\subsection*{カーテンウォール(CW)について}
カーテンウォールは高層ビルの発展に伴って19世紀以降に現れた非耐力壁の一種である.
19世紀に鉄骨造が発展し,壁が構造を支える必要性が無くなった.
また高層ビルの建設にあたり,壁の軽量化と高所での作業量を削減する必要があった.
カーテンウォールは非耐力壁であることから,壁の肉厚を削減でき壁の軽量化に寄与する.
更にプレファブ化によって現地での作業量を減らし,また工期短縮や足場の削減に寄与している.
加えてカーテンウォールは生産数が非常に多いことから,カーテンウォールを専門とする製造者が成立するため,
低コスト化や品質の安定化,技術の蓄積が行われやすいなどのメリットがある.
また一棟の高層ビルでも多くのカーテンウォールを使うことから,大規模なビルではオーダーメイドのカーテンウォールを用いて個性を表現することが多い.

カーテンウォールは主に金属系カーテンウォールとPC(プレキャストコンクリート)カーテンウォールに大別される.
金属系のカーテンウォールについてインターネットで検索したところフジテレビ本社ビルが金属系カーテンウォールを用いているとわかった.\cite{2008020675:online}
図\ref{fig:fig/huji.jpg}のように金属系カーテンウォールは金属光沢感が強いことがわかる.
一方でPCカーテンウォールは金属系に比べて重く,新しい技術である.
PCカーテンウォールは図\ref{fig:fig/pc_cw.jpg}のように重厚感のある表現が可能で,またコンクリートが主成分であることから耐火性が高く,重いことから遮音性も高い.
\mfig[width=8cm]{fig/huji.jpg}{フジテレビ本社ビル\cite{LIXIL70:online}}
\mfig[width=8cm]{fig/pc_cw.jpg}{PCカーテンウォール\cite{PC39:online}}

またカーテンウォールはその構造によっても分類でき,主に方立方式とパネル方式に分類される.
方立方式は方立(マリオン)という鉛直向きの枠材を一定ピッチで配置し,その間にガラスやパネルをはめ込むことで構成されるカーテンウォールで,金属系カーテンウォールで主に用いられる.
マリオン方式ではパネルが受ける風圧力などの水平力はマリオンを介して建物に伝達される.
方立方式の建物を検索したところ有楽町マリオンというものがあるとわかった.\cite{yuurakutyo59:online}図\ref{fig:fig/yuurakutyo.jpg}に有楽町マリオンの画像を示す.
画像からも明らかなように有楽町マリオンは方立を前面に押し出したデザインであることがわかる.
またマリオン方式では細長い部材が連続することから,熱による膨張を考慮するべきである.
そこで方立と建物の接続には図のようなファスナーという部品が用いられ,長孔によって変位を吸収するようになっている.
\mfig[width=8cm]{fig/yuurakutyo.jpg}{有楽町マリオン\cite{yuurakutyo59:online}}
\mfig[width=10cm]{fig/marion_fasnar.jpg}{ファスナーの例\cite{SR26:online}}

一方でパネル方式のカーテンウォールは1階分の高さのパネルを方下左右に並べることで構成され,図\ref{fig:fig/huji.jpg}のフジテレビ本社ビルもパネル方式である.\cite{LIXIL70:online}
パネル方式は金属系でもPCでも用いられる方式である.
高層ビルにおいては地震などにより水平力が加わると層の間に水平方向の変位が発生することがあり,これを層間変位と呼ぶ.
方立方式の場合パネルはマリオンを介して接続され,そのマリオンもファスナーによって接続されることから層間変位を吸収するのは比較的容易である.
一方でパネル方式の場合は面内方向の合成が高いパネルが直接建物に接続されることから,層間変位の吸収に特別の考慮が必要である.
層間変位の吸収にはロッキング方式やスライド方式などの方式がある.図\ref{fig:fig/soukan.jpg}に各方式の概略図を示す.
ロッキング方式は図\ref{fig:fig/soukan.jpg}の左側のように各パネルがピンを軸に回転することで層間変位を吸収する.
一方スライド方式は図\ref{fig:fig/soukan.jpg}右側のように各層のパネルが水平移動することにより層間変位を吸収する.
\mfig[width=10cm]{fig/soukan.jpg}{層間変位の吸収方式(左:ロッキング方式,右:スライド方式)\cite{PC39:online}}