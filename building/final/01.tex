\subsection*{窓について}
窓は開口部の一種であり,壁や屋根の一部にガラスなどの光を通す素材を取り付けることで
屋内に光を取り入れる機能がある.
窓には可動式のものとはめ殺しのものがあり,特に可動式のものは換気や清掃といった機能も持っている場合が多い.

以上からもわかるように,窓などの開口部は本来壁が持っている遮断性能を部分的に取り除くことで
ある機能を実現する機構だが,それに伴って好ましくない副作用をもたらす場合もある.
たとえば窓の主要な機能である光の取り入れは輻射による熱の伝達を増やすことに他ならなず,室内の調温機能にとっては外乱になる.
これは熱以外の防音性や気密性についても同様である.
実際の建築の際にはこれらのトレードオフを考慮した上で,要求性能を満たすように窓の方式などを選択する必要がある.

以下では可動式窓の開閉の方式についていくつか紹介する,
日本の窓では引違い窓が最も多く用いられている.\cite{mado:online}図\ref{fig:fig/hikitigai_zentai.jpg}は自宅で撮影した引違い窓である.
引き違い窓は窓全体を2毎に分割し,それぞれが左右に動くことで開閉機能を実現している.
引違い窓は非常に流通量が多いことから,既成品でもサイズが豊富にあり,様々な壁に適用できる.
また開き具合によって風通しが調整できるなどの特徴がある.

また私の自宅は飛行場に近いため,窓には防音性が高いサッシが用いられている.
図\ref{fig:fig/hikitigai_gasket.jpg}は防音サッシを外側から見た様子である.
また図\ref{fig:fig/hikitigai_kanamono.jpg}は防音サッシの金物である.
これらの図からわかるように防音サッシには隅にガスケットが取り付けられ,更に金物も窓の両側に取り付けられている.
この金物によりガスケットを締め付け,気密性を高めることで防音性を高めていると考えられる.
また図\ref{fig:fig/hikitigai_gasket.jpg}には窓の下部に結露水を流すための水抜き穴が見える.

図\ref{fig:fig/utitaore.jpg}は自宅で撮影した内倒し窓である.内倒し窓は窓枠の下辺を軸に室内側に倒れることで開く窓である.
引き違い窓と異なり内倒し窓は外部からの視線を遮ることができるため,浴室やプライバシーが求められる場所で用いられることが多い.\cite{mado:online}
実際図\ref{fig:fig/utitaore.jpg}は浴室に取り付けられていた窓である.
また内倒し窓はストッパーにより途中までしか開かないため,外部からの侵入防止や窓からの落下防止も期待できる.\cite{utitaosi:online}
\mfig[width=8cm, angle=270]{fig/hikitigai_zentai.jpg}{引違い窓の外観}
\mfig[width=5cm]{fig/hikitigai_gasket.jpg}{防音サッシのガスケットと水抜き穴}
\mfig[width=10cm]{fig/hikitigai_kanamono.jpg}{防音サッシの金物(鍵)}
\mfig[width=8cm]{fig/utitaore.jpg}{内倒し窓の外観}