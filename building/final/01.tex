\subsection*{窓について}
窓は開口部の一種であり,壁や屋根の一部にガラスなどの光を通す素材を取り付けることで
屋内に光を取り入れる機能がある.
窓には可動式のものとはめ殺しのものがあり,特に可動式のものは換気や清掃といった機能も持っている場合が多い.

以上からもわかるように,窓などの開口部は本来壁が持っている遮断性能を部分的に取り除くことで
ある機能を実現する機構だが,それに伴って好ましくない副作用をもたらす場合もある.
たとえば窓の主要な機能である光の取り入れは輻射による熱の伝達を増やすことに他ならなず,室内の調温機能にとっては外乱になる.
これは熱以外の防音性や気密性についても同様である.
実際の建築の際にはこれらのトレードオフを考慮した上で,要求性能を満たすように窓の方式などを選択する必要がある.