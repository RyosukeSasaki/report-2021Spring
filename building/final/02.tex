\subsection*{木造建築の壁について}
壁には屋内外を分ける外周壁と部屋を区切る間仕切り壁がある.
これらの壁は室内環境を維持するため断熱,防音,遮光,気密などの性能が求められる.
また日光や火災,雨などに対する対候性も必要不可欠である.
これらの性能に関する要求は環境や目的によって異なり,適当な構法を用いるべきである.

木造建築の壁には大きく分けて湿式壁と乾式壁がある.
湿式壁は水分を含んだ材料が乾燥し硬化することで作られる.一方で乾式壁は板材を用いる壁である.
湿式壁は継ぎ目のない広い面を作ることができ,また厚さの調整も容易である.
さらに乾式壁に比べて防音性や堅牢性も高い.\cite{mizuho_sissiki:online}
しかしながら湿式壁は左官工事によって作られるため,施工に熟練を要する.
また,湿式壁は乾燥と共に収縮し,次第に柱との間に隙間が生まれることがある.
これを防ぐため,場合によっては散りじゃくりという溝を柱に入れておき,そこに壁を埋め込むことで収縮しても隙間ができないようにすることがある.
また柱との密着性を高めるため,柱の表面はある程度荒い方が良い.
一方で乾式壁は施工が容易な一方,元の板材以上の広さの面を作ることはできない.
また石膏ボードなどの板材自体は軽量なため防音性や断熱性は低いと考えられるが,遮音材や断熱材を入れることで改善できる.

以下では壁の実例をいくつか示す.図\ref{fig:fig/suisya.jpg}は神奈川県大和市 泉の森の水車小屋で撮影した画像である.
図\ref{fig:fig/suisya_kabe_up.jpg}は壁を拡大した画像であり,上部が湿式壁であり下部に板張壁が用いられていることがわかる.
また上部の湿式壁は外部から柱が見えており,真壁である.
上側の湿式壁は荒い繊維が見えており繊維壁だとわかる.下地を見ることは出来なかったため,この建物が実際にどういった構法で建築されているのかは確認出来なかった.
下部の板張壁は板材を重ねた状態で細い棒材で抑えつけた押縁下見だとわかる.板材が外側に反りだすように重ねることで雨仕舞をしている.
このように湿式壁の腰部分を板張壁にすることで,湿式壁を保護していると考えられる.
\mfig[width=8cm]{fig/suisya.jpg}{神奈川県大和市 泉の森 水車小屋で撮影}
\mfig[width=8cm]{fig/suisya_kabe_up.jpg}{水車小屋の壁,上部が湿式壁で下部が乾式壁になっている}

また図\ref{fig:fig/kabe_before_after.png}には防音工事中の自宅で撮影した乾式壁の画像を示す.
工事中と施工後の画像を同じ壁面を同じ角度から撮影している.この比較からわかるように,柱は壁内部に隠れており,大壁である.
図\ref{fig:fig/kabe.jpg}のようにこの壁は石膏ボード張りの壁であり,その下には遮音シート,断熱材が敷き詰められている.
石膏ボードの目地には面取が施されており,パテ埋めで処理されていると考えられる.
遮音シートは高比重の塩ビ樹脂シートであり,密度の大きい物質を用いて遮音性を高めたものである.\cite{syaon14:online}
また断熱材はガラス繊維で構成され,このガラス繊維が多孔質のように機能することで屋内外の熱移動を遮断し,断熱性を高めている.\cite{mag:online}
更に図\ref{fig:fig/habaki.jpg}のように壁紙の上から幅木が取り付けられており,角を保護し,また壁紙や壁の端が見えないにくくなっている.
\mfig[width=8cm]{fig/kabe_before_after.png}{自宅で撮影した乾式壁(左:工事中,右:施工後)}
\mfig[width=10cm]{fig/kabe.jpg}{壁の構成(左:断熱材,中央:遮音シート,右:石膏ボード)}
\mfig[width=8cm]{fig/habaki.jpg}{幅木}