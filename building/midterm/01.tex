\subsection*{1 鉄骨造}
\subsubsection*{1.1 鋼材の規格}
構造用の鋼材にはSS400などの一般構造用鋼材, SN400などの建築用鋼材, SM400などの溶接構造用鋼材がある.
これらの記号の3桁の数字は強度区分を表しており$400$は$400$から$510\ \si{\newton.\metre^{-2}}$, $490$は$490$から$610\ \si{\newton.\metre^2}$の引張強度を持つようにJISで規定される.\cite{tekkotuQA28:online}
S45Cなどの機械用の鋼材が炭素含有量などの成分によって規定されるのに対し一般構造用鋼材は強度で規定されている.
SM材やSN材は溶接性を確保するため炭素の含有量に上限が規定されている.これは炭素が多く含まれている鋼材は溶接によって硬化し破壊されやすくなるためである.
またSN材はSM材での規定に加えて降伏比や衝撃値といった値に規定がある.これは建築に用いる際に耐震性を確保するための規定である.\cite{tekkotuQA28:online}
\subsubsection*{1.2 H型鋼}
H型鋼は図\ref{fig:fig1.png}のような断面形状を持つ構造用鋼材である.形状はI型鋼と似ているがI型鋼にはフランジにテーパーがついている.
H型鋼はフランジとウェブでそれぞれ曲げ耐力とせん断耐力を受け持っており,重量あたりの強度に優れている.
H型鋼を梁に用いる際は曲げ応力が大きく掛かることから,フランジの肉厚を厚くして,またこの間隔を遠ざけることによって効率よく曲げ耐力を得られる.
またウェブは平行な方向の荷重を受けると潰れるように座屈することが考えられるので図\ref{fig:fig2.png}のようにウェブ,フランジ双方と直行するような補強板を取り付けることがありこれをスチフナという.
H型鋼の仕口は溶接による剛接合とボルト留めによるピン接合がある.ピン接合の場合,図\ref{fig:fig3.png}のようにガセットプレートという板を溶接によって取り付け,ガセットプレートを介してH型鋼をボルト留めする.
またH型鋼を柱として用いる場合,その基礎はより強度の低い鉄筋コンクリートで作られることが多い.
こういった場合,H型鋼からの力はより広い面積に伝達する必要があることからベースプレートという板を用いて柱からの力をより広い断面積に伝達する.
図\ref{fig:fig4.png}に打設中の柱脚の画像を示す.左側の柱脚ではベースプレートが露出しており,周りに鉄筋が見えている.
一方で右側の柱脚はベースプレートがコンクリートで埋められている.このコンクリートを根巻コンクリートという.

またH型鋼は熱間の圧延によって製造される.圧延は祖圧延,中間圧延,仕上げ圧延という段階にわかれており,仕上げ圧延で用いるユニバーサル圧延機はフランジとウェブの厚さを容易に変更できるようになっている.\cite{Hgatakou:online}

以下に示す図\ref{fig:fig2.png}から図\ref{fig:fig4.png}は2019年11月20日にパークコート文京小石川 ザ タワー(施工:清水建設)の建設現場を見学した際に撮影した.
\begin{figure}[htbp]
  \begin{minipage}{0.5\hsize}
    \mfig[width=6cm]{fig1.png}{H型鋼とI型鋼の断面}
  \end{minipage}
  \begin{minipage}{0.5\hsize}
    \mfig[width=6cm]{fig2.png}{スチフナ}
  \end{minipage}
\end{figure}
\begin{figure}[htbp]
  \begin{minipage}{0.5\hsize}
    \mfig[width=6cm]{fig3.png}{ガセットプレート}
  \end{minipage}
  \begin{minipage}{0.5\hsize}
    \mfig[width=6cm]{fig4.png}{柱脚}
  \end{minipage}
\end{figure}
\mfig[width=10cm]{fig5.png}{H型鋼の圧延\cite{Hgatakou:online}}