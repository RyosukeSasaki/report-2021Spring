\subsection*{プレストレストコンクリート}
コンクリートは一般に圧縮力に強いが引張力には弱く,梁などが荷重を受けると引張応力が発生しヒビが入ることが考えられる.
プレストレストコンクリート(PC)ではプレキャスト,あるいは現場打ちの段階で図\ref{fig:fig8.png}のように鉄筋コンクリートに圧縮力を掛けておくことで,
異常な荷重などがかかった際に引張力を打ち消すことができる.
PCの利点として梁の必要な断面積を削減できるので,大スパンの梁を掛けられるようになる.
また荷重によりひびが発生しても荷重を取り除くとプレストレスにより材料が元の形状に戻ろうとするため,内部への水の侵入が少なくなる.
一方で欠点として高強度の鋼材やコンクリートを用いるためコストが高くなる,ストレスを掛けている鋼線は常に大きな荷重を受けているため熱などに弱くなるといった点がある.

PCはプレストレスをプレキャストの段階で掛けるのか,現場打ちの段階で掛けるのかによって名称が変わり,それぞれプレテンション,ポストテンションと呼ばれる.
プレテンションでは事前に緊張させた鋼線を型枠の中に配置しコンクリートを打設,
コンクリートが固まった後に鋼線のテンションを解くことによって鋼線が縮み,付着したコンクリートに圧縮力が掛かる.
一方でポストテンションではシースと呼ばれる管の内部にテンションを掛けていない鋼線を通してコンクリートを打設し,
コンクリートが固まった後に鋼線にテンションを掛けることで圧縮力を掛ける.
ここで鋼線とコンクリートは直接接触していないため,ストレスを与えるためには定着具を用いて力を伝達する必要がある.

プレストレストコンクリートが用いられた建築の例として図\ref{fig:fig9.jpg},図\ref{fig:fig10.jpg}のような橋や階段がある.
特に橋は橋脚のスパンを大きくするという点でプレストレストコンクリートが適していると考えられる.
\mfig[width=6cm]{fig8.png}{プレストレストコンクリートの原理\cite{alma990015457520204034}}
\begin{figure}[htbp]
  \begin{minipage}{0.5\hsize}
    \mfig[width=6cm]{fig9.jpg}{江島大橋\cite{konnatokoro:online}}
  \end{minipage}
  \begin{minipage}{0.5\hsize}
    \mfig[width=6cm]{fig10.jpg}{PCaPC構造の良さを生かした階段\cite{konnatokoro:online}}
  \end{minipage}
\end{figure}