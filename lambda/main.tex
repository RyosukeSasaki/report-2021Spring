\documentclass[uplatex,a4j,11pt,dvipdfmx]{jsarticle}
\bibliographystyle{jplain}

\usepackage{url}

\usepackage{graphicx}
\usepackage{gnuplot-lua-tikz}
\usepackage{pgfplots}
\usepackage{tikz}
\usepackage{amsmath,amsfonts,amssymb}
\usepackage{bm}
\usepackage{siunitx}

\makeatletter
\def\fgcaption{\def\@captype{figure}\caption}
\makeatother
\newcommand{\setsections}[3]{
\setcounter{section}{#1}
\setcounter{subsection}{#2}
\setcounter{subsubsection}{#3}
}
\newcommand{\mfig}[3][width=15cm]{
\begin{center}
\includegraphics[#1]{#2}
\fgcaption{#3 \label{fig:#2}}
\end{center}
}
\newcommand{\gnu}[2]{
\begin{figure}[hptb]
\begin{center}
\input{#2}
\caption{#1}
\label{fig:#2}
\end{center}
\end{figure}
}
\newcommand{\llr}{\langle\langle}
\newcommand{\rrr}{\rangle\rangle}

\begin{document}
\title{計算モデル論 レポート}
\author{佐々木良輔}
\date{}
\maketitle
\section*{課題1}
\subsection*{乗算関数$\llr mult\rrr$について}
乗算$m\times n$は$0+m+m+\cdots+m$と加算を$n$回繰り返すことで実現できる.
したがって授業スライド2のp47で$\llr add\rrr$を$\llr succ\rrr$繰り返しで定義したことの類推から
\begin{align*}
  \llr mult\rrr=:\lambda m.\lambda n.\bigg(n\ \Big(\llr add\rrr\ m\Big)\bigg)\ \llr 0\rrr
\end{align*}
で定義されると考えられる.\cite{utokyo}\cite{jugyo3}ここで$\llr add\rrr$は加算関数であり$\llr add\rrr\ \llr m\rrr\ \llr n\rrr=\llr m+n\rrr$である.実際に自然数に対して適用すると
\begin{align*}
  &\llr mult\rrr\ \llr m\rrr\ \llr n\rrr\\
  &\rightarrow_\beta \Bigg(\bigg(\lambda m.\Big(\lambda n.\Big(n\ \Big(\llr add\rrr\ m\Big)\Big)\ \llr 0\rrr\Big)\ \llr m\rrr\bigg)\ \llr n\rrr\Bigg)\\
  &\rightarrow_\beta \llr n\rrr\ \Big(\llr add\rrr\ \llr m\rrr\Big)\ \llr 0\rrr\\
  &\rightarrow_\beta \Bigg(\lambda x.\bigg(\lambda y.\Big(\underbrace{x(x\cdots(x(x}_{n個}\ y))\cdots)\Big)\bigg)\ \Big(\llr add\rrr\ \llr m\rrr\Big)\Bigg)\ \llr 0\rrr\\
  &\rightarrow_\beta \underbrace{\llr add\rrr\ \llr m\rrr\ \Bigg(\llr add\rrr\ \llr m\rrr\cdots\bigg(\llr add\rrr\ \llr m\rrr\ \Big(\llr add\rrr\ \llr m\rrr}_{n個}\ \llr 0\rrr\Big)\bigg)\Bigg)\\
  &\rightarrow_\beta \llr add\rrr\ \llr m\rrr\ \bigg(\llr add\rrr\ \llr m\rrr\cdots\Big(\llr add\rrr\ \llr m\rrr\ \llr m\rrr\Big)\bigg)\\
  &\rightarrow_\beta \llr add\rrr\ \llr m\rrr\ \Big(\llr add\rrr\ \llr m\rrr\cdots\llr 2m\rrr\Big)\\
  &\rightarrow_\beta \llr add\rrr\ \llr m\rrr\ \llr m\times(n-1)\rrr\\
  &\rightarrow_\beta\llr m\times n\rrr
\end{align*}
となり,確かに乗算が行われている.
\subsection*{冪乗関数$\llr pow\rrr$について}
乗算が加算の繰り返しで実現したのと同様に冪乗$m^n$も$1\times m\times\cdots\times m$と乗算を$n$回繰り返すことで実現できる.
したがって乗算関数と同様に定義すれば
\begin{align*}
  \llr pow\rrr=:\lambda m.\lambda n.\bigg(n\ \Big(\llr mult\rrr\ m\Big)\bigg)\ \llr 1\rrr
\end{align*}
で定義されると考えられる.\cite{utokyo}実際に自然数に対して適用すると,乗算関数の場合と同様の$\beta$変換を行うことで
\begin{align*}
  &\llr pow\rrr\ \llr m\rrr\ \llr n\rrr\\
  &\rightarrow_\beta \underbrace{\llr mult\rrr\ \llr m\rrr\ \Bigg(\llr mult\rrr\ \llr m\rrr\cdots\bigg(\llr mult\rrr\ \llr m\rrr\ \Big(\llr mult\rrr\ \llr m\rrr}_{n個}\ \llr 1\rrr\Big)\bigg)\Bigg)\\
  &\rightarrow_\beta \llr mult\rrr\ \llr m\rrr\ \bigg(\llr mult\rrr\ \llr m\rrr\cdots\Big(\llr mult\rrr\ \llr m\rrr\ \llr m\rrr\Big)\bigg)\\
  &\rightarrow_\beta \llr mult\rrr\ \llr m\rrr\ \Big(\llr mult\rrr\ \llr m\rrr\cdots\llr m^2\rrr\Big)\\
  &\rightarrow_\beta \llr mult\rrr\ \llr m\rrr\ \llr m^{n-1}\rrr\\
  &\rightarrow_\beta\llr m^n\rrr
\end{align*}
となり,確かに冪乗が行われている.
\bibliography{ref.bib}
\end{document}