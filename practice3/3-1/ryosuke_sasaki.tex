\documentclass[uplatex,a4j,11pt,dvipdfmx]{jsarticle}
\bibliographystyle{jplain}

%\usepackage{url}
\usepackage{graphicx}
\usepackage{gnuplot-lua-tikz}
\usepackage{pgfplots}
\usepackage{tikz}
\usepackage{amsmath,amsfonts,amssymb}
\usepackage{bm}
\usepackage{siunitx}

\makeatletter
\def\fgcaption{\def\@captype{figure}\caption}
\makeatother
\newcommand{\setsections}[3]{
\setcounter{section}{#1}
\setcounter{subsection}{#2}
\setcounter{subsubsection}{#3}
}
\newcommand{\mfig}[3][width=15cm]{
\begin{center}
\includegraphics[#1]{#2}
\fgcaption{#3 \label{fig:#2}}
\end{center}
}

\renewcommand{\appendixname}{補遺}

\begin{document}
%\title{物理学演習第3}
%\author{佐々木良輔}
%\date{}
%\maketitle
\subsection*{3-1.}
\subsubsection*{(1) 解答}
\begin{align*}
  Z=\left(2\cosh (\beta\mu H)\right)^N
\end{align*}
\dotfill\\
\subsubsection*{(2) 解説}
分配関数$Z$は(3.5)式から微視的状態$\nu$について
\begin{align*}
  Z&=\sum_\nu{\rm e}^{-\beta E}\\
  &=\sum_\nu{\rm e}^{\beta\mu H\sum_i \sigma_i}
\end{align*}
となる.ここで$\sigma_i=\pm1$を取るので
\begin{align}
  Z&=\sum_{\sigma_1=\pm1}\sum_{\sigma_2=\pm1}\cdots\sum_{\sigma_N=\pm1}{\rm e}^{\beta\mu H\sum_i \sigma_i}\nonumber\\
  &=\sum_{\sigma_1=\pm1}\sum_{\sigma_2=\pm1}\cdots\sum_{\sigma_N=\pm1}{\rm e}^{\beta\mu H(\sigma_1+\sigma_2+\cdots+\sigma_N)}\\
  &=\sum_{\sigma_1=\pm1}{\rm e}^{\beta\mu H\sigma_1} \sum_{\sigma_2=\pm1}{\rm e}^{\beta\mu H\sigma_2}\cdots \sum_{\sigma_N=\pm1}{\rm e}^{\beta\mu H\sigma_N}\nonumber\\
  &=({\rm e}^{\beta\mu H}+{\rm e}^{-\beta\mu H})^N\nonumber\\
  &=(2\cosh (\beta\mu H))^N\nonumber
\end{align}
となる.
\clearpage
\subsubsection*{(2) 解答}
\begin{align*}
  F=-\frac{N}{\beta}\log(2\cosh (\beta\mu H))
\end{align*}
\dotfill\\
\subsubsection*{(2) 解説}
(3.12)式から
\begin{align}
  F&=-\frac{1}{\beta}\log Z\\
  &=-\frac{1}{\beta}\log (2\cosh (\beta\mu H))^N\nonumber\\
  &=-\frac{N}{\beta}\log(2\cosh (\beta\mu H))
\end{align}
\clearpage
\subsubsection*{(3) 解答}
\begin{align*}
  \langle M\rangle&=N\mu\tanh(\beta\mu H)\\
\end{align*}
\dotfill\\
\subsubsection*{(3) 解説}
(3.5)式から$M$の期待値は
\begin{align*}
  \langle M\rangle&=\sum_\nu p_\nu M_\nu\\
  &=\sum_\nu \frac{1}{Z}{\rm e}^{\beta\mu H\sum_i\sigma_i}\mu\sum_{i=1}^N \sigma_i\\
  &=\frac{\mu}{Z}\sum_{\sigma_1=\pm1}\sum_{\sigma_2=\pm1}\cdots\sum_{\sigma_N=\pm1}{\rm e}^{\beta\mu H(\sigma_1+\sigma_2+\cdots+\sigma_N)}(\sigma_1+\sigma_2+\cdots+\sigma_N)
\end{align*}
ここで$\partial F/\partial H$を考えると(1), (2)より
\begin{align*}
  \frac{\partial F}{\partial H}&=\frac{\partial Z}{\partial H}\frac{\partial F}{\partial Z}\\
  &=\frac{\partial}{\partial H}\sum_{\sigma_1=\pm1}\sum_{\sigma_2=\pm1}\cdots\sum_{\sigma_N=\pm1}{\rm e}^{\beta\mu H(\sigma_1+\sigma_2+\cdots+\sigma_N)}\times\frac{-1}{\beta Z}\\
\end{align*}
ここで級数は有限であり,項別に微分可能なので
\begin{align*}
  \frac{\partial F}{\partial H}&=\sum_{\sigma_1=\pm1}\sum_{\sigma_2=\pm1}\cdots\sum_{\sigma_N=\pm1}\frac{\partial}{\partial H}{\rm e}^{\beta\mu H(\sigma_1+\sigma_2+\cdots+\sigma_N)}\times\frac{-1}{\beta Z}\\
  &=\sum_{\sigma_1=\pm1}\sum_{\sigma_2=\pm1}\cdots\sum_{\sigma_N=\pm1}{\rm e}^{\beta\mu H(\sigma_1+\sigma_2+\cdots+\sigma_N)}\beta\mu(\sigma_1+\sigma_2+\cdots+\sigma_N)\times\frac{-1}{\beta Z}\\
  &=\frac{-\mu}{Z}\sum_{\sigma_1=\pm1}\sum_{\sigma_2=\pm1}\cdots\sum_{\sigma_N=\pm1}{\rm e}^{\beta\mu H(\sigma_1+\sigma_2+\cdots+\sigma_N)}(\sigma_1+\sigma_2+\cdots+\sigma_N)
\end{align*}
したがって(3)式から
\begin{align}
  \langle M\rangle&=-\frac{\partial F}{\partial H}\nonumber\\
  &=\frac{N}{\beta}\frac{\partial}{\partial H}\log(2\cosh(\beta\mu H))\nonumber\\
  &=\frac{N}{\beta}\frac{2\beta\mu\sinh(\beta\mu H)}{2\cosh(\beta\mu H)}\nonumber\\
  &=N\mu\tanh(\beta\mu H)=N\mu\tanh(\mu H/k_BT)
\end{align}
\clearpage
\appendix
\section{}
原子が1個のとき$M$の期待値$\langle m\rangle$は
\begin{align*}
  \langle m\rangle&=\frac{-\mu{\rm e}^{-\beta\mu H}+\mu{\rm e}^{\beta\mu H}}{{\rm e}^{-\beta\mu H}+{\rm e}^{\beta\mu H}}\\
  &=\mu\tanh(\beta\mu H)
\end{align*}
であり,原子が$N$個のときの期待値$\langle M\rangle$は$\langle m\rangle$の$N$倍になっている.これは各原子が相互作用せず独立に動くことに整合する.
\section{}
(4)式において両辺を体積$V$で割る.
\begin{align*}
  \frac{\langle M\rangle}{V}&=\frac{N}{V}\mu\tanh(\mu H/k_BT)\\
  \langle I\rangle&=n\mu\tanh(\mu H/k_BT)
\end{align*}
ここで$\langle I\rangle$は磁化の期待値, $n$は単位体積あたりの原子数である.
$\mu H/k_BT \ll 1$のとき,すなわち温度が十分高いとき$\langle I\rangle$を1次まで展開すると$\tanh x\simeq x+o(x^3)$から
\begin{align*}
  \langle I\rangle&=n\mu\frac{\mu H}{k_BT}
\end{align*}
ここで磁化率$\chi=I/H$を用いて
\begin{align*}
  \chi&=\frac{n\mu^2}{k_B}\frac{1}{T}=:\frac{C}{T}
\end{align*}
これはキュリーの法則とよばれる.また$C=n\mu^2/k_B$はキュリー定数である.
\clearpage
\section{}
$\langle M\rangle$は図\ref{fig:M}のように振る舞う.
ここで粒子数$N=N_A$ (アボガドロ数), $\mu=\mu_B$ (ボーア磁子), $H=50\si{\tesla}$とした.
\begin{figure}[hptb]
\begin{center}
\begin{tikzpicture}[gnuplot]
%% generated with GNUPLOT 5.2p8 (Lua 5.3; terminal rev. Nov 2018, script rev. 108)
%% 2021年05月16日 02時28分54秒
\path (0.000,0.000) rectangle (12.000,8.000);
\gpcolor{color=gp lt color border}
\gpsetlinetype{gp lt border}
\gpsetdashtype{gp dt solid}
\gpsetlinewidth{1.00}
\draw[gp path] (1.320,0.985)--(1.500,0.985);
\draw[gp path] (11.447,0.985)--(11.267,0.985);
\node[gp node right] at (1.136,0.985) {$0.0$};
\draw[gp path] (1.320,1.823)--(1.500,1.823);
\draw[gp path] (11.447,1.823)--(11.267,1.823);
\node[gp node right] at (1.136,1.823) {$1.0$};
\draw[gp path] (1.320,2.662)--(1.500,2.662);
\draw[gp path] (11.447,2.662)--(11.267,2.662);
\node[gp node right] at (1.136,2.662) {$2.0$};
\draw[gp path] (1.320,3.500)--(1.500,3.500);
\draw[gp path] (11.447,3.500)--(11.267,3.500);
\node[gp node right] at (1.136,3.500) {$3.0$};
\draw[gp path] (1.320,4.338)--(1.500,4.338);
\draw[gp path] (11.447,4.338)--(11.267,4.338);
\node[gp node right] at (1.136,4.338) {$4.0$};
\draw[gp path] (1.320,5.176)--(1.500,5.176);
\draw[gp path] (11.447,5.176)--(11.267,5.176);
\node[gp node right] at (1.136,5.176) {$5.0$};
\draw[gp path] (1.320,6.014)--(1.500,6.014);
\draw[gp path] (11.447,6.014)--(11.267,6.014);
\node[gp node right] at (1.136,6.014) {$6.0$};
\draw[gp path] (1.320,6.853)--(1.500,6.853);
\draw[gp path] (11.447,6.853)--(11.267,6.853);
\node[gp node right] at (1.136,6.853) {$7.0$};
\draw[gp path] (1.320,7.691)--(1.500,7.691);
\draw[gp path] (11.447,7.691)--(11.267,7.691);
\node[gp node right] at (1.136,7.691) {$8.0$};
\draw[gp path] (1.320,0.985)--(1.320,1.165);
\draw[gp path] (1.320,7.691)--(1.320,7.511);
\node[gp node center] at (1.320,0.677) {$0$};
\draw[gp path] (3.008,0.985)--(3.008,1.165);
\draw[gp path] (3.008,7.691)--(3.008,7.511);
\node[gp node center] at (3.008,0.677) {$50$};
\draw[gp path] (4.696,0.985)--(4.696,1.165);
\draw[gp path] (4.696,7.691)--(4.696,7.511);
\node[gp node center] at (4.696,0.677) {$100$};
\draw[gp path] (6.384,0.985)--(6.384,1.165);
\draw[gp path] (6.384,7.691)--(6.384,7.511);
\node[gp node center] at (6.384,0.677) {$150$};
\draw[gp path] (8.071,0.985)--(8.071,1.165);
\draw[gp path] (8.071,7.691)--(8.071,7.511);
\node[gp node center] at (8.071,0.677) {$200$};
\draw[gp path] (9.759,0.985)--(9.759,1.165);
\draw[gp path] (9.759,7.691)--(9.759,7.511);
\node[gp node center] at (9.759,0.677) {$250$};
\draw[gp path] (11.447,0.985)--(11.447,1.165);
\draw[gp path] (11.447,7.691)--(11.447,7.511);
\node[gp node center] at (11.447,0.677) {$300$};
\draw[gp path] (1.320,7.691)--(1.320,0.985)--(11.447,0.985)--(11.447,7.691)--cycle;
\node[gp node center,rotate=-270] at (0.292,4.338) {$\langle M\rangle$ / $\times10^{-6}\ \si{\weber.\metre}$};
\node[gp node center] at (6.383,0.215) {$T$ / $\si{\kelvin}$};
\gpcolor{rgb color={0.580,0.000,0.827}}
\draw[gp path] (1.320,6.889)--(1.321,6.889)--(1.322,6.889)--(1.323,6.889)--(1.324,6.889)%
  --(1.325,6.889)--(1.326,6.889)--(1.327,6.889)--(1.328,6.889)--(1.329,6.889)--(1.330,6.889)%
  --(1.331,6.889)--(1.332,6.889)--(1.333,6.889)--(1.334,6.889)--(1.335,6.889)--(1.336,6.889)%
  --(1.337,6.889)--(1.338,6.889)--(1.339,6.889)--(1.340,6.889)--(1.341,6.889)--(1.342,6.889)%
  --(1.343,6.889)--(1.344,6.889)--(1.345,6.889)--(1.346,6.889)--(1.347,6.889)--(1.348,6.889)%
  --(1.349,6.889)--(1.350,6.889)--(1.351,6.889)--(1.352,6.889)--(1.353,6.889)--(1.354,6.889)%
  --(1.355,6.889)--(1.356,6.889)--(1.357,6.889)--(1.358,6.889)--(1.359,6.889)--(1.360,6.889)%
  --(1.361,6.889)--(1.362,6.889)--(1.363,6.889)--(1.364,6.889)--(1.365,6.889)--(1.366,6.889)%
  --(1.367,6.889)--(1.368,6.889)--(1.369,6.889)--(1.370,6.889)--(1.371,6.889)--(1.372,6.889)%
  --(1.373,6.889)--(1.374,6.889)--(1.375,6.889)--(1.376,6.889)--(1.377,6.889)--(1.378,6.889)%
  --(1.379,6.889)--(1.380,6.889)--(1.381,6.889)--(1.382,6.889)--(1.383,6.889)--(1.384,6.889)%
  --(1.385,6.889)--(1.386,6.889)--(1.387,6.889)--(1.388,6.889)--(1.389,6.889)--(1.390,6.889)%
  --(1.391,6.889)--(1.392,6.889)--(1.393,6.889)--(1.394,6.889)--(1.395,6.889)--(1.396,6.889)%
  --(1.397,6.889)--(1.398,6.889)--(1.399,6.889)--(1.400,6.889)--(1.401,6.889)--(1.402,6.889)%
  --(1.403,6.889)--(1.404,6.889)--(1.405,6.889)--(1.406,6.889)--(1.407,6.889)--(1.408,6.889)%
  --(1.409,6.889)--(1.410,6.889)--(1.411,6.889)--(1.412,6.889)--(1.413,6.889)--(1.414,6.889)%
  --(1.415,6.889)--(1.416,6.889)--(1.417,6.889)--(1.418,6.889)--(1.419,6.889)--(1.420,6.889)%
  --(1.421,6.889)--(1.422,6.889)--(1.423,6.889)--(1.424,6.889)--(1.425,6.889)--(1.426,6.889)%
  --(1.427,6.889)--(1.428,6.889)--(1.429,6.889)--(1.430,6.889)--(1.431,6.889)--(1.432,6.889)%
  --(1.433,6.889)--(1.434,6.889)--(1.435,6.889)--(1.436,6.889)--(1.437,6.889)--(1.438,6.889)%
  --(1.439,6.889)--(1.440,6.889)--(1.441,6.889)--(1.442,6.889)--(1.443,6.889)--(1.444,6.889)%
  --(1.445,6.889)--(1.446,6.889)--(1.447,6.889)--(1.448,6.889)--(1.449,6.889)--(1.450,6.889)%
  --(1.451,6.889)--(1.452,6.889)--(1.453,6.889)--(1.454,6.889)--(1.455,6.889)--(1.456,6.889)%
  --(1.457,6.889)--(1.458,6.889)--(1.459,6.889)--(1.460,6.889)--(1.461,6.889)--(1.462,6.889)%
  --(1.463,6.889)--(1.464,6.889)--(1.465,6.889)--(1.466,6.889)--(1.467,6.889)--(1.468,6.889)%
  --(1.469,6.889)--(1.470,6.889)--(1.471,6.889)--(1.472,6.889)--(1.473,6.889)--(1.474,6.889)%
  --(1.475,6.889)--(1.476,6.889)--(1.477,6.889)--(1.478,6.889)--(1.479,6.889)--(1.480,6.889)%
  --(1.481,6.889)--(1.482,6.889)--(1.483,6.889)--(1.484,6.889)--(1.485,6.889)--(1.486,6.889)%
  --(1.487,6.889)--(1.488,6.889)--(1.489,6.889)--(1.490,6.889)--(1.491,6.889)--(1.492,6.889)%
  --(1.493,6.889)--(1.494,6.889)--(1.495,6.889)--(1.496,6.889)--(1.497,6.889)--(1.498,6.889)%
  --(1.499,6.889)--(1.500,6.889)--(1.501,6.889)--(1.502,6.889)--(1.503,6.889)--(1.504,6.889)%
  --(1.505,6.889)--(1.506,6.889)--(1.507,6.889)--(1.508,6.889)--(1.509,6.889)--(1.510,6.889)%
  --(1.511,6.889)--(1.512,6.889)--(1.513,6.889)--(1.514,6.889)--(1.515,6.889)--(1.516,6.889)%
  --(1.517,6.889)--(1.518,6.889)--(1.519,6.889)--(1.520,6.889)--(1.521,6.889)--(1.522,6.889)%
  --(1.523,6.889)--(1.524,6.889)--(1.525,6.889)--(1.526,6.889)--(1.527,6.889)--(1.528,6.889)%
  --(1.529,6.889)--(1.530,6.889)--(1.531,6.889)--(1.532,6.889)--(1.533,6.889)--(1.534,6.889)%
  --(1.535,6.889)--(1.536,6.889)--(1.537,6.889)--(1.538,6.889)--(1.539,6.889)--(1.540,6.889)%
  --(1.541,6.889)--(1.542,6.889)--(1.543,6.889)--(1.544,6.889)--(1.545,6.889)--(1.546,6.889)%
  --(1.547,6.889)--(1.548,6.889)--(1.549,6.889)--(1.550,6.889)--(1.550,6.888)--(1.551,6.888)%
  --(1.552,6.888)--(1.553,6.888)--(1.554,6.888)--(1.555,6.888)--(1.556,6.888)--(1.557,6.888)%
  --(1.558,6.888)--(1.559,6.888)--(1.560,6.888)--(1.561,6.888)--(1.562,6.888)--(1.563,6.888)%
  --(1.564,6.888)--(1.565,6.888)--(1.566,6.888)--(1.567,6.888)--(1.568,6.888)--(1.569,6.888)%
  --(1.570,6.888)--(1.571,6.888)--(1.572,6.888)--(1.573,6.888)--(1.574,6.888)--(1.575,6.888)%
  --(1.575,6.887)--(1.576,6.887)--(1.577,6.887)--(1.578,6.887)--(1.579,6.887)--(1.580,6.887)%
  --(1.581,6.887)--(1.582,6.887)--(1.583,6.887)--(1.584,6.887)--(1.585,6.887)--(1.586,6.887)%
  --(1.587,6.887)--(1.588,6.887)--(1.589,6.887)--(1.590,6.887)--(1.590,6.886)--(1.591,6.886)%
  --(1.592,6.886)--(1.593,6.886)--(1.594,6.886)--(1.595,6.886)--(1.596,6.886)--(1.597,6.886)%
  --(1.598,6.886)--(1.599,6.886)--(1.600,6.886)--(1.600,6.885)--(1.601,6.885)--(1.602,6.885)%
  --(1.603,6.885)--(1.604,6.885)--(1.605,6.885)--(1.606,6.885)--(1.607,6.885)--(1.608,6.885)%
  --(1.609,6.885)--(1.609,6.884)--(1.610,6.884)--(1.611,6.884)--(1.612,6.884)--(1.613,6.884)%
  --(1.614,6.884)--(1.615,6.884)--(1.616,6.884)--(1.616,6.883)--(1.617,6.883)--(1.618,6.883)%
  --(1.619,6.883)--(1.620,6.883)--(1.621,6.883)--(1.622,6.883)--(1.623,6.883)--(1.623,6.882)%
  --(1.624,6.882)--(1.625,6.882)--(1.626,6.882)--(1.627,6.882)--(1.628,6.882)--(1.629,6.882)%
  --(1.629,6.881)--(1.630,6.881)--(1.631,6.881)--(1.632,6.881)--(1.633,6.881)--(1.634,6.881)%
  --(1.634,6.880)--(1.635,6.880)--(1.636,6.880)--(1.637,6.880)--(1.638,6.880)--(1.639,6.880)%
  --(1.639,6.879)--(1.640,6.879)--(1.641,6.879)--(1.642,6.879)--(1.643,6.879)--(1.643,6.878)%
  --(1.644,6.878)--(1.645,6.878)--(1.646,6.878)--(1.647,6.878)--(1.648,6.878)--(1.648,6.877)%
  --(1.649,6.877)--(1.650,6.877)--(1.651,6.877)--(1.652,6.877)--(1.652,6.876)--(1.653,6.876)%
  --(1.654,6.876)--(1.655,6.876)--(1.655,6.875)--(1.656,6.875)--(1.657,6.875)--(1.658,6.875)%
  --(1.659,6.875)--(1.659,6.874)--(1.660,6.874)--(1.661,6.874)--(1.662,6.874)--(1.662,6.873)%
  --(1.663,6.873)--(1.664,6.873)--(1.665,6.873)--(1.665,6.872)--(1.666,6.872)--(1.667,6.872)%
  --(1.668,6.872)--(1.669,6.872)--(1.669,6.871)--(1.670,6.871)--(1.671,6.871)--(1.672,6.871)%
  --(1.672,6.870)--(1.673,6.870)--(1.674,6.870)--(1.674,6.869)--(1.675,6.869)--(1.676,6.869)%
  --(1.677,6.869)--(1.677,6.868)--(1.678,6.868)--(1.679,6.868)--(1.680,6.868)--(1.680,6.867)%
  --(1.681,6.867)--(1.682,6.867)--(1.683,6.866)--(1.684,6.866)--(1.685,6.866)--(1.685,6.865)%
  --(1.686,6.865)--(1.687,6.865)--(1.687,6.864)--(1.688,6.864)--(1.689,6.864)--(1.690,6.864)%
  --(1.690,6.863)--(1.691,6.863)--(1.692,6.863)--(1.692,6.862)--(1.693,6.862)--(1.694,6.862)%
  --(1.694,6.861)--(1.695,6.861)--(1.696,6.861)--(1.697,6.861)--(1.697,6.860)--(1.698,6.860)%
  --(1.699,6.860)--(1.699,6.859)--(1.700,6.859)--(1.701,6.859)--(1.701,6.858)--(1.702,6.858)%
  --(1.703,6.858)--(1.703,6.857)--(1.704,6.857)--(1.705,6.857)--(1.705,6.856)--(1.706,6.856)%
  --(1.707,6.856)--(1.707,6.855)--(1.708,6.855)--(1.709,6.855)--(1.709,6.854)--(1.710,6.854)%
  --(1.711,6.854)--(1.711,6.853)--(1.712,6.853)--(1.713,6.853)--(1.713,6.852)--(1.714,6.852)%
  --(1.715,6.852)--(1.715,6.851)--(1.716,6.851)--(1.716,6.850)--(1.717,6.850)--(1.718,6.850)%
  --(1.718,6.849)--(1.719,6.849)--(1.720,6.849)--(1.720,6.848)--(1.721,6.848)--(1.722,6.848)%
  --(1.722,6.847)--(1.723,6.847)--(1.723,6.846)--(1.724,6.846)--(1.725,6.846)--(1.725,6.845)%
  --(1.726,6.845)--(1.727,6.845)--(1.727,6.844)--(1.728,6.844)--(1.728,6.843)--(1.729,6.843)%
  --(1.730,6.843)--(1.730,6.842)--(1.731,6.842)--(1.732,6.842)--(1.732,6.841)--(1.733,6.841)%
  --(1.733,6.840)--(1.734,6.840)--(1.735,6.840)--(1.735,6.839)--(1.736,6.839)--(1.736,6.838)%
  --(1.737,6.838)--(1.738,6.838)--(1.738,6.837)--(1.739,6.837)--(1.739,6.836)--(1.740,6.836)%
  --(1.741,6.836)--(1.741,6.835)--(1.742,6.835)--(1.742,6.834)--(1.743,6.834)--(1.744,6.834)%
  --(1.744,6.833)--(1.745,6.833)--(1.745,6.832)--(1.746,6.832)--(1.746,6.831)--(1.747,6.831)%
  --(1.748,6.831)--(1.748,6.830)--(1.749,6.830)--(1.749,6.829)--(1.750,6.829)--(1.751,6.829)%
  --(1.751,6.828)--(1.752,6.828)--(1.752,6.827)--(1.753,6.827)--(1.753,6.826)--(1.754,6.826)%
  --(1.755,6.826)--(1.755,6.825)--(1.756,6.825)--(1.756,6.824)--(1.757,6.824)--(1.757,6.823)%
  --(1.758,6.823)--(1.758,6.822)--(1.759,6.822)--(1.760,6.822)--(1.760,6.821)--(1.761,6.821)%
  --(1.761,6.820)--(1.762,6.820)--(1.762,6.819)--(1.763,6.819)--(1.763,6.818)--(1.764,6.818)%
  --(1.765,6.818)--(1.765,6.817)--(1.766,6.817)--(1.766,6.816)--(1.767,6.816)--(1.767,6.815)%
  --(1.768,6.815)--(1.768,6.814)--(1.769,6.814)--(1.770,6.814)--(1.770,6.813)--(1.771,6.813)%
  --(1.771,6.812)--(1.772,6.812)--(1.772,6.811)--(1.773,6.811)--(1.773,6.810)--(1.774,6.810)%
  --(1.774,6.809)--(1.775,6.809)--(1.775,6.808)--(1.776,6.808)--(1.776,6.807)--(1.777,6.807)%
  --(1.778,6.807)--(1.778,6.806)--(1.779,6.806)--(1.779,6.805)--(1.780,6.805)--(1.780,6.804)%
  --(1.781,6.804)--(1.781,6.803)--(1.782,6.803)--(1.782,6.802)--(1.783,6.802)--(1.783,6.801)%
  --(1.784,6.801)--(1.784,6.800)--(1.785,6.800)--(1.785,6.799)--(1.786,6.799)--(1.786,6.798)%
  --(1.787,6.798)--(1.787,6.797)--(1.788,6.797)--(1.789,6.796)--(1.790,6.796)--(1.790,6.795)%
  --(1.791,6.795)--(1.791,6.794)--(1.792,6.794)--(1.792,6.793)--(1.793,6.793)--(1.793,6.792)%
  --(1.794,6.792)--(1.794,6.791)--(1.795,6.791)--(1.795,6.790)--(1.796,6.790)--(1.796,6.789)%
  --(1.797,6.789)--(1.797,6.788)--(1.798,6.788)--(1.798,6.787)--(1.799,6.787)--(1.799,6.786)%
  --(1.800,6.786)--(1.800,6.785)--(1.801,6.785)--(1.801,6.784)--(1.802,6.784)--(1.802,6.783)%
  --(1.803,6.783)--(1.803,6.782)--(1.804,6.782)--(1.804,6.781)--(1.805,6.781)--(1.805,6.780)%
  --(1.806,6.780)--(1.806,6.779)--(1.807,6.779)--(1.807,6.778)--(1.807,6.777)--(1.808,6.777)%
  --(1.808,6.776)--(1.809,6.776)--(1.809,6.775)--(1.810,6.775)--(1.810,6.774)--(1.811,6.774)%
  --(1.811,6.773)--(1.812,6.773)--(1.812,6.772)--(1.813,6.772)--(1.813,6.771)--(1.814,6.771)%
  --(1.814,6.770)--(1.815,6.770)--(1.815,6.769)--(1.816,6.769)--(1.816,6.768)--(1.817,6.768)%
  --(1.817,6.767)--(1.818,6.767)--(1.818,6.766)--(1.819,6.766)--(1.819,6.765)--(1.819,6.764)%
  --(1.820,6.764)--(1.820,6.763)--(1.821,6.763)--(1.821,6.762)--(1.822,6.762)--(1.822,6.761)%
  --(1.823,6.761)--(1.823,6.760)--(1.824,6.760)--(1.824,6.759)--(1.825,6.759)--(1.825,6.758)%
  --(1.826,6.758)--(1.826,6.757)--(1.826,6.756)--(1.827,6.756)--(1.827,6.755)--(1.828,6.755)%
  --(1.828,6.754)--(1.829,6.754)--(1.829,6.753)--(1.830,6.753)--(1.830,6.752)--(1.831,6.752)%
  --(1.831,6.751)--(1.832,6.751)--(1.832,6.750)--(1.832,6.749)--(1.833,6.749)--(1.833,6.748)%
  --(1.834,6.748)--(1.834,6.747)--(1.835,6.747)--(1.835,6.746)--(1.836,6.746)--(1.836,6.745)%
  --(1.837,6.745)--(1.837,6.744)--(1.837,6.743)--(1.838,6.743)--(1.838,6.742)--(1.839,6.742)%
  --(1.839,6.741)--(1.840,6.741)--(1.840,6.740)--(1.841,6.740)--(1.841,6.739)--(1.841,6.738)%
  --(1.842,6.738)--(1.842,6.737)--(1.843,6.737)--(1.843,6.736)--(1.844,6.736)--(1.844,6.735)%
  --(1.845,6.735)--(1.845,6.734)--(1.845,6.733)--(1.846,6.733)--(1.846,6.732)--(1.847,6.732)%
  --(1.847,6.731)--(1.848,6.731)--(1.848,6.730)--(1.849,6.730)--(1.849,6.729)--(1.849,6.728)%
  --(1.850,6.728)--(1.850,6.727)--(1.851,6.727)--(1.851,6.726)--(1.852,6.726)--(1.852,6.725)%
  --(1.852,6.724)--(1.853,6.724)--(1.853,6.723)--(1.854,6.723)--(1.854,6.722)--(1.855,6.722)%
  --(1.855,6.721)--(1.856,6.720)--(1.856,6.719)--(1.857,6.719)--(1.857,6.718)--(1.858,6.718)%
  --(1.858,6.717)--(1.859,6.717)--(1.859,6.716)--(1.859,6.715)--(1.860,6.715)--(1.860,6.714)%
  --(1.861,6.714)--(1.861,6.713)--(1.862,6.713)--(1.862,6.712)--(1.862,6.711)--(1.863,6.711)%
  --(1.863,6.710)--(1.864,6.710)--(1.864,6.709)--(1.864,6.708)--(1.865,6.708)--(1.865,6.707)%
  --(1.866,6.707)--(1.866,6.706)--(1.867,6.706)--(1.867,6.705)--(1.867,6.704)--(1.868,6.704)%
  --(1.868,6.703)--(1.869,6.703)--(1.869,6.702)--(1.870,6.702)--(1.870,6.701)--(1.870,6.700)%
  --(1.871,6.700)--(1.871,6.699)--(1.872,6.699)--(1.872,6.698)--(1.872,6.697)--(1.873,6.697)%
  --(1.873,6.696)--(1.874,6.696)--(1.874,6.695)--(1.875,6.695)--(1.875,6.694)--(1.875,6.693)%
  --(1.876,6.693)--(1.876,6.692)--(1.877,6.692)--(1.877,6.691)--(1.877,6.690)--(1.878,6.690)%
  --(1.878,6.689)--(1.879,6.689)--(1.879,6.688)--(1.879,6.687)--(1.880,6.687)--(1.880,6.686)%
  --(1.881,6.686)--(1.881,6.685)--(1.882,6.685)--(1.882,6.684)--(1.882,6.683)--(1.883,6.683)%
  --(1.883,6.682)--(1.884,6.682)--(1.884,6.681)--(1.884,6.680)--(1.885,6.680)--(1.885,6.679)%
  --(1.886,6.679)--(1.886,6.678)--(1.886,6.677)--(1.887,6.677)--(1.887,6.676)--(1.888,6.676)%
  --(1.888,6.675)--(1.888,6.674)--(1.889,6.674)--(1.889,6.673)--(1.890,6.673)--(1.890,6.672)%
  --(1.890,6.671)--(1.891,6.671)--(1.891,6.670)--(1.892,6.670)--(1.892,6.669)--(1.892,6.668)%
  --(1.893,6.668)--(1.893,6.667)--(1.894,6.667)--(1.894,6.666)--(1.894,6.665)--(1.895,6.665)%
  --(1.895,6.664)--(1.896,6.664)--(1.896,6.663)--(1.896,6.662)--(1.897,6.662)--(1.897,6.661)%
  --(1.898,6.661)--(1.898,6.660)--(1.898,6.659)--(1.899,6.659)--(1.899,6.658)--(1.900,6.658)%
  --(1.900,6.657)--(1.900,6.656)--(1.901,6.656)--(1.901,6.655)--(1.902,6.655)--(1.902,6.654)%
  --(1.902,6.653)--(1.903,6.653)--(1.903,6.652)--(1.904,6.652)--(1.904,6.651)--(1.904,6.650)%
  --(1.905,6.650)--(1.905,6.649)--(1.906,6.649)--(1.906,6.648)--(1.906,6.647)--(1.907,6.647)%
  --(1.907,6.646)--(1.907,6.645)--(1.908,6.645)--(1.908,6.644)--(1.909,6.644)--(1.909,6.643)%
  --(1.909,6.642)--(1.910,6.642)--(1.910,6.641)--(1.911,6.641)--(1.911,6.640)--(1.911,6.639)%
  --(1.912,6.639)--(1.912,6.638)--(1.913,6.638)--(1.913,6.637)--(1.913,6.636)--(1.914,6.636)%
  --(1.914,6.635)--(1.914,6.634)--(1.915,6.634)--(1.915,6.633)--(1.916,6.633)--(1.916,6.632)%
  --(1.916,6.631)--(1.917,6.631)--(1.917,6.630)--(1.918,6.630)--(1.918,6.629)--(1.918,6.628)%
  --(1.919,6.628)--(1.919,6.627)--(1.919,6.626)--(1.920,6.626)--(1.920,6.625)--(1.921,6.625)%
  --(1.921,6.624)--(1.921,6.623)--(1.922,6.623)--(1.922,6.622)--(1.922,6.621)--(1.923,6.621)%
  --(1.923,6.620)--(1.924,6.620)--(1.924,6.619)--(1.924,6.618)--(1.925,6.618)--(1.925,6.617)%
  --(1.925,6.616)--(1.926,6.616)--(1.926,6.615)--(1.927,6.615)--(1.927,6.614)--(1.927,6.613)%
  --(1.928,6.613)--(1.928,6.612)--(1.929,6.612)--(1.929,6.611)--(1.929,6.610)--(1.930,6.610)%
  --(1.930,6.609)--(1.930,6.608)--(1.931,6.608)--(1.931,6.607)--(1.932,6.606)--(1.932,6.605)%
  --(1.933,6.605)--(1.933,6.604)--(1.933,6.603)--(1.934,6.603)--(1.934,6.602)--(1.934,6.601)%
  --(1.935,6.601)--(1.935,6.600)--(1.936,6.600)--(1.936,6.599)--(1.936,6.598)--(1.937,6.598)%
  --(1.937,6.597)--(1.937,6.596)--(1.938,6.596)--(1.938,6.595)--(1.939,6.595)--(1.939,6.594)%
  --(1.939,6.593)--(1.940,6.593)--(1.940,6.592)--(1.940,6.591)--(1.941,6.591)--(1.941,6.590)%
  --(1.942,6.590)--(1.942,6.589)--(1.942,6.588)--(1.943,6.588)--(1.943,6.587)--(1.943,6.586)%
  --(1.944,6.586)--(1.944,6.585)--(1.944,6.584)--(1.945,6.584)--(1.945,6.583)--(1.946,6.583)%
  --(1.946,6.582)--(1.946,6.581)--(1.947,6.581)--(1.947,6.580)--(1.947,6.579)--(1.948,6.579)%
  --(1.948,6.578)--(1.948,6.577)--(1.949,6.577)--(1.949,6.576)--(1.950,6.576)--(1.950,6.575)%
  --(1.950,6.574)--(1.951,6.574)--(1.951,6.573)--(1.951,6.572)--(1.952,6.572)--(1.952,6.571)%
  --(1.952,6.570)--(1.953,6.570)--(1.953,6.569)--(1.954,6.569)--(1.954,6.568)--(1.954,6.567)%
  --(1.955,6.567)--(1.955,6.566)--(1.955,6.565)--(1.956,6.565)--(1.956,6.564)--(1.956,6.563)%
  --(1.957,6.563)--(1.957,6.562)--(1.958,6.562)--(1.958,6.561)--(1.958,6.560)--(1.959,6.560)%
  --(1.959,6.559)--(1.959,6.558)--(1.960,6.558)--(1.960,6.557)--(1.960,6.556)--(1.961,6.556)%
  --(1.961,6.555)--(1.962,6.554)--(1.962,6.553)--(1.963,6.553)--(1.963,6.552)--(1.963,6.551)%
  --(1.964,6.551)--(1.964,6.550)--(1.964,6.549)--(1.965,6.549)--(1.965,6.548)--(1.965,6.547)%
  --(1.966,6.547)--(1.966,6.546)--(1.967,6.545)--(1.967,6.544)--(1.968,6.544)--(1.968,6.543)%
  --(1.968,6.542)--(1.969,6.542)--(1.969,6.541)--(1.969,6.540)--(1.970,6.540)--(1.970,6.539)%
  --(1.970,6.538)--(1.971,6.538)--(1.971,6.537)--(1.971,6.536)--(1.972,6.536)--(1.972,6.535)%
  --(1.973,6.535)--(1.973,6.534)--(1.973,6.533)--(1.974,6.533)--(1.974,6.532)--(1.974,6.531)%
  --(1.975,6.531)--(1.975,6.530)--(1.975,6.529)--(1.976,6.529)--(1.976,6.528)--(1.976,6.527)%
  --(1.977,6.527)--(1.977,6.526)--(1.977,6.525)--(1.978,6.525)--(1.978,6.524)--(1.979,6.524)%
  --(1.979,6.523)--(1.979,6.522)--(1.980,6.522)--(1.980,6.521)--(1.980,6.520)--(1.981,6.520)%
  --(1.981,6.519)--(1.981,6.518)--(1.982,6.518)--(1.982,6.517)--(1.982,6.516)--(1.983,6.516)%
  --(1.983,6.515)--(1.983,6.514)--(1.984,6.514)--(1.984,6.513)--(1.984,6.512)--(1.985,6.512)%
  --(1.985,6.511)--(1.985,6.510)--(1.986,6.510)--(1.986,6.509)--(1.987,6.509)--(1.987,6.508)%
  --(1.987,6.507)--(1.988,6.507)--(1.988,6.506)--(1.988,6.505)--(1.989,6.505)--(1.989,6.504)%
  --(1.989,6.503)--(1.990,6.503)--(1.990,6.502)--(1.990,6.501)--(1.991,6.501)--(1.991,6.500)%
  --(1.991,6.499)--(1.992,6.499)--(1.992,6.498)--(1.992,6.497)--(1.993,6.497)--(1.993,6.496)%
  --(1.993,6.495)--(1.994,6.495)--(1.994,6.494)--(1.994,6.493)--(1.995,6.493)--(1.995,6.492)%
  --(1.996,6.492)--(1.996,6.491)--(1.996,6.490)--(1.997,6.490)--(1.997,6.489)--(1.997,6.488)%
  --(1.998,6.488)--(1.998,6.487)--(1.998,6.486)--(1.999,6.486)--(1.999,6.485)--(1.999,6.484)%
  --(2.000,6.484)--(2.000,6.483)--(2.000,6.482)--(2.001,6.482)--(2.001,6.481)--(2.001,6.480)%
  --(2.002,6.480)--(2.002,6.479)--(2.002,6.478)--(2.003,6.478)--(2.003,6.477)--(2.003,6.476)%
  --(2.004,6.476)--(2.004,6.475)--(2.004,6.474)--(2.005,6.474)--(2.005,6.473)--(2.005,6.472)%
  --(2.006,6.472)--(2.006,6.471)--(2.006,6.470)--(2.007,6.470)--(2.007,6.469)--(2.007,6.468)%
  --(2.008,6.468)--(2.008,6.467)--(2.008,6.466)--(2.009,6.466)--(2.009,6.465)--(2.010,6.464)%
  --(2.010,6.463)--(2.011,6.463)--(2.011,6.462)--(2.011,6.461)--(2.012,6.461)--(2.012,6.460)%
  --(2.012,6.459)--(2.013,6.459)--(2.013,6.458)--(2.013,6.457)--(2.014,6.457)--(2.014,6.456)%
  --(2.014,6.455)--(2.015,6.455)--(2.015,6.454)--(2.015,6.453)--(2.016,6.453)--(2.016,6.452)%
  --(2.016,6.451)--(2.017,6.451)--(2.017,6.450)--(2.017,6.449)--(2.018,6.449)--(2.018,6.448)%
  --(2.018,6.447)--(2.019,6.447)--(2.019,6.446)--(2.019,6.445)--(2.020,6.445)--(2.020,6.444)%
  --(2.020,6.443)--(2.021,6.443)--(2.021,6.442)--(2.021,6.441)--(2.022,6.441)--(2.022,6.440)%
  --(2.022,6.439)--(2.023,6.439)--(2.023,6.438)--(2.023,6.437)--(2.024,6.437)--(2.024,6.436)%
  --(2.024,6.435)--(2.025,6.435)--(2.025,6.434)--(2.025,6.433)--(2.026,6.433)--(2.026,6.432)%
  --(2.026,6.431)--(2.027,6.431)--(2.027,6.430)--(2.027,6.429)--(2.028,6.429)--(2.028,6.428)%
  --(2.028,6.427)--(2.029,6.427)--(2.029,6.426)--(2.029,6.425)--(2.030,6.425)--(2.030,6.424)%
  --(2.030,6.423)--(2.031,6.423)--(2.031,6.422)--(2.031,6.421)--(2.032,6.421)--(2.032,6.420)%
  --(2.032,6.419)--(2.033,6.419)--(2.033,6.418)--(2.033,6.417)--(2.034,6.417)--(2.034,6.416)%
  --(2.034,6.415)--(2.035,6.415)--(2.035,6.414)--(2.035,6.413)--(2.036,6.413)--(2.036,6.412)%
  --(2.036,6.411)--(2.037,6.411)--(2.037,6.410)--(2.037,6.409)--(2.038,6.409)--(2.038,6.408)%
  --(2.038,6.407)--(2.039,6.407)--(2.039,6.406)--(2.039,6.405)--(2.040,6.405)--(2.040,6.404)%
  --(2.040,6.403)--(2.041,6.403)--(2.041,6.402)--(2.041,6.401)--(2.042,6.401)--(2.042,6.400)%
  --(2.042,6.399)--(2.043,6.399)--(2.043,6.398)--(2.043,6.397)--(2.044,6.396)--(2.044,6.395)%
  --(2.044,6.394)--(2.045,6.394)--(2.045,6.393)--(2.045,6.392)--(2.046,6.392)--(2.046,6.391)%
  --(2.046,6.390)--(2.047,6.390)--(2.047,6.389)--(2.047,6.388)--(2.048,6.388)--(2.048,6.387)%
  --(2.048,6.386)--(2.049,6.386)--(2.049,6.385)--(2.049,6.384)--(2.050,6.384)--(2.050,6.383)%
  --(2.050,6.382)--(2.051,6.382)--(2.051,6.381)--(2.051,6.380)--(2.052,6.380)--(2.052,6.379)%
  --(2.052,6.378)--(2.053,6.378)--(2.053,6.377)--(2.053,6.376)--(2.054,6.376)--(2.054,6.375)%
  --(2.054,6.374)--(2.055,6.374)--(2.055,6.373)--(2.055,6.372)--(2.056,6.372)--(2.056,6.371)%
  --(2.056,6.370)--(2.057,6.370)--(2.057,6.369)--(2.057,6.368)--(2.058,6.368)--(2.058,6.367)%
  --(2.058,6.366)--(2.059,6.366)--(2.059,6.365)--(2.059,6.364)--(2.059,6.363)--(2.060,6.363)%
  --(2.060,6.362)--(2.060,6.361)--(2.061,6.361)--(2.061,6.360)--(2.061,6.359)--(2.062,6.359)%
  --(2.062,6.358)--(2.062,6.357)--(2.063,6.357)--(2.063,6.356)--(2.063,6.355)--(2.064,6.355)%
  --(2.064,6.354)--(2.064,6.353)--(2.065,6.353)--(2.065,6.352)--(2.065,6.351)--(2.066,6.351)%
  --(2.066,6.350)--(2.066,6.349)--(2.067,6.349)--(2.067,6.348)--(2.067,6.347)--(2.068,6.347)%
  --(2.068,6.346)--(2.068,6.345)--(2.069,6.345)--(2.069,6.344)--(2.069,6.343)--(2.070,6.343)%
  --(2.070,6.342)--(2.070,6.341)--(2.070,6.340)--(2.071,6.340)--(2.071,6.339)--(2.071,6.338)%
  --(2.072,6.338)--(2.072,6.337)--(2.072,6.336)--(2.073,6.336)--(2.073,6.335)--(2.073,6.334)%
  --(2.074,6.334)--(2.074,6.333)--(2.074,6.332)--(2.075,6.332)--(2.075,6.331)--(2.075,6.330)%
  --(2.076,6.330)--(2.076,6.329)--(2.076,6.328)--(2.077,6.328)--(2.077,6.327)--(2.077,6.326)%
  --(2.078,6.326)--(2.078,6.325)--(2.078,6.324)--(2.079,6.324)--(2.079,6.323)--(2.079,6.322)%
  --(2.079,6.321)--(2.080,6.321)--(2.080,6.320)--(2.080,6.319)--(2.081,6.319)--(2.081,6.318)%
  --(2.081,6.317)--(2.082,6.317)--(2.082,6.316)--(2.082,6.315)--(2.083,6.315)--(2.083,6.314)%
  --(2.083,6.313)--(2.084,6.313)--(2.084,6.312)--(2.084,6.311)--(2.085,6.311)--(2.085,6.310)%
  --(2.085,6.309)--(2.086,6.309)--(2.086,6.308)--(2.086,6.307)--(2.086,6.306)--(2.087,6.306)%
  --(2.087,6.305)--(2.087,6.304)--(2.088,6.304)--(2.088,6.303)--(2.088,6.302)--(2.089,6.302)%
  --(2.089,6.301)--(2.089,6.300)--(2.090,6.300)--(2.090,6.299)--(2.090,6.298)--(2.091,6.298)%
  --(2.091,6.297)--(2.091,6.296)--(2.092,6.296)--(2.092,6.295)--(2.092,6.294)--(2.093,6.294)%
  --(2.093,6.293)--(2.093,6.292)--(2.093,6.291)--(2.094,6.291)--(2.094,6.290)--(2.094,6.289)%
  --(2.095,6.289)--(2.095,6.288)--(2.095,6.287)--(2.096,6.287)--(2.096,6.286)--(2.096,6.285)%
  --(2.097,6.285)--(2.097,6.284)--(2.097,6.283)--(2.098,6.283)--(2.098,6.282)--(2.098,6.281)%
  --(2.099,6.281)--(2.099,6.280)--(2.099,6.279)--(2.100,6.279)--(2.100,6.278)--(2.100,6.277)%
  --(2.100,6.276)--(2.101,6.276)--(2.101,6.275)--(2.101,6.274)--(2.102,6.274)--(2.102,6.273)%
  --(2.102,6.272)--(2.103,6.272)--(2.103,6.271)--(2.103,6.270)--(2.104,6.270)--(2.104,6.269)%
  --(2.104,6.268)--(2.105,6.268)--(2.105,6.267)--(2.105,6.266)--(2.106,6.266)--(2.106,6.265)%
  --(2.106,6.264)--(2.106,6.263)--(2.107,6.263)--(2.107,6.262)--(2.107,6.261)--(2.108,6.261)%
  --(2.108,6.260)--(2.108,6.259)--(2.109,6.259)--(2.109,6.258)--(2.109,6.257)--(2.110,6.257)%
  --(2.110,6.256)--(2.110,6.255)--(2.111,6.255)--(2.111,6.254)--(2.111,6.253)--(2.111,6.252)%
  --(2.112,6.252)--(2.112,6.251)--(2.112,6.250)--(2.113,6.250)--(2.113,6.249)--(2.113,6.248)%
  --(2.114,6.248)--(2.114,6.247)--(2.114,6.246)--(2.115,6.246)--(2.115,6.245)--(2.115,6.244)%
  --(2.116,6.244)--(2.116,6.243)--(2.116,6.242)--(2.117,6.242)--(2.117,6.241)--(2.117,6.240)%
  --(2.117,6.239)--(2.118,6.239)--(2.118,6.238)--(2.118,6.237)--(2.119,6.237)--(2.119,6.236)%
  --(2.119,6.235)--(2.120,6.235)--(2.120,6.234)--(2.120,6.233)--(2.121,6.233)--(2.121,6.232)%
  --(2.121,6.231)--(2.122,6.231)--(2.122,6.230)--(2.122,6.229)--(2.122,6.228)--(2.123,6.228)%
  --(2.123,6.227)--(2.123,6.226)--(2.124,6.226)--(2.124,6.225)--(2.124,6.224)--(2.125,6.224)%
  --(2.125,6.223)--(2.125,6.222)--(2.126,6.222)--(2.126,6.221)--(2.126,6.220)--(2.127,6.220)%
  --(2.127,6.219)--(2.127,6.218)--(2.127,6.217)--(2.128,6.217)--(2.128,6.216)--(2.128,6.215)%
  --(2.129,6.215)--(2.129,6.214)--(2.129,6.213)--(2.130,6.213)--(2.130,6.212)--(2.130,6.211)%
  --(2.131,6.211)--(2.131,6.210)--(2.131,6.209)--(2.132,6.209)--(2.132,6.208)--(2.132,6.207)%
  --(2.132,6.206)--(2.133,6.206)--(2.133,6.205)--(2.133,6.204)--(2.134,6.204)--(2.134,6.203)%
  --(2.134,6.202)--(2.135,6.202)--(2.135,6.201)--(2.135,6.200)--(2.136,6.200)--(2.136,6.199)%
  --(2.136,6.198)--(2.137,6.198)--(2.137,6.197)--(2.137,6.196)--(2.137,6.195)--(2.138,6.195)%
  --(2.138,6.194)--(2.138,6.193)--(2.139,6.193)--(2.139,6.192)--(2.139,6.191)--(2.140,6.191)%
  --(2.140,6.190)--(2.140,6.189)--(2.141,6.189)--(2.141,6.188)--(2.141,6.187)--(2.141,6.186)%
  --(2.142,6.186)--(2.142,6.185)--(2.142,6.184)--(2.143,6.184)--(2.143,6.183)--(2.143,6.182)%
  --(2.144,6.182)--(2.144,6.181)--(2.144,6.180)--(2.145,6.180)--(2.145,6.179)--(2.145,6.178)%
  --(2.146,6.178)--(2.146,6.177)--(2.146,6.176)--(2.146,6.175)--(2.147,6.175)--(2.147,6.174)%
  --(2.147,6.173)--(2.148,6.173)--(2.148,6.172)--(2.148,6.171)--(2.149,6.171)--(2.149,6.170)%
  --(2.149,6.169)--(2.150,6.169)--(2.150,6.168)--(2.150,6.167)--(2.150,6.166)--(2.151,6.166)%
  --(2.151,6.165)--(2.151,6.164)--(2.152,6.164)--(2.152,6.163)--(2.152,6.162)--(2.153,6.162)%
  --(2.153,6.161)--(2.153,6.160)--(2.154,6.160)--(2.154,6.159)--(2.154,6.158)--(2.154,6.157)%
  --(2.155,6.157)--(2.155,6.156)--(2.155,6.155)--(2.156,6.155)--(2.156,6.154)--(2.156,6.153)%
  --(2.157,6.153)--(2.157,6.152)--(2.157,6.151)--(2.158,6.151)--(2.158,6.150)--(2.158,6.149)%
  --(2.159,6.149)--(2.159,6.148)--(2.159,6.147)--(2.159,6.146)--(2.160,6.146)--(2.160,6.145)%
  --(2.160,6.144)--(2.161,6.144)--(2.161,6.143)--(2.161,6.142)--(2.162,6.142)--(2.162,6.141)%
  --(2.162,6.140)--(2.163,6.140)--(2.163,6.139)--(2.163,6.138)--(2.163,6.137)--(2.164,6.137)%
  --(2.164,6.136)--(2.164,6.135)--(2.165,6.135)--(2.165,6.134)--(2.165,6.133)--(2.166,6.133)%
  --(2.166,6.132)--(2.166,6.131)--(2.167,6.131)--(2.167,6.130)--(2.167,6.129)--(2.167,6.128)%
  --(2.168,6.128)--(2.168,6.127)--(2.168,6.126)--(2.169,6.126)--(2.169,6.125)--(2.169,6.124)%
  --(2.170,6.124)--(2.170,6.123)--(2.170,6.122)--(2.171,6.122)--(2.171,6.121)--(2.171,6.120)%
  --(2.171,6.119)--(2.172,6.119)--(2.172,6.118)--(2.172,6.117)--(2.173,6.117)--(2.173,6.116)%
  --(2.173,6.115)--(2.174,6.115)--(2.174,6.114)--(2.174,6.113)--(2.175,6.113)--(2.175,6.112)%
  --(2.175,6.111)--(2.175,6.110)--(2.176,6.110)--(2.176,6.109)--(2.176,6.108)--(2.177,6.108)%
  --(2.177,6.107)--(2.177,6.106)--(2.178,6.106)--(2.178,6.105)--(2.178,6.104)--(2.179,6.104)%
  --(2.179,6.103)--(2.179,6.102)--(2.179,6.101)--(2.180,6.101)--(2.180,6.100)--(2.180,6.099)%
  --(2.181,6.099)--(2.181,6.098)--(2.181,6.097)--(2.182,6.097)--(2.182,6.096)--(2.182,6.095)%
  --(2.183,6.095)--(2.183,6.094)--(2.183,6.093)--(2.183,6.092)--(2.184,6.092)--(2.184,6.091)%
  --(2.184,6.090)--(2.185,6.090)--(2.185,6.089)--(2.185,6.088)--(2.186,6.088)--(2.186,6.087)%
  --(2.186,6.086)--(2.187,6.085)--(2.187,6.084)--(2.187,6.083)--(2.188,6.083)--(2.188,6.082)%
  --(2.188,6.081)--(2.189,6.081)--(2.189,6.080)--(2.189,6.079)--(2.190,6.079)--(2.190,6.078)%
  --(2.190,6.077)--(2.190,6.076)--(2.191,6.076)--(2.191,6.075)--(2.191,6.074)--(2.192,6.074)%
  --(2.192,6.073)--(2.192,6.072)--(2.193,6.072)--(2.193,6.071)--(2.193,6.070)--(2.194,6.070)%
  --(2.194,6.069)--(2.194,6.068)--(2.194,6.067)--(2.195,6.067)--(2.195,6.066)--(2.195,6.065)%
  --(2.196,6.065)--(2.196,6.064)--(2.196,6.063)--(2.197,6.063)--(2.197,6.062)--(2.197,6.061)%
  --(2.198,6.061)--(2.198,6.060)--(2.198,6.059)--(2.198,6.058)--(2.199,6.058)--(2.199,6.057)%
  --(2.199,6.056)--(2.200,6.056)--(2.200,6.055)--(2.200,6.054)--(2.201,6.054)--(2.201,6.053)%
  --(2.201,6.052)--(2.202,6.051)--(2.202,6.050)--(2.202,6.049)--(2.203,6.049)--(2.203,6.048)%
  --(2.203,6.047)--(2.204,6.047)--(2.204,6.046)--(2.204,6.045)--(2.205,6.045)--(2.205,6.044)%
  --(2.205,6.043)--(2.205,6.042)--(2.206,6.042)--(2.206,6.041)--(2.206,6.040)--(2.207,6.040)%
  --(2.207,6.039)--(2.207,6.038)--(2.208,6.038)--(2.208,6.037)--(2.208,6.036)--(2.209,6.036)%
  --(2.209,6.035)--(2.209,6.034)--(2.209,6.033)--(2.210,6.033)--(2.210,6.032)--(2.210,6.031)%
  --(2.211,6.031)--(2.211,6.030)--(2.211,6.029)--(2.212,6.029)--(2.212,6.028)--(2.212,6.027)%
  --(2.213,6.026)--(2.213,6.025)--(2.213,6.024)--(2.214,6.024)--(2.214,6.023)--(2.214,6.022)%
  --(2.215,6.022)--(2.215,6.021)--(2.215,6.020)--(2.216,6.020)--(2.216,6.019)--(2.216,6.018)%
  --(2.216,6.017)--(2.217,6.017)--(2.217,6.016)--(2.217,6.015)--(2.218,6.015)--(2.218,6.014)%
  --(2.218,6.013)--(2.219,6.013)--(2.219,6.012)--(2.219,6.011)--(2.220,6.011)--(2.220,6.010)%
  --(2.220,6.009)--(2.220,6.008)--(2.221,6.008)--(2.221,6.007)--(2.221,6.006)--(2.222,6.006)%
  --(2.222,6.005)--(2.222,6.004)--(2.223,6.004)--(2.223,6.003)--(2.223,6.002)--(2.223,6.001)%
  --(2.224,6.001)--(2.224,6.000)--(2.224,5.999)--(2.225,5.999)--(2.225,5.998)--(2.225,5.997)%
  --(2.226,5.997)--(2.226,5.996)--(2.226,5.995)--(2.227,5.995)--(2.227,5.994)--(2.227,5.993)%
  --(2.227,5.992)--(2.228,5.992)--(2.228,5.991)--(2.228,5.990)--(2.229,5.990)--(2.229,5.989)%
  --(2.229,5.988)--(2.230,5.988)--(2.230,5.987)--(2.230,5.986)--(2.230,5.985)--(2.231,5.985)%
  --(2.231,5.984)--(2.231,5.983)--(2.232,5.983)--(2.232,5.982)--(2.232,5.981)--(2.233,5.981)%
  --(2.233,5.980)--(2.233,5.979)--(2.234,5.979)--(2.234,5.978)--(2.234,5.977)--(2.234,5.976)%
  --(2.235,5.976)--(2.235,5.975)--(2.235,5.974)--(2.236,5.974)--(2.236,5.973)--(2.236,5.972)%
  --(2.237,5.972)--(2.237,5.971)--(2.237,5.970)--(2.237,5.969)--(2.238,5.969)--(2.238,5.968)%
  --(2.238,5.967)--(2.239,5.967)--(2.239,5.966)--(2.239,5.965)--(2.240,5.965)--(2.240,5.964)%
  --(2.240,5.963)--(2.241,5.963)--(2.241,5.962)--(2.241,5.961)--(2.241,5.960)--(2.242,5.960)%
  --(2.242,5.959)--(2.242,5.958)--(2.243,5.958)--(2.243,5.957)--(2.243,5.956)--(2.244,5.956)%
  --(2.244,5.955)--(2.244,5.954)--(2.244,5.953)--(2.245,5.953)--(2.245,5.952)--(2.245,5.951)%
  --(2.246,5.951)--(2.246,5.950)--(2.246,5.949)--(2.247,5.949)--(2.247,5.948)--(2.247,5.947)%
  --(2.248,5.947)--(2.248,5.946)--(2.248,5.945)--(2.248,5.944)--(2.249,5.944)--(2.249,5.943)%
  --(2.249,5.942)--(2.250,5.942)--(2.250,5.941)--(2.250,5.940)--(2.251,5.940)--(2.251,5.939)%
  --(2.251,5.938)--(2.251,5.937)--(2.252,5.937)--(2.252,5.936)--(2.252,5.935)--(2.253,5.935)%
  --(2.253,5.934)--(2.253,5.933)--(2.254,5.933)--(2.254,5.932)--(2.254,5.931)--(2.255,5.931)%
  --(2.255,5.930)--(2.255,5.929)--(2.255,5.928)--(2.256,5.928)--(2.256,5.927)--(2.256,5.926)%
  --(2.257,5.926)--(2.257,5.925)--(2.257,5.924)--(2.258,5.924)--(2.258,5.923)--(2.258,5.922)%
  --(2.258,5.921)--(2.259,5.921)--(2.259,5.920)--(2.259,5.919)--(2.260,5.919)--(2.260,5.918)%
  --(2.260,5.917)--(2.261,5.917)--(2.261,5.916)--(2.261,5.915)--(2.262,5.915)--(2.262,5.914)%
  --(2.262,5.913)--(2.262,5.912)--(2.263,5.912)--(2.263,5.911)--(2.263,5.910)--(2.264,5.910)%
  --(2.264,5.909)--(2.264,5.908)--(2.265,5.908)--(2.265,5.907)--(2.265,5.906)--(2.265,5.905)%
  --(2.266,5.905)--(2.266,5.904)--(2.266,5.903)--(2.267,5.903)--(2.267,5.902)--(2.267,5.901)%
  --(2.268,5.901)--(2.268,5.900)--(2.268,5.899)--(2.269,5.899)--(2.269,5.898)--(2.269,5.897)%
  --(2.269,5.896)--(2.270,5.896)--(2.270,5.895)--(2.270,5.894)--(2.271,5.894)--(2.271,5.893)%
  --(2.271,5.892)--(2.272,5.892)--(2.272,5.891)--(2.272,5.890)--(2.272,5.889)--(2.273,5.889)%
  --(2.273,5.888)--(2.273,5.887)--(2.274,5.887)--(2.274,5.886)--(2.274,5.885)--(2.275,5.885)%
  --(2.275,5.884)--(2.275,5.883)--(2.276,5.883)--(2.276,5.882)--(2.276,5.881)--(2.276,5.880)%
  --(2.277,5.880)--(2.277,5.879)--(2.277,5.878)--(2.278,5.878)--(2.278,5.877)--(2.278,5.876)%
  --(2.279,5.876)--(2.279,5.875)--(2.279,5.874)--(2.279,5.873)--(2.280,5.873)--(2.280,5.872)%
  --(2.280,5.871)--(2.281,5.871)--(2.281,5.870)--(2.281,5.869)--(2.282,5.869)--(2.282,5.868)%
  --(2.282,5.867)--(2.283,5.867)--(2.283,5.866)--(2.283,5.865)--(2.283,5.864)--(2.284,5.864)%
  --(2.284,5.863)--(2.284,5.862)--(2.285,5.862)--(2.285,5.861)--(2.285,5.860)--(2.286,5.860)%
  --(2.286,5.859)--(2.286,5.858)--(2.286,5.857)--(2.287,5.857)--(2.287,5.856)--(2.287,5.855)%
  --(2.288,5.855)--(2.288,5.854)--(2.288,5.853)--(2.289,5.853)--(2.289,5.852)--(2.289,5.851)%
  --(2.290,5.851)--(2.290,5.850)--(2.290,5.849)--(2.290,5.848)--(2.291,5.848)--(2.291,5.847)%
  --(2.291,5.846)--(2.292,5.846)--(2.292,5.845)--(2.292,5.844)--(2.293,5.844)--(2.293,5.843)%
  --(2.293,5.842)--(2.293,5.841)--(2.294,5.841)--(2.294,5.840)--(2.294,5.839)--(2.295,5.839)%
  --(2.295,5.838)--(2.295,5.837)--(2.296,5.837)--(2.296,5.836)--(2.296,5.835)--(2.297,5.835)%
  --(2.297,5.834)--(2.297,5.833)--(2.297,5.832)--(2.298,5.832)--(2.298,5.831)--(2.298,5.830)%
  --(2.299,5.830)--(2.299,5.829)--(2.299,5.828)--(2.300,5.828)--(2.300,5.827)--(2.300,5.826)%
  --(2.300,5.825)--(2.301,5.825)--(2.301,5.824)--(2.301,5.823)--(2.302,5.823)--(2.302,5.822)%
  --(2.302,5.821)--(2.303,5.821)--(2.303,5.820)--(2.303,5.819)--(2.304,5.819)--(2.304,5.818)%
  --(2.304,5.817)--(2.304,5.816)--(2.305,5.816)--(2.305,5.815)--(2.305,5.814)--(2.306,5.814)%
  --(2.306,5.813)--(2.306,5.812)--(2.307,5.812)--(2.307,5.811)--(2.307,5.810)--(2.307,5.809)%
  --(2.308,5.809)--(2.308,5.808)--(2.308,5.807)--(2.309,5.807)--(2.309,5.806)--(2.309,5.805)%
  --(2.310,5.805)--(2.310,5.804)--(2.310,5.803)--(2.311,5.803)--(2.311,5.802)--(2.311,5.801)%
  --(2.311,5.800)--(2.312,5.800)--(2.312,5.799)--(2.312,5.798)--(2.313,5.798)--(2.313,5.797)%
  --(2.313,5.796)--(2.314,5.796)--(2.314,5.795)--(2.314,5.794)--(2.315,5.794)--(2.315,5.793)%
  --(2.315,5.792)--(2.315,5.791)--(2.316,5.791)--(2.316,5.790)--(2.316,5.789)--(2.317,5.789)%
  --(2.317,5.788)--(2.317,5.787)--(2.318,5.787)--(2.318,5.786)--(2.318,5.785)--(2.318,5.784)%
  --(2.319,5.784)--(2.319,5.783)--(2.319,5.782)--(2.320,5.782)--(2.320,5.781)--(2.320,5.780)%
  --(2.321,5.780)--(2.321,5.779)--(2.321,5.778)--(2.322,5.778)--(2.322,5.777)--(2.322,5.776)%
  --(2.322,5.775)--(2.323,5.775)--(2.323,5.774)--(2.323,5.773)--(2.324,5.773)--(2.324,5.772)%
  --(2.324,5.771)--(2.325,5.771)--(2.325,5.770)--(2.325,5.769)--(2.325,5.768)--(2.326,5.768)%
  --(2.326,5.767)--(2.326,5.766)--(2.327,5.766)--(2.327,5.765)--(2.327,5.764)--(2.328,5.764)%
  --(2.328,5.763)--(2.328,5.762)--(2.329,5.762)--(2.329,5.761)--(2.329,5.760)--(2.329,5.759)%
  --(2.330,5.759)--(2.330,5.758)--(2.330,5.757)--(2.331,5.757)--(2.331,5.756)--(2.331,5.755)%
  --(2.332,5.755)--(2.332,5.754)--(2.332,5.753)--(2.333,5.753)--(2.333,5.752)--(2.333,5.751)%
  --(2.333,5.750)--(2.334,5.750)--(2.334,5.749)--(2.334,5.748)--(2.335,5.748)--(2.335,5.747)%
  --(2.335,5.746)--(2.336,5.746)--(2.336,5.745)--(2.336,5.744)--(2.336,5.743)--(2.337,5.743)%
  --(2.337,5.742)--(2.337,5.741)--(2.338,5.741)--(2.338,5.740)--(2.338,5.739)--(2.339,5.739)%
  --(2.339,5.738)--(2.339,5.737)--(2.340,5.737)--(2.340,5.736)--(2.340,5.735)--(2.340,5.734)%
  --(2.341,5.734)--(2.341,5.733)--(2.341,5.732)--(2.342,5.732)--(2.342,5.731)--(2.342,5.730)%
  --(2.343,5.730)--(2.343,5.729)--(2.343,5.728)--(2.344,5.728)--(2.344,5.727)--(2.344,5.726)%
  --(2.344,5.725)--(2.345,5.725)--(2.345,5.724)--(2.345,5.723)--(2.346,5.723)--(2.346,5.722)%
  --(2.346,5.721)--(2.347,5.721)--(2.347,5.720)--(2.347,5.719)--(2.348,5.719)--(2.348,5.718)%
  --(2.348,5.717)--(2.348,5.716)--(2.349,5.716)--(2.349,5.715)--(2.349,5.714)--(2.350,5.714)%
  --(2.350,5.713)--(2.350,5.712)--(2.351,5.712)--(2.351,5.711)--(2.351,5.710)--(2.351,5.709)%
  --(2.352,5.709)--(2.352,5.708)--(2.352,5.707)--(2.353,5.707)--(2.353,5.706)--(2.353,5.705)%
  --(2.354,5.705)--(2.354,5.704)--(2.354,5.703)--(2.355,5.703)--(2.355,5.702)--(2.355,5.701)%
  --(2.355,5.700)--(2.356,5.700)--(2.356,5.699)--(2.356,5.698)--(2.357,5.698)--(2.357,5.697)%
  --(2.357,5.696)--(2.358,5.696)--(2.358,5.695)--(2.358,5.694)--(2.359,5.694)--(2.359,5.693)%
  --(2.359,5.692)--(2.359,5.691)--(2.360,5.691)--(2.360,5.690)--(2.360,5.689)--(2.361,5.689)%
  --(2.361,5.688)--(2.361,5.687)--(2.362,5.687)--(2.362,5.686)--(2.362,5.685)--(2.363,5.685)%
  --(2.363,5.684)--(2.363,5.683)--(2.363,5.682)--(2.364,5.682)--(2.364,5.681)--(2.364,5.680)%
  --(2.365,5.680)--(2.365,5.679)--(2.365,5.678)--(2.366,5.678)--(2.366,5.677)--(2.366,5.676)%
  --(2.367,5.676)--(2.367,5.675)--(2.367,5.674)--(2.367,5.673)--(2.368,5.673)--(2.368,5.672)%
  --(2.368,5.671)--(2.369,5.671)--(2.369,5.670)--(2.369,5.669)--(2.370,5.669)--(2.370,5.668)%
  --(2.370,5.667)--(2.371,5.667)--(2.371,5.666)--(2.371,5.665)--(2.371,5.664)--(2.372,5.664)%
  --(2.372,5.663)--(2.372,5.662)--(2.373,5.662)--(2.373,5.661)--(2.373,5.660)--(2.374,5.660)%
  --(2.374,5.659)--(2.374,5.658)--(2.375,5.658)--(2.375,5.657)--(2.375,5.656)--(2.375,5.655)%
  --(2.376,5.655)--(2.376,5.654)--(2.376,5.653)--(2.377,5.653)--(2.377,5.652)--(2.377,5.651)%
  --(2.378,5.651)--(2.378,5.650)--(2.378,5.649)--(2.379,5.649)--(2.379,5.648)--(2.379,5.647)%
  --(2.379,5.646)--(2.380,5.646)--(2.380,5.645)--(2.380,5.644)--(2.381,5.644)--(2.381,5.643)%
  --(2.381,5.642)--(2.382,5.642)--(2.382,5.641)--(2.382,5.640)--(2.383,5.640)--(2.383,5.639)%
  --(2.383,5.638)--(2.383,5.637)--(2.384,5.637)--(2.384,5.636)--(2.384,5.635)--(2.385,5.635)%
  --(2.385,5.634)--(2.385,5.633)--(2.386,5.633)--(2.386,5.632)--(2.386,5.631)--(2.387,5.631)%
  --(2.387,5.630)--(2.387,5.629)--(2.387,5.628)--(2.388,5.628)--(2.388,5.627)--(2.388,5.626)%
  --(2.389,5.626)--(2.389,5.625)--(2.389,5.624)--(2.390,5.624)--(2.390,5.623)--(2.390,5.622)%
  --(2.391,5.622)--(2.391,5.621)--(2.391,5.620)--(2.391,5.619)--(2.392,5.619)--(2.392,5.618)%
  --(2.392,5.617)--(2.393,5.617)--(2.393,5.616)--(2.393,5.615)--(2.394,5.615)--(2.394,5.614)%
  --(2.394,5.613)--(2.395,5.613)--(2.395,5.612)--(2.395,5.611)--(2.395,5.610)--(2.396,5.610)%
  --(2.396,5.609)--(2.396,5.608)--(2.397,5.608)--(2.397,5.607)--(2.397,5.606)--(2.398,5.606)%
  --(2.398,5.605)--(2.398,5.604)--(2.399,5.604)--(2.399,5.603)--(2.399,5.602)--(2.400,5.602)%
  --(2.400,5.601)--(2.400,5.600)--(2.400,5.599)--(2.401,5.599)--(2.401,5.598)--(2.401,5.597)%
  --(2.402,5.597)--(2.402,5.596)--(2.402,5.595)--(2.403,5.595)--(2.403,5.594)--(2.403,5.593)%
  --(2.404,5.593)--(2.404,5.592)--(2.404,5.591)--(2.404,5.590)--(2.405,5.590)--(2.405,5.589)%
  --(2.405,5.588)--(2.406,5.588)--(2.406,5.587)--(2.406,5.586)--(2.407,5.586)--(2.407,5.585)%
  --(2.407,5.584)--(2.408,5.584)--(2.408,5.583)--(2.408,5.582)--(2.409,5.581)--(2.409,5.580)%
  --(2.409,5.579)--(2.410,5.579)--(2.410,5.578)--(2.410,5.577)--(2.411,5.577)--(2.411,5.576)%
  --(2.411,5.575)--(2.412,5.575)--(2.412,5.574)--(2.412,5.573)--(2.413,5.573)--(2.413,5.572)%
  --(2.413,5.571)--(2.413,5.570)--(2.414,5.570)--(2.414,5.569)--(2.414,5.568)--(2.415,5.568)%
  --(2.415,5.567)--(2.415,5.566)--(2.416,5.566)--(2.416,5.565)--(2.416,5.564)--(2.417,5.564)%
  --(2.417,5.563)--(2.417,5.562)--(2.417,5.561)--(2.418,5.561)--(2.418,5.560)--(2.418,5.559)%
  --(2.419,5.559)--(2.419,5.558)--(2.419,5.557)--(2.420,5.557)--(2.420,5.556)--(2.420,5.555)%
  --(2.421,5.555)--(2.421,5.554)--(2.421,5.553)--(2.422,5.553)--(2.422,5.552)--(2.422,5.551)%
  --(2.422,5.550)--(2.423,5.550)--(2.423,5.549)--(2.423,5.548)--(2.424,5.548)--(2.424,5.547)%
  --(2.424,5.546)--(2.425,5.546)--(2.425,5.545)--(2.425,5.544)--(2.426,5.544)--(2.426,5.543)%
  --(2.426,5.542)--(2.427,5.542)--(2.427,5.541)--(2.427,5.540)--(2.427,5.539)--(2.428,5.539)%
  --(2.428,5.538)--(2.428,5.537)--(2.429,5.537)--(2.429,5.536)--(2.429,5.535)--(2.430,5.535)%
  --(2.430,5.534)--(2.430,5.533)--(2.431,5.533)--(2.431,5.532)--(2.431,5.531)--(2.432,5.531)%
  --(2.432,5.530)--(2.432,5.529)--(2.432,5.528)--(2.433,5.528)--(2.433,5.527)--(2.433,5.526)%
  --(2.434,5.526)--(2.434,5.525)--(2.434,5.524)--(2.435,5.524)--(2.435,5.523)--(2.435,5.522)%
  --(2.436,5.522)--(2.436,5.521)--(2.436,5.520)--(2.437,5.519)--(2.437,5.518)--(2.437,5.517)%
  --(2.438,5.517)--(2.438,5.516)--(2.438,5.515)--(2.439,5.515)--(2.439,5.514)--(2.439,5.513)%
  --(2.440,5.513)--(2.440,5.512)--(2.440,5.511)--(2.441,5.511)--(2.441,5.510)--(2.441,5.509)%
  --(2.441,5.508)--(2.442,5.508)--(2.442,5.507)--(2.442,5.506)--(2.443,5.506)--(2.443,5.505)%
  --(2.443,5.504)--(2.444,5.504)--(2.444,5.503)--(2.444,5.502)--(2.445,5.502)--(2.445,5.501)%
  --(2.445,5.500)--(2.446,5.500)--(2.446,5.499)--(2.446,5.498)--(2.447,5.497)--(2.447,5.496)%
  --(2.447,5.495)--(2.448,5.495)--(2.448,5.494)--(2.448,5.493)--(2.449,5.493)--(2.449,5.492)%
  --(2.449,5.491)--(2.450,5.491)--(2.450,5.490)--(2.450,5.489)--(2.451,5.489)--(2.451,5.488)%
  --(2.451,5.487)--(2.452,5.487)--(2.452,5.486)--(2.452,5.485)--(2.452,5.484)--(2.453,5.484)%
  --(2.453,5.483)--(2.453,5.482)--(2.454,5.482)--(2.454,5.481)--(2.454,5.480)--(2.455,5.480)%
  --(2.455,5.479)--(2.455,5.478)--(2.456,5.478)--(2.456,5.477)--(2.456,5.476)--(2.457,5.476)%
  --(2.457,5.475)--(2.457,5.474)--(2.457,5.473)--(2.458,5.473)--(2.458,5.472)--(2.458,5.471)%
  --(2.459,5.471)--(2.459,5.470)--(2.459,5.469)--(2.460,5.469)--(2.460,5.468)--(2.460,5.467)%
  --(2.461,5.467)--(2.461,5.466)--(2.461,5.465)--(2.462,5.465)--(2.462,5.464)--(2.462,5.463)%
  --(2.463,5.463)--(2.463,5.462)--(2.463,5.461)--(2.463,5.460)--(2.464,5.460)--(2.464,5.459)%
  --(2.464,5.458)--(2.465,5.458)--(2.465,5.457)--(2.465,5.456)--(2.466,5.456)--(2.466,5.455)%
  --(2.466,5.454)--(2.467,5.454)--(2.467,5.453)--(2.467,5.452)--(2.468,5.452)--(2.468,5.451)%
  --(2.468,5.450)--(2.468,5.449)--(2.469,5.449)--(2.469,5.448)--(2.469,5.447)--(2.470,5.447)%
  --(2.470,5.446)--(2.470,5.445)--(2.471,5.445)--(2.471,5.444)--(2.471,5.443)--(2.472,5.443)%
  --(2.472,5.442)--(2.472,5.441)--(2.473,5.441)--(2.473,5.440)--(2.473,5.439)--(2.474,5.439)%
  --(2.474,5.438)--(2.474,5.437)--(2.474,5.436)--(2.475,5.436)--(2.475,5.435)--(2.475,5.434)%
  --(2.476,5.434)--(2.476,5.433)--(2.476,5.432)--(2.477,5.432)--(2.477,5.431)--(2.477,5.430)%
  --(2.478,5.430)--(2.478,5.429)--(2.478,5.428)--(2.479,5.428)--(2.479,5.427)--(2.479,5.426)%
  --(2.480,5.426)--(2.480,5.425)--(2.480,5.424)--(2.480,5.423)--(2.481,5.423)--(2.481,5.422)%
  --(2.481,5.421)--(2.482,5.421)--(2.482,5.420)--(2.482,5.419)--(2.483,5.419)--(2.483,5.418)%
  --(2.483,5.417)--(2.484,5.417)--(2.484,5.416)--(2.484,5.415)--(2.485,5.415)--(2.485,5.414)%
  --(2.485,5.413)--(2.486,5.413)--(2.486,5.412)--(2.486,5.411)--(2.487,5.410)--(2.487,5.409)%
  --(2.487,5.408)--(2.488,5.408)--(2.488,5.407)--(2.488,5.406)--(2.489,5.406)--(2.489,5.405)%
  --(2.489,5.404)--(2.490,5.404)--(2.490,5.403)--(2.490,5.402)--(2.491,5.402)--(2.491,5.401)%
  --(2.491,5.400)--(2.492,5.400)--(2.492,5.399)--(2.492,5.398)--(2.493,5.398)--(2.493,5.397)%
  --(2.493,5.396)--(2.493,5.395)--(2.494,5.395)--(2.494,5.394)--(2.494,5.393)--(2.495,5.393)%
  --(2.495,5.392)--(2.495,5.391)--(2.496,5.391)--(2.496,5.390)--(2.496,5.389)--(2.497,5.389)%
  --(2.497,5.388)--(2.497,5.387)--(2.498,5.387)--(2.498,5.386)--(2.498,5.385)--(2.499,5.385)%
  --(2.499,5.384)--(2.499,5.383)--(2.500,5.383)--(2.500,5.382)--(2.500,5.381)--(2.500,5.380)%
  --(2.501,5.380)--(2.501,5.379)--(2.501,5.378)--(2.502,5.378)--(2.502,5.377)--(2.502,5.376)%
  --(2.503,5.376)--(2.503,5.375)--(2.503,5.374)--(2.504,5.374)--(2.504,5.373)--(2.504,5.372)%
  --(2.505,5.372)--(2.505,5.371)--(2.505,5.370)--(2.506,5.370)--(2.506,5.369)--(2.506,5.368)%
  --(2.507,5.368)--(2.507,5.367)--(2.507,5.366)--(2.507,5.365)--(2.508,5.365)--(2.508,5.364)%
  --(2.508,5.363)--(2.509,5.363)--(2.509,5.362)--(2.509,5.361)--(2.510,5.361)--(2.510,5.360)%
  --(2.510,5.359)--(2.511,5.359)--(2.511,5.358)--(2.511,5.357)--(2.512,5.357)--(2.512,5.356)%
  --(2.512,5.355)--(2.513,5.355)--(2.513,5.354)--(2.513,5.353)--(2.514,5.353)--(2.514,5.352)%
  --(2.514,5.351)--(2.515,5.351)--(2.515,5.350)--(2.515,5.349)--(2.515,5.348)--(2.516,5.348)%
  --(2.516,5.347)--(2.516,5.346)--(2.517,5.346)--(2.517,5.345)--(2.517,5.344)--(2.518,5.344)%
  --(2.518,5.343)--(2.518,5.342)--(2.519,5.342)--(2.519,5.341)--(2.519,5.340)--(2.520,5.340)%
  --(2.520,5.339)--(2.520,5.338)--(2.521,5.338)--(2.521,5.337)--(2.521,5.336)--(2.522,5.336)%
  --(2.522,5.335)--(2.522,5.334)--(2.523,5.334)--(2.523,5.333)--(2.523,5.332)--(2.523,5.331)%
  --(2.524,5.331)--(2.524,5.330)--(2.524,5.329)--(2.525,5.329)--(2.525,5.328)--(2.525,5.327)%
  --(2.526,5.327)--(2.526,5.326)--(2.526,5.325)--(2.527,5.325)--(2.527,5.324)--(2.527,5.323)%
  --(2.528,5.323)--(2.528,5.322)--(2.528,5.321)--(2.529,5.321)--(2.529,5.320)--(2.529,5.319)%
  --(2.530,5.319)--(2.530,5.318)--(2.530,5.317)--(2.531,5.317)--(2.531,5.316)--(2.531,5.315)%
  --(2.532,5.315)--(2.532,5.314)--(2.532,5.313)--(2.532,5.312)--(2.533,5.312)--(2.533,5.311)%
  --(2.533,5.310)--(2.534,5.310)--(2.534,5.309)--(2.534,5.308)--(2.535,5.308)--(2.535,5.307)%
  --(2.535,5.306)--(2.536,5.306)--(2.536,5.305)--(2.536,5.304)--(2.537,5.304)--(2.537,5.303)%
  --(2.537,5.302)--(2.538,5.302)--(2.538,5.301)--(2.538,5.300)--(2.539,5.300)--(2.539,5.299)%
  --(2.539,5.298)--(2.540,5.298)--(2.540,5.297)--(2.540,5.296)--(2.541,5.296)--(2.541,5.295)%
  --(2.541,5.294)--(2.542,5.294)--(2.542,5.293)--(2.542,5.292)--(2.542,5.291)--(2.543,5.291)%
  --(2.543,5.290)--(2.543,5.289)--(2.544,5.289)--(2.544,5.288)--(2.544,5.287)--(2.545,5.287)%
  --(2.545,5.286)--(2.545,5.285)--(2.546,5.285)--(2.546,5.284)--(2.546,5.283)--(2.547,5.283)%
  --(2.547,5.282)--(2.547,5.281)--(2.548,5.281)--(2.548,5.280)--(2.548,5.279)--(2.549,5.279)%
  --(2.549,5.278)--(2.549,5.277)--(2.550,5.277)--(2.550,5.276)--(2.550,5.275)--(2.551,5.275)%
  --(2.551,5.274)--(2.551,5.273)--(2.552,5.273)--(2.552,5.272)--(2.552,5.271)--(2.553,5.271)%
  --(2.553,5.270)--(2.553,5.269)--(2.553,5.268)--(2.554,5.268)--(2.554,5.267)--(2.554,5.266)%
  --(2.555,5.266)--(2.555,5.265)--(2.555,5.264)--(2.556,5.264)--(2.556,5.263)--(2.556,5.262)%
  --(2.557,5.262)--(2.557,5.261)--(2.557,5.260)--(2.558,5.260)--(2.558,5.259)--(2.558,5.258)%
  --(2.559,5.258)--(2.559,5.257)--(2.559,5.256)--(2.560,5.256)--(2.560,5.255)--(2.560,5.254)%
  --(2.561,5.254)--(2.561,5.253)--(2.561,5.252)--(2.562,5.252)--(2.562,5.251)--(2.562,5.250)%
  --(2.563,5.250)--(2.563,5.249)--(2.563,5.248)--(2.564,5.248)--(2.564,5.247)--(2.564,5.246)%
  --(2.565,5.246)--(2.565,5.245)--(2.565,5.244)--(2.566,5.244)--(2.566,5.243)--(2.566,5.242)%
  --(2.566,5.241)--(2.567,5.241)--(2.567,5.240)--(2.567,5.239)--(2.568,5.239)--(2.568,5.238)%
  --(2.568,5.237)--(2.569,5.237)--(2.569,5.236)--(2.569,5.235)--(2.570,5.235)--(2.570,5.234)%
  --(2.570,5.233)--(2.571,5.233)--(2.571,5.232)--(2.571,5.231)--(2.572,5.231)--(2.572,5.230)%
  --(2.572,5.229)--(2.573,5.229)--(2.573,5.228)--(2.573,5.227)--(2.574,5.227)--(2.574,5.226)%
  --(2.574,5.225)--(2.575,5.225)--(2.575,5.224)--(2.575,5.223)--(2.576,5.223)--(2.576,5.222)%
  --(2.576,5.221)--(2.577,5.221)--(2.577,5.220)--(2.577,5.219)--(2.578,5.219)--(2.578,5.218)%
  --(2.578,5.217)--(2.579,5.217)--(2.579,5.216)--(2.579,5.215)--(2.580,5.215)--(2.580,5.214)%
  --(2.580,5.213)--(2.581,5.213)--(2.581,5.212)--(2.581,5.211)--(2.582,5.211)--(2.582,5.210)%
  --(2.582,5.209)--(2.583,5.208)--(2.583,5.207)--(2.583,5.206)--(2.584,5.206)--(2.584,5.205)%
  --(2.584,5.204)--(2.585,5.204)--(2.585,5.203)--(2.585,5.202)--(2.586,5.202)--(2.586,5.201)%
  --(2.586,5.200)--(2.587,5.200)--(2.587,5.199)--(2.587,5.198)--(2.588,5.198)--(2.588,5.197)%
  --(2.588,5.196)--(2.589,5.196)--(2.589,5.195)--(2.589,5.194)--(2.590,5.194)--(2.590,5.193)%
  --(2.590,5.192)--(2.591,5.192)--(2.591,5.191)--(2.591,5.190)--(2.592,5.190)--(2.592,5.189)%
  --(2.592,5.188)--(2.593,5.188)--(2.593,5.187)--(2.593,5.186)--(2.594,5.186)--(2.594,5.185)%
  --(2.594,5.184)--(2.595,5.184)--(2.595,5.183)--(2.595,5.182)--(2.596,5.182)--(2.596,5.181)%
  --(2.596,5.180)--(2.597,5.180)--(2.597,5.179)--(2.597,5.178)--(2.598,5.178)--(2.598,5.177)%
  --(2.598,5.176)--(2.599,5.176)--(2.599,5.175)--(2.599,5.174)--(2.600,5.174)--(2.600,5.173)%
  --(2.600,5.172)--(2.601,5.172)--(2.601,5.171)--(2.601,5.170)--(2.602,5.170)--(2.602,5.169)%
  --(2.602,5.168)--(2.603,5.168)--(2.603,5.167)--(2.603,5.166)--(2.604,5.166)--(2.604,5.165)%
  --(2.604,5.164)--(2.605,5.164)--(2.605,5.163)--(2.605,5.162)--(2.606,5.162)--(2.606,5.161)%
  --(2.606,5.160)--(2.607,5.160)--(2.607,5.159)--(2.607,5.158)--(2.608,5.158)--(2.608,5.157)%
  --(2.608,5.156)--(2.609,5.156)--(2.609,5.155)--(2.609,5.154)--(2.610,5.154)--(2.610,5.153)%
  --(2.610,5.152)--(2.611,5.152)--(2.611,5.151)--(2.611,5.150)--(2.612,5.149)--(2.612,5.148)%
  --(2.613,5.147)--(2.613,5.146)--(2.613,5.145)--(2.614,5.145)--(2.614,5.144)--(2.614,5.143)%
  --(2.615,5.143)--(2.615,5.142)--(2.615,5.141)--(2.616,5.141)--(2.616,5.140)--(2.616,5.139)%
  --(2.617,5.139)--(2.617,5.138)--(2.617,5.137)--(2.618,5.137)--(2.618,5.136)--(2.618,5.135)%
  --(2.619,5.135)--(2.619,5.134)--(2.619,5.133)--(2.620,5.133)--(2.620,5.132)--(2.620,5.131)%
  --(2.621,5.131)--(2.621,5.130)--(2.621,5.129)--(2.622,5.129)--(2.622,5.128)--(2.622,5.127)%
  --(2.623,5.127)--(2.623,5.126)--(2.623,5.125)--(2.624,5.125)--(2.624,5.124)--(2.624,5.123)%
  --(2.625,5.123)--(2.625,5.122)--(2.625,5.121)--(2.626,5.121)--(2.626,5.120)--(2.626,5.119)%
  --(2.627,5.119)--(2.627,5.118)--(2.627,5.117)--(2.628,5.117)--(2.628,5.116)--(2.628,5.115)%
  --(2.629,5.115)--(2.629,5.114)--(2.629,5.113)--(2.630,5.113)--(2.630,5.112)--(2.630,5.111)%
  --(2.631,5.111)--(2.631,5.110)--(2.631,5.109)--(2.632,5.109)--(2.632,5.108)--(2.632,5.107)%
  --(2.633,5.107)--(2.633,5.106)--(2.633,5.105)--(2.634,5.105)--(2.634,5.104)--(2.634,5.103)%
  --(2.635,5.103)--(2.635,5.102)--(2.635,5.101)--(2.636,5.101)--(2.636,5.100)--(2.636,5.099)%
  --(2.637,5.099)--(2.637,5.098)--(2.638,5.097)--(2.638,5.096)--(2.639,5.095)--(2.639,5.094)%
  --(2.640,5.094)--(2.640,5.093)--(2.640,5.092)--(2.641,5.092)--(2.641,5.091)--(2.641,5.090)%
  --(2.642,5.090)--(2.642,5.089)--(2.642,5.088)--(2.643,5.088)--(2.643,5.087)--(2.643,5.086)%
  --(2.644,5.086)--(2.644,5.085)--(2.644,5.084)--(2.645,5.084)--(2.645,5.083)--(2.645,5.082)%
  --(2.646,5.082)--(2.646,5.081)--(2.646,5.080)--(2.647,5.080)--(2.647,5.079)--(2.647,5.078)%
  --(2.648,5.078)--(2.648,5.077)--(2.648,5.076)--(2.649,5.076)--(2.649,5.075)--(2.649,5.074)%
  --(2.650,5.074)--(2.650,5.073)--(2.650,5.072)--(2.651,5.072)--(2.651,5.071)--(2.651,5.070)%
  --(2.652,5.070)--(2.652,5.069)--(2.652,5.068)--(2.653,5.068)--(2.653,5.067)--(2.653,5.066)%
  --(2.654,5.066)--(2.654,5.065)--(2.654,5.064)--(2.655,5.064)--(2.655,5.063)--(2.655,5.062)%
  --(2.656,5.062)--(2.656,5.061)--(2.656,5.060)--(2.657,5.060)--(2.657,5.059)--(2.657,5.058)%
  --(2.658,5.058)--(2.658,5.057)--(2.658,5.056)--(2.659,5.056)--(2.659,5.055)--(2.659,5.054)%
  --(2.660,5.054)--(2.660,5.053)--(2.660,5.052)--(2.661,5.052)--(2.661,5.051)--(2.661,5.050)%
  --(2.662,5.050)--(2.662,5.049)--(2.662,5.048)--(2.663,5.048)--(2.663,5.047)--(2.663,5.046)%
  --(2.664,5.046)--(2.664,5.045)--(2.664,5.044)--(2.665,5.044)--(2.665,5.043)--(2.665,5.042)%
  --(2.666,5.042)--(2.666,5.041)--(2.666,5.040)--(2.667,5.040)--(2.667,5.039)--(2.667,5.038)%
  --(2.668,5.038)--(2.668,5.037)--(2.669,5.037)--(2.669,5.036)--(2.669,5.035)--(2.670,5.035)%
  --(2.670,5.034)--(2.670,5.033)--(2.671,5.033)--(2.671,5.032)--(2.671,5.031)--(2.672,5.031)%
  --(2.672,5.030)--(2.672,5.029)--(2.673,5.029)--(2.673,5.028)--(2.673,5.027)--(2.674,5.027)%
  --(2.674,5.026)--(2.674,5.025)--(2.675,5.025)--(2.675,5.024)--(2.675,5.023)--(2.676,5.023)%
  --(2.676,5.022)--(2.676,5.021)--(2.677,5.021)--(2.677,5.020)--(2.677,5.019)--(2.678,5.019)%
  --(2.678,5.018)--(2.678,5.017)--(2.679,5.017)--(2.679,5.016)--(2.679,5.015)--(2.680,5.015)%
  --(2.680,5.014)--(2.680,5.013)--(2.681,5.013)--(2.681,5.012)--(2.681,5.011)--(2.682,5.011)%
  --(2.682,5.010)--(2.682,5.009)--(2.683,5.009)--(2.683,5.008)--(2.683,5.007)--(2.684,5.007)%
  --(2.684,5.006)--(2.685,5.006)--(2.685,5.005)--(2.685,5.004)--(2.686,5.004)--(2.686,5.003)%
  --(2.686,5.002)--(2.687,5.002)--(2.687,5.001)--(2.687,5.000)--(2.688,5.000)--(2.688,4.999)%
  --(2.688,4.998)--(2.689,4.998)--(2.689,4.997)--(2.689,4.996)--(2.690,4.996)--(2.690,4.995)%
  --(2.690,4.994)--(2.691,4.994)--(2.691,4.993)--(2.691,4.992)--(2.692,4.992)--(2.692,4.991)%
  --(2.692,4.990)--(2.693,4.990)--(2.693,4.989)--(2.693,4.988)--(2.694,4.988)--(2.694,4.987)%
  --(2.694,4.986)--(2.695,4.986)--(2.695,4.985)--(2.695,4.984)--(2.696,4.984)--(2.696,4.983)%
  --(2.696,4.982)--(2.697,4.982)--(2.697,4.981)--(2.698,4.981)--(2.698,4.980)--(2.698,4.979)%
  --(2.699,4.979)--(2.699,4.978)--(2.699,4.977)--(2.700,4.977)--(2.700,4.976)--(2.700,4.975)%
  --(2.701,4.975)--(2.701,4.974)--(2.701,4.973)--(2.702,4.973)--(2.702,4.972)--(2.702,4.971)%
  --(2.703,4.971)--(2.703,4.970)--(2.703,4.969)--(2.704,4.969)--(2.704,4.968)--(2.704,4.967)%
  --(2.705,4.967)--(2.705,4.966)--(2.705,4.965)--(2.706,4.965)--(2.706,4.964)--(2.706,4.963)%
  --(2.707,4.963)--(2.707,4.962)--(2.708,4.961)--(2.708,4.960)--(2.709,4.960)--(2.709,4.959)%
  --(2.709,4.958)--(2.710,4.958)--(2.710,4.957)--(2.710,4.956)--(2.711,4.956)--(2.711,4.955)%
  --(2.711,4.954)--(2.712,4.954)--(2.712,4.953)--(2.712,4.952)--(2.713,4.952)--(2.713,4.951)%
  --(2.713,4.950)--(2.714,4.950)--(2.714,4.949)--(2.714,4.948)--(2.715,4.948)--(2.715,4.947)%
  --(2.715,4.946)--(2.716,4.946)--(2.716,4.945)--(2.716,4.944)--(2.717,4.944)--(2.717,4.943)%
  --(2.718,4.943)--(2.718,4.942)--(2.718,4.941)--(2.719,4.941)--(2.719,4.940)--(2.719,4.939)%
  --(2.720,4.939)--(2.720,4.938)--(2.720,4.937)--(2.721,4.937)--(2.721,4.936)--(2.721,4.935)%
  --(2.722,4.935)--(2.722,4.934)--(2.722,4.933)--(2.723,4.933)--(2.723,4.932)--(2.723,4.931)%
  --(2.724,4.931)--(2.724,4.930)--(2.724,4.929)--(2.725,4.929)--(2.725,4.928)--(2.726,4.927)%
  --(2.726,4.926)--(2.727,4.926)--(2.727,4.925)--(2.727,4.924)--(2.728,4.924)--(2.728,4.923)%
  --(2.728,4.922)--(2.729,4.922)--(2.729,4.921)--(2.729,4.920)--(2.730,4.920)--(2.730,4.919)%
  --(2.730,4.918)--(2.731,4.918)--(2.731,4.917)--(2.731,4.916)--(2.732,4.916)--(2.732,4.915)%
  --(2.732,4.914)--(2.733,4.914)--(2.733,4.913)--(2.734,4.913)--(2.734,4.912)--(2.734,4.911)%
  --(2.735,4.911)--(2.735,4.910)--(2.735,4.909)--(2.736,4.909)--(2.736,4.908)--(2.736,4.907)%
  --(2.737,4.907)--(2.737,4.906)--(2.737,4.905)--(2.738,4.905)--(2.738,4.904)--(2.738,4.903)%
  --(2.739,4.903)--(2.739,4.902)--(2.739,4.901)--(2.740,4.901)--(2.740,4.900)--(2.740,4.899)%
  --(2.741,4.899)--(2.741,4.898)--(2.742,4.898)--(2.742,4.897)--(2.742,4.896)--(2.743,4.896)%
  --(2.743,4.895)--(2.743,4.894)--(2.744,4.894)--(2.744,4.893)--(2.744,4.892)--(2.745,4.892)%
  --(2.745,4.891)--(2.745,4.890)--(2.746,4.890)--(2.746,4.889)--(2.746,4.888)--(2.747,4.888)%
  --(2.747,4.887)--(2.748,4.886)--(2.748,4.885)--(2.749,4.885)--(2.749,4.884)--(2.749,4.883)%
  --(2.750,4.883)--(2.750,4.882)--(2.750,4.881)--(2.751,4.881)--(2.751,4.880)--(2.751,4.879)%
  --(2.752,4.879)--(2.752,4.878)--(2.752,4.877)--(2.753,4.877)--(2.753,4.876)--(2.753,4.875)%
  --(2.754,4.875)--(2.754,4.874)--(2.755,4.874)--(2.755,4.873)--(2.755,4.872)--(2.756,4.872)%
  --(2.756,4.871)--(2.756,4.870)--(2.757,4.870)--(2.757,4.869)--(2.757,4.868)--(2.758,4.868)%
  --(2.758,4.867)--(2.758,4.866)--(2.759,4.866)--(2.759,4.865)--(2.759,4.864)--(2.760,4.864)%
  --(2.760,4.863)--(2.761,4.863)--(2.761,4.862)--(2.761,4.861)--(2.762,4.861)--(2.762,4.860)%
  --(2.762,4.859)--(2.763,4.859)--(2.763,4.858)--(2.763,4.857)--(2.764,4.857)--(2.764,4.856)%
  --(2.764,4.855)--(2.765,4.855)--(2.765,4.854)--(2.765,4.853)--(2.766,4.853)--(2.766,4.852)%
  --(2.767,4.852)--(2.767,4.851)--(2.767,4.850)--(2.768,4.850)--(2.768,4.849)--(2.768,4.848)%
  --(2.769,4.848)--(2.769,4.847)--(2.769,4.846)--(2.770,4.846)--(2.770,4.845)--(2.770,4.844)%
  --(2.771,4.844)--(2.771,4.843)--(2.771,4.842)--(2.772,4.842)--(2.772,4.841)--(2.773,4.841)%
  --(2.773,4.840)--(2.773,4.839)--(2.774,4.839)--(2.774,4.838)--(2.774,4.837)--(2.775,4.837)%
  --(2.775,4.836)--(2.775,4.835)--(2.776,4.835)--(2.776,4.834)--(2.776,4.833)--(2.777,4.833)%
  --(2.777,4.832)--(2.778,4.832)--(2.778,4.831)--(2.778,4.830)--(2.779,4.830)--(2.779,4.829)%
  --(2.779,4.828)--(2.780,4.828)--(2.780,4.827)--(2.780,4.826)--(2.781,4.826)--(2.781,4.825)%
  --(2.781,4.824)--(2.782,4.824)--(2.782,4.823)--(2.782,4.822)--(2.783,4.822)--(2.783,4.821)%
  --(2.784,4.821)--(2.784,4.820)--(2.784,4.819)--(2.785,4.819)--(2.785,4.818)--(2.785,4.817)%
  --(2.786,4.817)--(2.786,4.816)--(2.786,4.815)--(2.787,4.815)--(2.787,4.814)--(2.787,4.813)%
  --(2.788,4.813)--(2.788,4.812)--(2.789,4.812)--(2.789,4.811)--(2.789,4.810)--(2.790,4.810)%
  --(2.790,4.809)--(2.790,4.808)--(2.791,4.808)--(2.791,4.807)--(2.791,4.806)--(2.792,4.806)%
  --(2.792,4.805)--(2.792,4.804)--(2.793,4.804)--(2.793,4.803)--(2.794,4.803)--(2.794,4.802)%
  --(2.794,4.801)--(2.795,4.801)--(2.795,4.800)--(2.795,4.799)--(2.796,4.799)--(2.796,4.798)%
  --(2.796,4.797)--(2.797,4.797)--(2.797,4.796)--(2.798,4.795)--(2.798,4.794)--(2.799,4.794)%
  --(2.799,4.793)--(2.799,4.792)--(2.800,4.792)--(2.800,4.791)--(2.800,4.790)--(2.801,4.790)%
  --(2.801,4.789)--(2.801,4.788)--(2.802,4.788)--(2.802,4.787)--(2.803,4.787)--(2.803,4.786)%
  --(2.803,4.785)--(2.804,4.785)--(2.804,4.784)--(2.804,4.783)--(2.805,4.783)--(2.805,4.782)%
  --(2.805,4.781)--(2.806,4.781)--(2.806,4.780)--(2.806,4.779)--(2.807,4.779)--(2.807,4.778)%
  --(2.808,4.778)--(2.808,4.777)--(2.808,4.776)--(2.809,4.776)--(2.809,4.775)--(2.809,4.774)%
  --(2.810,4.774)--(2.810,4.773)--(2.810,4.772)--(2.811,4.772)--(2.811,4.771)--(2.812,4.771)%
  --(2.812,4.770)--(2.812,4.769)--(2.813,4.769)--(2.813,4.768)--(2.813,4.767)--(2.814,4.767)%
  --(2.814,4.766)--(2.814,4.765)--(2.815,4.765)--(2.815,4.764)--(2.815,4.763)--(2.816,4.763)%
  --(2.816,4.762)--(2.817,4.762)--(2.817,4.761)--(2.817,4.760)--(2.818,4.760)--(2.818,4.759)%
  --(2.818,4.758)--(2.819,4.758)--(2.819,4.757)--(2.819,4.756)--(2.820,4.756)--(2.820,4.755)%
  --(2.821,4.755)--(2.821,4.754)--(2.821,4.753)--(2.822,4.753)--(2.822,4.752)--(2.822,4.751)%
  --(2.823,4.751)--(2.823,4.750)--(2.823,4.749)--(2.824,4.749)--(2.824,4.748)--(2.825,4.748)%
  --(2.825,4.747)--(2.825,4.746)--(2.826,4.746)--(2.826,4.745)--(2.826,4.744)--(2.827,4.744)%
  --(2.827,4.743)--(2.827,4.742)--(2.828,4.742)--(2.828,4.741)--(2.829,4.741)--(2.829,4.740)%
  --(2.829,4.739)--(2.830,4.739)--(2.830,4.738)--(2.830,4.737)--(2.831,4.737)--(2.831,4.736)%
  --(2.831,4.735)--(2.832,4.735)--(2.832,4.734)--(2.833,4.734)--(2.833,4.733)--(2.833,4.732)%
  --(2.834,4.732)--(2.834,4.731)--(2.834,4.730)--(2.835,4.730)--(2.835,4.729)--(2.835,4.728)%
  --(2.836,4.728)--(2.836,4.727)--(2.837,4.727)--(2.837,4.726)--(2.837,4.725)--(2.838,4.725)%
  --(2.838,4.724)--(2.838,4.723)--(2.839,4.723)--(2.839,4.722)--(2.839,4.721)--(2.840,4.721)%
  --(2.840,4.720)--(2.841,4.720)--(2.841,4.719)--(2.841,4.718)--(2.842,4.718)--(2.842,4.717)%
  --(2.842,4.716)--(2.843,4.716)--(2.843,4.715)--(2.843,4.714)--(2.844,4.714)--(2.844,4.713)%
  --(2.845,4.713)--(2.845,4.712)--(2.845,4.711)--(2.846,4.711)--(2.846,4.710)--(2.846,4.709)%
  --(2.847,4.709)--(2.847,4.708)--(2.848,4.708)--(2.848,4.707)--(2.848,4.706)--(2.849,4.706)%
  --(2.849,4.705)--(2.849,4.704)--(2.850,4.704)--(2.850,4.703)--(2.850,4.702)--(2.851,4.702)%
  --(2.851,4.701)--(2.852,4.701)--(2.852,4.700)--(2.852,4.699)--(2.853,4.699)--(2.853,4.698)%
  --(2.853,4.697)--(2.854,4.697)--(2.854,4.696)--(2.855,4.695)--(2.855,4.694)--(2.856,4.694)%
  --(2.856,4.693)--(2.856,4.692)--(2.857,4.692)--(2.857,4.691)--(2.857,4.690)--(2.858,4.690)%
  --(2.858,4.689)--(2.859,4.689)--(2.859,4.688)--(2.859,4.687)--(2.860,4.687)--(2.860,4.686)%
  --(2.860,4.685)--(2.861,4.685)--(2.861,4.684)--(2.861,4.683)--(2.862,4.683)--(2.862,4.682)%
  --(2.863,4.682)--(2.863,4.681)--(2.863,4.680)--(2.864,4.680)--(2.864,4.679)--(2.864,4.678)%
  --(2.865,4.678)--(2.865,4.677)--(2.866,4.677)--(2.866,4.676)--(2.866,4.675)--(2.867,4.675)%
  --(2.867,4.674)--(2.867,4.673)--(2.868,4.673)--(2.868,4.672)--(2.869,4.672)--(2.869,4.671)%
  --(2.869,4.670)--(2.870,4.670)--(2.870,4.669)--(2.870,4.668)--(2.871,4.668)--(2.871,4.667)%
  --(2.871,4.666)--(2.872,4.666)--(2.872,4.665)--(2.873,4.665)--(2.873,4.664)--(2.873,4.663)%
  --(2.874,4.663)--(2.874,4.662)--(2.874,4.661)--(2.875,4.661)--(2.875,4.660)--(2.876,4.660)%
  --(2.876,4.659)--(2.876,4.658)--(2.877,4.658)--(2.877,4.657)--(2.877,4.656)--(2.878,4.656)%
  --(2.878,4.655)--(2.879,4.655)--(2.879,4.654)--(2.879,4.653)--(2.880,4.653)--(2.880,4.652)%
  --(2.880,4.651)--(2.881,4.651)--(2.881,4.650)--(2.881,4.649)--(2.882,4.649)--(2.882,4.648)%
  --(2.883,4.648)--(2.883,4.647)--(2.883,4.646)--(2.884,4.646)--(2.884,4.645)--(2.884,4.644)%
  --(2.885,4.644)--(2.885,4.643)--(2.886,4.643)--(2.886,4.642)--(2.886,4.641)--(2.887,4.641)%
  --(2.887,4.640)--(2.887,4.639)--(2.888,4.639)--(2.888,4.638)--(2.889,4.638)--(2.889,4.637)%
  --(2.889,4.636)--(2.890,4.636)--(2.890,4.635)--(2.890,4.634)--(2.891,4.634)--(2.891,4.633)%
  --(2.892,4.633)--(2.892,4.632)--(2.892,4.631)--(2.893,4.631)--(2.893,4.630)--(2.893,4.629)%
  --(2.894,4.629)--(2.894,4.628)--(2.895,4.628)--(2.895,4.627)--(2.895,4.626)--(2.896,4.626)%
  --(2.896,4.625)--(2.896,4.624)--(2.897,4.624)--(2.897,4.623)--(2.898,4.623)--(2.898,4.622)%
  --(2.898,4.621)--(2.899,4.621)--(2.899,4.620)--(2.899,4.619)--(2.900,4.619)--(2.900,4.618)%
  --(2.901,4.618)--(2.901,4.617)--(2.901,4.616)--(2.902,4.616)--(2.902,4.615)--(2.902,4.614)%
  --(2.903,4.614)--(2.903,4.613)--(2.904,4.613)--(2.904,4.612)--(2.904,4.611)--(2.905,4.611)%
  --(2.905,4.610)--(2.905,4.609)--(2.906,4.609)--(2.906,4.608)--(2.907,4.608)--(2.907,4.607)%
  --(2.907,4.606)--(2.908,4.606)--(2.908,4.605)--(2.908,4.604)--(2.909,4.604)--(2.909,4.603)%
  --(2.910,4.603)--(2.910,4.602)--(2.910,4.601)--(2.911,4.601)--(2.911,4.600)--(2.911,4.599)%
  --(2.912,4.599)--(2.912,4.598)--(2.913,4.598)--(2.913,4.597)--(2.913,4.596)--(2.914,4.596)%
  --(2.914,4.595)--(2.914,4.594)--(2.915,4.594)--(2.915,4.593)--(2.916,4.593)--(2.916,4.592)%
  --(2.916,4.591)--(2.917,4.591)--(2.917,4.590)--(2.918,4.589)--(2.918,4.588)--(2.919,4.588)%
  --(2.919,4.587)--(2.919,4.586)--(2.920,4.586)--(2.920,4.585)--(2.921,4.585)--(2.921,4.584)%
  --(2.921,4.583)--(2.922,4.583)--(2.922,4.582)--(2.922,4.581)--(2.923,4.581)--(2.923,4.580)%
  --(2.924,4.580)--(2.924,4.579)--(2.924,4.578)--(2.925,4.578)--(2.925,4.577)--(2.925,4.576)%
  --(2.926,4.576)--(2.926,4.575)--(2.927,4.575)--(2.927,4.574)--(2.927,4.573)--(2.928,4.573)%
  --(2.928,4.572)--(2.928,4.571)--(2.929,4.571)--(2.929,4.570)--(2.930,4.570)--(2.930,4.569)%
  --(2.930,4.568)--(2.931,4.568)--(2.931,4.567)--(2.932,4.567)--(2.932,4.566)--(2.932,4.565)%
  --(2.933,4.565)--(2.933,4.564)--(2.933,4.563)--(2.934,4.563)--(2.934,4.562)--(2.935,4.562)%
  --(2.935,4.561)--(2.935,4.560)--(2.936,4.560)--(2.936,4.559)--(2.936,4.558)--(2.937,4.558)%
  --(2.937,4.557)--(2.938,4.557)--(2.938,4.556)--(2.938,4.555)--(2.939,4.555)--(2.939,4.554)%
  --(2.940,4.554)--(2.940,4.553)--(2.940,4.552)--(2.941,4.552)--(2.941,4.551)--(2.941,4.550)%
  --(2.942,4.550)--(2.942,4.549)--(2.943,4.549)--(2.943,4.548)--(2.943,4.547)--(2.944,4.547)%
  --(2.944,4.546)--(2.945,4.545)--(2.945,4.544)--(2.946,4.544)--(2.946,4.543)--(2.946,4.542)%
  --(2.947,4.542)--(2.947,4.541)--(2.948,4.541)--(2.948,4.540)--(2.948,4.539)--(2.949,4.539)%
  --(2.949,4.538)--(2.949,4.537)--(2.950,4.537)--(2.950,4.536)--(2.951,4.536)--(2.951,4.535)%
  --(2.951,4.534)--(2.952,4.534)--(2.952,4.533)--(2.953,4.533)--(2.953,4.532)--(2.953,4.531)%
  --(2.954,4.531)--(2.954,4.530)--(2.954,4.529)--(2.955,4.529)--(2.955,4.528)--(2.956,4.528)%
  --(2.956,4.527)--(2.956,4.526)--(2.957,4.526)--(2.957,4.525)--(2.958,4.525)--(2.958,4.524)%
  --(2.958,4.523)--(2.959,4.523)--(2.959,4.522)--(2.959,4.521)--(2.960,4.521)--(2.960,4.520)%
  --(2.961,4.520)--(2.961,4.519)--(2.961,4.518)--(2.962,4.518)--(2.962,4.517)--(2.963,4.517)%
  --(2.963,4.516)--(2.963,4.515)--(2.964,4.515)--(2.964,4.514)--(2.964,4.513)--(2.965,4.513)%
  --(2.965,4.512)--(2.966,4.512)--(2.966,4.511)--(2.966,4.510)--(2.967,4.510)--(2.967,4.509)%
  --(2.968,4.509)--(2.968,4.508)--(2.968,4.507)--(2.969,4.507)--(2.969,4.506)--(2.969,4.505)%
  --(2.970,4.505)--(2.970,4.504)--(2.971,4.504)--(2.971,4.503)--(2.971,4.502)--(2.972,4.502)%
  --(2.972,4.501)--(2.973,4.501)--(2.973,4.500)--(2.973,4.499)--(2.974,4.499)--(2.974,4.498)%
  --(2.975,4.498)--(2.975,4.497)--(2.975,4.496)--(2.976,4.496)--(2.976,4.495)--(2.976,4.494)%
  --(2.977,4.494)--(2.977,4.493)--(2.978,4.493)--(2.978,4.492)--(2.978,4.491)--(2.979,4.491)%
  --(2.979,4.490)--(2.980,4.490)--(2.980,4.489)--(2.980,4.488)--(2.981,4.488)--(2.981,4.487)%
  --(2.982,4.487)--(2.982,4.486)--(2.982,4.485)--(2.983,4.485)--(2.983,4.484)--(2.983,4.483)%
  --(2.984,4.483)--(2.984,4.482)--(2.985,4.482)--(2.985,4.481)--(2.985,4.480)--(2.986,4.480)%
  --(2.986,4.479)--(2.987,4.479)--(2.987,4.478)--(2.987,4.477)--(2.988,4.477)--(2.988,4.476)%
  --(2.989,4.476)--(2.989,4.475)--(2.989,4.474)--(2.990,4.474)--(2.990,4.473)--(2.990,4.472)%
  --(2.991,4.472)--(2.991,4.471)--(2.992,4.471)--(2.992,4.470)--(2.992,4.469)--(2.993,4.469)%
  --(2.993,4.468)--(2.994,4.468)--(2.994,4.467)--(2.994,4.466)--(2.995,4.466)--(2.995,4.465)%
  --(2.996,4.465)--(2.996,4.464)--(2.996,4.463)--(2.997,4.463)--(2.997,4.462)--(2.998,4.462)%
  --(2.998,4.461)--(2.998,4.460)--(2.999,4.460)--(2.999,4.459)--(2.999,4.458)--(3.000,4.458)%
  --(3.000,4.457)--(3.001,4.457)--(3.001,4.456)--(3.001,4.455)--(3.002,4.455)--(3.002,4.454)%
  --(3.003,4.454)--(3.003,4.453)--(3.003,4.452)--(3.004,4.452)--(3.004,4.451)--(3.005,4.451)%
  --(3.005,4.450)--(3.005,4.449)--(3.006,4.449)--(3.006,4.448)--(3.007,4.448)--(3.007,4.447)%
  --(3.007,4.446)--(3.008,4.446)--(3.008,4.445)--(3.009,4.445)--(3.009,4.444)--(3.009,4.443)%
  --(3.010,4.443)--(3.010,4.442)--(3.010,4.441)--(3.011,4.441)--(3.011,4.440)--(3.012,4.440)%
  --(3.012,4.439)--(3.012,4.438)--(3.013,4.438)--(3.013,4.437)--(3.014,4.437)--(3.014,4.436)%
  --(3.014,4.435)--(3.015,4.435)--(3.015,4.434)--(3.016,4.434)--(3.016,4.433)--(3.016,4.432)%
  --(3.017,4.432)--(3.017,4.431)--(3.018,4.431)--(3.018,4.430)--(3.018,4.429)--(3.019,4.429)%
  --(3.019,4.428)--(3.020,4.428)--(3.020,4.427)--(3.020,4.426)--(3.021,4.426)--(3.021,4.425)%
  --(3.022,4.425)--(3.022,4.424)--(3.022,4.423)--(3.023,4.423)--(3.023,4.422)--(3.024,4.422)%
  --(3.024,4.421)--(3.024,4.420)--(3.025,4.420)--(3.025,4.419)--(3.025,4.418)--(3.026,4.418)%
  --(3.026,4.417)--(3.027,4.417)--(3.027,4.416)--(3.027,4.415)--(3.028,4.415)--(3.028,4.414)%
  --(3.029,4.414)--(3.029,4.413)--(3.029,4.412)--(3.030,4.412)--(3.030,4.411)--(3.031,4.411)%
  --(3.031,4.410)--(3.031,4.409)--(3.032,4.409)--(3.032,4.408)--(3.033,4.408)--(3.033,4.407)%
  --(3.033,4.406)--(3.034,4.406)--(3.034,4.405)--(3.035,4.405)--(3.035,4.404)--(3.035,4.403)%
  --(3.036,4.403)--(3.036,4.402)--(3.037,4.402)--(3.037,4.401)--(3.037,4.400)--(3.038,4.400)%
  --(3.038,4.399)--(3.039,4.399)--(3.039,4.398)--(3.039,4.397)--(3.040,4.397)--(3.040,4.396)%
  --(3.041,4.396)--(3.041,4.395)--(3.041,4.394)--(3.042,4.394)--(3.042,4.393)--(3.043,4.393)%
  --(3.043,4.392)--(3.043,4.391)--(3.044,4.391)--(3.044,4.390)--(3.045,4.390)--(3.045,4.389)%
  --(3.045,4.388)--(3.046,4.388)--(3.046,4.387)--(3.047,4.387)--(3.047,4.386)--(3.047,4.385)%
  --(3.048,4.385)--(3.048,4.384)--(3.049,4.384)--(3.049,4.383)--(3.049,4.382)--(3.050,4.382)%
  --(3.050,4.381)--(3.051,4.381)--(3.051,4.380)--(3.051,4.379)--(3.052,4.379)--(3.052,4.378)%
  --(3.053,4.378)--(3.053,4.377)--(3.053,4.376)--(3.054,4.376)--(3.054,4.375)--(3.055,4.375)%
  --(3.055,4.374)--(3.055,4.373)--(3.056,4.373)--(3.056,4.372)--(3.057,4.372)--(3.057,4.371)%
  --(3.057,4.370)--(3.058,4.370)--(3.058,4.369)--(3.059,4.369)--(3.059,4.368)--(3.059,4.367)%
  --(3.060,4.367)--(3.060,4.366)--(3.061,4.366)--(3.061,4.365)--(3.061,4.364)--(3.062,4.364)%
  --(3.062,4.363)--(3.063,4.363)--(3.063,4.362)--(3.063,4.361)--(3.064,4.361)--(3.064,4.360)%
  --(3.065,4.360)--(3.065,4.359)--(3.065,4.358)--(3.066,4.358)--(3.066,4.357)--(3.067,4.357)%
  --(3.067,4.356)--(3.067,4.355)--(3.068,4.355)--(3.068,4.354)--(3.069,4.354)--(3.069,4.353)%
  --(3.070,4.352)--(3.070,4.351)--(3.071,4.351)--(3.071,4.350)--(3.072,4.350)--(3.072,4.349)%
  --(3.072,4.348)--(3.073,4.348)--(3.073,4.347)--(3.074,4.347)--(3.074,4.346)--(3.074,4.345)%
  --(3.075,4.345)--(3.075,4.344)--(3.076,4.344)--(3.076,4.343)--(3.076,4.342)--(3.077,4.342)%
  --(3.077,4.341)--(3.078,4.341)--(3.078,4.340)--(3.078,4.339)--(3.079,4.339)--(3.079,4.338)%
  --(3.080,4.338)--(3.080,4.337)--(3.080,4.336)--(3.081,4.336)--(3.081,4.335)--(3.082,4.335)%
  --(3.082,4.334)--(3.082,4.333)--(3.083,4.333)--(3.083,4.332)--(3.084,4.332)--(3.084,4.331)%
  --(3.084,4.330)--(3.085,4.330)--(3.085,4.329)--(3.086,4.329)--(3.086,4.328)--(3.087,4.328)%
  --(3.087,4.327)--(3.087,4.326)--(3.088,4.326)--(3.088,4.325)--(3.089,4.325)--(3.089,4.324)%
  --(3.089,4.323)--(3.090,4.323)--(3.090,4.322)--(3.091,4.322)--(3.091,4.321)--(3.091,4.320)%
  --(3.092,4.320)--(3.092,4.319)--(3.093,4.319)--(3.093,4.318)--(3.093,4.317)--(3.094,4.317)%
  --(3.094,4.316)--(3.095,4.316)--(3.095,4.315)--(3.095,4.314)--(3.096,4.314)--(3.096,4.313)%
  --(3.097,4.313)--(3.097,4.312)--(3.098,4.312)--(3.098,4.311)--(3.098,4.310)--(3.099,4.310)%
  --(3.099,4.309)--(3.100,4.309)--(3.100,4.308)--(3.100,4.307)--(3.101,4.307)--(3.101,4.306)%
  --(3.102,4.306)--(3.102,4.305)--(3.102,4.304)--(3.103,4.304)--(3.103,4.303)--(3.104,4.303)%
  --(3.104,4.302)--(3.105,4.302)--(3.105,4.301)--(3.105,4.300)--(3.106,4.300)--(3.106,4.299)%
  --(3.107,4.299)--(3.107,4.298)--(3.107,4.297)--(3.108,4.297)--(3.108,4.296)--(3.109,4.296)%
  --(3.109,4.295)--(3.109,4.294)--(3.110,4.294)--(3.110,4.293)--(3.111,4.293)--(3.111,4.292)%
  --(3.111,4.291)--(3.112,4.291)--(3.112,4.290)--(3.113,4.290)--(3.113,4.289)--(3.114,4.289)%
  --(3.114,4.288)--(3.114,4.287)--(3.115,4.287)--(3.115,4.286)--(3.116,4.286)--(3.116,4.285)%
  --(3.116,4.284)--(3.117,4.284)--(3.117,4.283)--(3.118,4.283)--(3.118,4.282)--(3.118,4.281)%
  --(3.119,4.281)--(3.119,4.280)--(3.120,4.280)--(3.120,4.279)--(3.121,4.279)--(3.121,4.278)%
  --(3.121,4.277)--(3.122,4.277)--(3.122,4.276)--(3.123,4.276)--(3.123,4.275)--(3.123,4.274)%
  --(3.124,4.274)--(3.124,4.273)--(3.125,4.273)--(3.125,4.272)--(3.126,4.272)--(3.126,4.271)%
  --(3.126,4.270)--(3.127,4.270)--(3.127,4.269)--(3.128,4.269)--(3.128,4.268)--(3.128,4.267)%
  --(3.129,4.267)--(3.129,4.266)--(3.130,4.266)--(3.130,4.265)--(3.130,4.264)--(3.131,4.264)%
  --(3.131,4.263)--(3.132,4.263)--(3.132,4.262)--(3.133,4.262)--(3.133,4.261)--(3.133,4.260)%
  --(3.134,4.260)--(3.134,4.259)--(3.135,4.259)--(3.135,4.258)--(3.135,4.257)--(3.136,4.257)%
  --(3.136,4.256)--(3.137,4.256)--(3.137,4.255)--(3.138,4.255)--(3.138,4.254)--(3.138,4.253)%
  --(3.139,4.253)--(3.139,4.252)--(3.140,4.252)--(3.140,4.251)--(3.140,4.250)--(3.141,4.250)%
  --(3.141,4.249)--(3.142,4.249)--(3.142,4.248)--(3.143,4.248)--(3.143,4.247)--(3.143,4.246)%
  --(3.144,4.246)--(3.144,4.245)--(3.145,4.245)--(3.145,4.244)--(3.145,4.243)--(3.146,4.243)%
  --(3.146,4.242)--(3.147,4.242)--(3.147,4.241)--(3.148,4.241)--(3.148,4.240)--(3.148,4.239)%
  --(3.149,4.239)--(3.149,4.238)--(3.150,4.238)--(3.150,4.237)--(3.150,4.236)--(3.151,4.236)%
  --(3.151,4.235)--(3.152,4.235)--(3.152,4.234)--(3.153,4.234)--(3.153,4.233)--(3.153,4.232)%
  --(3.154,4.232)--(3.154,4.231)--(3.155,4.231)--(3.155,4.230)--(3.155,4.229)--(3.156,4.229)%
  --(3.156,4.228)--(3.157,4.228)--(3.157,4.227)--(3.158,4.227)--(3.158,4.226)--(3.158,4.225)%
  --(3.159,4.225)--(3.159,4.224)--(3.160,4.224)--(3.160,4.223)--(3.161,4.223)--(3.161,4.222)%
  --(3.161,4.221)--(3.162,4.221)--(3.162,4.220)--(3.163,4.220)--(3.163,4.219)--(3.163,4.218)%
  --(3.164,4.218)--(3.164,4.217)--(3.165,4.217)--(3.165,4.216)--(3.166,4.216)--(3.166,4.215)%
  --(3.166,4.214)--(3.167,4.214)--(3.167,4.213)--(3.168,4.213)--(3.168,4.212)--(3.169,4.212)%
  --(3.169,4.211)--(3.169,4.210)--(3.170,4.210)--(3.170,4.209)--(3.171,4.209)--(3.171,4.208)%
  --(3.171,4.207)--(3.172,4.207)--(3.172,4.206)--(3.173,4.206)--(3.173,4.205)--(3.174,4.205)%
  --(3.174,4.204)--(3.174,4.203)--(3.175,4.203)--(3.175,4.202)--(3.176,4.202)--(3.176,4.201)%
  --(3.177,4.201)--(3.177,4.200)--(3.177,4.199)--(3.178,4.199)--(3.178,4.198)--(3.179,4.198)%
  --(3.179,4.197)--(3.179,4.196)--(3.180,4.196)--(3.180,4.195)--(3.181,4.195)--(3.181,4.194)%
  --(3.182,4.194)--(3.182,4.193)--(3.182,4.192)--(3.183,4.192)--(3.183,4.191)--(3.184,4.191)%
  --(3.184,4.190)--(3.185,4.190)--(3.185,4.189)--(3.185,4.188)--(3.186,4.188)--(3.186,4.187)%
  --(3.187,4.187)--(3.187,4.186)--(3.188,4.186)--(3.188,4.185)--(3.188,4.184)--(3.189,4.184)%
  --(3.189,4.183)--(3.190,4.183)--(3.190,4.182)--(3.191,4.182)--(3.191,4.181)--(3.191,4.180)%
  --(3.192,4.180)--(3.192,4.179)--(3.193,4.179)--(3.193,4.178)--(3.193,4.177)--(3.194,4.177)%
  --(3.194,4.176)--(3.195,4.176)--(3.195,4.175)--(3.196,4.175)--(3.196,4.174)--(3.196,4.173)%
  --(3.197,4.173)--(3.197,4.172)--(3.198,4.172)--(3.198,4.171)--(3.199,4.171)--(3.199,4.170)%
  --(3.199,4.169)--(3.200,4.169)--(3.200,4.168)--(3.201,4.168)--(3.201,4.167)--(3.202,4.167)%
  --(3.202,4.166)--(3.202,4.165)--(3.203,4.165)--(3.203,4.164)--(3.204,4.164)--(3.204,4.163)%
  --(3.205,4.163)--(3.205,4.162)--(3.205,4.161)--(3.206,4.161)--(3.206,4.160)--(3.207,4.160)%
  --(3.207,4.159)--(3.208,4.159)--(3.208,4.158)--(3.208,4.157)--(3.209,4.157)--(3.209,4.156)%
  --(3.210,4.156)--(3.210,4.155)--(3.211,4.155)--(3.211,4.154)--(3.211,4.153)--(3.212,4.153)%
  --(3.212,4.152)--(3.213,4.152)--(3.213,4.151)--(3.214,4.151)--(3.214,4.150)--(3.214,4.149)%
  --(3.215,4.149)--(3.215,4.148)--(3.216,4.148)--(3.216,4.147)--(3.217,4.147)--(3.217,4.146)%
  --(3.217,4.145)--(3.218,4.145)--(3.218,4.144)--(3.219,4.144)--(3.219,4.143)--(3.220,4.143)%
  --(3.220,4.142)--(3.220,4.141)--(3.221,4.141)--(3.221,4.140)--(3.222,4.140)--(3.222,4.139)%
  --(3.223,4.139)--(3.223,4.138)--(3.223,4.137)--(3.224,4.137)--(3.224,4.136)--(3.225,4.136)%
  --(3.225,4.135)--(3.226,4.135)--(3.226,4.134)--(3.226,4.133)--(3.227,4.133)--(3.227,4.132)%
  --(3.228,4.132)--(3.228,4.131)--(3.229,4.131)--(3.229,4.130)--(3.230,4.129)--(3.230,4.128)%
  --(3.231,4.128)--(3.231,4.127)--(3.232,4.127)--(3.232,4.126)--(3.233,4.126)--(3.233,4.125)%
  --(3.233,4.124)--(3.234,4.124)--(3.234,4.123)--(3.235,4.123)--(3.235,4.122)--(3.236,4.122)%
  --(3.236,4.121)--(3.236,4.120)--(3.237,4.120)--(3.237,4.119)--(3.238,4.119)--(3.238,4.118)%
  --(3.239,4.118)--(3.239,4.117)--(3.239,4.116)--(3.240,4.116)--(3.240,4.115)--(3.241,4.115)%
  --(3.241,4.114)--(3.242,4.114)--(3.242,4.113)--(3.242,4.112)--(3.243,4.112)--(3.243,4.111)%
  --(3.244,4.111)--(3.244,4.110)--(3.245,4.110)--(3.245,4.109)--(3.246,4.109)--(3.246,4.108)%
  --(3.246,4.107)--(3.247,4.107)--(3.247,4.106)--(3.248,4.106)--(3.248,4.105)--(3.249,4.105)%
  --(3.249,4.104)--(3.249,4.103)--(3.250,4.103)--(3.250,4.102)--(3.251,4.102)--(3.251,4.101)%
  --(3.252,4.101)--(3.252,4.100)--(3.253,4.099)--(3.253,4.098)--(3.254,4.098)--(3.254,4.097)%
  --(3.255,4.097)--(3.255,4.096)--(3.256,4.096)--(3.256,4.095)--(3.256,4.094)--(3.257,4.094)%
  --(3.257,4.093)--(3.258,4.093)--(3.258,4.092)--(3.259,4.092)--(3.259,4.091)--(3.259,4.090)%
  --(3.260,4.090)--(3.260,4.089)--(3.261,4.089)--(3.261,4.088)--(3.262,4.088)--(3.262,4.087)%
  --(3.263,4.087)--(3.263,4.086)--(3.263,4.085)--(3.264,4.085)--(3.264,4.084)--(3.265,4.084)%
  --(3.265,4.083)--(3.266,4.083)--(3.266,4.082)--(3.266,4.081)--(3.267,4.081)--(3.267,4.080)%
  --(3.268,4.080)--(3.268,4.079)--(3.269,4.079)--(3.269,4.078)--(3.270,4.078)--(3.270,4.077)%
  --(3.270,4.076)--(3.271,4.076)--(3.271,4.075)--(3.272,4.075)--(3.272,4.074)--(3.273,4.074)%
  --(3.273,4.073)--(3.274,4.073)--(3.274,4.072)--(3.274,4.071)--(3.275,4.071)--(3.275,4.070)%
  --(3.276,4.070)--(3.276,4.069)--(3.277,4.069)--(3.277,4.068)--(3.277,4.067)--(3.278,4.067)%
  --(3.278,4.066)--(3.279,4.066)--(3.279,4.065)--(3.280,4.065)--(3.280,4.064)--(3.281,4.064)%
  --(3.281,4.063)--(3.281,4.062)--(3.282,4.062)--(3.282,4.061)--(3.283,4.061)--(3.283,4.060)%
  --(3.284,4.060)--(3.284,4.059)--(3.285,4.059)--(3.285,4.058)--(3.285,4.057)--(3.286,4.057)%
  --(3.286,4.056)--(3.287,4.056)--(3.287,4.055)--(3.288,4.055)--(3.288,4.054)--(3.289,4.054)%
  --(3.289,4.053)--(3.289,4.052)--(3.290,4.052)--(3.290,4.051)--(3.291,4.051)--(3.291,4.050)%
  --(3.292,4.050)--(3.292,4.049)--(3.292,4.048)--(3.293,4.048)--(3.293,4.047)--(3.294,4.047)%
  --(3.294,4.046)--(3.295,4.046)--(3.295,4.045)--(3.296,4.045)--(3.296,4.044)--(3.296,4.043)%
  --(3.297,4.043)--(3.297,4.042)--(3.298,4.042)--(3.298,4.041)--(3.299,4.041)--(3.299,4.040)%
  --(3.300,4.040)--(3.300,4.039)--(3.300,4.038)--(3.301,4.038)--(3.301,4.037)--(3.302,4.037)%
  --(3.302,4.036)--(3.303,4.036)--(3.303,4.035)--(3.304,4.035)--(3.304,4.034)--(3.304,4.033)%
  --(3.305,4.033)--(3.305,4.032)--(3.306,4.032)--(3.306,4.031)--(3.307,4.031)--(3.307,4.030)%
  --(3.308,4.030)--(3.308,4.029)--(3.308,4.028)--(3.309,4.028)--(3.309,4.027)--(3.310,4.027)%
  --(3.310,4.026)--(3.311,4.026)--(3.311,4.025)--(3.312,4.025)--(3.312,4.024)--(3.313,4.024)%
  --(3.313,4.023)--(3.313,4.022)--(3.314,4.022)--(3.314,4.021)--(3.315,4.021)--(3.315,4.020)%
  --(3.316,4.020)--(3.316,4.019)--(3.317,4.019)--(3.317,4.018)--(3.317,4.017)--(3.318,4.017)%
  --(3.318,4.016)--(3.319,4.016)--(3.319,4.015)--(3.320,4.015)--(3.320,4.014)--(3.321,4.014)%
  --(3.321,4.013)--(3.321,4.012)--(3.322,4.012)--(3.322,4.011)--(3.323,4.011)--(3.323,4.010)%
  --(3.324,4.010)--(3.324,4.009)--(3.325,4.009)--(3.325,4.008)--(3.325,4.007)--(3.326,4.007)%
  --(3.326,4.006)--(3.327,4.006)--(3.327,4.005)--(3.328,4.005)--(3.328,4.004)--(3.329,4.004)%
  --(3.329,4.003)--(3.330,4.003)--(3.330,4.002)--(3.330,4.001)--(3.331,4.001)--(3.331,4.000)%
  --(3.332,4.000)--(3.332,3.999)--(3.333,3.999)--(3.333,3.998)--(3.334,3.998)--(3.334,3.997)%
  --(3.334,3.996)--(3.335,3.996)--(3.335,3.995)--(3.336,3.995)--(3.336,3.994)--(3.337,3.994)%
  --(3.337,3.993)--(3.338,3.993)--(3.338,3.992)--(3.339,3.992)--(3.339,3.991)--(3.339,3.990)%
  --(3.340,3.990)--(3.340,3.989)--(3.341,3.989)--(3.341,3.988)--(3.342,3.988)--(3.342,3.987)%
  --(3.343,3.987)--(3.343,3.986)--(3.344,3.986)--(3.344,3.985)--(3.344,3.984)--(3.345,3.984)%
  --(3.345,3.983)--(3.346,3.983)--(3.346,3.982)--(3.347,3.982)--(3.347,3.981)--(3.348,3.981)%
  --(3.348,3.980)--(3.348,3.979)--(3.349,3.979)--(3.349,3.978)--(3.350,3.978)--(3.350,3.977)%
  --(3.351,3.977)--(3.351,3.976)--(3.352,3.976)--(3.352,3.975)--(3.353,3.975)--(3.353,3.974)%
  --(3.353,3.973)--(3.354,3.973)--(3.354,3.972)--(3.355,3.972)--(3.355,3.971)--(3.356,3.971)%
  --(3.356,3.970)--(3.357,3.970)--(3.357,3.969)--(3.358,3.969)--(3.358,3.968)--(3.358,3.967)%
  --(3.359,3.967)--(3.359,3.966)--(3.360,3.966)--(3.360,3.965)--(3.361,3.965)--(3.361,3.964)%
  --(3.362,3.964)--(3.362,3.963)--(3.363,3.963)--(3.363,3.962)--(3.363,3.961)--(3.364,3.961)%
  --(3.364,3.960)--(3.365,3.960)--(3.365,3.959)--(3.366,3.959)--(3.366,3.958)--(3.367,3.958)%
  --(3.367,3.957)--(3.368,3.957)--(3.368,3.956)--(3.368,3.955)--(3.369,3.955)--(3.369,3.954)%
  --(3.370,3.954)--(3.370,3.953)--(3.371,3.953)--(3.371,3.952)--(3.372,3.952)--(3.372,3.951)%
  --(3.373,3.951)--(3.373,3.950)--(3.374,3.950)--(3.374,3.949)--(3.374,3.948)--(3.375,3.948)%
  --(3.375,3.947)--(3.376,3.947)--(3.376,3.946)--(3.377,3.946)--(3.377,3.945)--(3.378,3.945)%
  --(3.378,3.944)--(3.379,3.944)--(3.379,3.943)--(3.379,3.942)--(3.380,3.942)--(3.380,3.941)%
  --(3.381,3.941)--(3.381,3.940)--(3.382,3.940)--(3.382,3.939)--(3.383,3.939)--(3.383,3.938)%
  --(3.384,3.938)--(3.384,3.937)--(3.385,3.937)--(3.385,3.936)--(3.385,3.935)--(3.386,3.935)%
  --(3.386,3.934)--(3.387,3.934)--(3.387,3.933)--(3.388,3.933)--(3.388,3.932)--(3.389,3.932)%
  --(3.389,3.931)--(3.390,3.931)--(3.390,3.930)--(3.390,3.929)--(3.391,3.929)--(3.391,3.928)%
  --(3.392,3.928)--(3.392,3.927)--(3.393,3.927)--(3.393,3.926)--(3.394,3.926)--(3.394,3.925)%
  --(3.395,3.925)--(3.395,3.924)--(3.396,3.924)--(3.396,3.923)--(3.396,3.922)--(3.397,3.922)%
  --(3.397,3.921)--(3.398,3.921)--(3.398,3.920)--(3.399,3.920)--(3.399,3.919)--(3.400,3.919)%
  --(3.400,3.918)--(3.401,3.918)--(3.401,3.917)--(3.402,3.917)--(3.402,3.916)--(3.402,3.915)%
  --(3.403,3.915)--(3.403,3.914)--(3.404,3.914)--(3.404,3.913)--(3.405,3.913)--(3.405,3.912)%
  --(3.406,3.912)--(3.406,3.911)--(3.407,3.911)--(3.407,3.910)--(3.408,3.910)--(3.408,3.909)%
  --(3.408,3.908)--(3.409,3.908)--(3.409,3.907)--(3.410,3.907)--(3.410,3.906)--(3.411,3.906)%
  --(3.411,3.905)--(3.412,3.905)--(3.412,3.904)--(3.413,3.904)--(3.413,3.903)--(3.414,3.903)%
  --(3.414,3.902)--(3.415,3.902)--(3.415,3.901)--(3.415,3.900)--(3.416,3.900)--(3.416,3.899)%
  --(3.417,3.899)--(3.417,3.898)--(3.418,3.898)--(3.418,3.897)--(3.419,3.897)--(3.419,3.896)%
  --(3.420,3.896)--(3.420,3.895)--(3.421,3.895)--(3.421,3.894)--(3.421,3.893)--(3.422,3.893)%
  --(3.422,3.892)--(3.423,3.892)--(3.423,3.891)--(3.424,3.891)--(3.424,3.890)--(3.425,3.890)%
  --(3.425,3.889)--(3.426,3.889)--(3.426,3.888)--(3.427,3.888)--(3.427,3.887)--(3.428,3.887)%
  --(3.428,3.886)--(3.428,3.885)--(3.429,3.885)--(3.429,3.884)--(3.430,3.884)--(3.430,3.883)%
  --(3.431,3.883)--(3.431,3.882)--(3.432,3.882)--(3.432,3.881)--(3.433,3.881)--(3.433,3.880)%
  --(3.434,3.880)--(3.434,3.879)--(3.435,3.879)--(3.435,3.878)--(3.435,3.877)--(3.436,3.877)%
  --(3.436,3.876)--(3.437,3.876)--(3.437,3.875)--(3.438,3.875)--(3.438,3.874)--(3.439,3.874)%
  --(3.439,3.873)--(3.440,3.873)--(3.440,3.872)--(3.441,3.872)--(3.441,3.871)--(3.442,3.871)%
  --(3.442,3.870)--(3.443,3.870)--(3.443,3.869)--(3.443,3.868)--(3.444,3.868)--(3.444,3.867)%
  --(3.445,3.867)--(3.445,3.866)--(3.446,3.866)--(3.446,3.865)--(3.447,3.865)--(3.447,3.864)%
  --(3.448,3.864)--(3.448,3.863)--(3.449,3.863)--(3.449,3.862)--(3.450,3.862)--(3.450,3.861)%
  --(3.451,3.861)--(3.451,3.860)--(3.451,3.859)--(3.452,3.859)--(3.452,3.858)--(3.453,3.858)%
  --(3.453,3.857)--(3.454,3.857)--(3.454,3.856)--(3.455,3.856)--(3.455,3.855)--(3.456,3.855)%
  --(3.456,3.854)--(3.457,3.854)--(3.457,3.853)--(3.458,3.853)--(3.458,3.852)--(3.459,3.852)%
  --(3.459,3.851)--(3.459,3.850)--(3.460,3.850)--(3.460,3.849)--(3.461,3.849)--(3.461,3.848)%
  --(3.462,3.848)--(3.462,3.847)--(3.463,3.847)--(3.463,3.846)--(3.464,3.846)--(3.464,3.845)%
  --(3.465,3.845)--(3.465,3.844)--(3.466,3.844)--(3.466,3.843)--(3.467,3.843)--(3.467,3.842)%
  --(3.468,3.842)--(3.468,3.841)--(3.468,3.840)--(3.469,3.840)--(3.469,3.839)--(3.470,3.839)%
  --(3.470,3.838)--(3.471,3.838)--(3.471,3.837)--(3.472,3.837)--(3.472,3.836)--(3.473,3.836)%
  --(3.473,3.835)--(3.474,3.835)--(3.474,3.834)--(3.475,3.834)--(3.475,3.833)--(3.476,3.833)%
  --(3.476,3.832)--(3.477,3.832)--(3.477,3.831)--(3.477,3.830)--(3.478,3.830)--(3.478,3.829)%
  --(3.479,3.829)--(3.479,3.828)--(3.480,3.828)--(3.480,3.827)--(3.481,3.827)--(3.481,3.826)%
  --(3.482,3.826)--(3.482,3.825)--(3.483,3.825)--(3.483,3.824)--(3.484,3.824)--(3.484,3.823)%
  --(3.485,3.823)--(3.485,3.822)--(3.486,3.822)--(3.486,3.821)--(3.487,3.821)--(3.487,3.820)%
  --(3.487,3.819)--(3.488,3.819)--(3.488,3.818)--(3.489,3.818)--(3.489,3.817)--(3.490,3.817)%
  --(3.490,3.816)--(3.491,3.816)--(3.491,3.815)--(3.492,3.815)--(3.492,3.814)--(3.493,3.814)%
  --(3.493,3.813)--(3.494,3.813)--(3.494,3.812)--(3.495,3.812)--(3.495,3.811)--(3.496,3.811)%
  --(3.496,3.810)--(3.497,3.810)--(3.497,3.809)--(3.498,3.808)--(3.498,3.807)--(3.499,3.807)%
  --(3.499,3.806)--(3.500,3.806)--(3.500,3.805)--(3.501,3.805)--(3.501,3.804)--(3.502,3.804)%
  --(3.502,3.803)--(3.503,3.803)--(3.503,3.802)--(3.504,3.802)--(3.504,3.801)--(3.505,3.801)%
  --(3.505,3.800)--(3.506,3.800)--(3.506,3.799)--(3.507,3.799)--(3.507,3.798)--(3.508,3.798)%
  --(3.508,3.797)--(3.509,3.797)--(3.509,3.796)--(3.510,3.795)--(3.510,3.794)--(3.511,3.794)%
  --(3.511,3.793)--(3.512,3.793)--(3.512,3.792)--(3.513,3.792)--(3.513,3.791)--(3.514,3.791)%
  --(3.514,3.790)--(3.515,3.790)--(3.515,3.789)--(3.516,3.789)--(3.516,3.788)--(3.517,3.788)%
  --(3.517,3.787)--(3.518,3.787)--(3.518,3.786)--(3.519,3.786)--(3.519,3.785)--(3.520,3.785)%
  --(3.520,3.784)--(3.521,3.784)--(3.521,3.783)--(3.522,3.783)--(3.522,3.782)--(3.523,3.782)%
  --(3.523,3.781)--(3.523,3.780)--(3.524,3.780)--(3.524,3.779)--(3.525,3.779)--(3.525,3.778)%
  --(3.526,3.778)--(3.526,3.777)--(3.527,3.777)--(3.527,3.776)--(3.528,3.776)--(3.528,3.775)%
  --(3.529,3.775)--(3.529,3.774)--(3.530,3.774)--(3.530,3.773)--(3.531,3.773)--(3.531,3.772)%
  --(3.532,3.772)--(3.532,3.771)--(3.533,3.771)--(3.533,3.770)--(3.534,3.770)--(3.534,3.769)%
  --(3.535,3.769)--(3.535,3.768)--(3.536,3.768)--(3.536,3.767)--(3.537,3.767)--(3.537,3.766)%
  --(3.538,3.766)--(3.538,3.765)--(3.538,3.764)--(3.539,3.764)--(3.539,3.763)--(3.540,3.763)%
  --(3.540,3.762)--(3.541,3.762)--(3.541,3.761)--(3.542,3.761)--(3.542,3.760)--(3.543,3.760)%
  --(3.543,3.759)--(3.544,3.759)--(3.544,3.758)--(3.545,3.758)--(3.545,3.757)--(3.546,3.757)%
  --(3.546,3.756)--(3.547,3.756)--(3.547,3.755)--(3.548,3.755)--(3.548,3.754)--(3.549,3.754)%
  --(3.549,3.753)--(3.550,3.753)--(3.550,3.752)--(3.551,3.752)--(3.551,3.751)--(3.552,3.751)%
  --(3.552,3.750)--(3.553,3.750)--(3.553,3.749)--(3.554,3.749)--(3.554,3.748)--(3.555,3.748)%
  --(3.555,3.747)--(3.556,3.747)--(3.556,3.746)--(3.557,3.746)--(3.557,3.745)--(3.558,3.744)%
  --(3.558,3.743)--(3.559,3.743)--(3.559,3.742)--(3.560,3.742)--(3.560,3.741)--(3.561,3.741)%
  --(3.561,3.740)--(3.562,3.740)--(3.562,3.739)--(3.563,3.739)--(3.563,3.738)--(3.564,3.738)%
  --(3.564,3.737)--(3.565,3.737)--(3.565,3.736)--(3.566,3.736)--(3.566,3.735)--(3.567,3.735)%
  --(3.567,3.734)--(3.568,3.734)--(3.568,3.733)--(3.569,3.733)--(3.569,3.732)--(3.570,3.732)%
  --(3.570,3.731)--(3.571,3.731)--(3.571,3.730)--(3.572,3.730)--(3.572,3.729)--(3.573,3.729)%
  --(3.573,3.728)--(3.574,3.728)--(3.574,3.727)--(3.575,3.727)--(3.575,3.726)--(3.576,3.726)%
  --(3.576,3.725)--(3.577,3.725)--(3.577,3.724)--(3.578,3.724)--(3.578,3.723)--(3.579,3.723)%
  --(3.579,3.722)--(3.580,3.722)--(3.580,3.721)--(3.581,3.721)--(3.581,3.720)--(3.582,3.720)%
  --(3.582,3.719)--(3.583,3.719)--(3.583,3.718)--(3.584,3.718)--(3.584,3.717)--(3.585,3.717)%
  --(3.585,3.716)--(3.586,3.716)--(3.586,3.715)--(3.586,3.714)--(3.587,3.714)--(3.587,3.713)%
  --(3.588,3.713)--(3.588,3.712)--(3.589,3.712)--(3.589,3.711)--(3.590,3.711)--(3.590,3.710)%
  --(3.591,3.710)--(3.591,3.709)--(3.592,3.709)--(3.592,3.708)--(3.593,3.708)--(3.593,3.707)%
  --(3.594,3.707)--(3.594,3.706)--(3.595,3.706)--(3.595,3.705)--(3.596,3.705)--(3.596,3.704)%
  --(3.597,3.704)--(3.597,3.703)--(3.598,3.703)--(3.598,3.702)--(3.599,3.702)--(3.599,3.701)%
  --(3.600,3.701)--(3.600,3.700)--(3.601,3.700)--(3.601,3.699)--(3.602,3.699)--(3.602,3.698)%
  --(3.603,3.698)--(3.603,3.697)--(3.604,3.697)--(3.604,3.696)--(3.605,3.696)--(3.605,3.695)%
  --(3.606,3.695)--(3.606,3.694)--(3.607,3.694)--(3.607,3.693)--(3.608,3.693)--(3.608,3.692)%
  --(3.609,3.692)--(3.609,3.691)--(3.610,3.691)--(3.610,3.690)--(3.611,3.690)--(3.611,3.689)%
  --(3.612,3.689)--(3.612,3.688)--(3.613,3.688)--(3.613,3.687)--(3.614,3.687)--(3.614,3.686)%
  --(3.615,3.686)--(3.615,3.685)--(3.616,3.685)--(3.616,3.684)--(3.617,3.684)--(3.617,3.683)%
  --(3.618,3.683)--(3.618,3.682)--(3.619,3.682)--(3.619,3.681)--(3.620,3.681)--(3.620,3.680)%
  --(3.621,3.680)--(3.621,3.679)--(3.622,3.679)--(3.622,3.678)--(3.623,3.678)--(3.623,3.677)%
  --(3.624,3.677)--(3.624,3.676)--(3.625,3.676)--(3.625,3.675)--(3.626,3.675)--(3.626,3.674)%
  --(3.627,3.674)--(3.627,3.673)--(3.628,3.673)--(3.628,3.672)--(3.629,3.672)--(3.629,3.671)%
  --(3.630,3.671)--(3.630,3.670)--(3.631,3.670)--(3.631,3.669)--(3.632,3.669)--(3.632,3.668)%
  --(3.633,3.668)--(3.633,3.667)--(3.634,3.667)--(3.634,3.666)--(3.635,3.666)--(3.635,3.665)%
  --(3.636,3.665)--(3.636,3.664)--(3.637,3.664)--(3.637,3.663)--(3.638,3.663)--(3.638,3.662)%
  --(3.639,3.662)--(3.639,3.661)--(3.640,3.661)--(3.640,3.660)--(3.641,3.660)--(3.641,3.659)%
  --(3.642,3.659)--(3.642,3.658)--(3.643,3.658)--(3.643,3.657)--(3.644,3.657)--(3.644,3.656)%
  --(3.645,3.656)--(3.645,3.655)--(3.646,3.655)--(3.646,3.654)--(3.647,3.654)--(3.647,3.653)%
  --(3.648,3.653)--(3.648,3.652)--(3.649,3.652)--(3.649,3.651)--(3.650,3.651)--(3.650,3.650)%
  --(3.651,3.650)--(3.651,3.649)--(3.652,3.649)--(3.652,3.648)--(3.653,3.648)--(3.653,3.647)%
  --(3.654,3.647)--(3.654,3.646)--(3.655,3.646)--(3.655,3.645)--(3.656,3.645)--(3.656,3.644)%
  --(3.657,3.644)--(3.657,3.643)--(3.658,3.643)--(3.658,3.642)--(3.659,3.642)--(3.660,3.642)%
  --(3.660,3.641)--(3.661,3.641)--(3.661,3.640)--(3.662,3.640)--(3.662,3.639)--(3.663,3.639)%
  --(3.663,3.638)--(3.664,3.638)--(3.664,3.637)--(3.665,3.637)--(3.665,3.636)--(3.666,3.636)%
  --(3.666,3.635)--(3.667,3.635)--(3.667,3.634)--(3.668,3.634)--(3.668,3.633)--(3.669,3.633)%
  --(3.669,3.632)--(3.670,3.632)--(3.670,3.631)--(3.671,3.631)--(3.671,3.630)--(3.672,3.630)%
  --(3.672,3.629)--(3.673,3.629)--(3.673,3.628)--(3.674,3.628)--(3.674,3.627)--(3.675,3.627)%
  --(3.675,3.626)--(3.676,3.626)--(3.676,3.625)--(3.677,3.625)--(3.677,3.624)--(3.678,3.624)%
  --(3.678,3.623)--(3.679,3.623)--(3.679,3.622)--(3.680,3.622)--(3.680,3.621)--(3.681,3.621)%
  --(3.681,3.620)--(3.682,3.620)--(3.682,3.619)--(3.683,3.619)--(3.683,3.618)--(3.684,3.618)%
  --(3.684,3.617)--(3.685,3.617)--(3.685,3.616)--(3.686,3.616)--(3.686,3.615)--(3.687,3.615)%
  --(3.687,3.614)--(3.688,3.614)--(3.689,3.613)--(3.690,3.613)--(3.690,3.612)--(3.691,3.612)%
  --(3.691,3.611)--(3.692,3.611)--(3.692,3.610)--(3.693,3.610)--(3.693,3.609)--(3.694,3.609)%
  --(3.694,3.608)--(3.695,3.608)--(3.695,3.607)--(3.696,3.607)--(3.696,3.606)--(3.697,3.606)%
  --(3.697,3.605)--(3.698,3.605)--(3.698,3.604)--(3.699,3.604)--(3.699,3.603)--(3.700,3.603)%
  --(3.700,3.602)--(3.701,3.602)--(3.701,3.601)--(3.702,3.601)--(3.702,3.600)--(3.703,3.600)%
  --(3.703,3.599)--(3.704,3.599)--(3.704,3.598)--(3.705,3.598)--(3.705,3.597)--(3.706,3.597)%
  --(3.706,3.596)--(3.707,3.596)--(3.707,3.595)--(3.708,3.595)--(3.708,3.594)--(3.709,3.594)%
  --(3.710,3.594)--(3.710,3.593)--(3.711,3.593)--(3.711,3.592)--(3.712,3.592)--(3.712,3.591)%
  --(3.713,3.591)--(3.713,3.590)--(3.714,3.590)--(3.714,3.589)--(3.715,3.589)--(3.715,3.588)%
  --(3.716,3.588)--(3.716,3.587)--(3.717,3.587)--(3.717,3.586)--(3.718,3.586)--(3.718,3.585)%
  --(3.719,3.585)--(3.719,3.584)--(3.720,3.584)--(3.720,3.583)--(3.721,3.583)--(3.721,3.582)%
  --(3.722,3.582)--(3.722,3.581)--(3.723,3.581)--(3.723,3.580)--(3.724,3.580)--(3.724,3.579)%
  --(3.725,3.579)--(3.726,3.579)--(3.726,3.578)--(3.727,3.578)--(3.727,3.577)--(3.728,3.577)%
  --(3.728,3.576)--(3.729,3.576)--(3.729,3.575)--(3.730,3.575)--(3.730,3.574)--(3.731,3.574)%
  --(3.731,3.573)--(3.732,3.573)--(3.732,3.572)--(3.733,3.572)--(3.733,3.571)--(3.734,3.571)%
  --(3.734,3.570)--(3.735,3.570)--(3.735,3.569)--(3.736,3.569)--(3.736,3.568)--(3.737,3.568)%
  --(3.737,3.567)--(3.738,3.567)--(3.738,3.566)--(3.739,3.566)--(3.740,3.566)--(3.740,3.565)%
  --(3.741,3.565)--(3.741,3.564)--(3.742,3.564)--(3.742,3.563)--(3.743,3.563)--(3.743,3.562)%
  --(3.744,3.562)--(3.744,3.561)--(3.745,3.561)--(3.745,3.560)--(3.746,3.560)--(3.746,3.559)%
  --(3.747,3.559)--(3.747,3.558)--(3.748,3.558)--(3.748,3.557)--(3.749,3.557)--(3.749,3.556)%
  --(3.750,3.556)--(3.750,3.555)--(3.751,3.555)--(3.751,3.554)--(3.752,3.554)--(3.753,3.554)%
  --(3.753,3.553)--(3.754,3.553)--(3.754,3.552)--(3.755,3.552)--(3.755,3.551)--(3.756,3.551)%
  --(3.756,3.550)--(3.757,3.550)--(3.757,3.549)--(3.758,3.549)--(3.758,3.548)--(3.759,3.548)%
  --(3.759,3.547)--(3.760,3.547)--(3.760,3.546)--(3.761,3.546)--(3.761,3.545)--(3.762,3.545)%
  --(3.762,3.544)--(3.763,3.544)--(3.763,3.543)--(3.764,3.543)--(3.765,3.543)--(3.765,3.542)%
  --(3.766,3.542)--(3.766,3.541)--(3.767,3.541)--(3.767,3.540)--(3.768,3.540)--(3.768,3.539)%
  --(3.769,3.539)--(3.769,3.538)--(3.770,3.538)--(3.770,3.537)--(3.771,3.537)--(3.771,3.536)%
  --(3.772,3.536)--(3.772,3.535)--(3.773,3.535)--(3.773,3.534)--(3.774,3.534)--(3.775,3.534)%
  --(3.775,3.533)--(3.776,3.533)--(3.776,3.532)--(3.777,3.532)--(3.777,3.531)--(3.778,3.531)%
  --(3.778,3.530)--(3.779,3.530)--(3.779,3.529)--(3.780,3.529)--(3.780,3.528)--(3.781,3.528)%
  --(3.781,3.527)--(3.782,3.527)--(3.782,3.526)--(3.783,3.526)--(3.783,3.525)--(3.784,3.525)%
  --(3.784,3.524)--(3.785,3.524)--(3.786,3.524)--(3.786,3.523)--(3.787,3.523)--(3.787,3.522)%
  --(3.788,3.522)--(3.788,3.521)--(3.789,3.521)--(3.789,3.520)--(3.790,3.520)--(3.790,3.519)%
  --(3.791,3.519)--(3.791,3.518)--(3.792,3.518)--(3.792,3.517)--(3.793,3.517)--(3.793,3.516)%
  --(3.794,3.516)--(3.795,3.516)--(3.795,3.515)--(3.796,3.515)--(3.796,3.514)--(3.797,3.514)%
  --(3.797,3.513)--(3.798,3.513)--(3.798,3.512)--(3.799,3.512)--(3.799,3.511)--(3.800,3.511)%
  --(3.800,3.510)--(3.801,3.510)--(3.801,3.509)--(3.802,3.509)--(3.802,3.508)--(3.803,3.508)%
  --(3.804,3.508)--(3.804,3.507)--(3.805,3.507)--(3.805,3.506)--(3.806,3.506)--(3.806,3.505)%
  --(3.807,3.505)--(3.807,3.504)--(3.808,3.504)--(3.808,3.503)--(3.809,3.503)--(3.809,3.502)%
  --(3.810,3.502)--(3.810,3.501)--(3.811,3.501)--(3.811,3.500)--(3.812,3.500)--(3.813,3.500)%
  --(3.813,3.499)--(3.814,3.499)--(3.814,3.498)--(3.815,3.498)--(3.815,3.497)--(3.816,3.497)%
  --(3.816,3.496)--(3.817,3.496)--(3.817,3.495)--(3.818,3.495)--(3.818,3.494)--(3.819,3.494)%
  --(3.819,3.493)--(3.820,3.493)--(3.821,3.493)--(3.821,3.492)--(3.822,3.492)--(3.822,3.491)%
  --(3.823,3.491)--(3.823,3.490)--(3.824,3.490)--(3.824,3.489)--(3.825,3.489)--(3.825,3.488)%
  --(3.826,3.488)--(3.826,3.487)--(3.827,3.487)--(3.827,3.486)--(3.828,3.486)--(3.829,3.486)%
  --(3.829,3.485)--(3.830,3.485)--(3.830,3.484)--(3.831,3.484)--(3.831,3.483)--(3.832,3.483)%
  --(3.832,3.482)--(3.833,3.482)--(3.833,3.481)--(3.834,3.481)--(3.834,3.480)--(3.835,3.480)%
  --(3.835,3.479)--(3.836,3.479)--(3.837,3.479)--(3.837,3.478)--(3.838,3.478)--(3.838,3.477)%
  --(3.839,3.477)--(3.839,3.476)--(3.840,3.476)--(3.840,3.475)--(3.841,3.475)--(3.841,3.474)%
  --(3.842,3.474)--(3.842,3.473)--(3.843,3.473)--(3.844,3.472)--(3.845,3.472)--(3.845,3.471)%
  --(3.846,3.471)--(3.846,3.470)--(3.847,3.470)--(3.847,3.469)--(3.848,3.469)--(3.848,3.468)%
  --(3.849,3.468)--(3.849,3.467)--(3.850,3.467)--(3.850,3.466)--(3.851,3.466)--(3.852,3.466)%
  --(3.852,3.465)--(3.853,3.465)--(3.853,3.464)--(3.854,3.464)--(3.854,3.463)--(3.855,3.463)%
  --(3.855,3.462)--(3.856,3.462)--(3.856,3.461)--(3.857,3.461)--(3.857,3.460)--(3.858,3.460)%
  --(3.859,3.460)--(3.859,3.459)--(3.860,3.459)--(3.860,3.458)--(3.861,3.458)--(3.861,3.457)%
  --(3.862,3.457)--(3.862,3.456)--(3.863,3.456)--(3.863,3.455)--(3.864,3.455)--(3.864,3.454)%
  --(3.865,3.454)--(3.866,3.454)--(3.866,3.453)--(3.867,3.453)--(3.867,3.452)--(3.868,3.452)%
  --(3.868,3.451)--(3.869,3.451)--(3.869,3.450)--(3.870,3.450)--(3.870,3.449)--(3.871,3.449)%
  --(3.872,3.449)--(3.872,3.448)--(3.873,3.448)--(3.873,3.447)--(3.874,3.447)--(3.874,3.446)%
  --(3.875,3.446)--(3.875,3.445)--(3.876,3.445)--(3.876,3.444)--(3.877,3.444)--(3.877,3.443)%
  --(3.878,3.443)--(3.879,3.443)--(3.879,3.442)--(3.880,3.442)--(3.880,3.441)--(3.881,3.441)%
  --(3.881,3.440)--(3.882,3.440)--(3.882,3.439)--(3.883,3.439)--(3.883,3.438)--(3.884,3.438)%
  --(3.885,3.438)--(3.885,3.437)--(3.886,3.437)--(3.886,3.436)--(3.887,3.436)--(3.887,3.435)%
  --(3.888,3.435)--(3.888,3.434)--(3.889,3.434)--(3.889,3.433)--(3.890,3.433)--(3.890,3.432)%
  --(3.891,3.432)--(3.892,3.432)--(3.892,3.431)--(3.893,3.431)--(3.893,3.430)--(3.894,3.430)%
  --(3.894,3.429)--(3.895,3.429)--(3.895,3.428)--(3.896,3.428)--(3.896,3.427)--(3.897,3.427)%
  --(3.898,3.427)--(3.898,3.426)--(3.899,3.426)--(3.899,3.425)--(3.900,3.425)--(3.900,3.424)%
  --(3.901,3.424)--(3.901,3.423)--(3.902,3.423)--(3.902,3.422)--(3.903,3.422)--(3.904,3.422)%
  --(3.904,3.421)--(3.905,3.421)--(3.905,3.420)--(3.906,3.420)--(3.906,3.419)--(3.907,3.419)%
  --(3.907,3.418)--(3.908,3.418)--(3.908,3.417)--(3.909,3.417)--(3.910,3.417)--(3.910,3.416)%
  --(3.911,3.416)--(3.911,3.415)--(3.912,3.415)--(3.912,3.414)--(3.913,3.414)--(3.913,3.413)%
  --(3.914,3.413)--(3.914,3.412)--(3.915,3.412)--(3.916,3.412)--(3.916,3.411)--(3.917,3.411)%
  --(3.917,3.410)--(3.918,3.410)--(3.918,3.409)--(3.919,3.409)--(3.919,3.408)--(3.920,3.408)%
  --(3.921,3.408)--(3.921,3.407)--(3.922,3.407)--(3.922,3.406)--(3.923,3.406)--(3.923,3.405)%
  --(3.924,3.405)--(3.924,3.404)--(3.925,3.404)--(3.925,3.403)--(3.926,3.403)--(3.927,3.403)%
  --(3.927,3.402)--(3.928,3.402)--(3.928,3.401)--(3.929,3.401)--(3.929,3.400)--(3.930,3.400)%
  --(3.930,3.399)--(3.931,3.399)--(3.932,3.399)--(3.932,3.398)--(3.933,3.398)--(3.933,3.397)%
  --(3.934,3.397)--(3.934,3.396)--(3.935,3.396)--(3.935,3.395)--(3.936,3.395)--(3.936,3.394)%
  --(3.937,3.394)--(3.938,3.394)--(3.938,3.393)--(3.939,3.393)--(3.939,3.392)--(3.940,3.392)%
  --(3.940,3.391)--(3.941,3.391)--(3.941,3.390)--(3.942,3.390)--(3.943,3.390)--(3.943,3.389)%
  --(3.944,3.389)--(3.944,3.388)--(3.945,3.388)--(3.945,3.387)--(3.946,3.387)--(3.946,3.386)%
  --(3.947,3.386)--(3.948,3.385)--(3.949,3.385)--(3.949,3.384)--(3.950,3.384)--(3.950,3.383)%
  --(3.951,3.383)--(3.951,3.382)--(3.952,3.382)--(3.952,3.381)--(3.953,3.381)--(3.954,3.381)%
  --(3.954,3.380)--(3.955,3.380)--(3.955,3.379)--(3.956,3.379)--(3.956,3.378)--(3.957,3.378)%
  --(3.957,3.377)--(3.958,3.377)--(3.959,3.377)--(3.959,3.376)--(3.960,3.376)--(3.960,3.375)%
  --(3.961,3.375)--(3.961,3.374)--(3.962,3.374)--(3.962,3.373)--(3.963,3.373)--(3.964,3.373)%
  --(3.964,3.372)--(3.965,3.372)--(3.965,3.371)--(3.966,3.371)--(3.966,3.370)--(3.967,3.370)%
  --(3.967,3.369)--(3.968,3.369)--(3.969,3.369)--(3.969,3.368)--(3.970,3.368)--(3.970,3.367)%
  --(3.971,3.367)--(3.971,3.366)--(3.972,3.366)--(3.972,3.365)--(3.973,3.365)--(3.974,3.365)%
  --(3.974,3.364)--(3.975,3.364)--(3.975,3.363)--(3.976,3.363)--(3.976,3.362)--(3.977,3.362)%
  --(3.977,3.361)--(3.978,3.361)--(3.979,3.361)--(3.979,3.360)--(3.980,3.360)--(3.980,3.359)%
  --(3.981,3.359)--(3.981,3.358)--(3.982,3.358)--(3.982,3.357)--(3.983,3.357)--(3.984,3.357)%
  --(3.984,3.356)--(3.985,3.356)--(3.985,3.355)--(3.986,3.355)--(3.986,3.354)--(3.987,3.354)%
  --(3.988,3.354)--(3.988,3.353)--(3.989,3.353)--(3.989,3.352)--(3.990,3.352)--(3.990,3.351)%
  --(3.991,3.351)--(3.991,3.350)--(3.992,3.350)--(3.993,3.350)--(3.993,3.349)--(3.994,3.349)%
  --(3.994,3.348)--(3.995,3.348)--(3.995,3.347)--(3.996,3.347)--(3.996,3.346)--(3.997,3.346)%
  --(3.998,3.346)--(3.998,3.345)--(3.999,3.345)--(3.999,3.344)--(4.000,3.344)--(4.000,3.343)%
  --(4.001,3.343)--(4.002,3.343)--(4.002,3.342)--(4.003,3.342)--(4.003,3.341)--(4.004,3.341)%
  --(4.004,3.340)--(4.005,3.340)--(4.005,3.339)--(4.006,3.339)--(4.007,3.339)--(4.007,3.338)%
  --(4.008,3.338)--(4.008,3.337)--(4.009,3.337)--(4.009,3.336)--(4.010,3.336)--(4.011,3.335)%
  --(4.012,3.335)--(4.012,3.334)--(4.013,3.334)--(4.013,3.333)--(4.014,3.333)--(4.014,3.332)%
  --(4.015,3.332)--(4.016,3.332)--(4.016,3.331)--(4.017,3.331)--(4.017,3.330)--(4.018,3.330)%
  --(4.018,3.329)--(4.019,3.329)--(4.020,3.329)--(4.020,3.328)--(4.021,3.328)--(4.021,3.327)%
  --(4.022,3.327)--(4.022,3.326)--(4.023,3.326)--(4.023,3.325)--(4.024,3.325)--(4.025,3.325)%
  --(4.025,3.324)--(4.026,3.324)--(4.026,3.323)--(4.027,3.323)--(4.027,3.322)--(4.028,3.322)%
  --(4.029,3.322)--(4.029,3.321)--(4.030,3.321)--(4.030,3.320)--(4.031,3.320)--(4.031,3.319)%
  --(4.032,3.319)--(4.032,3.318)--(4.033,3.318)--(4.034,3.318)--(4.034,3.317)--(4.035,3.317)%
  --(4.035,3.316)--(4.036,3.316)--(4.036,3.315)--(4.037,3.315)--(4.038,3.315)--(4.038,3.314)%
  --(4.039,3.314)--(4.039,3.313)--(4.040,3.313)--(4.040,3.312)--(4.041,3.312)--(4.042,3.312)%
  --(4.042,3.311)--(4.043,3.311)--(4.043,3.310)--(4.044,3.310)--(4.044,3.309)--(4.045,3.309)%
  --(4.046,3.309)--(4.046,3.308)--(4.047,3.308)--(4.047,3.307)--(4.048,3.307)--(4.048,3.306)%
  --(4.049,3.306)--(4.050,3.306)--(4.050,3.305)--(4.051,3.305)--(4.051,3.304)--(4.052,3.304)%
  --(4.052,3.303)--(4.053,3.303)--(4.053,3.302)--(4.054,3.302)--(4.055,3.302)--(4.055,3.301)%
  --(4.056,3.301)--(4.056,3.300)--(4.057,3.300)--(4.057,3.299)--(4.058,3.299)--(4.059,3.299)%
  --(4.059,3.298)--(4.060,3.298)--(4.060,3.297)--(4.061,3.297)--(4.061,3.296)--(4.062,3.296)%
  --(4.063,3.296)--(4.063,3.295)--(4.064,3.295)--(4.064,3.294)--(4.065,3.294)--(4.065,3.293)%
  --(4.066,3.293)--(4.067,3.293)--(4.067,3.292)--(4.068,3.292)--(4.068,3.291)--(4.069,3.291)%
  --(4.069,3.290)--(4.070,3.290)--(4.071,3.290)--(4.071,3.289)--(4.072,3.289)--(4.072,3.288)%
  --(4.073,3.288)--(4.073,3.287)--(4.074,3.287)--(4.075,3.287)--(4.075,3.286)--(4.076,3.286)%
  --(4.076,3.285)--(4.077,3.285)--(4.077,3.284)--(4.078,3.284)--(4.079,3.284)--(4.079,3.283)%
  --(4.080,3.283)--(4.080,3.282)--(4.081,3.282)--(4.081,3.281)--(4.082,3.281)--(4.083,3.281)%
  --(4.083,3.280)--(4.084,3.280)--(4.084,3.279)--(4.085,3.279)--(4.085,3.278)--(4.086,3.278)%
  --(4.087,3.278)--(4.087,3.277)--(4.088,3.277)--(4.088,3.276)--(4.089,3.276)--(4.089,3.275)%
  --(4.090,3.275)--(4.091,3.275)--(4.091,3.274)--(4.092,3.274)--(4.092,3.273)--(4.093,3.273)%
  --(4.094,3.273)--(4.094,3.272)--(4.095,3.272)--(4.095,3.271)--(4.096,3.271)--(4.096,3.270)%
  --(4.097,3.270)--(4.098,3.270)--(4.098,3.269)--(4.099,3.269)--(4.099,3.268)--(4.100,3.268)%
  --(4.100,3.267)--(4.101,3.267)--(4.102,3.267)--(4.102,3.266)--(4.103,3.266)--(4.103,3.265)%
  --(4.104,3.265)--(4.104,3.264)--(4.105,3.264)--(4.106,3.264)--(4.106,3.263)--(4.107,3.263)%
  --(4.107,3.262)--(4.108,3.262)--(4.108,3.261)--(4.109,3.261)--(4.110,3.261)--(4.110,3.260)%
  --(4.111,3.260)--(4.111,3.259)--(4.112,3.259)--(4.113,3.259)--(4.113,3.258)--(4.114,3.258)%
  --(4.114,3.257)--(4.115,3.257)--(4.115,3.256)--(4.116,3.256)--(4.117,3.256)--(4.117,3.255)%
  --(4.118,3.255)--(4.118,3.254)--(4.119,3.254)--(4.119,3.253)--(4.120,3.253)--(4.121,3.253)%
  --(4.121,3.252)--(4.122,3.252)--(4.122,3.251)--(4.123,3.251)--(4.124,3.251)--(4.124,3.250)%
  --(4.125,3.250)--(4.125,3.249)--(4.126,3.249)--(4.126,3.248)--(4.127,3.248)--(4.128,3.248)%
  --(4.128,3.247)--(4.129,3.247)--(4.129,3.246)--(4.130,3.246)--(4.130,3.245)--(4.131,3.245)%
  --(4.132,3.245)--(4.132,3.244)--(4.133,3.244)--(4.133,3.243)--(4.134,3.243)--(4.135,3.243)%
  --(4.135,3.242)--(4.136,3.242)--(4.136,3.241)--(4.137,3.241)--(4.137,3.240)--(4.138,3.240)%
  --(4.139,3.240)--(4.139,3.239)--(4.140,3.239)--(4.140,3.238)--(4.141,3.238)--(4.142,3.238)%
  --(4.142,3.237)--(4.143,3.237)--(4.143,3.236)--(4.144,3.236)--(4.144,3.235)--(4.145,3.235)%
  --(4.146,3.235)--(4.146,3.234)--(4.147,3.234)--(4.147,3.233)--(4.148,3.233)--(4.149,3.233)%
  --(4.149,3.232)--(4.150,3.232)--(4.150,3.231)--(4.151,3.231)--(4.151,3.230)--(4.152,3.230)%
  --(4.153,3.230)--(4.153,3.229)--(4.154,3.229)--(4.154,3.228)--(4.155,3.228)--(4.156,3.228)%
  --(4.156,3.227)--(4.157,3.227)--(4.157,3.226)--(4.158,3.226)--(4.158,3.225)--(4.159,3.225)%
  --(4.160,3.225)--(4.160,3.224)--(4.161,3.224)--(4.161,3.223)--(4.162,3.223)--(4.163,3.223)%
  --(4.163,3.222)--(4.164,3.222)--(4.164,3.221)--(4.165,3.221)--(4.165,3.220)--(4.166,3.220)%
  --(4.167,3.220)--(4.167,3.219)--(4.168,3.219)--(4.168,3.218)--(4.169,3.218)--(4.170,3.218)%
  --(4.170,3.217)--(4.171,3.217)--(4.171,3.216)--(4.172,3.216)--(4.172,3.215)--(4.173,3.215)%
  --(4.174,3.215)--(4.174,3.214)--(4.175,3.214)--(4.175,3.213)--(4.176,3.213)--(4.177,3.213)%
  --(4.177,3.212)--(4.178,3.212)--(4.178,3.211)--(4.179,3.211)--(4.180,3.211)--(4.180,3.210)%
  --(4.181,3.210)--(4.181,3.209)--(4.182,3.209)--(4.182,3.208)--(4.183,3.208)--(4.184,3.208)%
  --(4.184,3.207)--(4.185,3.207)--(4.185,3.206)--(4.186,3.206)--(4.187,3.206)--(4.187,3.205)%
  --(4.188,3.205)--(4.188,3.204)--(4.189,3.204)--(4.190,3.204)--(4.190,3.203)--(4.191,3.203)%
  --(4.191,3.202)--(4.192,3.202)--(4.192,3.201)--(4.193,3.201)--(4.194,3.201)--(4.194,3.200)%
  --(4.195,3.200)--(4.195,3.199)--(4.196,3.199)--(4.197,3.199)--(4.197,3.198)--(4.198,3.198)%
  --(4.198,3.197)--(4.199,3.197)--(4.200,3.197)--(4.200,3.196)--(4.201,3.196)--(4.201,3.195)%
  --(4.202,3.195)--(4.203,3.195)--(4.203,3.194)--(4.204,3.194)--(4.204,3.193)--(4.205,3.193)%
  --(4.205,3.192)--(4.206,3.192)--(4.207,3.192)--(4.207,3.191)--(4.208,3.191)--(4.208,3.190)%
  --(4.209,3.190)--(4.210,3.190)--(4.210,3.189)--(4.211,3.189)--(4.211,3.188)--(4.212,3.188)%
  --(4.213,3.188)--(4.213,3.187)--(4.214,3.187)--(4.214,3.186)--(4.215,3.186)--(4.216,3.186)%
  --(4.216,3.185)--(4.217,3.185)--(4.217,3.184)--(4.218,3.184)--(4.218,3.183)--(4.219,3.183)%
  --(4.220,3.183)--(4.220,3.182)--(4.221,3.182)--(4.221,3.181)--(4.222,3.181)--(4.223,3.181)%
  --(4.223,3.180)--(4.224,3.180)--(4.224,3.179)--(4.225,3.179)--(4.226,3.179)--(4.226,3.178)%
  --(4.227,3.178)--(4.227,3.177)--(4.228,3.177)--(4.229,3.177)--(4.229,3.176)--(4.230,3.176)%
  --(4.230,3.175)--(4.231,3.175)--(4.232,3.175)--(4.232,3.174)--(4.233,3.174)--(4.233,3.173)%
  --(4.234,3.173)--(4.235,3.173)--(4.235,3.172)--(4.236,3.172)--(4.236,3.171)--(4.237,3.171)%
  --(4.238,3.171)--(4.238,3.170)--(4.239,3.170)--(4.239,3.169)--(4.240,3.169)--(4.240,3.168)%
  --(4.241,3.168)--(4.242,3.168)--(4.242,3.167)--(4.243,3.167)--(4.243,3.166)--(4.244,3.166)%
  --(4.245,3.166)--(4.245,3.165)--(4.246,3.165)--(4.246,3.164)--(4.247,3.164)--(4.248,3.164)%
  --(4.248,3.163)--(4.249,3.163)--(4.249,3.162)--(4.250,3.162)--(4.251,3.162)--(4.251,3.161)%
  --(4.252,3.161)--(4.252,3.160)--(4.253,3.160)--(4.254,3.160)--(4.254,3.159)--(4.255,3.159)%
  --(4.255,3.158)--(4.256,3.158)--(4.257,3.158)--(4.257,3.157)--(4.258,3.157)--(4.258,3.156)%
  --(4.259,3.156)--(4.260,3.156)--(4.260,3.155)--(4.261,3.155)--(4.261,3.154)--(4.262,3.154)%
  --(4.263,3.154)--(4.263,3.153)--(4.264,3.153)--(4.264,3.152)--(4.265,3.152)--(4.266,3.152)%
  --(4.266,3.151)--(4.267,3.151)--(4.267,3.150)--(4.268,3.150)--(4.269,3.150)--(4.269,3.149)%
  --(4.270,3.149)--(4.270,3.148)--(4.271,3.148)--(4.272,3.148)--(4.272,3.147)--(4.273,3.147)%
  --(4.273,3.146)--(4.274,3.146)--(4.275,3.146)--(4.275,3.145)--(4.276,3.145)--(4.276,3.144)%
  --(4.277,3.144)--(4.278,3.144)--(4.278,3.143)--(4.279,3.143)--(4.279,3.142)--(4.280,3.142)%
  --(4.281,3.142)--(4.281,3.141)--(4.282,3.141)--(4.282,3.140)--(4.283,3.140)--(4.284,3.140)%
  --(4.284,3.139)--(4.285,3.139)--(4.285,3.138)--(4.286,3.138)--(4.287,3.138)--(4.287,3.137)%
  --(4.288,3.137)--(4.288,3.136)--(4.289,3.136)--(4.290,3.136)--(4.290,3.135)--(4.291,3.135)%
  --(4.291,3.134)--(4.292,3.134)--(4.293,3.134)--(4.293,3.133)--(4.294,3.133)--(4.294,3.132)%
  --(4.295,3.132)--(4.296,3.132)--(4.296,3.131)--(4.297,3.131)--(4.297,3.130)--(4.298,3.130)%
  --(4.299,3.130)--(4.299,3.129)--(4.300,3.129)--(4.301,3.129)--(4.301,3.128)--(4.302,3.128)%
  --(4.302,3.127)--(4.303,3.127)--(4.304,3.127)--(4.304,3.126)--(4.305,3.126)--(4.305,3.125)%
  --(4.306,3.125)--(4.307,3.125)--(4.307,3.124)--(4.308,3.124)--(4.308,3.123)--(4.309,3.123)%
  --(4.310,3.123)--(4.310,3.122)--(4.311,3.122)--(4.311,3.121)--(4.312,3.121)--(4.313,3.121)%
  --(4.313,3.120)--(4.314,3.120)--(4.314,3.119)--(4.315,3.119)--(4.316,3.119)--(4.316,3.118)%
  --(4.317,3.118)--(4.317,3.117)--(4.318,3.117)--(4.319,3.117)--(4.319,3.116)--(4.320,3.116)%
  --(4.321,3.116)--(4.321,3.115)--(4.322,3.115)--(4.322,3.114)--(4.323,3.114)--(4.324,3.114)%
  --(4.324,3.113)--(4.325,3.113)--(4.325,3.112)--(4.326,3.112)--(4.327,3.112)--(4.327,3.111)%
  --(4.328,3.111)--(4.328,3.110)--(4.329,3.110)--(4.330,3.110)--(4.330,3.109)--(4.331,3.109)%
  --(4.331,3.108)--(4.332,3.108)--(4.333,3.108)--(4.333,3.107)--(4.334,3.107)--(4.335,3.106)%
  --(4.336,3.106)--(4.336,3.105)--(4.337,3.105)--(4.338,3.105)--(4.338,3.104)--(4.339,3.104)%
  --(4.339,3.103)--(4.340,3.103)--(4.341,3.103)--(4.341,3.102)--(4.342,3.102)--(4.342,3.101)%
  --(4.343,3.101)--(4.344,3.101)--(4.344,3.100)--(4.345,3.100)--(4.345,3.099)--(4.346,3.099)%
  --(4.347,3.099)--(4.347,3.098)--(4.348,3.098)--(4.349,3.098)--(4.349,3.097)--(4.350,3.097)%
  --(4.350,3.096)--(4.351,3.096)--(4.352,3.096)--(4.352,3.095)--(4.353,3.095)--(4.353,3.094)%
  --(4.354,3.094)--(4.355,3.094)--(4.355,3.093)--(4.356,3.093)--(4.356,3.092)--(4.357,3.092)%
  --(4.358,3.092)--(4.358,3.091)--(4.359,3.091)--(4.360,3.091)--(4.360,3.090)--(4.361,3.090)%
  --(4.361,3.089)--(4.362,3.089)--(4.363,3.089)--(4.363,3.088)--(4.364,3.088)--(4.364,3.087)%
  --(4.365,3.087)--(4.366,3.087)--(4.366,3.086)--(4.367,3.086)--(4.368,3.086)--(4.368,3.085)%
  --(4.369,3.085)--(4.369,3.084)--(4.370,3.084)--(4.371,3.084)--(4.371,3.083)--(4.372,3.083)%
  --(4.372,3.082)--(4.373,3.082)--(4.374,3.082)--(4.374,3.081)--(4.375,3.081)--(4.376,3.081)%
  --(4.376,3.080)--(4.377,3.080)--(4.377,3.079)--(4.378,3.079)--(4.379,3.079)--(4.379,3.078)%
  --(4.380,3.078)--(4.380,3.077)--(4.381,3.077)--(4.382,3.077)--(4.382,3.076)--(4.383,3.076)%
  --(4.384,3.075)--(4.385,3.075)--(4.385,3.074)--(4.386,3.074)--(4.387,3.074)--(4.387,3.073)%
  --(4.388,3.073)--(4.388,3.072)--(4.389,3.072)--(4.390,3.072)--(4.390,3.071)--(4.391,3.071)%
  --(4.392,3.071)--(4.392,3.070)--(4.393,3.070)--(4.393,3.069)--(4.394,3.069)--(4.395,3.069)%
  --(4.395,3.068)--(4.396,3.068)--(4.396,3.067)--(4.397,3.067)--(4.398,3.067)--(4.398,3.066)%
  --(4.399,3.066)--(4.400,3.066)--(4.400,3.065)--(4.401,3.065)--(4.401,3.064)--(4.402,3.064)%
  --(4.403,3.064)--(4.403,3.063)--(4.404,3.063)--(4.404,3.062)--(4.405,3.062)--(4.406,3.062)%
  --(4.406,3.061)--(4.407,3.061)--(4.408,3.061)--(4.408,3.060)--(4.409,3.060)--(4.409,3.059)%
  --(4.410,3.059)--(4.411,3.059)--(4.411,3.058)--(4.412,3.058)--(4.413,3.058)--(4.413,3.057)%
  --(4.414,3.057)--(4.414,3.056)--(4.415,3.056)--(4.416,3.056)--(4.416,3.055)--(4.417,3.055)%
  --(4.417,3.054)--(4.418,3.054)--(4.419,3.054)--(4.419,3.053)--(4.420,3.053)--(4.421,3.053)%
  --(4.421,3.052)--(4.422,3.052)--(4.422,3.051)--(4.423,3.051)--(4.424,3.051)--(4.424,3.050)%
  --(4.425,3.050)--(4.426,3.050)--(4.426,3.049)--(4.427,3.049)--(4.427,3.048)--(4.428,3.048)%
  --(4.429,3.048)--(4.429,3.047)--(4.430,3.047)--(4.431,3.047)--(4.431,3.046)--(4.432,3.046)%
  --(4.432,3.045)--(4.433,3.045)--(4.434,3.045)--(4.434,3.044)--(4.435,3.044)--(4.436,3.044)%
  --(4.436,3.043)--(4.437,3.043)--(4.437,3.042)--(4.438,3.042)--(4.439,3.042)--(4.439,3.041)%
  --(4.440,3.041)--(4.441,3.041)--(4.441,3.040)--(4.442,3.040)--(4.442,3.039)--(4.443,3.039)%
  --(4.444,3.039)--(4.444,3.038)--(4.445,3.038)--(4.445,3.037)--(4.446,3.037)--(4.447,3.037)%
  --(4.447,3.036)--(4.448,3.036)--(4.449,3.036)--(4.449,3.035)--(4.450,3.035)--(4.450,3.034)%
  --(4.451,3.034)--(4.452,3.034)--(4.452,3.033)--(4.453,3.033)--(4.454,3.033)--(4.454,3.032)%
  --(4.455,3.032)--(4.455,3.031)--(4.456,3.031)--(4.457,3.031)--(4.457,3.030)--(4.458,3.030)%
  --(4.459,3.030)--(4.459,3.029)--(4.460,3.029)--(4.460,3.028)--(4.461,3.028)--(4.462,3.028)%
  --(4.462,3.027)--(4.463,3.027)--(4.464,3.027)--(4.464,3.026)--(4.465,3.026)--(4.466,3.026)%
  --(4.466,3.025)--(4.467,3.025)--(4.467,3.024)--(4.468,3.024)--(4.469,3.024)--(4.469,3.023)%
  --(4.470,3.023)--(4.471,3.023)--(4.471,3.022)--(4.472,3.022)--(4.472,3.021)--(4.473,3.021)%
  --(4.474,3.021)--(4.474,3.020)--(4.475,3.020)--(4.476,3.020)--(4.476,3.019)--(4.477,3.019)%
  --(4.477,3.018)--(4.478,3.018)--(4.479,3.018)--(4.479,3.017)--(4.480,3.017)--(4.481,3.017)%
  --(4.481,3.016)--(4.482,3.016)--(4.482,3.015)--(4.483,3.015)--(4.484,3.015)--(4.484,3.014)%
  --(4.485,3.014)--(4.486,3.014)--(4.486,3.013)--(4.487,3.013)--(4.487,3.012)--(4.488,3.012)%
  --(4.489,3.012)--(4.489,3.011)--(4.490,3.011)--(4.491,3.011)--(4.491,3.010)--(4.492,3.010)%
  --(4.493,3.010)--(4.493,3.009)--(4.494,3.009)--(4.494,3.008)--(4.495,3.008)--(4.496,3.008)%
  --(4.496,3.007)--(4.497,3.007)--(4.498,3.007)--(4.498,3.006)--(4.499,3.006)--(4.499,3.005)%
  --(4.500,3.005)--(4.501,3.005)--(4.501,3.004)--(4.502,3.004)--(4.503,3.004)--(4.503,3.003)%
  --(4.504,3.003)--(4.505,3.003)--(4.505,3.002)--(4.506,3.002)--(4.506,3.001)--(4.507,3.001)%
  --(4.508,3.001)--(4.508,3.000)--(4.509,3.000)--(4.510,3.000)--(4.510,2.999)--(4.511,2.999)%
  --(4.511,2.998)--(4.512,2.998)--(4.513,2.998)--(4.513,2.997)--(4.514,2.997)--(4.515,2.997)%
  --(4.515,2.996)--(4.516,2.996)--(4.517,2.996)--(4.517,2.995)--(4.518,2.995)--(4.518,2.994)%
  --(4.519,2.994)--(4.520,2.994)--(4.520,2.993)--(4.521,2.993)--(4.522,2.993)--(4.522,2.992)%
  --(4.523,2.992)--(4.524,2.992)--(4.524,2.991)--(4.525,2.991)--(4.525,2.990)--(4.526,2.990)%
  --(4.527,2.990)--(4.527,2.989)--(4.528,2.989)--(4.529,2.989)--(4.529,2.988)--(4.530,2.988)%
  --(4.530,2.987)--(4.531,2.987)--(4.532,2.987)--(4.532,2.986)--(4.533,2.986)--(4.534,2.986)%
  --(4.534,2.985)--(4.535,2.985)--(4.536,2.985)--(4.536,2.984)--(4.537,2.984)--(4.537,2.983)%
  --(4.538,2.983)--(4.539,2.983)--(4.539,2.982)--(4.540,2.982)--(4.541,2.982)--(4.541,2.981)%
  --(4.542,2.981)--(4.543,2.981)--(4.543,2.980)--(4.544,2.980)--(4.544,2.979)--(4.545,2.979)%
  --(4.546,2.979)--(4.546,2.978)--(4.547,2.978)--(4.548,2.978)--(4.548,2.977)--(4.549,2.977)%
  --(4.550,2.977)--(4.550,2.976)--(4.551,2.976)--(4.551,2.975)--(4.552,2.975)--(4.553,2.975)%
  --(4.553,2.974)--(4.554,2.974)--(4.555,2.974)--(4.555,2.973)--(4.556,2.973)--(4.557,2.973)%
  --(4.557,2.972)--(4.558,2.972)--(4.559,2.972)--(4.559,2.971)--(4.560,2.971)--(4.560,2.970)%
  --(4.561,2.970)--(4.562,2.970)--(4.562,2.969)--(4.563,2.969)--(4.564,2.969)--(4.564,2.968)%
  --(4.565,2.968)--(4.566,2.968)--(4.566,2.967)--(4.567,2.967)--(4.567,2.966)--(4.568,2.966)%
  --(4.569,2.966)--(4.569,2.965)--(4.570,2.965)--(4.571,2.965)--(4.571,2.964)--(4.572,2.964)%
  --(4.573,2.964)--(4.573,2.963)--(4.574,2.963)--(4.574,2.962)--(4.575,2.962)--(4.576,2.962)%
  --(4.576,2.961)--(4.577,2.961)--(4.578,2.961)--(4.578,2.960)--(4.579,2.960)--(4.580,2.960)%
  --(4.580,2.959)--(4.581,2.959)--(4.582,2.959)--(4.582,2.958)--(4.583,2.958)--(4.583,2.957)%
  --(4.584,2.957)--(4.585,2.957)--(4.585,2.956)--(4.586,2.956)--(4.587,2.956)--(4.587,2.955)%
  --(4.588,2.955)--(4.589,2.955)--(4.589,2.954)--(4.590,2.954)--(4.591,2.954)--(4.591,2.953)%
  --(4.592,2.953)--(4.592,2.952)--(4.593,2.952)--(4.594,2.952)--(4.594,2.951)--(4.595,2.951)%
  --(4.596,2.951)--(4.596,2.950)--(4.597,2.950)--(4.598,2.950)--(4.598,2.949)--(4.599,2.949)%
  --(4.600,2.949)--(4.600,2.948)--(4.601,2.948)--(4.601,2.947)--(4.602,2.947)--(4.603,2.947)%
  --(4.603,2.946)--(4.604,2.946)--(4.605,2.946)--(4.605,2.945)--(4.606,2.945)--(4.607,2.945)%
  --(4.607,2.944)--(4.608,2.944)--(4.609,2.944)--(4.609,2.943)--(4.610,2.943)--(4.610,2.942)%
  --(4.611,2.942)--(4.612,2.942)--(4.612,2.941)--(4.613,2.941)--(4.614,2.941)--(4.614,2.940)%
  --(4.615,2.940)--(4.616,2.940)--(4.616,2.939)--(4.617,2.939)--(4.618,2.939)--(4.618,2.938)%
  --(4.619,2.938)--(4.620,2.938)--(4.620,2.937)--(4.621,2.937)--(4.621,2.936)--(4.622,2.936)%
  --(4.623,2.936)--(4.623,2.935)--(4.624,2.935)--(4.625,2.935)--(4.625,2.934)--(4.626,2.934)%
  --(4.627,2.934)--(4.627,2.933)--(4.628,2.933)--(4.629,2.933)--(4.629,2.932)--(4.630,2.932)%
  --(4.631,2.932)--(4.631,2.931)--(4.632,2.931)--(4.632,2.930)--(4.633,2.930)--(4.634,2.930)%
  --(4.634,2.929)--(4.635,2.929)--(4.636,2.929)--(4.636,2.928)--(4.637,2.928)--(4.638,2.928)%
  --(4.638,2.927)--(4.639,2.927)--(4.640,2.927)--(4.640,2.926)--(4.641,2.926)--(4.642,2.926)%
  --(4.642,2.925)--(4.643,2.925)--(4.643,2.924)--(4.644,2.924)--(4.645,2.924)--(4.645,2.923)%
  --(4.646,2.923)--(4.647,2.923)--(4.647,2.922)--(4.648,2.922)--(4.649,2.922)--(4.649,2.921)%
  --(4.650,2.921)--(4.651,2.921)--(4.651,2.920)--(4.652,2.920)--(4.653,2.920)--(4.653,2.919)%
  --(4.654,2.919)--(4.655,2.919)--(4.655,2.918)--(4.656,2.918)--(4.656,2.917)--(4.657,2.917)%
  --(4.658,2.917)--(4.658,2.916)--(4.659,2.916)--(4.660,2.916)--(4.660,2.915)--(4.661,2.915)%
  --(4.662,2.915)--(4.662,2.914)--(4.663,2.914)--(4.664,2.914)--(4.664,2.913)--(4.665,2.913)%
  --(4.666,2.913)--(4.666,2.912)--(4.667,2.912)--(4.668,2.912)--(4.668,2.911)--(4.669,2.911)%
  --(4.670,2.911)--(4.670,2.910)--(4.671,2.910)--(4.671,2.909)--(4.672,2.909)--(4.673,2.909)%
  --(4.673,2.908)--(4.674,2.908)--(4.675,2.908)--(4.675,2.907)--(4.676,2.907)--(4.677,2.907)%
  --(4.677,2.906)--(4.678,2.906)--(4.679,2.906)--(4.679,2.905)--(4.680,2.905)--(4.681,2.905)%
  --(4.681,2.904)--(4.682,2.904)--(4.683,2.904)--(4.683,2.903)--(4.684,2.903)--(4.685,2.903)%
  --(4.685,2.902)--(4.686,2.902)--(4.687,2.901)--(4.688,2.901)--(4.688,2.900)--(4.689,2.900)%
  --(4.690,2.900)--(4.690,2.899)--(4.691,2.899)--(4.692,2.899)--(4.692,2.898)--(4.693,2.898)%
  --(4.694,2.898)--(4.694,2.897)--(4.695,2.897)--(4.696,2.897)--(4.696,2.896)--(4.697,2.896)%
  --(4.698,2.896)--(4.698,2.895)--(4.699,2.895)--(4.700,2.895)--(4.700,2.894)--(4.701,2.894)%
  --(4.702,2.894)--(4.702,2.893)--(4.703,2.893)--(4.704,2.893)--(4.704,2.892)--(4.705,2.892)%
  --(4.706,2.892)--(4.706,2.891)--(4.707,2.891)--(4.707,2.890)--(4.708,2.890)--(4.709,2.890)%
  --(4.709,2.889)--(4.710,2.889)--(4.711,2.889)--(4.711,2.888)--(4.712,2.888)--(4.713,2.888)%
  --(4.713,2.887)--(4.714,2.887)--(4.715,2.887)--(4.715,2.886)--(4.716,2.886)--(4.717,2.886)%
  --(4.717,2.885)--(4.718,2.885)--(4.719,2.885)--(4.719,2.884)--(4.720,2.884)--(4.721,2.884)%
  --(4.721,2.883)--(4.722,2.883)--(4.723,2.883)--(4.723,2.882)--(4.724,2.882)--(4.725,2.882)%
  --(4.725,2.881)--(4.726,2.881)--(4.727,2.881)--(4.727,2.880)--(4.728,2.880)--(4.729,2.880)%
  --(4.729,2.879)--(4.730,2.879)--(4.731,2.879)--(4.731,2.878)--(4.732,2.878)--(4.732,2.877)%
  --(4.733,2.877)--(4.734,2.877)--(4.734,2.876)--(4.735,2.876)--(4.736,2.876)--(4.736,2.875)%
  --(4.737,2.875)--(4.738,2.875)--(4.738,2.874)--(4.739,2.874)--(4.740,2.874)--(4.740,2.873)%
  --(4.741,2.873)--(4.742,2.873)--(4.742,2.872)--(4.743,2.872)--(4.744,2.872)--(4.744,2.871)%
  --(4.745,2.871)--(4.746,2.871)--(4.746,2.870)--(4.747,2.870)--(4.748,2.870)--(4.748,2.869)%
  --(4.749,2.869)--(4.750,2.869)--(4.750,2.868)--(4.751,2.868)--(4.752,2.868)--(4.752,2.867)%
  --(4.753,2.867)--(4.754,2.867)--(4.754,2.866)--(4.755,2.866)--(4.756,2.866)--(4.756,2.865)%
  --(4.757,2.865)--(4.758,2.865)--(4.758,2.864)--(4.759,2.864)--(4.760,2.864)--(4.760,2.863)%
  --(4.761,2.863)--(4.762,2.863)--(4.762,2.862)--(4.763,2.862)--(4.764,2.862)--(4.764,2.861)%
  --(4.765,2.861)--(4.766,2.861)--(4.766,2.860)--(4.767,2.860)--(4.768,2.860)--(4.768,2.859)%
  --(4.769,2.859)--(4.770,2.859)--(4.770,2.858)--(4.771,2.858)--(4.772,2.858)--(4.772,2.857)%
  --(4.773,2.857)--(4.774,2.857)--(4.774,2.856)--(4.775,2.856)--(4.776,2.856)--(4.776,2.855)%
  --(4.777,2.855)--(4.778,2.855)--(4.778,2.854)--(4.779,2.854)--(4.780,2.854)--(4.780,2.853)%
  --(4.781,2.853)--(4.782,2.853)--(4.782,2.852)--(4.783,2.852)--(4.784,2.852)--(4.784,2.851)%
  --(4.785,2.851)--(4.786,2.851)--(4.786,2.850)--(4.787,2.850)--(4.788,2.850)--(4.788,2.849)%
  --(4.789,2.849)--(4.790,2.849)--(4.790,2.848)--(4.791,2.848)--(4.792,2.848)--(4.792,2.847)%
  --(4.793,2.847)--(4.794,2.847)--(4.794,2.846)--(4.795,2.846)--(4.796,2.846)--(4.796,2.845)%
  --(4.797,2.845)--(4.798,2.845)--(4.798,2.844)--(4.799,2.844)--(4.800,2.844)--(4.800,2.843)%
  --(4.801,2.843)--(4.802,2.843)--(4.802,2.842)--(4.803,2.842)--(4.804,2.842)--(4.804,2.841)%
  --(4.805,2.841)--(4.806,2.841)--(4.806,2.840)--(4.807,2.840)--(4.808,2.840)--(4.808,2.839)%
  --(4.809,2.839)--(4.810,2.839)--(4.810,2.838)--(4.811,2.838)--(4.812,2.838)--(4.812,2.837)%
  --(4.813,2.837)--(4.814,2.837)--(4.814,2.836)--(4.815,2.836)--(4.816,2.836)--(4.816,2.835)%
  --(4.817,2.835)--(4.818,2.835)--(4.818,2.834)--(4.819,2.834)--(4.820,2.834)--(4.820,2.833)%
  --(4.821,2.833)--(4.822,2.833)--(4.822,2.832)--(4.823,2.832)--(4.824,2.832)--(4.824,2.831)%
  --(4.825,2.831)--(4.826,2.831)--(4.826,2.830)--(4.827,2.830)--(4.828,2.830)--(4.828,2.829)%
  --(4.829,2.829)--(4.830,2.829)--(4.830,2.828)--(4.831,2.828)--(4.832,2.828)--(4.832,2.827)%
  --(4.833,2.827)--(4.834,2.827)--(4.834,2.826)--(4.835,2.826)--(4.836,2.826)--(4.836,2.825)%
  --(4.837,2.825)--(4.838,2.825)--(4.838,2.824)--(4.839,2.824)--(4.840,2.824)--(4.840,2.823)%
  --(4.841,2.823)--(4.842,2.823)--(4.842,2.822)--(4.843,2.822)--(4.844,2.822)--(4.844,2.821)%
  --(4.845,2.821)--(4.846,2.821)--(4.846,2.820)--(4.847,2.820)--(4.848,2.820)--(4.848,2.819)%
  --(4.849,2.819)--(4.850,2.819)--(4.850,2.818)--(4.851,2.818)--(4.852,2.818)--(4.852,2.817)%
  --(4.853,2.817)--(4.854,2.817)--(4.855,2.817)--(4.855,2.816)--(4.856,2.816)--(4.857,2.816)%
  --(4.857,2.815)--(4.858,2.815)--(4.859,2.815)--(4.859,2.814)--(4.860,2.814)--(4.861,2.814)%
  --(4.861,2.813)--(4.862,2.813)--(4.863,2.813)--(4.863,2.812)--(4.864,2.812)--(4.865,2.812)%
  --(4.865,2.811)--(4.866,2.811)--(4.867,2.811)--(4.867,2.810)--(4.868,2.810)--(4.869,2.810)%
  --(4.869,2.809)--(4.870,2.809)--(4.871,2.809)--(4.871,2.808)--(4.872,2.808)--(4.873,2.808)%
  --(4.873,2.807)--(4.874,2.807)--(4.875,2.807)--(4.875,2.806)--(4.876,2.806)--(4.877,2.806)%
  --(4.877,2.805)--(4.878,2.805)--(4.879,2.805)--(4.879,2.804)--(4.880,2.804)--(4.881,2.804)%
  --(4.882,2.804)--(4.882,2.803)--(4.883,2.803)--(4.884,2.803)--(4.884,2.802)--(4.885,2.802)%
  --(4.886,2.802)--(4.886,2.801)--(4.887,2.801)--(4.888,2.801)--(4.888,2.800)--(4.889,2.800)%
  --(4.890,2.800)--(4.890,2.799)--(4.891,2.799)--(4.892,2.799)--(4.892,2.798)--(4.893,2.798)%
  --(4.894,2.798)--(4.894,2.797)--(4.895,2.797)--(4.896,2.797)--(4.896,2.796)--(4.897,2.796)%
  --(4.898,2.796)--(4.898,2.795)--(4.899,2.795)--(4.900,2.795)--(4.900,2.794)--(4.901,2.794)%
  --(4.902,2.794)--(4.903,2.794)--(4.903,2.793)--(4.904,2.793)--(4.905,2.793)--(4.905,2.792)%
  --(4.906,2.792)--(4.907,2.792)--(4.907,2.791)--(4.908,2.791)--(4.909,2.791)--(4.909,2.790)%
  --(4.910,2.790)--(4.911,2.790)--(4.911,2.789)--(4.912,2.789)--(4.913,2.789)--(4.913,2.788)%
  --(4.914,2.788)--(4.915,2.788)--(4.915,2.787)--(4.916,2.787)--(4.917,2.787)--(4.917,2.786)%
  --(4.918,2.786)--(4.919,2.786)--(4.920,2.786)--(4.920,2.785)--(4.921,2.785)--(4.922,2.785)%
  --(4.922,2.784)--(4.923,2.784)--(4.924,2.784)--(4.924,2.783)--(4.925,2.783)--(4.926,2.783)%
  --(4.926,2.782)--(4.927,2.782)--(4.928,2.782)--(4.928,2.781)--(4.929,2.781)--(4.930,2.781)%
  --(4.930,2.780)--(4.931,2.780)--(4.932,2.780)--(4.932,2.779)--(4.933,2.779)--(4.934,2.779)%
  --(4.935,2.779)--(4.935,2.778)--(4.936,2.778)--(4.937,2.778)--(4.937,2.777)--(4.938,2.777)%
  --(4.939,2.777)--(4.939,2.776)--(4.940,2.776)--(4.941,2.776)--(4.941,2.775)--(4.942,2.775)%
  --(4.943,2.775)--(4.943,2.774)--(4.944,2.774)--(4.945,2.774)--(4.945,2.773)--(4.946,2.773)%
  --(4.947,2.773)--(4.948,2.773)--(4.948,2.772)--(4.949,2.772)--(4.950,2.772)--(4.950,2.771)%
  --(4.951,2.771)--(4.952,2.771)--(4.952,2.770)--(4.953,2.770)--(4.954,2.770)--(4.954,2.769)%
  --(4.955,2.769)--(4.956,2.769)--(4.956,2.768)--(4.957,2.768)--(4.958,2.768)--(4.958,2.767)%
  --(4.959,2.767)--(4.960,2.767)--(4.961,2.767)--(4.961,2.766)--(4.962,2.766)--(4.963,2.766)%
  --(4.963,2.765)--(4.964,2.765)--(4.965,2.765)--(4.965,2.764)--(4.966,2.764)--(4.967,2.764)%
  --(4.967,2.763)--(4.968,2.763)--(4.969,2.763)--(4.969,2.762)--(4.970,2.762)--(4.971,2.762)%
  --(4.971,2.761)--(4.972,2.761)--(4.973,2.761)--(4.974,2.761)--(4.974,2.760)--(4.975,2.760)%
  --(4.976,2.760)--(4.976,2.759)--(4.977,2.759)--(4.978,2.759)--(4.978,2.758)--(4.979,2.758)%
  --(4.980,2.758)--(4.980,2.757)--(4.981,2.757)--(4.982,2.757)--(4.982,2.756)--(4.983,2.756)%
  --(4.984,2.756)--(4.985,2.756)--(4.985,2.755)--(4.986,2.755)--(4.987,2.755)--(4.987,2.754)%
  --(4.988,2.754)--(4.989,2.754)--(4.989,2.753)--(4.990,2.753)--(4.991,2.753)--(4.991,2.752)%
  --(4.992,2.752)--(4.993,2.752)--(4.994,2.752)--(4.994,2.751)--(4.995,2.751)--(4.996,2.751)%
  --(4.996,2.750)--(4.997,2.750)--(4.998,2.750)--(4.998,2.749)--(4.999,2.749)--(5.000,2.749)%
  --(5.000,2.748)--(5.001,2.748)--(5.002,2.748)--(5.002,2.747)--(5.003,2.747)--(5.004,2.747)%
  --(5.005,2.747)--(5.005,2.746)--(5.006,2.746)--(5.007,2.746)--(5.007,2.745)--(5.008,2.745)%
  --(5.009,2.745)--(5.009,2.744)--(5.010,2.744)--(5.011,2.744)--(5.011,2.743)--(5.012,2.743)%
  --(5.013,2.743)--(5.014,2.743)--(5.014,2.742)--(5.015,2.742)--(5.016,2.742)--(5.016,2.741)%
  --(5.017,2.741)--(5.018,2.741)--(5.018,2.740)--(5.019,2.740)--(5.020,2.740)--(5.020,2.739)%
  --(5.021,2.739)--(5.022,2.739)--(5.023,2.738)--(5.024,2.738)--(5.025,2.738)--(5.025,2.737)%
  --(5.026,2.737)--(5.027,2.737)--(5.027,2.736)--(5.028,2.736)--(5.029,2.736)--(5.029,2.735)%
  --(5.030,2.735)--(5.031,2.735)--(5.032,2.734)--(5.033,2.734)--(5.034,2.734)--(5.034,2.733)%
  --(5.035,2.733)--(5.036,2.733)--(5.036,2.732)--(5.037,2.732)--(5.038,2.732)--(5.038,2.731)%
  --(5.039,2.731)--(5.040,2.731)--(5.041,2.731)--(5.041,2.730)--(5.042,2.730)--(5.043,2.730)%
  --(5.043,2.729)--(5.044,2.729)--(5.045,2.729)--(5.045,2.728)--(5.046,2.728)--(5.047,2.728)%
  --(5.047,2.727)--(5.048,2.727)--(5.049,2.727)--(5.050,2.727)--(5.050,2.726)--(5.051,2.726)%
  --(5.052,2.726)--(5.052,2.725)--(5.053,2.725)--(5.054,2.725)--(5.054,2.724)--(5.055,2.724)%
  --(5.056,2.724)--(5.056,2.723)--(5.057,2.723)--(5.058,2.723)--(5.059,2.723)--(5.059,2.722)%
  --(5.060,2.722)--(5.061,2.722)--(5.061,2.721)--(5.062,2.721)--(5.063,2.721)--(5.063,2.720)%
  --(5.064,2.720)--(5.065,2.720)--(5.066,2.720)--(5.066,2.719)--(5.067,2.719)--(5.068,2.719)%
  --(5.068,2.718)--(5.069,2.718)--(5.070,2.718)--(5.070,2.717)--(5.071,2.717)--(5.072,2.717)%
  --(5.073,2.717)--(5.073,2.716)--(5.074,2.716)--(5.075,2.716)--(5.075,2.715)--(5.076,2.715)%
  --(5.077,2.715)--(5.077,2.714)--(5.078,2.714)--(5.079,2.714)--(5.079,2.713)--(5.080,2.713)%
  --(5.081,2.713)--(5.082,2.713)--(5.082,2.712)--(5.083,2.712)--(5.084,2.712)--(5.084,2.711)%
  --(5.085,2.711)--(5.086,2.711)--(5.086,2.710)--(5.087,2.710)--(5.088,2.710)--(5.089,2.710)%
  --(5.089,2.709)--(5.090,2.709)--(5.091,2.709)--(5.091,2.708)--(5.092,2.708)--(5.093,2.708)%
  --(5.093,2.707)--(5.094,2.707)--(5.095,2.707)--(5.096,2.707)--(5.096,2.706)--(5.097,2.706)%
  --(5.098,2.706)--(5.098,2.705)--(5.099,2.705)--(5.100,2.705)--(5.100,2.704)--(5.101,2.704)%
  --(5.102,2.704)--(5.103,2.704)--(5.103,2.703)--(5.104,2.703)--(5.105,2.703)--(5.105,2.702)%
  --(5.106,2.702)--(5.107,2.702)--(5.107,2.701)--(5.108,2.701)--(5.109,2.701)--(5.110,2.701)%
  --(5.110,2.700)--(5.111,2.700)--(5.112,2.700)--(5.112,2.699)--(5.113,2.699)--(5.114,2.699)%
  --(5.114,2.698)--(5.115,2.698)--(5.116,2.698)--(5.117,2.698)--(5.117,2.697)--(5.118,2.697)%
  --(5.119,2.697)--(5.119,2.696)--(5.120,2.696)--(5.121,2.696)--(5.121,2.695)--(5.122,2.695)%
  --(5.123,2.695)--(5.124,2.695)--(5.124,2.694)--(5.125,2.694)--(5.126,2.694)--(5.126,2.693)%
  --(5.127,2.693)--(5.128,2.693)--(5.129,2.693)--(5.129,2.692)--(5.130,2.692)--(5.131,2.692)%
  --(5.131,2.691)--(5.132,2.691)--(5.133,2.691)--(5.133,2.690)--(5.134,2.690)--(5.135,2.690)%
  --(5.136,2.690)--(5.136,2.689)--(5.137,2.689)--(5.138,2.689)--(5.138,2.688)--(5.139,2.688)%
  --(5.140,2.688)--(5.140,2.687)--(5.141,2.687)--(5.142,2.687)--(5.143,2.687)--(5.143,2.686)%
  --(5.144,2.686)--(5.145,2.686)--(5.145,2.685)--(5.146,2.685)--(5.147,2.685)--(5.148,2.685)%
  --(5.148,2.684)--(5.149,2.684)--(5.150,2.684)--(5.150,2.683)--(5.151,2.683)--(5.152,2.683)%
  --(5.152,2.682)--(5.153,2.682)--(5.154,2.682)--(5.155,2.682)--(5.155,2.681)--(5.156,2.681)%
  --(5.157,2.681)--(5.157,2.680)--(5.158,2.680)--(5.159,2.680)--(5.159,2.679)--(5.160,2.679)%
  --(5.161,2.679)--(5.162,2.679)--(5.162,2.678)--(5.163,2.678)--(5.164,2.678)--(5.164,2.677)%
  --(5.165,2.677)--(5.166,2.677)--(5.167,2.677)--(5.167,2.676)--(5.168,2.676)--(5.169,2.676)%
  --(5.169,2.675)--(5.170,2.675)--(5.171,2.675)--(5.172,2.675)--(5.172,2.674)--(5.173,2.674)%
  --(5.174,2.674)--(5.174,2.673)--(5.175,2.673)--(5.176,2.673)--(5.176,2.672)--(5.177,2.672)%
  --(5.178,2.672)--(5.179,2.672)--(5.179,2.671)--(5.180,2.671)--(5.181,2.671)--(5.181,2.670)%
  --(5.182,2.670)--(5.183,2.670)--(5.184,2.670)--(5.184,2.669)--(5.185,2.669)--(5.186,2.669)%
  --(5.186,2.668)--(5.187,2.668)--(5.188,2.668)--(5.188,2.667)--(5.189,2.667)--(5.190,2.667)%
  --(5.191,2.667)--(5.191,2.666)--(5.192,2.666)--(5.193,2.666)--(5.193,2.665)--(5.194,2.665)%
  --(5.195,2.665)--(5.196,2.665)--(5.196,2.664)--(5.197,2.664)--(5.198,2.664)--(5.198,2.663)%
  --(5.199,2.663)--(5.200,2.663)--(5.201,2.663)--(5.201,2.662)--(5.202,2.662)--(5.203,2.662)%
  --(5.203,2.661)--(5.204,2.661)--(5.205,2.661)--(5.206,2.661)--(5.206,2.660)--(5.207,2.660)%
  --(5.208,2.660)--(5.208,2.659)--(5.209,2.659)--(5.210,2.659)--(5.210,2.658)--(5.211,2.658)%
  --(5.212,2.658)--(5.213,2.658)--(5.213,2.657)--(5.214,2.657)--(5.215,2.657)--(5.215,2.656)%
  --(5.216,2.656)--(5.217,2.656)--(5.218,2.656)--(5.218,2.655)--(5.219,2.655)--(5.220,2.655)%
  --(5.220,2.654)--(5.221,2.654)--(5.222,2.654)--(5.223,2.654)--(5.223,2.653)--(5.224,2.653)%
  --(5.225,2.653)--(5.225,2.652)--(5.226,2.652)--(5.227,2.652)--(5.228,2.652)--(5.228,2.651)%
  --(5.229,2.651)--(5.230,2.651)--(5.230,2.650)--(5.231,2.650)--(5.232,2.650)--(5.233,2.650)%
  --(5.233,2.649)--(5.234,2.649)--(5.235,2.649)--(5.235,2.648)--(5.236,2.648)--(5.237,2.648)%
  --(5.238,2.648)--(5.238,2.647)--(5.239,2.647)--(5.240,2.647)--(5.240,2.646)--(5.241,2.646)%
  --(5.242,2.646)--(5.243,2.646)--(5.243,2.645)--(5.244,2.645)--(5.245,2.645)--(5.245,2.644)%
  --(5.246,2.644)--(5.247,2.644)--(5.248,2.644)--(5.248,2.643)--(5.249,2.643)--(5.250,2.643)%
  --(5.250,2.642)--(5.251,2.642)--(5.252,2.642)--(5.253,2.642)--(5.253,2.641)--(5.254,2.641)%
  --(5.255,2.641)--(5.255,2.640)--(5.256,2.640)--(5.257,2.640)--(5.258,2.640)--(5.258,2.639)%
  --(5.259,2.639)--(5.260,2.639)--(5.260,2.638)--(5.261,2.638)--(5.262,2.638)--(5.263,2.638)%
  --(5.263,2.637)--(5.264,2.637)--(5.265,2.637)--(5.265,2.636)--(5.266,2.636)--(5.267,2.636)%
  --(5.268,2.636)--(5.268,2.635)--(5.269,2.635)--(5.270,2.635)--(5.270,2.634)--(5.271,2.634)%
  --(5.272,2.634)--(5.273,2.634)--(5.273,2.633)--(5.274,2.633)--(5.275,2.633)--(5.275,2.632)%
  --(5.276,2.632)--(5.277,2.632)--(5.278,2.632)--(5.278,2.631)--(5.279,2.631)--(5.280,2.631)%
  --(5.280,2.630)--(5.281,2.630)--(5.282,2.630)--(5.283,2.630)--(5.283,2.629)--(5.284,2.629)%
  --(5.285,2.629)--(5.286,2.629)--(5.286,2.628)--(5.287,2.628)--(5.288,2.628)--(5.288,2.627)%
  --(5.289,2.627)--(5.290,2.627)--(5.291,2.627)--(5.291,2.626)--(5.292,2.626)--(5.293,2.626)%
  --(5.293,2.625)--(5.294,2.625)--(5.295,2.625)--(5.296,2.625)--(5.296,2.624)--(5.297,2.624)%
  --(5.298,2.624)--(5.298,2.623)--(5.299,2.623)--(5.300,2.623)--(5.301,2.623)--(5.301,2.622)%
  --(5.302,2.622)--(5.303,2.622)--(5.303,2.621)--(5.304,2.621)--(5.305,2.621)--(5.306,2.621)%
  --(5.306,2.620)--(5.307,2.620)--(5.308,2.620)--(5.309,2.620)--(5.309,2.619)--(5.310,2.619)%
  --(5.311,2.619)--(5.311,2.618)--(5.312,2.618)--(5.313,2.618)--(5.314,2.618)--(5.314,2.617)%
  --(5.315,2.617)--(5.316,2.617)--(5.316,2.616)--(5.317,2.616)--(5.318,2.616)--(5.319,2.616)%
  --(5.319,2.615)--(5.320,2.615)--(5.321,2.615)--(5.322,2.614)--(5.323,2.614)--(5.324,2.614)%
  --(5.324,2.613)--(5.325,2.613)--(5.326,2.613)--(5.327,2.613)--(5.327,2.612)--(5.328,2.612)%
  --(5.329,2.612)--(5.329,2.611)--(5.330,2.611)--(5.331,2.611)--(5.332,2.611)--(5.332,2.610)%
  --(5.333,2.610)--(5.334,2.610)--(5.334,2.609)--(5.335,2.609)--(5.336,2.609)--(5.337,2.609)%
  --(5.337,2.608)--(5.338,2.608)--(5.339,2.608)--(5.340,2.608)--(5.340,2.607)--(5.341,2.607)%
  --(5.342,2.607)--(5.342,2.606)--(5.343,2.606)--(5.344,2.606)--(5.345,2.606)--(5.345,2.605)%
  --(5.346,2.605)--(5.347,2.605)--(5.348,2.605)--(5.348,2.604)--(5.349,2.604)--(5.350,2.604)%
  --(5.350,2.603)--(5.351,2.603)--(5.352,2.603)--(5.353,2.603)--(5.353,2.602)--(5.354,2.602)%
  --(5.355,2.602)--(5.355,2.601)--(5.356,2.601)--(5.357,2.601)--(5.358,2.601)--(5.358,2.600)%
  --(5.359,2.600)--(5.360,2.600)--(5.361,2.600)--(5.361,2.599)--(5.362,2.599)--(5.363,2.599)%
  --(5.363,2.598)--(5.364,2.598)--(5.365,2.598)--(5.366,2.598)--(5.366,2.597)--(5.367,2.597)%
  --(5.368,2.597)--(5.369,2.597)--(5.369,2.596)--(5.370,2.596)--(5.371,2.596)--(5.371,2.595)%
  --(5.372,2.595)--(5.373,2.595)--(5.374,2.595)--(5.374,2.594)--(5.375,2.594)--(5.376,2.594)%
  --(5.377,2.594)--(5.377,2.593)--(5.378,2.593)--(5.379,2.593)--(5.379,2.592)--(5.380,2.592)%
  --(5.381,2.592)--(5.382,2.592)--(5.382,2.591)--(5.383,2.591)--(5.384,2.591)--(5.385,2.591)%
  --(5.385,2.590)--(5.386,2.590)--(5.387,2.590)--(5.387,2.589)--(5.388,2.589)--(5.389,2.589)%
  --(5.390,2.589)--(5.390,2.588)--(5.391,2.588)--(5.392,2.588)--(5.393,2.588)--(5.393,2.587)%
  --(5.394,2.587)--(5.395,2.587)--(5.395,2.586)--(5.396,2.586)--(5.397,2.586)--(5.398,2.586)%
  --(5.398,2.585)--(5.399,2.585)--(5.400,2.585)--(5.401,2.585)--(5.401,2.584)--(5.402,2.584)%
  --(5.403,2.584)--(5.403,2.583)--(5.404,2.583)--(5.405,2.583)--(5.406,2.583)--(5.406,2.582)%
  --(5.407,2.582)--(5.408,2.582)--(5.409,2.582)--(5.409,2.581)--(5.410,2.581)--(5.411,2.581)%
  --(5.411,2.580)--(5.412,2.580)--(5.413,2.580)--(5.414,2.580)--(5.414,2.579)--(5.415,2.579)%
  --(5.416,2.579)--(5.417,2.579)--(5.417,2.578)--(5.418,2.578)--(5.419,2.578)--(5.419,2.577)%
  --(5.420,2.577)--(5.421,2.577)--(5.422,2.577)--(5.422,2.576)--(5.423,2.576)--(5.424,2.576)%
  --(5.425,2.576)--(5.425,2.575)--(5.426,2.575)--(5.427,2.575)--(5.428,2.575)--(5.428,2.574)%
  --(5.429,2.574)--(5.430,2.574)--(5.430,2.573)--(5.431,2.573)--(5.432,2.573)--(5.433,2.573)%
  --(5.433,2.572)--(5.434,2.572)--(5.435,2.572)--(5.436,2.572)--(5.436,2.571)--(5.437,2.571)%
  --(5.438,2.571)--(5.439,2.570)--(5.440,2.570)--(5.441,2.570)--(5.441,2.569)--(5.442,2.569)%
  --(5.443,2.569)--(5.444,2.569)--(5.444,2.568)--(5.445,2.568)--(5.446,2.568)--(5.447,2.568)%
  --(5.447,2.567)--(5.448,2.567)--(5.449,2.567)--(5.449,2.566)--(5.450,2.566)--(5.451,2.566)%
  --(5.452,2.566)--(5.452,2.565)--(5.453,2.565)--(5.454,2.565)--(5.455,2.565)--(5.455,2.564)%
  --(5.456,2.564)--(5.457,2.564)--(5.458,2.564)--(5.458,2.563)--(5.459,2.563)--(5.460,2.563)%
  --(5.460,2.562)--(5.461,2.562)--(5.462,2.562)--(5.463,2.562)--(5.463,2.561)--(5.464,2.561)%
  --(5.465,2.561)--(5.466,2.561)--(5.466,2.560)--(5.467,2.560)--(5.468,2.560)--(5.469,2.560)%
  --(5.469,2.559)--(5.470,2.559)--(5.471,2.559)--(5.472,2.559)--(5.472,2.558)--(5.473,2.558)%
  --(5.474,2.558)--(5.474,2.557)--(5.475,2.557)--(5.476,2.557)--(5.477,2.557)--(5.477,2.556)%
  --(5.478,2.556)--(5.479,2.556)--(5.480,2.556)--(5.480,2.555)--(5.481,2.555)--(5.482,2.555)%
  --(5.483,2.555)--(5.483,2.554)--(5.484,2.554)--(5.485,2.554)--(5.485,2.553)--(5.486,2.553)%
  --(5.487,2.553)--(5.488,2.553)--(5.488,2.552)--(5.489,2.552)--(5.490,2.552)--(5.491,2.552)%
  --(5.491,2.551)--(5.492,2.551)--(5.493,2.551)--(5.494,2.551)--(5.494,2.550)--(5.495,2.550)%
  --(5.496,2.550)--(5.497,2.550)--(5.497,2.549)--(5.498,2.549)--(5.499,2.549)--(5.499,2.548)%
  --(5.500,2.548)--(5.501,2.548)--(5.502,2.548)--(5.502,2.547)--(5.503,2.547)--(5.504,2.547)%
  --(5.505,2.547)--(5.505,2.546)--(5.506,2.546)--(5.507,2.546)--(5.508,2.546)--(5.508,2.545)%
  --(5.509,2.545)--(5.510,2.545)--(5.511,2.545)--(5.511,2.544)--(5.512,2.544)--(5.513,2.544)%
  --(5.513,2.543)--(5.514,2.543)--(5.515,2.543)--(5.516,2.543)--(5.516,2.542)--(5.517,2.542)%
  --(5.518,2.542)--(5.519,2.542)--(5.519,2.541)--(5.520,2.541)--(5.521,2.541)--(5.522,2.541)%
  --(5.522,2.540)--(5.523,2.540)--(5.524,2.540)--(5.525,2.540)--(5.525,2.539)--(5.526,2.539)%
  --(5.527,2.539)--(5.528,2.539)--(5.528,2.538)--(5.529,2.538)--(5.530,2.538)--(5.530,2.537)%
  --(5.531,2.537)--(5.532,2.537)--(5.533,2.537)--(5.533,2.536)--(5.534,2.536)--(5.535,2.536)%
  --(5.536,2.536)--(5.536,2.535)--(5.537,2.535)--(5.538,2.535)--(5.539,2.535)--(5.539,2.534)%
  --(5.540,2.534)--(5.541,2.534)--(5.542,2.534)--(5.542,2.533)--(5.543,2.533)--(5.544,2.533)%
  --(5.545,2.533)--(5.545,2.532)--(5.546,2.532)--(5.547,2.532)--(5.548,2.532)--(5.548,2.531)%
  --(5.549,2.531)--(5.550,2.531)--(5.550,2.530)--(5.551,2.530)--(5.552,2.530)--(5.553,2.530)%
  --(5.553,2.529)--(5.554,2.529)--(5.555,2.529)--(5.556,2.529)--(5.556,2.528)--(5.557,2.528)%
  --(5.558,2.528)--(5.559,2.528)--(5.559,2.527)--(5.560,2.527)--(5.561,2.527)--(5.562,2.527)%
  --(5.562,2.526)--(5.563,2.526)--(5.564,2.526)--(5.565,2.526)--(5.565,2.525)--(5.566,2.525)%
  --(5.567,2.525)--(5.568,2.525)--(5.568,2.524)--(5.569,2.524)--(5.570,2.524)--(5.571,2.524)%
  --(5.571,2.523)--(5.572,2.523)--(5.573,2.523)--(5.574,2.523)--(5.574,2.522)--(5.575,2.522)%
  --(5.576,2.522)--(5.576,2.521)--(5.577,2.521)--(5.578,2.521)--(5.579,2.521)--(5.579,2.520)%
  --(5.580,2.520)--(5.581,2.520)--(5.582,2.520)--(5.582,2.519)--(5.583,2.519)--(5.584,2.519)%
  --(5.585,2.519)--(5.585,2.518)--(5.586,2.518)--(5.587,2.518)--(5.588,2.518)--(5.588,2.517)%
  --(5.589,2.517)--(5.590,2.517)--(5.591,2.517)--(5.591,2.516)--(5.592,2.516)--(5.593,2.516)%
  --(5.594,2.516)--(5.594,2.515)--(5.595,2.515)--(5.596,2.515)--(5.597,2.515)--(5.597,2.514)%
  --(5.598,2.514)--(5.599,2.514)--(5.600,2.514)--(5.600,2.513)--(5.601,2.513)--(5.602,2.513)%
  --(5.603,2.513)--(5.603,2.512)--(5.604,2.512)--(5.605,2.512)--(5.606,2.512)--(5.606,2.511)%
  --(5.607,2.511)--(5.608,2.511)--(5.609,2.511)--(5.609,2.510)--(5.610,2.510)--(5.611,2.510)%
  --(5.612,2.510)--(5.612,2.509)--(5.613,2.509)--(5.614,2.509)--(5.615,2.509)--(5.615,2.508)%
  --(5.616,2.508)--(5.617,2.508)--(5.617,2.507)--(5.618,2.507)--(5.619,2.507)--(5.620,2.507)%
  --(5.620,2.506)--(5.621,2.506)--(5.622,2.506)--(5.623,2.506)--(5.623,2.505)--(5.624,2.505)%
  --(5.625,2.505)--(5.626,2.505)--(5.626,2.504)--(5.627,2.504)--(5.628,2.504)--(5.629,2.504)%
  --(5.629,2.503)--(5.630,2.503)--(5.631,2.503)--(5.632,2.503)--(5.632,2.502)--(5.633,2.502)%
  --(5.634,2.502)--(5.635,2.502)--(5.635,2.501)--(5.636,2.501)--(5.637,2.501)--(5.638,2.501)%
  --(5.638,2.500)--(5.639,2.500)--(5.640,2.500)--(5.641,2.500)--(5.641,2.499)--(5.642,2.499)%
  --(5.643,2.499)--(5.644,2.499)--(5.644,2.498)--(5.645,2.498)--(5.646,2.498)--(5.647,2.498)%
  --(5.647,2.497)--(5.648,2.497)--(5.649,2.497)--(5.650,2.497)--(5.650,2.496)--(5.651,2.496)%
  --(5.652,2.496)--(5.653,2.496)--(5.653,2.495)--(5.654,2.495)--(5.655,2.495)--(5.656,2.495)%
  --(5.656,2.494)--(5.657,2.494)--(5.658,2.494)--(5.659,2.494)--(5.659,2.493)--(5.660,2.493)%
  --(5.661,2.493)--(5.662,2.493)--(5.662,2.492)--(5.663,2.492)--(5.664,2.492)--(5.665,2.492)%
  --(5.665,2.491)--(5.666,2.491)--(5.667,2.491)--(5.668,2.491)--(5.668,2.490)--(5.669,2.490)%
  --(5.670,2.490)--(5.671,2.490)--(5.671,2.489)--(5.672,2.489)--(5.673,2.489)--(5.674,2.489)%
  --(5.674,2.488)--(5.675,2.488)--(5.676,2.488)--(5.677,2.488)--(5.677,2.487)--(5.678,2.487)%
  --(5.679,2.487)--(5.680,2.487)--(5.680,2.486)--(5.681,2.486)--(5.682,2.486)--(5.683,2.486)%
  --(5.683,2.485)--(5.684,2.485)--(5.685,2.485)--(5.686,2.485)--(5.686,2.484)--(5.687,2.484)%
  --(5.688,2.484)--(5.689,2.484)--(5.690,2.484)--(5.690,2.483)--(5.691,2.483)--(5.692,2.483)%
  --(5.693,2.483)--(5.693,2.482)--(5.694,2.482)--(5.695,2.482)--(5.696,2.482)--(5.696,2.481)%
  --(5.697,2.481)--(5.698,2.481)--(5.699,2.481)--(5.699,2.480)--(5.700,2.480)--(5.701,2.480)%
  --(5.702,2.480)--(5.702,2.479)--(5.703,2.479)--(5.704,2.479)--(5.705,2.479)--(5.705,2.478)%
  --(5.706,2.478)--(5.707,2.478)--(5.708,2.478)--(5.708,2.477)--(5.709,2.477)--(5.710,2.477)%
  --(5.711,2.477)--(5.711,2.476)--(5.712,2.476)--(5.713,2.476)--(5.714,2.476)--(5.714,2.475)%
  --(5.715,2.475)--(5.716,2.475)--(5.717,2.475)--(5.717,2.474)--(5.718,2.474)--(5.719,2.474)%
  --(5.720,2.474)--(5.720,2.473)--(5.721,2.473)--(5.722,2.473)--(5.723,2.473)--(5.723,2.472)%
  --(5.724,2.472)--(5.725,2.472)--(5.726,2.472)--(5.726,2.471)--(5.727,2.471)--(5.728,2.471)%
  --(5.729,2.471)--(5.729,2.470)--(5.730,2.470)--(5.731,2.470)--(5.732,2.470)--(5.733,2.470)%
  --(5.733,2.469)--(5.734,2.469)--(5.735,2.469)--(5.736,2.469)--(5.736,2.468)--(5.737,2.468)%
  --(5.738,2.468)--(5.739,2.468)--(5.739,2.467)--(5.740,2.467)--(5.741,2.467)--(5.742,2.467)%
  --(5.742,2.466)--(5.743,2.466)--(5.744,2.466)--(5.745,2.466)--(5.745,2.465)--(5.746,2.465)%
  --(5.747,2.465)--(5.748,2.465)--(5.748,2.464)--(5.749,2.464)--(5.750,2.464)--(5.751,2.464)%
  --(5.751,2.463)--(5.752,2.463)--(5.753,2.463)--(5.754,2.463)--(5.754,2.462)--(5.755,2.462)%
  --(5.756,2.462)--(5.757,2.462)--(5.758,2.462)--(5.758,2.461)--(5.759,2.461)--(5.760,2.461)%
  --(5.761,2.461)--(5.761,2.460)--(5.762,2.460)--(5.763,2.460)--(5.764,2.460)--(5.764,2.459)%
  --(5.765,2.459)--(5.766,2.459)--(5.767,2.459)--(5.767,2.458)--(5.768,2.458)--(5.769,2.458)%
  --(5.770,2.458)--(5.770,2.457)--(5.771,2.457)--(5.772,2.457)--(5.773,2.457)--(5.773,2.456)%
  --(5.774,2.456)--(5.775,2.456)--(5.776,2.456)--(5.776,2.455)--(5.777,2.455)--(5.778,2.455)%
  --(5.779,2.455)--(5.780,2.455)--(5.780,2.454)--(5.781,2.454)--(5.782,2.454)--(5.783,2.454)%
  --(5.783,2.453)--(5.784,2.453)--(5.785,2.453)--(5.786,2.453)--(5.786,2.452)--(5.787,2.452)%
  --(5.788,2.452)--(5.789,2.452)--(5.789,2.451)--(5.790,2.451)--(5.791,2.451)--(5.792,2.451)%
  --(5.792,2.450)--(5.793,2.450)--(5.794,2.450)--(5.795,2.450)--(5.796,2.450)--(5.796,2.449)%
  --(5.797,2.449)--(5.798,2.449)--(5.799,2.449)--(5.799,2.448)--(5.800,2.448)--(5.801,2.448)%
  --(5.802,2.448)--(5.802,2.447)--(5.803,2.447)--(5.804,2.447)--(5.805,2.447)--(5.805,2.446)%
  --(5.806,2.446)--(5.807,2.446)--(5.808,2.446)--(5.808,2.445)--(5.809,2.445)--(5.810,2.445)%
  --(5.811,2.445)--(5.812,2.445)--(5.812,2.444)--(5.813,2.444)--(5.814,2.444)--(5.815,2.444)%
  --(5.815,2.443)--(5.816,2.443)--(5.817,2.443)--(5.818,2.443)--(5.818,2.442)--(5.819,2.442)%
  --(5.820,2.442)--(5.821,2.442)--(5.821,2.441)--(5.822,2.441)--(5.823,2.441)--(5.824,2.441)%
  --(5.824,2.440)--(5.825,2.440)--(5.826,2.440)--(5.827,2.440)--(5.828,2.440)--(5.828,2.439)%
  --(5.829,2.439)--(5.830,2.439)--(5.831,2.439)--(5.831,2.438)--(5.832,2.438)--(5.833,2.438)%
  --(5.834,2.438)--(5.834,2.437)--(5.835,2.437)--(5.836,2.437)--(5.837,2.437)--(5.837,2.436)%
  --(5.838,2.436)--(5.839,2.436)--(5.840,2.436)--(5.841,2.436)--(5.841,2.435)--(5.842,2.435)%
  --(5.843,2.435)--(5.844,2.435)--(5.844,2.434)--(5.845,2.434)--(5.846,2.434)--(5.847,2.434)%
  --(5.847,2.433)--(5.848,2.433)--(5.849,2.433)--(5.850,2.433)--(5.850,2.432)--(5.851,2.432)%
  --(5.852,2.432)--(5.853,2.432)--(5.854,2.432)--(5.854,2.431)--(5.855,2.431)--(5.856,2.431)%
  --(5.857,2.431)--(5.857,2.430)--(5.858,2.430)--(5.859,2.430)--(5.860,2.430)--(5.860,2.429)%
  --(5.861,2.429)--(5.862,2.429)--(5.863,2.429)--(5.863,2.428)--(5.864,2.428)--(5.865,2.428)%
  --(5.866,2.428)--(5.867,2.428)--(5.867,2.427)--(5.868,2.427)--(5.869,2.427)--(5.870,2.427)%
  --(5.870,2.426)--(5.871,2.426)--(5.872,2.426)--(5.873,2.426)--(5.873,2.425)--(5.874,2.425)%
  --(5.875,2.425)--(5.876,2.425)--(5.877,2.425)--(5.877,2.424)--(5.878,2.424)--(5.879,2.424)%
  --(5.880,2.424)--(5.880,2.423)--(5.881,2.423)--(5.882,2.423)--(5.883,2.423)--(5.883,2.422)%
  --(5.884,2.422)--(5.885,2.422)--(5.886,2.422)--(5.887,2.421)--(5.888,2.421)--(5.889,2.421)%
  --(5.890,2.421)--(5.890,2.420)--(5.891,2.420)--(5.892,2.420)--(5.893,2.420)--(5.893,2.419)%
  --(5.894,2.419)--(5.895,2.419)--(5.896,2.419)--(5.896,2.418)--(5.897,2.418)--(5.898,2.418)%
  --(5.899,2.418)--(5.900,2.418)--(5.900,2.417)--(5.901,2.417)--(5.902,2.417)--(5.903,2.417)%
  --(5.903,2.416)--(5.904,2.416)--(5.905,2.416)--(5.906,2.416)--(5.906,2.415)--(5.907,2.415)%
  --(5.908,2.415)--(5.909,2.415)--(5.910,2.415)--(5.910,2.414)--(5.911,2.414)--(5.912,2.414)%
  --(5.913,2.414)--(5.913,2.413)--(5.914,2.413)--(5.915,2.413)--(5.916,2.413)--(5.916,2.412)%
  --(5.917,2.412)--(5.918,2.412)--(5.919,2.412)--(5.920,2.412)--(5.920,2.411)--(5.921,2.411)%
  --(5.922,2.411)--(5.923,2.411)--(5.923,2.410)--(5.924,2.410)--(5.925,2.410)--(5.926,2.410)%
  --(5.927,2.410)--(5.927,2.409)--(5.928,2.409)--(5.929,2.409)--(5.930,2.409)--(5.930,2.408)%
  --(5.931,2.408)--(5.932,2.408)--(5.933,2.408)--(5.933,2.407)--(5.934,2.407)--(5.935,2.407)%
  --(5.936,2.407)--(5.937,2.407)--(5.937,2.406)--(5.938,2.406)--(5.939,2.406)--(5.940,2.406)%
  --(5.940,2.405)--(5.941,2.405)--(5.942,2.405)--(5.943,2.405)--(5.943,2.404)--(5.944,2.404)%
  --(5.945,2.404)--(5.946,2.404)--(5.947,2.404)--(5.947,2.403)--(5.948,2.403)--(5.949,2.403)%
  --(5.950,2.403)--(5.950,2.402)--(5.951,2.402)--(5.952,2.402)--(5.953,2.402)--(5.954,2.402)%
  --(5.954,2.401)--(5.955,2.401)--(5.956,2.401)--(5.957,2.401)--(5.957,2.400)--(5.958,2.400)%
  --(5.959,2.400)--(5.960,2.400)--(5.960,2.399)--(5.961,2.399)--(5.962,2.399)--(5.963,2.399)%
  --(5.964,2.399)--(5.964,2.398)--(5.965,2.398)--(5.966,2.398)--(5.967,2.398)--(5.967,2.397)%
  --(5.968,2.397)--(5.969,2.397)--(5.970,2.397)--(5.971,2.397)--(5.971,2.396)--(5.972,2.396)%
  --(5.973,2.396)--(5.974,2.396)--(5.974,2.395)--(5.975,2.395)--(5.976,2.395)--(5.977,2.395)%
  --(5.978,2.395)--(5.978,2.394)--(5.979,2.394)--(5.980,2.394)--(5.981,2.394)--(5.981,2.393)%
  --(5.982,2.393)--(5.983,2.393)--(5.984,2.393)--(5.984,2.392)--(5.985,2.392)--(5.986,2.392)%
  --(5.987,2.392)--(5.988,2.392)--(5.988,2.391)--(5.989,2.391)--(5.990,2.391)--(5.991,2.391)%
  --(5.991,2.390)--(5.992,2.390)--(5.993,2.390)--(5.994,2.390)--(5.995,2.390)--(5.995,2.389)%
  --(5.996,2.389)--(5.997,2.389)--(5.998,2.389)--(5.998,2.388)--(5.999,2.388)--(6.000,2.388)%
  --(6.001,2.388)--(6.002,2.388)--(6.002,2.387)--(6.003,2.387)--(6.004,2.387)--(6.005,2.387)%
  --(6.005,2.386)--(6.006,2.386)--(6.007,2.386)--(6.008,2.386)--(6.009,2.386)--(6.009,2.385)%
  --(6.010,2.385)--(6.011,2.385)--(6.012,2.385)--(6.012,2.384)--(6.013,2.384)--(6.014,2.384)%
  --(6.015,2.384)--(6.016,2.384)--(6.016,2.383)--(6.017,2.383)--(6.018,2.383)--(6.019,2.383)%
  --(6.019,2.382)--(6.020,2.382)--(6.021,2.382)--(6.022,2.382)--(6.023,2.382)--(6.023,2.381)%
  --(6.024,2.381)--(6.025,2.381)--(6.026,2.381)--(6.026,2.380)--(6.027,2.380)--(6.028,2.380)%
  --(6.029,2.380)--(6.030,2.380)--(6.030,2.379)--(6.031,2.379)--(6.032,2.379)--(6.033,2.379)%
  --(6.033,2.378)--(6.034,2.378)--(6.035,2.378)--(6.036,2.378)--(6.037,2.378)--(6.037,2.377)%
  --(6.038,2.377)--(6.039,2.377)--(6.040,2.377)--(6.040,2.376)--(6.041,2.376)--(6.042,2.376)%
  --(6.043,2.376)--(6.044,2.376)--(6.044,2.375)--(6.045,2.375)--(6.046,2.375)--(6.047,2.375)%
  --(6.047,2.374)--(6.048,2.374)--(6.049,2.374)--(6.050,2.374)--(6.051,2.374)--(6.051,2.373)%
  --(6.052,2.373)--(6.053,2.373)--(6.054,2.373)--(6.054,2.372)--(6.055,2.372)--(6.056,2.372)%
  --(6.057,2.372)--(6.058,2.372)--(6.058,2.371)--(6.059,2.371)--(6.060,2.371)--(6.061,2.371)%
  --(6.061,2.370)--(6.062,2.370)--(6.063,2.370)--(6.064,2.370)--(6.065,2.370)--(6.065,2.369)%
  --(6.066,2.369)--(6.067,2.369)--(6.068,2.369)--(6.069,2.369)--(6.069,2.368)--(6.070,2.368)%
  --(6.071,2.368)--(6.072,2.368)--(6.072,2.367)--(6.073,2.367)--(6.074,2.367)--(6.075,2.367)%
  --(6.076,2.367)--(6.076,2.366)--(6.077,2.366)--(6.078,2.366)--(6.079,2.366)--(6.079,2.365)%
  --(6.080,2.365)--(6.081,2.365)--(6.082,2.365)--(6.083,2.365)--(6.083,2.364)--(6.084,2.364)%
  --(6.085,2.364)--(6.086,2.364)--(6.086,2.363)--(6.087,2.363)--(6.088,2.363)--(6.089,2.363)%
  --(6.090,2.363)--(6.090,2.362)--(6.091,2.362)--(6.092,2.362)--(6.093,2.362)--(6.094,2.362)%
  --(6.094,2.361)--(6.095,2.361)--(6.096,2.361)--(6.097,2.361)--(6.097,2.360)--(6.098,2.360)%
  --(6.099,2.360)--(6.100,2.360)--(6.101,2.360)--(6.101,2.359)--(6.102,2.359)--(6.103,2.359)%
  --(6.104,2.359)--(6.104,2.358)--(6.105,2.358)--(6.106,2.358)--(6.107,2.358)--(6.108,2.358)%
  --(6.108,2.357)--(6.109,2.357)--(6.110,2.357)--(6.111,2.357)--(6.112,2.357)--(6.112,2.356)%
  --(6.113,2.356)--(6.114,2.356)--(6.115,2.356)--(6.115,2.355)--(6.116,2.355)--(6.117,2.355)%
  --(6.118,2.355)--(6.119,2.355)--(6.119,2.354)--(6.120,2.354)--(6.121,2.354)--(6.122,2.354)%
  --(6.123,2.354)--(6.123,2.353)--(6.124,2.353)--(6.125,2.353)--(6.126,2.353)--(6.126,2.352)%
  --(6.127,2.352)--(6.128,2.352)--(6.129,2.352)--(6.130,2.352)--(6.130,2.351)--(6.131,2.351)%
  --(6.132,2.351)--(6.133,2.351)--(6.134,2.351)--(6.134,2.350)--(6.135,2.350)--(6.136,2.350)%
  --(6.137,2.350)--(6.137,2.349)--(6.138,2.349)--(6.139,2.349)--(6.140,2.349)--(6.141,2.349)%
  --(6.141,2.348)--(6.142,2.348)--(6.143,2.348)--(6.144,2.348)--(6.145,2.348)--(6.145,2.347)%
  --(6.146,2.347)--(6.147,2.347)--(6.148,2.347)--(6.148,2.346)--(6.149,2.346)--(6.150,2.346)%
  --(6.151,2.346)--(6.152,2.346)--(6.152,2.345)--(6.153,2.345)--(6.154,2.345)--(6.155,2.345)%
  --(6.156,2.345)--(6.156,2.344)--(6.157,2.344)--(6.158,2.344)--(6.159,2.344)--(6.159,2.343)%
  --(6.160,2.343)--(6.161,2.343)--(6.162,2.343)--(6.163,2.343)--(6.163,2.342)--(6.164,2.342)%
  --(6.165,2.342)--(6.166,2.342)--(6.167,2.342)--(6.167,2.341)--(6.168,2.341)--(6.169,2.341)%
  --(6.170,2.341)--(6.170,2.340)--(6.171,2.340)--(6.172,2.340)--(6.173,2.340)--(6.174,2.340)%
  --(6.174,2.339)--(6.175,2.339)--(6.176,2.339)--(6.177,2.339)--(6.178,2.339)--(6.178,2.338)%
  --(6.179,2.338)--(6.180,2.338)--(6.181,2.338)--(6.182,2.338)--(6.182,2.337)--(6.183,2.337)%
  --(6.184,2.337)--(6.185,2.337)--(6.185,2.336)--(6.186,2.336)--(6.187,2.336)--(6.188,2.336)%
  --(6.189,2.336)--(6.189,2.335)--(6.190,2.335)--(6.191,2.335)--(6.192,2.335)--(6.193,2.335)%
  --(6.193,2.334)--(6.194,2.334)--(6.195,2.334)--(6.196,2.334)--(6.196,2.333)--(6.197,2.333)%
  --(6.198,2.333)--(6.199,2.333)--(6.200,2.333)--(6.200,2.332)--(6.201,2.332)--(6.202,2.332)%
  --(6.203,2.332)--(6.204,2.332)--(6.204,2.331)--(6.205,2.331)--(6.206,2.331)--(6.207,2.331)%
  --(6.208,2.331)--(6.208,2.330)--(6.209,2.330)--(6.210,2.330)--(6.211,2.330)--(6.212,2.330)%
  --(6.212,2.329)--(6.213,2.329)--(6.214,2.329)--(6.215,2.329)--(6.215,2.328)--(6.216,2.328)%
  --(6.217,2.328)--(6.218,2.328)--(6.219,2.328)--(6.219,2.327)--(6.220,2.327)--(6.221,2.327)%
  --(6.222,2.327)--(6.223,2.327)--(6.223,2.326)--(6.224,2.326)--(6.225,2.326)--(6.226,2.326)%
  --(6.227,2.326)--(6.227,2.325)--(6.228,2.325)--(6.229,2.325)--(6.230,2.325)--(6.230,2.324)%
  --(6.231,2.324)--(6.232,2.324)--(6.233,2.324)--(6.234,2.324)--(6.234,2.323)--(6.235,2.323)%
  --(6.236,2.323)--(6.237,2.323)--(6.238,2.323)--(6.238,2.322)--(6.239,2.322)--(6.240,2.322)%
  --(6.241,2.322)--(6.242,2.322)--(6.242,2.321)--(6.243,2.321)--(6.244,2.321)--(6.245,2.321)%
  --(6.246,2.321)--(6.246,2.320)--(6.247,2.320)--(6.248,2.320)--(6.249,2.320)--(6.250,2.319)%
  --(6.251,2.319)--(6.252,2.319)--(6.253,2.319)--(6.253,2.318)--(6.254,2.318)--(6.255,2.318)%
  --(6.256,2.318)--(6.257,2.318)--(6.257,2.317)--(6.258,2.317)--(6.259,2.317)--(6.260,2.317)%
  --(6.261,2.317)--(6.261,2.316)--(6.262,2.316)--(6.263,2.316)--(6.264,2.316)--(6.265,2.316)%
  --(6.265,2.315)--(6.266,2.315)--(6.267,2.315)--(6.268,2.315)--(6.269,2.315)--(6.269,2.314)%
  --(6.270,2.314)--(6.271,2.314)--(6.272,2.314)--(6.273,2.314)--(6.273,2.313)--(6.274,2.313)%
  --(6.275,2.313)--(6.276,2.313)--(6.276,2.312)--(6.277,2.312)--(6.278,2.312)--(6.279,2.312)%
  --(6.280,2.312)--(6.280,2.311)--(6.281,2.311)--(6.282,2.311)--(6.283,2.311)--(6.284,2.311)%
  --(6.284,2.310)--(6.285,2.310)--(6.286,2.310)--(6.287,2.310)--(6.288,2.310)--(6.288,2.309)%
  --(6.289,2.309)--(6.290,2.309)--(6.291,2.309)--(6.292,2.309)--(6.292,2.308)--(6.293,2.308)%
  --(6.294,2.308)--(6.295,2.308)--(6.296,2.308)--(6.296,2.307)--(6.297,2.307)--(6.298,2.307)%
  --(6.299,2.307)--(6.300,2.307)--(6.300,2.306)--(6.301,2.306)--(6.302,2.306)--(6.303,2.306)%
  --(6.304,2.306)--(6.304,2.305)--(6.305,2.305)--(6.306,2.305)--(6.307,2.305)--(6.308,2.305)%
  --(6.308,2.304)--(6.309,2.304)--(6.310,2.304)--(6.311,2.304)--(6.311,2.303)--(6.312,2.303)%
  --(6.313,2.303)--(6.314,2.303)--(6.315,2.303)--(6.315,2.302)--(6.316,2.302)--(6.317,2.302)%
  --(6.318,2.302)--(6.319,2.302)--(6.319,2.301)--(6.320,2.301)--(6.321,2.301)--(6.322,2.301)%
  --(6.323,2.301)--(6.323,2.300)--(6.324,2.300)--(6.325,2.300)--(6.326,2.300)--(6.327,2.300)%
  --(6.327,2.299)--(6.328,2.299)--(6.329,2.299)--(6.330,2.299)--(6.331,2.299)--(6.331,2.298)%
  --(6.332,2.298)--(6.333,2.298)--(6.334,2.298)--(6.335,2.298)--(6.335,2.297)--(6.336,2.297)%
  --(6.337,2.297)--(6.338,2.297)--(6.339,2.297)--(6.339,2.296)--(6.340,2.296)--(6.341,2.296)%
  --(6.342,2.296)--(6.343,2.296)--(6.343,2.295)--(6.344,2.295)--(6.345,2.295)--(6.346,2.295)%
  --(6.347,2.295)--(6.347,2.294)--(6.348,2.294)--(6.349,2.294)--(6.350,2.294)--(6.351,2.294)%
  --(6.351,2.293)--(6.352,2.293)--(6.353,2.293)--(6.354,2.293)--(6.355,2.293)--(6.355,2.292)%
  --(6.356,2.292)--(6.357,2.292)--(6.358,2.292)--(6.359,2.292)--(6.359,2.291)--(6.360,2.291)%
  --(6.361,2.291)--(6.362,2.291)--(6.363,2.291)--(6.363,2.290)--(6.364,2.290)--(6.365,2.290)%
  --(6.366,2.290)--(6.367,2.290)--(6.367,2.289)--(6.368,2.289)--(6.369,2.289)--(6.370,2.289)%
  --(6.371,2.289)--(6.371,2.288)--(6.372,2.288)--(6.373,2.288)--(6.374,2.288)--(6.375,2.288)%
  --(6.375,2.287)--(6.376,2.287)--(6.377,2.287)--(6.378,2.287)--(6.379,2.287)--(6.379,2.286)%
  --(6.380,2.286)--(6.381,2.286)--(6.382,2.286)--(6.383,2.286)--(6.383,2.285)--(6.384,2.285)%
  --(6.385,2.285)--(6.386,2.285)--(6.387,2.285)--(6.387,2.284)--(6.388,2.284)--(6.389,2.284)%
  --(6.390,2.284)--(6.391,2.284)--(6.391,2.283)--(6.392,2.283)--(6.393,2.283)--(6.394,2.283)%
  --(6.395,2.283)--(6.395,2.282)--(6.396,2.282)--(6.397,2.282)--(6.398,2.282)--(6.399,2.282)%
  --(6.399,2.281)--(6.400,2.281)--(6.401,2.281)--(6.402,2.281)--(6.403,2.281)--(6.403,2.280)%
  --(6.404,2.280)--(6.405,2.280)--(6.406,2.280)--(6.407,2.280)--(6.407,2.279)--(6.408,2.279)%
  --(6.409,2.279)--(6.410,2.279)--(6.411,2.279)--(6.411,2.278)--(6.412,2.278)--(6.413,2.278)%
  --(6.414,2.278)--(6.415,2.278)--(6.415,2.277)--(6.416,2.277)--(6.417,2.277)--(6.418,2.277)%
  --(6.419,2.277)--(6.419,2.276)--(6.420,2.276)--(6.421,2.276)--(6.422,2.276)--(6.423,2.276)%
  --(6.423,2.275)--(6.424,2.275)--(6.425,2.275)--(6.426,2.275)--(6.427,2.275)--(6.428,2.275)%
  --(6.428,2.274)--(6.429,2.274)--(6.430,2.274)--(6.431,2.274)--(6.432,2.274)--(6.432,2.273)%
  --(6.433,2.273)--(6.434,2.273)--(6.435,2.273)--(6.436,2.273)--(6.436,2.272)--(6.437,2.272)%
  --(6.438,2.272)--(6.439,2.272)--(6.440,2.272)--(6.440,2.271)--(6.441,2.271)--(6.442,2.271)%
  --(6.443,2.271)--(6.444,2.271)--(6.444,2.270)--(6.445,2.270)--(6.446,2.270)--(6.447,2.270)%
  --(6.448,2.270)--(6.448,2.269)--(6.449,2.269)--(6.450,2.269)--(6.451,2.269)--(6.452,2.269)%
  --(6.452,2.268)--(6.453,2.268)--(6.454,2.268)--(6.455,2.268)--(6.456,2.268)--(6.456,2.267)%
  --(6.457,2.267)--(6.458,2.267)--(6.459,2.267)--(6.460,2.267)--(6.460,2.266)--(6.461,2.266)%
  --(6.462,2.266)--(6.463,2.266)--(6.464,2.266)--(6.465,2.266)--(6.465,2.265)--(6.466,2.265)%
  --(6.467,2.265)--(6.468,2.265)--(6.469,2.265)--(6.469,2.264)--(6.470,2.264)--(6.471,2.264)%
  --(6.472,2.264)--(6.473,2.264)--(6.473,2.263)--(6.474,2.263)--(6.475,2.263)--(6.476,2.263)%
  --(6.477,2.263)--(6.477,2.262)--(6.478,2.262)--(6.479,2.262)--(6.480,2.262)--(6.481,2.262)%
  --(6.481,2.261)--(6.482,2.261)--(6.483,2.261)--(6.484,2.261)--(6.485,2.261)--(6.485,2.260)%
  --(6.486,2.260)--(6.487,2.260)--(6.488,2.260)--(6.489,2.260)--(6.490,2.260)--(6.490,2.259)%
  --(6.491,2.259)--(6.492,2.259)--(6.493,2.259)--(6.494,2.259)--(6.494,2.258)--(6.495,2.258)%
  --(6.496,2.258)--(6.497,2.258)--(6.498,2.258)--(6.498,2.257)--(6.499,2.257)--(6.500,2.257)%
  --(6.501,2.257)--(6.502,2.257)--(6.502,2.256)--(6.503,2.256)--(6.504,2.256)--(6.505,2.256)%
  --(6.506,2.256)--(6.506,2.255)--(6.507,2.255)--(6.508,2.255)--(6.509,2.255)--(6.510,2.255)%
  --(6.511,2.255)--(6.511,2.254)--(6.512,2.254)--(6.513,2.254)--(6.514,2.254)--(6.515,2.254)%
  --(6.515,2.253)--(6.516,2.253)--(6.517,2.253)--(6.518,2.253)--(6.519,2.253)--(6.519,2.252)%
  --(6.520,2.252)--(6.521,2.252)--(6.522,2.252)--(6.523,2.252)--(6.523,2.251)--(6.524,2.251)%
  --(6.525,2.251)--(6.526,2.251)--(6.527,2.251)--(6.528,2.251)--(6.528,2.250)--(6.529,2.250)%
  --(6.530,2.250)--(6.531,2.250)--(6.532,2.250)--(6.532,2.249)--(6.533,2.249)--(6.534,2.249)%
  --(6.535,2.249)--(6.536,2.249)--(6.536,2.248)--(6.537,2.248)--(6.538,2.248)--(6.539,2.248)%
  --(6.540,2.248)--(6.540,2.247)--(6.541,2.247)--(6.542,2.247)--(6.543,2.247)--(6.544,2.247)%
  --(6.545,2.247)--(6.545,2.246)--(6.546,2.246)--(6.547,2.246)--(6.548,2.246)--(6.549,2.246)%
  --(6.549,2.245)--(6.550,2.245)--(6.551,2.245)--(6.552,2.245)--(6.553,2.245)--(6.553,2.244)%
  --(6.554,2.244)--(6.555,2.244)--(6.556,2.244)--(6.557,2.244)--(6.557,2.243)--(6.558,2.243)%
  --(6.559,2.243)--(6.560,2.243)--(6.561,2.243)--(6.562,2.243)--(6.562,2.242)--(6.563,2.242)%
  --(6.564,2.242)--(6.565,2.242)--(6.566,2.242)--(6.566,2.241)--(6.567,2.241)--(6.568,2.241)%
  --(6.569,2.241)--(6.570,2.241)--(6.570,2.240)--(6.571,2.240)--(6.572,2.240)--(6.573,2.240)%
  --(6.574,2.240)--(6.575,2.240)--(6.575,2.239)--(6.576,2.239)--(6.577,2.239)--(6.578,2.239)%
  --(6.579,2.239)--(6.579,2.238)--(6.580,2.238)--(6.581,2.238)--(6.582,2.238)--(6.583,2.238)%
  --(6.583,2.237)--(6.584,2.237)--(6.585,2.237)--(6.586,2.237)--(6.587,2.237)--(6.588,2.237)%
  --(6.588,2.236)--(6.589,2.236)--(6.590,2.236)--(6.591,2.236)--(6.592,2.236)--(6.592,2.235)%
  --(6.593,2.235)--(6.594,2.235)--(6.595,2.235)--(6.596,2.235)--(6.596,2.234)--(6.597,2.234)%
  --(6.598,2.234)--(6.599,2.234)--(6.600,2.234)--(6.601,2.234)--(6.601,2.233)--(6.602,2.233)%
  --(6.603,2.233)--(6.604,2.233)--(6.605,2.233)--(6.605,2.232)--(6.606,2.232)--(6.607,2.232)%
  --(6.608,2.232)--(6.609,2.232)--(6.609,2.231)--(6.610,2.231)--(6.611,2.231)--(6.612,2.231)%
  --(6.613,2.231)--(6.614,2.231)--(6.614,2.230)--(6.615,2.230)--(6.616,2.230)--(6.617,2.230)%
  --(6.618,2.230)--(6.618,2.229)--(6.619,2.229)--(6.620,2.229)--(6.621,2.229)--(6.622,2.229)%
  --(6.623,2.229)--(6.623,2.228)--(6.624,2.228)--(6.625,2.228)--(6.626,2.228)--(6.627,2.228)%
  --(6.627,2.227)--(6.628,2.227)--(6.629,2.227)--(6.630,2.227)--(6.631,2.227)--(6.631,2.226)%
  --(6.632,2.226)--(6.633,2.226)--(6.634,2.226)--(6.635,2.226)--(6.636,2.226)--(6.636,2.225)%
  --(6.637,2.225)--(6.638,2.225)--(6.639,2.225)--(6.640,2.225)--(6.640,2.224)--(6.641,2.224)%
  --(6.642,2.224)--(6.643,2.224)--(6.644,2.224)--(6.645,2.224)--(6.645,2.223)--(6.646,2.223)%
  --(6.647,2.223)--(6.648,2.223)--(6.649,2.223)--(6.649,2.222)--(6.650,2.222)--(6.651,2.222)%
  --(6.652,2.222)--(6.653,2.222)--(6.654,2.222)--(6.654,2.221)--(6.655,2.221)--(6.656,2.221)%
  --(6.657,2.221)--(6.658,2.221)--(6.658,2.220)--(6.659,2.220)--(6.660,2.220)--(6.661,2.220)%
  --(6.662,2.220)--(6.662,2.219)--(6.663,2.219)--(6.664,2.219)--(6.665,2.219)--(6.666,2.219)%
  --(6.667,2.219)--(6.667,2.218)--(6.668,2.218)--(6.669,2.218)--(6.670,2.218)--(6.671,2.218)%
  --(6.671,2.217)--(6.672,2.217)--(6.673,2.217)--(6.674,2.217)--(6.675,2.217)--(6.676,2.217)%
  --(6.676,2.216)--(6.677,2.216)--(6.678,2.216)--(6.679,2.216)--(6.680,2.216)--(6.680,2.215)%
  --(6.681,2.215)--(6.682,2.215)--(6.683,2.215)--(6.684,2.215)--(6.685,2.215)--(6.685,2.214)%
  --(6.686,2.214)--(6.687,2.214)--(6.688,2.214)--(6.689,2.214)--(6.689,2.213)--(6.690,2.213)%
  --(6.691,2.213)--(6.692,2.213)--(6.693,2.213)--(6.694,2.213)--(6.694,2.212)--(6.695,2.212)%
  --(6.696,2.212)--(6.697,2.212)--(6.698,2.212)--(6.698,2.211)--(6.699,2.211)--(6.700,2.211)%
  --(6.701,2.211)--(6.702,2.211)--(6.703,2.211)--(6.703,2.210)--(6.704,2.210)--(6.705,2.210)%
  --(6.706,2.210)--(6.707,2.210)--(6.707,2.209)--(6.708,2.209)--(6.709,2.209)--(6.710,2.209)%
  --(6.711,2.209)--(6.712,2.209)--(6.712,2.208)--(6.713,2.208)--(6.714,2.208)--(6.715,2.208)%
  --(6.716,2.208)--(6.717,2.207)--(6.718,2.207)--(6.719,2.207)--(6.720,2.207)--(6.721,2.207)%
  --(6.721,2.206)--(6.722,2.206)--(6.723,2.206)--(6.724,2.206)--(6.725,2.206)--(6.726,2.206)%
  --(6.726,2.205)--(6.727,2.205)--(6.728,2.205)--(6.729,2.205)--(6.730,2.205)--(6.730,2.204)%
  --(6.731,2.204)--(6.732,2.204)--(6.733,2.204)--(6.734,2.204)--(6.735,2.204)--(6.735,2.203)%
  --(6.736,2.203)--(6.737,2.203)--(6.738,2.203)--(6.739,2.203)--(6.739,2.202)--(6.740,2.202)%
  --(6.741,2.202)--(6.742,2.202)--(6.743,2.202)--(6.744,2.202)--(6.744,2.201)--(6.745,2.201)%
  --(6.746,2.201)--(6.747,2.201)--(6.748,2.201)--(6.749,2.200)--(6.750,2.200)--(6.751,2.200)%
  --(6.752,2.200)--(6.753,2.200)--(6.753,2.199)--(6.754,2.199)--(6.755,2.199)--(6.756,2.199)%
  --(6.757,2.199)--(6.758,2.199)--(6.758,2.198)--(6.759,2.198)--(6.760,2.198)--(6.761,2.198)%
  --(6.762,2.198)--(6.762,2.197)--(6.763,2.197)--(6.764,2.197)--(6.765,2.197)--(6.766,2.197)%
  --(6.767,2.197)--(6.767,2.196)--(6.768,2.196)--(6.769,2.196)--(6.770,2.196)--(6.771,2.196)%
  --(6.772,2.196)--(6.772,2.195)--(6.773,2.195)--(6.774,2.195)--(6.775,2.195)--(6.776,2.195)%
  --(6.776,2.194)--(6.777,2.194)--(6.778,2.194)--(6.779,2.194)--(6.780,2.194)--(6.781,2.194)%
  --(6.781,2.193)--(6.782,2.193)--(6.783,2.193)--(6.784,2.193)--(6.785,2.193)--(6.786,2.193)%
  --(6.786,2.192)--(6.787,2.192)--(6.788,2.192)--(6.789,2.192)--(6.790,2.192)--(6.790,2.191)%
  --(6.791,2.191)--(6.792,2.191)--(6.793,2.191)--(6.794,2.191)--(6.795,2.191)--(6.795,2.190)%
  --(6.796,2.190)--(6.797,2.190)--(6.798,2.190)--(6.799,2.190)--(6.800,2.190)--(6.800,2.189)%
  --(6.801,2.189)--(6.802,2.189)--(6.803,2.189)--(6.804,2.189)--(6.804,2.188)--(6.805,2.188)%
  --(6.806,2.188)--(6.807,2.188)--(6.808,2.188)--(6.809,2.188)--(6.809,2.187)--(6.810,2.187)%
  --(6.811,2.187)--(6.812,2.187)--(6.813,2.187)--(6.814,2.187)--(6.814,2.186)--(6.815,2.186)%
  --(6.816,2.186)--(6.817,2.186)--(6.818,2.186)--(6.818,2.185)--(6.819,2.185)--(6.820,2.185)%
  --(6.821,2.185)--(6.822,2.185)--(6.823,2.185)--(6.823,2.184)--(6.824,2.184)--(6.825,2.184)%
  --(6.826,2.184)--(6.827,2.184)--(6.828,2.184)--(6.828,2.183)--(6.829,2.183)--(6.830,2.183)%
  --(6.831,2.183)--(6.832,2.183)--(6.832,2.182)--(6.833,2.182)--(6.834,2.182)--(6.835,2.182)%
  --(6.836,2.182)--(6.837,2.182)--(6.837,2.181)--(6.838,2.181)--(6.839,2.181)--(6.840,2.181)%
  --(6.841,2.181)--(6.842,2.181)--(6.842,2.180)--(6.843,2.180)--(6.844,2.180)--(6.845,2.180)%
  --(6.846,2.180)--(6.847,2.180)--(6.847,2.179)--(6.848,2.179)--(6.849,2.179)--(6.850,2.179)%
  --(6.851,2.179)--(6.851,2.178)--(6.852,2.178)--(6.853,2.178)--(6.854,2.178)--(6.855,2.178)%
  --(6.856,2.178)--(6.856,2.177)--(6.857,2.177)--(6.858,2.177)--(6.859,2.177)--(6.860,2.177)%
  --(6.861,2.177)--(6.861,2.176)--(6.862,2.176)--(6.863,2.176)--(6.864,2.176)--(6.865,2.176)%
  --(6.866,2.176)--(6.866,2.175)--(6.867,2.175)--(6.868,2.175)--(6.869,2.175)--(6.870,2.175)%
  --(6.871,2.175)--(6.871,2.174)--(6.872,2.174)--(6.873,2.174)--(6.874,2.174)--(6.875,2.174)%
  --(6.875,2.173)--(6.876,2.173)--(6.877,2.173)--(6.878,2.173)--(6.879,2.173)--(6.880,2.173)%
  --(6.880,2.172)--(6.881,2.172)--(6.882,2.172)--(6.883,2.172)--(6.884,2.172)--(6.885,2.172)%
  --(6.885,2.171)--(6.886,2.171)--(6.887,2.171)--(6.888,2.171)--(6.889,2.171)--(6.890,2.171)%
  --(6.890,2.170)--(6.891,2.170)--(6.892,2.170)--(6.893,2.170)--(6.894,2.170)--(6.895,2.170)%
  --(6.895,2.169)--(6.896,2.169)--(6.897,2.169)--(6.898,2.169)--(6.899,2.169)--(6.900,2.169)%
  --(6.900,2.168)--(6.901,2.168)--(6.902,2.168)--(6.903,2.168)--(6.904,2.168)--(6.904,2.167)%
  --(6.905,2.167)--(6.906,2.167)--(6.907,2.167)--(6.908,2.167)--(6.909,2.167)--(6.909,2.166)%
  --(6.910,2.166)--(6.911,2.166)--(6.912,2.166)--(6.913,2.166)--(6.914,2.166)--(6.914,2.165)%
  --(6.915,2.165)--(6.916,2.165)--(6.917,2.165)--(6.918,2.165)--(6.919,2.165)--(6.919,2.164)%
  --(6.920,2.164)--(6.921,2.164)--(6.922,2.164)--(6.923,2.164)--(6.924,2.164)--(6.924,2.163)%
  --(6.925,2.163)--(6.926,2.163)--(6.927,2.163)--(6.928,2.163)--(6.929,2.163)--(6.929,2.162)%
  --(6.930,2.162)--(6.931,2.162)--(6.932,2.162)--(6.933,2.162)--(6.934,2.162)--(6.934,2.161)%
  --(6.935,2.161)--(6.936,2.161)--(6.937,2.161)--(6.938,2.161)--(6.939,2.161)--(6.939,2.160)%
  --(6.940,2.160)--(6.941,2.160)--(6.942,2.160)--(6.943,2.160)--(6.943,2.159)--(6.944,2.159)%
  --(6.945,2.159)--(6.946,2.159)--(6.947,2.159)--(6.948,2.159)--(6.948,2.158)--(6.949,2.158)%
  --(6.950,2.158)--(6.951,2.158)--(6.952,2.158)--(6.953,2.158)--(6.953,2.157)--(6.954,2.157)%
  --(6.955,2.157)--(6.956,2.157)--(6.957,2.157)--(6.958,2.157)--(6.958,2.156)--(6.959,2.156)%
  --(6.960,2.156)--(6.961,2.156)--(6.962,2.156)--(6.963,2.156)--(6.963,2.155)--(6.964,2.155)%
  --(6.965,2.155)--(6.966,2.155)--(6.967,2.155)--(6.968,2.155)--(6.968,2.154)--(6.969,2.154)%
  --(6.970,2.154)--(6.971,2.154)--(6.972,2.154)--(6.973,2.154)--(6.973,2.153)--(6.974,2.153)%
  --(6.975,2.153)--(6.976,2.153)--(6.977,2.153)--(6.978,2.153)--(6.978,2.152)--(6.979,2.152)%
  --(6.980,2.152)--(6.981,2.152)--(6.982,2.152)--(6.983,2.152)--(6.983,2.151)--(6.984,2.151)%
  --(6.985,2.151)--(6.986,2.151)--(6.987,2.151)--(6.988,2.151)--(6.988,2.150)--(6.989,2.150)%
  --(6.990,2.150)--(6.991,2.150)--(6.992,2.150)--(6.993,2.150)--(6.993,2.149)--(6.994,2.149)%
  --(6.995,2.149)--(6.996,2.149)--(6.997,2.149)--(6.998,2.149)--(6.998,2.148)--(6.999,2.148)%
  --(7.000,2.148)--(7.001,2.148)--(7.002,2.148)--(7.003,2.148)--(7.003,2.147)--(7.004,2.147)%
  --(7.005,2.147)--(7.006,2.147)--(7.007,2.147)--(7.008,2.147)--(7.008,2.146)--(7.009,2.146)%
  --(7.010,2.146)--(7.011,2.146)--(7.012,2.146)--(7.013,2.146)--(7.013,2.145)--(7.014,2.145)%
  --(7.015,2.145)--(7.016,2.145)--(7.017,2.145)--(7.018,2.145)--(7.018,2.144)--(7.019,2.144)%
  --(7.020,2.144)--(7.021,2.144)--(7.022,2.144)--(7.023,2.144)--(7.023,2.143)--(7.024,2.143)%
  --(7.025,2.143)--(7.026,2.143)--(7.027,2.143)--(7.028,2.143)--(7.028,2.142)--(7.029,2.142)%
  --(7.030,2.142)--(7.031,2.142)--(7.032,2.142)--(7.033,2.142)--(7.033,2.141)--(7.034,2.141)%
  --(7.035,2.141)--(7.036,2.141)--(7.037,2.141)--(7.038,2.141)--(7.038,2.140)--(7.039,2.140)%
  --(7.040,2.140)--(7.041,2.140)--(7.042,2.140)--(7.043,2.140)--(7.044,2.140)--(7.044,2.139)%
  --(7.045,2.139)--(7.046,2.139)--(7.047,2.139)--(7.048,2.139)--(7.049,2.139)--(7.049,2.138)%
  --(7.050,2.138)--(7.051,2.138)--(7.052,2.138)--(7.053,2.138)--(7.054,2.138)--(7.054,2.137)%
  --(7.055,2.137)--(7.056,2.137)--(7.057,2.137)--(7.058,2.137)--(7.059,2.137)--(7.059,2.136)%
  --(7.060,2.136)--(7.061,2.136)--(7.062,2.136)--(7.063,2.136)--(7.064,2.136)--(7.064,2.135)%
  --(7.065,2.135)--(7.066,2.135)--(7.067,2.135)--(7.068,2.135)--(7.069,2.135)--(7.069,2.134)%
  --(7.070,2.134)--(7.071,2.134)--(7.072,2.134)--(7.073,2.134)--(7.074,2.134)--(7.074,2.133)%
  --(7.075,2.133)--(7.076,2.133)--(7.077,2.133)--(7.078,2.133)--(7.079,2.133)--(7.079,2.132)%
  --(7.080,2.132)--(7.081,2.132)--(7.082,2.132)--(7.083,2.132)--(7.084,2.132)--(7.085,2.131)%
  --(7.086,2.131)--(7.087,2.131)--(7.088,2.131)--(7.089,2.131)--(7.090,2.131)--(7.090,2.130)%
  --(7.091,2.130)--(7.092,2.130)--(7.093,2.130)--(7.094,2.130)--(7.095,2.130)--(7.095,2.129)%
  --(7.096,2.129)--(7.097,2.129)--(7.098,2.129)--(7.099,2.129)--(7.100,2.129)--(7.100,2.128)%
  --(7.101,2.128)--(7.102,2.128)--(7.103,2.128)--(7.104,2.128)--(7.105,2.128)--(7.105,2.127)%
  --(7.106,2.127)--(7.107,2.127)--(7.108,2.127)--(7.109,2.127)--(7.110,2.127)--(7.110,2.126)%
  --(7.111,2.126)--(7.112,2.126)--(7.113,2.126)--(7.114,2.126)--(7.115,2.126)--(7.116,2.126)%
  --(7.116,2.125)--(7.117,2.125)--(7.118,2.125)--(7.119,2.125)--(7.120,2.125)--(7.121,2.125)%
  --(7.121,2.124)--(7.122,2.124)--(7.123,2.124)--(7.124,2.124)--(7.125,2.124)--(7.126,2.124)%
  --(7.126,2.123)--(7.127,2.123)--(7.128,2.123)--(7.129,2.123)--(7.130,2.123)--(7.131,2.123)%
  --(7.131,2.122)--(7.132,2.122)--(7.133,2.122)--(7.134,2.122)--(7.135,2.122)--(7.136,2.122)%
  --(7.137,2.122)--(7.137,2.121)--(7.138,2.121)--(7.139,2.121)--(7.140,2.121)--(7.141,2.121)%
  --(7.142,2.121)--(7.142,2.120)--(7.143,2.120)--(7.144,2.120)--(7.145,2.120)--(7.146,2.120)%
  --(7.147,2.120)--(7.147,2.119)--(7.148,2.119)--(7.149,2.119)--(7.150,2.119)--(7.151,2.119)%
  --(7.152,2.119)--(7.152,2.118)--(7.153,2.118)--(7.154,2.118)--(7.155,2.118)--(7.156,2.118)%
  --(7.157,2.118)--(7.158,2.118)--(7.158,2.117)--(7.159,2.117)--(7.160,2.117)--(7.161,2.117)%
  --(7.162,2.117)--(7.163,2.117)--(7.163,2.116)--(7.164,2.116)--(7.165,2.116)--(7.166,2.116)%
  --(7.167,2.116)--(7.168,2.116)--(7.168,2.115)--(7.169,2.115)--(7.170,2.115)--(7.171,2.115)%
  --(7.172,2.115)--(7.173,2.115)--(7.174,2.114)--(7.175,2.114)--(7.176,2.114)--(7.177,2.114)%
  --(7.178,2.114)--(7.179,2.114)--(7.179,2.113)--(7.180,2.113)--(7.181,2.113)--(7.182,2.113)%
  --(7.183,2.113)--(7.184,2.113)--(7.184,2.112)--(7.185,2.112)--(7.186,2.112)--(7.187,2.112)%
  --(7.188,2.112)--(7.189,2.112)--(7.189,2.111)--(7.190,2.111)--(7.191,2.111)--(7.192,2.111)%
  --(7.193,2.111)--(7.194,2.111)--(7.195,2.111)--(7.195,2.110)--(7.196,2.110)--(7.197,2.110)%
  --(7.198,2.110)--(7.199,2.110)--(7.200,2.110)--(7.200,2.109)--(7.201,2.109)--(7.202,2.109)%
  --(7.203,2.109)--(7.204,2.109)--(7.205,2.109)--(7.206,2.109)--(7.206,2.108)--(7.207,2.108)%
  --(7.208,2.108)--(7.209,2.108)--(7.210,2.108)--(7.211,2.108)--(7.211,2.107)--(7.212,2.107)%
  --(7.213,2.107)--(7.214,2.107)--(7.215,2.107)--(7.216,2.107)--(7.216,2.106)--(7.217,2.106)%
  --(7.218,2.106)--(7.219,2.106)--(7.220,2.106)--(7.221,2.106)--(7.222,2.106)--(7.222,2.105)%
  --(7.223,2.105)--(7.224,2.105)--(7.225,2.105)--(7.226,2.105)--(7.227,2.105)--(7.227,2.104)%
  --(7.228,2.104)--(7.229,2.104)--(7.230,2.104)--(7.231,2.104)--(7.232,2.104)--(7.233,2.104)%
  --(7.233,2.103)--(7.234,2.103)--(7.235,2.103)--(7.236,2.103)--(7.237,2.103)--(7.238,2.103)%
  --(7.238,2.102)--(7.239,2.102)--(7.240,2.102)--(7.241,2.102)--(7.242,2.102)--(7.243,2.102)%
  --(7.243,2.101)--(7.244,2.101)--(7.245,2.101)--(7.246,2.101)--(7.247,2.101)--(7.248,2.101)%
  --(7.249,2.101)--(7.249,2.100)--(7.250,2.100)--(7.251,2.100)--(7.252,2.100)--(7.253,2.100)%
  --(7.254,2.100)--(7.254,2.099)--(7.255,2.099)--(7.256,2.099)--(7.257,2.099)--(7.258,2.099)%
  --(7.259,2.099)--(7.260,2.099)--(7.260,2.098)--(7.261,2.098)--(7.262,2.098)--(7.263,2.098)%
  --(7.264,2.098)--(7.265,2.098)--(7.265,2.097)--(7.266,2.097)--(7.267,2.097)--(7.268,2.097)%
  --(7.269,2.097)--(7.270,2.097)--(7.271,2.097)--(7.271,2.096)--(7.272,2.096)--(7.273,2.096)%
  --(7.274,2.096)--(7.275,2.096)--(7.276,2.096)--(7.276,2.095)--(7.277,2.095)--(7.278,2.095)%
  --(7.279,2.095)--(7.280,2.095)--(7.281,2.095)--(7.282,2.095)--(7.282,2.094)--(7.283,2.094)%
  --(7.284,2.094)--(7.285,2.094)--(7.286,2.094)--(7.287,2.094)--(7.287,2.093)--(7.288,2.093)%
  --(7.289,2.093)--(7.290,2.093)--(7.291,2.093)--(7.292,2.093)--(7.293,2.093)--(7.293,2.092)%
  --(7.294,2.092)--(7.295,2.092)--(7.296,2.092)--(7.297,2.092)--(7.298,2.092)--(7.298,2.091)%
  --(7.299,2.091)--(7.300,2.091)--(7.301,2.091)--(7.302,2.091)--(7.303,2.091)--(7.304,2.091)%
  --(7.304,2.090)--(7.305,2.090)--(7.306,2.090)--(7.307,2.090)--(7.308,2.090)--(7.309,2.090)%
  --(7.309,2.089)--(7.310,2.089)--(7.311,2.089)--(7.312,2.089)--(7.313,2.089)--(7.314,2.089)%
  --(7.315,2.089)--(7.315,2.088)--(7.316,2.088)--(7.317,2.088)--(7.318,2.088)--(7.319,2.088)%
  --(7.320,2.088)--(7.320,2.087)--(7.321,2.087)--(7.322,2.087)--(7.323,2.087)--(7.324,2.087)%
  --(7.325,2.087)--(7.326,2.087)--(7.326,2.086)--(7.327,2.086)--(7.328,2.086)--(7.329,2.086)%
  --(7.330,2.086)--(7.331,2.086)--(7.332,2.086)--(7.332,2.085)--(7.333,2.085)--(7.334,2.085)%
  --(7.335,2.085)--(7.336,2.085)--(7.337,2.085)--(7.337,2.084)--(7.338,2.084)--(7.339,2.084)%
  --(7.340,2.084)--(7.341,2.084)--(7.342,2.084)--(7.343,2.084)--(7.343,2.083)--(7.344,2.083)%
  --(7.345,2.083)--(7.346,2.083)--(7.347,2.083)--(7.348,2.083)--(7.348,2.082)--(7.349,2.082)%
  --(7.350,2.082)--(7.351,2.082)--(7.352,2.082)--(7.353,2.082)--(7.354,2.082)--(7.354,2.081)%
  --(7.355,2.081)--(7.356,2.081)--(7.357,2.081)--(7.358,2.081)--(7.359,2.081)--(7.360,2.081)%
  --(7.360,2.080)--(7.361,2.080)--(7.362,2.080)--(7.363,2.080)--(7.364,2.080)--(7.365,2.080)%
  --(7.365,2.079)--(7.366,2.079)--(7.367,2.079)--(7.368,2.079)--(7.369,2.079)--(7.370,2.079)%
  --(7.371,2.079)--(7.371,2.078)--(7.372,2.078)--(7.373,2.078)--(7.374,2.078)--(7.375,2.078)%
  --(7.376,2.078)--(7.377,2.078)--(7.377,2.077)--(7.378,2.077)--(7.379,2.077)--(7.380,2.077)%
  --(7.381,2.077)--(7.382,2.077)--(7.382,2.076)--(7.383,2.076)--(7.384,2.076)--(7.385,2.076)%
  --(7.386,2.076)--(7.387,2.076)--(7.388,2.076)--(7.388,2.075)--(7.389,2.075)--(7.390,2.075)%
  --(7.391,2.075)--(7.392,2.075)--(7.393,2.075)--(7.394,2.075)--(7.394,2.074)--(7.395,2.074)%
  --(7.396,2.074)--(7.397,2.074)--(7.398,2.074)--(7.399,2.074)--(7.399,2.073)--(7.400,2.073)%
  --(7.401,2.073)--(7.402,2.073)--(7.403,2.073)--(7.404,2.073)--(7.405,2.073)--(7.405,2.072)%
  --(7.406,2.072)--(7.407,2.072)--(7.408,2.072)--(7.409,2.072)--(7.410,2.072)--(7.411,2.072)%
  --(7.411,2.071)--(7.412,2.071)--(7.413,2.071)--(7.414,2.071)--(7.415,2.071)--(7.416,2.071)%
  --(7.417,2.071)--(7.417,2.070)--(7.418,2.070)--(7.419,2.070)--(7.420,2.070)--(7.421,2.070)%
  --(7.422,2.070)--(7.422,2.069)--(7.423,2.069)--(7.424,2.069)--(7.425,2.069)--(7.426,2.069)%
  --(7.427,2.069)--(7.428,2.069)--(7.428,2.068)--(7.429,2.068)--(7.430,2.068)--(7.431,2.068)%
  --(7.432,2.068)--(7.433,2.068)--(7.434,2.068)--(7.434,2.067)--(7.435,2.067)--(7.436,2.067)%
  --(7.437,2.067)--(7.438,2.067)--(7.439,2.067)--(7.440,2.067)--(7.440,2.066)--(7.441,2.066)%
  --(7.442,2.066)--(7.443,2.066)--(7.444,2.066)--(7.445,2.066)--(7.445,2.065)--(7.446,2.065)%
  --(7.447,2.065)--(7.448,2.065)--(7.449,2.065)--(7.450,2.065)--(7.451,2.065)--(7.451,2.064)%
  --(7.452,2.064)--(7.453,2.064)--(7.454,2.064)--(7.455,2.064)--(7.456,2.064)--(7.457,2.064)%
  --(7.457,2.063)--(7.458,2.063)--(7.459,2.063)--(7.460,2.063)--(7.461,2.063)--(7.462,2.063)%
  --(7.463,2.063)--(7.463,2.062)--(7.464,2.062)--(7.465,2.062)--(7.466,2.062)--(7.467,2.062)%
  --(7.468,2.062)--(7.469,2.062)--(7.469,2.061)--(7.470,2.061)--(7.471,2.061)--(7.472,2.061)%
  --(7.473,2.061)--(7.474,2.061)--(7.475,2.061)--(7.475,2.060)--(7.476,2.060)--(7.477,2.060)%
  --(7.478,2.060)--(7.479,2.060)--(7.480,2.060)--(7.480,2.059)--(7.481,2.059)--(7.482,2.059)%
  --(7.483,2.059)--(7.484,2.059)--(7.485,2.059)--(7.486,2.059)--(7.486,2.058)--(7.487,2.058)%
  --(7.488,2.058)--(7.489,2.058)--(7.490,2.058)--(7.491,2.058)--(7.492,2.058)--(7.492,2.057)%
  --(7.493,2.057)--(7.494,2.057)--(7.495,2.057)--(7.496,2.057)--(7.497,2.057)--(7.498,2.057)%
  --(7.498,2.056)--(7.499,2.056)--(7.500,2.056)--(7.501,2.056)--(7.502,2.056)--(7.503,2.056)%
  --(7.504,2.056)--(7.504,2.055)--(7.505,2.055)--(7.506,2.055)--(7.507,2.055)--(7.508,2.055)%
  --(7.509,2.055)--(7.510,2.055)--(7.510,2.054)--(7.511,2.054)--(7.512,2.054)--(7.513,2.054)%
  --(7.514,2.054)--(7.515,2.054)--(7.516,2.054)--(7.516,2.053)--(7.517,2.053)--(7.518,2.053)%
  --(7.519,2.053)--(7.520,2.053)--(7.521,2.053)--(7.522,2.053)--(7.522,2.052)--(7.523,2.052)%
  --(7.524,2.052)--(7.525,2.052)--(7.526,2.052)--(7.527,2.052)--(7.528,2.052)--(7.528,2.051)%
  --(7.529,2.051)--(7.530,2.051)--(7.531,2.051)--(7.532,2.051)--(7.533,2.051)--(7.534,2.051)%
  --(7.534,2.050)--(7.535,2.050)--(7.536,2.050)--(7.537,2.050)--(7.538,2.050)--(7.539,2.050)%
  --(7.540,2.050)--(7.540,2.049)--(7.541,2.049)--(7.542,2.049)--(7.543,2.049)--(7.544,2.049)%
  --(7.545,2.049)--(7.546,2.049)--(7.546,2.048)--(7.547,2.048)--(7.548,2.048)--(7.549,2.048)%
  --(7.550,2.048)--(7.551,2.048)--(7.552,2.048)--(7.552,2.047)--(7.553,2.047)--(7.554,2.047)%
  --(7.555,2.047)--(7.556,2.047)--(7.557,2.047)--(7.558,2.047)--(7.558,2.046)--(7.559,2.046)%
  --(7.560,2.046)--(7.561,2.046)--(7.562,2.046)--(7.563,2.046)--(7.564,2.046)--(7.564,2.045)%
  --(7.565,2.045)--(7.566,2.045)--(7.567,2.045)--(7.568,2.045)--(7.569,2.045)--(7.570,2.045)%
  --(7.570,2.044)--(7.571,2.044)--(7.572,2.044)--(7.573,2.044)--(7.574,2.044)--(7.575,2.044)%
  --(7.576,2.044)--(7.576,2.043)--(7.577,2.043)--(7.578,2.043)--(7.579,2.043)--(7.580,2.043)%
  --(7.581,2.043)--(7.582,2.043)--(7.582,2.042)--(7.583,2.042)--(7.584,2.042)--(7.585,2.042)%
  --(7.586,2.042)--(7.587,2.042)--(7.588,2.042)--(7.588,2.041)--(7.589,2.041)--(7.590,2.041)%
  --(7.591,2.041)--(7.592,2.041)--(7.593,2.041)--(7.594,2.041)--(7.594,2.040)--(7.595,2.040)%
  --(7.596,2.040)--(7.597,2.040)--(7.598,2.040)--(7.599,2.040)--(7.600,2.040)--(7.600,2.039)%
  --(7.601,2.039)--(7.602,2.039)--(7.603,2.039)--(7.604,2.039)--(7.605,2.039)--(7.606,2.039)%
  --(7.606,2.038)--(7.607,2.038)--(7.608,2.038)--(7.609,2.038)--(7.610,2.038)--(7.611,2.038)%
  --(7.612,2.038)--(7.612,2.037)--(7.613,2.037)--(7.614,2.037)--(7.615,2.037)--(7.616,2.037)%
  --(7.617,2.037)--(7.618,2.037)--(7.618,2.036)--(7.619,2.036)--(7.620,2.036)--(7.621,2.036)%
  --(7.622,2.036)--(7.623,2.036)--(7.624,2.036)--(7.624,2.035)--(7.625,2.035)--(7.626,2.035)%
  --(7.627,2.035)--(7.628,2.035)--(7.629,2.035)--(7.630,2.035)--(7.631,2.035)--(7.631,2.034)%
  --(7.632,2.034)--(7.633,2.034)--(7.634,2.034)--(7.635,2.034)--(7.636,2.034)--(7.637,2.034)%
  --(7.637,2.033)--(7.638,2.033)--(7.639,2.033)--(7.640,2.033)--(7.641,2.033)--(7.642,2.033)%
  --(7.643,2.033)--(7.643,2.032)--(7.644,2.032)--(7.645,2.032)--(7.646,2.032)--(7.647,2.032)%
  --(7.648,2.032)--(7.649,2.032)--(7.649,2.031)--(7.650,2.031)--(7.651,2.031)--(7.652,2.031)%
  --(7.653,2.031)--(7.654,2.031)--(7.655,2.031)--(7.655,2.030)--(7.656,2.030)--(7.657,2.030)%
  --(7.658,2.030)--(7.659,2.030)--(7.660,2.030)--(7.661,2.030)--(7.661,2.029)--(7.662,2.029)%
  --(7.663,2.029)--(7.664,2.029)--(7.665,2.029)--(7.666,2.029)--(7.667,2.029)--(7.668,2.029)%
  --(7.668,2.028)--(7.669,2.028)--(7.670,2.028)--(7.671,2.028)--(7.672,2.028)--(7.673,2.028)%
  --(7.674,2.028)--(7.674,2.027)--(7.675,2.027)--(7.676,2.027)--(7.677,2.027)--(7.678,2.027)%
  --(7.679,2.027)--(7.680,2.027)--(7.680,2.026)--(7.681,2.026)--(7.682,2.026)--(7.683,2.026)%
  --(7.684,2.026)--(7.685,2.026)--(7.686,2.026)--(7.686,2.025)--(7.687,2.025)--(7.688,2.025)%
  --(7.689,2.025)--(7.690,2.025)--(7.691,2.025)--(7.692,2.025)--(7.693,2.025)--(7.693,2.024)%
  --(7.694,2.024)--(7.695,2.024)--(7.696,2.024)--(7.697,2.024)--(7.698,2.024)--(7.699,2.024)%
  --(7.699,2.023)--(7.700,2.023)--(7.701,2.023)--(7.702,2.023)--(7.703,2.023)--(7.704,2.023)%
  --(7.705,2.023)--(7.705,2.022)--(7.706,2.022)--(7.707,2.022)--(7.708,2.022)--(7.709,2.022)%
  --(7.710,2.022)--(7.711,2.022)--(7.711,2.021)--(7.712,2.021)--(7.713,2.021)--(7.714,2.021)%
  --(7.715,2.021)--(7.716,2.021)--(7.717,2.021)--(7.718,2.021)--(7.718,2.020)--(7.719,2.020)%
  --(7.720,2.020)--(7.721,2.020)--(7.722,2.020)--(7.723,2.020)--(7.724,2.020)--(7.724,2.019)%
  --(7.725,2.019)--(7.726,2.019)--(7.727,2.019)--(7.728,2.019)--(7.729,2.019)--(7.730,2.019)%
  --(7.730,2.018)--(7.731,2.018)--(7.732,2.018)--(7.733,2.018)--(7.734,2.018)--(7.735,2.018)%
  --(7.736,2.018)--(7.737,2.018)--(7.737,2.017)--(7.738,2.017)--(7.739,2.017)--(7.740,2.017)%
  --(7.741,2.017)--(7.742,2.017)--(7.743,2.017)--(7.743,2.016)--(7.744,2.016)--(7.745,2.016)%
  --(7.746,2.016)--(7.747,2.016)--(7.748,2.016)--(7.749,2.016)--(7.749,2.015)--(7.750,2.015)%
  --(7.751,2.015)--(7.752,2.015)--(7.753,2.015)--(7.754,2.015)--(7.755,2.015)--(7.756,2.015)%
  --(7.756,2.014)--(7.757,2.014)--(7.758,2.014)--(7.759,2.014)--(7.760,2.014)--(7.761,2.014)%
  --(7.762,2.014)--(7.762,2.013)--(7.763,2.013)--(7.764,2.013)--(7.765,2.013)--(7.766,2.013)%
  --(7.767,2.013)--(7.768,2.013)--(7.769,2.013)--(7.769,2.012)--(7.770,2.012)--(7.771,2.012)%
  --(7.772,2.012)--(7.773,2.012)--(7.774,2.012)--(7.775,2.012)--(7.775,2.011)--(7.776,2.011)%
  --(7.777,2.011)--(7.778,2.011)--(7.779,2.011)--(7.780,2.011)--(7.781,2.011)--(7.781,2.010)%
  --(7.782,2.010)--(7.783,2.010)--(7.784,2.010)--(7.785,2.010)--(7.786,2.010)--(7.787,2.010)%
  --(7.788,2.010)--(7.788,2.009)--(7.789,2.009)--(7.790,2.009)--(7.791,2.009)--(7.792,2.009)%
  --(7.793,2.009)--(7.794,2.009)--(7.794,2.008)--(7.795,2.008)--(7.796,2.008)--(7.797,2.008)%
  --(7.798,2.008)--(7.799,2.008)--(7.800,2.008)--(7.801,2.008)--(7.801,2.007)--(7.802,2.007)%
  --(7.803,2.007)--(7.804,2.007)--(7.805,2.007)--(7.806,2.007)--(7.807,2.007)--(7.807,2.006)%
  --(7.808,2.006)--(7.809,2.006)--(7.810,2.006)--(7.811,2.006)--(7.812,2.006)--(7.813,2.006)%
  --(7.814,2.006)--(7.814,2.005)--(7.815,2.005)--(7.816,2.005)--(7.817,2.005)--(7.818,2.005)%
  --(7.819,2.005)--(7.820,2.005)--(7.820,2.004)--(7.821,2.004)--(7.822,2.004)--(7.823,2.004)%
  --(7.824,2.004)--(7.825,2.004)--(7.826,2.004)--(7.827,2.004)--(7.827,2.003)--(7.828,2.003)%
  --(7.829,2.003)--(7.830,2.003)--(7.831,2.003)--(7.832,2.003)--(7.833,2.003)--(7.833,2.002)%
  --(7.834,2.002)--(7.835,2.002)--(7.836,2.002)--(7.837,2.002)--(7.838,2.002)--(7.839,2.002)%
  --(7.840,2.002)--(7.840,2.001)--(7.841,2.001)--(7.842,2.001)--(7.843,2.001)--(7.844,2.001)%
  --(7.845,2.001)--(7.846,2.001)--(7.846,2.000)--(7.847,2.000)--(7.848,2.000)--(7.849,2.000)%
  --(7.850,2.000)--(7.851,2.000)--(7.852,2.000)--(7.853,2.000)--(7.853,1.999)--(7.854,1.999)%
  --(7.855,1.999)--(7.856,1.999)--(7.857,1.999)--(7.858,1.999)--(7.859,1.999)--(7.859,1.998)%
  --(7.860,1.998)--(7.861,1.998)--(7.862,1.998)--(7.863,1.998)--(7.864,1.998)--(7.865,1.998)%
  --(7.866,1.998)--(7.866,1.997)--(7.867,1.997)--(7.868,1.997)--(7.869,1.997)--(7.870,1.997)%
  --(7.871,1.997)--(7.872,1.997)--(7.873,1.997)--(7.873,1.996)--(7.874,1.996)--(7.875,1.996)%
  --(7.876,1.996)--(7.877,1.996)--(7.878,1.996)--(7.879,1.996)--(7.879,1.995)--(7.880,1.995)%
  --(7.881,1.995)--(7.882,1.995)--(7.883,1.995)--(7.884,1.995)--(7.885,1.995)--(7.886,1.995)%
  --(7.886,1.994)--(7.887,1.994)--(7.888,1.994)--(7.889,1.994)--(7.890,1.994)--(7.891,1.994)%
  --(7.892,1.994)--(7.893,1.994)--(7.893,1.993)--(7.894,1.993)--(7.895,1.993)--(7.896,1.993)%
  --(7.897,1.993)--(7.898,1.993)--(7.899,1.993)--(7.899,1.992)--(7.900,1.992)--(7.901,1.992)%
  --(7.902,1.992)--(7.903,1.992)--(7.904,1.992)--(7.905,1.992)--(7.906,1.992)--(7.906,1.991)%
  --(7.907,1.991)--(7.908,1.991)--(7.909,1.991)--(7.910,1.991)--(7.911,1.991)--(7.912,1.991)%
  --(7.913,1.991)--(7.913,1.990)--(7.914,1.990)--(7.915,1.990)--(7.916,1.990)--(7.917,1.990)%
  --(7.918,1.990)--(7.919,1.990)--(7.919,1.989)--(7.920,1.989)--(7.921,1.989)--(7.922,1.989)%
  --(7.923,1.989)--(7.924,1.989)--(7.925,1.989)--(7.926,1.989)--(7.926,1.988)--(7.927,1.988)%
  --(7.928,1.988)--(7.929,1.988)--(7.930,1.988)--(7.931,1.988)--(7.932,1.988)--(7.933,1.988)%
  --(7.933,1.987)--(7.934,1.987)--(7.935,1.987)--(7.936,1.987)--(7.937,1.987)--(7.938,1.987)%
  --(7.939,1.987)--(7.939,1.986)--(7.940,1.986)--(7.941,1.986)--(7.942,1.986)--(7.943,1.986)%
  --(7.944,1.986)--(7.945,1.986)--(7.946,1.986)--(7.946,1.985)--(7.947,1.985)--(7.948,1.985)%
  --(7.949,1.985)--(7.950,1.985)--(7.951,1.985)--(7.952,1.985)--(7.953,1.985)--(7.953,1.984)%
  --(7.954,1.984)--(7.955,1.984)--(7.956,1.984)--(7.957,1.984)--(7.958,1.984)--(7.959,1.984)%
  --(7.960,1.984)--(7.960,1.983)--(7.961,1.983)--(7.962,1.983)--(7.963,1.983)--(7.964,1.983)%
  --(7.965,1.983)--(7.966,1.983)--(7.966,1.982)--(7.967,1.982)--(7.968,1.982)--(7.969,1.982)%
  --(7.970,1.982)--(7.971,1.982)--(7.972,1.982)--(7.973,1.982)--(7.973,1.981)--(7.974,1.981)%
  --(7.975,1.981)--(7.976,1.981)--(7.977,1.981)--(7.978,1.981)--(7.979,1.981)--(7.980,1.981)%
  --(7.980,1.980)--(7.981,1.980)--(7.982,1.980)--(7.983,1.980)--(7.984,1.980)--(7.985,1.980)%
  --(7.986,1.980)--(7.987,1.980)--(7.987,1.979)--(7.988,1.979)--(7.989,1.979)--(7.990,1.979)%
  --(7.991,1.979)--(7.992,1.979)--(7.993,1.979)--(7.994,1.979)--(7.994,1.978)--(7.995,1.978)%
  --(7.996,1.978)--(7.997,1.978)--(7.998,1.978)--(7.999,1.978)--(8.000,1.978)--(8.001,1.978)%
  --(8.001,1.977)--(8.002,1.977)--(8.003,1.977)--(8.004,1.977)--(8.005,1.977)--(8.006,1.977)%
  --(8.007,1.977)--(8.007,1.976)--(8.008,1.976)--(8.009,1.976)--(8.010,1.976)--(8.011,1.976)%
  --(8.012,1.976)--(8.013,1.976)--(8.014,1.976)--(8.014,1.975)--(8.015,1.975)--(8.016,1.975)%
  --(8.017,1.975)--(8.018,1.975)--(8.019,1.975)--(8.020,1.975)--(8.021,1.975)--(8.021,1.974)%
  --(8.022,1.974)--(8.023,1.974)--(8.024,1.974)--(8.025,1.974)--(8.026,1.974)--(8.027,1.974)%
  --(8.028,1.974)--(8.028,1.973)--(8.029,1.973)--(8.030,1.973)--(8.031,1.973)--(8.032,1.973)%
  --(8.033,1.973)--(8.034,1.973)--(8.035,1.973)--(8.035,1.972)--(8.036,1.972)--(8.037,1.972)%
  --(8.038,1.972)--(8.039,1.972)--(8.040,1.972)--(8.041,1.972)--(8.042,1.972)--(8.042,1.971)%
  --(8.043,1.971)--(8.044,1.971)--(8.045,1.971)--(8.046,1.971)--(8.047,1.971)--(8.048,1.971)%
  --(8.049,1.971)--(8.049,1.970)--(8.050,1.970)--(8.051,1.970)--(8.052,1.970)--(8.053,1.970)%
  --(8.054,1.970)--(8.055,1.970)--(8.056,1.970)--(8.056,1.969)--(8.057,1.969)--(8.058,1.969)%
  --(8.059,1.969)--(8.060,1.969)--(8.061,1.969)--(8.062,1.969)--(8.063,1.969)--(8.063,1.968)%
  --(8.064,1.968)--(8.065,1.968)--(8.066,1.968)--(8.067,1.968)--(8.068,1.968)--(8.069,1.968)%
  --(8.070,1.968)--(8.070,1.967)--(8.071,1.967)--(8.072,1.967)--(8.073,1.967)--(8.074,1.967)%
  --(8.075,1.967)--(8.076,1.967)--(8.077,1.967)--(8.077,1.966)--(8.078,1.966)--(8.079,1.966)%
  --(8.080,1.966)--(8.081,1.966)--(8.082,1.966)--(8.083,1.966)--(8.084,1.966)--(8.084,1.965)%
  --(8.085,1.965)--(8.086,1.965)--(8.087,1.965)--(8.088,1.965)--(8.089,1.965)--(8.090,1.965)%
  --(8.091,1.965)--(8.091,1.964)--(8.092,1.964)--(8.093,1.964)--(8.094,1.964)--(8.095,1.964)%
  --(8.096,1.964)--(8.097,1.964)--(8.098,1.964)--(8.098,1.963)--(8.099,1.963)--(8.100,1.963)%
  --(8.101,1.963)--(8.102,1.963)--(8.103,1.963)--(8.104,1.963)--(8.105,1.963)--(8.105,1.962)%
  --(8.106,1.962)--(8.107,1.962)--(8.108,1.962)--(8.109,1.962)--(8.110,1.962)--(8.111,1.962)%
  --(8.112,1.962)--(8.112,1.961)--(8.113,1.961)--(8.114,1.961)--(8.115,1.961)--(8.116,1.961)%
  --(8.117,1.961)--(8.118,1.961)--(8.119,1.961)--(8.119,1.960)--(8.120,1.960)--(8.121,1.960)%
  --(8.122,1.960)--(8.123,1.960)--(8.124,1.960)--(8.125,1.960)--(8.126,1.960)--(8.126,1.959)%
  --(8.127,1.959)--(8.128,1.959)--(8.129,1.959)--(8.130,1.959)--(8.131,1.959)--(8.132,1.959)%
  --(8.133,1.959)--(8.133,1.958)--(8.134,1.958)--(8.135,1.958)--(8.136,1.958)--(8.137,1.958)%
  --(8.138,1.958)--(8.139,1.958)--(8.140,1.958)--(8.141,1.958)--(8.141,1.957)--(8.142,1.957)%
  --(8.143,1.957)--(8.144,1.957)--(8.145,1.957)--(8.146,1.957)--(8.147,1.957)--(8.148,1.957)%
  --(8.148,1.956)--(8.149,1.956)--(8.150,1.956)--(8.151,1.956)--(8.152,1.956)--(8.153,1.956)%
  --(8.154,1.956)--(8.155,1.956)--(8.155,1.955)--(8.156,1.955)--(8.157,1.955)--(8.158,1.955)%
  --(8.159,1.955)--(8.160,1.955)--(8.161,1.955)--(8.162,1.955)--(8.162,1.954)--(8.163,1.954)%
  --(8.164,1.954)--(8.165,1.954)--(8.166,1.954)--(8.167,1.954)--(8.168,1.954)--(8.169,1.954)%
  --(8.169,1.953)--(8.170,1.953)--(8.171,1.953)--(8.172,1.953)--(8.173,1.953)--(8.174,1.953)%
  --(8.175,1.953)--(8.176,1.953)--(8.176,1.952)--(8.177,1.952)--(8.178,1.952)--(8.179,1.952)%
  --(8.180,1.952)--(8.181,1.952)--(8.182,1.952)--(8.183,1.952)--(8.184,1.952)--(8.184,1.951)%
  --(8.185,1.951)--(8.186,1.951)--(8.187,1.951)--(8.188,1.951)--(8.189,1.951)--(8.190,1.951)%
  --(8.191,1.951)--(8.191,1.950)--(8.192,1.950)--(8.193,1.950)--(8.194,1.950)--(8.195,1.950)%
  --(8.196,1.950)--(8.197,1.950)--(8.198,1.950)--(8.198,1.949)--(8.199,1.949)--(8.200,1.949)%
  --(8.201,1.949)--(8.202,1.949)--(8.203,1.949)--(8.204,1.949)--(8.205,1.949)--(8.205,1.948)%
  --(8.206,1.948)--(8.207,1.948)--(8.208,1.948)--(8.209,1.948)--(8.210,1.948)--(8.211,1.948)%
  --(8.212,1.948)--(8.213,1.948)--(8.213,1.947)--(8.214,1.947)--(8.215,1.947)--(8.216,1.947)%
  --(8.217,1.947)--(8.218,1.947)--(8.219,1.947)--(8.220,1.947)--(8.220,1.946)--(8.221,1.946)%
  --(8.222,1.946)--(8.223,1.946)--(8.224,1.946)--(8.225,1.946)--(8.226,1.946)--(8.227,1.946)%
  --(8.227,1.945)--(8.228,1.945)--(8.229,1.945)--(8.230,1.945)--(8.231,1.945)--(8.232,1.945)%
  --(8.233,1.945)--(8.234,1.945)--(8.235,1.945)--(8.235,1.944)--(8.236,1.944)--(8.237,1.944)%
  --(8.238,1.944)--(8.239,1.944)--(8.240,1.944)--(8.241,1.944)--(8.242,1.944)--(8.242,1.943)%
  --(8.243,1.943)--(8.244,1.943)--(8.245,1.943)--(8.246,1.943)--(8.247,1.943)--(8.248,1.943)%
  --(8.249,1.943)--(8.249,1.942)--(8.250,1.942)--(8.251,1.942)--(8.252,1.942)--(8.253,1.942)%
  --(8.254,1.942)--(8.255,1.942)--(8.256,1.942)--(8.257,1.942)--(8.257,1.941)--(8.258,1.941)%
  --(8.259,1.941)--(8.260,1.941)--(8.261,1.941)--(8.262,1.941)--(8.263,1.941)--(8.264,1.941)%
  --(8.264,1.940)--(8.265,1.940)--(8.266,1.940)--(8.267,1.940)--(8.268,1.940)--(8.269,1.940)%
  --(8.270,1.940)--(8.271,1.940)--(8.272,1.940)--(8.272,1.939)--(8.273,1.939)--(8.274,1.939)%
  --(8.275,1.939)--(8.276,1.939)--(8.277,1.939)--(8.278,1.939)--(8.279,1.939)--(8.279,1.938)%
  --(8.280,1.938)--(8.281,1.938)--(8.282,1.938)--(8.283,1.938)--(8.284,1.938)--(8.285,1.938)%
  --(8.286,1.938)--(8.286,1.937)--(8.287,1.937)--(8.288,1.937)--(8.289,1.937)--(8.290,1.937)%
  --(8.291,1.937)--(8.292,1.937)--(8.293,1.937)--(8.294,1.937)--(8.294,1.936)--(8.295,1.936)%
  --(8.296,1.936)--(8.297,1.936)--(8.298,1.936)--(8.299,1.936)--(8.300,1.936)--(8.301,1.936)%
  --(8.301,1.935)--(8.302,1.935)--(8.303,1.935)--(8.304,1.935)--(8.305,1.935)--(8.306,1.935)%
  --(8.307,1.935)--(8.308,1.935)--(8.309,1.935)--(8.309,1.934)--(8.310,1.934)--(8.311,1.934)%
  --(8.312,1.934)--(8.313,1.934)--(8.314,1.934)--(8.315,1.934)--(8.316,1.934)--(8.316,1.933)%
  --(8.317,1.933)--(8.318,1.933)--(8.319,1.933)--(8.320,1.933)--(8.321,1.933)--(8.322,1.933)%
  --(8.323,1.933)--(8.324,1.933)--(8.324,1.932)--(8.325,1.932)--(8.326,1.932)--(8.327,1.932)%
  --(8.328,1.932)--(8.329,1.932)--(8.330,1.932)--(8.331,1.932)--(8.331,1.931)--(8.332,1.931)%
  --(8.333,1.931)--(8.334,1.931)--(8.335,1.931)--(8.336,1.931)--(8.337,1.931)--(8.338,1.931)%
  --(8.339,1.931)--(8.339,1.930)--(8.340,1.930)--(8.341,1.930)--(8.342,1.930)--(8.343,1.930)%
  --(8.344,1.930)--(8.345,1.930)--(8.346,1.930)--(8.346,1.929)--(8.347,1.929)--(8.348,1.929)%
  --(8.349,1.929)--(8.350,1.929)--(8.351,1.929)--(8.352,1.929)--(8.353,1.929)--(8.354,1.929)%
  --(8.354,1.928)--(8.355,1.928)--(8.356,1.928)--(8.357,1.928)--(8.358,1.928)--(8.359,1.928)%
  --(8.360,1.928)--(8.361,1.928)--(8.362,1.928)--(8.362,1.927)--(8.363,1.927)--(8.364,1.927)%
  --(8.365,1.927)--(8.366,1.927)--(8.367,1.927)--(8.368,1.927)--(8.369,1.927)--(8.369,1.926)%
  --(8.370,1.926)--(8.371,1.926)--(8.372,1.926)--(8.373,1.926)--(8.374,1.926)--(8.375,1.926)%
  --(8.376,1.926)--(8.377,1.926)--(8.377,1.925)--(8.378,1.925)--(8.379,1.925)--(8.380,1.925)%
  --(8.381,1.925)--(8.382,1.925)--(8.383,1.925)--(8.384,1.925)--(8.385,1.925)--(8.385,1.924)%
  --(8.386,1.924)--(8.387,1.924)--(8.388,1.924)--(8.389,1.924)--(8.390,1.924)--(8.391,1.924)%
  --(8.392,1.924)--(8.392,1.923)--(8.393,1.923)--(8.394,1.923)--(8.395,1.923)--(8.396,1.923)%
  --(8.397,1.923)--(8.398,1.923)--(8.399,1.923)--(8.400,1.923)--(8.400,1.922)--(8.401,1.922)%
  --(8.402,1.922)--(8.403,1.922)--(8.404,1.922)--(8.405,1.922)--(8.406,1.922)--(8.407,1.922)%
  --(8.408,1.922)--(8.408,1.921)--(8.409,1.921)--(8.410,1.921)--(8.411,1.921)--(8.412,1.921)%
  --(8.413,1.921)--(8.414,1.921)--(8.415,1.921)--(8.415,1.920)--(8.416,1.920)--(8.417,1.920)%
  --(8.418,1.920)--(8.419,1.920)--(8.420,1.920)--(8.421,1.920)--(8.422,1.920)--(8.423,1.920)%
  --(8.423,1.919)--(8.424,1.919)--(8.425,1.919)--(8.426,1.919)--(8.427,1.919)--(8.428,1.919)%
  --(8.429,1.919)--(8.430,1.919)--(8.431,1.919)--(8.431,1.918)--(8.432,1.918)--(8.433,1.918)%
  --(8.434,1.918)--(8.435,1.918)--(8.436,1.918)--(8.437,1.918)--(8.438,1.918)--(8.438,1.917)%
  --(8.439,1.917)--(8.440,1.917)--(8.441,1.917)--(8.442,1.917)--(8.443,1.917)--(8.444,1.917)%
  --(8.445,1.917)--(8.446,1.917)--(8.446,1.916)--(8.447,1.916)--(8.448,1.916)--(8.449,1.916)%
  --(8.450,1.916)--(8.451,1.916)--(8.452,1.916)--(8.453,1.916)--(8.454,1.916)--(8.454,1.915)%
  --(8.455,1.915)--(8.456,1.915)--(8.457,1.915)--(8.458,1.915)--(8.459,1.915)--(8.460,1.915)%
  --(8.461,1.915)--(8.462,1.915)--(8.462,1.914)--(8.463,1.914)--(8.464,1.914)--(8.465,1.914)%
  --(8.466,1.914)--(8.467,1.914)--(8.468,1.914)--(8.469,1.914)--(8.470,1.914)--(8.470,1.913)%
  --(8.471,1.913)--(8.472,1.913)--(8.473,1.913)--(8.474,1.913)--(8.475,1.913)--(8.476,1.913)%
  --(8.477,1.913)--(8.477,1.912)--(8.478,1.912)--(8.479,1.912)--(8.480,1.912)--(8.481,1.912)%
  --(8.482,1.912)--(8.483,1.912)--(8.484,1.912)--(8.485,1.912)--(8.485,1.911)--(8.486,1.911)%
  --(8.487,1.911)--(8.488,1.911)--(8.489,1.911)--(8.490,1.911)--(8.491,1.911)--(8.492,1.911)%
  --(8.493,1.911)--(8.493,1.910)--(8.494,1.910)--(8.495,1.910)--(8.496,1.910)--(8.497,1.910)%
  --(8.498,1.910)--(8.499,1.910)--(8.500,1.910)--(8.501,1.910)--(8.501,1.909)--(8.502,1.909)%
  --(8.503,1.909)--(8.504,1.909)--(8.505,1.909)--(8.506,1.909)--(8.507,1.909)--(8.508,1.909)%
  --(8.509,1.909)--(8.509,1.908)--(8.510,1.908)--(8.511,1.908)--(8.512,1.908)--(8.513,1.908)%
  --(8.514,1.908)--(8.515,1.908)--(8.516,1.908)--(8.517,1.908)--(8.517,1.907)--(8.518,1.907)%
  --(8.519,1.907)--(8.520,1.907)--(8.521,1.907)--(8.522,1.907)--(8.523,1.907)--(8.524,1.907)%
  --(8.525,1.907)--(8.525,1.906)--(8.526,1.906)--(8.527,1.906)--(8.528,1.906)--(8.529,1.906)%
  --(8.530,1.906)--(8.531,1.906)--(8.532,1.906)--(8.533,1.906)--(8.533,1.905)--(8.534,1.905)%
  --(8.535,1.905)--(8.536,1.905)--(8.537,1.905)--(8.538,1.905)--(8.539,1.905)--(8.540,1.905)%
  --(8.541,1.905)--(8.541,1.904)--(8.542,1.904)--(8.543,1.904)--(8.544,1.904)--(8.545,1.904)%
  --(8.546,1.904)--(8.547,1.904)--(8.548,1.904)--(8.549,1.904)--(8.549,1.903)--(8.550,1.903)%
  --(8.551,1.903)--(8.552,1.903)--(8.553,1.903)--(8.554,1.903)--(8.555,1.903)--(8.556,1.903)%
  --(8.557,1.903)--(8.557,1.902)--(8.558,1.902)--(8.559,1.902)--(8.560,1.902)--(8.561,1.902)%
  --(8.562,1.902)--(8.563,1.902)--(8.564,1.902)--(8.565,1.902)--(8.565,1.901)--(8.566,1.901)%
  --(8.567,1.901)--(8.568,1.901)--(8.569,1.901)--(8.570,1.901)--(8.571,1.901)--(8.572,1.901)%
  --(8.573,1.901)--(8.573,1.900)--(8.574,1.900)--(8.575,1.900)--(8.576,1.900)--(8.577,1.900)%
  --(8.578,1.900)--(8.579,1.900)--(8.580,1.900)--(8.581,1.900)--(8.581,1.899)--(8.582,1.899)%
  --(8.583,1.899)--(8.584,1.899)--(8.585,1.899)--(8.586,1.899)--(8.587,1.899)--(8.588,1.899)%
  --(8.589,1.899)--(8.589,1.898)--(8.590,1.898)--(8.591,1.898)--(8.592,1.898)--(8.593,1.898)%
  --(8.594,1.898)--(8.595,1.898)--(8.596,1.898)--(8.597,1.898)--(8.597,1.897)--(8.598,1.897)%
  --(8.599,1.897)--(8.600,1.897)--(8.601,1.897)--(8.602,1.897)--(8.603,1.897)--(8.604,1.897)%
  --(8.605,1.897)--(8.605,1.896)--(8.606,1.896)--(8.607,1.896)--(8.608,1.896)--(8.609,1.896)%
  --(8.610,1.896)--(8.611,1.896)--(8.612,1.896)--(8.613,1.896)--(8.613,1.895)--(8.614,1.895)%
  --(8.615,1.895)--(8.616,1.895)--(8.617,1.895)--(8.618,1.895)--(8.619,1.895)--(8.620,1.895)%
  --(8.621,1.895)--(8.622,1.895)--(8.622,1.894)--(8.623,1.894)--(8.624,1.894)--(8.625,1.894)%
  --(8.626,1.894)--(8.627,1.894)--(8.628,1.894)--(8.629,1.894)--(8.630,1.894)--(8.630,1.893)%
  --(8.631,1.893)--(8.632,1.893)--(8.633,1.893)--(8.634,1.893)--(8.635,1.893)--(8.636,1.893)%
  --(8.637,1.893)--(8.638,1.893)--(8.638,1.892)--(8.639,1.892)--(8.640,1.892)--(8.641,1.892)%
  --(8.642,1.892)--(8.643,1.892)--(8.644,1.892)--(8.645,1.892)--(8.646,1.892)--(8.646,1.891)%
  --(8.647,1.891)--(8.648,1.891)--(8.649,1.891)--(8.650,1.891)--(8.651,1.891)--(8.652,1.891)%
  --(8.653,1.891)--(8.654,1.891)--(8.654,1.890)--(8.655,1.890)--(8.656,1.890)--(8.657,1.890)%
  --(8.658,1.890)--(8.659,1.890)--(8.660,1.890)--(8.661,1.890)--(8.662,1.890)--(8.663,1.890)%
  --(8.663,1.889)--(8.664,1.889)--(8.665,1.889)--(8.666,1.889)--(8.667,1.889)--(8.668,1.889)%
  --(8.669,1.889)--(8.670,1.889)--(8.671,1.889)--(8.671,1.888)--(8.672,1.888)--(8.673,1.888)%
  --(8.674,1.888)--(8.675,1.888)--(8.676,1.888)--(8.677,1.888)--(8.678,1.888)--(8.679,1.888)%
  --(8.679,1.887)--(8.680,1.887)--(8.681,1.887)--(8.682,1.887)--(8.683,1.887)--(8.684,1.887)%
  --(8.685,1.887)--(8.686,1.887)--(8.687,1.887)--(8.687,1.886)--(8.688,1.886)--(8.689,1.886)%
  --(8.690,1.886)--(8.691,1.886)--(8.692,1.886)--(8.693,1.886)--(8.694,1.886)--(8.695,1.886)%
  --(8.696,1.886)--(8.696,1.885)--(8.697,1.885)--(8.698,1.885)--(8.699,1.885)--(8.700,1.885)%
  --(8.701,1.885)--(8.702,1.885)--(8.703,1.885)--(8.704,1.885)--(8.704,1.884)--(8.705,1.884)%
  --(8.706,1.884)--(8.707,1.884)--(8.708,1.884)--(8.709,1.884)--(8.710,1.884)--(8.711,1.884)%
  --(8.712,1.884)--(8.712,1.883)--(8.713,1.883)--(8.714,1.883)--(8.715,1.883)--(8.716,1.883)%
  --(8.717,1.883)--(8.718,1.883)--(8.719,1.883)--(8.720,1.883)--(8.721,1.883)--(8.721,1.882)%
  --(8.722,1.882)--(8.723,1.882)--(8.724,1.882)--(8.725,1.882)--(8.726,1.882)--(8.727,1.882)%
  --(8.728,1.882)--(8.729,1.882)--(8.729,1.881)--(8.730,1.881)--(8.731,1.881)--(8.732,1.881)%
  --(8.733,1.881)--(8.734,1.881)--(8.735,1.881)--(8.736,1.881)--(8.737,1.881)--(8.737,1.880)%
  --(8.738,1.880)--(8.739,1.880)--(8.740,1.880)--(8.741,1.880)--(8.742,1.880)--(8.743,1.880)%
  --(8.744,1.880)--(8.745,1.880)--(8.746,1.880)--(8.746,1.879)--(8.747,1.879)--(8.748,1.879)%
  --(8.749,1.879)--(8.750,1.879)--(8.751,1.879)--(8.752,1.879)--(8.753,1.879)--(8.754,1.879)%
  --(8.754,1.878)--(8.755,1.878)--(8.756,1.878)--(8.757,1.878)--(8.758,1.878)--(8.759,1.878)%
  --(8.760,1.878)--(8.761,1.878)--(8.762,1.878)--(8.763,1.878)--(8.763,1.877)--(8.764,1.877)%
  --(8.765,1.877)--(8.766,1.877)--(8.767,1.877)--(8.768,1.877)--(8.769,1.877)--(8.770,1.877)%
  --(8.771,1.877)--(8.771,1.876)--(8.772,1.876)--(8.773,1.876)--(8.774,1.876)--(8.775,1.876)%
  --(8.776,1.876)--(8.777,1.876)--(8.778,1.876)--(8.779,1.876)--(8.780,1.876)--(8.780,1.875)%
  --(8.781,1.875)--(8.782,1.875)--(8.783,1.875)--(8.784,1.875)--(8.785,1.875)--(8.786,1.875)%
  --(8.787,1.875)--(8.788,1.875)--(8.788,1.874)--(8.789,1.874)--(8.790,1.874)--(8.791,1.874)%
  --(8.792,1.874)--(8.793,1.874)--(8.794,1.874)--(8.795,1.874)--(8.796,1.874)--(8.797,1.874)%
  --(8.797,1.873)--(8.798,1.873)--(8.799,1.873)--(8.800,1.873)--(8.801,1.873)--(8.802,1.873)%
  --(8.803,1.873)--(8.804,1.873)--(8.805,1.873)--(8.805,1.872)--(8.806,1.872)--(8.807,1.872)%
  --(8.808,1.872)--(8.809,1.872)--(8.810,1.872)--(8.811,1.872)--(8.812,1.872)--(8.813,1.872)%
  --(8.814,1.872)--(8.814,1.871)--(8.815,1.871)--(8.816,1.871)--(8.817,1.871)--(8.818,1.871)%
  --(8.819,1.871)--(8.820,1.871)--(8.821,1.871)--(8.822,1.871)--(8.823,1.871)--(8.823,1.870)%
  --(8.824,1.870)--(8.825,1.870)--(8.826,1.870)--(8.827,1.870)--(8.828,1.870)--(8.829,1.870)%
  --(8.830,1.870)--(8.831,1.870)--(8.831,1.869)--(8.832,1.869)--(8.833,1.869)--(8.834,1.869)%
  --(8.835,1.869)--(8.836,1.869)--(8.837,1.869)--(8.838,1.869)--(8.839,1.869)--(8.840,1.869)%
  --(8.840,1.868)--(8.841,1.868)--(8.842,1.868)--(8.843,1.868)--(8.844,1.868)--(8.845,1.868)%
  --(8.846,1.868)--(8.847,1.868)--(8.848,1.868)--(8.848,1.867)--(8.849,1.867)--(8.850,1.867)%
  --(8.851,1.867)--(8.852,1.867)--(8.853,1.867)--(8.854,1.867)--(8.855,1.867)--(8.856,1.867)%
  --(8.857,1.867)--(8.857,1.866)--(8.858,1.866)--(8.859,1.866)--(8.860,1.866)--(8.861,1.866)%
  --(8.862,1.866)--(8.863,1.866)--(8.864,1.866)--(8.865,1.866)--(8.866,1.866)--(8.866,1.865)%
  --(8.867,1.865)--(8.868,1.865)--(8.869,1.865)--(8.870,1.865)--(8.871,1.865)--(8.872,1.865)%
  --(8.873,1.865)--(8.874,1.865)--(8.875,1.865)--(8.875,1.864)--(8.876,1.864)--(8.877,1.864)%
  --(8.878,1.864)--(8.879,1.864)--(8.880,1.864)--(8.881,1.864)--(8.882,1.864)--(8.883,1.864)%
  --(8.883,1.863)--(8.884,1.863)--(8.885,1.863)--(8.886,1.863)--(8.887,1.863)--(8.888,1.863)%
  --(8.889,1.863)--(8.890,1.863)--(8.891,1.863)--(8.892,1.863)--(8.892,1.862)--(8.893,1.862)%
  --(8.894,1.862)--(8.895,1.862)--(8.896,1.862)--(8.897,1.862)--(8.898,1.862)--(8.899,1.862)%
  --(8.900,1.862)--(8.901,1.862)--(8.901,1.861)--(8.902,1.861)--(8.903,1.861)--(8.904,1.861)%
  --(8.905,1.861)--(8.906,1.861)--(8.907,1.861)--(8.908,1.861)--(8.909,1.861)--(8.910,1.861)%
  --(8.910,1.860)--(8.911,1.860)--(8.912,1.860)--(8.913,1.860)--(8.914,1.860)--(8.915,1.860)%
  --(8.916,1.860)--(8.917,1.860)--(8.918,1.860)--(8.918,1.859)--(8.919,1.859)--(8.920,1.859)%
  --(8.921,1.859)--(8.922,1.859)--(8.923,1.859)--(8.924,1.859)--(8.925,1.859)--(8.926,1.859)%
  --(8.927,1.859)--(8.927,1.858)--(8.928,1.858)--(8.929,1.858)--(8.930,1.858)--(8.931,1.858)%
  --(8.932,1.858)--(8.933,1.858)--(8.934,1.858)--(8.935,1.858)--(8.936,1.858)--(8.936,1.857)%
  --(8.937,1.857)--(8.938,1.857)--(8.939,1.857)--(8.940,1.857)--(8.941,1.857)--(8.942,1.857)%
  --(8.943,1.857)--(8.944,1.857)--(8.945,1.857)--(8.945,1.856)--(8.946,1.856)--(8.947,1.856)%
  --(8.948,1.856)--(8.949,1.856)--(8.950,1.856)--(8.951,1.856)--(8.952,1.856)--(8.953,1.856)%
  --(8.954,1.856)--(8.954,1.855)--(8.955,1.855)--(8.956,1.855)--(8.957,1.855)--(8.958,1.855)%
  --(8.959,1.855)--(8.960,1.855)--(8.961,1.855)--(8.962,1.855)--(8.963,1.855)--(8.963,1.854)%
  --(8.964,1.854)--(8.965,1.854)--(8.966,1.854)--(8.967,1.854)--(8.968,1.854)--(8.969,1.854)%
  --(8.970,1.854)--(8.971,1.854)--(8.972,1.854)--(8.972,1.853)--(8.973,1.853)--(8.974,1.853)%
  --(8.975,1.853)--(8.976,1.853)--(8.977,1.853)--(8.978,1.853)--(8.979,1.853)--(8.980,1.853)%
  --(8.981,1.853)--(8.981,1.852)--(8.982,1.852)--(8.983,1.852)--(8.984,1.852)--(8.985,1.852)%
  --(8.986,1.852)--(8.987,1.852)--(8.988,1.852)--(8.989,1.852)--(8.990,1.852)--(8.990,1.851)%
  --(8.991,1.851)--(8.992,1.851)--(8.993,1.851)--(8.994,1.851)--(8.995,1.851)--(8.996,1.851)%
  --(8.997,1.851)--(8.998,1.851)--(8.999,1.851)--(8.999,1.850)--(9.000,1.850)--(9.001,1.850)%
  --(9.002,1.850)--(9.003,1.850)--(9.004,1.850)--(9.005,1.850)--(9.006,1.850)--(9.007,1.850)%
  --(9.008,1.850)--(9.008,1.849)--(9.009,1.849)--(9.010,1.849)--(9.011,1.849)--(9.012,1.849)%
  --(9.013,1.849)--(9.014,1.849)--(9.015,1.849)--(9.016,1.849)--(9.017,1.849)--(9.017,1.848)%
  --(9.018,1.848)--(9.019,1.848)--(9.020,1.848)--(9.021,1.848)--(9.022,1.848)--(9.023,1.848)%
  --(9.024,1.848)--(9.025,1.848)--(9.026,1.848)--(9.026,1.847)--(9.027,1.847)--(9.028,1.847)%
  --(9.029,1.847)--(9.030,1.847)--(9.031,1.847)--(9.032,1.847)--(9.033,1.847)--(9.034,1.847)%
  --(9.035,1.847)--(9.035,1.846)--(9.036,1.846)--(9.037,1.846)--(9.038,1.846)--(9.039,1.846)%
  --(9.040,1.846)--(9.041,1.846)--(9.042,1.846)--(9.043,1.846)--(9.044,1.846)--(9.044,1.845)%
  --(9.045,1.845)--(9.046,1.845)--(9.047,1.845)--(9.048,1.845)--(9.049,1.845)--(9.050,1.845)%
  --(9.051,1.845)--(9.052,1.845)--(9.053,1.845)--(9.053,1.844)--(9.054,1.844)--(9.055,1.844)%
  --(9.056,1.844)--(9.057,1.844)--(9.058,1.844)--(9.059,1.844)--(9.060,1.844)--(9.061,1.844)%
  --(9.062,1.844)--(9.062,1.843)--(9.063,1.843)--(9.064,1.843)--(9.065,1.843)--(9.066,1.843)%
  --(9.067,1.843)--(9.068,1.843)--(9.069,1.843)--(9.070,1.843)--(9.071,1.843)--(9.071,1.842)%
  --(9.072,1.842)--(9.073,1.842)--(9.074,1.842)--(9.075,1.842)--(9.076,1.842)--(9.077,1.842)%
  --(9.078,1.842)--(9.079,1.842)--(9.080,1.842)--(9.080,1.841)--(9.081,1.841)--(9.082,1.841)%
  --(9.083,1.841)--(9.084,1.841)--(9.085,1.841)--(9.086,1.841)--(9.087,1.841)--(9.088,1.841)%
  --(9.089,1.841)--(9.090,1.841)--(9.090,1.840)--(9.091,1.840)--(9.092,1.840)--(9.093,1.840)%
  --(9.094,1.840)--(9.095,1.840)--(9.096,1.840)--(9.097,1.840)--(9.098,1.840)--(9.099,1.840)%
  --(9.099,1.839)--(9.100,1.839)--(9.101,1.839)--(9.102,1.839)--(9.103,1.839)--(9.104,1.839)%
  --(9.105,1.839)--(9.106,1.839)--(9.107,1.839)--(9.108,1.839)--(9.108,1.838)--(9.109,1.838)%
  --(9.110,1.838)--(9.111,1.838)--(9.112,1.838)--(9.113,1.838)--(9.114,1.838)--(9.115,1.838)%
  --(9.116,1.838)--(9.117,1.838)--(9.117,1.837)--(9.118,1.837)--(9.119,1.837)--(9.120,1.837)%
  --(9.121,1.837)--(9.122,1.837)--(9.123,1.837)--(9.124,1.837)--(9.125,1.837)--(9.126,1.837)%
  --(9.127,1.837)--(9.127,1.836)--(9.128,1.836)--(9.129,1.836)--(9.130,1.836)--(9.131,1.836)%
  --(9.132,1.836)--(9.133,1.836)--(9.134,1.836)--(9.135,1.836)--(9.136,1.836)--(9.136,1.835)%
  --(9.137,1.835)--(9.138,1.835)--(9.139,1.835)--(9.140,1.835)--(9.141,1.835)--(9.142,1.835)%
  --(9.143,1.835)--(9.144,1.835)--(9.145,1.835)--(9.145,1.834)--(9.146,1.834)--(9.147,1.834)%
  --(9.148,1.834)--(9.149,1.834)--(9.150,1.834)--(9.151,1.834)--(9.152,1.834)--(9.153,1.834)%
  --(9.154,1.834)--(9.155,1.834)--(9.155,1.833)--(9.156,1.833)--(9.157,1.833)--(9.158,1.833)%
  --(9.159,1.833)--(9.160,1.833)--(9.161,1.833)--(9.162,1.833)--(9.163,1.833)--(9.164,1.833)%
  --(9.164,1.832)--(9.165,1.832)--(9.166,1.832)--(9.167,1.832)--(9.168,1.832)--(9.169,1.832)%
  --(9.170,1.832)--(9.171,1.832)--(9.172,1.832)--(9.173,1.832)--(9.173,1.831)--(9.174,1.831)%
  --(9.175,1.831)--(9.176,1.831)--(9.177,1.831)--(9.178,1.831)--(9.179,1.831)--(9.180,1.831)%
  --(9.181,1.831)--(9.182,1.831)--(9.183,1.831)--(9.183,1.830)--(9.184,1.830)--(9.185,1.830)%
  --(9.186,1.830)--(9.187,1.830)--(9.188,1.830)--(9.189,1.830)--(9.190,1.830)--(9.191,1.830)%
  --(9.192,1.830)--(9.192,1.829)--(9.193,1.829)--(9.194,1.829)--(9.195,1.829)--(9.196,1.829)%
  --(9.197,1.829)--(9.198,1.829)--(9.199,1.829)--(9.200,1.829)--(9.201,1.829)--(9.202,1.829)%
  --(9.202,1.828)--(9.203,1.828)--(9.204,1.828)--(9.205,1.828)--(9.206,1.828)--(9.207,1.828)%
  --(9.208,1.828)--(9.209,1.828)--(9.210,1.828)--(9.211,1.828)--(9.211,1.827)--(9.212,1.827)%
  --(9.213,1.827)--(9.214,1.827)--(9.215,1.827)--(9.216,1.827)--(9.217,1.827)--(9.218,1.827)%
  --(9.219,1.827)--(9.220,1.827)--(9.221,1.827)--(9.221,1.826)--(9.222,1.826)--(9.223,1.826)%
  --(9.224,1.826)--(9.225,1.826)--(9.226,1.826)--(9.227,1.826)--(9.228,1.826)--(9.229,1.826)%
  --(9.230,1.826)--(9.230,1.825)--(9.231,1.825)--(9.232,1.825)--(9.233,1.825)--(9.234,1.825)%
  --(9.235,1.825)--(9.236,1.825)--(9.237,1.825)--(9.238,1.825)--(9.239,1.825)--(9.240,1.825)%
  --(9.240,1.824)--(9.241,1.824)--(9.242,1.824)--(9.243,1.824)--(9.244,1.824)--(9.245,1.824)%
  --(9.246,1.824)--(9.247,1.824)--(9.248,1.824)--(9.249,1.824)--(9.249,1.823)--(9.250,1.823)%
  --(9.251,1.823)--(9.252,1.823)--(9.253,1.823)--(9.254,1.823)--(9.255,1.823)--(9.256,1.823)%
  --(9.257,1.823)--(9.258,1.823)--(9.259,1.823)--(9.259,1.822)--(9.260,1.822)--(9.261,1.822)%
  --(9.262,1.822)--(9.263,1.822)--(9.264,1.822)--(9.265,1.822)--(9.266,1.822)--(9.267,1.822)%
  --(9.268,1.822)--(9.269,1.822)--(9.269,1.821)--(9.270,1.821)--(9.271,1.821)--(9.272,1.821)%
  --(9.273,1.821)--(9.274,1.821)--(9.275,1.821)--(9.276,1.821)--(9.277,1.821)--(9.278,1.821)%
  --(9.278,1.820)--(9.279,1.820)--(9.280,1.820)--(9.281,1.820)--(9.282,1.820)--(9.283,1.820)%
  --(9.284,1.820)--(9.285,1.820)--(9.286,1.820)--(9.287,1.820)--(9.288,1.820)--(9.288,1.819)%
  --(9.289,1.819)--(9.290,1.819)--(9.291,1.819)--(9.292,1.819)--(9.293,1.819)--(9.294,1.819)%
  --(9.295,1.819)--(9.296,1.819)--(9.297,1.819)--(9.298,1.819)--(9.298,1.818)--(9.299,1.818)%
  --(9.300,1.818)--(9.301,1.818)--(9.302,1.818)--(9.303,1.818)--(9.304,1.818)--(9.305,1.818)%
  --(9.306,1.818)--(9.307,1.818)--(9.307,1.817)--(9.308,1.817)--(9.309,1.817)--(9.310,1.817)%
  --(9.311,1.817)--(9.312,1.817)--(9.313,1.817)--(9.314,1.817)--(9.315,1.817)--(9.316,1.817)%
  --(9.317,1.817)--(9.317,1.816)--(9.318,1.816)--(9.319,1.816)--(9.320,1.816)--(9.321,1.816)%
  --(9.322,1.816)--(9.323,1.816)--(9.324,1.816)--(9.325,1.816)--(9.326,1.816)--(9.327,1.816)%
  --(9.327,1.815)--(9.328,1.815)--(9.329,1.815)--(9.330,1.815)--(9.331,1.815)--(9.332,1.815)%
  --(9.333,1.815)--(9.334,1.815)--(9.335,1.815)--(9.336,1.815)--(9.337,1.815)--(9.337,1.814)%
  --(9.338,1.814)--(9.339,1.814)--(9.340,1.814)--(9.341,1.814)--(9.342,1.814)--(9.343,1.814)%
  --(9.344,1.814)--(9.345,1.814)--(9.346,1.814)--(9.346,1.813)--(9.347,1.813)--(9.348,1.813)%
  --(9.349,1.813)--(9.350,1.813)--(9.351,1.813)--(9.352,1.813)--(9.353,1.813)--(9.354,1.813)%
  --(9.355,1.813)--(9.356,1.813)--(9.356,1.812)--(9.357,1.812)--(9.358,1.812)--(9.359,1.812)%
  --(9.360,1.812)--(9.361,1.812)--(9.362,1.812)--(9.363,1.812)--(9.364,1.812)--(9.365,1.812)%
  --(9.366,1.812)--(9.366,1.811)--(9.367,1.811)--(9.368,1.811)--(9.369,1.811)--(9.370,1.811)%
  --(9.371,1.811)--(9.372,1.811)--(9.373,1.811)--(9.374,1.811)--(9.375,1.811)--(9.376,1.811)%
  --(9.376,1.810)--(9.377,1.810)--(9.378,1.810)--(9.379,1.810)--(9.380,1.810)--(9.381,1.810)%
  --(9.382,1.810)--(9.383,1.810)--(9.384,1.810)--(9.385,1.810)--(9.386,1.810)--(9.386,1.809)%
  --(9.387,1.809)--(9.388,1.809)--(9.389,1.809)--(9.390,1.809)--(9.391,1.809)--(9.392,1.809)%
  --(9.393,1.809)--(9.394,1.809)--(9.395,1.809)--(9.396,1.809)--(9.396,1.808)--(9.397,1.808)%
  --(9.398,1.808)--(9.399,1.808)--(9.400,1.808)--(9.401,1.808)--(9.402,1.808)--(9.403,1.808)%
  --(9.404,1.808)--(9.405,1.808)--(9.406,1.808)--(9.406,1.807)--(9.407,1.807)--(9.408,1.807)%
  --(9.409,1.807)--(9.410,1.807)--(9.411,1.807)--(9.412,1.807)--(9.413,1.807)--(9.414,1.807)%
  --(9.415,1.807)--(9.416,1.807)--(9.416,1.806)--(9.417,1.806)--(9.418,1.806)--(9.419,1.806)%
  --(9.420,1.806)--(9.421,1.806)--(9.422,1.806)--(9.423,1.806)--(9.424,1.806)--(9.425,1.806)%
  --(9.426,1.806)--(9.426,1.805)--(9.427,1.805)--(9.428,1.805)--(9.429,1.805)--(9.430,1.805)%
  --(9.431,1.805)--(9.432,1.805)--(9.433,1.805)--(9.434,1.805)--(9.435,1.805)--(9.436,1.805)%
  --(9.436,1.804)--(9.437,1.804)--(9.438,1.804)--(9.439,1.804)--(9.440,1.804)--(9.441,1.804)%
  --(9.442,1.804)--(9.443,1.804)--(9.444,1.804)--(9.445,1.804)--(9.446,1.804)--(9.446,1.803)%
  --(9.447,1.803)--(9.448,1.803)--(9.449,1.803)--(9.450,1.803)--(9.451,1.803)--(9.452,1.803)%
  --(9.453,1.803)--(9.454,1.803)--(9.455,1.803)--(9.456,1.803)--(9.456,1.802)--(9.457,1.802)%
  --(9.458,1.802)--(9.459,1.802)--(9.460,1.802)--(9.461,1.802)--(9.462,1.802)--(9.463,1.802)%
  --(9.464,1.802)--(9.465,1.802)--(9.466,1.802)--(9.466,1.801)--(9.467,1.801)--(9.468,1.801)%
  --(9.469,1.801)--(9.470,1.801)--(9.471,1.801)--(9.472,1.801)--(9.473,1.801)--(9.474,1.801)%
  --(9.475,1.801)--(9.476,1.801)--(9.476,1.800)--(9.477,1.800)--(9.478,1.800)--(9.479,1.800)%
  --(9.480,1.800)--(9.481,1.800)--(9.482,1.800)--(9.483,1.800)--(9.484,1.800)--(9.485,1.800)%
  --(9.486,1.800)--(9.486,1.799)--(9.487,1.799)--(9.488,1.799)--(9.489,1.799)--(9.490,1.799)%
  --(9.491,1.799)--(9.492,1.799)--(9.493,1.799)--(9.494,1.799)--(9.495,1.799)--(9.496,1.799)%
  --(9.496,1.798)--(9.497,1.798)--(9.498,1.798)--(9.499,1.798)--(9.500,1.798)--(9.501,1.798)%
  --(9.502,1.798)--(9.503,1.798)--(9.504,1.798)--(9.505,1.798)--(9.506,1.798)--(9.507,1.798)%
  --(9.507,1.797)--(9.508,1.797)--(9.509,1.797)--(9.510,1.797)--(9.511,1.797)--(9.512,1.797)%
  --(9.513,1.797)--(9.514,1.797)--(9.515,1.797)--(9.516,1.797)--(9.517,1.797)--(9.517,1.796)%
  --(9.518,1.796)--(9.519,1.796)--(9.520,1.796)--(9.521,1.796)--(9.522,1.796)--(9.523,1.796)%
  --(9.524,1.796)--(9.525,1.796)--(9.526,1.796)--(9.527,1.796)--(9.527,1.795)--(9.528,1.795)%
  --(9.529,1.795)--(9.530,1.795)--(9.531,1.795)--(9.532,1.795)--(9.533,1.795)--(9.534,1.795)%
  --(9.535,1.795)--(9.536,1.795)--(9.537,1.795)--(9.537,1.794)--(9.538,1.794)--(9.539,1.794)%
  --(9.540,1.794)--(9.541,1.794)--(9.542,1.794)--(9.543,1.794)--(9.544,1.794)--(9.545,1.794)%
  --(9.546,1.794)--(9.547,1.794)--(9.548,1.794)--(9.548,1.793)--(9.549,1.793)--(9.550,1.793)%
  --(9.551,1.793)--(9.552,1.793)--(9.553,1.793)--(9.554,1.793)--(9.555,1.793)--(9.556,1.793)%
  --(9.557,1.793)--(9.558,1.793)--(9.558,1.792)--(9.559,1.792)--(9.560,1.792)--(9.561,1.792)%
  --(9.562,1.792)--(9.563,1.792)--(9.564,1.792)--(9.565,1.792)--(9.566,1.792)--(9.567,1.792)%
  --(9.568,1.792)--(9.568,1.791)--(9.569,1.791)--(9.570,1.791)--(9.571,1.791)--(9.572,1.791)%
  --(9.573,1.791)--(9.574,1.791)--(9.575,1.791)--(9.576,1.791)--(9.577,1.791)--(9.578,1.791)%
  --(9.579,1.791)--(9.579,1.790)--(9.580,1.790)--(9.581,1.790)--(9.582,1.790)--(9.583,1.790)%
  --(9.584,1.790)--(9.585,1.790)--(9.586,1.790)--(9.587,1.790)--(9.588,1.790)--(9.589,1.790)%
  --(9.589,1.789)--(9.590,1.789)--(9.591,1.789)--(9.592,1.789)--(9.593,1.789)--(9.594,1.789)%
  --(9.595,1.789)--(9.596,1.789)--(9.597,1.789)--(9.598,1.789)--(9.599,1.789)--(9.599,1.788)%
  --(9.600,1.788)--(9.601,1.788)--(9.602,1.788)--(9.603,1.788)--(9.604,1.788)--(9.605,1.788)%
  --(9.606,1.788)--(9.607,1.788)--(9.608,1.788)--(9.609,1.788)--(9.610,1.788)--(9.610,1.787)%
  --(9.611,1.787)--(9.612,1.787)--(9.613,1.787)--(9.614,1.787)--(9.615,1.787)--(9.616,1.787)%
  --(9.617,1.787)--(9.618,1.787)--(9.619,1.787)--(9.620,1.787)--(9.620,1.786)--(9.621,1.786)%
  --(9.622,1.786)--(9.623,1.786)--(9.624,1.786)--(9.625,1.786)--(9.626,1.786)--(9.627,1.786)%
  --(9.628,1.786)--(9.629,1.786)--(9.630,1.786)--(9.631,1.786)--(9.631,1.785)--(9.632,1.785)%
  --(9.633,1.785)--(9.634,1.785)--(9.635,1.785)--(9.636,1.785)--(9.637,1.785)--(9.638,1.785)%
  --(9.639,1.785)--(9.640,1.785)--(9.641,1.785)--(9.641,1.784)--(9.642,1.784)--(9.643,1.784)%
  --(9.644,1.784)--(9.645,1.784)--(9.646,1.784)--(9.647,1.784)--(9.648,1.784)--(9.649,1.784)%
  --(9.650,1.784)--(9.651,1.784)--(9.652,1.784)--(9.652,1.783)--(9.653,1.783)--(9.654,1.783)%
  --(9.655,1.783)--(9.656,1.783)--(9.657,1.783)--(9.658,1.783)--(9.659,1.783)--(9.660,1.783)%
  --(9.661,1.783)--(9.662,1.783)--(9.662,1.782)--(9.663,1.782)--(9.664,1.782)--(9.665,1.782)%
  --(9.666,1.782)--(9.667,1.782)--(9.668,1.782)--(9.669,1.782)--(9.670,1.782)--(9.671,1.782)%
  --(9.672,1.782)--(9.673,1.782)--(9.673,1.781)--(9.674,1.781)--(9.675,1.781)--(9.676,1.781)%
  --(9.677,1.781)--(9.678,1.781)--(9.679,1.781)--(9.680,1.781)--(9.681,1.781)--(9.682,1.781)%
  --(9.683,1.781)--(9.684,1.781)--(9.684,1.780)--(9.685,1.780)--(9.686,1.780)--(9.687,1.780)%
  --(9.688,1.780)--(9.689,1.780)--(9.690,1.780)--(9.691,1.780)--(9.692,1.780)--(9.693,1.780)%
  --(9.694,1.780)--(9.694,1.779)--(9.695,1.779)--(9.696,1.779)--(9.697,1.779)--(9.698,1.779)%
  --(9.699,1.779)--(9.700,1.779)--(9.701,1.779)--(9.702,1.779)--(9.703,1.779)--(9.704,1.779)%
  --(9.705,1.779)--(9.705,1.778)--(9.706,1.778)--(9.707,1.778)--(9.708,1.778)--(9.709,1.778)%
  --(9.710,1.778)--(9.711,1.778)--(9.712,1.778)--(9.713,1.778)--(9.714,1.778)--(9.715,1.778)%
  --(9.716,1.778)--(9.716,1.777)--(9.717,1.777)--(9.718,1.777)--(9.719,1.777)--(9.720,1.777)%
  --(9.721,1.777)--(9.722,1.777)--(9.723,1.777)--(9.724,1.777)--(9.725,1.777)--(9.726,1.777)%
  --(9.726,1.776)--(9.727,1.776)--(9.728,1.776)--(9.729,1.776)--(9.730,1.776)--(9.731,1.776)%
  --(9.732,1.776)--(9.733,1.776)--(9.734,1.776)--(9.735,1.776)--(9.736,1.776)--(9.737,1.776)%
  --(9.737,1.775)--(9.738,1.775)--(9.739,1.775)--(9.740,1.775)--(9.741,1.775)--(9.742,1.775)%
  --(9.743,1.775)--(9.744,1.775)--(9.745,1.775)--(9.746,1.775)--(9.747,1.775)--(9.748,1.775)%
  --(9.748,1.774)--(9.749,1.774)--(9.750,1.774)--(9.751,1.774)--(9.752,1.774)--(9.753,1.774)%
  --(9.754,1.774)--(9.755,1.774)--(9.756,1.774)--(9.757,1.774)--(9.758,1.774)--(9.759,1.774)%
  --(9.759,1.773)--(9.760,1.773)--(9.761,1.773)--(9.762,1.773)--(9.763,1.773)--(9.764,1.773)%
  --(9.765,1.773)--(9.766,1.773)--(9.767,1.773)--(9.768,1.773)--(9.769,1.773)--(9.770,1.773)%
  --(9.770,1.772)--(9.771,1.772)--(9.772,1.772)--(9.773,1.772)--(9.774,1.772)--(9.775,1.772)%
  --(9.776,1.772)--(9.777,1.772)--(9.778,1.772)--(9.779,1.772)--(9.780,1.772)--(9.781,1.772)%
  --(9.781,1.771)--(9.782,1.771)--(9.783,1.771)--(9.784,1.771)--(9.785,1.771)--(9.786,1.771)%
  --(9.787,1.771)--(9.788,1.771)--(9.789,1.771)--(9.790,1.771)--(9.791,1.771)--(9.791,1.770)%
  --(9.792,1.770)--(9.793,1.770)--(9.794,1.770)--(9.795,1.770)--(9.796,1.770)--(9.797,1.770)%
  --(9.798,1.770)--(9.799,1.770)--(9.800,1.770)--(9.801,1.770)--(9.802,1.770)--(9.802,1.769)%
  --(9.803,1.769)--(9.804,1.769)--(9.805,1.769)--(9.806,1.769)--(9.807,1.769)--(9.808,1.769)%
  --(9.809,1.769)--(9.810,1.769)--(9.811,1.769)--(9.812,1.769)--(9.813,1.769)--(9.813,1.768)%
  --(9.814,1.768)--(9.815,1.768)--(9.816,1.768)--(9.817,1.768)--(9.818,1.768)--(9.819,1.768)%
  --(9.820,1.768)--(9.821,1.768)--(9.822,1.768)--(9.823,1.768)--(9.824,1.768)--(9.824,1.767)%
  --(9.825,1.767)--(9.826,1.767)--(9.827,1.767)--(9.828,1.767)--(9.829,1.767)--(9.830,1.767)%
  --(9.831,1.767)--(9.832,1.767)--(9.833,1.767)--(9.834,1.767)--(9.835,1.767)--(9.835,1.766)%
  --(9.836,1.766)--(9.837,1.766)--(9.838,1.766)--(9.839,1.766)--(9.840,1.766)--(9.841,1.766)%
  --(9.842,1.766)--(9.843,1.766)--(9.844,1.766)--(9.845,1.766)--(9.846,1.766)--(9.846,1.765)%
  --(9.847,1.765)--(9.848,1.765)--(9.849,1.765)--(9.850,1.765)--(9.851,1.765)--(9.852,1.765)%
  --(9.853,1.765)--(9.854,1.765)--(9.855,1.765)--(9.856,1.765)--(9.857,1.765)--(9.857,1.764)%
  --(9.858,1.764)--(9.859,1.764)--(9.860,1.764)--(9.861,1.764)--(9.862,1.764)--(9.863,1.764)%
  --(9.864,1.764)--(9.865,1.764)--(9.866,1.764)--(9.867,1.764)--(9.868,1.764)--(9.869,1.764)%
  --(9.869,1.763)--(9.870,1.763)--(9.871,1.763)--(9.872,1.763)--(9.873,1.763)--(9.874,1.763)%
  --(9.875,1.763)--(9.876,1.763)--(9.877,1.763)--(9.878,1.763)--(9.879,1.763)--(9.880,1.763)%
  --(9.880,1.762)--(9.881,1.762)--(9.882,1.762)--(9.883,1.762)--(9.884,1.762)--(9.885,1.762)%
  --(9.886,1.762)--(9.887,1.762)--(9.888,1.762)--(9.889,1.762)--(9.890,1.762)--(9.891,1.762)%
  --(9.891,1.761)--(9.892,1.761)--(9.893,1.761)--(9.894,1.761)--(9.895,1.761)--(9.896,1.761)%
  --(9.897,1.761)--(9.898,1.761)--(9.899,1.761)--(9.900,1.761)--(9.901,1.761)--(9.902,1.761)%
  --(9.902,1.760)--(9.903,1.760)--(9.904,1.760)--(9.905,1.760)--(9.906,1.760)--(9.907,1.760)%
  --(9.908,1.760)--(9.909,1.760)--(9.910,1.760)--(9.911,1.760)--(9.912,1.760)--(9.913,1.760)%
  --(9.913,1.759)--(9.914,1.759)--(9.915,1.759)--(9.916,1.759)--(9.917,1.759)--(9.918,1.759)%
  --(9.919,1.759)--(9.920,1.759)--(9.921,1.759)--(9.922,1.759)--(9.923,1.759)--(9.924,1.759)%
  --(9.924,1.758)--(9.925,1.758)--(9.926,1.758)--(9.927,1.758)--(9.928,1.758)--(9.929,1.758)%
  --(9.930,1.758)--(9.931,1.758)--(9.932,1.758)--(9.933,1.758)--(9.934,1.758)--(9.935,1.758)%
  --(9.936,1.758)--(9.936,1.757)--(9.937,1.757)--(9.938,1.757)--(9.939,1.757)--(9.940,1.757)%
  --(9.941,1.757)--(9.942,1.757)--(9.943,1.757)--(9.944,1.757)--(9.945,1.757)--(9.946,1.757)%
  --(9.947,1.757)--(9.947,1.756)--(9.948,1.756)--(9.949,1.756)--(9.950,1.756)--(9.951,1.756)%
  --(9.952,1.756)--(9.953,1.756)--(9.954,1.756)--(9.955,1.756)--(9.956,1.756)--(9.957,1.756)%
  --(9.958,1.756)--(9.958,1.755)--(9.959,1.755)--(9.960,1.755)--(9.961,1.755)--(9.962,1.755)%
  --(9.963,1.755)--(9.964,1.755)--(9.965,1.755)--(9.966,1.755)--(9.967,1.755)--(9.968,1.755)%
  --(9.969,1.755)--(9.970,1.755)--(9.970,1.754)--(9.971,1.754)--(9.972,1.754)--(9.973,1.754)%
  --(9.974,1.754)--(9.975,1.754)--(9.976,1.754)--(9.977,1.754)--(9.978,1.754)--(9.979,1.754)%
  --(9.980,1.754)--(9.981,1.754)--(9.981,1.753)--(9.982,1.753)--(9.983,1.753)--(9.984,1.753)%
  --(9.985,1.753)--(9.986,1.753)--(9.987,1.753)--(9.988,1.753)--(9.989,1.753)--(9.990,1.753)%
  --(9.991,1.753)--(9.992,1.753)--(9.992,1.752)--(9.993,1.752)--(9.994,1.752)--(9.995,1.752)%
  --(9.996,1.752)--(9.997,1.752)--(9.998,1.752)--(9.999,1.752)--(10.000,1.752)--(10.001,1.752)%
  --(10.002,1.752)--(10.003,1.752)--(10.004,1.752)--(10.004,1.751)--(10.005,1.751)--(10.006,1.751)%
  --(10.007,1.751)--(10.008,1.751)--(10.009,1.751)--(10.010,1.751)--(10.011,1.751)--(10.012,1.751)%
  --(10.013,1.751)--(10.014,1.751)--(10.015,1.751)--(10.015,1.750)--(10.016,1.750)--(10.017,1.750)%
  --(10.018,1.750)--(10.019,1.750)--(10.020,1.750)--(10.021,1.750)--(10.022,1.750)--(10.023,1.750)%
  --(10.024,1.750)--(10.025,1.750)--(10.026,1.750)--(10.027,1.750)--(10.027,1.749)--(10.028,1.749)%
  --(10.029,1.749)--(10.030,1.749)--(10.031,1.749)--(10.032,1.749)--(10.033,1.749)--(10.034,1.749)%
  --(10.035,1.749)--(10.036,1.749)--(10.037,1.749)--(10.038,1.749)--(10.038,1.748)--(10.039,1.748)%
  --(10.040,1.748)--(10.041,1.748)--(10.042,1.748)--(10.043,1.748)--(10.044,1.748)--(10.045,1.748)%
  --(10.046,1.748)--(10.047,1.748)--(10.048,1.748)--(10.049,1.748)--(10.050,1.748)--(10.050,1.747)%
  --(10.051,1.747)--(10.052,1.747)--(10.053,1.747)--(10.054,1.747)--(10.055,1.747)--(10.056,1.747)%
  --(10.057,1.747)--(10.058,1.747)--(10.059,1.747)--(10.060,1.747)--(10.061,1.747)--(10.062,1.747)%
  --(10.062,1.746)--(10.063,1.746)--(10.064,1.746)--(10.065,1.746)--(10.066,1.746)--(10.067,1.746)%
  --(10.068,1.746)--(10.069,1.746)--(10.070,1.746)--(10.071,1.746)--(10.072,1.746)--(10.073,1.746)%
  --(10.073,1.745)--(10.074,1.745)--(10.075,1.745)--(10.076,1.745)--(10.077,1.745)--(10.078,1.745)%
  --(10.079,1.745)--(10.080,1.745)--(10.081,1.745)--(10.082,1.745)--(10.083,1.745)--(10.084,1.745)%
  --(10.085,1.745)--(10.085,1.744)--(10.086,1.744)--(10.087,1.744)--(10.088,1.744)--(10.089,1.744)%
  --(10.090,1.744)--(10.091,1.744)--(10.092,1.744)--(10.093,1.744)--(10.094,1.744)--(10.095,1.744)%
  --(10.096,1.744)--(10.097,1.744)--(10.097,1.743)--(10.098,1.743)--(10.099,1.743)--(10.100,1.743)%
  --(10.101,1.743)--(10.102,1.743)--(10.103,1.743)--(10.104,1.743)--(10.105,1.743)--(10.106,1.743)%
  --(10.107,1.743)--(10.108,1.743)--(10.108,1.742)--(10.109,1.742)--(10.110,1.742)--(10.111,1.742)%
  --(10.112,1.742)--(10.113,1.742)--(10.114,1.742)--(10.115,1.742)--(10.116,1.742)--(10.117,1.742)%
  --(10.118,1.742)--(10.119,1.742)--(10.120,1.742)--(10.120,1.741)--(10.121,1.741)--(10.122,1.741)%
  --(10.123,1.741)--(10.124,1.741)--(10.125,1.741)--(10.126,1.741)--(10.127,1.741)--(10.128,1.741)%
  --(10.129,1.741)--(10.130,1.741)--(10.131,1.741)--(10.132,1.741)--(10.132,1.740)--(10.133,1.740)%
  --(10.134,1.740)--(10.135,1.740)--(10.136,1.740)--(10.137,1.740)--(10.138,1.740)--(10.139,1.740)%
  --(10.140,1.740)--(10.141,1.740)--(10.142,1.740)--(10.143,1.740)--(10.144,1.740)--(10.144,1.739)%
  --(10.145,1.739)--(10.146,1.739)--(10.147,1.739)--(10.148,1.739)--(10.149,1.739)--(10.150,1.739)%
  --(10.151,1.739)--(10.152,1.739)--(10.153,1.739)--(10.154,1.739)--(10.155,1.739)--(10.155,1.738)%
  --(10.156,1.738)--(10.157,1.738)--(10.158,1.738)--(10.159,1.738)--(10.160,1.738)--(10.161,1.738)%
  --(10.162,1.738)--(10.163,1.738)--(10.164,1.738)--(10.165,1.738)--(10.166,1.738)--(10.167,1.738)%
  --(10.167,1.737)--(10.168,1.737)--(10.169,1.737)--(10.170,1.737)--(10.171,1.737)--(10.172,1.737)%
  --(10.173,1.737)--(10.174,1.737)--(10.175,1.737)--(10.176,1.737)--(10.177,1.737)--(10.178,1.737)%
  --(10.179,1.737)--(10.179,1.736)--(10.180,1.736)--(10.181,1.736)--(10.182,1.736)--(10.183,1.736)%
  --(10.184,1.736)--(10.185,1.736)--(10.186,1.736)--(10.187,1.736)--(10.188,1.736)--(10.189,1.736)%
  --(10.190,1.736)--(10.191,1.736)--(10.191,1.735)--(10.192,1.735)--(10.193,1.735)--(10.194,1.735)%
  --(10.195,1.735)--(10.196,1.735)--(10.197,1.735)--(10.198,1.735)--(10.199,1.735)--(10.200,1.735)%
  --(10.201,1.735)--(10.202,1.735)--(10.203,1.735)--(10.203,1.734)--(10.204,1.734)--(10.205,1.734)%
  --(10.206,1.734)--(10.207,1.734)--(10.208,1.734)--(10.209,1.734)--(10.210,1.734)--(10.211,1.734)%
  --(10.212,1.734)--(10.213,1.734)--(10.214,1.734)--(10.215,1.734)--(10.215,1.733)--(10.216,1.733)%
  --(10.217,1.733)--(10.218,1.733)--(10.219,1.733)--(10.220,1.733)--(10.221,1.733)--(10.222,1.733)%
  --(10.223,1.733)--(10.224,1.733)--(10.225,1.733)--(10.226,1.733)--(10.227,1.733)--(10.227,1.732)%
  --(10.228,1.732)--(10.229,1.732)--(10.230,1.732)--(10.231,1.732)--(10.232,1.732)--(10.233,1.732)%
  --(10.234,1.732)--(10.235,1.732)--(10.236,1.732)--(10.237,1.732)--(10.238,1.732)--(10.239,1.732)%
  --(10.239,1.731)--(10.240,1.731)--(10.241,1.731)--(10.242,1.731)--(10.243,1.731)--(10.244,1.731)%
  --(10.245,1.731)--(10.246,1.731)--(10.247,1.731)--(10.248,1.731)--(10.249,1.731)--(10.250,1.731)%
  --(10.251,1.731)--(10.251,1.730)--(10.252,1.730)--(10.253,1.730)--(10.254,1.730)--(10.255,1.730)%
  --(10.256,1.730)--(10.257,1.730)--(10.258,1.730)--(10.259,1.730)--(10.260,1.730)--(10.261,1.730)%
  --(10.262,1.730)--(10.263,1.730)--(10.263,1.729)--(10.264,1.729)--(10.265,1.729)--(10.266,1.729)%
  --(10.267,1.729)--(10.268,1.729)--(10.269,1.729)--(10.270,1.729)--(10.271,1.729)--(10.272,1.729)%
  --(10.273,1.729)--(10.274,1.729)--(10.275,1.729)--(10.276,1.729)--(10.276,1.728)--(10.277,1.728)%
  --(10.278,1.728)--(10.279,1.728)--(10.280,1.728)--(10.281,1.728)--(10.282,1.728)--(10.283,1.728)%
  --(10.284,1.728)--(10.285,1.728)--(10.286,1.728)--(10.287,1.728)--(10.288,1.728)--(10.288,1.727)%
  --(10.289,1.727)--(10.290,1.727)--(10.291,1.727)--(10.292,1.727)--(10.293,1.727)--(10.294,1.727)%
  --(10.295,1.727)--(10.296,1.727)--(10.297,1.727)--(10.298,1.727)--(10.299,1.727)--(10.300,1.727)%
  --(10.300,1.726)--(10.301,1.726)--(10.302,1.726)--(10.303,1.726)--(10.304,1.726)--(10.305,1.726)%
  --(10.306,1.726)--(10.307,1.726)--(10.308,1.726)--(10.309,1.726)--(10.310,1.726)--(10.311,1.726)%
  --(10.312,1.726)--(10.312,1.725)--(10.313,1.725)--(10.314,1.725)--(10.315,1.725)--(10.316,1.725)%
  --(10.317,1.725)--(10.318,1.725)--(10.319,1.725)--(10.320,1.725)--(10.321,1.725)--(10.322,1.725)%
  --(10.323,1.725)--(10.324,1.725)--(10.325,1.725)--(10.325,1.724)--(10.326,1.724)--(10.327,1.724)%
  --(10.328,1.724)--(10.329,1.724)--(10.330,1.724)--(10.331,1.724)--(10.332,1.724)--(10.333,1.724)%
  --(10.334,1.724)--(10.335,1.724)--(10.336,1.724)--(10.337,1.724)--(10.337,1.723)--(10.338,1.723)%
  --(10.339,1.723)--(10.340,1.723)--(10.341,1.723)--(10.342,1.723)--(10.343,1.723)--(10.344,1.723)%
  --(10.345,1.723)--(10.346,1.723)--(10.347,1.723)--(10.348,1.723)--(10.349,1.723)--(10.349,1.722)%
  --(10.350,1.722)--(10.351,1.722)--(10.352,1.722)--(10.353,1.722)--(10.354,1.722)--(10.355,1.722)%
  --(10.356,1.722)--(10.357,1.722)--(10.358,1.722)--(10.359,1.722)--(10.360,1.722)--(10.361,1.722)%
  --(10.362,1.722)--(10.362,1.721)--(10.363,1.721)--(10.364,1.721)--(10.365,1.721)--(10.366,1.721)%
  --(10.367,1.721)--(10.368,1.721)--(10.369,1.721)--(10.370,1.721)--(10.371,1.721)--(10.372,1.721)%
  --(10.373,1.721)--(10.374,1.721)--(10.374,1.720)--(10.375,1.720)--(10.376,1.720)--(10.377,1.720)%
  --(10.378,1.720)--(10.379,1.720)--(10.380,1.720)--(10.381,1.720)--(10.382,1.720)--(10.383,1.720)%
  --(10.384,1.720)--(10.385,1.720)--(10.386,1.720)--(10.386,1.719)--(10.387,1.719)--(10.388,1.719)%
  --(10.389,1.719)--(10.390,1.719)--(10.391,1.719)--(10.392,1.719)--(10.393,1.719)--(10.394,1.719)%
  --(10.395,1.719)--(10.396,1.719)--(10.397,1.719)--(10.398,1.719)--(10.399,1.719)--(10.399,1.718)%
  --(10.400,1.718)--(10.401,1.718)--(10.402,1.718)--(10.403,1.718)--(10.404,1.718)--(10.405,1.718)%
  --(10.406,1.718)--(10.407,1.718)--(10.408,1.718)--(10.409,1.718)--(10.410,1.718)--(10.411,1.718)%
  --(10.412,1.717)--(10.413,1.717)--(10.414,1.717)--(10.415,1.717)--(10.416,1.717)--(10.417,1.717)%
  --(10.418,1.717)--(10.419,1.717)--(10.420,1.717)--(10.421,1.717)--(10.422,1.717)--(10.423,1.717)%
  --(10.424,1.717)--(10.424,1.716)--(10.425,1.716)--(10.426,1.716)--(10.427,1.716)--(10.428,1.716)%
  --(10.429,1.716)--(10.430,1.716)--(10.431,1.716)--(10.432,1.716)--(10.433,1.716)--(10.434,1.716)%
  --(10.435,1.716)--(10.436,1.716)--(10.437,1.716)--(10.437,1.715)--(10.438,1.715)--(10.439,1.715)%
  --(10.440,1.715)--(10.441,1.715)--(10.442,1.715)--(10.443,1.715)--(10.444,1.715)--(10.445,1.715)%
  --(10.446,1.715)--(10.447,1.715)--(10.448,1.715)--(10.449,1.715)--(10.449,1.714)--(10.450,1.714)%
  --(10.451,1.714)--(10.452,1.714)--(10.453,1.714)--(10.454,1.714)--(10.455,1.714)--(10.456,1.714)%
  --(10.457,1.714)--(10.458,1.714)--(10.459,1.714)--(10.460,1.714)--(10.461,1.714)--(10.462,1.714)%
  --(10.462,1.713)--(10.463,1.713)--(10.464,1.713)--(10.465,1.713)--(10.466,1.713)--(10.467,1.713)%
  --(10.468,1.713)--(10.469,1.713)--(10.470,1.713)--(10.471,1.713)--(10.472,1.713)--(10.473,1.713)%
  --(10.474,1.713)--(10.475,1.713)--(10.475,1.712)--(10.476,1.712)--(10.477,1.712)--(10.478,1.712)%
  --(10.479,1.712)--(10.480,1.712)--(10.481,1.712)--(10.482,1.712)--(10.483,1.712)--(10.484,1.712)%
  --(10.485,1.712)--(10.486,1.712)--(10.487,1.712)--(10.487,1.711)--(10.488,1.711)--(10.489,1.711)%
  --(10.490,1.711)--(10.491,1.711)--(10.492,1.711)--(10.493,1.711)--(10.494,1.711)--(10.495,1.711)%
  --(10.496,1.711)--(10.497,1.711)--(10.498,1.711)--(10.499,1.711)--(10.500,1.711)--(10.500,1.710)%
  --(10.501,1.710)--(10.502,1.710)--(10.503,1.710)--(10.504,1.710)--(10.505,1.710)--(10.506,1.710)%
  --(10.507,1.710)--(10.508,1.710)--(10.509,1.710)--(10.510,1.710)--(10.511,1.710)--(10.512,1.710)%
  --(10.513,1.710)--(10.513,1.709)--(10.514,1.709)--(10.515,1.709)--(10.516,1.709)--(10.517,1.709)%
  --(10.518,1.709)--(10.519,1.709)--(10.520,1.709)--(10.521,1.709)--(10.522,1.709)--(10.523,1.709)%
  --(10.524,1.709)--(10.525,1.709)--(10.526,1.709)--(10.526,1.708)--(10.527,1.708)--(10.528,1.708)%
  --(10.529,1.708)--(10.530,1.708)--(10.531,1.708)--(10.532,1.708)--(10.533,1.708)--(10.534,1.708)%
  --(10.535,1.708)--(10.536,1.708)--(10.537,1.708)--(10.538,1.708)--(10.539,1.708)--(10.539,1.707)%
  --(10.540,1.707)--(10.541,1.707)--(10.542,1.707)--(10.543,1.707)--(10.544,1.707)--(10.545,1.707)%
  --(10.546,1.707)--(10.547,1.707)--(10.548,1.707)--(10.549,1.707)--(10.550,1.707)--(10.551,1.707)%
  --(10.552,1.707)--(10.552,1.706)--(10.553,1.706)--(10.554,1.706)--(10.555,1.706)--(10.556,1.706)%
  --(10.557,1.706)--(10.558,1.706)--(10.559,1.706)--(10.560,1.706)--(10.561,1.706)--(10.562,1.706)%
  --(10.563,1.706)--(10.564,1.706)--(10.564,1.705)--(10.565,1.705)--(10.566,1.705)--(10.567,1.705)%
  --(10.568,1.705)--(10.569,1.705)--(10.570,1.705)--(10.571,1.705)--(10.572,1.705)--(10.573,1.705)%
  --(10.574,1.705)--(10.575,1.705)--(10.576,1.705)--(10.577,1.705)--(10.577,1.704)--(10.578,1.704)%
  --(10.579,1.704)--(10.580,1.704)--(10.581,1.704)--(10.582,1.704)--(10.583,1.704)--(10.584,1.704)%
  --(10.585,1.704)--(10.586,1.704)--(10.587,1.704)--(10.588,1.704)--(10.589,1.704)--(10.590,1.704)%
  --(10.590,1.703)--(10.591,1.703)--(10.592,1.703)--(10.593,1.703)--(10.594,1.703)--(10.595,1.703)%
  --(10.596,1.703)--(10.597,1.703)--(10.598,1.703)--(10.599,1.703)--(10.600,1.703)--(10.601,1.703)%
  --(10.602,1.703)--(10.603,1.703)--(10.604,1.703)--(10.604,1.702)--(10.605,1.702)--(10.606,1.702)%
  --(10.607,1.702)--(10.608,1.702)--(10.609,1.702)--(10.610,1.702)--(10.611,1.702)--(10.612,1.702)%
  --(10.613,1.702)--(10.614,1.702)--(10.615,1.702)--(10.616,1.702)--(10.617,1.702)--(10.617,1.701)%
  --(10.618,1.701)--(10.619,1.701)--(10.620,1.701)--(10.621,1.701)--(10.622,1.701)--(10.623,1.701)%
  --(10.624,1.701)--(10.625,1.701)--(10.626,1.701)--(10.627,1.701)--(10.628,1.701)--(10.629,1.701)%
  --(10.630,1.701)--(10.630,1.700)--(10.631,1.700)--(10.632,1.700)--(10.633,1.700)--(10.634,1.700)%
  --(10.635,1.700)--(10.636,1.700)--(10.637,1.700)--(10.638,1.700)--(10.639,1.700)--(10.640,1.700)%
  --(10.641,1.700)--(10.642,1.700)--(10.643,1.700)--(10.643,1.699)--(10.644,1.699)--(10.645,1.699)%
  --(10.646,1.699)--(10.647,1.699)--(10.648,1.699)--(10.649,1.699)--(10.650,1.699)--(10.651,1.699)%
  --(10.652,1.699)--(10.653,1.699)--(10.654,1.699)--(10.655,1.699)--(10.656,1.699)--(10.656,1.698)%
  --(10.657,1.698)--(10.658,1.698)--(10.659,1.698)--(10.660,1.698)--(10.661,1.698)--(10.662,1.698)%
  --(10.663,1.698)--(10.664,1.698)--(10.665,1.698)--(10.666,1.698)--(10.667,1.698)--(10.668,1.698)%
  --(10.669,1.698)--(10.669,1.697)--(10.670,1.697)--(10.671,1.697)--(10.672,1.697)--(10.673,1.697)%
  --(10.674,1.697)--(10.675,1.697)--(10.676,1.697)--(10.677,1.697)--(10.678,1.697)--(10.679,1.697)%
  --(10.680,1.697)--(10.681,1.697)--(10.682,1.697)--(10.683,1.697)--(10.683,1.696)--(10.684,1.696)%
  --(10.685,1.696)--(10.686,1.696)--(10.687,1.696)--(10.688,1.696)--(10.689,1.696)--(10.690,1.696)%
  --(10.691,1.696)--(10.692,1.696)--(10.693,1.696)--(10.694,1.696)--(10.695,1.696)--(10.696,1.696)%
  --(10.696,1.695)--(10.697,1.695)--(10.698,1.695)--(10.699,1.695)--(10.700,1.695)--(10.701,1.695)%
  --(10.702,1.695)--(10.703,1.695)--(10.704,1.695)--(10.705,1.695)--(10.706,1.695)--(10.707,1.695)%
  --(10.708,1.695)--(10.709,1.695)--(10.709,1.694)--(10.710,1.694)--(10.711,1.694)--(10.712,1.694)%
  --(10.713,1.694)--(10.714,1.694)--(10.715,1.694)--(10.716,1.694)--(10.717,1.694)--(10.718,1.694)%
  --(10.719,1.694)--(10.720,1.694)--(10.721,1.694)--(10.722,1.694)--(10.723,1.694)--(10.723,1.693)%
  --(10.724,1.693)--(10.725,1.693)--(10.726,1.693)--(10.727,1.693)--(10.728,1.693)--(10.729,1.693)%
  --(10.730,1.693)--(10.731,1.693)--(10.732,1.693)--(10.733,1.693)--(10.734,1.693)--(10.735,1.693)%
  --(10.736,1.693)--(10.736,1.692)--(10.737,1.692)--(10.738,1.692)--(10.739,1.692)--(10.740,1.692)%
  --(10.741,1.692)--(10.742,1.692)--(10.743,1.692)--(10.744,1.692)--(10.745,1.692)--(10.746,1.692)%
  --(10.747,1.692)--(10.748,1.692)--(10.749,1.692)--(10.749,1.691)--(10.750,1.691)--(10.751,1.691)%
  --(10.752,1.691)--(10.753,1.691)--(10.754,1.691)--(10.755,1.691)--(10.756,1.691)--(10.757,1.691)%
  --(10.758,1.691)--(10.759,1.691)--(10.760,1.691)--(10.761,1.691)--(10.762,1.691)--(10.763,1.691)%
  --(10.763,1.690)--(10.764,1.690)--(10.765,1.690)--(10.766,1.690)--(10.767,1.690)--(10.768,1.690)%
  --(10.769,1.690)--(10.770,1.690)--(10.771,1.690)--(10.772,1.690)--(10.773,1.690)--(10.774,1.690)%
  --(10.775,1.690)--(10.776,1.690)--(10.777,1.690)--(10.777,1.689)--(10.778,1.689)--(10.779,1.689)%
  --(10.780,1.689)--(10.781,1.689)--(10.782,1.689)--(10.783,1.689)--(10.784,1.689)--(10.785,1.689)%
  --(10.786,1.689)--(10.787,1.689)--(10.788,1.689)--(10.789,1.689)--(10.790,1.689)--(10.790,1.688)%
  --(10.791,1.688)--(10.792,1.688)--(10.793,1.688)--(10.794,1.688)--(10.795,1.688)--(10.796,1.688)%
  --(10.797,1.688)--(10.798,1.688)--(10.799,1.688)--(10.800,1.688)--(10.801,1.688)--(10.802,1.688)%
  --(10.803,1.688)--(10.804,1.688)--(10.804,1.687)--(10.805,1.687)--(10.806,1.687)--(10.807,1.687)%
  --(10.808,1.687)--(10.809,1.687)--(10.810,1.687)--(10.811,1.687)--(10.812,1.687)--(10.813,1.687)%
  --(10.814,1.687)--(10.815,1.687)--(10.816,1.687)--(10.817,1.687)--(10.817,1.686)--(10.818,1.686)%
  --(10.819,1.686)--(10.820,1.686)--(10.821,1.686)--(10.822,1.686)--(10.823,1.686)--(10.824,1.686)%
  --(10.825,1.686)--(10.826,1.686)--(10.827,1.686)--(10.828,1.686)--(10.829,1.686)--(10.830,1.686)%
  --(10.831,1.686)--(10.831,1.685)--(10.832,1.685)--(10.833,1.685)--(10.834,1.685)--(10.835,1.685)%
  --(10.836,1.685)--(10.837,1.685)--(10.838,1.685)--(10.839,1.685)--(10.840,1.685)--(10.841,1.685)%
  --(10.842,1.685)--(10.843,1.685)--(10.844,1.685)--(10.845,1.685)--(10.845,1.684)--(10.846,1.684)%
  --(10.847,1.684)--(10.848,1.684)--(10.849,1.684)--(10.850,1.684)--(10.851,1.684)--(10.852,1.684)%
  --(10.853,1.684)--(10.854,1.684)--(10.855,1.684)--(10.856,1.684)--(10.857,1.684)--(10.858,1.684)%
  --(10.859,1.684)--(10.859,1.683)--(10.860,1.683)--(10.861,1.683)--(10.862,1.683)--(10.863,1.683)%
  --(10.864,1.683)--(10.865,1.683)--(10.866,1.683)--(10.867,1.683)--(10.868,1.683)--(10.869,1.683)%
  --(10.870,1.683)--(10.871,1.683)--(10.872,1.683)--(10.872,1.682)--(10.873,1.682)--(10.874,1.682)%
  --(10.875,1.682)--(10.876,1.682)--(10.877,1.682)--(10.878,1.682)--(10.879,1.682)--(10.880,1.682)%
  --(10.881,1.682)--(10.882,1.682)--(10.883,1.682)--(10.884,1.682)--(10.885,1.682)--(10.886,1.682)%
  --(10.886,1.681)--(10.887,1.681)--(10.888,1.681)--(10.889,1.681)--(10.890,1.681)--(10.891,1.681)%
  --(10.892,1.681)--(10.893,1.681)--(10.894,1.681)--(10.895,1.681)--(10.896,1.681)--(10.897,1.681)%
  --(10.898,1.681)--(10.899,1.681)--(10.900,1.681)--(10.900,1.680)--(10.901,1.680)--(10.902,1.680)%
  --(10.903,1.680)--(10.904,1.680)--(10.905,1.680)--(10.906,1.680)--(10.907,1.680)--(10.908,1.680)%
  --(10.909,1.680)--(10.910,1.680)--(10.911,1.680)--(10.912,1.680)--(10.913,1.680)--(10.914,1.680)%
  --(10.914,1.679)--(10.915,1.679)--(10.916,1.679)--(10.917,1.679)--(10.918,1.679)--(10.919,1.679)%
  --(10.920,1.679)--(10.921,1.679)--(10.922,1.679)--(10.923,1.679)--(10.924,1.679)--(10.925,1.679)%
  --(10.926,1.679)--(10.927,1.679)--(10.928,1.679)--(10.928,1.678)--(10.929,1.678)--(10.930,1.678)%
  --(10.931,1.678)--(10.932,1.678)--(10.933,1.678)--(10.934,1.678)--(10.935,1.678)--(10.936,1.678)%
  --(10.937,1.678)--(10.938,1.678)--(10.939,1.678)--(10.940,1.678)--(10.941,1.678)--(10.942,1.678)%
  --(10.942,1.677)--(10.943,1.677)--(10.944,1.677)--(10.945,1.677)--(10.946,1.677)--(10.947,1.677)%
  --(10.948,1.677)--(10.949,1.677)--(10.950,1.677)--(10.951,1.677)--(10.952,1.677)--(10.953,1.677)%
  --(10.954,1.677)--(10.955,1.677)--(10.956,1.677)--(10.956,1.676)--(10.957,1.676)--(10.958,1.676)%
  --(10.959,1.676)--(10.960,1.676)--(10.961,1.676)--(10.962,1.676)--(10.963,1.676)--(10.964,1.676)%
  --(10.965,1.676)--(10.966,1.676)--(10.967,1.676)--(10.968,1.676)--(10.969,1.676)--(10.970,1.676)%
  --(10.970,1.675)--(10.971,1.675)--(10.972,1.675)--(10.973,1.675)--(10.974,1.675)--(10.975,1.675)%
  --(10.976,1.675)--(10.977,1.675)--(10.978,1.675)--(10.979,1.675)--(10.980,1.675)--(10.981,1.675)%
  --(10.982,1.675)--(10.983,1.675)--(10.984,1.675)--(10.984,1.674)--(10.985,1.674)--(10.986,1.674)%
  --(10.987,1.674)--(10.988,1.674)--(10.989,1.674)--(10.990,1.674)--(10.991,1.674)--(10.992,1.674)%
  --(10.993,1.674)--(10.994,1.674)--(10.995,1.674)--(10.996,1.674)--(10.997,1.674)--(10.998,1.674)%
  --(10.998,1.673)--(10.999,1.673)--(11.000,1.673)--(11.001,1.673)--(11.002,1.673)--(11.003,1.673)%
  --(11.004,1.673)--(11.005,1.673)--(11.006,1.673)--(11.007,1.673)--(11.008,1.673)--(11.009,1.673)%
  --(11.010,1.673)--(11.011,1.673)--(11.012,1.673)--(11.013,1.673)--(11.013,1.672)--(11.014,1.672)%
  --(11.015,1.672)--(11.016,1.672)--(11.017,1.672)--(11.018,1.672)--(11.019,1.672)--(11.020,1.672)%
  --(11.021,1.672)--(11.022,1.672)--(11.023,1.672)--(11.024,1.672)--(11.025,1.672)--(11.026,1.672)%
  --(11.027,1.672)--(11.027,1.671)--(11.028,1.671)--(11.029,1.671)--(11.030,1.671)--(11.031,1.671)%
  --(11.032,1.671)--(11.033,1.671)--(11.034,1.671)--(11.035,1.671)--(11.036,1.671)--(11.037,1.671)%
  --(11.038,1.671)--(11.039,1.671)--(11.040,1.671)--(11.041,1.671)--(11.041,1.670)--(11.042,1.670)%
  --(11.043,1.670)--(11.044,1.670)--(11.045,1.670)--(11.046,1.670)--(11.047,1.670)--(11.048,1.670)%
  --(11.049,1.670)--(11.050,1.670)--(11.051,1.670)--(11.052,1.670)--(11.053,1.670)--(11.054,1.670)%
  --(11.055,1.670)--(11.055,1.669)--(11.056,1.669)--(11.057,1.669)--(11.058,1.669)--(11.059,1.669)%
  --(11.060,1.669)--(11.061,1.669)--(11.062,1.669)--(11.063,1.669)--(11.064,1.669)--(11.065,1.669)%
  --(11.066,1.669)--(11.067,1.669)--(11.068,1.669)--(11.069,1.669)--(11.070,1.669)--(11.070,1.668)%
  --(11.071,1.668)--(11.072,1.668)--(11.073,1.668)--(11.074,1.668)--(11.075,1.668)--(11.076,1.668)%
  --(11.077,1.668)--(11.078,1.668)--(11.079,1.668)--(11.080,1.668)--(11.081,1.668)--(11.082,1.668)%
  --(11.083,1.668)--(11.084,1.668)--(11.084,1.667)--(11.085,1.667)--(11.086,1.667)--(11.087,1.667)%
  --(11.088,1.667)--(11.089,1.667)--(11.090,1.667)--(11.091,1.667)--(11.092,1.667)--(11.093,1.667)%
  --(11.094,1.667)--(11.095,1.667)--(11.096,1.667)--(11.097,1.667)--(11.098,1.667)--(11.099,1.667)%
  --(11.099,1.666)--(11.100,1.666)--(11.101,1.666)--(11.102,1.666)--(11.103,1.666)--(11.104,1.666)%
  --(11.105,1.666)--(11.106,1.666)--(11.107,1.666)--(11.108,1.666)--(11.109,1.666)--(11.110,1.666)%
  --(11.111,1.666)--(11.112,1.666)--(11.113,1.666)--(11.113,1.665)--(11.114,1.665)--(11.115,1.665)%
  --(11.116,1.665)--(11.117,1.665)--(11.118,1.665)--(11.119,1.665)--(11.120,1.665)--(11.121,1.665)%
  --(11.122,1.665)--(11.123,1.665)--(11.124,1.665)--(11.125,1.665)--(11.126,1.665)--(11.127,1.665)%
  --(11.128,1.665)--(11.128,1.664)--(11.129,1.664)--(11.130,1.664)--(11.131,1.664)--(11.132,1.664)%
  --(11.133,1.664)--(11.134,1.664)--(11.135,1.664)--(11.136,1.664)--(11.137,1.664)--(11.138,1.664)%
  --(11.139,1.664)--(11.140,1.664)--(11.141,1.664)--(11.142,1.664)--(11.142,1.663)--(11.143,1.663)%
  --(11.144,1.663)--(11.145,1.663)--(11.146,1.663)--(11.147,1.663)--(11.148,1.663)--(11.149,1.663)%
  --(11.150,1.663)--(11.151,1.663)--(11.152,1.663)--(11.153,1.663)--(11.154,1.663)--(11.155,1.663)%
  --(11.156,1.663)--(11.157,1.663)--(11.157,1.662)--(11.158,1.662)--(11.159,1.662)--(11.160,1.662)%
  --(11.161,1.662)--(11.162,1.662)--(11.163,1.662)--(11.164,1.662)--(11.165,1.662)--(11.166,1.662)%
  --(11.167,1.662)--(11.168,1.662)--(11.169,1.662)--(11.170,1.662)--(11.171,1.662)--(11.172,1.662)%
  --(11.172,1.661)--(11.173,1.661)--(11.174,1.661)--(11.175,1.661)--(11.176,1.661)--(11.177,1.661)%
  --(11.178,1.661)--(11.179,1.661)--(11.180,1.661)--(11.181,1.661)--(11.182,1.661)--(11.183,1.661)%
  --(11.184,1.661)--(11.185,1.661)--(11.186,1.661)--(11.186,1.660)--(11.187,1.660)--(11.188,1.660)%
  --(11.189,1.660)--(11.190,1.660)--(11.191,1.660)--(11.192,1.660)--(11.193,1.660)--(11.194,1.660)%
  --(11.195,1.660)--(11.196,1.660)--(11.197,1.660)--(11.198,1.660)--(11.199,1.660)--(11.200,1.660)%
  --(11.201,1.660)--(11.201,1.659)--(11.202,1.659)--(11.203,1.659)--(11.204,1.659)--(11.205,1.659)%
  --(11.206,1.659)--(11.207,1.659)--(11.208,1.659)--(11.209,1.659)--(11.210,1.659)--(11.211,1.659)%
  --(11.212,1.659)--(11.213,1.659)--(11.214,1.659)--(11.215,1.659)--(11.216,1.659)--(11.216,1.658)%
  --(11.217,1.658)--(11.218,1.658)--(11.219,1.658)--(11.220,1.658)--(11.221,1.658)--(11.222,1.658)%
  --(11.223,1.658)--(11.224,1.658)--(11.225,1.658)--(11.226,1.658)--(11.227,1.658)--(11.228,1.658)%
  --(11.229,1.658)--(11.230,1.658)--(11.231,1.658)--(11.231,1.657)--(11.232,1.657)--(11.233,1.657)%
  --(11.234,1.657)--(11.235,1.657)--(11.236,1.657)--(11.237,1.657)--(11.238,1.657)--(11.239,1.657)%
  --(11.240,1.657)--(11.241,1.657)--(11.242,1.657)--(11.243,1.657)--(11.244,1.657)--(11.245,1.657)%
  --(11.246,1.657)--(11.246,1.656)--(11.247,1.656)--(11.248,1.656)--(11.249,1.656)--(11.250,1.656)%
  --(11.251,1.656)--(11.252,1.656)--(11.253,1.656)--(11.254,1.656)--(11.255,1.656)--(11.256,1.656)%
  --(11.257,1.656)--(11.258,1.656)--(11.259,1.656)--(11.260,1.656)--(11.261,1.656)--(11.261,1.655)%
  --(11.262,1.655)--(11.263,1.655)--(11.264,1.655)--(11.265,1.655)--(11.266,1.655)--(11.267,1.655)%
  --(11.268,1.655)--(11.269,1.655)--(11.270,1.655)--(11.271,1.655)--(11.272,1.655)--(11.273,1.655)%
  --(11.274,1.655)--(11.275,1.655)--(11.276,1.655)--(11.276,1.654)--(11.277,1.654)--(11.278,1.654)%
  --(11.279,1.654)--(11.280,1.654)--(11.281,1.654)--(11.282,1.654)--(11.283,1.654)--(11.284,1.654)%
  --(11.285,1.654)--(11.286,1.654)--(11.287,1.654)--(11.288,1.654)--(11.289,1.654)--(11.290,1.654)%
  --(11.291,1.654)--(11.291,1.653)--(11.292,1.653)--(11.293,1.653)--(11.294,1.653)--(11.295,1.653)%
  --(11.296,1.653)--(11.297,1.653)--(11.298,1.653)--(11.299,1.653)--(11.300,1.653)--(11.301,1.653)%
  --(11.302,1.653)--(11.303,1.653)--(11.304,1.653)--(11.305,1.653)--(11.306,1.653)--(11.306,1.652)%
  --(11.307,1.652)--(11.308,1.652)--(11.309,1.652)--(11.310,1.652)--(11.311,1.652)--(11.312,1.652)%
  --(11.313,1.652)--(11.314,1.652)--(11.315,1.652)--(11.316,1.652)--(11.317,1.652)--(11.318,1.652)%
  --(11.319,1.652)--(11.320,1.652)--(11.321,1.652)--(11.321,1.651)--(11.322,1.651)--(11.323,1.651)%
  --(11.324,1.651)--(11.325,1.651)--(11.326,1.651)--(11.327,1.651)--(11.328,1.651)--(11.329,1.651)%
  --(11.330,1.651)--(11.331,1.651)--(11.332,1.651)--(11.333,1.651)--(11.334,1.651)--(11.335,1.651)%
  --(11.336,1.651)--(11.336,1.650)--(11.337,1.650)--(11.338,1.650)--(11.339,1.650)--(11.340,1.650)%
  --(11.341,1.650)--(11.342,1.650)--(11.343,1.650)--(11.344,1.650)--(11.345,1.650)--(11.346,1.650)%
  --(11.347,1.650)--(11.348,1.650)--(11.349,1.650)--(11.350,1.650)--(11.351,1.650)--(11.351,1.649)%
  --(11.352,1.649)--(11.353,1.649)--(11.354,1.649)--(11.355,1.649)--(11.356,1.649)--(11.357,1.649)%
  --(11.358,1.649)--(11.359,1.649)--(11.360,1.649)--(11.361,1.649)--(11.362,1.649)--(11.363,1.649)%
  --(11.364,1.649)--(11.365,1.649)--(11.366,1.649)--(11.366,1.648)--(11.367,1.648)--(11.368,1.648)%
  --(11.369,1.648)--(11.370,1.648)--(11.371,1.648)--(11.372,1.648)--(11.373,1.648)--(11.374,1.648)%
  --(11.375,1.648)--(11.376,1.648)--(11.377,1.648)--(11.378,1.648)--(11.379,1.648)--(11.380,1.648)%
  --(11.381,1.648)--(11.382,1.648)--(11.382,1.647)--(11.383,1.647)--(11.384,1.647)--(11.385,1.647)%
  --(11.386,1.647)--(11.387,1.647)--(11.388,1.647)--(11.389,1.647)--(11.390,1.647)--(11.391,1.647)%
  --(11.392,1.647)--(11.393,1.647)--(11.394,1.647)--(11.395,1.647)--(11.396,1.647)--(11.397,1.647)%
  --(11.397,1.646)--(11.398,1.646)--(11.399,1.646)--(11.400,1.646)--(11.401,1.646)--(11.402,1.646)%
  --(11.403,1.646)--(11.404,1.646)--(11.405,1.646)--(11.406,1.646)--(11.407,1.646)--(11.408,1.646)%
  --(11.409,1.646)--(11.410,1.646)--(11.411,1.646)--(11.412,1.646)--(11.412,1.645)--(11.413,1.645)%
  --(11.414,1.645)--(11.415,1.645)--(11.416,1.645)--(11.417,1.645)--(11.418,1.645)--(11.419,1.645)%
  --(11.420,1.645)--(11.421,1.645)--(11.422,1.645)--(11.423,1.645)--(11.424,1.645)--(11.425,1.645)%
  --(11.426,1.645)--(11.427,1.645)--(11.428,1.645)--(11.428,1.644)--(11.429,1.644)--(11.430,1.644)%
  --(11.431,1.644)--(11.432,1.644)--(11.433,1.644)--(11.434,1.644)--(11.435,1.644)--(11.436,1.644)%
  --(11.437,1.644)--(11.438,1.644)--(11.439,1.644)--(11.440,1.644)--(11.441,1.644)--(11.442,1.644)%
  --(11.443,1.644)--(11.443,1.643)--(11.444,1.643)--(11.445,1.643)--(11.446,1.643)--(11.447,1.643);
\gpcolor{color=gp lt color border}
\draw[gp path] (1.320,7.691)--(1.320,0.985)--(11.447,0.985)--(11.447,7.691)--cycle;
%% coordinates of the plot area
\gpdefrectangularnode{gp plot 1}{\pgfpoint{1.320cm}{0.985cm}}{\pgfpoint{11.447cm}{7.691cm}}
\end{tikzpicture}

\caption{$\langle M\rangle$の挙動}
\label{fig:M}
\end{center}
\end{figure}
\end{document}
