\documentclass[uplatex,a4j,11pt,dvipdfmx]{jsarticle}
\bibliographystyle{jplain}

\usepackage{url}

\usepackage{graphicx}
\usepackage{gnuplot-lua-tikz}
\usepackage{pgfplots}
\usepackage{tikz}
\usepackage{amsmath,amsfonts,amssymb}
\usepackage{bm}
\usepackage{siunitx}

\makeatletter
\def\fgcaption{\def\@captype{figure}\caption}
\makeatother
\newcommand{\setsections}[3]{
\setcounter{section}{#1}
\setcounter{subsection}{#2}
\setcounter{subsubsection}{#3}
}
\newcommand{\mfig}[3][width=15cm]{
\begin{center}
\includegraphics[#1]{#2}
\fgcaption{#3 \label{fig:#2}}
\end{center}
}
\newcommand{\gnu}[2]{
\begin{figure}[hptb]
\begin{center}
\input{#2}
\caption{#1}
\label{fig:#2}
\end{center}
\end{figure}
}

\begin{document}
\title{}
\author{61908697 佐々木良輔}
\date{}
\maketitle
\subsection*{3.16}
\subsubsection*{(1)}
分散関係より
\begin{align}
  \begin{split}
    k&=\frac{\pi}{L}n=\frac{\omega_{\bm{k}}}{c}\qquad
    \therefore n=\frac{L\omega_{\bm{k}}}{\pi c}
  \end{split}
\end{align}
ここで状態が密に存在しているならば振動数が$\omega$から$\omega+{\rm d}\omega$に含まれる状態数$D(\omega){\rm d}\omega$は$|\bm{n}|=L\omega/2\pi c$で厚さ${\rm d}\omega$の球殻の第一象限の体積に等しく,また偏光として2自由度があることを考えると
\begin{align}
  \begin{split}
    D(\omega){\rm d}\omega&=4\pi\left(\frac{L\omega}{\pi c}\right)^2\times\frac{1}{8}\times2\times{\rm d}n\\
    &=\frac{L^3\omega^2}{\pi^2c^3}{\rm d}\omega
  \end{split}
\end{align}
$L^3=V$より
\begin{align}
  D(\omega)=\frac{V\omega^2}{\pi^2c^3}
\end{align}
\subsubsection*{(2)}
\begin{align}
  \begin{split}
    Z_{\bm{k},\lambda}&=\sum_{n_{\bm{k},\lambda}=0}^{\infty}{\rm e}^{-n_{\bm{k},\lambda}\hbar\omega_{\bm{k}}/k_BT}\\
    &=\frac{1}{1-{\rm e}^{-\hbar\omega_{\bm{k}}/k_BT}}
  \end{split}
\end{align}
1行目から2行目の変形で無限等比級数の和の公式を用いた.
\subsubsection*{(3)}
\begin{align}
  \begin{split}
    \langle n_{\bm{k},\lambda}\rangle&=\frac{1}{Z}\sum_{\nu}n_{\bm{k},\lambda}{\rm e}^{-\beta E_\nu}\\
    &=\sum_{\bm{k},\lambda}\frac{\sum_{n_{\bm{k},\lambda}}n_{\bm{k},\lambda}{\rm e}^{-\beta n_{\bm{k},\lambda}\hbar\omega_{\bm{k}}}}{\sum_{n_{\bm{k},\lambda}}{\rm e}^{-\beta n_{\bm{k},\lambda}\hbar\omega_{\bm{k}}}}\\
    &=-\frac{1}{\beta}\frac{\partial}{\partial(\hbar\omega_{\bm{k}})}\log Z
  \end{split}
\end{align}
となるので$\varepsilon_{\bm{k}}=\hbar\omega_{\bm{k}}$と置くと
\begin{align}
  \begin{split}
    \langle n_{{\bm{k}},\lambda}\rangle&=-\frac{1}{\beta}\frac{\partial}{\partial \varepsilon_k}\log\prod_{{\bm{k}},\lambda}\frac{1}{1-{\rm e}^{-\beta\varepsilon_{\bm{k}}}}\\
    &=\frac{1}{\beta}\frac{\partial}{\partial\varepsilon_{\bm{k}}}\sum_{\bm{k}}\log(1-{\rm e}^{-\beta\varepsilon_{\bm{k}}})\\
    &=\frac{1}{{\rm e}^{\hbar\omega_{\bm{k}}/k_BT}-1}
  \end{split}
\end{align}
となる.また波数${\bm{k}}$,偏光$\lambda$の光のエネルギーの期待値は
\begin{align}
  \begin{split}
    \langle E_{{\bm{k}},\lambda}\rangle&=\langle n_{{\bm{k}},\lambda}\hbar\omega_{\bm{k}}\rangle\\
    &=\hbar\omega_{\bm{k}}\langle n_{{\bm{k}},\lambda}\rangle\\
    &=\frac{\hbar\omega_{\bm{k}}}{{\rm e}^{\hbar\omega_{\bm{k}}/k_BT}-1}
  \end{split}
\end{align}
\subsubsection*{(4)}
振動数が$\omega$から$\omega+{\rm d}\omega$の振動子の状態数は(3)から
\begin{align}
  D(\omega){\rm d}\omega=\frac{V\omega^2}{\pi^2c^3}{\rm d}\omega
\end{align}
また振動数$\omega$の光のエネルギー期待値は(7)から
\begin{align}
  \frac{\hbar\omega}{{\rm e}^{\hbar\omega/k_BT}-1}
\end{align}
である.振動数が$\omega$から$\omega+{\rm d}\omega$の範囲でこの期待値が変化しないとすれば,
振動数が$\omega$から$\omega+{\rm d}\omega$の電磁波の単位体積あたりのエネルギー$\epsilon(\omega){\rm d}\omega$は
\begin{align}
  \begin{split}
    \epsilon(\omega){\rm d}\omega&=\frac{V\omega^2}{\pi^2c^3}{\rm d}\omega\frac{\hbar\omega}{{\rm e}^{\hbar\omega/k_BT}-1}\times\frac{1}{V}\\
    &=\frac{\hbar}{\pi^2c^3}\frac{\omega^3}{{\rm e}^{\hbar\omega/k_BT}-1}{\rm d}\omega
  \end{split}
\end{align}
\subsubsection*{(5)}
\begin{align}
  \epsilon(\omega)=\frac{\hbar}{\pi^2c^3}\frac{\omega^3}{{\rm e}^{\hbar\omega/k_BT}-1}{\rm d}\omega=:\frac{\hbar}{\pi^2c^3}f(\omega)
\end{align}
とする.ここで以下を考える.
\begin{align}
  \frac{\rm d}{{\rm d}\omega}f(\omega)=\frac{\omega^2\left({\rm e}^{\beta\hbar\omega}(3-\beta\hbar\omega)-3\right)}{({\rm e}^{\beta\hbar\omega}-1)^2}=0
\end{align}
この$\omega\neq0$かつ$\omega<\infty$なる解は
\begin{align}
  {\rm e}^{\beta\hbar\omega}(3-\beta\hbar\omega)-3=0
\end{align}
を満たす.ここで$x=\beta\hbar\omega$を用いて
\begin{align}
  g(x):=e^x(3-x)-3
\end{align}
とする.ここで$g(0)=0$, $\lim_{x\rightarrow\infty}g(x)=-\infty$である.
\begin{align}
  \frac{\rm d}{{\rm d}x}g(x)=e^x(2-x)
\end{align}
これは$0\leq x\leq 2$で正, $x>2$で負である.したがって$g(x)$の増減表は表\ref{tab:zougen}のようになる.
\begin{table}[h]
\caption{$g(x)$の増減表}
\label{tab:zougen}
\centering
\begin{tabular}{c|cccccc}
\hline
$x$&0&$\cdots$&2&$\cdots$&$x_0$&$\infty$\\
\hline \hline
$g(x)$&0&$\nearrow$&&$\searrow$&0&$-\infty$\\
$g'(x)$&\multicolumn{2}{c}{$+$}&0&\multicolumn{3}{c}{$-$}\\
\hline
\end{tabular}
\end{table}\\
このことから$g(x)=0$は$x_0$にて唯一の解を持つ.すなわち$f(x)$の極値の数は1つである.
ここで$f(0)=0$, $\lim_{x\rightarrow\infty}f(x)=0$であり,更に$x\geq0$において$f(x)\geq0$である.
したがって$x\geq0$において$f(x)$の唯一の極値は最大値である.
ここで
\begin{align}
  w{\rm e}^w=-\frac{3}{e^3}
\end{align}
なる$w$を用いると$x_0=w+3$は
\begin{align}
  \begin{split}
    g(x_0)&=3{\rm e}^{w+3}-(w+3){\rm e}^{w+3}-3\\
    &=3{\rm e}^{w+3}-3{\rm e}^{w+3}-{\rm e}^3w{\rm e}^w-3=0
  \end{split}
\end{align}
となるので$x_0$は$g(x)$の解である.
また$x=\beta\hbar\omega$としていたので$f(\omega)=0$の解$\omega_0$は
\begin{align}
  \begin{split}
    \omega_0=\frac{x_0}{\hbar\beta}=\frac{x_0k_BT}{\hbar}
  \end{split}
\end{align}
となる.したがって$\epsilon(\omega)$は$\omega_0$で最大値を取り,そのときの振動数は$T$に比例する.
\subsubsection*{(6)}
\begin{align}
  \begin{split}
    \frac{E}{V}&=\int^\infty_0\epsilon(\omega){\rm d}\omega\\
    &=\frac{\hbar}{\pi^2c^3}\int^\infty_0\frac{\omega^3}{{\rm e}^{\beta\hbar\omega}-1}{\rm d}\omega
  \end{split}
\end{align}
ここで$\omega=x/\beta\hbar$とすると
\begin{align}
  \begin{split}
    \frac{E}{V}&=\frac{\hbar}{\pi^2c^3}\frac{1}{\hbar^4\beta^4}\int^\infty_0\frac{x^3}{{\rm e}^x-1}{\rm d}x\\
    &=\frac{1}{\pi^2c^3\hbar^3\beta^4}\zeta(4)\Gamma(4)
  \end{split}
\end{align}
ここで$\zeta(4)=\pi^4/90$, $\Gamma(4)=6$より以下を得る.
\begin{align}
  \frac{E}{V}=\frac{\pi^2(k_BT)^4}{15c^3\hbar^3}
\end{align}
\subsubsection*{(7)}
自由エネルギー$F$は
\begin{align}
  F&=-\frac{1}{\beta}\log Z=\frac{1}{\beta}\sum_{\bm{k}}\log(1-{\rm e}^{-\beta\hbar\omega_{\bm{k}}})
\end{align}
ここで$\bm{k}$が十分に密だとして総和を積分に置換する.状態密度$D(\omega)$を用いて
\begin{align}
  \begin{split}
    F&=\frac{1}{\beta}\int_0^{\infty}\log(1-{\rm e}^{-\beta\hbar\omega})\frac{V\omega^2}{\pi^2c^3}{\rm d}\omega\\
    &=\frac{1}{\beta}\frac{V}{\pi^2c^3}\int_0^\infty\log(1-{\rm e}^{-\beta\hbar\omega})\omega^2{\rm d}\omega  
  \end{split}
\end{align}
$x=\beta\hbar\omega$とすると
\begin{align}
  \begin{split}
    F&=\frac{1}{\beta^4\hbar^3}\frac{V}{\pi^2c^3}\int_0^\infty\log(1-{\rm e}^{-x})x^2{\rm d}x\\
    &=\frac{1}{\beta^4\hbar^3}\frac{V}{\pi^2c^3}\left( \left[ \frac{x^3}{3}\log(1-{\rm e}^{-x}) \right]_0^\infty -\int_0^\infty\frac{x^3}{3}\frac{1}{e^x-1}{\rm d}x\right)\\
    &=-\frac{\pi^4}{45}\frac{1}{\beta^4\hbar^3}\frac{V}{\pi^2c^3}\\
    &=-\frac{\pi^2V(k_BT)^4}{45c^3\hbar^3}=-\frac{1}{3}E
  \end{split}
\end{align}
エントロピー$S$は
\begin{align}
  \begin{split}
    S&=-\frac{\partial F}{\partial T}\\
    &=\frac{4\pi^2Vk_B^4}{45c^3\hbar^3}T^3
  \end{split}
\end{align}
定積比熱$C_V$は
\begin{align}
  \begin{split}
    C_V&=\frac{\partial E}{\partial T}\\
    &=-3\frac{\partial F}{\partial T}\\
    &=-\frac{4\pi^2Vk_B^4}{15c^3\hbar^3}T^3
  \end{split}
\end{align}
圧力$P$は
\begin{align}
  \begin{split}
    P&=-\frac{F\partial F}{\partial V}\\
    &=-\frac{\pi^2(k_BT)^4}{45c^3\hbar^3}=\frac{1}{3}\frac{E}{V}
  \end{split}
\end{align}
となる.
\subsubsection*{(8)}
光子の総数$N$は
\begin{align}
  N=\sum_{\bm{k}}\langle n_{\bm{k},\lambda}\rangle
\end{align}
前問と同様に総和を積分に置換すると
\begin{align}
  \begin{split}
    N&=\int^\infty_0\frac{V\omega^2}{\pi^2c^3}\frac{1}{{\rm e}^{\beta\hbar\omega}-1}{\rm d}\omega\\
    &=\frac{V}{\pi^2c^3}\frac{1}{\beta^3\hbar^3}\int^\infty_0\frac{x^2}{{\rm e}^{x}-1}{\rm d}x\\
    &=\frac{V}{\pi^2c^3}\frac{k_B^3T^3}{\hbar^3}\Gamma(3)\zeta(3)\\
    &=\frac{2Vk_B^3\zeta(3)}{\pi^2c^3\hbar^3}T^3
  \end{split}
\end{align}
となり,光子の総数は$T^3$に比例する.
\subsubsection*{(9)}
\begin{align}
  \begin{split}
    \frac{N}{V}&=\frac{2k_B^3\zeta(3)}{\pi^2c^3\hbar^3}T^3\\
    &=\frac{2\times(1.381\times10^{-23})^3\times1.202}{3.142^2\times(2.998\times10^8)^3\times(1.055\times10^{-34})^3}300^3\\
    &=5.474\times10^{14}
  \end{split}
\end{align}
\end{document}