\documentclass[uplatex,a4j,11pt,dvipdfmx]{jsarticle}
\bibliographystyle{jplain}

\usepackage{url}

\usepackage{graphicx}
\usepackage{gnuplot-lua-tikz}
\usepackage{pgfplots}
\usepackage{tikz}
\usepackage{amsmath,amsfonts,amssymb}
\usepackage{bm}
\usepackage{siunitx}

\makeatletter
\def\fgcaption{\def\@captype{figure}\caption}
\makeatother
\newcommand{\setsections}[3]{
\setcounter{section}{#1}
\setcounter{subsection}{#2}
\setcounter{subsubsection}{#3}
}
\newcommand{\mfig}[3][width=15cm]{
\begin{center}
\includegraphics[#1]{#2}
\fgcaption{#3 \label{fig:#2}}
\end{center}
}
\newcommand{\gnu}[2]{
\begin{figure}[hptb]
\begin{center}
\input{#2}
\caption{#1}
\label{fig:#2}
\end{center}
\end{figure}
}

\begin{document}
\title{}
\author{61908697 佐々木良輔}
\date{}
\maketitle
\subsubsection*{問題}
全ての微視的状態に任意の個数の粒子が存在しうる場合(ボース分布)を考える.つまり$N_\lambda$個の同種
の玉を$M_\lambda$個の入れ物に入れるのと同じだけの重率がある場合,エントロピーが
\begin{align}
  S=-k_B\sum_j\left(f_j\log f_j-(1+f_j)\log(1+f_j)\right)
\end{align}
となることを示し,熱平衡状態で
\begin{align}
  f_j=\frac{1}{\exp\left(\frac{\epsilon_\lambda-\mu}{k_BT}-1\right)}
\end{align}
となることを示せ.\\
\hrulefill
\subsubsection*{回答}
準位$\lambda$に幾つでも粒子を入れることができるので,状態数は
\begin{align}
  W_\lambda={}_{M_\lambda+N_\lambda-1}{\rm C}_{N_\lambda}=\frac{(M_\lambda+N_\lambda-1)!}{N_\lambda!(M_\lambda-1)!}
\end{align}
したがってエントロピー$S$はStirlingの公式を用いて
\begin{align}
  \begin{split}
    S&=k_B\log\prod_\lambda W_\lambda\\
    &=k_B\sum_\lambda((M_\lambda+N_\lambda-1)\log(M_\lambda+N_\lambda-1)\\
    &\qquad-N_\lambda\log N_\lambda-(M_\lambda-1)\log(M_\lambda-1))\\
  \end{split}
\end{align}
ここで$N_\lambda\gg1$, $M_\lambda\gg1$なので
\begin{align}
  \begin{split}
    S&=k_B\sum_\lambda((M_\lambda+N_\lambda)\log(M_\lambda+N_\lambda)-N_\lambda\log N_\lambda-M_\lambda\log M_\lambda)\\
    &=-k_B\sum_\lambda M_\lambda\left(\frac{N_\lambda}{M_\lambda}\log\frac{N_\lambda}{M_\lambda}-\left(1+\frac{N_\lambda}{M_\lambda}\right)\log\left(1+\frac{N_\lambda}{M_\lambda}\right)\right)
  \end{split}
\end{align}
ここで準位$\lambda$には$M_\lambda$の状態があるので$\sum_\lambda M_\lambda=\sum_j$となる.したがって
\begin{align}
  \begin{split}
    S&=-k_B\sum_j\left(f_j\log f_j-(1+f_j)\log(1+f_j)\right)
  \end{split}
\end{align}
熱平衡状態ではエントロピーが最大になるので,その分布を求める.ここで全粒子数と全エネルギーは
\begin{align}
  N=\sum_\lambda N_\lambda=\sum_jf_j
\end{align}
\begin{align}
  E=\sum_\lambda N_\lambda\epsilon_\lambda=\sum_jf_j\epsilon_\lambda^{(j)}
\end{align}
で保存する.ここで$\epsilon_\lambda^{(j)}$は粒子$j$のエネルギーである.ラグランジュの未定乗数法を用いると
\begin{align}
  \tilde{S}=S-a\left(\sum_jf_j-N\right)-b\left(\sum_jf_j\epsilon_\lambda^{(j)}-E\right)
\end{align}
ここで両辺を$f_j$で偏微分すると
\begin{align}
  \begin{split}
    \tilde{S}=k_B\left(-\log f_j+\log(1+f_j)\right)-a-b\epsilon_\lambda^{(j)}=0\\
    \log\left(\frac{1+f_j}{f_j}\right)=\frac{a+b\epsilon_\lambda^{(j)}}{k_B}\\
    f_j=\frac{1}{\exp\left(\frac{a+b\epsilon_\lambda^{(j)}}{k_B}\right)-1}
  \end{split}
\end{align}
となる.ここで熱力学関係式から
\begin{align}
  \begin{split}
    \frac{1}{T}&=\frac{\partial S}{\partial E}=\sum_j\frac{\partial S}{\partial f_j}\frac{\partial f_j}{\partial E}\\
    &=\sum_j-k_B(\log f_j-\log(1+f_j))\frac{\partial f_j}{\partial E}\\
    &=\sum_jk_B\log\left(\frac{1+f_j}{f_j}\right)\frac{\partial f_j}{\partial E}\\
    &=\sum_j(a+b\epsilon_\lambda^{(j)})\frac{\partial f_j}{\partial E}\\
    &=a\frac{\partial}{\partial E}\left(\sum_jf_j\right)+b\frac{\partial}{\partial E}\left(\sum_jf_j\epsilon_j^{(j)}\right)\\
    &=a\frac{\partial N}{\partial E}+b\frac{\partial E}{\partial E}=b
  \end{split}
\end{align}
\clearpage
\begin{align}
  \begin{split}
    -\frac{\mu}{T}&=\frac{\partial S}{\partial N}=\sum_j\frac{\partial S}{\partial f_j}\frac{\partial f_j}{\partial N}\\
    &=a\frac{\partial N}{\partial N}+b\frac{\partial E}{\partial N}=a
  \end{split}
\end{align}
したがって(10)に代入すると
\begin{align}
  f_j=\frac{1}{\exp\left(\frac{\epsilon_\lambda-\mu}{k_BT}\right)-1}
\end{align}

\end{document}
