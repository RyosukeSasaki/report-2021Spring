\subsection*{(4)}
\subsubsection*{解}
\begin{align*}
  \Omega_1(E)&=\frac{E}{\hbar\omega}-\frac{1}{2}\\
  D_1(E)&=\frac{1}{\hbar\omega}
\end{align*}
\hrulefill
\subsubsection*{解説}
固有エネルギー$\varepsilon$が$E$以下となる状態数$\Omega_1(E)$は
\begin{align}
  \varepsilon=\left(n+\frac{1}{2}\right)\hbar\omega&\leq E\\
  n&\leq\frac{E}{\hbar\omega}-\frac{1}{2}
\end{align}
を満たす$n$の数である.
これは$n$空間において$\frac{E}{\hbar\omega}-\frac{1}{2}$以下の格子点の数と等しい.
したがって$\Omega_1(E)$は
\begin{align}
  \Omega_1(E)&=\left(\frac{E}{\hbar\omega}-\frac{1}{2}\right)\div 1\\
  &=\frac{E}{\hbar\omega}-\frac{1}{2}
\end{align}
また状態密度$D_1(E)$は
\begin{align}
  D_1(E)&=\frac{{\rm d}}{{\rm d}E}\Omega_1(E)\\
  &=\frac{1}{\hbar\omega}
\end{align}
となる.