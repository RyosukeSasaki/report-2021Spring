\subsection*{(2)}
\subsubsection*{解}
\begin{align*}
  \Omega(E)=\frac{L^3(2mE)^{\frac{3}{2}}}{3\pi^2\hbar^3}
\end{align*}
\hrulefill
\subsubsection*{解説}
固有エネルギー$\varepsilon$が$E$以下となる状態数$\Omega(E)$は
\begin{align}
  \varepsilon=\cfrac{2\pi^2\hbar^2}{mL^2}(n_x^2+x_y^2+n_z^2)&\leq E\\
  n_x^2+x_y^2+n_z^2&\leq E\frac{mL^2}{2\pi^2\hbar^2}
\end{align}
を満たす$n_x$, $n_y$, $n_z$の組の数である.
これは$\bm{n}=(n_x,n_y,n_z)$空間において半径$\frac{L}{\pi\hbar}\sqrt{\frac{mE}{2}}$の球に含まれる格子点の数と等しい.
$\bm{n}$空間には単位体積あたり$1$つの格子点が含まれるので,その数$\Omega(E)$は
\begin{align}
  \Omega(E)&=\frac{4}{3}\pi\left(\frac{L}{\pi\hbar}\sqrt{\frac{mE}{2}}\right)^3\div 1\\
  &=\frac{L^3(2mE)^{\frac{3}{2}}}{3\pi^2\hbar^3}
\end{align}