\subsection*{3-8}
\subsubsection*{(1)}
角振動数$\omega$の量子的調和振動子のエネルギーは$n\in\mathbb{N}$を用いて$\hbar\omega(n+1/2)$なので
\begin{align}
  Z_1&=\sum_n{\rm e}^{-\beta\hbar\omega(n+\frac{1}{2})}\nonumber\\
  &={\rm e}^{-\beta\hbar\omega/2}\sum_{n=0}^{\infty}{\rm e}^{-\beta\hbar\omega n}\nonumber\\
  &={\rm e}^{-\beta\hbar\omega/2}\frac{1}{1-{\rm e}^{-\beta\hbar\omega}}
\end{align}
\subsubsection*{(2)}
$\bm{n}=(n_x,n_y,n_z)$ ($n_i\in\mathbb{Z}$)を用いて
\begin{align}
  \bm{k}=\frac{2\pi}{L}\bm{n}
\end{align}
\subsubsection*{(3)}
$\omega=v|\bm{k}|$と(2)の結果から
\begin{align}
  |\bm{n}|=\frac{L\omega}{2\pi v}
\end{align}
したがって角振動数が$\omega$以下の状態数$N(\omega)$は$L$に対して格子間隔が十分小さい時半径$L\omega/2\pi v$の球の体積で近似できるので
\begin{align}
  N(\omega)&=\frac{4}{3}\pi\frac{L^3\omega^3}{8\pi^3v^3}\nonumber\\
  &=\frac{V\omega^3}{6\pi^2v^3}
\end{align}
状態密度$D_v(\omega)$は$N(\omega)$を微分し
\begin{align}
  D_v(\omega)&=\frac{{\rm d}}{{\rm d}\omega}N(\omega)\nonumber\\
  &=\frac{V\omega^2}{2\pi^2v^3}
\end{align}
\subsubsection*{(4)}
\begin{align}
  \int^{\omega_D}_0D(\omega){\rm d}\omega&=N(\omega_D)\nonumber\\
  3N&=\frac{V\omega_D^3}{6\pi^2}\left(\frac{1}{v_l^3}+\frac{1}{v_t^3}\right)
\end{align}
したがって
\begin{align}
  \frac{9N\omega^2}{\omega_D^3}&=\frac{9N\omega^2}{\cfrac{3N}{\cfrac{V}{6\pi^2}\left(\cfrac{1}{v_l^3}+\cfrac{1}{v_t^3}\right)}}\nonumber\\
  &=\frac{\omega^2V}{2\pi^2}\left(\frac{1}{v_l^3}+\frac{1}{v_t^3}\right)=D(\omega)
\end{align}
\subsubsection*{(5)}
\begin{align}
  F&=-k_BT\sum_k\log Z_1(\omega_k)\nonumber\\
  &=k_BT\sum_k\left(\frac{\hbar\omega_k}{2k_BT}+\log\left(1-{\rm e}^{-\hbar\omega_k/k_BT}\right)\right)
\end{align}
角振動数$\omega$から$\omega+{\rm d}\omega$の間にある状態数は$D(\omega){\rm d}\omega$なので$\omega_k$に関する和は積分に書き換えられ
\begin{align}
  F&=k_BT\int_0^\infty{\rm d}\omega\ D(\omega)\left(\frac{\hbar\omega}{2k_BT}+\log\left(1-{\rm e}^{-\hbar\omega/k_BT}\right)\right)\nonumber\\
  &=\frac{9Nk_BT}{\omega_D^3}\int_0^\infty{\rm d}\omega\ \omega^2\left(\frac{\hbar\omega}{2k_BT}+\log\left(1-{\rm e}^{-\hbar\omega/k_BT}\right)\right)
\end{align}
$\omega$の範囲は$0$から$\omega_D$で全ての状態数を網羅しているので
\begin{align}
  F=\frac{9Nk_BT}{\omega_D^3}\int_0^{\omega_D}{\rm d}\omega\ \omega^2\left(\frac{\hbar\omega}{2k_BT}+\log\left(1-{\rm e}^{-\hbar\omega/k_BT}\right)\right)
\end{align}
\subsubsection*{(6)}
エネルギー$E$は$E=\partial(\beta F)/\partial\beta$なので
\begin{align}
  E&=\frac{\partial}{\partial\beta}\frac{9N}{\omega_D^3}\int_0^{\omega_D}{\rm d}\omega\ \omega^2\left(\frac{\hbar\omega\beta}{2}+\log\left(1-{\rm e}^{-\hbar\omega\beta}\right)\right)\nonumber\\
  &=\frac{9N}{\omega_D^3}\int_0^{\omega_D}{\rm d}\omega\ \omega^2\left(\frac{\hbar\omega}{2}+\frac{\hbar\omega}{{\rm e}^{\beta\hbar\omega}-1}\right)
\end{align}
\subsubsection*{(7)}
\begin{align}
  C_V=\frac{\partial E}{\partial T}&=\frac{9N}{\omega_D^3}\int_0^{\omega_D}{\rm d}\omega\ \frac{\partial}{\partial T}\omega^2\left(\frac{\hbar\omega}{2}+\frac{\hbar\omega}{{\rm e}^{\beta\hbar\omega}-1}\right)\nonumber\\
  &=\frac{9N}{\omega_D^3}\int_0^{\omega_D}{\rm d}\omega\ \omega^2\frac{\hbar\omega}{k_BT^2}\frac{\hbar\omega{\rm e}^{\hbar\omega/k_BT}}{({\rm e}^{\hbar\omega/k_BT}-1)^2}
\end{align}
ここで$\hbar\omega/k_BT=t$, $\hbar\omega_D/k_B=\Theta_D$とすると
\begin{align}
  C_V&=\frac{9N}{\omega_D^3}\int_0^{\Theta_D/T}{\rm d}t\ k_B\frac{k_B^3T^3}{\hbar^3}t^4\frac{e^t}{(e^t-1)^2}\nonumber\\
  &=3Nk_B\frac{3T^3}{\Theta_D^3}\int_0^{\Theta_D/T}{\rm d}t\ \frac{t^4e^t}{(e^t-1)^2}\nonumber\\
  &=3Nk_Bf\left(\frac{\Theta_D}{T}\right)
\end{align}
ここで$f(x)=3/x^3\int_0^x{\rm d}t\ t^4e^t/(e^t-1)^2$とした.
\subsubsection*{(8)}
$T\gg\theta_D$のとき被積分項が$t^4e^t/(e^t-1)^2\sim t^2$とできることから
\begin{align}
  f(x)&\sim\frac{3}{x^3}\int_0^x{\rm d}t\ t^2\nonumber\\
  &=1
\end{align}
したがって
\begin{align}
  C_V=3Nk_B
\end{align}
\subsubsection*{(9)}
$T\ll\theta_D$のとき$x\rightarrow\infty$より積分公式
\begin{align}
  \int_0^x{\rm d}t\ \frac{t^4e^t}{(e^t-1)^2}\sim\frac{4\pi^4}{15}
\end{align}
から
\begin{align}
  f(x)&\sim 3Nk_B\frac{3T^3}{\Theta_D^3}\frac{4\pi^4}{15}
  \rightarrow 0
\end{align}
となる.
\subsubsection*{(10)}
上の結果から$C_V$は$T\rightarrow0$で$0$に, $T\rightarrow\infty$で$3Nk_B$に漸近するとわかる.図\ref{fig:desmos-graph.png}にその概形を示す.
\mfig[width=6cm]{desmos-graph.png}{$C_V$の概形(Desmos計算機にて描画)}