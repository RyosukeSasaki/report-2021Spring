\subsection*{(2)数理モデルの応用の提案}
私がアルバイトとして参加しているZip Infrastructure株式会社(\url{https://zip-infra.co.jp/#zippar})
では新たな公共交通として自走式のロープウェイを開発している.
自走式のロープウェイは従来のロープウェイと異なり運行間隔の調整や分岐による行き先の変更が可能であり,
都市型の交通手段としての実装を目指している.
更に都市型ロープウェイは図\ref{fig:fig/zippar.png}のように建物の内部に乗り入れることが可能であり,高層ビル間での移動手段としての利用も期待される.
これは階層間や離れたビル間を結ぶデッキと考えることができ,最適な駅の配置をヴェーバー問題として検討できると考えられる.
またビル間でのトラフィックを重力モデルを用いて予想することで,運行間隔の事前調整なども可能であると考えられる.
\mfig[width=8cm]{fig/zippar.png}{自走式ロープウェイの高層ビルへの乗り入れのイメージ(コンセプトムービーから引用)}