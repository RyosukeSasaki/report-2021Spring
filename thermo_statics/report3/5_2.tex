\clearpage
\subsection*{(2)}
微視的状態を$\nu$と,粒子$i$のエネルギーを$\epsilon_i=0,\varepsilon$とするとエネルギー$E_\nu$は
\begin{align}
  E_\nu=\sum_{i=1}^N\epsilon_i
\end{align}
したがって分配関数$Z$は
\begin{align}
  \begin{split}
    Z&=\sum_{\epsilon_1}\sum_{\epsilon_2}\cdots\sum_{\epsilon_N}{\rm e}^{-\beta\sum_i\epsilon_i}\\
    &=\sum_{\epsilon_1}{\rm e}^{-\beta\epsilon_1}\sum_{\epsilon_2}{\rm e}^{-\beta\epsilon_2}\cdots\sum_{\epsilon_N}{\rm e}^{-\beta\epsilon_N}\\
    &=(1+{\rm e}^{-\beta\varepsilon})^N
  \end{split}
\end{align}
また自由エネルギー$F$は
\begin{align}
  F&=-\frac{1}{\beta}\log Z\\
  &=-\frac{1}{\beta}N\log(1+{\rm e}^{-\beta\varepsilon})
\end{align}
ここで高分子の長さ$l$は微視的状態$\nu$においてエネルギー$\varepsilon$にある粒子の数を$n_\nu$とすると$l=an_\nu$である.
また$E_\nu=\varepsilon n_\nu$なので$l$の期待値は状態が$\nu$にある確率$p_\nu$を用いて
\begin{align}
  \begin{split}
    \langle l\rangle&=\sum_\nu p_\nu an_\nu\\
    &=\sum_\nu \frac{1}{Z}{\rm e}^{-\beta E_\nu}a\frac{E_\nu}{\varepsilon}\\
    &=\frac{a}{\varepsilon}\frac{\sum_\nu E_\nu{\rm e}^{-\beta E_\nu}}{Z}\\
    &=-\frac{a}{\varepsilon}\frac{1}{Z}\frac{\partial}{\partial\beta}\sum_\nu{\rm e}^{-\beta E_\nu}\\
    &=-\frac{a}{\varepsilon}\frac{1}{Z}\frac{\partial Z}{\partial\beta}\\
    &=\frac{a}{\varepsilon}\frac{\partial(\beta F)}{\partial\beta}\\
    &=\frac{Na}{1+{\rm e}^{-\beta\varepsilon}}=\frac{Na}{1+{\rm e}^{-\varepsilon/k_BT}}
  \end{split}
\end{align}