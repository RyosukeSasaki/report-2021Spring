\subsection*{6-1}
この問題での分配関数は$Z=(1+{\rm e}^{-\beta\varepsilon})^N$なので大分配関数$\Xi$は
\begin{align}
  \begin{split}
    \Xi&=\sum_{N=0}^{\infty}{\rm e}^{\beta\mu N}(1+{\rm e}^{\beta\varepsilon})^N\\
    &=\sum_{N=0}^{\infty}\left({\rm e}^{\beta\mu}(1+{\rm e}^{-\beta\varepsilon})\right)^N\\
    &=\frac{1}{1-{\rm e}^{\beta\mu}(1+{\rm e}^{-\beta\varepsilon})}
  \end{split}
\end{align}
したがってグランドポテンシャル$\Omega$は
\begin{align}
  \Omega=\frac{1}{\beta}\log(1-{\rm e}^{\beta\mu}(1+{\rm e}^{-\beta\varepsilon}))
\end{align}
粒子数の期待値$\langle N\rangle$は
\begin{align}
  \begin{split}
    \langle N\rangle&=-\frac{\partial\Omega}{\partial\mu}\\
    &=-\frac{1}{\beta}\frac{-\beta{\rm e}^{\beta\mu}(1+{\rm e}^{-\beta\varepsilon})}{1-{\rm e}^{\beta\mu}(1+{\rm e}^{-\beta\varepsilon})}\\
    &=\frac{{\rm e}^{\beta\mu}(1+{\rm e}^{-\beta\varepsilon})}{1-{\rm e}^{\beta\mu}(1+{\rm e}^{-\beta\varepsilon})}
  \end{split}
\end{align}
これを用いて化学ポテンシャル$\mu$は
\begin{align}
  \begin{split}
    \langle N\rangle(1-{\rm e}^{\beta\mu}(1+{\rm e}^{-\beta\varepsilon}))&={\rm e}^{\beta\mu}(1+{\rm e}^{-\beta\varepsilon})\\
    {\rm e}^{\beta\mu}&=\frac{\langle N\rangle}{1+\langle N\rangle}\frac{1}{1+{\rm e}^{-\beta\varepsilon}}
  \end{split}
\end{align}
ここで$\langle N\rangle\gg 1$とすると
\begin{align}
  \mu=-\frac{1}{\beta}\log(1+{\rm e}^{-\beta\varepsilon})
\end{align}
またグランドポテンシャルはルジャンドル変換
\begin{align}
  \Omega=F-\mu N
\end{align}
によって定義されるので,自由エネルギー$F$は
\begin{align}
  \begin{split}
    F&=\Omega+\mu N\\
    &=\frac{1}{\beta}\log(1-{\rm e}^{\beta\mu}(1+{\rm e}^{-\beta\varepsilon}))-\frac{1}{\beta}\log(1+{\rm e}^{-\beta\varepsilon})\langle N\rangle\\
    &=-\frac{1}{\beta}\left(\log\langle N\rangle-\log({\rm e}^{\beta\mu}(1+{\rm e}^{-\beta\varepsilon}))\right)-\frac{1}{\beta}\langle N\rangle\log(1+{\rm e}^{-\beta\varepsilon})\\
  \end{split}
\end{align}
ここで${\rm e}^{\beta\mu}(1+{\rm e}^{-\beta\varepsilon})=1$なので
\begin{align}
  F=-\frac{1}{\beta}\log\langle N\rangle-\frac{1}{\beta}\langle N\rangle\log(1+{\rm e}^{-\beta\varepsilon})
\end{align}
さらに$F$は示量性の変数なので$O(N)$以下の項を捨てると
\begin{align}
  F&=-\frac{1}{\beta}\langle N\rangle\log(1+{\rm e}^{-\beta\varepsilon})\\
  &=-k_BT\langle N\rangle\log(1+{\rm e}^{-\beta\varepsilon})
\end{align}
これはミクロカノニカル及ぼカノニカルアンサンブルで得られた結果と一致する.
\subsection*{6-2}
この問題での分配関数は$Z=1/N!(1/\hbar\omega\beta)^N$なので大分配関数$\Xi$は
\begin{align}
  \begin{split}
    \Xi&=\sum_{N=0}^\infty{\rm e}^{\beta\mu N}\frac{1}{N!}\left(\frac{1}{\hbar\omega\beta}\right)^N\\
    &=\exp\left(\frac{{\rm e}^{\beta\mu}}{\hbar\omega\beta}\right)
  \end{split}
\end{align}
ここでテイラー展開の表式を用いた.
したがってグランドポテンシャル$\Omega$は
\begin{align}
  \begin{split}
    \Omega&=-\frac{1}{\beta}\log\Xi\\
    &=-\frac{1}{\beta}\frac{e^{\beta\mu}}{\hbar\omega\beta}
  \end{split}
\end{align}
粒子数の期待値$\langle N\rangle$は
\begin{align}
  \langle N\rangle=-\frac{\partial\Xi}{\partial\mu}=\frac{{\rm e}^{\beta\mu}}{\hbar\omega\beta}
\end{align}
したがって化学ポテンシャル$\mu$は
\begin{align}
  \mu=\frac{1}{\beta}\log(\langle N\rangle\hbar\omega\beta)
\end{align}
以上から自由エネルギー$F$は
\begin{align}
  \begin{split}
    F&=\Omega+\mu N\\
    &=-\frac{1}{\beta}\frac{{\rm e}^{\beta\mu}}{\hbar\omega\beta}+\mu\frac{{\rm e}^{\beta\mu}}{\hbar\omega\beta}\\
    &=\frac{N}{\beta}(\log(N\hbar\omega\beta)-1)
  \end{split}
\end{align}
また同じ問題をカノニカルアンサンブルで考えると
\begin{align}
  \begin{split}
    F&=-\frac{1}{\beta}\log\frac{\left(\frac{1}{\hbar\omega\beta}\right)^N}{N!}\\
    &=-\frac{1}{\beta}(\log\frac{1}{\hbar\omega\beta}-N\log N-N)\\
    &=\frac{N}{\beta}(\log(N\hbar\omega\beta)-1)
  \end{split}
\end{align}
となり一致する.