\documentclass[uplatex,a4j,11pt,dvipdfmx]{jsarticle}
\bibliographystyle{jplain}

\usepackage{url}

\usepackage{graphicx}
\usepackage{amsmath,amsfonts,amssymb}
\usepackage{siunitx}

\usepackage{tcolorbox}
\newtcbox{\code}[1][]{
  colback=gray!10!white,
  colframe=gray!20!white,
  boxrule=1pt,
  left=0mm,right=0mm,top=0mm,bottom=0mm,
  box align=base,
  nobeforeafter,
  fontupper=\ttfamily
}

\makeatletter
\def\fgcaption{\def\@captype{figure}\caption}
\makeatother
\newcommand{\setsections}[3]{
\setcounter{section}{#1}
\setcounter{subsection}{#2}
\setcounter{subsubsection}{#3}
}
\newcommand{\mfig}[3][width=15cm]{
\begin{center}
\includegraphics[#1]{#2}
\fgcaption{#3 \label{fig:#2}}
\end{center}
}
\newcommand{\gnu}[2]{
\begin{figure}[hptb]
\begin{center}
\input{#2}
\caption{#1}
\label{fig:#2}
\end{center}
\end{figure}
}

\begin{document}
\title{}
\author{61908697 佐々木良輔}
\date{}
\maketitle
cpu kernelとgpu kernelについてそれぞれ5回時間を計測した.
ただしgpu kernelについてはReductionをホスト側で行う場合とデバイス側で行う場合とでそれぞれ5回時間を計測した.単位は全て${\rm ms}$である.
デバイスでReductionを行うときは\code{\_\_shfl\_down\_sync}を用いて同一Warp内の部分和を集計しそれをホスト側で再度集計している.

表\ref{tab:}からcpu kernelに比べてgpu kernelは$1/700$程度の時間で計算が実行できている.
またReductionをデバイス側で行った場合はそうでない場合に比べて1割程度高速化している.
\begin{table}[h]
\caption{}
\label{tab:}
\centering
\begin{tabular}{cccc}
\hline
No.&gpu kernel (Reduction in host)&gpu kernel (Reduction in device)&cpu kernel\\
\hline \hline
1&2.589760&2.370848&16721.085938\\
2&2.568512&2.399328&16708.503906\\
3&2.569280&2.369952&16679.330078\\
4&2.680256&2.368896&16724.927734\\
5&2.568864&2.384512&16635.873047\\
\hline
平均&2.595&2.379&$1.670\times10^4$\\
\hline
\end{tabular}
\end{table}
\end{document}