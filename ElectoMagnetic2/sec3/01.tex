\setcounter{subsection}{3}
\subsubsection{}
まず同次方程式の一般解を求める. $x(t)=e^{\lambda t}$と置くと
\begin{align}
  \begin{split}
    0&=m\ddot{x}(t)+2m\gamma\dot{x}(t)+m\omega_0^2x(t)\\
    0&=m\lambda^2+2m\gamma\lambda+m\omega_0^2\\
    \lambda&=-\gamma\pm\sqrt{\gamma^2-\omega_0^2}
  \end{split}
\end{align}
ここで$\gamma\geq\omega$のときは過減衰解, $\omega>\gamma$のときは減衰振動解である.ここでは減衰振動解を考える.
すなわち$\omega_1=\sqrt{\gamma^2-\omega_0^2}$を用いて
\begin{align}
  x(t)={\rm e}^{-\gamma}\left(C_1{\rm e}^{i\omega_1 t}+C_2{\rm e}^{-i\omega_1 t}\right)
\end{align}
解は実数に限られるので$C_1^*=C_2$である必要がある.新たに定数を$A,\alpha$と取り直せば
\begin{align}
  x(t)=A{\rm e}^{-\gamma}\sin(\omega_1 t+\alpha)
\end{align}
となる.次に非同次方程式の特解を$x(t)=C_1{\rm e}^{i\omega t}+C_2{\rm e}^{-i\omega t}$と仮定すると
\begin{align}
  \begin{split}
    eE_0\sin\omega t&=m\ddot{x}(t)+2m\gamma\dot{x}(t)+m\omega_0^2x(t)\\
    eE_0\sin\omega t&=-C_1m\omega^2{\rm e}^{i\omega t}+2iC_1m\gamma\omega{\rm e}^{i\omega t}+C_1m\omega_0^2{\rm e}^{i\omega t}\\
    &-C_2m\omega^2{\rm e}^{-i\omega t}-2iC_2m\gamma\omega{\rm e}^{-i\omega t}+C_2m\omega_0^2{\rm e}^{-i\omega t}\\
  \end{split}
\end{align}
これが恒等的に成り立つとき両辺の正弦,余弦成分が一致するので
\begin{align}
  \begin{cases}
    \begin{split}
      eE_0&=m\left(i(\omega_0^2-\omega^2)(C_1-C_2)-2\gamma\omega(C_1+C_2)\right)\\
      0&=m\left(i(\omega_0^2-\omega^2)(C_1+C_2)-2\gamma\omega(C_1-C_2)\right)\\
    \end{split}
  \end{cases}
\end{align}
これを連立すると$a=\omega_0^2-\omega^2,b=2\gamma\omega$を用いて
\begin{align}
  C_1=C_2^*=\frac{eE_0}{m}\frac{1}{2ia-2b}
\end{align}
したがって
\begin{align}
  \begin{split}
    x(t)&=\frac{eE_0}{m}\frac{a\sin\omega t+b\cos\omega t}{a^2+b^2}\\
    &=\frac{eE_0}{m}\frac{\sin(\omega t-\delta)}{\sqrt{(\omega_0-\omega^2)^2+4\gamma^2\omega^2}}
  \end{split}
\end{align}
ここで$\tan\delta=2\gamma\omega/(\omega_0^2-\omega^2)$である.
以上から一般解は以下のように成る.
\begin{align}
  x(t)=A{\rm e}^{-\gamma}\sin(\omega_1 t+\alpha)+\frac{eE_0}{m}\frac{\sin(\omega t-\delta)}{\sqrt{(\omega_0-\omega^2)^2+4\gamma^2\omega^2}}
\end{align}
\subsubsection{}
\begin{align*}
  r_e&=\frac{(1.602\times10^{-19})^2\ \si{\coulomb^2}}{4\pi(8.854\times10^{-12})(9.109\times10^{-31})(2.998\times10^8)^2\ \si{\coulomb.\volt^{-1}.\metre^{-1}.\kilo\gram.\metre^2.\second^{-2}}}\\
  &=2.817\times10^{-15}\ \si{\metre}
\end{align*}
\subsubsection{}
Rayleigh散乱を仮定すると散乱強度$i$は
\begin{align}
  \begin{split}
    i&\propto\frac{1}{\lambda^4}\\
    \therefore\ i&\lambda^4\propto 1
  \end{split}
\end{align}
となるので$i\lambda^4$は定数に成ると考えられる.図に波長$\lambda$と$i\lambda^4$の関係を示す.
このgnuplotのfit機能を用いて得た回帰曲線は$y=0.782x+2.71$となった.また相関係数は$0.202$となり,わずかに正の相関が見えるが相関関係はほぼ無いと考えられる.
したがって$i\lambda^4$はほぼ概ね定数であるといえ,地上での観測されたスペクトルにはRayleigh散乱の影響が大きいと考えられる.
\gnu{$\lambda$と$i\lambda^4$の関係}{graph/fig1.tex}
\subsubsection{}
\begin{align}
  \sigma_T=\frac{8\pi}{3}\left(\frac{1}{137}\right)^2\left(\frac{2.426\times10^{-12}}{2\pi}\right)^2=6.656\times10^{-29}\ \si{\metre^2}
\end{align}