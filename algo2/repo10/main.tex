\documentclass[uplatex,a4j,11pt,dvipdfmx]{jsarticle}
\bibliographystyle{jplain}

\usepackage{url}

\usepackage{graphicx}
\usepackage{gnuplot-lua-tikz}
\usepackage{pgfplots}
\usepackage{tikz}
\usepackage{amsmath,amsfonts,amssymb}
\usepackage{bm}
\usepackage{siunitx}
\usepackage{listings,jvlisting}

\makeatletter
\def\fgcaption{\def\@captype{figure}\caption}
\makeatother
\newcommand{\setsections}[3]{
\setcounter{section}{#1}
\setcounter{subsection}{#2}
\setcounter{subsubsection}{#3}
}
\newcommand{\mfig}[3][width=15cm]{
\begin{center}
\includegraphics[#1]{#2}
\fgcaption{#3 \label{fig:#2}}
\end{center}
}
\newcommand{\gnu}[2]{
\begin{figure}[hptb]
\begin{center}
\input{#2}
\caption{#1}
\label{fig:#2}
\end{center}
\end{figure}
}
\lstset{
  basicstyle={\ttfamily},
  identifierstyle={\small},
  commentstyle={\smallitshape},
  keywordstyle={\small\bfseries},
  ndkeywordstyle={\small},
  stringstyle={\small\ttfamily},
  frame={tb},
  breaklines=true,
  columns=[l]{fullflexible},
  numbers=left,
  xrightmargin=0zw,
  xleftmargin=3zw,
  numberstyle={\scriptsize},
  stepnumber=1,
  numbersep=1zw,
  lineskip=-0.5ex
}
%ここまでソースコードの表示に関する設定

\begin{document}
\title{アルゴリズム2B 第10回レポート}
\author{佐々木良輔}
\date{}
\maketitle
\subsubsection*{(1)}
今回入力ファイルとして,訪れたい温泉街のリストを作成した.表\ref{tab:nyuuryoku}にその内容を,またソースコード\ref{fuga}に入力ファイルを示す.
1列目に都市番号,
2列目に都市名,
3, 4列目にその温泉街を含む都道府県の県庁所在地の緯度,経度を示している.
\begin{table}[h]
\caption{入力リストの内容}
\label{tab:nyuuryoku}
\centering
\begin{tabular}{c|ccc}
\hline
都市番号&名称&緯度&経度\\
\hline \hline
0&箱根(神奈川)&35.4475073&139.6423446\\
1&草津(群馬)&36.3906675&139.0604061\\
2&熱海(静岡)&34.9771201&138.3830845\\
3&道後(愛媛)&33.8416238&132.7656808\\
4&松嶋(宮城)&38.2688373&140.8721\\
5&高山(岐阜)&35.3912272&136.7222906\\
6&伊勢島(三重)&34.7302829&136.5085883\\
7&加賀(石川)&36.5946816&136.6255726\\
8&宇奈月(富山)&36.6952907&137.2113383\\
9&有馬(兵庫)&34.6912688&135.1830706\\
\hline
\end{tabular}
\end{table}
\clearpage
\begin{lstlisting}[caption=input file,label=fuga]
  10
  0   Hakone      35.4475073  139.6423446
  1   Kusatu      36.3906675  139.0604061
  2   Atami       34.9771201  138.3830845
  3   Dougo       33.8416238  132.7656808
  4   Matsushima  38.2688373  140.8721
  5   Takayama    35.3912272  136.7222906
  6   Iseshima    34.7302829  136.5085883
  7   Kaga        36.5946816  136.6255726
  8   Unatsuki    36.6952907  137.2113383
  9   Arima       34.6912688  135.1830706
\end{lstlisting}
\subsubsection*{(2)}
以上の入力ファイルについて,
BF法及びBB法で最短コスト経路を計算すると
\begin{align}
  箱根→熱海→道後→有馬→伊勢島→高山→加賀→宇奈月→草津→松嶋→箱根
\end{align}
の経路で最もコストが低くなるとわかった.これは箱根を起点に関西まで行き,
その後北陸,東北を経て戻ってくるような一筆書きのルートになっており,尤もらしい結果である.
また,このときのコストはどちらのアルゴリズムでも25.87と算出された.

また計算量をEvaluatePath関数の呼び出し回数で近似すると,
BF法, BB法それぞれでの計算量は362880, 1388でありBB法は計算量が遥かに削減できているとわかる.
\end{document}