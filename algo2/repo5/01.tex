\subsection*{問1}
ここでは密度行列のDjikstra法とBellman-Ford法を比較した.
図\ref{fig:graph/dijkstra.tex}にDijkstra法でのノード数$N$とUpdateの回数を示す.
またそれぞれの系列について$y(x)=ax^n$という関数で最小二乗fittingを行った.
表\ref{tab:dijkstra_n}にDense graphの場合とSparse graphの場合で得られた$n$を示す.
したがってそれぞれの系列で計算量は$O(N^2)$程度になっていることがわかる.

また図\ref{fig:graph/bellman.tex}にBellman-Ford法でのノード数$N$とUpdateの回数を示す.
これについても同様に$y(x)=ax^n$でfittingを行い,その結果は表のようになった.
したがってDense graphでは概ね$O(N^2)$の計算量であるのに対し, Sparse graphでは計算量が減っていると考えられる.
\begin{table}[h]
\caption{Dijkstra法での各系列の次数}
\label{tab:dijkstra_n}
\centering
\begin{tabular}{c|c}
\hline
&$n$\\
\hline \hline
Dense graph&$2.00\pm0.00$\\
Sparse graph&$2.00\pm0.00$\\
\hline
\end{tabular}
\end{table}
\begin{table}[h]
  \caption{Bellman-Ford法での各系列の次数}
  \label{tab:dijkstra_n}
  \centering
  \begin{tabular}{c|c}
  \hline
  &$n$\\
  \hline \hline
  Dense graph&$1.99\pm0.00$\\
  Sparse graph&$1.41\pm0.00$\\
  \hline
  \end{tabular}
  \end{table}
\gnu{Dijkstra法でのノード数と計算量の関係}{graph/dijkstra.tex}
\gnu{Bellman-Ford法でのノード数と計算量の関係}{graph/bellman.tex}