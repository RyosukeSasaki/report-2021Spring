\subsection*{問5}
c++でAho-Corasick法を実装し,円周率から任意の数列を検索した.
ソースコードをListing \ref{hoge}に示す.ソースコードは\cite{AhoCoras40:online}を参考に実装した.
入力テキストとして\cite{PI10058:online}から円周率100万桁を取得し,整形したものを用いた.
また,検索パターンを11111, 22222, ..., 99999とし,検索パターンの個数を1から9個に変化させながら検索に要した時間を計測した.
図\ref{fig:graph/ac.tex}にその結果を示す.
\gnu{検索パターン数と検索時間}{graph/ac.tex}