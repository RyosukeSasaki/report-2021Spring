\documentclass[uplatex,a4j,11pt,dvipdfmx]{jsarticle}
\bibliographystyle{jplain}

\usepackage{url}

\usepackage{graphicx}
\usepackage{gnuplot-lua-tikz}
\usepackage{pgfplots}
\usepackage{tikz}
\usepackage{amsmath,amsfonts,amssymb}
\usepackage{bm}
\usepackage{siunitx}

\makeatletter
\def\fgcaption{\def\@captype{figure}\caption}
\makeatother
\newcommand{\setsections}[3]{
\setcounter{section}{#1}
\setcounter{subsection}{#2}
\setcounter{subsubsection}{#3}
}
\newcommand{\mfig}[3][width=15cm]{
\begin{center}
\includegraphics[#1]{#2}
\fgcaption{#3 \label{fig:#2}}
\end{center}
}

\begin{document}
\title{アルゴリズム2A 第4回レポート}
\author{61908697 佐々木良輔}
\date{}
\maketitle
\section{結果}
bubble sort, heap sort, counting sortについて比較した.
また計算時間は他のプロセスのCPU使用率によって変化するため,コピー回数,比較回数で評価した.
交換1回はコピー3回とした.
\subsection{データ数による比較}
rand10000.txtのデータの上から7500, 5000, 2500行を抽出しrand7500.txt, rand5000.txt, rand2500.txtを作成した.
それぞれについてbubble sort, heap sort, counting sortを行い比較した.表\ref{tab:2}に結果を示す.
\begin{table}[h]
  \caption{データ数による比較}
  \label{tab:2}
  \centering
  \begin{tabular}{c|cc|cc|cc}
  \hline
  &\multicolumn{2}{c|}{bubble sort}&\multicolumn{2}{c|}{heap sort}&\multicolumn{2}{c}{counting sort}\\
  行数&copy&compare&copy&compare&copy&compare\\
  \hline \hline
  10000 & $1.50\times10^{8}$ & $2.49\times10^{7}$ & $1.74\times10^{5}$ & $2.35\times10^{5}$ & $2.00\times10^{4}$ & $0.00$\\
  7500  & $8.43\times10^{7}$ & $1.39\times10^{7}$ & $1.28\times10^{5}$ & $1.70\times10^{5}$ & $1.50\times10^{4}$ & $0.00$\\
  5000  & $3.74\times10^{7}$ & $6.34\times10^{6}$ & $8.21\times10^{4}$ & $1.08\times10^{5}$ & $1.00\times10^{4}$ & $0.00$\\
  2500  & $9.35\times10^{6}$ & $1.56\times10^{6}$ & $3.86\times10^{4}$ & $4.89\times10^{4}$ & $5.00\times10^{3}$ & $0.00$\\
  \hline
  \end{tabular}
  \end{table}
\subsection{昇順,降順,ランダム入力の比較}
表\ref{tab:1}に結果を示す.
\begin{table}[h]
\caption{昇順,降順,ランダム入力の比較}
\label{tab:1}
\centering
\begin{tabular}{c|cc|cc|cc}
\hline
&\multicolumn{2}{c|}{bubble sort}&\multicolumn{2}{c|}{heap sort}&\multicolumn{2}{c}{counting sort}\\
&copy&compare&copy&compare&copy&compare\\
\hline \hline
descend & $1.50\times10^{8}$ & $5.00\times10^{7}$ & $1.67\times10^{5}$ & $2.27\times10^{5}$ & $2.00\times10^{4}$ & $0.00$\\
ascend  & $0.00$             & $1.00\times10^{4}$ & $1.79\times10^{5}$ & $2.40\times10^{5}$ & $2.00\times10^{4}$ & $0.00$\\
random  & $1.50\times10^{8}$ & $2.49\times10^{7}$ & $1.74\times10^{5}$ & $2.35\times10^{5}$ & $2.00\times10^{4}$ & $0.00$\\
\hline
\end{tabular}
\end{table}
\clearpage
\section{考察}
\subsection{データ数による比較}
図\ref{fig:col_b},図,図にbubble sort, heap sort, counting sortそれぞれの計算量のデータ数依存性を示す.
またそれぞれの図に$y=ax^2$, $y=ax\log x$, $y=ax$でフィットした曲線を示している.
これらからbubble sortの計算量は$y=ax^2$, heap sortの計算量は$y=ax\log x$, counting sortの計算量は$y=ax$でよくフィットしていることがわかる.
\begin{figure}[hptb]
  \begin{center}
    \begin{tikzpicture}[gnuplot]
%% generated with GNUPLOT 5.2p8 (Lua 5.3; terminal rev. Nov 2018, script rev. 108)
%% 2021年05月10日 23時03分43秒
\path (0.000,0.000) rectangle (12.500,8.750);
\gpcolor{color=gp lt color border}
\gpsetlinetype{gp lt border}
\gpsetdashtype{gp dt solid}
\gpsetlinewidth{1.00}
\draw[gp path] (3.273,0.985)--(3.453,0.985);
\draw[gp path] (10.730,0.985)--(10.550,0.985);
\node[gp node right] at (3.089,0.985) {$0$};
\draw[gp path] (3.273,1.731)--(3.453,1.731);
\draw[gp path] (10.730,1.731)--(10.550,1.731);
\node[gp node right] at (3.089,1.731) {$2\times10^{7}$};
\draw[gp path] (3.273,2.476)--(3.453,2.476);
\draw[gp path] (10.730,2.476)--(10.550,2.476);
\node[gp node right] at (3.089,2.476) {$4\times10^{7}$};
\draw[gp path] (3.273,3.222)--(3.453,3.222);
\draw[gp path] (10.730,3.222)--(10.550,3.222);
\node[gp node right] at (3.089,3.222) {$6\times10^{7}$};
\draw[gp path] (3.273,3.967)--(3.453,3.967);
\draw[gp path] (10.730,3.967)--(10.550,3.967);
\node[gp node right] at (3.089,3.967) {$8\times10^{7}$};
\draw[gp path] (3.273,4.713)--(3.453,4.713);
\draw[gp path] (10.730,4.713)--(10.550,4.713);
\node[gp node right] at (3.089,4.713) {$1\times10^{8}$};
\draw[gp path] (3.273,5.459)--(3.453,5.459);
\draw[gp path] (10.730,5.459)--(10.550,5.459);
\node[gp node right] at (3.089,5.459) {$1.2\times10^{8}$};
\draw[gp path] (3.273,6.204)--(3.453,6.204);
\draw[gp path] (10.730,6.204)--(10.550,6.204);
\node[gp node right] at (3.089,6.204) {$1.4\times10^{8}$};
\draw[gp path] (3.273,6.950)--(3.453,6.950);
\draw[gp path] (10.730,6.950)--(10.550,6.950);
\node[gp node right] at (3.089,6.950) {$1.6\times10^{8}$};
\draw[gp path] (3.273,7.695)--(3.453,7.695);
\draw[gp path] (10.730,7.695)--(10.550,7.695);
\node[gp node right] at (3.089,7.695) {$1.8\times10^{8}$};
\draw[gp path] (3.273,8.441)--(3.453,8.441);
\draw[gp path] (10.730,8.441)--(10.550,8.441);
\node[gp node right] at (3.089,8.441) {$2\times10^{8}$};
\draw[gp path] (3.273,0.985)--(3.273,1.165);
\draw[gp path] (3.273,8.441)--(3.273,8.261);
\node[gp node center] at (3.273,0.677) {$0$};
\draw[gp path] (4.629,0.985)--(4.629,1.165);
\draw[gp path] (4.629,8.441)--(4.629,8.261);
\node[gp node center] at (4.629,0.677) {$2000$};
\draw[gp path] (5.985,0.985)--(5.985,1.165);
\draw[gp path] (5.985,8.441)--(5.985,8.261);
\node[gp node center] at (5.985,0.677) {$4000$};
\draw[gp path] (7.340,0.985)--(7.340,1.165);
\draw[gp path] (7.340,8.441)--(7.340,8.261);
\node[gp node center] at (7.340,0.677) {$6000$};
\draw[gp path] (8.696,0.985)--(8.696,1.165);
\draw[gp path] (8.696,8.441)--(8.696,8.261);
\node[gp node center] at (8.696,0.677) {$8000$};
\draw[gp path] (10.052,0.985)--(10.052,1.165);
\draw[gp path] (10.052,8.441)--(10.052,8.261);
\node[gp node center] at (10.052,0.677) {$10000$};
\draw[gp path] (3.273,8.441)--(3.273,0.985)--(10.730,0.985)--(10.730,8.441)--cycle;
\node[gp node center,rotate=-270] at (1.509,4.713) {$計算量$};
\node[gp node center] at (7.001,0.215) {$データ数$};
\draw[gp path] (3.457,7.337)--(3.457,8.261)--(7.317,8.261)--(7.317,7.337)--cycle;
\node[gp node right] at (6.033,8.030) {コピー回数};
\gpcolor{rgb color={0.000,0.447,0.698}}
\gpsetpointsize{4.00}
\gppoint{gp mark 1}{(4.968,1.333)}
\gppoint{gp mark 1}{(6.663,2.381)}
\gppoint{gp mark 1}{(8.357,4.127)}
\gppoint{gp mark 1}{(10.052,6.573)}
\gppoint{gp mark 1}{(6.675,8.030)}
\gpcolor{color=gp lt color border}
\node[gp node right] at (6.033,7.568) {比較回数};
\gpcolor{rgb color={0.898,0.118,0.063}}
\gppoint{gp mark 2}{(4.968,1.043)}
\gppoint{gp mark 2}{(6.663,1.221)}
\gppoint{gp mark 2}{(8.357,1.505)}
\gppoint{gp mark 2}{(10.052,1.914)}
\gppoint{gp mark 2}{(6.675,7.568)}
\gpcolor{rgb color={0.000,0.447,0.698}}
\draw[gp path] (3.273,0.985)--(3.348,0.986)--(3.424,0.988)--(3.499,0.991)--(3.574,0.996)%
  --(3.650,1.002)--(3.725,1.010)--(3.800,1.019)--(3.876,1.029)--(3.951,1.041)--(4.026,1.054)%
  --(4.102,1.068)--(4.177,1.084)--(4.252,1.102)--(4.328,1.120)--(4.403,1.140)--(4.478,1.162)%
  --(4.553,1.184)--(4.629,1.208)--(4.704,1.234)--(4.779,1.261)--(4.855,1.289)--(4.930,1.319)%
  --(5.005,1.350)--(5.081,1.382)--(5.156,1.416)--(5.231,1.451)--(5.307,1.488)--(5.382,1.526)%
  --(5.457,1.565)--(5.533,1.606)--(5.608,1.648)--(5.683,1.691)--(5.759,1.736)--(5.834,1.782)%
  --(5.909,1.830)--(5.985,1.879)--(6.060,1.929)--(6.135,1.981)--(6.211,2.034)--(6.286,2.089)%
  --(6.361,2.144)--(6.437,2.202)--(6.512,2.260)--(6.587,2.320)--(6.663,2.382)--(6.738,2.445)%
  --(6.813,2.509)--(6.889,2.574)--(6.964,2.641)--(7.039,2.709)--(7.114,2.779)--(7.190,2.850)%
  --(7.265,2.923)--(7.340,2.996)--(7.416,3.072)--(7.491,3.148)--(7.566,3.226)--(7.642,3.305)%
  --(7.717,3.386)--(7.792,3.468)--(7.868,3.552)--(7.943,3.636)--(8.018,3.723)--(8.094,3.810)%
  --(8.169,3.899)--(8.244,3.990)--(8.320,4.081)--(8.395,4.174)--(8.470,4.269)--(8.546,4.365)%
  --(8.621,4.462)--(8.696,4.561)--(8.772,4.661)--(8.847,4.762)--(8.922,4.865)--(8.998,4.969)%
  --(9.073,5.075)--(9.148,5.182)--(9.224,5.290)--(9.299,5.399)--(9.374,5.511)--(9.450,5.623)%
  --(9.525,5.737)--(9.600,5.852)--(9.675,5.969)--(9.751,6.087)--(9.826,6.206)--(9.901,6.327)%
  --(9.977,6.449)--(10.052,6.572)--(10.127,6.697)--(10.203,6.823)--(10.278,6.951)--(10.353,7.080)%
  --(10.429,7.210)--(10.504,7.342)--(10.579,7.475)--(10.655,7.610)--(10.730,7.745);
\gpcolor{rgb color={0.898,0.118,0.063}}
\draw[gp path] (3.273,0.985)--(3.348,0.985)--(3.424,0.985)--(3.499,0.986)--(3.574,0.987)%
  --(3.650,0.988)--(3.725,0.989)--(3.800,0.991)--(3.876,0.992)--(3.951,0.994)--(4.026,0.996)%
  --(4.102,0.999)--(4.177,1.002)--(4.252,1.004)--(4.328,1.007)--(4.403,1.011)--(4.478,1.014)%
  --(4.553,1.018)--(4.629,1.022)--(4.704,1.026)--(4.779,1.031)--(4.855,1.036)--(4.930,1.040)%
  --(5.005,1.046)--(5.081,1.051)--(5.156,1.057)--(5.231,1.063)--(5.307,1.069)--(5.382,1.075)%
  --(5.457,1.081)--(5.533,1.088)--(5.608,1.095)--(5.683,1.102)--(5.759,1.110)--(5.834,1.118)%
  --(5.909,1.125)--(5.985,1.134)--(6.060,1.142)--(6.135,1.151)--(6.211,1.159)--(6.286,1.168)%
  --(6.361,1.178)--(6.437,1.187)--(6.512,1.197)--(6.587,1.207)--(6.663,1.217)--(6.738,1.228)%
  --(6.813,1.238)--(6.889,1.249)--(6.964,1.260)--(7.039,1.272)--(7.114,1.283)--(7.190,1.295)%
  --(7.265,1.307)--(7.340,1.319)--(7.416,1.332)--(7.491,1.345)--(7.566,1.358)--(7.642,1.371)%
  --(7.717,1.384)--(7.792,1.398)--(7.868,1.412)--(7.943,1.426)--(8.018,1.440)--(8.094,1.455)%
  --(8.169,1.469)--(8.244,1.484)--(8.320,1.500)--(8.395,1.515)--(8.470,1.531)--(8.546,1.547)%
  --(8.621,1.563)--(8.696,1.579)--(8.772,1.596)--(8.847,1.613)--(8.922,1.630)--(8.998,1.647)%
  --(9.073,1.665)--(9.148,1.683)--(9.224,1.701)--(9.299,1.719)--(9.374,1.737)--(9.450,1.756)%
  --(9.525,1.775)--(9.600,1.794)--(9.675,1.813)--(9.751,1.833)--(9.826,1.853)--(9.901,1.873)%
  --(9.977,1.893)--(10.052,1.914)--(10.127,1.934)--(10.203,1.955)--(10.278,1.977)--(10.353,1.998)%
  --(10.429,2.020)--(10.504,2.042)--(10.579,2.064)--(10.655,2.086)--(10.730,2.109);
\gpcolor{color=gp lt color border}
\draw[gp path] (3.273,8.441)--(3.273,0.985)--(10.730,0.985)--(10.730,8.441)--cycle;
%% coordinates of the plot area
\gpdefrectangularnode{gp plot 1}{\pgfpoint{3.273cm}{0.985cm}}{\pgfpoint{10.730cm}{8.441cm}}
\end{tikzpicture}
%% gnuplot variables

    \caption{bubble sortのコピー,比較回数}
    \label{fig:col_b}
  \end{center}
\end{figure}
\begin{figure}[hptb]
  \begin{center}
    \begin{tikzpicture}[gnuplot]
%% generated with GNUPLOT 5.2p8 (Lua 5.3; terminal rev. Nov 2018, script rev. 108)
%% 2021年05月10日 23時03分43秒
\path (0.000,0.000) rectangle (12.500,8.750);
\gpcolor{color=gp lt color border}
\gpsetlinetype{gp lt border}
\gpsetdashtype{gp dt solid}
\gpsetlinewidth{1.00}
\draw[gp path] (3.181,0.985)--(3.361,0.985);
\draw[gp path] (10.638,0.985)--(10.458,0.985);
\node[gp node right] at (2.997,0.985) {$0$};
\draw[gp path] (3.181,2.228)--(3.361,2.228);
\draw[gp path] (10.638,2.228)--(10.458,2.228);
\node[gp node right] at (2.997,2.228) {$50000$};
\draw[gp path] (3.181,3.470)--(3.361,3.470);
\draw[gp path] (10.638,3.470)--(10.458,3.470);
\node[gp node right] at (2.997,3.470) {$100000$};
\draw[gp path] (3.181,4.713)--(3.361,4.713);
\draw[gp path] (10.638,4.713)--(10.458,4.713);
\node[gp node right] at (2.997,4.713) {$150000$};
\draw[gp path] (3.181,5.956)--(3.361,5.956);
\draw[gp path] (10.638,5.956)--(10.458,5.956);
\node[gp node right] at (2.997,5.956) {$200000$};
\draw[gp path] (3.181,7.198)--(3.361,7.198);
\draw[gp path] (10.638,7.198)--(10.458,7.198);
\node[gp node right] at (2.997,7.198) {$250000$};
\draw[gp path] (3.181,8.441)--(3.361,8.441);
\draw[gp path] (10.638,8.441)--(10.458,8.441);
\node[gp node right] at (2.997,8.441) {$300000$};
\draw[gp path] (3.181,0.985)--(3.181,1.165);
\draw[gp path] (3.181,8.441)--(3.181,8.261);
\node[gp node center] at (3.181,0.677) {$0$};
\draw[gp path] (4.537,0.985)--(4.537,1.165);
\draw[gp path] (4.537,8.441)--(4.537,8.261);
\node[gp node center] at (4.537,0.677) {$2000$};
\draw[gp path] (5.893,0.985)--(5.893,1.165);
\draw[gp path] (5.893,8.441)--(5.893,8.261);
\node[gp node center] at (5.893,0.677) {$4000$};
\draw[gp path] (7.248,0.985)--(7.248,1.165);
\draw[gp path] (7.248,8.441)--(7.248,8.261);
\node[gp node center] at (7.248,0.677) {$6000$};
\draw[gp path] (8.604,0.985)--(8.604,1.165);
\draw[gp path] (8.604,8.441)--(8.604,8.261);
\node[gp node center] at (8.604,0.677) {$8000$};
\draw[gp path] (9.960,0.985)--(9.960,1.165);
\draw[gp path] (9.960,8.441)--(9.960,8.261);
\node[gp node center] at (9.960,0.677) {$10000$};
\draw[gp path] (3.181,8.441)--(3.181,0.985)--(10.638,0.985)--(10.638,8.441)--cycle;
\node[gp node center,rotate=-270] at (1.601,4.713) {$計算量$};
\node[gp node center] at (6.909,0.215) {$データ数$};
\draw[gp path] (3.365,7.337)--(3.365,8.261)--(7.225,8.261)--(7.225,7.337)--cycle;
\node[gp node right] at (5.941,8.030) {コピー回数};
\gpcolor{rgb color={0.000,0.447,0.698}}
\gpsetpointsize{4.00}
\gppoint{gp mark 1}{(4.876,1.945)}
\gppoint{gp mark 1}{(6.571,3.026)}
\gppoint{gp mark 1}{(8.265,4.155)}
\gppoint{gp mark 1}{(9.960,5.315)}
\gppoint{gp mark 1}{(6.583,8.030)}
\gpcolor{color=gp lt color border}
\node[gp node right] at (5.941,7.568) {比較回数};
\gpcolor{rgb color={0.898,0.118,0.063}}
\gppoint{gp mark 2}{(4.876,2.201)}
\gppoint{gp mark 2}{(6.571,3.663)}
\gppoint{gp mark 2}{(8.265,5.216)}
\gppoint{gp mark 2}{(9.960,6.835)}
\gppoint{gp mark 2}{(6.583,7.568)}
\gpcolor{rgb color={0.000,0.447,0.698}}
\draw[gp path] (3.256,1.010)--(3.332,1.042)--(3.407,1.077)--(3.482,1.113)--(3.558,1.151)%
  --(3.633,1.190)--(3.708,1.230)--(3.784,1.270)--(3.859,1.312)--(3.934,1.353)--(4.010,1.396)%
  --(4.085,1.439)--(4.160,1.482)--(4.236,1.526)--(4.311,1.570)--(4.386,1.614)--(4.461,1.659)%
  --(4.537,1.704)--(4.612,1.749)--(4.687,1.795)--(4.763,1.841)--(4.838,1.887)--(4.913,1.933)%
  --(4.989,1.980)--(5.064,2.026)--(5.139,2.074)--(5.215,2.121)--(5.290,2.168)--(5.365,2.216)%
  --(5.441,2.264)--(5.516,2.311)--(5.591,2.360)--(5.667,2.408)--(5.742,2.456)--(5.817,2.505)%
  --(5.893,2.554)--(5.968,2.603)--(6.043,2.652)--(6.119,2.701)--(6.194,2.750)--(6.269,2.800)%
  --(6.345,2.849)--(6.420,2.899)--(6.495,2.949)--(6.571,2.999)--(6.646,3.049)--(6.721,3.099)%
  --(6.797,3.149)--(6.872,3.200)--(6.947,3.250)--(7.022,3.301)--(7.098,3.351)--(7.173,3.402)%
  --(7.248,3.453)--(7.324,3.504)--(7.399,3.555)--(7.474,3.606)--(7.550,3.658)--(7.625,3.709)%
  --(7.700,3.761)--(7.776,3.812)--(7.851,3.864)--(7.926,3.915)--(8.002,3.967)--(8.077,4.019)%
  --(8.152,4.071)--(8.228,4.123)--(8.303,4.175)--(8.378,4.228)--(8.454,4.280)--(8.529,4.332)%
  --(8.604,4.385)--(8.680,4.437)--(8.755,4.490)--(8.830,4.542)--(8.906,4.595)--(8.981,4.648)%
  --(9.056,4.701)--(9.132,4.754)--(9.207,4.807)--(9.282,4.860)--(9.358,4.913)--(9.433,4.966)%
  --(9.508,5.019)--(9.583,5.073)--(9.659,5.126)--(9.734,5.179)--(9.809,5.233)--(9.885,5.286)%
  --(9.960,5.340)--(10.035,5.394)--(10.111,5.447)--(10.186,5.501)--(10.261,5.555)--(10.337,5.609)%
  --(10.412,5.663)--(10.487,5.717)--(10.563,5.771)--(10.638,5.825);
\gpcolor{rgb color={0.898,0.118,0.063}}
\draw[gp path] (3.256,1.018)--(3.332,1.061)--(3.407,1.108)--(3.482,1.157)--(3.558,1.207)%
  --(3.633,1.259)--(3.708,1.313)--(3.784,1.367)--(3.859,1.422)--(3.934,1.478)--(4.010,1.535)%
  --(4.085,1.592)--(4.160,1.650)--(4.236,1.709)--(4.311,1.768)--(4.386,1.827)--(4.461,1.887)%
  --(4.537,1.948)--(4.612,2.008)--(4.687,2.069)--(4.763,2.131)--(4.838,2.193)--(4.913,2.255)%
  --(4.989,2.317)--(5.064,2.380)--(5.139,2.443)--(5.215,2.506)--(5.290,2.569)--(5.365,2.633)%
  --(5.441,2.697)--(5.516,2.761)--(5.591,2.826)--(5.667,2.890)--(5.742,2.955)--(5.817,3.020)%
  --(5.893,3.086)--(5.968,3.151)--(6.043,3.217)--(6.119,3.283)--(6.194,3.349)--(6.269,3.415)%
  --(6.345,3.481)--(6.420,3.548)--(6.495,3.615)--(6.571,3.681)--(6.646,3.748)--(6.721,3.816)%
  --(6.797,3.883)--(6.872,3.950)--(6.947,4.018)--(7.022,4.086)--(7.098,4.154)--(7.173,4.222)%
  --(7.248,4.290)--(7.324,4.358)--(7.399,4.427)--(7.474,4.495)--(7.550,4.564)--(7.625,4.633)%
  --(7.700,4.702)--(7.776,4.771)--(7.851,4.840)--(7.926,4.909)--(8.002,4.979)--(8.077,5.048)%
  --(8.152,5.118)--(8.228,5.187)--(8.303,5.257)--(8.378,5.327)--(8.454,5.397)--(8.529,5.467)%
  --(8.604,5.537)--(8.680,5.608)--(8.755,5.678)--(8.830,5.749)--(8.906,5.819)--(8.981,5.890)%
  --(9.056,5.961)--(9.132,6.032)--(9.207,6.103)--(9.282,6.174)--(9.358,6.245)--(9.433,6.316)%
  --(9.508,6.387)--(9.583,6.459)--(9.659,6.530)--(9.734,6.602)--(9.809,6.673)--(9.885,6.745)%
  --(9.960,6.817)--(10.035,6.889)--(10.111,6.961)--(10.186,7.033)--(10.261,7.105)--(10.337,7.177)%
  --(10.412,7.249)--(10.487,7.321)--(10.563,7.394)--(10.638,7.466);
\gpcolor{color=gp lt color border}
\draw[gp path] (3.181,8.441)--(3.181,0.985)--(10.638,0.985)--(10.638,8.441)--cycle;
%% coordinates of the plot area
\gpdefrectangularnode{gp plot 1}{\pgfpoint{3.181cm}{0.985cm}}{\pgfpoint{10.638cm}{8.441cm}}
\end{tikzpicture}
%% gnuplot variables

    \caption{heap sortのコピー,比較回数}
    \label{fig:col_h}
  \end{center}
\end{figure}
\begin{figure}[hptb]
  \begin{center}
    \begin{tikzpicture}[gnuplot]
%% generated with GNUPLOT 5.2p8 (Lua 5.3; terminal rev. Nov 2018, script rev. 108)
%% 2021年05月10日 23時03分43秒
\path (0.000,0.000) rectangle (12.500,8.750);
\gpcolor{color=gp lt color border}
\gpsetlinetype{gp lt border}
\gpsetdashtype{gp dt solid}
\gpsetlinewidth{1.00}
\draw[gp path] (3.089,0.985)--(3.269,0.985);
\draw[gp path] (10.546,0.985)--(10.366,0.985);
\node[gp node right] at (2.905,0.985) {$0$};
\draw[gp path] (3.089,2.476)--(3.269,2.476);
\draw[gp path] (10.546,2.476)--(10.366,2.476);
\node[gp node right] at (2.905,2.476) {$5000$};
\draw[gp path] (3.089,3.967)--(3.269,3.967);
\draw[gp path] (10.546,3.967)--(10.366,3.967);
\node[gp node right] at (2.905,3.967) {$10000$};
\draw[gp path] (3.089,5.459)--(3.269,5.459);
\draw[gp path] (10.546,5.459)--(10.366,5.459);
\node[gp node right] at (2.905,5.459) {$15000$};
\draw[gp path] (3.089,6.950)--(3.269,6.950);
\draw[gp path] (10.546,6.950)--(10.366,6.950);
\node[gp node right] at (2.905,6.950) {$20000$};
\draw[gp path] (3.089,8.441)--(3.269,8.441);
\draw[gp path] (10.546,8.441)--(10.366,8.441);
\node[gp node right] at (2.905,8.441) {$25000$};
\draw[gp path] (3.089,0.985)--(3.089,1.165);
\draw[gp path] (3.089,8.441)--(3.089,8.261);
\node[gp node center] at (3.089,0.677) {$0$};
\draw[gp path] (4.445,0.985)--(4.445,1.165);
\draw[gp path] (4.445,8.441)--(4.445,8.261);
\node[gp node center] at (4.445,0.677) {$2000$};
\draw[gp path] (5.801,0.985)--(5.801,1.165);
\draw[gp path] (5.801,8.441)--(5.801,8.261);
\node[gp node center] at (5.801,0.677) {$4000$};
\draw[gp path] (7.156,0.985)--(7.156,1.165);
\draw[gp path] (7.156,8.441)--(7.156,8.261);
\node[gp node center] at (7.156,0.677) {$6000$};
\draw[gp path] (8.512,0.985)--(8.512,1.165);
\draw[gp path] (8.512,8.441)--(8.512,8.261);
\node[gp node center] at (8.512,0.677) {$8000$};
\draw[gp path] (9.868,0.985)--(9.868,1.165);
\draw[gp path] (9.868,8.441)--(9.868,8.261);
\node[gp node center] at (9.868,0.677) {$10000$};
\draw[gp path] (3.089,8.441)--(3.089,0.985)--(10.546,0.985)--(10.546,8.441)--cycle;
\node[gp node center,rotate=-270] at (1.693,4.713) {$計算量$};
\node[gp node center] at (6.817,0.215) {$データ数$};
\draw[gp path] (3.273,7.337)--(3.273,8.261)--(7.133,8.261)--(7.133,7.337)--cycle;
\node[gp node right] at (5.849,8.030) {コピー回数};
\gpcolor{rgb color={0.000,0.447,0.698}}
\gpsetpointsize{4.00}
\gppoint{gp mark 1}{(4.784,2.476)}
\gppoint{gp mark 1}{(6.479,3.967)}
\gppoint{gp mark 1}{(8.173,5.459)}
\gppoint{gp mark 1}{(9.868,6.950)}
\gppoint{gp mark 1}{(6.491,8.030)}
\gpcolor{color=gp lt color border}
\node[gp node right] at (5.849,7.568) {比較回数};
\gpcolor{rgb color={0.898,0.118,0.063}}
\gppoint{gp mark 2}{(4.784,0.985)}
\gppoint{gp mark 2}{(6.479,0.985)}
\gppoint{gp mark 2}{(8.173,0.985)}
\gppoint{gp mark 2}{(9.868,0.985)}
\gppoint{gp mark 2}{(6.491,7.568)}
\gpcolor{rgb color={0.000,0.447,0.698}}
\draw[gp path] (3.089,0.985)--(3.164,1.051)--(3.240,1.118)--(3.315,1.184)--(3.390,1.250)%
  --(3.466,1.316)--(3.541,1.383)--(3.616,1.449)--(3.692,1.515)--(3.767,1.581)--(3.842,1.648)%
  --(3.918,1.714)--(3.993,1.780)--(4.068,1.847)--(4.144,1.913)--(4.219,1.979)--(4.294,2.045)%
  --(4.369,2.112)--(4.445,2.178)--(4.520,2.244)--(4.595,2.311)--(4.671,2.377)--(4.746,2.443)%
  --(4.821,2.509)--(4.897,2.576)--(4.972,2.642)--(5.047,2.708)--(5.123,2.774)--(5.198,2.841)%
  --(5.273,2.907)--(5.349,2.973)--(5.424,3.040)--(5.499,3.106)--(5.575,3.172)--(5.650,3.238)%
  --(5.725,3.305)--(5.801,3.371)--(5.876,3.437)--(5.951,3.503)--(6.027,3.570)--(6.102,3.636)%
  --(6.177,3.702)--(6.253,3.769)--(6.328,3.835)--(6.403,3.901)--(6.479,3.967)--(6.554,4.034)%
  --(6.629,4.100)--(6.705,4.166)--(6.780,4.233)--(6.855,4.299)--(6.930,4.365)--(7.006,4.431)%
  --(7.081,4.498)--(7.156,4.564)--(7.232,4.630)--(7.307,4.696)--(7.382,4.763)--(7.458,4.829)%
  --(7.533,4.895)--(7.608,4.962)--(7.684,5.028)--(7.759,5.094)--(7.834,5.160)--(7.910,5.227)%
  --(7.985,5.293)--(8.060,5.359)--(8.136,5.425)--(8.211,5.492)--(8.286,5.558)--(8.362,5.624)%
  --(8.437,5.691)--(8.512,5.757)--(8.588,5.823)--(8.663,5.889)--(8.738,5.956)--(8.814,6.022)%
  --(8.889,6.088)--(8.964,6.154)--(9.040,6.221)--(9.115,6.287)--(9.190,6.353)--(9.266,6.420)%
  --(9.341,6.486)--(9.416,6.552)--(9.491,6.618)--(9.567,6.685)--(9.642,6.751)--(9.717,6.817)%
  --(9.793,6.884)--(9.868,6.950)--(9.943,7.016)--(10.019,7.082)--(10.094,7.149)--(10.169,7.215)%
  --(10.245,7.281)--(10.320,7.347)--(10.395,7.414)--(10.471,7.480)--(10.546,7.546);
\gpcolor{rgb color={0.898,0.118,0.063}}
\draw[gp path] (3.089,0.985)--(3.164,0.985)--(3.240,0.985)--(3.315,0.985)--(3.390,0.985)%
  --(3.466,0.985)--(3.541,0.985)--(3.616,0.985)--(3.692,0.985)--(3.767,0.985)--(3.842,0.985)%
  --(3.918,0.985)--(3.993,0.985)--(4.068,0.985)--(4.144,0.985)--(4.219,0.985)--(4.294,0.985)%
  --(4.369,0.985)--(4.445,0.985)--(4.520,0.985)--(4.595,0.985)--(4.671,0.985)--(4.746,0.985)%
  --(4.821,0.985)--(4.897,0.985)--(4.972,0.985)--(5.047,0.985)--(5.123,0.985)--(5.198,0.985)%
  --(5.273,0.985)--(5.349,0.985)--(5.424,0.985)--(5.499,0.985)--(5.575,0.985)--(5.650,0.985)%
  --(5.725,0.985)--(5.801,0.985)--(5.876,0.985)--(5.951,0.985)--(6.027,0.985)--(6.102,0.985)%
  --(6.177,0.985)--(6.253,0.985)--(6.328,0.985)--(6.403,0.985)--(6.479,0.985)--(6.554,0.985)%
  --(6.629,0.985)--(6.705,0.985)--(6.780,0.985)--(6.855,0.985)--(6.930,0.985)--(7.006,0.985)%
  --(7.081,0.985)--(7.156,0.985)--(7.232,0.985)--(7.307,0.985)--(7.382,0.985)--(7.458,0.985)%
  --(7.533,0.985)--(7.608,0.985)--(7.684,0.985)--(7.759,0.985)--(7.834,0.985)--(7.910,0.985)%
  --(7.985,0.985)--(8.060,0.985)--(8.136,0.985)--(8.211,0.985)--(8.286,0.985)--(8.362,0.985)%
  --(8.437,0.985)--(8.512,0.985)--(8.588,0.985)--(8.663,0.985)--(8.738,0.985)--(8.814,0.985)%
  --(8.889,0.985)--(8.964,0.985)--(9.040,0.985)--(9.115,0.985)--(9.190,0.985)--(9.266,0.985)%
  --(9.341,0.985)--(9.416,0.985)--(9.491,0.985)--(9.567,0.985)--(9.642,0.985)--(9.717,0.985)%
  --(9.793,0.985)--(9.868,0.985)--(9.943,0.985)--(10.019,0.985)--(10.094,0.985)--(10.169,0.985)%
  --(10.245,0.985)--(10.320,0.985)--(10.395,0.985)--(10.471,0.985)--(10.546,0.985);
\gpcolor{color=gp lt color border}
\draw[gp path] (3.089,8.441)--(3.089,0.985)--(10.546,0.985)--(10.546,8.441)--cycle;
%% coordinates of the plot area
\gpdefrectangularnode{gp plot 1}{\pgfpoint{3.089cm}{0.985cm}}{\pgfpoint{10.546cm}{8.441cm}}
\end{tikzpicture}
%% gnuplot variables

    \caption{counting sortのコピー,比較回数}
    \label{fig:col_c}
  \end{center}
\end{figure}
\clearpage
\subsection{昇順,降順,ランダム入力の比較}
\subsubsection{bubble sortについて}
表\ref{tab:2}からdescend, randomの場合の計算量は$10^8$程度のオーダーになっているのに対し,
ascendの場合は$10^4$程度のオーダーになっていることがわかる.
これはascendが既にsortされており,交換が発生しなかった時点でループを抜けたためである.
これはbubble sortの計算量が最悪で$O(n^2)\sim10^8$, 平均で$O(n^2)\sim10^8$, 最良で$O(n)\sim10^4$であることに整合する.
\subsubsection{heap sortについて}
表\ref{tab:2}からdescend, ascend, random全ての場合において計算量は$10^5$程度のオーダーである.
これはheap sortが全ての場合において$O(n\log_2n)\sim10^5$となることに整合する.
\subsubsection{counting sortについて}
表\ref{tab:2}からdescend, ascend, random全ての場合において計算量は$10^4$程度のオーダーである.
これはcounting sortが全ての場合において$O(n)\sim10^4$となることに整合する.
\end{document}