\subsection*{問3 最小二乗法の拡張}
\subsubsection*{最小二乗法について}
最小二乗法は測定で得られた離散的なデータを関数で近似するために用いられる.
ここでは特に近似を行う関数形が$m$次多項式の場合を考える.
すなわち関数は
\begin{align}
  g(x)=\sum_{k=0}^{m} a_kx^k=a_0+a_1x+\cdots+a_mx^m
\end{align}
という形で表される.ここで$n$組の離散データの組が$(x_i,y_i)$で与えられるとき
最小二乗法ではこれらのデータと$g(x)$の残差の二乗和$E$が最小となるように$a_k$を定める.
すなわち目的関数は以下で与えられる.ただし$a_0,\ldots,a_m$を$a_k$と略記している.
\begin{align}
  {\rm minimize}: E(a_k)=\sum_{i=0}^{n-1}(y_i-g(x_i))^2
\end{align}
ここで$a_k$について$E(a_k)$が唯一極小値を持つことを仮定すると$E(a_k)$が最小となる条件は
\begin{align}
  \begin{split}
    \frac{\partial E}{\partial a_k}&=\frac{\partial}{\partial a_k}\sum_{i=0}^{n-1}(y_i-(a_0+\cdots+a_kx_i^k+\cdots+a_mx_i^m))^2\\
    &=2\sum_{i=0}^{n-1}(y_i-g(x_i))(-x_i^k)=0
  \end{split}
\end{align}
すなわち
\begin{align}
  \sum_{i=0}^{n-1}(a_0x_i^k+a_1x_i^{k+1}+\cdots+a_mx_i^{k+m})=\sum_{i=0}^{n-1}y_ix_i^k
\end{align}
となる.
$k=0,\ldots,m$までを並べて行列とすると
\begin{align}
  \left(
    \begin{array}{cccc}
      n&\overline{x}&\cdots&\overline{x^m}\\
      \overline{x}&\overline{x^2}&\cdots&\overline{x^{m+1}}\\
      \vdots&\vdots&\vdots&\vdots\\
      \overline{x^m}&\overline{x^{m+1}}&\cdots&\overline{x^{2m}}
    \end{array}
  \right)
  \left(
    \begin{array}{c}
      a_0\\a_1\\\vdots\\a_m      
    \end{array}
  \right)=\left(
    \begin{array}{c}
      \overline{y}\\\overline{xy}\\\vdots\\\overline{x^my}
    \end{array}
  \right)
\end{align}
ここで$\overline{x^ky}$などは$\sum_{i=0}^{n-1}x^k_iy_i$を意味する.以上から(5)式の連立方程式を解くことで$E(a_k)$を最小化する$a_k$が得られる.
また$m$次多項式での最小二乗法は係数行列の生成に$O(m^2\times n)$の総和計算,連立方程式を解くのには$O(m^2)$の計算量が必要と見積もられる.
\subsubsection*{実装について}
以上の最小二乗法を数値計算するプログラム(lsm\_polynomial.c)をC言語で実装した.ソースコードは補遺Aに示した.
入力データは1列目に時刻,
2列目に値を持ったテキストファイルである.
また入力ファイルの1行1列にはデータ数を,
1行2列には多項式の次数を与える.
連立方程式のソルバーとしてガウスの消去法を用いた.
\subsubsection*{動作確認}
まず動作確認として,第10回授業で配布されたxy9.txtと同一のデータを直線回帰し,その結果を同じく第10回授業で配布されたlsm.cの出力結果と比較した.
それぞれのプログラムで得られた回帰直線は以下の通りである.
この結果から2つのlsm\_polynomial.cはlsm.cと全く同一の出力をしており,正常に動作していると考えられる.
\begin{table}[h]
\caption{動作確認の結果}
\label{tab:dousakakunin}
\centering
\begin{tabular}{cc}
\hline
プログラム&回帰直線\\
\hline \hline
lsm.c&$y=12.068x-5.011$\\
lsm\_polynomial.c&$y=12.068x-5.011$\\
\hline
\end{tabular}
\end{table}
\subsubsection*{性能評価}
幾つかの次数の多項式から生成されたデータを用いてプログラムの性能を評価した.
ここではデータの生成プログラムとしてgenerate\_data.cを作成した.
これは任意の多項式関数に対して正規分布に従うノイズを乗せたものをファイルに保存するプログラムである.
ソースコードは補遺Aに示した.
正規分布に従う乱数は\cite{ransu}を参考にした.
ここではデータ数$n=101$である.
また性能は計算に要したCPU時間の5回平均を用いて評価した.
表\ref{tab:fitting}に最小二乗法によるfittingの結果を示す.それぞれのfitting結果のプロットを図から図に示した.
これらの結果から作成したプログラムは正常にfittingを行っていると言える.
\begin{table}[h]
  \caption{fitting結果}
  \label{tab:fitting}
  \centering
  \begin{tabular}{c|ll}
    \hline
    次数&元の関数&fitting結果\\
    \hline \hline
    1&$10+5x$&$9.8+5.0x$\\
    2&$10+3x+x^2$&$9.9+3.0x+1.0x^2$\\
    3&$10+5x+0.2x^2+1.2x^3$&$9.8+5.1x+0.2x^2+1.2x^3$\\
    4&$10+5x+0.4x^2+0.5x^3+2x^4$&$9.7+5.1x+0.4x^2+0.5x^3+2.0x^4$\\
    5&$10+5x+0.1x^2+0.6x^3+x^4+1.3x^5$&$9.7+5.1x+0.2x^2+0.6x^3+1.0x^4+1.3x^5$\\
    \hline
  \end{tabular}
\end{table}
\clearpage
\gnu{1次の回帰曲線}{graph/01.tex}
\gnu{2次の回帰曲線}{graph/02.tex}
\gnu{3次の回帰曲線}{graph/03.tex}
\gnu{4次の回帰曲線}{graph/04.tex}
\gnu{5次の回帰曲線}{graph/05.tex}
\clearpage

図\ref{fig:graph/init.tex},図\ref{fig:graph/calc.tex}に次数と係数行列の初期化時間,計算時間の関係を示す.
回帰曲線はgnuplotの非線形最小二乗法機能を用いて作成し,それぞれの関数は表\ref{tab:per_kaiki}のようになった.
以上の結果から初期化時間,計算時間ともに大凡次数の二乗に比例していることが確かめられた.
また最小二乗法の計算においては係数行列の初期化が連立方程式の計算に比べて非常に多くの時間を要することがわかった.
これは連立方程式の計算時間は次数のみに依存するのに対し,係数行列の初期化時間は元データ数に線形に比例することに依ると考えられる.
係数行列の初期化を高速化する手段として,係数行列の対称性から不要な総和計算を省略することが考えられる.
これにより上手く実装を行えば$O(m^2\times n)$だった計算量は$O(m\times n)$にできると考えられる.
\begin{table}[h]
\caption{回帰曲線}
\label{tab:per_kaiki}
\centering
\begin{tabular}{cc}
\hline
初期化時間&$y=21.4x^{1.73}+62.6$\\
計算時間&$y=8.85\times10^{-2}x^{2.11}+2.74$\\
\hline
\end{tabular}
\end{table}
\gnu{次数と初期化時間の関係}{graph/init.tex}
\gnu{次数と計算時間の関係}{graph/calc.tex}