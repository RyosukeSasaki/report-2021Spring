\subsection*{問3 最小二乗法の拡張}
最小二乗法は測定で得られた離散的なデータを関数で近似するために用いられる.
ここでは特に近似を行う関数形が$m$次多項式の場合を考える.
すなわち関数は
\begin{align}
  g(x)=\sum_{k=0}^{m} a_kx^k=a_0+a_1x+\cdots+a_mx^m
\end{align}
という形で表される.ここで$n$組の離散データの組が$(x_i,y_i)$で与えられるとき
最小二乗法ではこれらのデータと$g(x)$の残差の二乗和$E$が最小となるように$a_k$を定める.
すなわち目的関数は以下で与えられる.ただし$a_0,\ldots,a_m$を$a_k$と略記している.
\begin{align}
  {\rm minimize}: E(a_k)=\sum_{i=0}^{n-1}(y_i-g(x_i))^2
\end{align}
ここで$a_k$について$E(a_k)$が唯一極小値を持つことを仮定すると$E(a_k)$が最小となる条件は
\begin{align}
  \begin{split}
    \frac{\partial E}{\partial a_k}&=\frac{\partial}{\partial a_k}\sum_{i=0}^{n-1}(y_i-(a_0+\cdots+a_kx_i^k+\cdots+a_mx_i^m))^2\\
    &=2\sum_{i=0}^{n-1}(y_i-g(x_i))(-x_i^k)=0
  \end{split}
\end{align}
すなわち
\begin{align}
  \sum_{i=0}^{n-1}(a_0x_i^k+a_1x_i^{k+1}+\cdots+a_mx_i^{k+m})=\sum_{i=0}^{n-1}y_ix_i^k
\end{align}
となる.
$k=0,\ldots,m$までを並べて行列とすると
\begin{align}
  \left(
    \begin{array}{cccc}
      n&\overline{x}&\cdots&\overline{x^m}\\
      \overline{x}&\overline{x^2}&\cdots&\overline{x^{m+1}}\\
      \vdots&\vdots&\vdots&\vdots\\
      \overline{x^m}&\overline{x^{m+1}}&\cdots&\overline{x^{2m}}
    \end{array}
  \right)
  \left(
    \begin{array}{c}
      a_0\\a_1\\\vdots\\a_m      
    \end{array}
  \right)=\left(
    \begin{array}{c}
      \overline{y}\\\overline{xy}\\\vdots\\\overline{x^my}
    \end{array}
  \right)
\end{align}
ここで$\overline{x^ky}=\sum_{i=0}^{n-1}x^k_iy_i$である.以上から(5)式の連立方程式を解くことで$E(a_k)$を最小化する$a_k$が得られる.