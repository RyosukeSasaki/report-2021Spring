\subsection{結果}
\subsubsection{測定1-B-01}
図\ref{fig:graph/1-B/1-B-01.tex}に測定1-B-01の結果を示す.
破線は$V_o$の直線部分を最小二乗fitしたものであり回帰曲線は以下の関数である.
\begin{align}
  \label{equ:amp_gain}
  y=5.843x+9.202\times10^{-31}
\end{align}
\gnu{$V_i$に対する$V_o$の変化}{graph/1-B/1-B-01.tex}
\subsubsection{測定1-B-02}
図\ref{fig:04/amp_1kHz.png}に周波数$1\ \si{\kilo\hertz}$の時の波形を示す.
青線が入力信号$V_i$, 赤線が出力信号$V_o$である.縦軸の単位は$\si{\volt}$,横軸の単位は$\si{\second}$である.
また図に測定1-B-02で得たBode線図を示す.
\mfig[width=10cm]{04/amp_1kHz.png}{$1\ \si{\kilo\hertz}$の時の波形}
\gnu{Bode線図の実測値}{graph/1-B/bode.tex}