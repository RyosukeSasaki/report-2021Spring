\subsection{完全積分回路の原理}
\subsubsection{回路方程式}
図\ref{fig:06/2-a-01.png}に完全積分回路の回路図を示す.
$R_1$を流れる電流は全てコンデンサに流れるので
コンデンサ両端の電圧$v_C$は
\begin{align}
  \begin{split}
    v_C=\frac{q}{C}&=\frac{1}{C}\int^t_0I{\rm d}t+v_C(0)\\
    &=\frac{1}{CR_1}\int^t_0v_1{\rm d}t+v_C(0)
  \end{split}
\end{align}
ここでコンデンサの初期電荷による項を$v_C(0)$としている.仮想短絡によりオペアンプの$-$端子の電位は0なので
出力電圧$v_o$は時刻$t=0$でコンデンサに電荷が無いとすれば
\begin{align}
  v_o=-v_C=-\frac{1}{CR_1}\int^t_0v_1{\rm d}t
\end{align}
となり確かに入力電圧の積分値が出力となっている.
\mfig[width=10cm]{06/2-a-01.png}{積分回路}
\subsubsection{伝達関数}
ラプラス変換の積分則は以下のようになる.
\begin{align}
  \mathcal{L}
\end{align}