\subsection{原理}
\subsubsection{回路方程式}
図\ref{fig:06/2-a-01.png}に完全積分回路の回路図を示す.
$R_1$を流れる電流は全てコンデンサに流れるので
コンデンサ両端の電圧$v_C$は
\begin{align}
  \begin{split}
    v_C=\frac{q}{C}&=\frac{1}{C}\int^t_0i{\rm d}t+v_C(0)\\
    &=\frac{1}{CR_1}\int^t_0v_i{\rm d}t+v_C(0)
  \end{split}
\end{align}
ここでコンデンサの初期電荷による項を$v_C(0)$としている.仮想短絡によりオペアンプの$-$端子の電位は0なので
出力電圧$v_o$は時刻$t=0$でコンデンサに電荷が無いとすれば
\begin{align}
  \label{equ:integrator}
  v_o=-v_C=-\frac{1}{CR_1}\int^t_0v_i{\rm d}t
\end{align}
となり確かに入力電圧の積分値が出力となっている.
\mfig[width=10cm]{06/2-a-01.png}{積分回路}
\clearpage
\subsubsection{伝達関数}
ラプラス変換の積分則は以下のようになる.
\begin{align}
  \mathcal{L}\left(\int^t_0f(t){\rm d}t\right)=\frac{1}{s}F(s)
\end{align}
ここで初期値は0としている.したがって(\ref{equ:integrator})をラプラス変換すると
\begin{align}
  \begin{split}
    V_o&=-\frac{1}{CR_1s}V_i(s)\\
    G(s)&=-\frac{1}{CR_1s}
  \end{split}
\end{align}
よってGainと位相差は
\begin{align}
  \begin{split}
    G_V(\omega)&=20\log\left|-\frac{1}{CR_1j\omega}\right|=-20\log(CR_1\omega)\\
    \phi(\omega)&=\angle\left(-\frac{1}{CR_1j\omega}\right)=\angle(j)=\frac{\pi}{2}
  \end{split}
\end{align}
以上から図\ref{fig:06/2-a-01.png}で示した値を代入してBode線図を描くと図\ref{fig:graph/2-A-01/theo-bode.tex}のようになる.
\gnu{Bode線図の理論値}{graph/2-A-01/theo-bode.tex}