\subsection{考察}
図から伝達関数から得られたBode線図と測定値はよく一致していることがわかる.

またGainを見ると低周波域,高周波域の両方で減衰が起き,全帯域の一部のみが透過していることがわかる.
したがってこの回路は確かにBPFとして機能していることがわかる.
また,位相差は$\pi/2$から$-\pi/2$まで変化し,透過域では位相差0になっていることがわかる.
このことはBPFの透過域においては前段のLPFと後段のHPF両方が位相差0で透過していることからも理解できる.

また,透過域はLPFのカットオフ周波数$1/2\pi C_1R_2$とHPFのカットオフ周波数$1/2\pi C_2R_4$の間になるため,これらの値を適当に設定することで任意の周波数帯を透過するBPFを設計できる.