\subsubsection{伝達関数}
図に2次のBPFの回路図を示す.
これは前節,前々節で取り扱ったHPFとLPFを接続したものである.
この一般の入力に対してこの回路方程式を解くのは困難だが,この伝達関数がHPFとBPFの線形和になることから
Gainと位相差は以下のように求まる.
\begin{align}
  \begin{split}
    G_V(\omega)&=20\log\left(\frac{R_2R_4}{R_1R_3}\right)+20\log\sqrt{1+(C_1R_2\omega)^2}-20\log\frac{\omega}{\sqrt{\omega^2+\left(\frac{1}{C_2R_4}\right)^2}}\\
  \end{split}
\end{align}
\begin{align}
  \begin{split}
    \phi(\omega)=\frac{\pi}{2}-\arctan(C_1R_2\omega)-\arctan(C_2R_4\omega)
  \end{split}
\end{align}
よってBode線図をプロットすると図\ref{fig:graph/3-B/theo-bode.tex}のようになる
\mfig[width=12cm]{13/ex3-C.png}{2次BPF}
\gnu{Bode線図の理論値}{graph/3-B/theo-bode.tex}