\subsection{方法}
\subsubsection{測定2-C-(a)-01}
図\ref{fig:09/ex2-C-a.png}の回路をQucs Spice上で作成した.
この回路において入力信号の周波数を$10$から$10\si{\kilo}\ \si{\hertz}$範囲で変化させ,カットオフ周波数を探した.
その際には出力電圧の振幅が$\sqrt{2}/2\ \si{\volt}$に小数点以下5桁まで一致するところを手動で二分探査を行って探した.
またその際の波形を記録した.
\subsubsection{測定2-C-(a)-02}
正弦波の周波数を$f=10,22,47,100,220,470,1000,2200,4700,10000\ \si{\hertz}$と変化させたときの
$v_i$, $v_o$の振幅及び位相差を記録し, Bode線図を作成した.
その際Qucs Spiceの方程式機能を用いて$v_i$, $v_o$のピークの時間と電圧を取得し,その値から算出した.