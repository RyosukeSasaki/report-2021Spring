\subsection{考察}
\subsubsection{測定2-C-(a)-01}
$R_1=1.984\ \si{\kilo\hertz}$, $C=0.099\ \si{\micro\farad}$であったことからカットオフ周波数は
\begin{align}
  f_c=\frac{1}{2\pi CR_1}=810.30\ \si{\hertz}
\end{align}
である.したがって測定値との相対誤差は$6.2\times10^{-3}\%$でありよく一致している.
またカットオフ周波数においては
\begin{align}
  \phi(\omega)=-\arctan(1)=-\frac{\pi}{4}
\end{align}
であり$\pi/4$だけ位相が進む.図\ref{fig:09/ex2-C-a_sqrt2.png}を見るとたしかに位相が進んでいる.
\subsubsection{測定2-C-(a)-02}
\label{subsubsec:passiveLPF}
図\ref{fig:graph/2-C-01/bode.tex}を見ると伝達関数から得られたBode線図と測定値の傾向は一致していることがわかる.
また,受動LPFと能動LPFではGainは同様の振る舞いを示すのに対して,位相差は$\pi$だけシフトしている.
受動LPFでは低周波域で位相差が0に近いため,
低周波域をそのままの位相で透過したい場合は受動LPFを,
低周波域を逆相で透過したい場合は能動LPFを用いるべきである.

また,能動LPFであれば$R_2$を$R_1$より大きくすれば低周波域でのGainを正にすることができるが,
受動LPFではGainが正になることはない.