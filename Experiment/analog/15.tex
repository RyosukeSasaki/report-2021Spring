\section{実験4-A:ダイオードの特性}
\subsection{原理}
ダイオードのに流れる電流は電圧に対して指数関数的に増大する.
すなわち
\begin{align}
  I=I_s({\rm e}^{qV/kT}-1)
\end{align}
ここで$I_s$は飽和電流, $q$は電気素量, $k$はボルツマン定数, $T$は温度である.
この特性を図\ref{fig:15/circ.png}に示した回路で測定する.
\mfig[width=12cm]{15/circ.png}{特性測定回路}
\subsection{方法}
Qucs Spiceで図\ref{fig:15/circ.png}に示した回路を作成した.
ダイオードの飽和電流は$10\ \si{\micro\ampere}$である.
VDCを0から$0.5\ \si{\volt}$までパラメータスイープ機能を用いて掃引し,ダイオードの電圧-電流特性を調べた.
\subsection{結果}
図\ref{fig:graph/4-A/4-a-1.tex}にダイオードの電圧-電流特性を示す.
\gnu{ダイオードの電圧-電流特性}{graph/4-A/4-a-1.tex}
\subsection{考察}
図\ref{fig:graph/4-A/4-a-1.tex}からたしかにダイオードの電圧-電流特性は指数関数的であることがわかる.
このことから,ダイオードは順方向の電圧については大きな電流を流せるのに対し,逆方向の電圧に対してはほぼ電流を流さないということがわかる.