\subsection{方法}
\subsubsection{加算回路}
図\ref{fig:05/adder.png}の回路をQucs Spice上で作成した.
$R_1=1.012\ \si{\kilo\ohm}$, $R_2=198.9\ \si{\ohm}$, $R_3=1.005\ \si{\kilo\ohm}$である.
また$v_1$はDC$-5\ \si{\volt}$, $v_2$は$V_{p-p}=0.5\ \si{\volt}$, $f=3\ \si{\kilo\hertz}$の正弦波である.
(\ref{equ:adder})から
\begin{align}
  v_o=\frac{1.005\times10^3}{1.012\times10^3}5-\frac{1.005\times10^3}{198.9}0.5\sin\omega t
\end{align}
このことから出力波形は$5\ \si{\volt}$中心で$V_{p-p}=2.5\ \si{\volt}$, $f=3\ \si{\kilo\hertz}$で入力信号と逆相の正弦波になる.
\subsubsection{減算回路}
図\ref{fig:05/subtractor.png}の回路をQucs Spice上で作成した.
$R_1=333.1\ \si{\ohm}$, $R_2=333.7\ \si{\ohm}$, $R_3=0.997\ \si{\kilo\ohm}$, $R_4=1.002\ \si{\kilo\ohm}$である.
また$v_1$は$V_{p-p}=0.5\ \si{\volt}$, $f=5\ \si{\kilo\hertz}$の正弦波, $v_2$は$V_{p-p}=1\ \si{\volt}$, $f=1\ \si{\kilo\hertz}$の矩形波である.
(\ref{equ:subtractor})から
\begin{align}
  v_o=2.995v_2-2.993v_1
\end{align}
このことから出力波形は$V_{p-p}=3\ \si{\volt}$, $f=1\ \si{\kilo\hertz}$の矩形波から
$V_{p-p}=1.5\ \si{\volt}$, $f=5\ \si{\kilo\hertz}$の正弦波を引いたものになる.