\subsection{方法}
\subsubsection{測定2-B-01}
図\ref{fig:08/ex2-b-01.png}の回路をQucs Spice上で作成しTransient Simulationを行い波形を記録した.
$R_1=100.1\ \si{\kilo\ohm}$, $C=0.102\ \si{\micro\farad}$である.
$v_1$は$V_{p-p}=1\ \si{\volt}$, $f=100\ \si{\hertz}$の矩形波とした.
\subsubsection{測定2-B-02}
上の回路において$R_1=0.979\ \si{\kilo\ohm}$, $10.16\si{\kilo\ohm}$に変えて再びTransient Simulationを行い波形を記録した.
その中で入力波形に対して出力波形が尤もらしかった,すなわち出力のピーク幅が狭かった回路に対して入力を正弦波,
三角波に変えた時の出力波形を記録した.