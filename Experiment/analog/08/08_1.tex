\subsection{原理}
\subsubsection{回路方程式}
図に微分回路の回路図を示す.
仮想短絡からコンデンサの右側の電位は0である.
またコンデンサに溜まっている電荷は$q=Cv_i$なので流れる電流$i_C$は
\begin{align}
  i_C=C\frac{{\rm d}v_i}{{\rm d}t}
\end{align}
この電流は全て抵抗に流れ込むとできるので出力電圧は
\begin{align}
  \label{equ:differentiator_vo}
  v_o=-R_1i_C=-CR_1\frac{{\rm d}v_i}{{\rm d}t}
\end{align}
となり出力電圧はたしかに入力電圧の微分で与えられる.
\mfig[width=10cm]{08/ex2-b-01.png}{微分回路}
\subsubsection{伝達関数}
ラプラス変換の微分則(\ref{equ:Laplace_differential})から(\ref{equ:differentiator_vo})は
\begin{align}
  \begin{split}
    V_o&=-sCR_1V_i\\
    G(s)&=-sCR_1
  \end{split}
\end{align}
よってGainと位相差は
\begin{align}
  \begin{split}
    G_V(\omega)&=20\log\left|-CR_1j\omega\right|=20\log(CR_1\omega)\\
    \angle(\omega)&=-\frac{\pi}{2}
  \end{split}
\end{align}
以上から図\ref{fig:08/2-b-01.png}で示した値を代入してBode線図を描くと図のようになる.
