\subsection{結果}
\subsubsection{測定2-A-02-(a)}
図\ref{fig:07/ex2-A-2-1.png}に出力波形を示す.青線が入力波形,赤線が出力波形である.
また図\ref{fig:07/ex2-A-2-2.png}に出力波形の立ち上がり付近を拡大した波形を示す.
この波形から時定数$\tau$は以下のように求まった.
\begin{align}
  \tau=0.10300\ \si{\milli\second}
\end{align}
\mfig[width=12cm]{07/ex2-A-2-1.png}{不完全積分回路の出力波形}
\mfig[width=12cm]{07/ex2-A-2-2.png}{出力波形の立ち上がり}
\subsubsection{測定2-A-02-(b)}
表\ref{tab:2-A-02-(b)}の条件3のとき,
矩形波の入力に対する出力波形は図\ref{fig:07/ex2-A-2-3-rect.png}のようになり三角波に近い波形を出力した.
図\ref{fig:07/ex2-A-2-3-sin.png},図\ref{fig:07/ex2-A-2-3-tri.png}に条件3において入力信号を正弦波,三角波としたときの出力波形を示す.
\mfig[width=12cm]{07/ex2-A-2-3-rect.png}{矩形波に対する出力波形}
\mfig[width=12cm]{07/ex2-A-2-3-sin.png}{正弦波に対する出力波形}
\mfig[width=12cm]{07/ex2-A-2-3-tri.png}{三角波に対する出力波形}