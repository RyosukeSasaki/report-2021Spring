\subsection{結果}
\subsubsection{測定2-A-02-(a)}
図に出力波形を示す.青線が入力波形,赤線が出力波形である.
また図に出力波形の立ち上がり付近を拡大した波形を示す.
この波形から時定数$\tau$は以下のように求まった.
\begin{align}
  \tau=0.10300\ \si{\milli\second}
\end{align}
\mfig[width=12cm]{07/ex2-A-2-1.png}{不完全積分回路の出力波形}
\mfig[width=12cm]{07/ex2-A-2-2.png}{出力波形の立ち上がり}
\subsubsection{測定2-A-02-(b)}
表\ref{tab:2-A-02-(b)}の条件3のとき,
矩形波の入力に対する出力波は図のようになり三角波に近い波形を出力した.
図から図に条件3において入力信号を矩形波,正弦波,三角波としたときの出力波形を示す.
\mfig[width=12cm]{07/ex2-A-2-3-rect.png}{矩形波に対する出力波形}
\mfig[width=12cm]{07/ex2-A-2-3-sin.png}{正弦波に対する出力波形}
\mfig[width=12cm]{07/ex2-A-2-3-tri.png}{三角波に対する出力波形}