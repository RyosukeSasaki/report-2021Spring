\subsection{原理}
\subsubsection{回路方程式}
図\ref{fig:03/1-a.png}に反転増幅器の回路図を示す.
オペアンプの入力インピーダンスが$R_2$に比べて非常に高いことから入力電流$i$はすべて$R_2$に流れ込むとして良い.
したがってOhmの法則から
\begin{align}
  \label{equ:inv_amp_circuit_eq}
  i=\frac{v_i-v_d}{R_1}=\frac{v_d-v_o}{R_2}
\end{align}
ここで(\ref{equ:01-VoA})式から
\begin{align}
  v_o=A(0-v_d)
\end{align}
が成り立つためこれらを連立すると
\begin{align}
  v_o=-v_i\frac{R_2}{R_1}\frac{1}{1+\frac{R_1+R_2}{AR_1}}
\end{align}
$A$が十分大きいとき
\begin{align}
  \label{equ:inverted_amp_Vo}
  v_o=-\frac{R_2}{R_1}v_i
\end{align}
となる.以上から反転増幅器は入力信号に対して位相が反転し,電圧が$R_2/R_1$倍の信号が出力されることがわかる.
\mfig[width=10cm]{03/1-a.png}{反転増幅器}
\subsubsection{伝達関数}
(\ref{equ:inverted_amp_Vo})から伝達関数を求めると以下のようになる.
\begin{align}
  \begin{split}
    \label{equ:inv_amp_transform_eq}
    V_o&=-\frac{R_2}{R_1}V_i\\
    G(s)=\frac{V_o}{V_i}&=-\frac{R_2}{R_1}
  \end{split}
\end{align}
これに図\ref{fig:03/1-a.png}で示した値を代入してBode線図を描くと図\ref{fig:graph/1-A/theo-bode.tex}のようになる.
\gnu{Bode線図の理論値}{graph/1-A/theo-bode.tex}