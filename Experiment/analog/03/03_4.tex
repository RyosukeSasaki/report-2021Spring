\subsection{考察}
\subsubsection{測定1-A-01}
$R_1=0.999\ \si{\kilo\ohm}$, $R_2=3.381\ \si{\kilo\ohm}$
としたことから反転増幅器の増幅率は
\begin{align}
  \frac{R_2}{R_1}=3.38
\end{align}
となる.一方で(\ref{equ:inv_amp_gain})からシミュレーションにより得られた増幅率は$3.39$である.
実験値と理論値の相対誤差は$0.3\%$であり,十分に一致していると考えられる.

また図\ref{fig:graph/1-A/1-A-01.tex}からオペアンプの入力電圧$\pm15\ \si{\volt}$を超える電圧は出力されていないことがわかる.
さらに$v_o$が飽和した時$v_d$も変化していることがわかる.
これについて(\ref{equ:inv_amp_circuit_eq})から
\begin{align}
  \begin{split}
    \frac{v_i-v_d}{R_1}&=\frac{v_d-v_o}{R_2}\\
    v_d&=\frac{R_2v_i}{R_1+R_2}+\frac{R_1v_o}{R_1+R_2}
  \end{split}
\end{align}
ここで$v_o$が飽和した時$v_o=\pm15\ \si{\volt}$となる.このことから
\begin{align}
  \label{equ:v_d_theo}
  v_{d\pm}=0.772v_i\pm3.42
\end{align}
を得る.一方で図\ref{fig:graph/1-A/1-A-01.tex}の$V_d$の直線部を最小二乗fitして回帰曲線を求めると図\ref{fig:graph/1-A/1-A-02.tex}のようになり,
回帰曲線は以下の関数になった.
\begin{align}
  y=
  \begin{cases}
    \begin{split}
      0.7718x+3.420\\
      0.7719x-3.419
    \end{split}
  \end{cases}
\end{align}
それぞれ有効数字3桁の範囲で(\ref{equ:v_d_theo})と完全に一致しており,上での立式は正しいと考えられる.
\gnu{$V_d$の回帰曲線}{graph/1-A/1-A-02.tex}
\subsubsection{測定1-A-02}
図\ref{fig:03/inv_amp_1kHz.png}を見るとたしかに$V_{p-p}$が0.33倍程度で,位相が反転した波形が出力されていることがわかる.

図\ref{fig:graph/1-A/theo-bode.tex}と図\ref{fig:graph/1-A/bode.tex}を見比べるとどちらも
周波数依存性の無いグラフでありその値もほぼ等しい.
したがって(\ref{equ:inv_amp_transform_eq})で得た伝達関数は正しいとわかる.

位相がどちらも$\pi$程度で一定であるのに対し,
Gainは$0.3\%$程度の差がある.これはgnuplotが十分な桁数を持ってグラフを描画しているのに対し,
Qucs Spiceが小数点以下3桁を切り捨てて出力したことが原因だと考えられる.