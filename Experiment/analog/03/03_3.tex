\subsection{結果}
\subsubsection{測定1-A-01}
図\ref{fig:graph/1-A/1-A-01.tex}に測定1-A-01の結果を示す.左側の軸が$V_o$,右側の軸が$V_d$である.
破線は$V_o$の直線部分を最小二乗fitしたものであり回帰曲線は以下の関数である.
\begin{align}
  \label{equ:inv_amp_gain}
  y=-3.385x+8.973\times10^{-31}
\end{align}
\gnu{$V_i$に対する$V_o$, $V_d$の変化}{graph/1-A/1-A-01.tex}
\subsubsection{測定1-A-02}
図\ref{fig:03/inv_amp_1kHz.png}に周波数$1\ \si{\kilo\hertz}$の時の波形を示す.
青線が入力信号$V_i$, 赤線が出力信号$V_o$である.縦軸の単位は$\si{\volt}$,横軸の単位は$\si{\second}$である.
また図\ref{fig:graph/1-A/bode.tex}に測定1-A-02で得たBode線図を示す.
\mfig[width=10cm]{03/inv_amp_1kHz.png}{$1\ \si{\kilo\hertz}$の時の波形}
\gnu{Bode線図の実測値}{graph/1-A/bode.tex}