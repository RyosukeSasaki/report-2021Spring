\subsection{考察}
図\ref{fig:graph/3-C/bode.tex}から伝達関数から得られたBode線図と測定値はよく一致していることがわかる.

また, 1次LPFのBode線図(図\ref{fig:graph/3-A-01/bode.tex})と比較すると,2次LPFのほうが減衰が早いことがわかる.
これは1段目のLPFで減衰された信号を2段目で更に減衰していると考えることで理解できる.
更に,その減衰量は2次LPFのGainが2つの1次LPFのGainの線形和で表されることから,2倍で減衰しているとわかる.
実際に図\ref{fig:graph/3-A-01/bode.tex}と図\ref{fig:graph/3-C/bode.tex}ではGainがおよそ2倍傾きで小さくなることが見て取れる.

また, 1次LPFは透過域で位相が$\pi$シフトしていたのに対して2次LPFでは位相はシフトしていない.
これは$\pi$のシフトを2回繰り返しているということからも明らかである.