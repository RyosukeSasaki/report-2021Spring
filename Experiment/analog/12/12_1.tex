\subsection{原理}
\subsubsection{伝達関数}
図\ref{fig:12/ex3-A-02.png}に能動LPFの回路図を示す.
これは微分回路の入力インピーダンスが$1/sC$から$R_1+1/sC$となった回路と言えるので
(58)式において$1/sC$を$R_1+1/sC$に, $R_1$を$R_2$に置き換えれば
\begin{align}
  G(s)=-\frac{R_2}{R_1}\frac{s}{s+\frac{1}{CR_2}}
\end{align}
よってGainと位相差は
\begin{align}
  \begin{split}
    G_V(\omega)&=20\log\left|-\frac{R_2}{R_1}\frac{j\omega}{j\omega+\frac{1}{CR_2}}\right|\\
    &=20\log\left(\frac{R_2}{R_1}\right)+20\log\frac{\omega}{\sqrt{\omega^2+\left(\frac{1}{CR_2}\right)^2}}
  \end{split}
\end{align}
\begin{align}
  \begin{split}
    \phi(\omega)&=\angle\left(-\frac{R_2}{R_1}\frac{s}{s+\frac{1}{CR_2}}\right)\\
    &=-\frac{\pi}{2}-\arctan(CR_2\omega)
  \end{split}
\end{align}
よってBode線図は図\ref{fig:graph/3-A-02/theo-bode.tex}のようになる.
\mfig[width=10cm]{12/ex3-A-02.png}{能動HPF}
\gnu{Bode線図の理論値}{graph/3-A-02/theo-bode.tex}