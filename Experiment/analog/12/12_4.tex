\subsection{考察}
\subsubsection{測定3-A-(b)-01}
$R_2=2.009\ \si{\kilo\hertz}$, $C=0.102\ \si{\micro\farad}$であったことから$R_1\simeq R_2$とするとカットオフ周波数は
\begin{align}
  f_c=769.14\ \si{\hertz}
\end{align}
である.したがって測定値との相対誤差は$0.68\%$でありよく一致している.
\subsubsection{測定3-A-(b)-02}
図\ref{fig:graph/3-A-02/bode.tex}から伝達関数から得られたBode線図と測定値はよく一致していることがわかる.

また,能動HPFと受動HPFを比較するとGainのは類似な振る舞いを示すのに対して,
位相差は能動HPFでは高周波域で$\pi$, 受動HPFでは高周波域で0となることがわかる.
したがって入力信号の位相を維持した出力を得たいときは受動HPFを,逆相の出力を得たいときは能動HPFを用いるべきである.

また,能動LPFであればR2をR1より大きくすれば高周波域でのGainを正にすることができるが,受動LPFではGainが正になることはない.