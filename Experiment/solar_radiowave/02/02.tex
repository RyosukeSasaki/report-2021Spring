\section{実験装置}
\subsection{装置の構成}
電波望遠鏡は受信部, IF信号処理系,検波器,DC増幅器,A/D変換器,測定用計算機からなる.
\mfig[width=12cm]{fig/05_text_Solar.eps}{実験装置のブロック図(実験テキストより引用)}
\subsubsection{受信部}
受信部はアンテナ,コンバーター,赤道儀,三脚からなる.

アンテナはBS放送受信用のTDK BCS-45DHVを用いている.
コンバーターはアンテナと一体になっており, $11.70$-$12.75\si{\giga\hertz}$帯の受信信号と
オシレーターからの$10.678\si{\giga\hertz}$信号を加算合成し,
LPFに通すことで$1022$-$2072\si{\mega\hertz}$の中間周波数信号を生成する.
このコンバーターには$54\si{\deci\bel}$利得があり,電源はIF信号処理系から同軸ケーブルで供給される.
室内機においてはパラボラアンテナは取り外され,集光部とコンバーターのみで運用する.

赤道儀はVixen GP2-399000を用いている.極軸は天の北極方向に合わせる必要があり,
今回は慶應義塾大学の緯度である北緯$35.5559\si{\degree}$に$\pm0.5\si{\degree}$の範囲で設定する.
\subsubsection{IF信号処理系}
IF信号処理系は主に信号増幅器から構成される.信号増幅器は一般的にBS・CS信号のブースターとして用いられる
マスプロ電工 BCB20LSを用いている.
このブースターは伝送周波数帯域$1000$-$2150\si{\mega\hertz}$において$18\si{\deci\bel}$以上の利得が保証されている.
また$+15\si{\volt}$は同軸ケーブルで供給され,電源にはマスプロ電工 BPS5Bを用いている.
\subsubsection{検波器}
検波器は交流の入力信号の包絡線を取り出し直流に変化する素子である.
今回は性能が補償された検波器としてKeysight 8472B Low-Barrier Schottky Diode Detectorと
Herotek DZR124AA Zero-Bias Schottky Diode Detectorを用いる.
それぞれの入力周波数範囲はDC-$18\si{\giga\hertz}$, $10$-$12.4\si{\giga\hertz}$であり,
入力電力に比例した負電流を出力する.
\subsubsection{DC増幅器}
前段の検波器の出力電流から得られる電圧出力は数十-数百$\si{\milli\volt}$だが,
後段のADCの測定レンジ$0$-$5\si{\milli\volt}$であることから更に増幅する必要がある.
検波器の出力は負電流であるので,反転増幅器を通すことで測定に適した正の電圧を得ることができる.
ここで増幅倍率は$(27\si{\kilo}\times51\si{\kilo})/(27\si{\kilo}+51\si{\kilo})/1\si{\kilo}=17.7$倍となる.
\mfig[width=10cm]{fig/inverted_amp.png}{反転増幅器}
\subsubsection{AD変換器,測定用計算機}
AD変換器としてArduino Uno (AVR ATmega328P)内臓のADCを用いる.これは測定レンジ$0$-$5\si{\volt}$, 分解能$4.89\si{\milli\volt}$である.
AD変換器で測定された電圧値は測定用計算機であるApple MacBook Air(室外機), Apple iMac(室内機)を用いて記録される.
データの記録はProcessing IDEで開発されたmizutamaソフトウェアによって行われる.
ADCのサンプルレートは$100\si{\hertz}$であり,これを$1\si{\second}$の時間間隔で積分したものを記録する.
\subsection{予備実験}
予備実験として室内機を用いてお湯の温度変化を電波望遠鏡を用いて測定した.
\subsubsection{実験方法}
アルミ缶に熱湯を入れ温度計を用いて初期温度を測定した.次に室内機の集光部をお湯に浸し,DC増幅器の出力電圧を記録した.
\subsubsection{実験結果}
図\ref{fig:hot_water}に測定結果を示す.
集光部への入力電力と検波器の出力電圧に完全な線形性をを仮定すると,
お湯の温度が低下するにしたがって黒体輻射の輻射強度が低下し,信号電圧も低下すると考えられる.
図\ref{fig:hot_water}から実際に温度の低下に伴って信号電圧が低下していることがわかる.
\begin{figure}[hptb]
\begin{center}
\begin{tikzpicture}[gnuplot]
%% generated with GNUPLOT 5.2p8 (Lua 5.3; terminal rev. Nov 2018, script rev. 108)
%% 2021年05月13日 15時25分02秒
\path (0.000,0.000) rectangle (12.000,8.000);
\gpcolor{color=gp lt color border}
\gpsetlinetype{gp lt border}
\gpsetdashtype{gp dt solid}
\gpsetlinewidth{1.00}
\draw[gp path] (1.320,0.985)--(1.500,0.985);
\draw[gp path] (11.447,0.985)--(11.267,0.985);
\node[gp node right] at (1.136,0.985) {$3$};
\draw[gp path] (1.320,1.943)--(1.410,1.943);
\draw[gp path] (11.447,1.943)--(11.357,1.943);
\draw[gp path] (1.320,2.901)--(1.500,2.901);
\draw[gp path] (11.447,2.901)--(11.267,2.901);
\node[gp node right] at (1.136,2.901) {$3.1$};
\draw[gp path] (1.320,3.859)--(1.410,3.859);
\draw[gp path] (11.447,3.859)--(11.357,3.859);
\draw[gp path] (1.320,4.817)--(1.500,4.817);
\draw[gp path] (11.447,4.817)--(11.267,4.817);
\node[gp node right] at (1.136,4.817) {$3.2$};
\draw[gp path] (1.320,5.775)--(1.410,5.775);
\draw[gp path] (11.447,5.775)--(11.357,5.775);
\draw[gp path] (1.320,6.733)--(1.500,6.733);
\draw[gp path] (11.447,6.733)--(11.267,6.733);
\node[gp node right] at (1.136,6.733) {$3.3$};
\draw[gp path] (1.320,7.691)--(1.410,7.691);
\draw[gp path] (11.447,7.691)--(11.357,7.691);
\draw[gp path] (1.320,0.985)--(1.320,1.165);
\draw[gp path] (1.320,7.691)--(1.320,7.511);
\node[gp node center] at (1.320,0.677) {$0$};
\draw[gp path] (2.526,0.985)--(2.526,1.075);
\draw[gp path] (2.526,7.691)--(2.526,7.601);
\draw[gp path] (3.731,0.985)--(3.731,1.165);
\draw[gp path] (3.731,7.691)--(3.731,7.511);
\node[gp node center] at (3.731,0.677) {$1000$};
\draw[gp path] (4.937,0.985)--(4.937,1.075);
\draw[gp path] (4.937,7.691)--(4.937,7.601);
\draw[gp path] (6.142,0.985)--(6.142,1.165);
\draw[gp path] (6.142,7.691)--(6.142,7.511);
\node[gp node center] at (6.142,0.677) {$2000$};
\draw[gp path] (7.348,0.985)--(7.348,1.075);
\draw[gp path] (7.348,7.691)--(7.348,7.601);
\draw[gp path] (8.554,0.985)--(8.554,1.165);
\draw[gp path] (8.554,7.691)--(8.554,7.511);
\node[gp node center] at (8.554,0.677) {$3000$};
\draw[gp path] (9.759,0.985)--(9.759,1.075);
\draw[gp path] (9.759,7.691)--(9.759,7.601);
\draw[gp path] (10.965,0.985)--(10.965,1.165);
\draw[gp path] (10.965,7.691)--(10.965,7.511);
\node[gp node center] at (10.965,0.677) {$4000$};
\draw[gp path] (1.320,7.691)--(1.320,0.985)--(11.447,0.985)--(11.447,7.691)--cycle;
\node[gp node center,rotate=-270] at (0.292,4.338) {電圧 / $\si{\volt}$};
\node[gp node center] at (6.383,0.215) {経過時間$t$ / $\si{\second}$};
\gpsetpointsize{4.00}
\gppoint{gp mark 1}{(1.325,6.412)}
\gppoint{gp mark 1}{(1.327,6.419)}
\gppoint{gp mark 1}{(1.330,6.396)}
\gppoint{gp mark 1}{(1.332,6.478)}
\gppoint{gp mark 1}{(1.334,6.398)}
\gppoint{gp mark 1}{(1.337,6.420)}
\gppoint{gp mark 1}{(1.339,6.467)}
\gppoint{gp mark 1}{(1.342,6.429)}
\gppoint{gp mark 1}{(1.344,6.342)}
\gppoint{gp mark 1}{(1.347,6.342)}
\gppoint{gp mark 1}{(1.349,6.450)}
\gppoint{gp mark 1}{(1.351,6.306)}
\gppoint{gp mark 1}{(1.354,6.424)}
\gppoint{gp mark 1}{(1.356,6.284)}
\gppoint{gp mark 1}{(1.359,6.258)}
\gppoint{gp mark 1}{(1.361,6.296)}
\gppoint{gp mark 1}{(1.363,6.262)}
\gppoint{gp mark 1}{(1.366,6.385)}
\gppoint{gp mark 1}{(1.368,6.186)}
\gppoint{gp mark 1}{(1.371,6.175)}
\gppoint{gp mark 1}{(1.373,6.244)}
\gppoint{gp mark 1}{(1.375,6.238)}
\gppoint{gp mark 1}{(1.378,6.220)}
\gppoint{gp mark 1}{(1.380,6.367)}
\gppoint{gp mark 1}{(1.383,6.260)}
\gppoint{gp mark 1}{(1.385,6.338)}
\gppoint{gp mark 1}{(1.388,6.148)}
\gppoint{gp mark 1}{(1.390,6.291)}
\gppoint{gp mark 1}{(1.392,6.129)}
\gppoint{gp mark 1}{(1.395,6.158)}
\gppoint{gp mark 1}{(1.397,6.256)}
\gppoint{gp mark 1}{(1.400,6.102)}
\gppoint{gp mark 1}{(1.402,6.102)}
\gppoint{gp mark 1}{(1.404,6.068)}
\gppoint{gp mark 1}{(1.407,6.029)}
\gppoint{gp mark 1}{(1.409,6.092)}
\gppoint{gp mark 1}{(1.412,6.166)}
\gppoint{gp mark 1}{(1.414,6.040)}
\gppoint{gp mark 1}{(1.416,6.093)}
\gppoint{gp mark 1}{(1.419,6.075)}
\gppoint{gp mark 1}{(1.421,6.069)}
\gppoint{gp mark 1}{(1.424,6.121)}
\gppoint{gp mark 1}{(1.426,5.989)}
\gppoint{gp mark 1}{(1.429,6.120)}
\gppoint{gp mark 1}{(1.431,6.087)}
\gppoint{gp mark 1}{(1.433,5.975)}
\gppoint{gp mark 1}{(1.436,6.037)}
\gppoint{gp mark 1}{(1.438,5.960)}
\gppoint{gp mark 1}{(1.441,5.938)}
\gppoint{gp mark 1}{(1.443,6.023)}
\gppoint{gp mark 1}{(1.445,5.900)}
\gppoint{gp mark 1}{(1.448,6.052)}
\gppoint{gp mark 1}{(1.450,5.949)}
\gppoint{gp mark 1}{(1.453,5.916)}
\gppoint{gp mark 1}{(1.455,5.976)}
\gppoint{gp mark 1}{(1.457,5.877)}
\gppoint{gp mark 1}{(1.460,5.958)}
\gppoint{gp mark 1}{(1.462,5.838)}
\gppoint{gp mark 1}{(1.465,5.998)}
\gppoint{gp mark 1}{(1.467,5.871)}
\gppoint{gp mark 1}{(1.469,5.840)}
\gppoint{gp mark 1}{(1.472,5.816)}
\gppoint{gp mark 1}{(1.474,5.998)}
\gppoint{gp mark 1}{(1.477,5.930)}
\gppoint{gp mark 1}{(1.479,5.790)}
\gppoint{gp mark 1}{(1.482,5.867)}
\gppoint{gp mark 1}{(1.484,5.759)}
\gppoint{gp mark 1}{(1.486,5.838)}
\gppoint{gp mark 1}{(1.489,5.781)}
\gppoint{gp mark 1}{(1.491,5.877)}
\gppoint{gp mark 1}{(1.494,5.699)}
\gppoint{gp mark 1}{(1.496,5.834)}
\gppoint{gp mark 1}{(1.498,5.853)}
\gppoint{gp mark 1}{(1.501,5.815)}
\gppoint{gp mark 1}{(1.503,5.723)}
\gppoint{gp mark 1}{(1.506,5.802)}
\gppoint{gp mark 1}{(1.508,5.795)}
\gppoint{gp mark 1}{(1.510,5.726)}
\gppoint{gp mark 1}{(1.513,5.747)}
\gppoint{gp mark 1}{(1.515,5.705)}
\gppoint{gp mark 1}{(1.518,5.740)}
\gppoint{gp mark 1}{(1.520,5.782)}
\gppoint{gp mark 1}{(1.523,5.697)}
\gppoint{gp mark 1}{(1.525,5.725)}
\gppoint{gp mark 1}{(1.527,5.664)}
\gppoint{gp mark 1}{(1.530,5.936)}
\gppoint{gp mark 1}{(1.532,5.729)}
\gppoint{gp mark 1}{(1.535,5.756)}
\gppoint{gp mark 1}{(1.537,5.685)}
\gppoint{gp mark 1}{(1.539,5.812)}
\gppoint{gp mark 1}{(1.542,5.831)}
\gppoint{gp mark 1}{(1.544,5.655)}
\gppoint{gp mark 1}{(1.547,5.661)}
\gppoint{gp mark 1}{(1.549,5.705)}
\gppoint{gp mark 1}{(1.551,5.657)}
\gppoint{gp mark 1}{(1.554,5.552)}
\gppoint{gp mark 1}{(1.556,5.703)}
\gppoint{gp mark 1}{(1.559,5.695)}
\gppoint{gp mark 1}{(1.561,5.790)}
\gppoint{gp mark 1}{(1.564,5.633)}
\gppoint{gp mark 1}{(1.566,5.786)}
\gppoint{gp mark 1}{(1.568,5.640)}
\gppoint{gp mark 1}{(1.571,5.586)}
\gppoint{gp mark 1}{(1.573,5.519)}
\gppoint{gp mark 1}{(1.576,5.588)}
\gppoint{gp mark 1}{(1.578,5.701)}
\gppoint{gp mark 1}{(1.580,5.556)}
\gppoint{gp mark 1}{(1.583,5.753)}
\gppoint{gp mark 1}{(1.585,5.567)}
\gppoint{gp mark 1}{(1.588,5.521)}
\gppoint{gp mark 1}{(1.590,5.600)}
\gppoint{gp mark 1}{(1.592,5.561)}
\gppoint{gp mark 1}{(1.595,5.697)}
\gppoint{gp mark 1}{(1.597,5.491)}
\gppoint{gp mark 1}{(1.600,5.535)}
\gppoint{gp mark 1}{(1.602,5.698)}
\gppoint{gp mark 1}{(1.605,5.587)}
\gppoint{gp mark 1}{(1.607,5.536)}
\gppoint{gp mark 1}{(1.609,5.519)}
\gppoint{gp mark 1}{(1.612,5.525)}
\gppoint{gp mark 1}{(1.614,5.517)}
\gppoint{gp mark 1}{(1.617,5.496)}
\gppoint{gp mark 1}{(1.619,5.564)}
\gppoint{gp mark 1}{(1.621,5.468)}
\gppoint{gp mark 1}{(1.624,5.482)}
\gppoint{gp mark 1}{(1.626,5.518)}
\gppoint{gp mark 1}{(1.629,5.458)}
\gppoint{gp mark 1}{(1.631,5.508)}
\gppoint{gp mark 1}{(1.633,5.589)}
\gppoint{gp mark 1}{(1.636,5.532)}
\gppoint{gp mark 1}{(1.638,5.519)}
\gppoint{gp mark 1}{(1.641,5.510)}
\gppoint{gp mark 1}{(1.643,5.538)}
\gppoint{gp mark 1}{(1.646,5.475)}
\gppoint{gp mark 1}{(1.648,5.593)}
\gppoint{gp mark 1}{(1.650,5.519)}
\gppoint{gp mark 1}{(1.653,5.555)}
\gppoint{gp mark 1}{(1.655,5.551)}
\gppoint{gp mark 1}{(1.658,5.502)}
\gppoint{gp mark 1}{(1.660,5.480)}
\gppoint{gp mark 1}{(1.662,5.481)}
\gppoint{gp mark 1}{(1.665,5.450)}
\gppoint{gp mark 1}{(1.667,5.443)}
\gppoint{gp mark 1}{(1.670,5.523)}
\gppoint{gp mark 1}{(1.672,5.218)}
\gppoint{gp mark 1}{(1.674,5.436)}
\gppoint{gp mark 1}{(1.677,5.313)}
\gppoint{gp mark 1}{(1.679,5.406)}
\gppoint{gp mark 1}{(1.682,5.448)}
\gppoint{gp mark 1}{(1.684,5.377)}
\gppoint{gp mark 1}{(1.687,5.271)}
\gppoint{gp mark 1}{(1.689,5.200)}
\gppoint{gp mark 1}{(1.691,5.370)}
\gppoint{gp mark 1}{(1.694,5.371)}
\gppoint{gp mark 1}{(1.696,5.411)}
\gppoint{gp mark 1}{(1.699,5.326)}
\gppoint{gp mark 1}{(1.701,5.395)}
\gppoint{gp mark 1}{(1.703,5.271)}
\gppoint{gp mark 1}{(1.706,5.346)}
\gppoint{gp mark 1}{(1.708,5.324)}
\gppoint{gp mark 1}{(1.711,5.329)}
\gppoint{gp mark 1}{(1.713,5.402)}
\gppoint{gp mark 1}{(1.715,5.314)}
\gppoint{gp mark 1}{(1.718,5.383)}
\gppoint{gp mark 1}{(1.720,5.309)}
\gppoint{gp mark 1}{(1.723,5.432)}
\gppoint{gp mark 1}{(1.725,5.355)}
\gppoint{gp mark 1}{(1.727,5.305)}
\gppoint{gp mark 1}{(1.730,5.433)}
\gppoint{gp mark 1}{(1.732,5.299)}
\gppoint{gp mark 1}{(1.735,5.311)}
\gppoint{gp mark 1}{(1.737,5.246)}
\gppoint{gp mark 1}{(1.740,5.198)}
\gppoint{gp mark 1}{(1.742,5.264)}
\gppoint{gp mark 1}{(1.744,5.376)}
\gppoint{gp mark 1}{(1.747,5.330)}
\gppoint{gp mark 1}{(1.749,5.332)}
\gppoint{gp mark 1}{(1.752,5.251)}
\gppoint{gp mark 1}{(1.754,5.330)}
\gppoint{gp mark 1}{(1.756,5.218)}
\gppoint{gp mark 1}{(1.759,5.368)}
\gppoint{gp mark 1}{(1.761,5.460)}
\gppoint{gp mark 1}{(1.764,5.162)}
\gppoint{gp mark 1}{(1.766,5.319)}
\gppoint{gp mark 1}{(1.768,5.168)}
\gppoint{gp mark 1}{(1.771,5.301)}
\gppoint{gp mark 1}{(1.773,5.311)}
\gppoint{gp mark 1}{(1.776,5.282)}
\gppoint{gp mark 1}{(1.778,5.193)}
\gppoint{gp mark 1}{(1.781,5.156)}
\gppoint{gp mark 1}{(1.783,5.002)}
\gppoint{gp mark 1}{(1.785,5.252)}
\gppoint{gp mark 1}{(1.788,5.062)}
\gppoint{gp mark 1}{(1.790,5.125)}
\gppoint{gp mark 1}{(1.793,5.197)}
\gppoint{gp mark 1}{(1.795,5.134)}
\gppoint{gp mark 1}{(1.797,4.992)}
\gppoint{gp mark 1}{(1.800,5.137)}
\gppoint{gp mark 1}{(1.802,5.137)}
\gppoint{gp mark 1}{(1.805,5.020)}
\gppoint{gp mark 1}{(1.807,4.977)}
\gppoint{gp mark 1}{(1.809,5.155)}
\gppoint{gp mark 1}{(1.812,5.090)}
\gppoint{gp mark 1}{(1.814,5.053)}
\gppoint{gp mark 1}{(1.817,5.100)}
\gppoint{gp mark 1}{(1.819,4.943)}
\gppoint{gp mark 1}{(1.822,5.046)}
\gppoint{gp mark 1}{(1.824,5.065)}
\gppoint{gp mark 1}{(1.826,5.015)}
\gppoint{gp mark 1}{(1.829,5.000)}
\gppoint{gp mark 1}{(1.831,4.971)}
\gppoint{gp mark 1}{(1.834,4.955)}
\gppoint{gp mark 1}{(1.836,4.948)}
\gppoint{gp mark 1}{(1.838,5.109)}
\gppoint{gp mark 1}{(1.841,4.996)}
\gppoint{gp mark 1}{(1.843,4.981)}
\gppoint{gp mark 1}{(1.846,4.989)}
\gppoint{gp mark 1}{(1.848,4.974)}
\gppoint{gp mark 1}{(1.850,5.024)}
\gppoint{gp mark 1}{(1.853,5.039)}
\gppoint{gp mark 1}{(1.855,5.011)}
\gppoint{gp mark 1}{(1.858,5.050)}
\gppoint{gp mark 1}{(1.860,4.961)}
\gppoint{gp mark 1}{(1.863,4.978)}
\gppoint{gp mark 1}{(1.865,4.914)}
\gppoint{gp mark 1}{(1.867,4.894)}
\gppoint{gp mark 1}{(1.870,4.956)}
\gppoint{gp mark 1}{(1.872,4.913)}
\gppoint{gp mark 1}{(1.875,4.913)}
\gppoint{gp mark 1}{(1.877,5.001)}
\gppoint{gp mark 1}{(1.879,4.982)}
\gppoint{gp mark 1}{(1.882,4.842)}
\gppoint{gp mark 1}{(1.884,5.030)}
\gppoint{gp mark 1}{(1.887,5.015)}
\gppoint{gp mark 1}{(1.889,4.965)}
\gppoint{gp mark 1}{(1.891,4.922)}
\gppoint{gp mark 1}{(1.894,5.067)}
\gppoint{gp mark 1}{(1.896,4.950)}
\gppoint{gp mark 1}{(1.899,4.861)}
\gppoint{gp mark 1}{(1.901,5.010)}
\gppoint{gp mark 1}{(1.904,4.916)}
\gppoint{gp mark 1}{(1.906,4.989)}
\gppoint{gp mark 1}{(1.908,4.892)}
\gppoint{gp mark 1}{(1.911,4.935)}
\gppoint{gp mark 1}{(1.913,5.022)}
\gppoint{gp mark 1}{(1.916,4.849)}
\gppoint{gp mark 1}{(1.918,5.024)}
\gppoint{gp mark 1}{(1.920,4.947)}
\gppoint{gp mark 1}{(1.923,4.957)}
\gppoint{gp mark 1}{(1.925,5.005)}
\gppoint{gp mark 1}{(1.928,5.008)}
\gppoint{gp mark 1}{(1.930,4.939)}
\gppoint{gp mark 1}{(1.932,4.906)}
\gppoint{gp mark 1}{(1.935,4.877)}
\gppoint{gp mark 1}{(1.937,4.910)}
\gppoint{gp mark 1}{(1.940,4.904)}
\gppoint{gp mark 1}{(1.942,4.881)}
\gppoint{gp mark 1}{(1.944,4.864)}
\gppoint{gp mark 1}{(1.947,4.803)}
\gppoint{gp mark 1}{(1.949,4.956)}
\gppoint{gp mark 1}{(1.952,4.902)}
\gppoint{gp mark 1}{(1.954,4.795)}
\gppoint{gp mark 1}{(1.957,4.824)}
\gppoint{gp mark 1}{(1.959,4.928)}
\gppoint{gp mark 1}{(1.961,4.846)}
\gppoint{gp mark 1}{(1.964,4.762)}
\gppoint{gp mark 1}{(1.966,4.931)}
\gppoint{gp mark 1}{(1.969,4.836)}
\gppoint{gp mark 1}{(1.971,4.926)}
\gppoint{gp mark 1}{(1.973,4.885)}
\gppoint{gp mark 1}{(1.976,4.917)}
\gppoint{gp mark 1}{(1.978,4.925)}
\gppoint{gp mark 1}{(1.981,4.983)}
\gppoint{gp mark 1}{(1.983,4.901)}
\gppoint{gp mark 1}{(1.985,4.824)}
\gppoint{gp mark 1}{(1.988,4.916)}
\gppoint{gp mark 1}{(1.990,4.906)}
\gppoint{gp mark 1}{(1.993,4.818)}
\gppoint{gp mark 1}{(1.995,4.794)}
\gppoint{gp mark 1}{(1.998,4.841)}
\gppoint{gp mark 1}{(2.000,4.841)}
\gppoint{gp mark 1}{(2.002,4.928)}
\gppoint{gp mark 1}{(2.005,4.869)}
\gppoint{gp mark 1}{(2.007,4.775)}
\gppoint{gp mark 1}{(2.010,4.839)}
\gppoint{gp mark 1}{(2.012,4.884)}
\gppoint{gp mark 1}{(2.014,4.846)}
\gppoint{gp mark 1}{(2.017,4.797)}
\gppoint{gp mark 1}{(2.019,4.753)}
\gppoint{gp mark 1}{(2.022,4.880)}
\gppoint{gp mark 1}{(2.024,4.786)}
\gppoint{gp mark 1}{(2.026,4.772)}
\gppoint{gp mark 1}{(2.029,4.899)}
\gppoint{gp mark 1}{(2.031,4.857)}
\gppoint{gp mark 1}{(2.034,4.943)}
\gppoint{gp mark 1}{(2.036,4.815)}
\gppoint{gp mark 1}{(2.039,4.907)}
\gppoint{gp mark 1}{(2.041,4.834)}
\gppoint{gp mark 1}{(2.043,4.852)}
\gppoint{gp mark 1}{(2.046,4.721)}
\gppoint{gp mark 1}{(2.048,4.797)}
\gppoint{gp mark 1}{(2.051,4.755)}
\gppoint{gp mark 1}{(2.053,4.827)}
\gppoint{gp mark 1}{(2.055,4.753)}
\gppoint{gp mark 1}{(2.058,4.787)}
\gppoint{gp mark 1}{(2.060,4.787)}
\gppoint{gp mark 1}{(2.063,4.939)}
\gppoint{gp mark 1}{(2.065,4.761)}
\gppoint{gp mark 1}{(2.067,4.662)}
\gppoint{gp mark 1}{(2.070,4.787)}
\gppoint{gp mark 1}{(2.072,4.685)}
\gppoint{gp mark 1}{(2.075,4.724)}
\gppoint{gp mark 1}{(2.077,4.712)}
\gppoint{gp mark 1}{(2.080,4.832)}
\gppoint{gp mark 1}{(2.082,4.711)}
\gppoint{gp mark 1}{(2.084,4.674)}
\gppoint{gp mark 1}{(2.087,4.678)}
\gppoint{gp mark 1}{(2.089,4.648)}
\gppoint{gp mark 1}{(2.092,4.806)}
\gppoint{gp mark 1}{(2.094,4.812)}
\gppoint{gp mark 1}{(2.096,4.711)}
\gppoint{gp mark 1}{(2.099,4.736)}
\gppoint{gp mark 1}{(2.101,4.918)}
\gppoint{gp mark 1}{(2.104,4.729)}
\gppoint{gp mark 1}{(2.106,4.712)}
\gppoint{gp mark 1}{(2.108,4.713)}
\gppoint{gp mark 1}{(2.111,4.716)}
\gppoint{gp mark 1}{(2.113,4.741)}
\gppoint{gp mark 1}{(2.116,4.699)}
\gppoint{gp mark 1}{(2.118,4.779)}
\gppoint{gp mark 1}{(2.121,4.700)}
\gppoint{gp mark 1}{(2.123,4.760)}
\gppoint{gp mark 1}{(2.125,4.654)}
\gppoint{gp mark 1}{(2.128,4.675)}
\gppoint{gp mark 1}{(2.130,4.760)}
\gppoint{gp mark 1}{(2.133,4.636)}
\gppoint{gp mark 1}{(2.135,4.699)}
\gppoint{gp mark 1}{(2.137,4.727)}
\gppoint{gp mark 1}{(2.140,4.633)}
\gppoint{gp mark 1}{(2.142,4.587)}
\gppoint{gp mark 1}{(2.145,4.732)}
\gppoint{gp mark 1}{(2.147,4.619)}
\gppoint{gp mark 1}{(2.149,4.589)}
\gppoint{gp mark 1}{(2.152,4.706)}
\gppoint{gp mark 1}{(2.154,4.727)}
\gppoint{gp mark 1}{(2.157,4.689)}
\gppoint{gp mark 1}{(2.159,4.687)}
\gppoint{gp mark 1}{(2.162,4.681)}
\gppoint{gp mark 1}{(2.164,4.588)}
\gppoint{gp mark 1}{(2.166,4.746)}
\gppoint{gp mark 1}{(2.169,4.665)}
\gppoint{gp mark 1}{(2.171,4.750)}
\gppoint{gp mark 1}{(2.174,4.685)}
\gppoint{gp mark 1}{(2.176,4.581)}
\gppoint{gp mark 1}{(2.178,4.657)}
\gppoint{gp mark 1}{(2.181,4.706)}
\gppoint{gp mark 1}{(2.183,4.563)}
\gppoint{gp mark 1}{(2.186,4.665)}
\gppoint{gp mark 1}{(2.188,4.685)}
\gppoint{gp mark 1}{(2.190,4.602)}
\gppoint{gp mark 1}{(2.193,4.694)}
\gppoint{gp mark 1}{(2.195,4.704)}
\gppoint{gp mark 1}{(2.198,4.597)}
\gppoint{gp mark 1}{(2.200,4.577)}
\gppoint{gp mark 1}{(2.202,4.544)}
\gppoint{gp mark 1}{(2.205,4.764)}
\gppoint{gp mark 1}{(2.207,4.811)}
\gppoint{gp mark 1}{(2.210,4.677)}
\gppoint{gp mark 1}{(2.212,4.679)}
\gppoint{gp mark 1}{(2.215,4.520)}
\gppoint{gp mark 1}{(2.217,4.579)}
\gppoint{gp mark 1}{(2.219,4.693)}
\gppoint{gp mark 1}{(2.222,4.597)}
\gppoint{gp mark 1}{(2.224,4.585)}
\gppoint{gp mark 1}{(2.227,4.567)}
\gppoint{gp mark 1}{(2.229,4.582)}
\gppoint{gp mark 1}{(2.231,4.682)}
\gppoint{gp mark 1}{(2.234,4.649)}
\gppoint{gp mark 1}{(2.236,4.719)}
\gppoint{gp mark 1}{(2.239,4.614)}
\gppoint{gp mark 1}{(2.241,4.635)}
\gppoint{gp mark 1}{(2.243,4.594)}
\gppoint{gp mark 1}{(2.246,4.711)}
\gppoint{gp mark 1}{(2.248,4.677)}
\gppoint{gp mark 1}{(2.251,4.611)}
\gppoint{gp mark 1}{(2.253,4.634)}
\gppoint{gp mark 1}{(2.256,4.555)}
\gppoint{gp mark 1}{(2.258,4.563)}
\gppoint{gp mark 1}{(2.260,4.545)}
\gppoint{gp mark 1}{(2.263,4.588)}
\gppoint{gp mark 1}{(2.265,4.578)}
\gppoint{gp mark 1}{(2.268,4.582)}
\gppoint{gp mark 1}{(2.270,4.611)}
\gppoint{gp mark 1}{(2.272,4.525)}
\gppoint{gp mark 1}{(2.275,4.656)}
\gppoint{gp mark 1}{(2.277,4.572)}
\gppoint{gp mark 1}{(2.280,4.560)}
\gppoint{gp mark 1}{(2.282,4.637)}
\gppoint{gp mark 1}{(2.284,4.441)}
\gppoint{gp mark 1}{(2.287,4.529)}
\gppoint{gp mark 1}{(2.289,4.657)}
\gppoint{gp mark 1}{(2.292,4.571)}
\gppoint{gp mark 1}{(2.294,4.528)}
\gppoint{gp mark 1}{(2.297,4.515)}
\gppoint{gp mark 1}{(2.299,4.530)}
\gppoint{gp mark 1}{(2.301,4.628)}
\gppoint{gp mark 1}{(2.304,4.534)}
\gppoint{gp mark 1}{(2.306,4.500)}
\gppoint{gp mark 1}{(2.309,4.567)}
\gppoint{gp mark 1}{(2.311,4.530)}
\gppoint{gp mark 1}{(2.313,4.500)}
\gppoint{gp mark 1}{(2.316,4.429)}
\gppoint{gp mark 1}{(2.318,4.692)}
\gppoint{gp mark 1}{(2.321,4.500)}
\gppoint{gp mark 1}{(2.323,4.522)}
\gppoint{gp mark 1}{(2.325,4.510)}
\gppoint{gp mark 1}{(2.328,4.490)}
\gppoint{gp mark 1}{(2.330,4.484)}
\gppoint{gp mark 1}{(2.333,4.589)}
\gppoint{gp mark 1}{(2.335,4.432)}
\gppoint{gp mark 1}{(2.338,4.536)}
\gppoint{gp mark 1}{(2.340,4.548)}
\gppoint{gp mark 1}{(2.342,4.447)}
\gppoint{gp mark 1}{(2.345,4.535)}
\gppoint{gp mark 1}{(2.347,4.469)}
\gppoint{gp mark 1}{(2.350,4.434)}
\gppoint{gp mark 1}{(2.352,4.476)}
\gppoint{gp mark 1}{(2.354,4.587)}
\gppoint{gp mark 1}{(2.357,4.534)}
\gppoint{gp mark 1}{(2.359,4.478)}
\gppoint{gp mark 1}{(2.362,4.520)}
\gppoint{gp mark 1}{(2.364,4.517)}
\gppoint{gp mark 1}{(2.366,4.573)}
\gppoint{gp mark 1}{(2.369,4.597)}
\gppoint{gp mark 1}{(2.371,4.508)}
\gppoint{gp mark 1}{(2.374,4.624)}
\gppoint{gp mark 1}{(2.376,4.412)}
\gppoint{gp mark 1}{(2.379,4.451)}
\gppoint{gp mark 1}{(2.381,4.486)}
\gppoint{gp mark 1}{(2.383,4.522)}
\gppoint{gp mark 1}{(2.386,4.419)}
\gppoint{gp mark 1}{(2.388,4.484)}
\gppoint{gp mark 1}{(2.391,4.535)}
\gppoint{gp mark 1}{(2.393,4.563)}
\gppoint{gp mark 1}{(2.395,4.433)}
\gppoint{gp mark 1}{(2.398,4.526)}
\gppoint{gp mark 1}{(2.400,4.587)}
\gppoint{gp mark 1}{(2.403,4.466)}
\gppoint{gp mark 1}{(2.405,4.550)}
\gppoint{gp mark 1}{(2.407,4.458)}
\gppoint{gp mark 1}{(2.410,4.435)}
\gppoint{gp mark 1}{(2.412,4.470)}
\gppoint{gp mark 1}{(2.415,4.463)}
\gppoint{gp mark 1}{(2.417,4.364)}
\gppoint{gp mark 1}{(2.420,4.435)}
\gppoint{gp mark 1}{(2.422,4.504)}
\gppoint{gp mark 1}{(2.424,4.515)}
\gppoint{gp mark 1}{(2.427,4.469)}
\gppoint{gp mark 1}{(2.429,4.505)}
\gppoint{gp mark 1}{(2.432,4.437)}
\gppoint{gp mark 1}{(2.434,4.440)}
\gppoint{gp mark 1}{(2.436,4.413)}
\gppoint{gp mark 1}{(2.439,4.408)}
\gppoint{gp mark 1}{(2.441,4.456)}
\gppoint{gp mark 1}{(2.444,4.459)}
\gppoint{gp mark 1}{(2.446,4.459)}
\gppoint{gp mark 1}{(2.448,4.458)}
\gppoint{gp mark 1}{(2.451,4.533)}
\gppoint{gp mark 1}{(2.453,4.471)}
\gppoint{gp mark 1}{(2.456,4.471)}
\gppoint{gp mark 1}{(2.458,4.479)}
\gppoint{gp mark 1}{(2.460,4.386)}
\gppoint{gp mark 1}{(2.463,4.371)}
\gppoint{gp mark 1}{(2.465,4.429)}
\gppoint{gp mark 1}{(2.468,4.529)}
\gppoint{gp mark 1}{(2.470,4.362)}
\gppoint{gp mark 1}{(2.473,4.500)}
\gppoint{gp mark 1}{(2.475,4.448)}
\gppoint{gp mark 1}{(2.477,4.413)}
\gppoint{gp mark 1}{(2.480,4.362)}
\gppoint{gp mark 1}{(2.482,4.420)}
\gppoint{gp mark 1}{(2.485,4.367)}
\gppoint{gp mark 1}{(2.487,4.397)}
\gppoint{gp mark 1}{(2.489,4.491)}
\gppoint{gp mark 1}{(2.492,4.433)}
\gppoint{gp mark 1}{(2.494,4.514)}
\gppoint{gp mark 1}{(2.497,4.454)}
\gppoint{gp mark 1}{(2.499,4.481)}
\gppoint{gp mark 1}{(2.501,4.419)}
\gppoint{gp mark 1}{(2.504,4.435)}
\gppoint{gp mark 1}{(2.506,4.367)}
\gppoint{gp mark 1}{(2.509,4.497)}
\gppoint{gp mark 1}{(2.511,4.588)}
\gppoint{gp mark 1}{(2.514,4.451)}
\gppoint{gp mark 1}{(2.516,4.362)}
\gppoint{gp mark 1}{(2.518,4.478)}
\gppoint{gp mark 1}{(2.521,4.318)}
\gppoint{gp mark 1}{(2.523,4.329)}
\gppoint{gp mark 1}{(2.526,4.320)}
\gppoint{gp mark 1}{(2.528,4.445)}
\gppoint{gp mark 1}{(2.530,4.328)}
\gppoint{gp mark 1}{(2.533,4.409)}
\gppoint{gp mark 1}{(2.535,4.370)}
\gppoint{gp mark 1}{(2.538,4.464)}
\gppoint{gp mark 1}{(2.540,4.339)}
\gppoint{gp mark 1}{(2.542,4.442)}
\gppoint{gp mark 1}{(2.545,4.331)}
\gppoint{gp mark 1}{(2.547,4.453)}
\gppoint{gp mark 1}{(2.550,4.334)}
\gppoint{gp mark 1}{(2.552,4.340)}
\gppoint{gp mark 1}{(2.555,4.397)}
\gppoint{gp mark 1}{(2.557,4.310)}
\gppoint{gp mark 1}{(2.559,4.356)}
\gppoint{gp mark 1}{(2.562,4.435)}
\gppoint{gp mark 1}{(2.564,4.471)}
\gppoint{gp mark 1}{(2.567,4.367)}
\gppoint{gp mark 1}{(2.569,4.350)}
\gppoint{gp mark 1}{(2.571,4.284)}
\gppoint{gp mark 1}{(2.574,4.387)}
\gppoint{gp mark 1}{(2.576,4.309)}
\gppoint{gp mark 1}{(2.579,4.425)}
\gppoint{gp mark 1}{(2.581,4.337)}
\gppoint{gp mark 1}{(2.583,4.430)}
\gppoint{gp mark 1}{(2.586,4.282)}
\gppoint{gp mark 1}{(2.588,4.339)}
\gppoint{gp mark 1}{(2.591,4.392)}
\gppoint{gp mark 1}{(2.593,4.404)}
\gppoint{gp mark 1}{(2.596,4.382)}
\gppoint{gp mark 1}{(2.598,4.363)}
\gppoint{gp mark 1}{(2.600,4.199)}
\gppoint{gp mark 1}{(2.603,4.244)}
\gppoint{gp mark 1}{(2.605,4.433)}
\gppoint{gp mark 1}{(2.608,4.452)}
\gppoint{gp mark 1}{(2.610,4.393)}
\gppoint{gp mark 1}{(2.612,4.330)}
\gppoint{gp mark 1}{(2.615,4.269)}
\gppoint{gp mark 1}{(2.617,4.376)}
\gppoint{gp mark 1}{(2.620,4.368)}
\gppoint{gp mark 1}{(2.622,4.471)}
\gppoint{gp mark 1}{(2.624,4.325)}
\gppoint{gp mark 1}{(2.627,4.279)}
\gppoint{gp mark 1}{(2.629,4.264)}
\gppoint{gp mark 1}{(2.632,4.335)}
\gppoint{gp mark 1}{(2.634,4.479)}
\gppoint{gp mark 1}{(2.637,4.284)}
\gppoint{gp mark 1}{(2.639,4.148)}
\gppoint{gp mark 1}{(2.641,4.352)}
\gppoint{gp mark 1}{(2.644,4.346)}
\gppoint{gp mark 1}{(2.646,4.390)}
\gppoint{gp mark 1}{(2.649,4.367)}
\gppoint{gp mark 1}{(2.651,4.293)}
\gppoint{gp mark 1}{(2.653,4.439)}
\gppoint{gp mark 1}{(2.656,4.323)}
\gppoint{gp mark 1}{(2.658,4.300)}
\gppoint{gp mark 1}{(2.661,4.379)}
\gppoint{gp mark 1}{(2.663,4.338)}
\gppoint{gp mark 1}{(2.665,4.351)}
\gppoint{gp mark 1}{(2.668,4.356)}
\gppoint{gp mark 1}{(2.670,4.307)}
\gppoint{gp mark 1}{(2.673,4.302)}
\gppoint{gp mark 1}{(2.675,4.355)}
\gppoint{gp mark 1}{(2.678,4.319)}
\gppoint{gp mark 1}{(2.680,4.285)}
\gppoint{gp mark 1}{(2.682,4.257)}
\gppoint{gp mark 1}{(2.685,4.167)}
\gppoint{gp mark 1}{(2.687,4.301)}
\gppoint{gp mark 1}{(2.690,4.309)}
\gppoint{gp mark 1}{(2.692,4.378)}
\gppoint{gp mark 1}{(2.694,4.359)}
\gppoint{gp mark 1}{(2.697,4.325)}
\gppoint{gp mark 1}{(2.699,4.258)}
\gppoint{gp mark 1}{(2.702,4.397)}
\gppoint{gp mark 1}{(2.704,4.370)}
\gppoint{gp mark 1}{(2.706,4.285)}
\gppoint{gp mark 1}{(2.709,4.364)}
\gppoint{gp mark 1}{(2.711,4.249)}
\gppoint{gp mark 1}{(2.714,4.271)}
\gppoint{gp mark 1}{(2.716,4.315)}
\gppoint{gp mark 1}{(2.718,4.223)}
\gppoint{gp mark 1}{(2.721,4.387)}
\gppoint{gp mark 1}{(2.723,4.228)}
\gppoint{gp mark 1}{(2.726,4.441)}
\gppoint{gp mark 1}{(2.728,4.371)}
\gppoint{gp mark 1}{(2.731,4.810)}
\gppoint{gp mark 1}{(2.733,4.900)}
\gppoint{gp mark 1}{(2.735,4.789)}
\gppoint{gp mark 1}{(2.738,4.618)}
\gppoint{gp mark 1}{(2.740,4.616)}
\gppoint{gp mark 1}{(2.743,4.507)}
\gppoint{gp mark 1}{(2.745,4.563)}
\gppoint{gp mark 1}{(2.747,4.755)}
\gppoint{gp mark 1}{(2.750,4.725)}
\gppoint{gp mark 1}{(2.752,4.548)}
\gppoint{gp mark 1}{(2.755,4.521)}
\gppoint{gp mark 1}{(2.757,4.586)}
\gppoint{gp mark 1}{(2.759,4.567)}
\gppoint{gp mark 1}{(2.762,4.541)}
\gppoint{gp mark 1}{(2.764,4.665)}
\gppoint{gp mark 1}{(2.767,4.669)}
\gppoint{gp mark 1}{(2.769,4.589)}
\gppoint{gp mark 1}{(2.772,4.563)}
\gppoint{gp mark 1}{(2.774,4.564)}
\gppoint{gp mark 1}{(2.776,4.626)}
\gppoint{gp mark 1}{(2.779,4.530)}
\gppoint{gp mark 1}{(2.781,4.683)}
\gppoint{gp mark 1}{(2.784,4.606)}
\gppoint{gp mark 1}{(2.786,4.522)}
\gppoint{gp mark 1}{(2.788,4.620)}
\gppoint{gp mark 1}{(2.791,4.580)}
\gppoint{gp mark 1}{(2.793,4.564)}
\gppoint{gp mark 1}{(2.796,4.490)}
\gppoint{gp mark 1}{(2.798,4.476)}
\gppoint{gp mark 1}{(2.800,4.596)}
\gppoint{gp mark 1}{(2.803,4.594)}
\gppoint{gp mark 1}{(2.805,4.553)}
\gppoint{gp mark 1}{(2.808,4.279)}
\gppoint{gp mark 1}{(2.810,4.292)}
\gppoint{gp mark 1}{(2.813,4.281)}
\gppoint{gp mark 1}{(2.815,4.318)}
\gppoint{gp mark 1}{(2.817,4.251)}
\gppoint{gp mark 1}{(2.820,4.316)}
\gppoint{gp mark 1}{(2.822,4.272)}
\gppoint{gp mark 1}{(2.825,4.214)}
\gppoint{gp mark 1}{(2.827,4.122)}
\gppoint{gp mark 1}{(2.829,4.235)}
\gppoint{gp mark 1}{(2.832,4.338)}
\gppoint{gp mark 1}{(2.834,4.291)}
\gppoint{gp mark 1}{(2.837,4.120)}
\gppoint{gp mark 1}{(2.839,4.271)}
\gppoint{gp mark 1}{(2.841,4.194)}
\gppoint{gp mark 1}{(2.844,4.248)}
\gppoint{gp mark 1}{(2.846,4.418)}
\gppoint{gp mark 1}{(2.849,4.133)}
\gppoint{gp mark 1}{(2.851,4.208)}
\gppoint{gp mark 1}{(2.854,4.165)}
\gppoint{gp mark 1}{(2.856,4.366)}
\gppoint{gp mark 1}{(2.858,4.326)}
\gppoint{gp mark 1}{(2.861,4.273)}
\gppoint{gp mark 1}{(2.863,4.260)}
\gppoint{gp mark 1}{(2.866,4.246)}
\gppoint{gp mark 1}{(2.868,4.233)}
\gppoint{gp mark 1}{(2.870,4.248)}
\gppoint{gp mark 1}{(2.873,4.222)}
\gppoint{gp mark 1}{(2.875,4.283)}
\gppoint{gp mark 1}{(2.878,4.153)}
\gppoint{gp mark 1}{(2.880,4.260)}
\gppoint{gp mark 1}{(2.882,4.205)}
\gppoint{gp mark 1}{(2.885,4.331)}
\gppoint{gp mark 1}{(2.887,4.334)}
\gppoint{gp mark 1}{(2.890,4.209)}
\gppoint{gp mark 1}{(2.892,4.226)}
\gppoint{gp mark 1}{(2.895,4.094)}
\gppoint{gp mark 1}{(2.897,4.134)}
\gppoint{gp mark 1}{(2.899,4.220)}
\gppoint{gp mark 1}{(2.902,4.236)}
\gppoint{gp mark 1}{(2.904,4.146)}
\gppoint{gp mark 1}{(2.907,4.170)}
\gppoint{gp mark 1}{(2.909,4.171)}
\gppoint{gp mark 1}{(2.911,4.156)}
\gppoint{gp mark 1}{(2.914,4.211)}
\gppoint{gp mark 1}{(2.916,4.125)}
\gppoint{gp mark 1}{(2.919,4.261)}
\gppoint{gp mark 1}{(2.921,4.137)}
\gppoint{gp mark 1}{(2.923,4.254)}
\gppoint{gp mark 1}{(2.926,4.264)}
\gppoint{gp mark 1}{(2.928,4.189)}
\gppoint{gp mark 1}{(2.931,4.146)}
\gppoint{gp mark 1}{(2.933,4.321)}
\gppoint{gp mark 1}{(2.935,4.201)}
\gppoint{gp mark 1}{(2.938,4.183)}
\gppoint{gp mark 1}{(2.940,4.070)}
\gppoint{gp mark 1}{(2.943,4.093)}
\gppoint{gp mark 1}{(2.945,4.218)}
\gppoint{gp mark 1}{(2.948,4.112)}
\gppoint{gp mark 1}{(2.950,4.167)}
\gppoint{gp mark 1}{(2.952,4.194)}
\gppoint{gp mark 1}{(2.955,4.180)}
\gppoint{gp mark 1}{(2.957,4.282)}
\gppoint{gp mark 1}{(2.960,4.260)}
\gppoint{gp mark 1}{(2.962,4.117)}
\gppoint{gp mark 1}{(2.964,4.188)}
\gppoint{gp mark 1}{(2.967,4.121)}
\gppoint{gp mark 1}{(2.969,4.279)}
\gppoint{gp mark 1}{(2.972,4.181)}
\gppoint{gp mark 1}{(2.974,4.233)}
\gppoint{gp mark 1}{(2.976,4.142)}
\gppoint{gp mark 1}{(2.979,4.117)}
\gppoint{gp mark 1}{(2.981,4.159)}
\gppoint{gp mark 1}{(2.984,4.193)}
\gppoint{gp mark 1}{(2.986,4.199)}
\gppoint{gp mark 1}{(2.989,4.143)}
\gppoint{gp mark 1}{(2.991,4.266)}
\gppoint{gp mark 1}{(2.993,4.158)}
\gppoint{gp mark 1}{(2.996,4.176)}
\gppoint{gp mark 1}{(2.998,4.157)}
\gppoint{gp mark 1}{(3.001,4.025)}
\gppoint{gp mark 1}{(3.003,4.297)}
\gppoint{gp mark 1}{(3.005,4.226)}
\gppoint{gp mark 1}{(3.008,4.179)}
\gppoint{gp mark 1}{(3.010,4.186)}
\gppoint{gp mark 1}{(3.013,3.969)}
\gppoint{gp mark 1}{(3.015,4.247)}
\gppoint{gp mark 1}{(3.017,4.060)}
\gppoint{gp mark 1}{(3.020,4.173)}
\gppoint{gp mark 1}{(3.022,4.209)}
\gppoint{gp mark 1}{(3.025,4.098)}
\gppoint{gp mark 1}{(3.027,4.115)}
\gppoint{gp mark 1}{(3.030,4.112)}
\gppoint{gp mark 1}{(3.032,3.985)}
\gppoint{gp mark 1}{(3.034,4.146)}
\gppoint{gp mark 1}{(3.037,4.008)}
\gppoint{gp mark 1}{(3.039,4.162)}
\gppoint{gp mark 1}{(3.042,4.174)}
\gppoint{gp mark 1}{(3.044,4.031)}
\gppoint{gp mark 1}{(3.046,4.009)}
\gppoint{gp mark 1}{(3.049,4.098)}
\gppoint{gp mark 1}{(3.051,4.172)}
\gppoint{gp mark 1}{(3.054,4.076)}
\gppoint{gp mark 1}{(3.056,4.215)}
\gppoint{gp mark 1}{(3.058,4.198)}
\gppoint{gp mark 1}{(3.061,3.958)}
\gppoint{gp mark 1}{(3.063,4.160)}
\gppoint{gp mark 1}{(3.066,4.113)}
\gppoint{gp mark 1}{(3.068,3.997)}
\gppoint{gp mark 1}{(3.071,4.128)}
\gppoint{gp mark 1}{(3.073,4.149)}
\gppoint{gp mark 1}{(3.075,4.160)}
\gppoint{gp mark 1}{(3.078,4.098)}
\gppoint{gp mark 1}{(3.080,4.032)}
\gppoint{gp mark 1}{(3.083,4.060)}
\gppoint{gp mark 1}{(3.085,4.065)}
\gppoint{gp mark 1}{(3.087,4.205)}
\gppoint{gp mark 1}{(3.090,4.016)}
\gppoint{gp mark 1}{(3.092,4.034)}
\gppoint{gp mark 1}{(3.095,4.050)}
\gppoint{gp mark 1}{(3.097,4.103)}
\gppoint{gp mark 1}{(3.099,4.023)}
\gppoint{gp mark 1}{(3.102,4.097)}
\gppoint{gp mark 1}{(3.104,4.067)}
\gppoint{gp mark 1}{(3.107,4.017)}
\gppoint{gp mark 1}{(3.109,4.121)}
\gppoint{gp mark 1}{(3.112,4.150)}
\gppoint{gp mark 1}{(3.114,3.955)}
\gppoint{gp mark 1}{(3.116,4.002)}
\gppoint{gp mark 1}{(3.119,4.007)}
\gppoint{gp mark 1}{(3.121,4.128)}
\gppoint{gp mark 1}{(3.124,4.125)}
\gppoint{gp mark 1}{(3.126,4.025)}
\gppoint{gp mark 1}{(3.128,4.038)}
\gppoint{gp mark 1}{(3.131,4.110)}
\gppoint{gp mark 1}{(3.133,3.996)}
\gppoint{gp mark 1}{(3.136,4.089)}
\gppoint{gp mark 1}{(3.138,4.124)}
\gppoint{gp mark 1}{(3.140,3.982)}
\gppoint{gp mark 1}{(3.143,4.066)}
\gppoint{gp mark 1}{(3.145,4.044)}
\gppoint{gp mark 1}{(3.148,4.058)}
\gppoint{gp mark 1}{(3.150,4.154)}
\gppoint{gp mark 1}{(3.153,4.056)}
\gppoint{gp mark 1}{(3.155,4.050)}
\gppoint{gp mark 1}{(3.157,4.075)}
\gppoint{gp mark 1}{(3.160,4.092)}
\gppoint{gp mark 1}{(3.162,4.112)}
\gppoint{gp mark 1}{(3.165,3.872)}
\gppoint{gp mark 1}{(3.167,4.052)}
\gppoint{gp mark 1}{(3.169,4.105)}
\gppoint{gp mark 1}{(3.172,4.034)}
\gppoint{gp mark 1}{(3.174,4.006)}
\gppoint{gp mark 1}{(3.177,4.139)}
\gppoint{gp mark 1}{(3.179,4.044)}
\gppoint{gp mark 1}{(3.181,3.967)}
\gppoint{gp mark 1}{(3.184,4.008)}
\gppoint{gp mark 1}{(3.186,4.113)}
\gppoint{gp mark 1}{(3.189,4.091)}
\gppoint{gp mark 1}{(3.191,4.009)}
\gppoint{gp mark 1}{(3.193,3.999)}
\gppoint{gp mark 1}{(3.196,3.997)}
\gppoint{gp mark 1}{(3.198,4.067)}
\gppoint{gp mark 1}{(3.201,4.047)}
\gppoint{gp mark 1}{(3.203,4.063)}
\gppoint{gp mark 1}{(3.206,4.187)}
\gppoint{gp mark 1}{(3.208,4.024)}
\gppoint{gp mark 1}{(3.210,3.979)}
\gppoint{gp mark 1}{(3.213,4.034)}
\gppoint{gp mark 1}{(3.215,4.127)}
\gppoint{gp mark 1}{(3.218,4.087)}
\gppoint{gp mark 1}{(3.220,3.901)}
\gppoint{gp mark 1}{(3.222,3.999)}
\gppoint{gp mark 1}{(3.225,4.004)}
\gppoint{gp mark 1}{(3.227,4.018)}
\gppoint{gp mark 1}{(3.230,3.901)}
\gppoint{gp mark 1}{(3.232,3.973)}
\gppoint{gp mark 1}{(3.234,4.059)}
\gppoint{gp mark 1}{(3.237,3.911)}
\gppoint{gp mark 1}{(3.239,4.014)}
\gppoint{gp mark 1}{(3.242,3.980)}
\gppoint{gp mark 1}{(3.244,3.949)}
\gppoint{gp mark 1}{(3.247,3.999)}
\gppoint{gp mark 1}{(3.249,4.014)}
\gppoint{gp mark 1}{(3.251,3.963)}
\gppoint{gp mark 1}{(3.254,3.976)}
\gppoint{gp mark 1}{(3.256,4.112)}
\gppoint{gp mark 1}{(3.259,4.028)}
\gppoint{gp mark 1}{(3.261,3.933)}
\gppoint{gp mark 1}{(3.263,4.020)}
\gppoint{gp mark 1}{(3.266,3.955)}
\gppoint{gp mark 1}{(3.268,4.042)}
\gppoint{gp mark 1}{(3.271,3.971)}
\gppoint{gp mark 1}{(3.273,3.979)}
\gppoint{gp mark 1}{(3.275,4.091)}
\gppoint{gp mark 1}{(3.278,3.991)}
\gppoint{gp mark 1}{(3.280,4.126)}
\gppoint{gp mark 1}{(3.283,4.102)}
\gppoint{gp mark 1}{(3.285,3.909)}
\gppoint{gp mark 1}{(3.288,3.982)}
\gppoint{gp mark 1}{(3.290,4.089)}
\gppoint{gp mark 1}{(3.292,3.956)}
\gppoint{gp mark 1}{(3.295,4.010)}
\gppoint{gp mark 1}{(3.297,3.947)}
\gppoint{gp mark 1}{(3.300,3.973)}
\gppoint{gp mark 1}{(3.302,4.016)}
\gppoint{gp mark 1}{(3.304,4.016)}
\gppoint{gp mark 1}{(3.307,4.067)}
\gppoint{gp mark 1}{(3.309,4.028)}
\gppoint{gp mark 1}{(3.312,3.971)}
\gppoint{gp mark 1}{(3.314,3.980)}
\gppoint{gp mark 1}{(3.316,3.898)}
\gppoint{gp mark 1}{(3.319,3.873)}
\gppoint{gp mark 1}{(3.321,3.897)}
\gppoint{gp mark 1}{(3.324,3.998)}
\gppoint{gp mark 1}{(3.326,3.994)}
\gppoint{gp mark 1}{(3.329,3.948)}
\gppoint{gp mark 1}{(3.331,3.957)}
\gppoint{gp mark 1}{(3.333,3.948)}
\gppoint{gp mark 1}{(3.336,4.056)}
\gppoint{gp mark 1}{(3.338,3.972)}
\gppoint{gp mark 1}{(3.341,4.043)}
\gppoint{gp mark 1}{(3.343,3.859)}
\gppoint{gp mark 1}{(3.345,3.975)}
\gppoint{gp mark 1}{(3.348,4.051)}
\gppoint{gp mark 1}{(3.350,3.989)}
\gppoint{gp mark 1}{(3.353,3.944)}
\gppoint{gp mark 1}{(3.355,3.967)}
\gppoint{gp mark 1}{(3.357,3.999)}
\gppoint{gp mark 1}{(3.360,3.991)}
\gppoint{gp mark 1}{(3.362,3.941)}
\gppoint{gp mark 1}{(3.365,4.041)}
\gppoint{gp mark 1}{(3.367,4.038)}
\gppoint{gp mark 1}{(3.370,3.932)}
\gppoint{gp mark 1}{(3.372,4.006)}
\gppoint{gp mark 1}{(3.374,3.877)}
\gppoint{gp mark 1}{(3.377,4.038)}
\gppoint{gp mark 1}{(3.379,3.958)}
\gppoint{gp mark 1}{(3.382,4.072)}
\gppoint{gp mark 1}{(3.384,3.899)}
\gppoint{gp mark 1}{(3.386,4.013)}
\gppoint{gp mark 1}{(3.389,3.914)}
\gppoint{gp mark 1}{(3.391,3.987)}
\gppoint{gp mark 1}{(3.394,4.056)}
\gppoint{gp mark 1}{(3.396,3.930)}
\gppoint{gp mark 1}{(3.398,3.904)}
\gppoint{gp mark 1}{(3.401,3.930)}
\gppoint{gp mark 1}{(3.403,3.898)}
\gppoint{gp mark 1}{(3.406,4.014)}
\gppoint{gp mark 1}{(3.408,3.873)}
\gppoint{gp mark 1}{(3.411,4.069)}
\gppoint{gp mark 1}{(3.413,3.984)}
\gppoint{gp mark 1}{(3.415,3.899)}
\gppoint{gp mark 1}{(3.418,3.875)}
\gppoint{gp mark 1}{(3.420,4.057)}
\gppoint{gp mark 1}{(3.423,3.926)}
\gppoint{gp mark 1}{(3.425,3.938)}
\gppoint{gp mark 1}{(3.427,3.925)}
\gppoint{gp mark 1}{(3.430,4.006)}
\gppoint{gp mark 1}{(3.432,3.893)}
\gppoint{gp mark 1}{(3.435,3.878)}
\gppoint{gp mark 1}{(3.437,3.940)}
\gppoint{gp mark 1}{(3.439,3.953)}
\gppoint{gp mark 1}{(3.442,3.834)}
\gppoint{gp mark 1}{(3.444,3.826)}
\gppoint{gp mark 1}{(3.447,3.879)}
\gppoint{gp mark 1}{(3.449,3.937)}
\gppoint{gp mark 1}{(3.451,3.955)}
\gppoint{gp mark 1}{(3.454,3.882)}
\gppoint{gp mark 1}{(3.456,3.819)}
\gppoint{gp mark 1}{(3.459,3.904)}
\gppoint{gp mark 1}{(3.461,3.926)}
\gppoint{gp mark 1}{(3.464,3.892)}
\gppoint{gp mark 1}{(3.466,3.927)}
\gppoint{gp mark 1}{(3.468,3.982)}
\gppoint{gp mark 1}{(3.471,3.955)}
\gppoint{gp mark 1}{(3.473,3.941)}
\gppoint{gp mark 1}{(3.476,3.976)}
\gppoint{gp mark 1}{(3.478,3.897)}
\gppoint{gp mark 1}{(3.480,3.931)}
\gppoint{gp mark 1}{(3.483,3.814)}
\gppoint{gp mark 1}{(3.485,3.909)}
\gppoint{gp mark 1}{(3.488,3.975)}
\gppoint{gp mark 1}{(3.490,3.966)}
\gppoint{gp mark 1}{(3.492,3.921)}
\gppoint{gp mark 1}{(3.495,3.847)}
\gppoint{gp mark 1}{(3.497,3.841)}
\gppoint{gp mark 1}{(3.500,3.898)}
\gppoint{gp mark 1}{(3.502,3.892)}
\gppoint{gp mark 1}{(3.505,3.866)}
\gppoint{gp mark 1}{(3.507,4.003)}
\gppoint{gp mark 1}{(3.509,3.824)}
\gppoint{gp mark 1}{(3.512,3.857)}
\gppoint{gp mark 1}{(3.514,3.963)}
\gppoint{gp mark 1}{(3.517,3.810)}
\gppoint{gp mark 1}{(3.519,3.962)}
\gppoint{gp mark 1}{(3.521,3.994)}
\gppoint{gp mark 1}{(3.524,3.865)}
\gppoint{gp mark 1}{(3.526,3.911)}
\gppoint{gp mark 1}{(3.529,4.003)}
\gppoint{gp mark 1}{(3.531,3.873)}
\gppoint{gp mark 1}{(3.533,3.972)}
\gppoint{gp mark 1}{(3.536,3.910)}
\gppoint{gp mark 1}{(3.538,3.885)}
\gppoint{gp mark 1}{(3.541,3.933)}
\gppoint{gp mark 1}{(3.543,3.763)}
\gppoint{gp mark 1}{(3.546,3.793)}
\gppoint{gp mark 1}{(3.548,3.807)}
\gppoint{gp mark 1}{(3.550,3.881)}
\gppoint{gp mark 1}{(3.553,3.830)}
\gppoint{gp mark 1}{(3.555,3.927)}
\gppoint{gp mark 1}{(3.558,3.951)}
\gppoint{gp mark 1}{(3.560,3.877)}
\gppoint{gp mark 1}{(3.562,3.988)}
\gppoint{gp mark 1}{(3.565,3.975)}
\gppoint{gp mark 1}{(3.567,3.740)}
\gppoint{gp mark 1}{(3.570,3.896)}
\gppoint{gp mark 1}{(3.572,3.895)}
\gppoint{gp mark 1}{(3.574,3.851)}
\gppoint{gp mark 1}{(3.577,3.780)}
\gppoint{gp mark 1}{(3.579,3.820)}
\gppoint{gp mark 1}{(3.582,3.770)}
\gppoint{gp mark 1}{(3.584,3.982)}
\gppoint{gp mark 1}{(3.587,3.956)}
\gppoint{gp mark 1}{(3.589,3.921)}
\gppoint{gp mark 1}{(3.591,3.941)}
\gppoint{gp mark 1}{(3.594,3.837)}
\gppoint{gp mark 1}{(3.596,3.907)}
\gppoint{gp mark 1}{(3.599,3.955)}
\gppoint{gp mark 1}{(3.601,4.059)}
\gppoint{gp mark 1}{(3.603,3.926)}
\gppoint{gp mark 1}{(3.606,3.923)}
\gppoint{gp mark 1}{(3.608,3.953)}
\gppoint{gp mark 1}{(3.611,4.010)}
\gppoint{gp mark 1}{(3.613,3.946)}
\gppoint{gp mark 1}{(3.615,3.808)}
\gppoint{gp mark 1}{(3.618,3.857)}
\gppoint{gp mark 1}{(3.620,3.948)}
\gppoint{gp mark 1}{(3.623,3.898)}
\gppoint{gp mark 1}{(3.625,4.273)}
\gppoint{gp mark 1}{(3.628,4.115)}
\gppoint{gp mark 1}{(3.630,4.364)}
\gppoint{gp mark 1}{(3.632,4.105)}
\gppoint{gp mark 1}{(3.635,4.123)}
\gppoint{gp mark 1}{(3.637,4.089)}
\gppoint{gp mark 1}{(3.640,4.271)}
\gppoint{gp mark 1}{(3.642,4.246)}
\gppoint{gp mark 1}{(3.644,4.289)}
\gppoint{gp mark 1}{(3.647,4.155)}
\gppoint{gp mark 1}{(3.649,4.235)}
\gppoint{gp mark 1}{(3.652,4.214)}
\gppoint{gp mark 1}{(3.654,4.166)}
\gppoint{gp mark 1}{(3.656,4.311)}
\gppoint{gp mark 1}{(3.659,4.199)}
\gppoint{gp mark 1}{(3.661,4.156)}
\gppoint{gp mark 1}{(3.664,4.291)}
\gppoint{gp mark 1}{(3.666,4.047)}
\gppoint{gp mark 1}{(3.668,4.132)}
\gppoint{gp mark 1}{(3.671,4.156)}
\gppoint{gp mark 1}{(3.673,4.261)}
\gppoint{gp mark 1}{(3.676,4.237)}
\gppoint{gp mark 1}{(3.678,4.384)}
\gppoint{gp mark 1}{(3.681,4.414)}
\gppoint{gp mark 1}{(3.683,4.252)}
\gppoint{gp mark 1}{(3.685,4.457)}
\gppoint{gp mark 1}{(3.688,4.457)}
\gppoint{gp mark 1}{(3.690,4.672)}
\gppoint{gp mark 1}{(3.693,4.780)}
\gppoint{gp mark 1}{(3.695,5.208)}
\gppoint{gp mark 1}{(3.697,5.514)}
\gppoint{gp mark 1}{(3.700,5.782)}
\gppoint{gp mark 1}{(3.702,6.064)}
\gppoint{gp mark 1}{(3.705,6.263)}
\gppoint{gp mark 1}{(3.707,6.509)}
\gppoint{gp mark 1}{(3.709,6.451)}
\gppoint{gp mark 1}{(3.712,6.591)}
\gppoint{gp mark 1}{(3.714,6.546)}
\gppoint{gp mark 1}{(3.717,6.550)}
\gppoint{gp mark 1}{(3.719,6.618)}
\gppoint{gp mark 1}{(3.722,6.644)}
\gppoint{gp mark 1}{(3.724,6.681)}
\gppoint{gp mark 1}{(3.726,6.720)}
\gppoint{gp mark 1}{(3.729,6.654)}
\gppoint{gp mark 1}{(3.731,6.506)}
\gppoint{gp mark 1}{(3.734,6.681)}
\gppoint{gp mark 1}{(3.736,6.712)}
\gppoint{gp mark 1}{(3.738,6.676)}
\gppoint{gp mark 1}{(3.741,6.729)}
\gppoint{gp mark 1}{(3.743,6.705)}
\gppoint{gp mark 1}{(3.746,6.641)}
\gppoint{gp mark 1}{(3.748,6.624)}
\gppoint{gp mark 1}{(3.750,6.711)}
\gppoint{gp mark 1}{(3.753,6.651)}
\gppoint{gp mark 1}{(3.755,6.829)}
\gppoint{gp mark 1}{(3.758,6.671)}
\gppoint{gp mark 1}{(3.760,6.619)}
\gppoint{gp mark 1}{(3.763,6.665)}
\gppoint{gp mark 1}{(3.765,6.608)}
\gppoint{gp mark 1}{(3.767,6.567)}
\gppoint{gp mark 1}{(3.770,6.525)}
\gppoint{gp mark 1}{(3.772,6.785)}
\gppoint{gp mark 1}{(3.775,6.461)}
\gppoint{gp mark 1}{(3.777,6.656)}
\gppoint{gp mark 1}{(3.779,6.692)}
\gppoint{gp mark 1}{(3.782,6.588)}
\gppoint{gp mark 1}{(3.784,6.625)}
\gppoint{gp mark 1}{(3.787,6.619)}
\gppoint{gp mark 1}{(3.789,6.655)}
\gppoint{gp mark 1}{(3.791,6.710)}
\gppoint{gp mark 1}{(3.794,6.593)}
\gppoint{gp mark 1}{(3.796,6.479)}
\gppoint{gp mark 1}{(3.799,6.586)}
\gppoint{gp mark 1}{(3.801,6.493)}
\gppoint{gp mark 1}{(3.804,6.616)}
\gppoint{gp mark 1}{(3.806,6.494)}
\gppoint{gp mark 1}{(3.808,6.569)}
\gppoint{gp mark 1}{(3.811,6.695)}
\gppoint{gp mark 1}{(3.813,6.711)}
\gppoint{gp mark 1}{(3.816,6.567)}
\gppoint{gp mark 1}{(3.818,6.654)}
\gppoint{gp mark 1}{(3.820,6.642)}
\gppoint{gp mark 1}{(3.823,6.604)}
\gppoint{gp mark 1}{(3.825,6.623)}
\gppoint{gp mark 1}{(3.828,6.647)}
\gppoint{gp mark 1}{(3.830,6.718)}
\gppoint{gp mark 1}{(3.832,6.593)}
\gppoint{gp mark 1}{(3.835,6.671)}
\gppoint{gp mark 1}{(3.837,6.670)}
\gppoint{gp mark 1}{(3.840,6.726)}
\gppoint{gp mark 1}{(3.842,6.710)}
\gppoint{gp mark 1}{(3.845,6.658)}
\gppoint{gp mark 1}{(3.847,6.584)}
\gppoint{gp mark 1}{(3.849,6.573)}
\gppoint{gp mark 1}{(3.852,6.509)}
\gppoint{gp mark 1}{(3.854,6.677)}
\gppoint{gp mark 1}{(3.857,6.664)}
\gppoint{gp mark 1}{(3.859,6.614)}
\gppoint{gp mark 1}{(3.861,6.582)}
\gppoint{gp mark 1}{(3.864,6.745)}
\gppoint{gp mark 1}{(3.866,6.610)}
\gppoint{gp mark 1}{(3.869,6.559)}
\gppoint{gp mark 1}{(3.871,6.578)}
\gppoint{gp mark 1}{(3.873,6.577)}
\gppoint{gp mark 1}{(3.876,6.696)}
\gppoint{gp mark 1}{(3.878,6.679)}
\gppoint{gp mark 1}{(3.881,6.584)}
\gppoint{gp mark 1}{(3.883,6.553)}
\gppoint{gp mark 1}{(3.886,6.564)}
\gppoint{gp mark 1}{(3.888,6.485)}
\gppoint{gp mark 1}{(3.890,6.582)}
\gppoint{gp mark 1}{(3.893,6.668)}
\gppoint{gp mark 1}{(3.895,6.525)}
\gppoint{gp mark 1}{(3.898,6.535)}
\gppoint{gp mark 1}{(3.900,6.549)}
\gppoint{gp mark 1}{(3.902,6.587)}
\gppoint{gp mark 1}{(3.905,6.691)}
\gppoint{gp mark 1}{(3.907,6.615)}
\gppoint{gp mark 1}{(3.910,6.645)}
\gppoint{gp mark 1}{(3.912,6.598)}
\gppoint{gp mark 1}{(3.914,6.585)}
\gppoint{gp mark 1}{(3.917,6.593)}
\gppoint{gp mark 1}{(3.919,6.619)}
\gppoint{gp mark 1}{(3.922,6.582)}
\gppoint{gp mark 1}{(3.924,6.539)}
\gppoint{gp mark 1}{(3.926,6.575)}
\gppoint{gp mark 1}{(3.929,6.511)}
\gppoint{gp mark 1}{(3.931,6.601)}
\gppoint{gp mark 1}{(3.934,6.602)}
\gppoint{gp mark 1}{(3.936,6.551)}
\gppoint{gp mark 1}{(3.939,6.619)}
\gppoint{gp mark 1}{(3.941,6.486)}
\gppoint{gp mark 1}{(3.943,6.612)}
\gppoint{gp mark 1}{(3.946,6.500)}
\gppoint{gp mark 1}{(3.948,6.565)}
\gppoint{gp mark 1}{(3.951,6.717)}
\gppoint{gp mark 1}{(3.953,6.610)}
\gppoint{gp mark 1}{(3.955,6.612)}
\gppoint{gp mark 1}{(3.958,6.467)}
\gppoint{gp mark 1}{(3.960,6.541)}
\gppoint{gp mark 1}{(3.963,6.581)}
\gppoint{gp mark 1}{(3.965,6.533)}
\gppoint{gp mark 1}{(3.967,6.563)}
\gppoint{gp mark 1}{(3.970,6.707)}
\gppoint{gp mark 1}{(3.972,6.390)}
\gppoint{gp mark 1}{(3.975,6.614)}
\gppoint{gp mark 1}{(3.977,6.504)}
\gppoint{gp mark 1}{(3.980,6.603)}
\gppoint{gp mark 1}{(3.982,6.554)}
\gppoint{gp mark 1}{(3.984,6.635)}
\gppoint{gp mark 1}{(3.987,6.578)}
\gppoint{gp mark 1}{(3.989,6.642)}
\gppoint{gp mark 1}{(3.992,6.520)}
\gppoint{gp mark 1}{(3.994,6.505)}
\gppoint{gp mark 1}{(3.996,6.625)}
\gppoint{gp mark 1}{(3.999,6.642)}
\gppoint{gp mark 1}{(4.001,6.582)}
\gppoint{gp mark 1}{(4.004,6.677)}
\gppoint{gp mark 1}{(4.006,6.529)}
\gppoint{gp mark 1}{(4.008,6.549)}
\gppoint{gp mark 1}{(4.011,6.495)}
\gppoint{gp mark 1}{(4.013,6.479)}
\gppoint{gp mark 1}{(4.016,6.525)}
\gppoint{gp mark 1}{(4.018,6.477)}
\gppoint{gp mark 1}{(4.021,6.504)}
\gppoint{gp mark 1}{(4.023,6.569)}
\gppoint{gp mark 1}{(4.025,6.519)}
\gppoint{gp mark 1}{(4.028,6.674)}
\gppoint{gp mark 1}{(4.030,6.621)}
\gppoint{gp mark 1}{(4.033,6.465)}
\gppoint{gp mark 1}{(4.035,6.575)}
\gppoint{gp mark 1}{(4.037,6.599)}
\gppoint{gp mark 1}{(4.040,6.506)}
\gppoint{gp mark 1}{(4.042,6.577)}
\gppoint{gp mark 1}{(4.045,6.612)}
\gppoint{gp mark 1}{(4.047,6.581)}
\gppoint{gp mark 1}{(4.049,6.453)}
\gppoint{gp mark 1}{(4.052,6.491)}
\gppoint{gp mark 1}{(4.054,6.389)}
\gppoint{gp mark 1}{(4.057,6.452)}
\gppoint{gp mark 1}{(4.059,6.503)}
\gppoint{gp mark 1}{(4.062,6.547)}
\gppoint{gp mark 1}{(4.064,6.585)}
\gppoint{gp mark 1}{(4.066,6.475)}
\gppoint{gp mark 1}{(4.069,6.549)}
\gppoint{gp mark 1}{(4.071,6.473)}
\gppoint{gp mark 1}{(4.074,6.549)}
\gppoint{gp mark 1}{(4.076,6.479)}
\gppoint{gp mark 1}{(4.078,6.540)}
\gppoint{gp mark 1}{(4.081,6.437)}
\gppoint{gp mark 1}{(4.083,6.588)}
\gppoint{gp mark 1}{(4.086,6.480)}
\gppoint{gp mark 1}{(4.088,6.405)}
\gppoint{gp mark 1}{(4.090,6.510)}
\gppoint{gp mark 1}{(4.093,6.446)}
\gppoint{gp mark 1}{(4.095,6.431)}
\gppoint{gp mark 1}{(4.098,6.542)}
\gppoint{gp mark 1}{(4.100,6.401)}
\gppoint{gp mark 1}{(4.103,6.502)}
\gppoint{gp mark 1}{(4.105,6.492)}
\gppoint{gp mark 1}{(4.107,6.517)}
\gppoint{gp mark 1}{(4.110,6.362)}
\gppoint{gp mark 1}{(4.112,6.505)}
\gppoint{gp mark 1}{(4.115,6.553)}
\gppoint{gp mark 1}{(4.117,6.458)}
\gppoint{gp mark 1}{(4.119,6.371)}
\gppoint{gp mark 1}{(4.122,6.581)}
\gppoint{gp mark 1}{(4.124,6.515)}
\gppoint{gp mark 1}{(4.127,6.464)}
\gppoint{gp mark 1}{(4.129,6.384)}
\gppoint{gp mark 1}{(4.131,6.471)}
\gppoint{gp mark 1}{(4.134,6.450)}
\gppoint{gp mark 1}{(4.136,6.537)}
\gppoint{gp mark 1}{(4.139,6.516)}
\gppoint{gp mark 1}{(4.141,6.451)}
\gppoint{gp mark 1}{(4.144,6.652)}
\gppoint{gp mark 1}{(4.146,6.500)}
\gppoint{gp mark 1}{(4.148,6.392)}
\gppoint{gp mark 1}{(4.151,6.526)}
\gppoint{gp mark 1}{(4.153,6.362)}
\gppoint{gp mark 1}{(4.156,6.394)}
\gppoint{gp mark 1}{(4.158,6.450)}
\gppoint{gp mark 1}{(4.160,6.410)}
\gppoint{gp mark 1}{(4.163,6.490)}
\gppoint{gp mark 1}{(4.165,6.436)}
\gppoint{gp mark 1}{(4.168,6.505)}
\gppoint{gp mark 1}{(4.170,6.505)}
\gppoint{gp mark 1}{(4.172,6.400)}
\gppoint{gp mark 1}{(4.175,6.479)}
\gppoint{gp mark 1}{(4.177,6.466)}
\gppoint{gp mark 1}{(4.180,6.361)}
\gppoint{gp mark 1}{(4.182,6.509)}
\gppoint{gp mark 1}{(4.184,6.469)}
\gppoint{gp mark 1}{(4.187,6.401)}
\gppoint{gp mark 1}{(4.189,6.457)}
\gppoint{gp mark 1}{(4.192,6.462)}
\gppoint{gp mark 1}{(4.194,6.428)}
\gppoint{gp mark 1}{(4.197,6.571)}
\gppoint{gp mark 1}{(4.199,6.460)}
\gppoint{gp mark 1}{(4.201,6.351)}
\gppoint{gp mark 1}{(4.204,6.450)}
\gppoint{gp mark 1}{(4.206,6.409)}
\gppoint{gp mark 1}{(4.209,6.475)}
\gppoint{gp mark 1}{(4.211,6.475)}
\gppoint{gp mark 1}{(4.213,6.524)}
\gppoint{gp mark 1}{(4.216,6.288)}
\gppoint{gp mark 1}{(4.218,6.334)}
\gppoint{gp mark 1}{(4.221,6.358)}
\gppoint{gp mark 1}{(4.223,6.315)}
\gppoint{gp mark 1}{(4.225,6.353)}
\gppoint{gp mark 1}{(4.228,6.427)}
\gppoint{gp mark 1}{(4.230,6.398)}
\gppoint{gp mark 1}{(4.233,6.499)}
\gppoint{gp mark 1}{(4.235,6.296)}
\gppoint{gp mark 1}{(4.238,6.340)}
\gppoint{gp mark 1}{(4.240,6.322)}
\gppoint{gp mark 1}{(4.242,6.296)}
\gppoint{gp mark 1}{(4.245,6.280)}
\gppoint{gp mark 1}{(4.247,6.419)}
\gppoint{gp mark 1}{(4.250,6.410)}
\gppoint{gp mark 1}{(4.252,6.391)}
\gppoint{gp mark 1}{(4.254,6.421)}
\gppoint{gp mark 1}{(4.257,6.408)}
\gppoint{gp mark 1}{(4.259,6.458)}
\gppoint{gp mark 1}{(4.262,6.403)}
\gppoint{gp mark 1}{(4.264,6.384)}
\gppoint{gp mark 1}{(4.266,6.292)}
\gppoint{gp mark 1}{(4.269,6.359)}
\gppoint{gp mark 1}{(4.271,6.386)}
\gppoint{gp mark 1}{(4.274,6.364)}
\gppoint{gp mark 1}{(4.276,6.354)}
\gppoint{gp mark 1}{(4.279,6.359)}
\gppoint{gp mark 1}{(4.281,6.369)}
\gppoint{gp mark 1}{(4.283,6.446)}
\gppoint{gp mark 1}{(4.286,6.312)}
\gppoint{gp mark 1}{(4.288,6.275)}
\gppoint{gp mark 1}{(4.291,6.280)}
\gppoint{gp mark 1}{(4.293,6.477)}
\gppoint{gp mark 1}{(4.295,6.259)}
\gppoint{gp mark 1}{(4.298,6.450)}
\gppoint{gp mark 1}{(4.300,6.357)}
\gppoint{gp mark 1}{(4.303,6.357)}
\gppoint{gp mark 1}{(4.305,6.262)}
\gppoint{gp mark 1}{(4.307,6.292)}
\gppoint{gp mark 1}{(4.310,6.324)}
\gppoint{gp mark 1}{(4.312,6.184)}
\gppoint{gp mark 1}{(4.315,6.364)}
\gppoint{gp mark 1}{(4.317,6.311)}
\gppoint{gp mark 1}{(4.320,6.362)}
\gppoint{gp mark 1}{(4.322,6.352)}
\gppoint{gp mark 1}{(4.324,6.358)}
\gppoint{gp mark 1}{(4.327,6.211)}
\gppoint{gp mark 1}{(4.329,6.370)}
\gppoint{gp mark 1}{(4.332,6.390)}
\gppoint{gp mark 1}{(4.334,6.339)}
\gppoint{gp mark 1}{(4.336,6.211)}
\gppoint{gp mark 1}{(4.339,6.233)}
\gppoint{gp mark 1}{(4.341,6.335)}
\gppoint{gp mark 1}{(4.344,6.334)}
\gppoint{gp mark 1}{(4.346,6.342)}
\gppoint{gp mark 1}{(4.348,6.348)}
\gppoint{gp mark 1}{(4.351,6.278)}
\gppoint{gp mark 1}{(4.353,6.296)}
\gppoint{gp mark 1}{(4.356,6.288)}
\gppoint{gp mark 1}{(4.358,6.365)}
\gppoint{gp mark 1}{(4.361,6.342)}
\gppoint{gp mark 1}{(4.363,6.357)}
\gppoint{gp mark 1}{(4.365,6.334)}
\gppoint{gp mark 1}{(4.368,6.378)}
\gppoint{gp mark 1}{(4.370,6.238)}
\gppoint{gp mark 1}{(4.373,6.288)}
\gppoint{gp mark 1}{(4.375,6.254)}
\gppoint{gp mark 1}{(4.377,6.398)}
\gppoint{gp mark 1}{(4.380,6.364)}
\gppoint{gp mark 1}{(4.382,6.188)}
\gppoint{gp mark 1}{(4.385,6.265)}
\gppoint{gp mark 1}{(4.387,6.308)}
\gppoint{gp mark 1}{(4.389,6.354)}
\gppoint{gp mark 1}{(4.392,6.238)}
\gppoint{gp mark 1}{(4.394,6.239)}
\gppoint{gp mark 1}{(4.397,6.313)}
\gppoint{gp mark 1}{(4.399,6.322)}
\gppoint{gp mark 1}{(4.402,6.391)}
\gppoint{gp mark 1}{(4.404,6.292)}
\gppoint{gp mark 1}{(4.406,6.320)}
\gppoint{gp mark 1}{(4.409,6.252)}
\gppoint{gp mark 1}{(4.411,6.299)}
\gppoint{gp mark 1}{(4.414,6.237)}
\gppoint{gp mark 1}{(4.416,6.316)}
\gppoint{gp mark 1}{(4.418,6.238)}
\gppoint{gp mark 1}{(4.421,6.223)}
\gppoint{gp mark 1}{(4.423,6.172)}
\gppoint{gp mark 1}{(4.426,6.307)}
\gppoint{gp mark 1}{(4.428,6.266)}
\gppoint{gp mark 1}{(4.430,6.344)}
\gppoint{gp mark 1}{(4.433,6.266)}
\gppoint{gp mark 1}{(4.435,6.293)}
\gppoint{gp mark 1}{(4.438,6.343)}
\gppoint{gp mark 1}{(4.440,6.248)}
\gppoint{gp mark 1}{(4.442,6.206)}
\gppoint{gp mark 1}{(4.445,6.147)}
\gppoint{gp mark 1}{(4.447,6.198)}
\gppoint{gp mark 1}{(4.450,6.134)}
\gppoint{gp mark 1}{(4.452,6.231)}
\gppoint{gp mark 1}{(4.455,6.260)}
\gppoint{gp mark 1}{(4.457,6.263)}
\gppoint{gp mark 1}{(4.459,6.188)}
\gppoint{gp mark 1}{(4.462,6.256)}
\gppoint{gp mark 1}{(4.464,6.154)}
\gppoint{gp mark 1}{(4.467,6.269)}
\gppoint{gp mark 1}{(4.469,6.195)}
\gppoint{gp mark 1}{(4.471,6.195)}
\gppoint{gp mark 1}{(4.474,6.324)}
\gppoint{gp mark 1}{(4.476,6.110)}
\gppoint{gp mark 1}{(4.479,6.203)}
\gppoint{gp mark 1}{(4.481,6.132)}
\gppoint{gp mark 1}{(4.483,6.135)}
\gppoint{gp mark 1}{(4.486,6.297)}
\gppoint{gp mark 1}{(4.488,6.135)}
\gppoint{gp mark 1}{(4.491,6.260)}
\gppoint{gp mark 1}{(4.493,6.264)}
\gppoint{gp mark 1}{(4.496,6.245)}
\gppoint{gp mark 1}{(4.498,6.236)}
\gppoint{gp mark 1}{(4.500,6.180)}
\gppoint{gp mark 1}{(4.503,6.367)}
\gppoint{gp mark 1}{(4.505,6.239)}
\gppoint{gp mark 1}{(4.508,6.292)}
\gppoint{gp mark 1}{(4.510,6.145)}
\gppoint{gp mark 1}{(4.512,6.189)}
\gppoint{gp mark 1}{(4.515,6.228)}
\gppoint{gp mark 1}{(4.517,6.169)}
\gppoint{gp mark 1}{(4.520,6.210)}
\gppoint{gp mark 1}{(4.522,6.209)}
\gppoint{gp mark 1}{(4.524,6.246)}
\gppoint{gp mark 1}{(4.527,6.217)}
\gppoint{gp mark 1}{(4.529,6.197)}
\gppoint{gp mark 1}{(4.532,6.213)}
\gppoint{gp mark 1}{(4.534,6.154)}
\gppoint{gp mark 1}{(4.537,6.149)}
\gppoint{gp mark 1}{(4.539,6.208)}
\gppoint{gp mark 1}{(4.541,6.209)}
\gppoint{gp mark 1}{(4.544,6.186)}
\gppoint{gp mark 1}{(4.546,6.161)}
\gppoint{gp mark 1}{(4.549,6.154)}
\gppoint{gp mark 1}{(4.551,6.301)}
\gppoint{gp mark 1}{(4.553,6.073)}
\gppoint{gp mark 1}{(4.556,6.200)}
\gppoint{gp mark 1}{(4.558,6.210)}
\gppoint{gp mark 1}{(4.561,6.245)}
\gppoint{gp mark 1}{(4.563,6.149)}
\gppoint{gp mark 1}{(4.565,6.297)}
\gppoint{gp mark 1}{(4.568,6.122)}
\gppoint{gp mark 1}{(4.570,6.104)}
\gppoint{gp mark 1}{(4.573,6.169)}
\gppoint{gp mark 1}{(4.575,6.268)}
\gppoint{gp mark 1}{(4.578,6.234)}
\gppoint{gp mark 1}{(4.580,6.070)}
\gppoint{gp mark 1}{(4.582,6.149)}
\gppoint{gp mark 1}{(4.585,6.182)}
\gppoint{gp mark 1}{(4.587,6.162)}
\gppoint{gp mark 1}{(4.590,6.134)}
\gppoint{gp mark 1}{(4.592,6.161)}
\gppoint{gp mark 1}{(4.594,6.062)}
\gppoint{gp mark 1}{(4.597,6.171)}
\gppoint{gp mark 1}{(4.599,6.210)}
\gppoint{gp mark 1}{(4.602,6.029)}
\gppoint{gp mark 1}{(4.604,6.058)}
\gppoint{gp mark 1}{(4.606,6.147)}
\gppoint{gp mark 1}{(4.609,6.136)}
\gppoint{gp mark 1}{(4.611,6.157)}
\gppoint{gp mark 1}{(4.614,6.179)}
\gppoint{gp mark 1}{(4.616,6.208)}
\gppoint{gp mark 1}{(4.619,6.144)}
\gppoint{gp mark 1}{(4.621,6.087)}
\gppoint{gp mark 1}{(4.623,6.163)}
\gppoint{gp mark 1}{(4.626,6.077)}
\gppoint{gp mark 1}{(4.628,6.161)}
\gppoint{gp mark 1}{(4.631,6.265)}
\gppoint{gp mark 1}{(4.633,6.123)}
\gppoint{gp mark 1}{(4.635,6.101)}
\gppoint{gp mark 1}{(4.638,6.035)}
\gppoint{gp mark 1}{(4.640,6.064)}
\gppoint{gp mark 1}{(4.643,6.110)}
\gppoint{gp mark 1}{(4.645,6.126)}
\gppoint{gp mark 1}{(4.647,6.081)}
\gppoint{gp mark 1}{(4.650,6.173)}
\gppoint{gp mark 1}{(4.652,6.053)}
\gppoint{gp mark 1}{(4.655,6.037)}
\gppoint{gp mark 1}{(4.657,6.221)}
\gppoint{gp mark 1}{(4.659,6.126)}
\gppoint{gp mark 1}{(4.662,6.192)}
\gppoint{gp mark 1}{(4.664,6.107)}
\gppoint{gp mark 1}{(4.667,6.136)}
\gppoint{gp mark 1}{(4.669,6.013)}
\gppoint{gp mark 1}{(4.672,6.110)}
\gppoint{gp mark 1}{(4.674,6.190)}
\gppoint{gp mark 1}{(4.676,6.104)}
\gppoint{gp mark 1}{(4.679,6.121)}
\gppoint{gp mark 1}{(4.681,5.974)}
\gppoint{gp mark 1}{(4.684,6.148)}
\gppoint{gp mark 1}{(4.686,6.059)}
\gppoint{gp mark 1}{(4.688,6.103)}
\gppoint{gp mark 1}{(4.691,6.092)}
\gppoint{gp mark 1}{(4.693,6.088)}
\gppoint{gp mark 1}{(4.696,6.137)}
\gppoint{gp mark 1}{(4.698,6.119)}
\gppoint{gp mark 1}{(4.700,6.007)}
\gppoint{gp mark 1}{(4.703,6.094)}
\gppoint{gp mark 1}{(4.705,6.166)}
\gppoint{gp mark 1}{(4.708,6.151)}
\gppoint{gp mark 1}{(4.710,6.136)}
\gppoint{gp mark 1}{(4.713,6.156)}
\gppoint{gp mark 1}{(4.715,6.129)}
\gppoint{gp mark 1}{(4.717,5.999)}
\gppoint{gp mark 1}{(4.720,6.142)}
\gppoint{gp mark 1}{(4.722,6.131)}
\gppoint{gp mark 1}{(4.725,6.097)}
\gppoint{gp mark 1}{(4.727,6.042)}
\gppoint{gp mark 1}{(4.729,6.220)}
\gppoint{gp mark 1}{(4.732,6.148)}
\gppoint{gp mark 1}{(4.734,5.971)}
\gppoint{gp mark 1}{(4.737,5.964)}
\gppoint{gp mark 1}{(4.739,6.033)}
\gppoint{gp mark 1}{(4.741,6.053)}
\gppoint{gp mark 1}{(4.744,6.090)}
\gppoint{gp mark 1}{(4.746,6.107)}
\gppoint{gp mark 1}{(4.749,6.069)}
\gppoint{gp mark 1}{(4.751,6.113)}
\gppoint{gp mark 1}{(4.754,6.112)}
\gppoint{gp mark 1}{(4.756,6.126)}
\gppoint{gp mark 1}{(4.758,6.079)}
\gppoint{gp mark 1}{(4.761,6.163)}
\gppoint{gp mark 1}{(4.763,6.042)}
\gppoint{gp mark 1}{(4.766,6.084)}
\gppoint{gp mark 1}{(4.768,6.046)}
\gppoint{gp mark 1}{(4.770,6.104)}
\gppoint{gp mark 1}{(4.773,6.110)}
\gppoint{gp mark 1}{(4.775,6.078)}
\gppoint{gp mark 1}{(4.778,6.004)}
\gppoint{gp mark 1}{(4.780,6.013)}
\gppoint{gp mark 1}{(4.782,5.862)}
\gppoint{gp mark 1}{(4.785,6.014)}
\gppoint{gp mark 1}{(4.787,5.933)}
\gppoint{gp mark 1}{(4.790,5.950)}
\gppoint{gp mark 1}{(4.792,5.959)}
\gppoint{gp mark 1}{(4.795,5.969)}
\gppoint{gp mark 1}{(4.797,6.048)}
\gppoint{gp mark 1}{(4.799,5.942)}
\gppoint{gp mark 1}{(4.802,6.174)}
\gppoint{gp mark 1}{(4.804,5.982)}
\gppoint{gp mark 1}{(4.807,5.940)}
\gppoint{gp mark 1}{(4.809,6.072)}
\gppoint{gp mark 1}{(4.811,6.063)}
\gppoint{gp mark 1}{(4.814,6.033)}
\gppoint{gp mark 1}{(4.816,6.109)}
\gppoint{gp mark 1}{(4.819,5.921)}
\gppoint{gp mark 1}{(4.821,6.015)}
\gppoint{gp mark 1}{(4.823,6.005)}
\gppoint{gp mark 1}{(4.826,6.107)}
\gppoint{gp mark 1}{(4.828,5.974)}
\gppoint{gp mark 1}{(4.831,6.098)}
\gppoint{gp mark 1}{(4.833,6.019)}
\gppoint{gp mark 1}{(4.836,6.010)}
\gppoint{gp mark 1}{(4.838,5.916)}
\gppoint{gp mark 1}{(4.840,6.081)}
\gppoint{gp mark 1}{(4.843,5.900)}
\gppoint{gp mark 1}{(4.845,6.079)}
\gppoint{gp mark 1}{(4.848,5.973)}
\gppoint{gp mark 1}{(4.850,5.964)}
\gppoint{gp mark 1}{(4.852,5.982)}
\gppoint{gp mark 1}{(4.855,6.003)}
\gppoint{gp mark 1}{(4.857,6.041)}
\gppoint{gp mark 1}{(4.860,5.855)}
\gppoint{gp mark 1}{(4.862,5.985)}
\gppoint{gp mark 1}{(4.864,6.039)}
\gppoint{gp mark 1}{(4.867,6.021)}
\gppoint{gp mark 1}{(4.869,6.023)}
\gppoint{gp mark 1}{(4.872,6.062)}
\gppoint{gp mark 1}{(4.874,5.891)}
\gppoint{gp mark 1}{(4.877,5.983)}
\gppoint{gp mark 1}{(4.879,5.965)}
\gppoint{gp mark 1}{(4.881,5.940)}
\gppoint{gp mark 1}{(4.884,5.996)}
\gppoint{gp mark 1}{(4.886,5.878)}
\gppoint{gp mark 1}{(4.889,5.856)}
\gppoint{gp mark 1}{(4.891,5.955)}
\gppoint{gp mark 1}{(4.893,6.021)}
\gppoint{gp mark 1}{(4.896,5.952)}
\gppoint{gp mark 1}{(4.898,5.985)}
\gppoint{gp mark 1}{(4.901,5.886)}
\gppoint{gp mark 1}{(4.903,5.978)}
\gppoint{gp mark 1}{(4.905,6.069)}
\gppoint{gp mark 1}{(4.908,5.842)}
\gppoint{gp mark 1}{(4.910,5.944)}
\gppoint{gp mark 1}{(4.913,5.966)}
\gppoint{gp mark 1}{(4.915,5.913)}
\gppoint{gp mark 1}{(4.917,5.891)}
\gppoint{gp mark 1}{(4.920,5.871)}
\gppoint{gp mark 1}{(4.922,5.990)}
\gppoint{gp mark 1}{(4.925,5.946)}
\gppoint{gp mark 1}{(4.927,5.827)}
\gppoint{gp mark 1}{(4.930,5.939)}
\gppoint{gp mark 1}{(4.932,5.919)}
\gppoint{gp mark 1}{(4.934,5.881)}
\gppoint{gp mark 1}{(4.937,5.925)}
\gppoint{gp mark 1}{(4.939,5.896)}
\gppoint{gp mark 1}{(4.942,5.864)}
\gppoint{gp mark 1}{(4.944,5.924)}
\gppoint{gp mark 1}{(4.946,5.951)}
\gppoint{gp mark 1}{(4.949,5.872)}
\gppoint{gp mark 1}{(4.951,6.004)}
\gppoint{gp mark 1}{(4.954,5.867)}
\gppoint{gp mark 1}{(4.956,5.927)}
\gppoint{gp mark 1}{(4.958,6.042)}
\gppoint{gp mark 1}{(4.961,5.927)}
\gppoint{gp mark 1}{(4.963,5.882)}
\gppoint{gp mark 1}{(4.966,5.857)}
\gppoint{gp mark 1}{(4.968,5.840)}
\gppoint{gp mark 1}{(4.971,5.868)}
\gppoint{gp mark 1}{(4.973,5.839)}
\gppoint{gp mark 1}{(4.975,5.984)}
\gppoint{gp mark 1}{(4.978,5.973)}
\gppoint{gp mark 1}{(4.980,5.854)}
\gppoint{gp mark 1}{(4.983,5.939)}
\gppoint{gp mark 1}{(4.985,5.840)}
\gppoint{gp mark 1}{(4.987,5.887)}
\gppoint{gp mark 1}{(4.990,5.956)}
\gppoint{gp mark 1}{(4.992,5.919)}
\gppoint{gp mark 1}{(4.995,5.701)}
\gppoint{gp mark 1}{(4.997,5.816)}
\gppoint{gp mark 1}{(4.999,5.836)}
\gppoint{gp mark 1}{(5.002,5.878)}
\gppoint{gp mark 1}{(5.004,5.836)}
\gppoint{gp mark 1}{(5.007,5.835)}
\gppoint{gp mark 1}{(5.009,5.802)}
\gppoint{gp mark 1}{(5.012,5.935)}
\gppoint{gp mark 1}{(5.014,5.987)}
\gppoint{gp mark 1}{(5.016,5.893)}
\gppoint{gp mark 1}{(5.019,5.814)}
\gppoint{gp mark 1}{(5.021,5.997)}
\gppoint{gp mark 1}{(5.024,5.754)}
\gppoint{gp mark 1}{(5.026,5.856)}
\gppoint{gp mark 1}{(5.028,5.869)}
\gppoint{gp mark 1}{(5.031,5.915)}
\gppoint{gp mark 1}{(5.033,5.858)}
\gppoint{gp mark 1}{(5.036,5.820)}
\gppoint{gp mark 1}{(5.038,5.821)}
\gppoint{gp mark 1}{(5.040,5.858)}
\gppoint{gp mark 1}{(5.043,5.808)}
\gppoint{gp mark 1}{(5.045,5.854)}
\gppoint{gp mark 1}{(5.048,5.895)}
\gppoint{gp mark 1}{(5.050,5.828)}
\gppoint{gp mark 1}{(5.053,5.892)}
\gppoint{gp mark 1}{(5.055,5.859)}
\gppoint{gp mark 1}{(5.057,5.893)}
\gppoint{gp mark 1}{(5.060,5.850)}
\gppoint{gp mark 1}{(5.062,5.754)}
\gppoint{gp mark 1}{(5.065,5.795)}
\gppoint{gp mark 1}{(5.067,5.962)}
\gppoint{gp mark 1}{(5.069,5.879)}
\gppoint{gp mark 1}{(5.072,5.831)}
\gppoint{gp mark 1}{(5.074,5.852)}
\gppoint{gp mark 1}{(5.077,5.817)}
\gppoint{gp mark 1}{(5.079,5.893)}
\gppoint{gp mark 1}{(5.081,5.774)}
\gppoint{gp mark 1}{(5.084,5.826)}
\gppoint{gp mark 1}{(5.086,5.942)}
\gppoint{gp mark 1}{(5.089,5.880)}
\gppoint{gp mark 1}{(5.091,5.805)}
\gppoint{gp mark 1}{(5.094,5.879)}
\gppoint{gp mark 1}{(5.096,5.826)}
\gppoint{gp mark 1}{(5.098,5.844)}
\gppoint{gp mark 1}{(5.101,5.837)}
\gppoint{gp mark 1}{(5.103,5.825)}
\gppoint{gp mark 1}{(5.106,5.753)}
\gppoint{gp mark 1}{(5.108,5.788)}
\gppoint{gp mark 1}{(5.110,5.810)}
\gppoint{gp mark 1}{(5.113,5.805)}
\gppoint{gp mark 1}{(5.115,5.725)}
\gppoint{gp mark 1}{(5.118,5.866)}
\gppoint{gp mark 1}{(5.120,5.931)}
\gppoint{gp mark 1}{(5.122,5.744)}
\gppoint{gp mark 1}{(5.125,5.908)}
\gppoint{gp mark 1}{(5.127,5.867)}
\gppoint{gp mark 1}{(5.130,5.785)}
\gppoint{gp mark 1}{(5.132,5.805)}
\gppoint{gp mark 1}{(5.135,5.800)}
\gppoint{gp mark 1}{(5.137,5.759)}
\gppoint{gp mark 1}{(5.139,5.873)}
\gppoint{gp mark 1}{(5.142,5.697)}
\gppoint{gp mark 1}{(5.144,5.722)}
\gppoint{gp mark 1}{(5.147,5.915)}
\gppoint{gp mark 1}{(5.149,5.782)}
\gppoint{gp mark 1}{(5.151,5.678)}
\gppoint{gp mark 1}{(5.154,5.822)}
\gppoint{gp mark 1}{(5.156,5.785)}
\gppoint{gp mark 1}{(5.159,5.795)}
\gppoint{gp mark 1}{(5.161,5.817)}
\gppoint{gp mark 1}{(5.163,5.871)}
\gppoint{gp mark 1}{(5.166,5.797)}
\gppoint{gp mark 1}{(5.168,5.771)}
\gppoint{gp mark 1}{(5.171,5.848)}
\gppoint{gp mark 1}{(5.173,5.884)}
\gppoint{gp mark 1}{(5.175,5.811)}
\gppoint{gp mark 1}{(5.178,5.753)}
\gppoint{gp mark 1}{(5.180,5.902)}
\gppoint{gp mark 1}{(5.183,5.675)}
\gppoint{gp mark 1}{(5.185,5.765)}
\gppoint{gp mark 1}{(5.188,5.785)}
\gppoint{gp mark 1}{(5.190,5.910)}
\gppoint{gp mark 1}{(5.192,5.658)}
\gppoint{gp mark 1}{(5.195,5.795)}
\gppoint{gp mark 1}{(5.197,5.849)}
\gppoint{gp mark 1}{(5.200,5.805)}
\gppoint{gp mark 1}{(5.202,5.840)}
\gppoint{gp mark 1}{(5.204,5.821)}
\gppoint{gp mark 1}{(5.207,5.810)}
\gppoint{gp mark 1}{(5.209,5.788)}
\gppoint{gp mark 1}{(5.212,5.736)}
\gppoint{gp mark 1}{(5.214,5.790)}
\gppoint{gp mark 1}{(5.216,5.765)}
\gppoint{gp mark 1}{(5.219,5.765)}
\gppoint{gp mark 1}{(5.221,5.722)}
\gppoint{gp mark 1}{(5.224,5.764)}
\gppoint{gp mark 1}{(5.226,5.784)}
\gppoint{gp mark 1}{(5.229,5.830)}
\gppoint{gp mark 1}{(5.231,5.657)}
\gppoint{gp mark 1}{(5.233,5.723)}
\gppoint{gp mark 1}{(5.236,5.739)}
\gppoint{gp mark 1}{(5.238,5.733)}
\gppoint{gp mark 1}{(5.241,5.724)}
\gppoint{gp mark 1}{(5.243,5.733)}
\gppoint{gp mark 1}{(5.245,5.695)}
\gppoint{gp mark 1}{(5.248,5.734)}
\gppoint{gp mark 1}{(5.250,5.806)}
\gppoint{gp mark 1}{(5.253,5.739)}
\gppoint{gp mark 1}{(5.255,5.729)}
\gppoint{gp mark 1}{(5.257,5.661)}
\gppoint{gp mark 1}{(5.260,5.611)}
\gppoint{gp mark 1}{(5.262,5.721)}
\gppoint{gp mark 1}{(5.265,5.656)}
\gppoint{gp mark 1}{(5.267,5.716)}
\gppoint{gp mark 1}{(5.270,5.676)}
\gppoint{gp mark 1}{(5.272,5.701)}
\gppoint{gp mark 1}{(5.274,5.616)}
\gppoint{gp mark 1}{(5.277,5.657)}
\gppoint{gp mark 1}{(5.279,5.732)}
\gppoint{gp mark 1}{(5.282,5.638)}
\gppoint{gp mark 1}{(5.284,5.752)}
\gppoint{gp mark 1}{(5.286,5.848)}
\gppoint{gp mark 1}{(5.289,5.776)}
\gppoint{gp mark 1}{(5.291,5.749)}
\gppoint{gp mark 1}{(5.294,5.568)}
\gppoint{gp mark 1}{(5.296,5.752)}
\gppoint{gp mark 1}{(5.298,5.612)}
\gppoint{gp mark 1}{(5.301,5.775)}
\gppoint{gp mark 1}{(5.303,5.742)}
\gppoint{gp mark 1}{(5.306,5.780)}
\gppoint{gp mark 1}{(5.308,5.646)}
\gppoint{gp mark 1}{(5.311,5.805)}
\gppoint{gp mark 1}{(5.313,5.737)}
\gppoint{gp mark 1}{(5.315,5.713)}
\gppoint{gp mark 1}{(5.318,5.662)}
\gppoint{gp mark 1}{(5.320,5.659)}
\gppoint{gp mark 1}{(5.323,5.754)}
\gppoint{gp mark 1}{(5.325,5.782)}
\gppoint{gp mark 1}{(5.327,5.656)}
\gppoint{gp mark 1}{(5.330,5.723)}
\gppoint{gp mark 1}{(5.332,5.694)}
\gppoint{gp mark 1}{(5.335,5.694)}
\gppoint{gp mark 1}{(5.337,5.689)}
\gppoint{gp mark 1}{(5.339,5.677)}
\gppoint{gp mark 1}{(5.342,5.707)}
\gppoint{gp mark 1}{(5.344,5.734)}
\gppoint{gp mark 1}{(5.347,5.666)}
\gppoint{gp mark 1}{(5.349,5.709)}
\gppoint{gp mark 1}{(5.352,5.596)}
\gppoint{gp mark 1}{(5.354,5.697)}
\gppoint{gp mark 1}{(5.356,5.652)}
\gppoint{gp mark 1}{(5.359,5.768)}
\gppoint{gp mark 1}{(5.361,5.789)}
\gppoint{gp mark 1}{(5.364,5.729)}
\gppoint{gp mark 1}{(5.366,5.647)}
\gppoint{gp mark 1}{(5.368,5.678)}
\gppoint{gp mark 1}{(5.371,5.754)}
\gppoint{gp mark 1}{(5.373,5.686)}
\gppoint{gp mark 1}{(5.376,5.684)}
\gppoint{gp mark 1}{(5.378,5.651)}
\gppoint{gp mark 1}{(5.380,5.615)}
\gppoint{gp mark 1}{(5.383,5.575)}
\gppoint{gp mark 1}{(5.385,5.695)}
\gppoint{gp mark 1}{(5.388,5.738)}
\gppoint{gp mark 1}{(5.390,5.588)}
\gppoint{gp mark 1}{(5.393,5.717)}
\gppoint{gp mark 1}{(5.395,5.562)}
\gppoint{gp mark 1}{(5.397,5.634)}
\gppoint{gp mark 1}{(5.400,5.609)}
\gppoint{gp mark 1}{(5.402,5.463)}
\gppoint{gp mark 1}{(5.405,5.568)}
\gppoint{gp mark 1}{(5.407,5.601)}
\gppoint{gp mark 1}{(5.409,5.429)}
\gppoint{gp mark 1}{(5.412,5.527)}
\gppoint{gp mark 1}{(5.414,5.607)}
\gppoint{gp mark 1}{(5.417,5.697)}
\gppoint{gp mark 1}{(5.419,5.646)}
\gppoint{gp mark 1}{(5.421,5.752)}
\gppoint{gp mark 1}{(5.424,5.679)}
\gppoint{gp mark 1}{(5.426,5.637)}
\gppoint{gp mark 1}{(5.429,5.740)}
\gppoint{gp mark 1}{(5.431,5.664)}
\gppoint{gp mark 1}{(5.433,5.762)}
\gppoint{gp mark 1}{(5.436,5.621)}
\gppoint{gp mark 1}{(5.438,5.591)}
\gppoint{gp mark 1}{(5.441,5.634)}
\gppoint{gp mark 1}{(5.443,5.703)}
\gppoint{gp mark 1}{(5.446,5.623)}
\gppoint{gp mark 1}{(5.448,5.586)}
\gppoint{gp mark 1}{(5.450,5.606)}
\gppoint{gp mark 1}{(5.453,5.588)}
\gppoint{gp mark 1}{(5.455,5.676)}
\gppoint{gp mark 1}{(5.458,5.606)}
\gppoint{gp mark 1}{(5.460,5.599)}
\gppoint{gp mark 1}{(5.462,5.658)}
\gppoint{gp mark 1}{(5.465,5.586)}
\gppoint{gp mark 1}{(5.467,5.544)}
\gppoint{gp mark 1}{(5.470,5.502)}
\gppoint{gp mark 1}{(5.472,5.547)}
\gppoint{gp mark 1}{(5.474,5.593)}
\gppoint{gp mark 1}{(5.477,5.578)}
\gppoint{gp mark 1}{(5.479,5.466)}
\gppoint{gp mark 1}{(5.482,5.672)}
\gppoint{gp mark 1}{(5.484,5.440)}
\gppoint{gp mark 1}{(5.487,5.636)}
\gppoint{gp mark 1}{(5.489,5.640)}
\gppoint{gp mark 1}{(5.491,5.611)}
\gppoint{gp mark 1}{(5.494,5.570)}
\gppoint{gp mark 1}{(5.496,5.645)}
\gppoint{gp mark 1}{(5.499,5.581)}
\gppoint{gp mark 1}{(5.501,5.646)}
\gppoint{gp mark 1}{(5.503,5.654)}
\gppoint{gp mark 1}{(5.506,5.549)}
\gppoint{gp mark 1}{(5.508,5.507)}
\gppoint{gp mark 1}{(5.511,5.484)}
\gppoint{gp mark 1}{(5.513,5.508)}
\gppoint{gp mark 1}{(5.515,5.609)}
\gppoint{gp mark 1}{(5.518,5.512)}
\gppoint{gp mark 1}{(5.520,5.572)}
\gppoint{gp mark 1}{(5.523,5.517)}
\gppoint{gp mark 1}{(5.525,5.603)}
\gppoint{gp mark 1}{(5.528,5.643)}
\gppoint{gp mark 1}{(5.530,5.551)}
\gppoint{gp mark 1}{(5.532,5.489)}
\gppoint{gp mark 1}{(5.535,5.492)}
\gppoint{gp mark 1}{(5.537,5.624)}
\gppoint{gp mark 1}{(5.540,5.513)}
\gppoint{gp mark 1}{(5.542,5.521)}
\gppoint{gp mark 1}{(5.544,5.598)}
\gppoint{gp mark 1}{(5.547,5.673)}
\gppoint{gp mark 1}{(5.549,5.494)}
\gppoint{gp mark 1}{(5.552,5.511)}
\gppoint{gp mark 1}{(5.554,5.435)}
\gppoint{gp mark 1}{(5.556,5.565)}
\gppoint{gp mark 1}{(5.559,5.630)}
\gppoint{gp mark 1}{(5.561,5.616)}
\gppoint{gp mark 1}{(5.564,5.570)}
\gppoint{gp mark 1}{(5.566,5.570)}
\gppoint{gp mark 1}{(5.569,5.544)}
\gppoint{gp mark 1}{(5.571,5.434)}
\gppoint{gp mark 1}{(5.573,5.574)}
\gppoint{gp mark 1}{(5.576,5.528)}
\gppoint{gp mark 1}{(5.578,5.523)}
\gppoint{gp mark 1}{(5.581,5.493)}
\gppoint{gp mark 1}{(5.583,5.506)}
\gppoint{gp mark 1}{(5.585,5.600)}
\gppoint{gp mark 1}{(5.588,5.607)}
\gppoint{gp mark 1}{(5.590,5.561)}
\gppoint{gp mark 1}{(5.593,5.518)}
\gppoint{gp mark 1}{(5.595,5.559)}
\gppoint{gp mark 1}{(5.597,5.570)}
\gppoint{gp mark 1}{(5.600,5.559)}
\gppoint{gp mark 1}{(5.602,5.456)}
\gppoint{gp mark 1}{(5.605,5.536)}
\gppoint{gp mark 1}{(5.607,5.403)}
\gppoint{gp mark 1}{(5.610,5.689)}
\gppoint{gp mark 1}{(5.612,5.477)}
\gppoint{gp mark 1}{(5.614,5.559)}
\gppoint{gp mark 1}{(5.617,5.461)}
\gppoint{gp mark 1}{(5.619,5.438)}
\gppoint{gp mark 1}{(5.622,5.614)}
\gppoint{gp mark 1}{(5.624,5.545)}
\gppoint{gp mark 1}{(5.626,5.514)}
\gppoint{gp mark 1}{(5.629,5.566)}
\gppoint{gp mark 1}{(5.631,5.537)}
\gppoint{gp mark 1}{(5.634,5.397)}
\gppoint{gp mark 1}{(5.636,5.386)}
\gppoint{gp mark 1}{(5.638,5.485)}
\gppoint{gp mark 1}{(5.641,5.496)}
\gppoint{gp mark 1}{(5.643,5.429)}
\gppoint{gp mark 1}{(5.646,5.396)}
\gppoint{gp mark 1}{(5.648,5.423)}
\gppoint{gp mark 1}{(5.650,5.377)}
\gppoint{gp mark 1}{(5.653,5.480)}
\gppoint{gp mark 1}{(5.655,5.430)}
\gppoint{gp mark 1}{(5.658,5.449)}
\gppoint{gp mark 1}{(5.660,5.556)}
\gppoint{gp mark 1}{(5.663,5.506)}
\gppoint{gp mark 1}{(5.665,5.370)}
\gppoint{gp mark 1}{(5.667,5.487)}
\gppoint{gp mark 1}{(5.670,5.488)}
\gppoint{gp mark 1}{(5.672,5.492)}
\gppoint{gp mark 1}{(5.675,5.457)}
\gppoint{gp mark 1}{(5.677,5.435)}
\gppoint{gp mark 1}{(5.679,5.590)}
\gppoint{gp mark 1}{(5.682,5.520)}
\gppoint{gp mark 1}{(5.684,5.577)}
\gppoint{gp mark 1}{(5.687,5.374)}
\gppoint{gp mark 1}{(5.689,5.411)}
\gppoint{gp mark 1}{(5.691,5.471)}
\gppoint{gp mark 1}{(5.694,5.518)}
\gppoint{gp mark 1}{(5.696,5.445)}
\gppoint{gp mark 1}{(5.699,5.407)}
\gppoint{gp mark 1}{(5.701,5.531)}
\gppoint{gp mark 1}{(5.704,5.471)}
\gppoint{gp mark 1}{(5.706,5.564)}
\gppoint{gp mark 1}{(5.708,5.554)}
\gppoint{gp mark 1}{(5.711,5.431)}
\gppoint{gp mark 1}{(5.713,5.416)}
\gppoint{gp mark 1}{(5.716,5.458)}
\gppoint{gp mark 1}{(5.718,5.439)}
\gppoint{gp mark 1}{(5.720,5.464)}
\gppoint{gp mark 1}{(5.723,5.521)}
\gppoint{gp mark 1}{(5.725,5.370)}
\gppoint{gp mark 1}{(5.728,5.466)}
\gppoint{gp mark 1}{(5.730,5.494)}
\gppoint{gp mark 1}{(5.732,5.544)}
\gppoint{gp mark 1}{(5.735,5.474)}
\gppoint{gp mark 1}{(5.737,5.446)}
\gppoint{gp mark 1}{(5.740,5.516)}
\gppoint{gp mark 1}{(5.742,5.552)}
\gppoint{gp mark 1}{(5.745,5.375)}
\gppoint{gp mark 1}{(5.747,5.479)}
\gppoint{gp mark 1}{(5.749,5.355)}
\gppoint{gp mark 1}{(5.752,5.480)}
\gppoint{gp mark 1}{(5.754,5.500)}
\gppoint{gp mark 1}{(5.757,5.466)}
\gppoint{gp mark 1}{(5.759,5.386)}
\gppoint{gp mark 1}{(5.761,5.488)}
\gppoint{gp mark 1}{(5.764,5.380)}
\gppoint{gp mark 1}{(5.766,5.376)}
\gppoint{gp mark 1}{(5.769,5.431)}
\gppoint{gp mark 1}{(5.771,5.395)}
\gppoint{gp mark 1}{(5.773,5.451)}
\gppoint{gp mark 1}{(5.776,5.428)}
\gppoint{gp mark 1}{(5.778,5.366)}
\gppoint{gp mark 1}{(5.781,5.460)}
\gppoint{gp mark 1}{(5.783,5.458)}
\gppoint{gp mark 1}{(5.786,5.350)}
\gppoint{gp mark 1}{(5.788,5.533)}
\gppoint{gp mark 1}{(5.790,5.428)}
\gppoint{gp mark 1}{(5.793,5.439)}
\gppoint{gp mark 1}{(5.795,5.530)}
\gppoint{gp mark 1}{(5.798,5.388)}
\gppoint{gp mark 1}{(5.800,5.472)}
\gppoint{gp mark 1}{(5.802,5.391)}
\gppoint{gp mark 1}{(5.805,5.491)}
\gppoint{gp mark 1}{(5.807,5.443)}
\gppoint{gp mark 1}{(5.810,5.449)}
\gppoint{gp mark 1}{(5.812,5.398)}
\gppoint{gp mark 1}{(5.814,5.351)}
\gppoint{gp mark 1}{(5.817,5.444)}
\gppoint{gp mark 1}{(5.819,5.332)}
\gppoint{gp mark 1}{(5.822,5.370)}
\gppoint{gp mark 1}{(5.824,5.303)}
\gppoint{gp mark 1}{(5.827,5.364)}
\gppoint{gp mark 1}{(5.829,5.615)}
\gppoint{gp mark 1}{(5.831,5.505)}
\gppoint{gp mark 1}{(5.834,5.392)}
\gppoint{gp mark 1}{(5.836,5.329)}
\gppoint{gp mark 1}{(5.839,5.440)}
\gppoint{gp mark 1}{(5.841,5.340)}
\gppoint{gp mark 1}{(5.843,5.459)}
\gppoint{gp mark 1}{(5.846,5.399)}
\gppoint{gp mark 1}{(5.848,5.343)}
\gppoint{gp mark 1}{(5.851,5.254)}
\gppoint{gp mark 1}{(5.853,5.427)}
\gppoint{gp mark 1}{(5.855,5.395)}
\gppoint{gp mark 1}{(5.858,5.370)}
\gppoint{gp mark 1}{(5.860,5.443)}
\gppoint{gp mark 1}{(5.863,5.380)}
\gppoint{gp mark 1}{(5.865,5.370)}
\gppoint{gp mark 1}{(5.868,5.394)}
\gppoint{gp mark 1}{(5.870,5.289)}
\gppoint{gp mark 1}{(5.872,5.286)}
\gppoint{gp mark 1}{(5.875,5.462)}
\gppoint{gp mark 1}{(5.877,5.291)}
\gppoint{gp mark 1}{(5.880,5.415)}
\gppoint{gp mark 1}{(5.882,5.416)}
\gppoint{gp mark 1}{(5.884,5.309)}
\gppoint{gp mark 1}{(5.887,5.367)}
\gppoint{gp mark 1}{(5.889,5.296)}
\gppoint{gp mark 1}{(5.892,5.317)}
\gppoint{gp mark 1}{(5.894,5.455)}
\gppoint{gp mark 1}{(5.896,5.280)}
\gppoint{gp mark 1}{(5.899,5.337)}
\gppoint{gp mark 1}{(5.901,5.378)}
\gppoint{gp mark 1}{(5.904,5.383)}
\gppoint{gp mark 1}{(5.906,5.313)}
\gppoint{gp mark 1}{(5.908,5.371)}
\gppoint{gp mark 1}{(5.911,5.420)}
\gppoint{gp mark 1}{(5.913,5.256)}
\gppoint{gp mark 1}{(5.916,5.329)}
\gppoint{gp mark 1}{(5.918,5.447)}
\gppoint{gp mark 1}{(5.921,5.472)}
\gppoint{gp mark 1}{(5.923,5.356)}
\gppoint{gp mark 1}{(5.925,5.296)}
\gppoint{gp mark 1}{(5.928,5.384)}
\gppoint{gp mark 1}{(5.930,5.424)}
\gppoint{gp mark 1}{(5.933,5.368)}
\gppoint{gp mark 1}{(5.935,5.384)}
\gppoint{gp mark 1}{(5.937,5.267)}
\gppoint{gp mark 1}{(5.940,5.331)}
\gppoint{gp mark 1}{(5.942,5.372)}
\gppoint{gp mark 1}{(5.945,5.398)}
\gppoint{gp mark 1}{(5.947,5.342)}
\gppoint{gp mark 1}{(5.949,5.221)}
\gppoint{gp mark 1}{(5.952,5.339)}
\gppoint{gp mark 1}{(5.954,5.454)}
\gppoint{gp mark 1}{(5.957,5.435)}
\gppoint{gp mark 1}{(5.959,5.293)}
\gppoint{gp mark 1}{(5.962,5.358)}
\gppoint{gp mark 1}{(5.964,5.329)}
\gppoint{gp mark 1}{(5.966,5.397)}
\gppoint{gp mark 1}{(5.969,5.252)}
\gppoint{gp mark 1}{(5.971,5.302)}
\gppoint{gp mark 1}{(5.974,5.236)}
\gppoint{gp mark 1}{(5.976,5.337)}
\gppoint{gp mark 1}{(5.978,5.299)}
\gppoint{gp mark 1}{(5.981,5.359)}
\gppoint{gp mark 1}{(5.983,5.384)}
\gppoint{gp mark 1}{(5.986,5.262)}
\gppoint{gp mark 1}{(5.988,5.304)}
\gppoint{gp mark 1}{(5.990,5.396)}
\gppoint{gp mark 1}{(5.993,5.402)}
\gppoint{gp mark 1}{(5.995,5.299)}
\gppoint{gp mark 1}{(5.998,5.262)}
\gppoint{gp mark 1}{(6.000,5.352)}
\gppoint{gp mark 1}{(6.003,5.359)}
\gppoint{gp mark 1}{(6.005,5.378)}
\gppoint{gp mark 1}{(6.007,5.270)}
\gppoint{gp mark 1}{(6.010,5.397)}
\gppoint{gp mark 1}{(6.012,5.312)}
\gppoint{gp mark 1}{(6.015,5.271)}
\gppoint{gp mark 1}{(6.017,5.308)}
\gppoint{gp mark 1}{(6.019,5.328)}
\gppoint{gp mark 1}{(6.022,5.331)}
\gppoint{gp mark 1}{(6.024,5.289)}
\gppoint{gp mark 1}{(6.027,5.335)}
\gppoint{gp mark 1}{(6.029,5.305)}
\gppoint{gp mark 1}{(6.031,5.355)}
\gppoint{gp mark 1}{(6.034,5.315)}
\gppoint{gp mark 1}{(6.036,5.357)}
\gppoint{gp mark 1}{(6.039,5.217)}
\gppoint{gp mark 1}{(6.041,5.280)}
\gppoint{gp mark 1}{(6.044,5.431)}
\gppoint{gp mark 1}{(6.046,5.178)}
\gppoint{gp mark 1}{(6.048,5.272)}
\gppoint{gp mark 1}{(6.051,5.298)}
\gppoint{gp mark 1}{(6.053,5.287)}
\gppoint{gp mark 1}{(6.056,5.309)}
\gppoint{gp mark 1}{(6.058,5.313)}
\gppoint{gp mark 1}{(6.060,5.330)}
\gppoint{gp mark 1}{(6.063,5.374)}
\gppoint{gp mark 1}{(6.065,5.247)}
\gppoint{gp mark 1}{(6.068,5.202)}
\gppoint{gp mark 1}{(6.070,5.192)}
\gppoint{gp mark 1}{(6.072,5.177)}
\gppoint{gp mark 1}{(6.075,5.296)}
\gppoint{gp mark 1}{(6.077,5.370)}
\gppoint{gp mark 1}{(6.080,5.269)}
\gppoint{gp mark 1}{(6.082,5.313)}
\gppoint{gp mark 1}{(6.085,5.265)}
\gppoint{gp mark 1}{(6.087,5.333)}
\gppoint{gp mark 1}{(6.089,5.178)}
\gppoint{gp mark 1}{(6.092,5.380)}
\gppoint{gp mark 1}{(6.094,5.388)}
\gppoint{gp mark 1}{(6.097,5.199)}
\gppoint{gp mark 1}{(6.099,5.312)}
\gppoint{gp mark 1}{(6.101,5.323)}
\gppoint{gp mark 1}{(6.104,5.207)}
\gppoint{gp mark 1}{(6.106,5.262)}
\gppoint{gp mark 1}{(6.109,5.214)}
\gppoint{gp mark 1}{(6.111,5.295)}
\gppoint{gp mark 1}{(6.113,5.173)}
\gppoint{gp mark 1}{(6.116,5.267)}
\gppoint{gp mark 1}{(6.118,5.170)}
\gppoint{gp mark 1}{(6.121,5.234)}
\gppoint{gp mark 1}{(6.123,5.200)}
\gppoint{gp mark 1}{(6.126,5.335)}
\gppoint{gp mark 1}{(6.128,5.388)}
\gppoint{gp mark 1}{(6.130,5.162)}
\gppoint{gp mark 1}{(6.133,5.407)}
\gppoint{gp mark 1}{(6.135,5.257)}
\gppoint{gp mark 1}{(6.138,5.420)}
\gppoint{gp mark 1}{(6.140,5.227)}
\gppoint{gp mark 1}{(6.142,5.157)}
\gppoint{gp mark 1}{(6.145,5.257)}
\gppoint{gp mark 1}{(6.147,5.438)}
\gppoint{gp mark 1}{(6.150,5.228)}
\gppoint{gp mark 1}{(6.152,5.283)}
\gppoint{gp mark 1}{(6.154,5.152)}
\gppoint{gp mark 1}{(6.157,5.240)}
\gppoint{gp mark 1}{(6.159,5.245)}
\gppoint{gp mark 1}{(6.162,5.261)}
\gppoint{gp mark 1}{(6.164,5.245)}
\gppoint{gp mark 1}{(6.166,5.381)}
\gppoint{gp mark 1}{(6.169,5.268)}
\gppoint{gp mark 1}{(6.171,5.293)}
\gppoint{gp mark 1}{(6.174,5.250)}
\gppoint{gp mark 1}{(6.176,5.339)}
\gppoint{gp mark 1}{(6.179,5.289)}
\gppoint{gp mark 1}{(6.181,5.257)}
\gppoint{gp mark 1}{(6.183,5.165)}
\gppoint{gp mark 1}{(6.186,5.118)}
\gppoint{gp mark 1}{(6.188,5.268)}
\gppoint{gp mark 1}{(6.191,5.179)}
\gppoint{gp mark 1}{(6.193,5.274)}
\gppoint{gp mark 1}{(6.195,5.281)}
\gppoint{gp mark 1}{(6.198,5.308)}
\gppoint{gp mark 1}{(6.200,5.304)}
\gppoint{gp mark 1}{(6.203,5.222)}
\gppoint{gp mark 1}{(6.205,5.312)}
\gppoint{gp mark 1}{(6.207,5.257)}
\gppoint{gp mark 1}{(6.210,5.199)}
\gppoint{gp mark 1}{(6.212,5.297)}
\gppoint{gp mark 1}{(6.215,5.227)}
\gppoint{gp mark 1}{(6.217,5.281)}
\gppoint{gp mark 1}{(6.220,5.134)}
\gppoint{gp mark 1}{(6.222,5.226)}
\gppoint{gp mark 1}{(6.224,5.096)}
\gppoint{gp mark 1}{(6.227,5.207)}
\gppoint{gp mark 1}{(6.229,5.377)}
\gppoint{gp mark 1}{(6.232,5.171)}
\gppoint{gp mark 1}{(6.234,5.116)}
\gppoint{gp mark 1}{(6.236,5.205)}
\gppoint{gp mark 1}{(6.239,5.255)}
\gppoint{gp mark 1}{(6.241,5.350)}
\gppoint{gp mark 1}{(6.244,5.230)}
\gppoint{gp mark 1}{(6.246,5.200)}
\gppoint{gp mark 1}{(6.248,5.251)}
\gppoint{gp mark 1}{(6.251,5.233)}
\gppoint{gp mark 1}{(6.253,5.278)}
\gppoint{gp mark 1}{(6.256,5.130)}
\gppoint{gp mark 1}{(6.258,5.144)}
\gppoint{gp mark 1}{(6.261,5.189)}
\gppoint{gp mark 1}{(6.263,5.231)}
\gppoint{gp mark 1}{(6.265,5.235)}
\gppoint{gp mark 1}{(6.268,5.178)}
\gppoint{gp mark 1}{(6.270,5.255)}
\gppoint{gp mark 1}{(6.273,5.168)}
\gppoint{gp mark 1}{(6.275,5.304)}
\gppoint{gp mark 1}{(6.277,5.227)}
\gppoint{gp mark 1}{(6.280,5.257)}
\gppoint{gp mark 1}{(6.282,5.284)}
\gppoint{gp mark 1}{(6.285,5.208)}
\gppoint{gp mark 1}{(6.287,5.203)}
\gppoint{gp mark 1}{(6.289,5.140)}
\gppoint{gp mark 1}{(6.292,5.350)}
\gppoint{gp mark 1}{(6.294,5.230)}
\gppoint{gp mark 1}{(6.297,5.142)}
\gppoint{gp mark 1}{(6.299,5.234)}
\gppoint{gp mark 1}{(6.302,5.163)}
\gppoint{gp mark 1}{(6.304,5.222)}
\gppoint{gp mark 1}{(6.306,5.266)}
\gppoint{gp mark 1}{(6.309,5.237)}
\gppoint{gp mark 1}{(6.311,5.185)}
\gppoint{gp mark 1}{(6.314,5.103)}
\gppoint{gp mark 1}{(6.316,5.224)}
\gppoint{gp mark 1}{(6.318,5.220)}
\gppoint{gp mark 1}{(6.321,5.189)}
\gppoint{gp mark 1}{(6.323,5.194)}
\gppoint{gp mark 1}{(6.326,5.148)}
\gppoint{gp mark 1}{(6.328,5.234)}
\gppoint{gp mark 1}{(6.330,5.247)}
\gppoint{gp mark 1}{(6.333,5.173)}
\gppoint{gp mark 1}{(6.335,5.183)}
\gppoint{gp mark 1}{(6.338,5.237)}
\gppoint{gp mark 1}{(6.340,5.153)}
\gppoint{gp mark 1}{(6.343,5.138)}
\gppoint{gp mark 1}{(6.345,5.189)}
\gppoint{gp mark 1}{(6.347,5.138)}
\gppoint{gp mark 1}{(6.350,5.141)}
\gppoint{gp mark 1}{(6.352,5.163)}
\gppoint{gp mark 1}{(6.355,5.153)}
\gppoint{gp mark 1}{(6.357,5.245)}
\gppoint{gp mark 1}{(6.359,5.145)}
\gppoint{gp mark 1}{(6.362,5.252)}
\gppoint{gp mark 1}{(6.364,5.283)}
\gppoint{gp mark 1}{(6.367,5.124)}
\gppoint{gp mark 1}{(6.369,5.144)}
\gppoint{gp mark 1}{(6.371,5.181)}
\gppoint{gp mark 1}{(6.374,5.132)}
\gppoint{gp mark 1}{(6.376,5.252)}
\gppoint{gp mark 1}{(6.379,5.146)}
\gppoint{gp mark 1}{(6.381,5.099)}
\gppoint{gp mark 1}{(6.384,5.246)}
\gppoint{gp mark 1}{(6.386,5.145)}
\gppoint{gp mark 1}{(6.388,5.240)}
\gppoint{gp mark 1}{(6.391,5.199)}
\gppoint{gp mark 1}{(6.393,5.182)}
\gppoint{gp mark 1}{(6.396,5.152)}
\gppoint{gp mark 1}{(6.398,5.104)}
\gppoint{gp mark 1}{(6.400,5.201)}
\gppoint{gp mark 1}{(6.403,5.260)}
\gppoint{gp mark 1}{(6.405,5.149)}
\gppoint{gp mark 1}{(6.408,5.206)}
\gppoint{gp mark 1}{(6.410,5.205)}
\gppoint{gp mark 1}{(6.412,5.197)}
\gppoint{gp mark 1}{(6.415,5.128)}
\gppoint{gp mark 1}{(6.417,5.211)}
\gppoint{gp mark 1}{(6.420,5.180)}
\gppoint{gp mark 1}{(6.422,5.059)}
\gppoint{gp mark 1}{(6.424,5.272)}
\gppoint{gp mark 1}{(6.427,5.193)}
\gppoint{gp mark 1}{(6.429,5.168)}
\gppoint{gp mark 1}{(6.432,5.177)}
\gppoint{gp mark 1}{(6.434,5.155)}
\gppoint{gp mark 1}{(6.437,5.178)}
\gppoint{gp mark 1}{(6.439,5.128)}
\gppoint{gp mark 1}{(6.441,5.151)}
\gppoint{gp mark 1}{(6.444,5.219)}
\gppoint{gp mark 1}{(6.446,5.303)}
\gppoint{gp mark 1}{(6.449,5.249)}
\gppoint{gp mark 1}{(6.451,5.211)}
\gppoint{gp mark 1}{(6.453,5.242)}
\gppoint{gp mark 1}{(6.456,5.168)}
\gppoint{gp mark 1}{(6.458,5.258)}
\gppoint{gp mark 1}{(6.461,5.229)}
\gppoint{gp mark 1}{(6.463,5.283)}
\gppoint{gp mark 1}{(6.465,5.218)}
\gppoint{gp mark 1}{(6.468,5.124)}
\gppoint{gp mark 1}{(6.470,5.302)}
\gppoint{gp mark 1}{(6.473,5.183)}
\gppoint{gp mark 1}{(6.475,5.158)}
\gppoint{gp mark 1}{(6.478,5.131)}
\gppoint{gp mark 1}{(6.480,5.140)}
\gppoint{gp mark 1}{(6.482,5.152)}
\gppoint{gp mark 1}{(6.485,5.216)}
\gppoint{gp mark 1}{(6.487,5.254)}
\gppoint{gp mark 1}{(6.490,5.279)}
\gppoint{gp mark 1}{(6.492,5.152)}
\gppoint{gp mark 1}{(6.494,5.079)}
\gppoint{gp mark 1}{(6.497,5.190)}
\gppoint{gp mark 1}{(6.499,5.203)}
\gppoint{gp mark 1}{(6.502,5.110)}
\gppoint{gp mark 1}{(6.504,5.128)}
\gppoint{gp mark 1}{(6.506,5.134)}
\gppoint{gp mark 1}{(6.509,5.162)}
\gppoint{gp mark 1}{(6.511,5.175)}
\gppoint{gp mark 1}{(6.514,5.204)}
\gppoint{gp mark 1}{(6.516,5.215)}
\gppoint{gp mark 1}{(6.519,5.191)}
\gppoint{gp mark 1}{(6.521,5.118)}
\gppoint{gp mark 1}{(6.523,5.109)}
\gppoint{gp mark 1}{(6.526,5.147)}
\gppoint{gp mark 1}{(6.528,5.136)}
\gppoint{gp mark 1}{(6.531,5.078)}
\gppoint{gp mark 1}{(6.533,5.042)}
\gppoint{gp mark 1}{(6.535,5.236)}
\gppoint{gp mark 1}{(6.538,5.075)}
\gppoint{gp mark 1}{(6.540,5.152)}
\gppoint{gp mark 1}{(6.543,5.108)}
\gppoint{gp mark 1}{(6.545,5.051)}
\gppoint{gp mark 1}{(6.547,5.193)}
\gppoint{gp mark 1}{(6.550,5.031)}
\gppoint{gp mark 1}{(6.552,5.121)}
\gppoint{gp mark 1}{(6.555,5.176)}
\gppoint{gp mark 1}{(6.557,5.068)}
\gppoint{gp mark 1}{(6.560,5.197)}
\gppoint{gp mark 1}{(6.562,5.067)}
\gppoint{gp mark 1}{(6.564,5.056)}
\gppoint{gp mark 1}{(6.567,5.136)}
\gppoint{gp mark 1}{(6.569,5.188)}
\gppoint{gp mark 1}{(6.572,5.102)}
\gppoint{gp mark 1}{(6.574,5.071)}
\gppoint{gp mark 1}{(6.576,5.208)}
\gppoint{gp mark 1}{(6.579,5.084)}
\gppoint{gp mark 1}{(6.581,5.217)}
\gppoint{gp mark 1}{(6.584,5.242)}
\gppoint{gp mark 1}{(6.586,5.078)}
\gppoint{gp mark 1}{(6.588,5.116)}
\gppoint{gp mark 1}{(6.591,5.222)}
\gppoint{gp mark 1}{(6.593,5.065)}
\gppoint{gp mark 1}{(6.596,5.104)}
\gppoint{gp mark 1}{(6.598,5.001)}
\gppoint{gp mark 1}{(6.601,5.148)}
\gppoint{gp mark 1}{(6.603,5.083)}
\gppoint{gp mark 1}{(6.605,5.089)}
\gppoint{gp mark 1}{(6.608,5.115)}
\gppoint{gp mark 1}{(6.610,5.202)}
\gppoint{gp mark 1}{(6.613,5.024)}
\gppoint{gp mark 1}{(6.615,5.015)}
\gppoint{gp mark 1}{(6.617,5.073)}
\gppoint{gp mark 1}{(6.620,5.077)}
\gppoint{gp mark 1}{(6.622,5.150)}
\gppoint{gp mark 1}{(6.625,5.178)}
\gppoint{gp mark 1}{(6.627,5.200)}
\gppoint{gp mark 1}{(6.629,5.074)}
\gppoint{gp mark 1}{(6.632,5.152)}
\gppoint{gp mark 1}{(6.634,5.013)}
\gppoint{gp mark 1}{(6.637,5.193)}
\gppoint{gp mark 1}{(6.639,5.149)}
\gppoint{gp mark 1}{(6.641,5.201)}
\gppoint{gp mark 1}{(6.644,4.986)}
\gppoint{gp mark 1}{(6.646,5.173)}
\gppoint{gp mark 1}{(6.649,5.099)}
\gppoint{gp mark 1}{(6.651,5.091)}
\gppoint{gp mark 1}{(6.654,5.147)}
\gppoint{gp mark 1}{(6.656,5.121)}
\gppoint{gp mark 1}{(6.658,5.104)}
\gppoint{gp mark 1}{(6.661,5.109)}
\gppoint{gp mark 1}{(6.663,5.024)}
\gppoint{gp mark 1}{(6.666,5.027)}
\gppoint{gp mark 1}{(6.668,5.196)}
\gppoint{gp mark 1}{(6.670,5.039)}
\gppoint{gp mark 1}{(6.673,5.070)}
\gppoint{gp mark 1}{(6.675,5.076)}
\gppoint{gp mark 1}{(6.678,5.021)}
\gppoint{gp mark 1}{(6.680,5.069)}
\gppoint{gp mark 1}{(6.682,5.072)}
\gppoint{gp mark 1}{(6.685,5.067)}
\gppoint{gp mark 1}{(6.687,5.122)}
\gppoint{gp mark 1}{(6.690,5.141)}
\gppoint{gp mark 1}{(6.692,5.105)}
\gppoint{gp mark 1}{(6.695,5.092)}
\gppoint{gp mark 1}{(6.697,5.024)}
\gppoint{gp mark 1}{(6.699,5.043)}
\gppoint{gp mark 1}{(6.702,5.126)}
\gppoint{gp mark 1}{(6.704,4.904)}
\gppoint{gp mark 1}{(6.707,5.113)}
\gppoint{gp mark 1}{(6.709,4.999)}
\gppoint{gp mark 1}{(6.711,5.084)}
\gppoint{gp mark 1}{(6.714,5.037)}
\gppoint{gp mark 1}{(6.716,5.139)}
\gppoint{gp mark 1}{(6.719,4.983)}
\gppoint{gp mark 1}{(6.721,5.099)}
\gppoint{gp mark 1}{(6.723,5.171)}
\gppoint{gp mark 1}{(6.726,5.005)}
\gppoint{gp mark 1}{(6.728,5.017)}
\gppoint{gp mark 1}{(6.731,5.154)}
\gppoint{gp mark 1}{(6.733,4.901)}
\gppoint{gp mark 1}{(6.736,5.040)}
\gppoint{gp mark 1}{(6.738,5.025)}
\gppoint{gp mark 1}{(6.740,5.220)}
\gppoint{gp mark 1}{(6.743,4.977)}
\gppoint{gp mark 1}{(6.745,5.085)}
\gppoint{gp mark 1}{(6.748,5.077)}
\gppoint{gp mark 1}{(6.750,5.070)}
\gppoint{gp mark 1}{(6.752,5.065)}
\gppoint{gp mark 1}{(6.755,5.145)}
\gppoint{gp mark 1}{(6.757,5.041)}
\gppoint{gp mark 1}{(6.760,5.103)}
\gppoint{gp mark 1}{(6.762,4.967)}
\gppoint{gp mark 1}{(6.764,5.003)}
\gppoint{gp mark 1}{(6.767,5.001)}
\gppoint{gp mark 1}{(6.769,5.082)}
\gppoint{gp mark 1}{(6.772,4.917)}
\gppoint{gp mark 1}{(6.774,5.002)}
\gppoint{gp mark 1}{(6.777,5.011)}
\gppoint{gp mark 1}{(6.779,5.094)}
\gppoint{gp mark 1}{(6.781,5.080)}
\gppoint{gp mark 1}{(6.784,5.057)}
\gppoint{gp mark 1}{(6.786,5.009)}
\gppoint{gp mark 1}{(6.789,5.031)}
\gppoint{gp mark 1}{(6.791,4.997)}
\gppoint{gp mark 1}{(6.793,5.016)}
\gppoint{gp mark 1}{(6.796,4.992)}
\gppoint{gp mark 1}{(6.798,4.964)}
\gppoint{gp mark 1}{(6.801,4.949)}
\gppoint{gp mark 1}{(6.803,5.011)}
\gppoint{gp mark 1}{(6.805,5.115)}
\gppoint{gp mark 1}{(6.808,5.152)}
\gppoint{gp mark 1}{(6.810,5.004)}
\gppoint{gp mark 1}{(6.813,5.047)}
\gppoint{gp mark 1}{(6.815,5.045)}
\gppoint{gp mark 1}{(6.818,5.076)}
\gppoint{gp mark 1}{(6.820,4.940)}
\gppoint{gp mark 1}{(6.822,4.934)}
\gppoint{gp mark 1}{(6.825,4.973)}
\gppoint{gp mark 1}{(6.827,4.955)}
\gppoint{gp mark 1}{(6.830,5.042)}
\gppoint{gp mark 1}{(6.832,4.975)}
\gppoint{gp mark 1}{(6.834,4.928)}
\gppoint{gp mark 1}{(6.837,5.045)}
\gppoint{gp mark 1}{(6.839,4.972)}
\gppoint{gp mark 1}{(6.842,4.991)}
\gppoint{gp mark 1}{(6.844,5.001)}
\gppoint{gp mark 1}{(6.846,4.966)}
\gppoint{gp mark 1}{(6.849,5.060)}
\gppoint{gp mark 1}{(6.851,4.945)}
\gppoint{gp mark 1}{(6.854,4.891)}
\gppoint{gp mark 1}{(6.856,5.098)}
\gppoint{gp mark 1}{(6.859,4.975)}
\gppoint{gp mark 1}{(6.861,4.964)}
\gppoint{gp mark 1}{(6.863,5.003)}
\gppoint{gp mark 1}{(6.866,4.871)}
\gppoint{gp mark 1}{(6.868,5.107)}
\gppoint{gp mark 1}{(6.871,4.989)}
\gppoint{gp mark 1}{(6.873,4.987)}
\gppoint{gp mark 1}{(6.875,4.995)}
\gppoint{gp mark 1}{(6.878,5.111)}
\gppoint{gp mark 1}{(6.880,4.986)}
\gppoint{gp mark 1}{(6.883,4.990)}
\gppoint{gp mark 1}{(6.885,4.947)}
\gppoint{gp mark 1}{(6.887,4.907)}
\gppoint{gp mark 1}{(6.890,5.040)}
\gppoint{gp mark 1}{(6.892,5.082)}
\gppoint{gp mark 1}{(6.895,4.916)}
\gppoint{gp mark 1}{(6.897,5.025)}
\gppoint{gp mark 1}{(6.899,4.943)}
\gppoint{gp mark 1}{(6.902,4.940)}
\gppoint{gp mark 1}{(6.904,4.919)}
\gppoint{gp mark 1}{(6.907,4.942)}
\gppoint{gp mark 1}{(6.909,4.961)}
\gppoint{gp mark 1}{(6.912,5.020)}
\gppoint{gp mark 1}{(6.914,5.107)}
\gppoint{gp mark 1}{(6.916,4.852)}
\gppoint{gp mark 1}{(6.919,4.898)}
\gppoint{gp mark 1}{(6.921,4.846)}
\gppoint{gp mark 1}{(6.924,5.071)}
\gppoint{gp mark 1}{(6.926,4.980)}
\gppoint{gp mark 1}{(6.928,5.016)}
\gppoint{gp mark 1}{(6.931,4.998)}
\gppoint{gp mark 1}{(6.933,5.071)}
\gppoint{gp mark 1}{(6.936,4.865)}
\gppoint{gp mark 1}{(6.938,4.934)}
\gppoint{gp mark 1}{(6.940,4.909)}
\gppoint{gp mark 1}{(6.943,4.993)}
\gppoint{gp mark 1}{(6.945,4.937)}
\gppoint{gp mark 1}{(6.948,5.007)}
\gppoint{gp mark 1}{(6.950,4.938)}
\gppoint{gp mark 1}{(6.953,4.968)}
\gppoint{gp mark 1}{(6.955,5.067)}
\gppoint{gp mark 1}{(6.957,4.980)}
\gppoint{gp mark 1}{(6.960,5.066)}
\gppoint{gp mark 1}{(6.962,4.898)}
\gppoint{gp mark 1}{(6.965,4.986)}
\gppoint{gp mark 1}{(6.967,5.040)}
\gppoint{gp mark 1}{(6.969,5.072)}
\gppoint{gp mark 1}{(6.972,4.972)}
\gppoint{gp mark 1}{(6.974,5.031)}
\gppoint{gp mark 1}{(6.977,4.913)}
\gppoint{gp mark 1}{(6.979,4.861)}
\gppoint{gp mark 1}{(6.981,4.979)}
\gppoint{gp mark 1}{(6.984,5.002)}
\gppoint{gp mark 1}{(6.986,5.069)}
\gppoint{gp mark 1}{(6.989,4.855)}
\gppoint{gp mark 1}{(6.991,5.001)}
\gppoint{gp mark 1}{(6.994,4.938)}
\gppoint{gp mark 1}{(6.996,4.922)}
\gppoint{gp mark 1}{(6.998,5.029)}
\gppoint{gp mark 1}{(7.001,4.945)}
\gppoint{gp mark 1}{(7.003,4.978)}
\gppoint{gp mark 1}{(7.006,4.928)}
\gppoint{gp mark 1}{(7.008,4.903)}
\gppoint{gp mark 1}{(7.010,4.918)}
\gppoint{gp mark 1}{(7.013,4.937)}
\gppoint{gp mark 1}{(7.015,4.901)}
\gppoint{gp mark 1}{(7.018,4.982)}
\gppoint{gp mark 1}{(7.020,4.869)}
\gppoint{gp mark 1}{(7.022,4.916)}
\gppoint{gp mark 1}{(7.025,4.924)}
\gppoint{gp mark 1}{(7.027,4.959)}
\gppoint{gp mark 1}{(7.030,4.869)}
\gppoint{gp mark 1}{(7.032,4.854)}
\gppoint{gp mark 1}{(7.035,4.967)}
\gppoint{gp mark 1}{(7.037,4.824)}
\gppoint{gp mark 1}{(7.039,4.867)}
\gppoint{gp mark 1}{(7.042,4.899)}
\gppoint{gp mark 1}{(7.044,4.903)}
\gppoint{gp mark 1}{(7.047,4.972)}
\gppoint{gp mark 1}{(7.049,4.893)}
\gppoint{gp mark 1}{(7.051,4.957)}
\gppoint{gp mark 1}{(7.054,4.883)}
\gppoint{gp mark 1}{(7.056,4.961)}
\gppoint{gp mark 1}{(7.059,4.945)}
\gppoint{gp mark 1}{(7.061,4.942)}
\gppoint{gp mark 1}{(7.063,4.731)}
\gppoint{gp mark 1}{(7.066,4.987)}
\gppoint{gp mark 1}{(7.068,5.016)}
\gppoint{gp mark 1}{(7.071,4.912)}
\gppoint{gp mark 1}{(7.073,4.935)}
\gppoint{gp mark 1}{(7.076,4.864)}
\gppoint{gp mark 1}{(7.078,4.768)}
\gppoint{gp mark 1}{(7.080,4.944)}
\gppoint{gp mark 1}{(7.083,4.765)}
\gppoint{gp mark 1}{(7.085,5.002)}
\gppoint{gp mark 1}{(7.088,5.008)}
\gppoint{gp mark 1}{(7.090,4.888)}
\gppoint{gp mark 1}{(7.092,4.979)}
\gppoint{gp mark 1}{(7.095,4.904)}
\gppoint{gp mark 1}{(7.097,4.949)}
\gppoint{gp mark 1}{(7.100,4.973)}
\gppoint{gp mark 1}{(7.102,4.848)}
\gppoint{gp mark 1}{(7.104,4.855)}
\gppoint{gp mark 1}{(7.107,4.799)}
\gppoint{gp mark 1}{(7.109,4.853)}
\gppoint{gp mark 1}{(7.112,4.919)}
\gppoint{gp mark 1}{(7.114,4.791)}
\gppoint{gp mark 1}{(7.117,4.872)}
\gppoint{gp mark 1}{(7.119,4.795)}
\gppoint{gp mark 1}{(7.121,4.880)}
\gppoint{gp mark 1}{(7.124,4.955)}
\gppoint{gp mark 1}{(7.126,4.880)}
\gppoint{gp mark 1}{(7.129,4.981)}
\gppoint{gp mark 1}{(7.131,4.967)}
\gppoint{gp mark 1}{(7.133,4.838)}
\gppoint{gp mark 1}{(7.136,4.872)}
\gppoint{gp mark 1}{(7.138,4.882)}
\gppoint{gp mark 1}{(7.141,4.875)}
\gppoint{gp mark 1}{(7.143,4.799)}
\gppoint{gp mark 1}{(7.145,4.844)}
\gppoint{gp mark 1}{(7.148,4.786)}
\gppoint{gp mark 1}{(7.150,4.765)}
\gppoint{gp mark 1}{(7.153,4.824)}
\gppoint{gp mark 1}{(7.155,4.822)}
\gppoint{gp mark 1}{(7.157,4.960)}
\gppoint{gp mark 1}{(7.160,4.832)}
\gppoint{gp mark 1}{(7.162,4.987)}
\gppoint{gp mark 1}{(7.165,4.875)}
\gppoint{gp mark 1}{(7.167,4.862)}
\gppoint{gp mark 1}{(7.170,4.869)}
\gppoint{gp mark 1}{(7.172,4.854)}
\gppoint{gp mark 1}{(7.174,4.856)}
\gppoint{gp mark 1}{(7.177,4.841)}
\gppoint{gp mark 1}{(7.179,4.810)}
\gppoint{gp mark 1}{(7.182,4.824)}
\gppoint{gp mark 1}{(7.184,4.813)}
\gppoint{gp mark 1}{(7.186,4.787)}
\gppoint{gp mark 1}{(7.189,4.828)}
\gppoint{gp mark 1}{(7.191,4.878)}
\gppoint{gp mark 1}{(7.194,4.823)}
\gppoint{gp mark 1}{(7.196,4.740)}
\gppoint{gp mark 1}{(7.198,4.834)}
\gppoint{gp mark 1}{(7.201,4.798)}
\gppoint{gp mark 1}{(7.203,4.954)}
\gppoint{gp mark 1}{(7.206,4.824)}
\gppoint{gp mark 1}{(7.208,4.879)}
\gppoint{gp mark 1}{(7.211,4.941)}
\gppoint{gp mark 1}{(7.213,4.769)}
\gppoint{gp mark 1}{(7.215,4.699)}
\gppoint{gp mark 1}{(7.218,4.828)}
\gppoint{gp mark 1}{(7.220,4.770)}
\gppoint{gp mark 1}{(7.223,4.753)}
\gppoint{gp mark 1}{(7.225,4.899)}
\gppoint{gp mark 1}{(7.227,4.816)}
\gppoint{gp mark 1}{(7.230,4.916)}
\gppoint{gp mark 1}{(7.232,4.827)}
\gppoint{gp mark 1}{(7.235,4.873)}
\gppoint{gp mark 1}{(7.237,4.765)}
\gppoint{gp mark 1}{(7.239,4.786)}
\gppoint{gp mark 1}{(7.242,4.897)}
\gppoint{gp mark 1}{(7.244,4.854)}
\gppoint{gp mark 1}{(7.247,4.783)}
\gppoint{gp mark 1}{(7.249,4.910)}
\gppoint{gp mark 1}{(7.252,4.730)}
\gppoint{gp mark 1}{(7.254,4.787)}
\gppoint{gp mark 1}{(7.256,4.879)}
\gppoint{gp mark 1}{(7.259,4.759)}
\gppoint{gp mark 1}{(7.261,4.888)}
\gppoint{gp mark 1}{(7.264,4.824)}
\gppoint{gp mark 1}{(7.266,4.869)}
\gppoint{gp mark 1}{(7.268,4.702)}
\gppoint{gp mark 1}{(7.271,4.876)}
\gppoint{gp mark 1}{(7.273,4.834)}
\gppoint{gp mark 1}{(7.276,4.810)}
\gppoint{gp mark 1}{(7.278,4.865)}
\gppoint{gp mark 1}{(7.280,4.854)}
\gppoint{gp mark 1}{(7.283,4.689)}
\gppoint{gp mark 1}{(7.285,4.755)}
\gppoint{gp mark 1}{(7.288,4.767)}
\gppoint{gp mark 1}{(7.290,4.878)}
\gppoint{gp mark 1}{(7.293,4.835)}
\gppoint{gp mark 1}{(7.295,4.795)}
\gppoint{gp mark 1}{(7.297,4.727)}
\gppoint{gp mark 1}{(7.300,4.785)}
\gppoint{gp mark 1}{(7.302,4.854)}
\gppoint{gp mark 1}{(7.305,4.707)}
\gppoint{gp mark 1}{(7.307,4.765)}
\gppoint{gp mark 1}{(7.309,4.882)}
\gppoint{gp mark 1}{(7.312,4.736)}
\gppoint{gp mark 1}{(7.314,4.868)}
\gppoint{gp mark 1}{(7.317,4.795)}
\gppoint{gp mark 1}{(7.319,4.930)}
\gppoint{gp mark 1}{(7.321,4.844)}
\gppoint{gp mark 1}{(7.324,4.774)}
\gppoint{gp mark 1}{(7.326,4.707)}
\gppoint{gp mark 1}{(7.329,4.757)}
\gppoint{gp mark 1}{(7.331,4.803)}
\gppoint{gp mark 1}{(7.334,4.686)}
\gppoint{gp mark 1}{(7.336,4.756)}
\gppoint{gp mark 1}{(7.338,4.832)}
\gppoint{gp mark 1}{(7.341,4.780)}
\gppoint{gp mark 1}{(7.343,4.739)}
\gppoint{gp mark 1}{(7.346,4.880)}
\gppoint{gp mark 1}{(7.348,4.830)}
\gppoint{gp mark 1}{(7.350,4.784)}
\gppoint{gp mark 1}{(7.353,4.837)}
\gppoint{gp mark 1}{(7.355,4.802)}
\gppoint{gp mark 1}{(7.358,4.693)}
\gppoint{gp mark 1}{(7.360,4.751)}
\gppoint{gp mark 1}{(7.362,4.792)}
\gppoint{gp mark 1}{(7.365,4.776)}
\gppoint{gp mark 1}{(7.367,4.740)}
\gppoint{gp mark 1}{(7.370,4.832)}
\gppoint{gp mark 1}{(7.372,4.886)}
\gppoint{gp mark 1}{(7.374,4.803)}
\gppoint{gp mark 1}{(7.377,4.824)}
\gppoint{gp mark 1}{(7.379,4.780)}
\gppoint{gp mark 1}{(7.382,4.651)}
\gppoint{gp mark 1}{(7.384,4.701)}
\gppoint{gp mark 1}{(7.387,4.844)}
\gppoint{gp mark 1}{(7.389,4.765)}
\gppoint{gp mark 1}{(7.391,4.795)}
\gppoint{gp mark 1}{(7.394,4.765)}
\gppoint{gp mark 1}{(7.396,4.839)}
\gppoint{gp mark 1}{(7.399,4.767)}
\gppoint{gp mark 1}{(7.401,4.633)}
\gppoint{gp mark 1}{(7.403,4.648)}
\gppoint{gp mark 1}{(7.406,4.802)}
\gppoint{gp mark 1}{(7.408,4.768)}
\gppoint{gp mark 1}{(7.411,4.839)}
\gppoint{gp mark 1}{(7.413,4.772)}
\gppoint{gp mark 1}{(7.415,4.721)}
\gppoint{gp mark 1}{(7.418,4.774)}
\gppoint{gp mark 1}{(7.420,4.761)}
\gppoint{gp mark 1}{(7.423,4.741)}
\gppoint{gp mark 1}{(7.425,4.645)}
\gppoint{gp mark 1}{(7.428,4.680)}
\gppoint{gp mark 1}{(7.430,4.647)}
\gppoint{gp mark 1}{(7.432,4.791)}
\gppoint{gp mark 1}{(7.435,4.682)}
\gppoint{gp mark 1}{(7.437,4.648)}
\gppoint{gp mark 1}{(7.440,4.814)}
\gppoint{gp mark 1}{(7.442,4.770)}
\gppoint{gp mark 1}{(7.444,4.690)}
\gppoint{gp mark 1}{(7.447,4.633)}
\gppoint{gp mark 1}{(7.449,4.771)}
\gppoint{gp mark 1}{(7.452,4.724)}
\gppoint{gp mark 1}{(7.454,4.818)}
\gppoint{gp mark 1}{(7.456,4.677)}
\gppoint{gp mark 1}{(7.459,4.732)}
\gppoint{gp mark 1}{(7.461,4.655)}
\gppoint{gp mark 1}{(7.464,4.818)}
\gppoint{gp mark 1}{(7.466,4.675)}
\gppoint{gp mark 1}{(7.469,4.744)}
\gppoint{gp mark 1}{(7.471,4.792)}
\gppoint{gp mark 1}{(7.473,4.645)}
\gppoint{gp mark 1}{(7.476,4.779)}
\gppoint{gp mark 1}{(7.478,4.709)}
\gppoint{gp mark 1}{(7.481,4.730)}
\gppoint{gp mark 1}{(7.483,4.751)}
\gppoint{gp mark 1}{(7.485,4.646)}
\gppoint{gp mark 1}{(7.488,4.762)}
\gppoint{gp mark 1}{(7.490,4.765)}
\gppoint{gp mark 1}{(7.493,4.711)}
\gppoint{gp mark 1}{(7.495,4.811)}
\gppoint{gp mark 1}{(7.497,4.666)}
\gppoint{gp mark 1}{(7.500,4.646)}
\gppoint{gp mark 1}{(7.502,4.728)}
\gppoint{gp mark 1}{(7.505,4.706)}
\gppoint{gp mark 1}{(7.507,4.677)}
\gppoint{gp mark 1}{(7.510,4.709)}
\gppoint{gp mark 1}{(7.512,4.656)}
\gppoint{gp mark 1}{(7.514,4.652)}
\gppoint{gp mark 1}{(7.517,4.731)}
\gppoint{gp mark 1}{(7.519,4.768)}
\gppoint{gp mark 1}{(7.522,4.790)}
\gppoint{gp mark 1}{(7.524,4.692)}
\gppoint{gp mark 1}{(7.526,4.579)}
\gppoint{gp mark 1}{(7.529,4.683)}
\gppoint{gp mark 1}{(7.531,4.697)}
\gppoint{gp mark 1}{(7.534,4.795)}
\gppoint{gp mark 1}{(7.536,4.683)}
\gppoint{gp mark 1}{(7.538,4.665)}
\gppoint{gp mark 1}{(7.541,4.676)}
\gppoint{gp mark 1}{(7.543,4.646)}
\gppoint{gp mark 1}{(7.546,4.703)}
\gppoint{gp mark 1}{(7.548,4.732)}
\gppoint{gp mark 1}{(7.551,4.761)}
\gppoint{gp mark 1}{(7.553,4.645)}
\gppoint{gp mark 1}{(7.555,4.658)}
\gppoint{gp mark 1}{(7.558,4.710)}
\gppoint{gp mark 1}{(7.560,4.657)}
\gppoint{gp mark 1}{(7.563,4.618)}
\gppoint{gp mark 1}{(7.565,4.801)}
\gppoint{gp mark 1}{(7.567,4.672)}
\gppoint{gp mark 1}{(7.570,4.693)}
\gppoint{gp mark 1}{(7.572,4.659)}
\gppoint{gp mark 1}{(7.575,4.815)}
\gppoint{gp mark 1}{(7.577,4.737)}
\gppoint{gp mark 1}{(7.579,4.670)}
\gppoint{gp mark 1}{(7.582,4.652)}
\gppoint{gp mark 1}{(7.584,4.731)}
\gppoint{gp mark 1}{(7.587,4.621)}
\gppoint{gp mark 1}{(7.589,4.717)}
\gppoint{gp mark 1}{(7.592,4.623)}
\gppoint{gp mark 1}{(7.594,4.822)}
\gppoint{gp mark 1}{(7.596,4.725)}
\gppoint{gp mark 1}{(7.599,4.662)}
\gppoint{gp mark 1}{(7.601,4.705)}
\gppoint{gp mark 1}{(7.604,4.640)}
\gppoint{gp mark 1}{(7.606,4.664)}
\gppoint{gp mark 1}{(7.608,4.621)}
\gppoint{gp mark 1}{(7.611,4.671)}
\gppoint{gp mark 1}{(7.613,4.710)}
\gppoint{gp mark 1}{(7.616,4.560)}
\gppoint{gp mark 1}{(7.618,4.673)}
\gppoint{gp mark 1}{(7.620,4.736)}
\gppoint{gp mark 1}{(7.623,4.678)}
\gppoint{gp mark 1}{(7.625,4.591)}
\gppoint{gp mark 1}{(7.628,4.695)}
\gppoint{gp mark 1}{(7.630,4.648)}
\gppoint{gp mark 1}{(7.632,4.625)}
\gppoint{gp mark 1}{(7.635,4.727)}
\gppoint{gp mark 1}{(7.637,4.718)}
\gppoint{gp mark 1}{(7.640,4.655)}
\gppoint{gp mark 1}{(7.642,4.618)}
\gppoint{gp mark 1}{(7.645,4.610)}
\gppoint{gp mark 1}{(7.647,4.544)}
\gppoint{gp mark 1}{(7.649,4.669)}
\gppoint{gp mark 1}{(7.652,4.483)}
\gppoint{gp mark 1}{(7.654,4.533)}
\gppoint{gp mark 1}{(7.657,4.660)}
\gppoint{gp mark 1}{(7.659,4.673)}
\gppoint{gp mark 1}{(7.661,4.613)}
\gppoint{gp mark 1}{(7.664,4.660)}
\gppoint{gp mark 1}{(7.666,4.751)}
\gppoint{gp mark 1}{(7.669,4.662)}
\gppoint{gp mark 1}{(7.671,4.714)}
\gppoint{gp mark 1}{(7.673,4.599)}
\gppoint{gp mark 1}{(7.676,4.618)}
\gppoint{gp mark 1}{(7.678,4.518)}
\gppoint{gp mark 1}{(7.681,4.526)}
\gppoint{gp mark 1}{(7.683,4.589)}
\gppoint{gp mark 1}{(7.686,4.706)}
\gppoint{gp mark 1}{(7.688,4.646)}
\gppoint{gp mark 1}{(7.690,4.643)}
\gppoint{gp mark 1}{(7.693,4.677)}
\gppoint{gp mark 1}{(7.695,4.528)}
\gppoint{gp mark 1}{(7.698,4.589)}
\gppoint{gp mark 1}{(7.700,4.530)}
\gppoint{gp mark 1}{(7.702,4.611)}
\gppoint{gp mark 1}{(7.705,4.579)}
\gppoint{gp mark 1}{(7.707,4.585)}
\gppoint{gp mark 1}{(7.710,4.669)}
\gppoint{gp mark 1}{(7.712,4.551)}
\gppoint{gp mark 1}{(7.714,4.601)}
\gppoint{gp mark 1}{(7.717,4.536)}
\gppoint{gp mark 1}{(7.719,4.577)}
\gppoint{gp mark 1}{(7.722,4.648)}
\gppoint{gp mark 1}{(7.724,4.651)}
\gppoint{gp mark 1}{(7.727,4.618)}
\gppoint{gp mark 1}{(7.729,4.583)}
\gppoint{gp mark 1}{(7.731,4.638)}
\gppoint{gp mark 1}{(7.734,4.746)}
\gppoint{gp mark 1}{(7.736,4.652)}
\gppoint{gp mark 1}{(7.739,4.485)}
\gppoint{gp mark 1}{(7.741,4.581)}
\gppoint{gp mark 1}{(7.743,4.540)}
\gppoint{gp mark 1}{(7.746,4.574)}
\gppoint{gp mark 1}{(7.748,4.572)}
\gppoint{gp mark 1}{(7.751,4.731)}
\gppoint{gp mark 1}{(7.753,4.618)}
\gppoint{gp mark 1}{(7.755,4.695)}
\gppoint{gp mark 1}{(7.758,4.577)}
\gppoint{gp mark 1}{(7.760,4.665)}
\gppoint{gp mark 1}{(7.763,4.625)}
\gppoint{gp mark 1}{(7.765,4.557)}
\gppoint{gp mark 1}{(7.768,4.614)}
\gppoint{gp mark 1}{(7.770,4.588)}
\gppoint{gp mark 1}{(7.772,4.570)}
\gppoint{gp mark 1}{(7.775,4.597)}
\gppoint{gp mark 1}{(7.777,4.644)}
\gppoint{gp mark 1}{(7.780,4.500)}
\gppoint{gp mark 1}{(7.782,4.568)}
\gppoint{gp mark 1}{(7.784,4.504)}
\gppoint{gp mark 1}{(7.787,4.704)}
\gppoint{gp mark 1}{(7.789,4.499)}
\gppoint{gp mark 1}{(7.792,4.696)}
\gppoint{gp mark 1}{(7.794,4.599)}
\gppoint{gp mark 1}{(7.796,4.548)}
\gppoint{gp mark 1}{(7.799,4.544)}
\gppoint{gp mark 1}{(7.801,4.587)}
\gppoint{gp mark 1}{(7.804,4.566)}
\gppoint{gp mark 1}{(7.806,4.586)}
\gppoint{gp mark 1}{(7.809,4.497)}
\gppoint{gp mark 1}{(7.811,4.603)}
\gppoint{gp mark 1}{(7.813,4.581)}
\gppoint{gp mark 1}{(7.816,4.540)}
\gppoint{gp mark 1}{(7.818,4.662)}
\gppoint{gp mark 1}{(7.821,4.599)}
\gppoint{gp mark 1}{(7.823,4.618)}
\gppoint{gp mark 1}{(7.825,4.538)}
\gppoint{gp mark 1}{(7.828,4.481)}
\gppoint{gp mark 1}{(7.830,4.516)}
\gppoint{gp mark 1}{(7.833,4.589)}
\gppoint{gp mark 1}{(7.835,4.487)}
\gppoint{gp mark 1}{(7.837,4.516)}
\gppoint{gp mark 1}{(7.840,4.480)}
\gppoint{gp mark 1}{(7.842,4.586)}
\gppoint{gp mark 1}{(7.845,4.594)}
\gppoint{gp mark 1}{(7.847,4.691)}
\gppoint{gp mark 1}{(7.850,4.554)}
\gppoint{gp mark 1}{(7.852,4.512)}
\gppoint{gp mark 1}{(7.854,4.584)}
\gppoint{gp mark 1}{(7.857,4.455)}
\gppoint{gp mark 1}{(7.859,4.641)}
\gppoint{gp mark 1}{(7.862,4.516)}
\gppoint{gp mark 1}{(7.864,4.474)}
\gppoint{gp mark 1}{(7.866,4.482)}
\gppoint{gp mark 1}{(7.869,4.562)}
\gppoint{gp mark 1}{(7.871,4.435)}
\gppoint{gp mark 1}{(7.874,4.581)}
\gppoint{gp mark 1}{(7.876,4.498)}
\gppoint{gp mark 1}{(7.878,4.356)}
\gppoint{gp mark 1}{(7.881,4.501)}
\gppoint{gp mark 1}{(7.883,4.442)}
\gppoint{gp mark 1}{(7.886,4.491)}
\gppoint{gp mark 1}{(7.888,4.553)}
\gppoint{gp mark 1}{(7.890,4.530)}
\gppoint{gp mark 1}{(7.893,4.613)}
\gppoint{gp mark 1}{(7.895,4.450)}
\gppoint{gp mark 1}{(7.898,4.352)}
\gppoint{gp mark 1}{(7.900,4.494)}
\gppoint{gp mark 1}{(7.903,4.562)}
\gppoint{gp mark 1}{(7.905,4.506)}
\gppoint{gp mark 1}{(7.907,4.552)}
\gppoint{gp mark 1}{(7.910,4.339)}
\gppoint{gp mark 1}{(7.912,4.553)}
\gppoint{gp mark 1}{(7.915,4.504)}
\gppoint{gp mark 1}{(7.917,4.524)}
\gppoint{gp mark 1}{(7.919,4.505)}
\gppoint{gp mark 1}{(7.922,4.589)}
\gppoint{gp mark 1}{(7.924,4.549)}
\gppoint{gp mark 1}{(7.927,4.565)}
\gppoint{gp mark 1}{(7.929,4.504)}
\gppoint{gp mark 1}{(7.931,4.524)}
\gppoint{gp mark 1}{(7.934,4.587)}
\gppoint{gp mark 1}{(7.936,4.496)}
\gppoint{gp mark 1}{(7.939,4.583)}
\gppoint{gp mark 1}{(7.941,4.512)}
\gppoint{gp mark 1}{(7.944,4.566)}
\gppoint{gp mark 1}{(7.946,4.608)}
\gppoint{gp mark 1}{(7.948,4.499)}
\gppoint{gp mark 1}{(7.951,4.546)}
\gppoint{gp mark 1}{(7.953,4.471)}
\gppoint{gp mark 1}{(7.956,4.424)}
\gppoint{gp mark 1}{(7.958,4.446)}
\gppoint{gp mark 1}{(7.960,4.467)}
\gppoint{gp mark 1}{(7.963,4.438)}
\gppoint{gp mark 1}{(7.965,4.422)}
\gppoint{gp mark 1}{(7.968,4.704)}
\gppoint{gp mark 1}{(7.970,4.492)}
\gppoint{gp mark 1}{(7.972,4.443)}
\gppoint{gp mark 1}{(7.975,4.448)}
\gppoint{gp mark 1}{(7.977,4.397)}
\gppoint{gp mark 1}{(7.980,4.384)}
\gppoint{gp mark 1}{(7.982,4.474)}
\gppoint{gp mark 1}{(7.985,4.559)}
\gppoint{gp mark 1}{(7.987,4.489)}
\gppoint{gp mark 1}{(7.989,4.470)}
\gppoint{gp mark 1}{(7.992,4.435)}
\gppoint{gp mark 1}{(7.994,4.549)}
\gppoint{gp mark 1}{(7.997,4.507)}
\gppoint{gp mark 1}{(7.999,4.546)}
\gppoint{gp mark 1}{(8.001,4.494)}
\gppoint{gp mark 1}{(8.004,4.429)}
\gppoint{gp mark 1}{(8.006,4.506)}
\gppoint{gp mark 1}{(8.009,4.396)}
\gppoint{gp mark 1}{(8.011,4.449)}
\gppoint{gp mark 1}{(8.013,4.446)}
\gppoint{gp mark 1}{(8.016,4.452)}
\gppoint{gp mark 1}{(8.018,4.516)}
\gppoint{gp mark 1}{(8.021,4.394)}
\gppoint{gp mark 1}{(8.023,4.455)}
\gppoint{gp mark 1}{(8.026,4.603)}
\gppoint{gp mark 1}{(8.028,4.471)}
\gppoint{gp mark 1}{(8.030,4.506)}
\gppoint{gp mark 1}{(8.033,4.476)}
\gppoint{gp mark 1}{(8.035,4.510)}
\gppoint{gp mark 1}{(8.038,4.467)}
\gppoint{gp mark 1}{(8.040,4.524)}
\gppoint{gp mark 1}{(8.042,4.323)}
\gppoint{gp mark 1}{(8.045,4.468)}
\gppoint{gp mark 1}{(8.047,4.510)}
\gppoint{gp mark 1}{(8.050,4.506)}
\gppoint{gp mark 1}{(8.052,4.503)}
\gppoint{gp mark 1}{(8.054,4.517)}
\gppoint{gp mark 1}{(8.057,4.453)}
\gppoint{gp mark 1}{(8.059,4.445)}
\gppoint{gp mark 1}{(8.062,4.426)}
\gppoint{gp mark 1}{(8.064,4.507)}
\gppoint{gp mark 1}{(8.067,4.415)}
\gppoint{gp mark 1}{(8.069,4.460)}
\gppoint{gp mark 1}{(8.071,4.530)}
\gppoint{gp mark 1}{(8.074,4.321)}
\gppoint{gp mark 1}{(8.076,4.479)}
\gppoint{gp mark 1}{(8.079,4.468)}
\gppoint{gp mark 1}{(8.081,4.403)}
\gppoint{gp mark 1}{(8.083,4.478)}
\gppoint{gp mark 1}{(8.086,4.412)}
\gppoint{gp mark 1}{(8.088,4.551)}
\gppoint{gp mark 1}{(8.091,4.469)}
\gppoint{gp mark 1}{(8.093,4.485)}
\gppoint{gp mark 1}{(8.095,4.589)}
\gppoint{gp mark 1}{(8.098,4.449)}
\gppoint{gp mark 1}{(8.100,4.465)}
\gppoint{gp mark 1}{(8.103,4.383)}
\gppoint{gp mark 1}{(8.105,4.400)}
\gppoint{gp mark 1}{(8.108,4.482)}
\gppoint{gp mark 1}{(8.110,4.441)}
\gppoint{gp mark 1}{(8.112,4.413)}
\gppoint{gp mark 1}{(8.115,4.436)}
\gppoint{gp mark 1}{(8.117,4.415)}
\gppoint{gp mark 1}{(8.120,4.533)}
\gppoint{gp mark 1}{(8.122,4.429)}
\gppoint{gp mark 1}{(8.124,4.484)}
\gppoint{gp mark 1}{(8.127,4.423)}
\gppoint{gp mark 1}{(8.129,4.493)}
\gppoint{gp mark 1}{(8.132,4.551)}
\gppoint{gp mark 1}{(8.134,4.512)}
\gppoint{gp mark 1}{(8.136,4.322)}
\gppoint{gp mark 1}{(8.139,4.332)}
\gppoint{gp mark 1}{(8.141,4.429)}
\gppoint{gp mark 1}{(8.144,4.395)}
\gppoint{gp mark 1}{(8.146,4.375)}
\gppoint{gp mark 1}{(8.148,4.395)}
\gppoint{gp mark 1}{(8.151,4.487)}
\gppoint{gp mark 1}{(8.153,4.434)}
\gppoint{gp mark 1}{(8.156,4.269)}
\gppoint{gp mark 1}{(8.158,4.406)}
\gppoint{gp mark 1}{(8.161,4.471)}
\gppoint{gp mark 1}{(8.163,4.352)}
\gppoint{gp mark 1}{(8.165,4.449)}
\gppoint{gp mark 1}{(8.168,4.462)}
\gppoint{gp mark 1}{(8.170,4.343)}
\gppoint{gp mark 1}{(8.173,4.546)}
\gppoint{gp mark 1}{(8.175,4.341)}
\gppoint{gp mark 1}{(8.177,4.415)}
\gppoint{gp mark 1}{(8.180,4.430)}
\gppoint{gp mark 1}{(8.182,4.497)}
\gppoint{gp mark 1}{(8.185,4.386)}
\gppoint{gp mark 1}{(8.187,4.482)}
\gppoint{gp mark 1}{(8.189,4.446)}
\gppoint{gp mark 1}{(8.192,4.462)}
\gppoint{gp mark 1}{(8.194,4.283)}
\gppoint{gp mark 1}{(8.197,4.336)}
\gppoint{gp mark 1}{(8.199,4.471)}
\gppoint{gp mark 1}{(8.202,4.448)}
\gppoint{gp mark 1}{(8.204,4.441)}
\gppoint{gp mark 1}{(8.206,4.338)}
\gppoint{gp mark 1}{(8.209,4.581)}
\gppoint{gp mark 1}{(8.211,4.440)}
\gppoint{gp mark 1}{(8.214,4.445)}
\gppoint{gp mark 1}{(8.216,4.374)}
\gppoint{gp mark 1}{(8.218,4.418)}
\gppoint{gp mark 1}{(8.221,4.412)}
\gppoint{gp mark 1}{(8.223,4.397)}
\gppoint{gp mark 1}{(8.226,4.560)}
\gppoint{gp mark 1}{(8.228,4.409)}
\gppoint{gp mark 1}{(8.230,4.357)}
\gppoint{gp mark 1}{(8.233,4.422)}
\gppoint{gp mark 1}{(8.235,4.350)}
\gppoint{gp mark 1}{(8.238,4.472)}
\gppoint{gp mark 1}{(8.240,4.412)}
\gppoint{gp mark 1}{(8.243,4.433)}
\gppoint{gp mark 1}{(8.245,4.341)}
\gppoint{gp mark 1}{(8.247,4.481)}
\gppoint{gp mark 1}{(8.250,4.546)}
\gppoint{gp mark 1}{(8.252,4.514)}
\gppoint{gp mark 1}{(8.255,4.378)}
\gppoint{gp mark 1}{(8.257,4.385)}
\gppoint{gp mark 1}{(8.259,4.341)}
\gppoint{gp mark 1}{(8.262,4.417)}
\gppoint{gp mark 1}{(8.264,4.368)}
\gppoint{gp mark 1}{(8.267,4.527)}
\gppoint{gp mark 1}{(8.269,4.421)}
\gppoint{gp mark 1}{(8.271,4.394)}
\gppoint{gp mark 1}{(8.274,4.290)}
\gppoint{gp mark 1}{(8.276,4.423)}
\gppoint{gp mark 1}{(8.279,4.452)}
\gppoint{gp mark 1}{(8.281,4.454)}
\gppoint{gp mark 1}{(8.284,4.312)}
\gppoint{gp mark 1}{(8.286,4.298)}
\gppoint{gp mark 1}{(8.288,4.408)}
\gppoint{gp mark 1}{(8.291,4.336)}
\gppoint{gp mark 1}{(8.293,4.256)}
\gppoint{gp mark 1}{(8.296,4.437)}
\gppoint{gp mark 1}{(8.298,4.418)}
\gppoint{gp mark 1}{(8.300,4.478)}
\gppoint{gp mark 1}{(8.303,4.348)}
\gppoint{gp mark 1}{(8.305,4.368)}
\gppoint{gp mark 1}{(8.308,4.338)}
\gppoint{gp mark 1}{(8.310,4.444)}
\gppoint{gp mark 1}{(8.312,4.454)}
\gppoint{gp mark 1}{(8.315,4.397)}
\gppoint{gp mark 1}{(8.317,4.302)}
\gppoint{gp mark 1}{(8.320,4.448)}
\gppoint{gp mark 1}{(8.322,4.283)}
\gppoint{gp mark 1}{(8.325,4.421)}
\gppoint{gp mark 1}{(8.327,4.450)}
\gppoint{gp mark 1}{(8.329,4.409)}
\gppoint{gp mark 1}{(8.332,4.472)}
\gppoint{gp mark 1}{(8.334,4.513)}
\gppoint{gp mark 1}{(8.337,4.383)}
\gppoint{gp mark 1}{(8.339,4.306)}
\gppoint{gp mark 1}{(8.341,4.367)}
\gppoint{gp mark 1}{(8.344,4.378)}
\gppoint{gp mark 1}{(8.346,4.395)}
\gppoint{gp mark 1}{(8.349,4.383)}
\gppoint{gp mark 1}{(8.351,4.420)}
\gppoint{gp mark 1}{(8.353,4.361)}
\gppoint{gp mark 1}{(8.356,4.345)}
\gppoint{gp mark 1}{(8.358,4.417)}
\gppoint{gp mark 1}{(8.361,4.419)}
\gppoint{gp mark 1}{(8.363,4.331)}
\gppoint{gp mark 1}{(8.365,4.371)}
\gppoint{gp mark 1}{(8.368,4.271)}
\gppoint{gp mark 1}{(8.370,4.252)}
\gppoint{gp mark 1}{(8.373,4.411)}
\gppoint{gp mark 1}{(8.375,4.368)}
\gppoint{gp mark 1}{(8.378,4.329)}
\gppoint{gp mark 1}{(8.380,4.312)}
\gppoint{gp mark 1}{(8.382,4.386)}
\gppoint{gp mark 1}{(8.385,4.366)}
\gppoint{gp mark 1}{(8.387,4.394)}
\gppoint{gp mark 1}{(8.390,4.363)}
\gppoint{gp mark 1}{(8.392,4.338)}
\gppoint{gp mark 1}{(8.394,4.365)}
\gppoint{gp mark 1}{(8.397,4.407)}
\gppoint{gp mark 1}{(8.399,4.362)}
\gppoint{gp mark 1}{(8.402,4.308)}
\gppoint{gp mark 1}{(8.404,4.385)}
\gppoint{gp mark 1}{(8.406,4.385)}
\gppoint{gp mark 1}{(8.409,4.353)}
\gppoint{gp mark 1}{(8.411,4.450)}
\gppoint{gp mark 1}{(8.414,4.362)}
\gppoint{gp mark 1}{(8.416,4.389)}
\gppoint{gp mark 1}{(8.419,4.340)}
\gppoint{gp mark 1}{(8.421,4.406)}
\gppoint{gp mark 1}{(8.423,4.308)}
\gppoint{gp mark 1}{(8.426,4.283)}
\gppoint{gp mark 1}{(8.428,4.392)}
\gppoint{gp mark 1}{(8.431,4.263)}
\gppoint{gp mark 1}{(8.433,4.377)}
\gppoint{gp mark 1}{(8.435,4.263)}
\gppoint{gp mark 1}{(8.438,4.325)}
\gppoint{gp mark 1}{(8.440,4.312)}
\gppoint{gp mark 1}{(8.443,4.315)}
\gppoint{gp mark 1}{(8.445,4.336)}
\gppoint{gp mark 1}{(8.447,4.343)}
\gppoint{gp mark 1}{(8.450,4.384)}
\gppoint{gp mark 1}{(8.452,4.362)}
\gppoint{gp mark 1}{(8.455,4.324)}
\gppoint{gp mark 1}{(8.457,4.282)}
\gppoint{gp mark 1}{(8.460,4.324)}
\gppoint{gp mark 1}{(8.462,4.311)}
\gppoint{gp mark 1}{(8.464,4.229)}
\gppoint{gp mark 1}{(8.467,4.413)}
\gppoint{gp mark 1}{(8.469,4.213)}
\gppoint{gp mark 1}{(8.472,4.133)}
\gppoint{gp mark 1}{(8.474,4.233)}
\gppoint{gp mark 1}{(8.476,4.365)}
\gppoint{gp mark 1}{(8.479,4.279)}
\gppoint{gp mark 1}{(8.481,4.316)}
\gppoint{gp mark 1}{(8.484,4.316)}
\gppoint{gp mark 1}{(8.486,4.294)}
\gppoint{gp mark 1}{(8.488,4.367)}
\gppoint{gp mark 1}{(8.491,4.329)}
\gppoint{gp mark 1}{(8.493,4.308)}
\gppoint{gp mark 1}{(8.496,4.252)}
\gppoint{gp mark 1}{(8.498,4.293)}
\gppoint{gp mark 1}{(8.501,4.245)}
\gppoint{gp mark 1}{(8.503,4.275)}
\gppoint{gp mark 1}{(8.505,4.253)}
\gppoint{gp mark 1}{(8.508,4.107)}
\gppoint{gp mark 1}{(8.510,4.255)}
\gppoint{gp mark 1}{(8.513,4.343)}
\gppoint{gp mark 1}{(8.515,4.279)}
\gppoint{gp mark 1}{(8.517,4.427)}
\gppoint{gp mark 1}{(8.520,4.317)}
\gppoint{gp mark 1}{(8.522,4.225)}
\gppoint{gp mark 1}{(8.525,4.195)}
\gppoint{gp mark 1}{(8.527,4.339)}
\gppoint{gp mark 1}{(8.529,4.225)}
\gppoint{gp mark 1}{(8.532,4.252)}
\gppoint{gp mark 1}{(8.534,4.332)}
\gppoint{gp mark 1}{(8.537,4.145)}
\gppoint{gp mark 1}{(8.539,4.187)}
\gppoint{gp mark 1}{(8.542,4.212)}
\gppoint{gp mark 1}{(8.544,4.227)}
\gppoint{gp mark 1}{(8.546,4.291)}
\gppoint{gp mark 1}{(8.549,4.347)}
\gppoint{gp mark 1}{(8.551,4.271)}
\gppoint{gp mark 1}{(8.554,4.268)}
\gppoint{gp mark 1}{(8.556,4.250)}
\gppoint{gp mark 1}{(8.558,4.224)}
\gppoint{gp mark 1}{(8.561,4.318)}
\gppoint{gp mark 1}{(8.563,4.220)}
\gppoint{gp mark 1}{(8.566,4.216)}
\gppoint{gp mark 1}{(8.568,4.251)}
\gppoint{gp mark 1}{(8.570,4.346)}
\gppoint{gp mark 1}{(8.573,4.234)}
\gppoint{gp mark 1}{(8.575,4.291)}
\gppoint{gp mark 1}{(8.578,4.239)}
\gppoint{gp mark 1}{(8.580,4.352)}
\gppoint{gp mark 1}{(8.583,4.287)}
\gppoint{gp mark 1}{(8.585,4.254)}
\gppoint{gp mark 1}{(8.587,4.315)}
\gppoint{gp mark 1}{(8.590,4.288)}
\gppoint{gp mark 1}{(8.592,4.273)}
\gppoint{gp mark 1}{(8.595,4.245)}
\gppoint{gp mark 1}{(8.597,4.238)}
\gppoint{gp mark 1}{(8.599,4.352)}
\gppoint{gp mark 1}{(8.602,4.219)}
\gppoint{gp mark 1}{(8.604,4.269)}
\gppoint{gp mark 1}{(8.607,4.271)}
\gppoint{gp mark 1}{(8.609,4.129)}
\gppoint{gp mark 1}{(8.611,4.377)}
\gppoint{gp mark 1}{(8.614,4.395)}
\gppoint{gp mark 1}{(8.616,4.197)}
\gppoint{gp mark 1}{(8.619,4.268)}
\gppoint{gp mark 1}{(8.621,4.335)}
\gppoint{gp mark 1}{(8.623,4.367)}
\gppoint{gp mark 1}{(8.626,4.251)}
\gppoint{gp mark 1}{(8.628,4.253)}
\gppoint{gp mark 1}{(8.631,4.313)}
\gppoint{gp mark 1}{(8.633,4.235)}
\gppoint{gp mark 1}{(8.636,4.145)}
\gppoint{gp mark 1}{(8.638,4.173)}
\gppoint{gp mark 1}{(8.640,4.252)}
\gppoint{gp mark 1}{(8.643,4.220)}
\gppoint{gp mark 1}{(8.645,4.313)}
\gppoint{gp mark 1}{(8.648,4.226)}
\gppoint{gp mark 1}{(8.650,4.290)}
\gppoint{gp mark 1}{(8.652,4.270)}
\gppoint{gp mark 1}{(8.655,4.237)}
\gppoint{gp mark 1}{(8.657,4.191)}
\gppoint{gp mark 1}{(8.660,4.087)}
\gppoint{gp mark 1}{(8.662,4.198)}
\gppoint{gp mark 1}{(8.664,4.285)}
\gppoint{gp mark 1}{(8.667,4.332)}
\gppoint{gp mark 1}{(8.669,4.248)}
\gppoint{gp mark 1}{(8.672,4.264)}
\gppoint{gp mark 1}{(8.674,4.178)}
\gppoint{gp mark 1}{(8.677,4.255)}
\gppoint{gp mark 1}{(8.679,4.338)}
\gppoint{gp mark 1}{(8.681,4.168)}
\gppoint{gp mark 1}{(8.684,4.145)}
\gppoint{gp mark 1}{(8.686,4.352)}
\gppoint{gp mark 1}{(8.689,4.220)}
\gppoint{gp mark 1}{(8.691,4.275)}
\gppoint{gp mark 1}{(8.693,4.233)}
\gppoint{gp mark 1}{(8.696,4.161)}
\gppoint{gp mark 1}{(8.698,4.262)}
\gppoint{gp mark 1}{(8.701,4.213)}
\gppoint{gp mark 1}{(8.703,4.229)}
\gppoint{gp mark 1}{(8.705,4.108)}
\gppoint{gp mark 1}{(8.708,4.223)}
\gppoint{gp mark 1}{(8.710,4.288)}
\gppoint{gp mark 1}{(8.713,4.319)}
\gppoint{gp mark 1}{(8.715,4.118)}
\gppoint{gp mark 1}{(8.718,4.282)}
\gppoint{gp mark 1}{(8.720,4.112)}
\gppoint{gp mark 1}{(8.722,4.296)}
\gppoint{gp mark 1}{(8.725,4.276)}
\gppoint{gp mark 1}{(8.727,4.162)}
\gppoint{gp mark 1}{(8.730,4.299)}
\gppoint{gp mark 1}{(8.732,4.249)}
\gppoint{gp mark 1}{(8.734,4.175)}
\gppoint{gp mark 1}{(8.737,4.301)}
\gppoint{gp mark 1}{(8.739,4.160)}
\gppoint{gp mark 1}{(8.742,4.226)}
\gppoint{gp mark 1}{(8.744,4.291)}
\gppoint{gp mark 1}{(8.746,4.204)}
\gppoint{gp mark 1}{(8.749,4.206)}
\gppoint{gp mark 1}{(8.751,4.235)}
\gppoint{gp mark 1}{(8.754,4.112)}
\gppoint{gp mark 1}{(8.756,4.255)}
\gppoint{gp mark 1}{(8.759,4.161)}
\gppoint{gp mark 1}{(8.761,4.243)}
\gppoint{gp mark 1}{(8.763,4.196)}
\gppoint{gp mark 1}{(8.766,4.193)}
\gppoint{gp mark 1}{(8.768,4.149)}
\gppoint{gp mark 1}{(8.771,4.309)}
\gppoint{gp mark 1}{(8.773,4.235)}
\gppoint{gp mark 1}{(8.775,4.150)}
\gppoint{gp mark 1}{(8.778,4.101)}
\gppoint{gp mark 1}{(8.780,4.301)}
\gppoint{gp mark 1}{(8.783,4.129)}
\gppoint{gp mark 1}{(8.785,4.167)}
\gppoint{gp mark 1}{(8.787,4.161)}
\gppoint{gp mark 1}{(8.790,4.253)}
\gppoint{gp mark 1}{(8.792,4.181)}
\gppoint{gp mark 1}{(8.795,4.198)}
\gppoint{gp mark 1}{(8.797,4.253)}
\gppoint{gp mark 1}{(8.800,4.236)}
\gppoint{gp mark 1}{(8.802,4.241)}
\gppoint{gp mark 1}{(8.804,4.132)}
\gppoint{gp mark 1}{(8.807,4.175)}
\gppoint{gp mark 1}{(8.809,4.150)}
\gppoint{gp mark 1}{(8.812,4.203)}
\gppoint{gp mark 1}{(8.814,4.065)}
\gppoint{gp mark 1}{(8.816,4.134)}
\gppoint{gp mark 1}{(8.819,4.147)}
\gppoint{gp mark 1}{(8.821,4.145)}
\gppoint{gp mark 1}{(8.824,4.172)}
\gppoint{gp mark 1}{(8.826,4.285)}
\gppoint{gp mark 1}{(8.828,4.101)}
\gppoint{gp mark 1}{(8.831,4.109)}
\gppoint{gp mark 1}{(8.833,4.081)}
\gppoint{gp mark 1}{(8.836,4.314)}
\gppoint{gp mark 1}{(8.838,4.141)}
\gppoint{gp mark 1}{(8.841,4.156)}
\gppoint{gp mark 1}{(8.843,4.094)}
\gppoint{gp mark 1}{(8.845,4.177)}
\gppoint{gp mark 1}{(8.848,4.114)}
\gppoint{gp mark 1}{(8.850,4.150)}
\gppoint{gp mark 1}{(8.853,4.198)}
\gppoint{gp mark 1}{(8.855,4.172)}
\gppoint{gp mark 1}{(8.857,4.152)}
\gppoint{gp mark 1}{(8.860,4.294)}
\gppoint{gp mark 1}{(8.862,4.078)}
\gppoint{gp mark 1}{(8.865,4.221)}
\gppoint{gp mark 1}{(8.867,4.056)}
\gppoint{gp mark 1}{(8.869,4.163)}
\gppoint{gp mark 1}{(8.872,4.198)}
\gppoint{gp mark 1}{(8.874,4.128)}
\gppoint{gp mark 1}{(8.877,4.141)}
\gppoint{gp mark 1}{(8.879,4.122)}
\gppoint{gp mark 1}{(8.881,4.146)}
\gppoint{gp mark 1}{(8.884,4.161)}
\gppoint{gp mark 1}{(8.886,4.271)}
\gppoint{gp mark 1}{(8.889,4.180)}
\gppoint{gp mark 1}{(8.891,4.141)}
\gppoint{gp mark 1}{(8.894,4.182)}
\gppoint{gp mark 1}{(8.896,4.067)}
\gppoint{gp mark 1}{(8.898,4.205)}
\gppoint{gp mark 1}{(8.901,4.222)}
\gppoint{gp mark 1}{(8.903,4.095)}
\gppoint{gp mark 1}{(8.906,4.222)}
\gppoint{gp mark 1}{(8.908,4.156)}
\gppoint{gp mark 1}{(8.910,4.106)}
\gppoint{gp mark 1}{(8.913,4.129)}
\gppoint{gp mark 1}{(8.915,4.097)}
\gppoint{gp mark 1}{(8.918,4.208)}
\gppoint{gp mark 1}{(8.920,4.294)}
\gppoint{gp mark 1}{(8.922,4.203)}
\gppoint{gp mark 1}{(8.925,4.009)}
\gppoint{gp mark 1}{(8.927,4.195)}
\gppoint{gp mark 1}{(8.930,4.130)}
\gppoint{gp mark 1}{(8.932,4.223)}
\gppoint{gp mark 1}{(8.935,4.229)}
\gppoint{gp mark 1}{(8.937,4.168)}
\gppoint{gp mark 1}{(8.939,4.254)}
\gppoint{gp mark 1}{(8.942,4.064)}
\gppoint{gp mark 1}{(8.944,4.149)}
\gppoint{gp mark 1}{(8.947,4.205)}
\gppoint{gp mark 1}{(8.949,4.186)}
\gppoint{gp mark 1}{(8.951,4.270)}
\gppoint{gp mark 1}{(8.954,4.050)}
\gppoint{gp mark 1}{(8.956,4.199)}
\gppoint{gp mark 1}{(8.959,4.039)}
\gppoint{gp mark 1}{(8.961,4.099)}
\gppoint{gp mark 1}{(8.963,4.140)}
\gppoint{gp mark 1}{(8.966,4.185)}
\gppoint{gp mark 1}{(8.968,4.035)}
\gppoint{gp mark 1}{(8.971,4.101)}
\gppoint{gp mark 1}{(8.973,4.167)}
\gppoint{gp mark 1}{(8.976,4.207)}
\gppoint{gp mark 1}{(8.978,4.136)}
\gppoint{gp mark 1}{(8.980,4.150)}
\gppoint{gp mark 1}{(8.983,4.207)}
\gppoint{gp mark 1}{(8.985,4.134)}
\gppoint{gp mark 1}{(8.988,4.180)}
\gppoint{gp mark 1}{(8.990,4.073)}
\gppoint{gp mark 1}{(8.992,4.209)}
\gppoint{gp mark 1}{(8.995,4.007)}
\gppoint{gp mark 1}{(8.997,4.095)}
\gppoint{gp mark 1}{(9.000,4.161)}
\gppoint{gp mark 1}{(9.002,4.098)}
\gppoint{gp mark 1}{(9.004,4.020)}
\gppoint{gp mark 1}{(9.007,4.178)}
\gppoint{gp mark 1}{(9.009,4.043)}
\gppoint{gp mark 1}{(9.012,4.107)}
\gppoint{gp mark 1}{(9.014,4.133)}
\gppoint{gp mark 1}{(9.017,4.097)}
\gppoint{gp mark 1}{(9.019,4.112)}
\gppoint{gp mark 1}{(9.021,4.038)}
\gppoint{gp mark 1}{(9.024,4.209)}
\gppoint{gp mark 1}{(9.026,4.243)}
\gppoint{gp mark 1}{(9.029,4.178)}
\gppoint{gp mark 1}{(9.031,4.139)}
\gppoint{gp mark 1}{(9.033,4.063)}
\gppoint{gp mark 1}{(9.036,4.194)}
\gppoint{gp mark 1}{(9.038,4.117)}
\gppoint{gp mark 1}{(9.041,4.113)}
\gppoint{gp mark 1}{(9.043,4.062)}
\gppoint{gp mark 1}{(9.045,4.013)}
\gppoint{gp mark 1}{(9.048,4.026)}
\gppoint{gp mark 1}{(9.050,4.148)}
\gppoint{gp mark 1}{(9.053,4.199)}
\gppoint{gp mark 1}{(9.055,4.170)}
\gppoint{gp mark 1}{(9.058,4.070)}
\gppoint{gp mark 1}{(9.060,4.002)}
\gppoint{gp mark 1}{(9.062,4.118)}
\gppoint{gp mark 1}{(9.065,4.153)}
\gppoint{gp mark 1}{(9.067,4.082)}
\gppoint{gp mark 1}{(9.070,4.117)}
\gppoint{gp mark 1}{(9.072,4.094)}
\gppoint{gp mark 1}{(9.074,4.124)}
\gppoint{gp mark 1}{(9.077,4.008)}
\gppoint{gp mark 1}{(9.079,4.108)}
\gppoint{gp mark 1}{(9.082,4.121)}
\gppoint{gp mark 1}{(9.084,4.186)}
\gppoint{gp mark 1}{(9.086,4.043)}
\gppoint{gp mark 1}{(9.089,4.278)}
\gppoint{gp mark 1}{(9.091,4.052)}
\gppoint{gp mark 1}{(9.094,4.077)}
\gppoint{gp mark 1}{(9.096,3.912)}
\gppoint{gp mark 1}{(9.099,4.100)}
\gppoint{gp mark 1}{(9.101,4.099)}
\gppoint{gp mark 1}{(9.103,4.159)}
\gppoint{gp mark 1}{(9.106,4.191)}
\gppoint{gp mark 1}{(9.108,4.118)}
\gppoint{gp mark 1}{(9.111,4.067)}
\gppoint{gp mark 1}{(9.113,4.127)}
\gppoint{gp mark 1}{(9.115,3.991)}
\gppoint{gp mark 1}{(9.118,4.105)}
\gppoint{gp mark 1}{(9.120,4.119)}
\gppoint{gp mark 1}{(9.123,3.973)}
\gppoint{gp mark 1}{(9.125,4.027)}
\gppoint{gp mark 1}{(9.127,4.132)}
\gppoint{gp mark 1}{(9.130,4.034)}
\gppoint{gp mark 1}{(9.132,4.130)}
\gppoint{gp mark 1}{(9.135,4.094)}
\gppoint{gp mark 1}{(9.137,3.994)}
\gppoint{gp mark 1}{(9.139,4.028)}
\gppoint{gp mark 1}{(9.142,4.073)}
\gppoint{gp mark 1}{(9.144,4.075)}
\gppoint{gp mark 1}{(9.147,3.992)}
\gppoint{gp mark 1}{(9.149,4.090)}
\gppoint{gp mark 1}{(9.152,4.044)}
\gppoint{gp mark 1}{(9.154,4.073)}
\gppoint{gp mark 1}{(9.156,4.062)}
\gppoint{gp mark 1}{(9.159,4.124)}
\gppoint{gp mark 1}{(9.161,4.112)}
\gppoint{gp mark 1}{(9.164,4.030)}
\gppoint{gp mark 1}{(9.166,4.153)}
\gppoint{gp mark 1}{(9.168,4.026)}
\gppoint{gp mark 1}{(9.171,4.094)}
\gppoint{gp mark 1}{(9.173,4.074)}
\gppoint{gp mark 1}{(9.176,4.069)}
\gppoint{gp mark 1}{(9.178,4.062)}
\gppoint{gp mark 1}{(9.180,4.075)}
\gppoint{gp mark 1}{(9.183,4.029)}
\gppoint{gp mark 1}{(9.185,3.938)}
\gppoint{gp mark 1}{(9.188,4.066)}
\gppoint{gp mark 1}{(9.190,4.050)}
\gppoint{gp mark 1}{(9.193,4.016)}
\gppoint{gp mark 1}{(9.195,4.063)}
\gppoint{gp mark 1}{(9.197,4.030)}
\gppoint{gp mark 1}{(9.200,4.015)}
\gppoint{gp mark 1}{(9.202,4.017)}
\gppoint{gp mark 1}{(9.205,4.084)}
\gppoint{gp mark 1}{(9.207,4.064)}
\gppoint{gp mark 1}{(9.209,4.020)}
\gppoint{gp mark 1}{(9.212,4.021)}
\gppoint{gp mark 1}{(9.214,4.026)}
\gppoint{gp mark 1}{(9.217,4.136)}
\gppoint{gp mark 1}{(9.219,4.075)}
\gppoint{gp mark 1}{(9.221,4.046)}
\gppoint{gp mark 1}{(9.224,3.984)}
\gppoint{gp mark 1}{(9.226,4.073)}
\gppoint{gp mark 1}{(9.229,4.029)}
\gppoint{gp mark 1}{(9.231,3.985)}
\gppoint{gp mark 1}{(9.234,4.057)}
\gppoint{gp mark 1}{(9.236,3.982)}
\gppoint{gp mark 1}{(9.238,3.936)}
\gppoint{gp mark 1}{(9.241,4.017)}
\gppoint{gp mark 1}{(9.243,4.074)}
\gppoint{gp mark 1}{(9.246,3.979)}
\gppoint{gp mark 1}{(9.248,3.991)}
\gppoint{gp mark 1}{(9.250,4.042)}
\gppoint{gp mark 1}{(9.253,4.003)}
\gppoint{gp mark 1}{(9.255,3.998)}
\gppoint{gp mark 1}{(9.258,4.024)}
\gppoint{gp mark 1}{(9.260,4.157)}
\gppoint{gp mark 1}{(9.262,4.019)}
\gppoint{gp mark 1}{(9.265,4.075)}
\gppoint{gp mark 1}{(9.267,4.054)}
\gppoint{gp mark 1}{(9.270,4.110)}
\gppoint{gp mark 1}{(9.272,4.067)}
\gppoint{gp mark 1}{(9.275,3.928)}
\gppoint{gp mark 1}{(9.277,4.025)}
\gppoint{gp mark 1}{(9.279,4.067)}
\gppoint{gp mark 1}{(9.282,3.863)}
\gppoint{gp mark 1}{(9.284,4.043)}
\gppoint{gp mark 1}{(9.287,4.040)}
\gppoint{gp mark 1}{(9.289,4.018)}
\gppoint{gp mark 1}{(9.291,4.146)}
\gppoint{gp mark 1}{(9.294,4.052)}
\gppoint{gp mark 1}{(9.296,4.043)}
\gppoint{gp mark 1}{(9.299,4.157)}
\gppoint{gp mark 1}{(9.301,4.000)}
\gppoint{gp mark 1}{(9.303,3.978)}
\gppoint{gp mark 1}{(9.306,4.102)}
\gppoint{gp mark 1}{(9.308,3.963)}
\gppoint{gp mark 1}{(9.311,4.034)}
\gppoint{gp mark 1}{(9.313,4.080)}
\gppoint{gp mark 1}{(9.316,4.073)}
\gppoint{gp mark 1}{(9.318,4.002)}
\gppoint{gp mark 1}{(9.320,4.046)}
\gppoint{gp mark 1}{(9.323,4.105)}
\gppoint{gp mark 1}{(9.325,4.059)}
\gppoint{gp mark 1}{(9.328,3.874)}
\gppoint{gp mark 1}{(9.330,4.102)}
\gppoint{gp mark 1}{(9.332,3.959)}
\gppoint{gp mark 1}{(9.335,4.059)}
\gppoint{gp mark 1}{(9.337,3.967)}
\gppoint{gp mark 1}{(9.340,4.045)}
\gppoint{gp mark 1}{(9.342,4.025)}
\gppoint{gp mark 1}{(9.344,3.934)}
\gppoint{gp mark 1}{(9.347,3.992)}
\gppoint{gp mark 1}{(9.349,3.976)}
\gppoint{gp mark 1}{(9.352,3.938)}
\gppoint{gp mark 1}{(9.354,4.184)}
\gppoint{gp mark 1}{(9.356,4.007)}
\gppoint{gp mark 1}{(9.359,4.058)}
\gppoint{gp mark 1}{(9.361,3.924)}
\gppoint{gp mark 1}{(9.364,4.015)}
\gppoint{gp mark 1}{(9.366,4.074)}
\gppoint{gp mark 1}{(9.369,4.049)}
\gppoint{gp mark 1}{(9.371,4.048)}
\gppoint{gp mark 1}{(9.373,3.908)}
\gppoint{gp mark 1}{(9.376,3.960)}
\gppoint{gp mark 1}{(9.378,3.904)}
\gppoint{gp mark 1}{(9.381,3.943)}
\gppoint{gp mark 1}{(9.383,3.888)}
\gppoint{gp mark 1}{(9.385,3.944)}
\gppoint{gp mark 1}{(9.388,4.001)}
\gppoint{gp mark 1}{(9.390,3.993)}
\gppoint{gp mark 1}{(9.393,4.047)}
\gppoint{gp mark 1}{(9.395,3.998)}
\gppoint{gp mark 1}{(9.397,4.059)}
\gppoint{gp mark 1}{(9.400,3.915)}
\gppoint{gp mark 1}{(9.402,3.999)}
\gppoint{gp mark 1}{(9.405,4.028)}
\gppoint{gp mark 1}{(9.407,3.970)}
\gppoint{gp mark 1}{(9.410,4.013)}
\gppoint{gp mark 1}{(9.412,3.928)}
\gppoint{gp mark 1}{(9.414,3.956)}
\gppoint{gp mark 1}{(9.417,3.994)}
\gppoint{gp mark 1}{(9.419,3.950)}
\gppoint{gp mark 1}{(9.422,3.936)}
\gppoint{gp mark 1}{(9.424,3.947)}
\gppoint{gp mark 1}{(9.426,3.931)}
\gppoint{gp mark 1}{(9.429,4.045)}
\gppoint{gp mark 1}{(9.431,3.881)}
\gppoint{gp mark 1}{(9.434,4.021)}
\gppoint{gp mark 1}{(9.436,3.964)}
\gppoint{gp mark 1}{(9.438,4.000)}
\gppoint{gp mark 1}{(9.441,3.895)}
\gppoint{gp mark 1}{(9.443,3.903)}
\gppoint{gp mark 1}{(9.446,3.985)}
\gppoint{gp mark 1}{(9.448,4.069)}
\gppoint{gp mark 1}{(9.451,3.957)}
\gppoint{gp mark 1}{(9.453,3.948)}
\gppoint{gp mark 1}{(9.455,4.019)}
\gppoint{gp mark 1}{(9.458,3.851)}
\gppoint{gp mark 1}{(9.460,3.951)}
\gppoint{gp mark 1}{(9.463,4.036)}
\gppoint{gp mark 1}{(9.465,3.829)}
\gppoint{gp mark 1}{(9.467,3.988)}
\gppoint{gp mark 1}{(9.470,3.930)}
\gppoint{gp mark 1}{(9.472,3.992)}
\gppoint{gp mark 1}{(9.475,3.964)}
\gppoint{gp mark 1}{(9.477,3.972)}
\gppoint{gp mark 1}{(9.479,4.025)}
\gppoint{gp mark 1}{(9.482,4.009)}
\gppoint{gp mark 1}{(9.484,4.039)}
\gppoint{gp mark 1}{(9.487,3.947)}
\gppoint{gp mark 1}{(9.489,3.851)}
\gppoint{gp mark 1}{(9.492,3.994)}
\gppoint{gp mark 1}{(9.494,3.957)}
\gppoint{gp mark 1}{(9.496,3.991)}
\gppoint{gp mark 1}{(9.499,3.968)}
\gppoint{gp mark 1}{(9.501,3.898)}
\gppoint{gp mark 1}{(9.504,3.939)}
\gppoint{gp mark 1}{(9.506,4.027)}
\gppoint{gp mark 1}{(9.508,3.890)}
\gppoint{gp mark 1}{(9.511,3.827)}
\gppoint{gp mark 1}{(9.513,3.848)}
\gppoint{gp mark 1}{(9.516,3.969)}
\gppoint{gp mark 1}{(9.518,3.826)}
\gppoint{gp mark 1}{(9.520,4.013)}
\gppoint{gp mark 1}{(9.523,3.983)}
\gppoint{gp mark 1}{(9.525,3.944)}
\gppoint{gp mark 1}{(9.528,4.062)}
\gppoint{gp mark 1}{(9.530,3.910)}
\gppoint{gp mark 1}{(9.533,3.962)}
\gppoint{gp mark 1}{(9.535,3.978)}
\gppoint{gp mark 1}{(9.537,3.922)}
\gppoint{gp mark 1}{(9.540,4.044)}
\gppoint{gp mark 1}{(9.542,3.956)}
\gppoint{gp mark 1}{(9.545,4.031)}
\gppoint{gp mark 1}{(9.547,3.949)}
\gppoint{gp mark 1}{(9.549,3.864)}
\gppoint{gp mark 1}{(9.552,3.945)}
\gppoint{gp mark 1}{(9.554,3.821)}
\gppoint{gp mark 1}{(9.557,3.967)}
\gppoint{gp mark 1}{(9.559,3.840)}
\gppoint{gp mark 1}{(9.561,3.919)}
\gppoint{gp mark 1}{(9.564,3.917)}
\gppoint{gp mark 1}{(9.566,3.922)}
\gppoint{gp mark 1}{(9.569,3.978)}
\gppoint{gp mark 1}{(9.571,3.912)}
\gppoint{gp mark 1}{(9.574,3.968)}
\gppoint{gp mark 1}{(9.576,3.963)}
\gppoint{gp mark 1}{(9.578,3.875)}
\gppoint{gp mark 1}{(9.581,3.978)}
\gppoint{gp mark 1}{(9.583,3.889)}
\gppoint{gp mark 1}{(9.586,3.901)}
\gppoint{gp mark 1}{(9.588,3.910)}
\gppoint{gp mark 1}{(9.590,3.886)}
\gppoint{gp mark 1}{(9.593,3.896)}
\gppoint{gp mark 1}{(9.595,3.852)}
\gppoint{gp mark 1}{(9.598,3.990)}
\gppoint{gp mark 1}{(9.600,3.960)}
\gppoint{gp mark 1}{(9.602,3.892)}
\gppoint{gp mark 1}{(9.605,3.917)}
\gppoint{gp mark 1}{(9.607,3.852)}
\gppoint{gp mark 1}{(9.610,3.880)}
\gppoint{gp mark 1}{(9.612,4.117)}
\gppoint{gp mark 1}{(9.614,3.892)}
\gppoint{gp mark 1}{(9.617,3.829)}
\gppoint{gp mark 1}{(9.619,3.838)}
\gppoint{gp mark 1}{(9.622,3.918)}
\gppoint{gp mark 1}{(9.624,3.861)}
\gppoint{gp mark 1}{(9.627,3.860)}
\gppoint{gp mark 1}{(9.629,3.939)}
\gppoint{gp mark 1}{(9.631,3.953)}
\gppoint{gp mark 1}{(9.634,3.940)}
\gppoint{gp mark 1}{(9.636,4.029)}
\gppoint{gp mark 1}{(9.639,3.859)}
\gppoint{gp mark 1}{(9.641,3.809)}
\gppoint{gp mark 1}{(9.643,3.896)}
\gppoint{gp mark 1}{(9.646,3.860)}
\gppoint{gp mark 1}{(9.648,3.926)}
\gppoint{gp mark 1}{(9.651,3.932)}
\gppoint{gp mark 1}{(9.653,3.805)}
\gppoint{gp mark 1}{(9.655,3.888)}
\gppoint{gp mark 1}{(9.658,3.854)}
\gppoint{gp mark 1}{(9.660,3.875)}
\gppoint{gp mark 1}{(9.663,3.912)}
\gppoint{gp mark 1}{(9.665,3.862)}
\gppoint{gp mark 1}{(9.668,3.897)}
\gppoint{gp mark 1}{(9.670,3.979)}
\gppoint{gp mark 1}{(9.672,3.970)}
\gppoint{gp mark 1}{(9.675,3.888)}
\gppoint{gp mark 1}{(9.677,3.965)}
\gppoint{gp mark 1}{(9.680,3.882)}
\gppoint{gp mark 1}{(9.682,3.926)}
\gppoint{gp mark 1}{(9.684,3.907)}
\gppoint{gp mark 1}{(9.687,3.884)}
\gppoint{gp mark 1}{(9.689,3.834)}
\gppoint{gp mark 1}{(9.692,3.982)}
\gppoint{gp mark 1}{(9.694,3.846)}
\gppoint{gp mark 1}{(9.696,3.905)}
\gppoint{gp mark 1}{(9.699,3.871)}
\gppoint{gp mark 1}{(9.701,3.936)}
\gppoint{gp mark 1}{(9.704,3.878)}
\gppoint{gp mark 1}{(9.706,3.892)}
\gppoint{gp mark 1}{(9.709,3.959)}
\gppoint{gp mark 1}{(9.711,3.956)}
\gppoint{gp mark 1}{(9.713,3.888)}
\gppoint{gp mark 1}{(9.716,3.865)}
\gppoint{gp mark 1}{(9.718,3.960)}
\gppoint{gp mark 1}{(9.721,3.824)}
\gppoint{gp mark 1}{(9.723,3.837)}
\gppoint{gp mark 1}{(9.725,3.824)}
\gppoint{gp mark 1}{(9.728,3.888)}
\gppoint{gp mark 1}{(9.730,3.831)}
\gppoint{gp mark 1}{(9.733,3.860)}
\gppoint{gp mark 1}{(9.735,3.951)}
\gppoint{gp mark 1}{(9.737,3.837)}
\gppoint{gp mark 1}{(9.740,3.820)}
\gppoint{gp mark 1}{(9.742,4.011)}
\gppoint{gp mark 1}{(9.745,3.794)}
\gppoint{gp mark 1}{(9.747,3.934)}
\gppoint{gp mark 1}{(9.750,3.862)}
\gppoint{gp mark 1}{(9.752,3.930)}
\gppoint{gp mark 1}{(9.754,3.971)}
\gppoint{gp mark 1}{(9.757,3.842)}
\gppoint{gp mark 1}{(9.759,3.840)}
\gppoint{gp mark 1}{(9.762,3.941)}
\gppoint{gp mark 1}{(9.764,3.887)}
\gppoint{gp mark 1}{(9.766,3.868)}
\gppoint{gp mark 1}{(9.769,3.868)}
\gppoint{gp mark 1}{(9.771,3.792)}
\gppoint{gp mark 1}{(9.774,3.874)}
\gppoint{gp mark 1}{(9.776,3.811)}
\gppoint{gp mark 1}{(9.778,3.886)}
\gppoint{gp mark 1}{(9.781,3.736)}
\gppoint{gp mark 1}{(9.783,3.853)}
\gppoint{gp mark 1}{(9.786,3.857)}
\gppoint{gp mark 1}{(9.788,3.870)}
\gppoint{gp mark 1}{(9.791,3.891)}
\gppoint{gp mark 1}{(9.793,3.940)}
\gppoint{gp mark 1}{(9.795,3.831)}
\gppoint{gp mark 1}{(9.798,3.993)}
\gppoint{gp mark 1}{(9.800,3.848)}
\gppoint{gp mark 1}{(9.803,3.892)}
\gppoint{gp mark 1}{(9.805,3.835)}
\gppoint{gp mark 1}{(9.807,3.848)}
\gppoint{gp mark 1}{(9.810,3.812)}
\gppoint{gp mark 1}{(9.812,3.820)}
\gppoint{gp mark 1}{(9.815,3.780)}
\gppoint{gp mark 1}{(9.817,3.660)}
\gppoint{gp mark 1}{(9.819,4.001)}
\gppoint{gp mark 1}{(9.822,3.876)}
\gppoint{gp mark 1}{(9.824,3.795)}
\gppoint{gp mark 1}{(9.827,3.890)}
\gppoint{gp mark 1}{(9.829,3.999)}
\gppoint{gp mark 1}{(9.832,3.881)}
\gppoint{gp mark 1}{(9.834,3.786)}
\gppoint{gp mark 1}{(9.836,3.900)}
\gppoint{gp mark 1}{(9.839,3.869)}
\gppoint{gp mark 1}{(9.841,3.701)}
\gppoint{gp mark 1}{(9.844,3.865)}
\gppoint{gp mark 1}{(9.846,3.876)}
\gppoint{gp mark 1}{(9.848,3.790)}
\gppoint{gp mark 1}{(9.851,3.835)}
\gppoint{gp mark 1}{(9.853,3.847)}
\gppoint{gp mark 1}{(9.856,4.007)}
\gppoint{gp mark 1}{(9.858,3.913)}
\gppoint{gp mark 1}{(9.860,3.778)}
\gppoint{gp mark 1}{(9.863,3.836)}
\gppoint{gp mark 1}{(9.865,3.888)}
\gppoint{gp mark 1}{(9.868,3.903)}
\gppoint{gp mark 1}{(9.870,3.819)}
\gppoint{gp mark 1}{(9.872,3.786)}
\gppoint{gp mark 1}{(9.875,3.763)}
\gppoint{gp mark 1}{(9.877,3.899)}
\gppoint{gp mark 1}{(9.880,3.842)}
\gppoint{gp mark 1}{(9.882,3.894)}
\gppoint{gp mark 1}{(9.885,3.748)}
\gppoint{gp mark 1}{(9.887,3.822)}
\gppoint{gp mark 1}{(9.889,3.785)}
\gppoint{gp mark 1}{(9.892,3.824)}
\gppoint{gp mark 1}{(9.894,3.706)}
\gppoint{gp mark 1}{(9.897,3.807)}
\gppoint{gp mark 1}{(9.899,3.793)}
\gppoint{gp mark 1}{(9.901,3.722)}
\gppoint{gp mark 1}{(9.904,3.884)}
\gppoint{gp mark 1}{(9.906,3.854)}
\gppoint{gp mark 1}{(9.909,3.877)}
\gppoint{gp mark 1}{(9.911,3.834)}
\gppoint{gp mark 1}{(9.913,3.837)}
\gppoint{gp mark 1}{(9.916,3.805)}
\gppoint{gp mark 1}{(9.918,3.720)}
\gppoint{gp mark 1}{(9.921,3.804)}
\gppoint{gp mark 1}{(9.923,3.793)}
\gppoint{gp mark 1}{(9.926,3.866)}
\gppoint{gp mark 1}{(9.928,3.888)}
\gppoint{gp mark 1}{(9.930,3.841)}
\gppoint{gp mark 1}{(9.933,3.769)}
\gppoint{gp mark 1}{(9.935,3.793)}
\gppoint{gp mark 1}{(9.938,3.831)}
\gppoint{gp mark 1}{(9.940,3.858)}
\gppoint{gp mark 1}{(9.942,3.799)}
\gppoint{gp mark 1}{(9.945,3.859)}
\gppoint{gp mark 1}{(9.947,3.837)}
\gppoint{gp mark 1}{(9.950,3.859)}
\gppoint{gp mark 1}{(9.952,3.831)}
\gppoint{gp mark 1}{(9.954,3.708)}
\gppoint{gp mark 1}{(9.957,3.748)}
\gppoint{gp mark 1}{(9.959,3.794)}
\gppoint{gp mark 1}{(9.962,3.852)}
\gppoint{gp mark 1}{(9.964,3.777)}
\gppoint{gp mark 1}{(9.967,3.781)}
\gppoint{gp mark 1}{(9.969,3.737)}
\gppoint{gp mark 1}{(9.971,3.734)}
\gppoint{gp mark 1}{(9.974,3.803)}
\gppoint{gp mark 1}{(9.976,3.851)}
\gppoint{gp mark 1}{(9.979,3.850)}
\gppoint{gp mark 1}{(9.981,3.901)}
\gppoint{gp mark 1}{(9.983,3.858)}
\gppoint{gp mark 1}{(9.986,3.825)}
\gppoint{gp mark 1}{(9.988,3.729)}
\gppoint{gp mark 1}{(9.991,3.762)}
\gppoint{gp mark 1}{(9.993,3.763)}
\gppoint{gp mark 1}{(9.995,3.820)}
\gppoint{gp mark 1}{(9.998,3.801)}
\gppoint{gp mark 1}{(10.000,3.692)}
\gppoint{gp mark 1}{(10.003,3.809)}
\gppoint{gp mark 1}{(10.005,3.815)}
\gppoint{gp mark 1}{(10.008,3.814)}
\gppoint{gp mark 1}{(10.010,3.947)}
\gppoint{gp mark 1}{(10.012,3.763)}
\gppoint{gp mark 1}{(10.015,3.805)}
\gppoint{gp mark 1}{(10.017,3.830)}
\gppoint{gp mark 1}{(10.020,3.873)}
\gppoint{gp mark 1}{(10.022,3.918)}
\gppoint{gp mark 1}{(10.024,3.831)}
\gppoint{gp mark 1}{(10.027,3.733)}
\gppoint{gp mark 1}{(10.029,3.801)}
\gppoint{gp mark 1}{(10.032,3.873)}
\gppoint{gp mark 1}{(10.034,3.913)}
\gppoint{gp mark 1}{(10.036,3.828)}
\gppoint{gp mark 1}{(10.039,3.822)}
\gppoint{gp mark 1}{(10.041,3.719)}
\gppoint{gp mark 1}{(10.044,3.911)}
\gppoint{gp mark 1}{(10.046,3.897)}
\gppoint{gp mark 1}{(10.049,3.876)}
\gppoint{gp mark 1}{(10.051,3.796)}
\gppoint{gp mark 1}{(10.053,3.794)}
\gppoint{gp mark 1}{(10.056,3.777)}
\gppoint{gp mark 1}{(10.058,3.754)}
\gppoint{gp mark 1}{(10.061,3.860)}
\gppoint{gp mark 1}{(10.063,3.697)}
\gppoint{gp mark 1}{(10.065,3.853)}
\gppoint{gp mark 1}{(10.068,3.763)}
\gppoint{gp mark 1}{(10.070,3.852)}
\gppoint{gp mark 1}{(10.073,3.655)}
\gppoint{gp mark 1}{(10.075,3.771)}
\gppoint{gp mark 1}{(10.077,3.759)}
\gppoint{gp mark 1}{(10.080,3.749)}
\gppoint{gp mark 1}{(10.082,3.811)}
\gppoint{gp mark 1}{(10.085,3.789)}
\gppoint{gp mark 1}{(10.087,3.767)}
\gppoint{gp mark 1}{(10.089,3.756)}
\gppoint{gp mark 1}{(10.092,3.733)}
\gppoint{gp mark 1}{(10.094,3.822)}
\gppoint{gp mark 1}{(10.097,3.799)}
\gppoint{gp mark 1}{(10.099,3.763)}
\gppoint{gp mark 1}{(10.102,3.914)}
\gppoint{gp mark 1}{(10.104,3.659)}
\gppoint{gp mark 1}{(10.106,3.937)}
\gppoint{gp mark 1}{(10.109,3.832)}
\gppoint{gp mark 1}{(10.111,3.773)}
\gppoint{gp mark 1}{(10.114,3.785)}
\gppoint{gp mark 1}{(10.116,3.729)}
\gppoint{gp mark 1}{(10.118,3.820)}
\gppoint{gp mark 1}{(10.121,3.778)}
\gppoint{gp mark 1}{(10.123,3.798)}
\gppoint{gp mark 1}{(10.126,3.717)}
\gppoint{gp mark 1}{(10.128,3.683)}
\gppoint{gp mark 1}{(10.130,3.728)}
\gppoint{gp mark 1}{(10.133,3.713)}
\gppoint{gp mark 1}{(10.135,3.734)}
\gppoint{gp mark 1}{(10.138,3.804)}
\gppoint{gp mark 1}{(10.140,3.712)}
\gppoint{gp mark 1}{(10.143,3.730)}
\gppoint{gp mark 1}{(10.145,3.607)}
\gppoint{gp mark 1}{(10.147,3.734)}
\gppoint{gp mark 1}{(10.150,3.666)}
\gppoint{gp mark 1}{(10.152,3.859)}
\gppoint{gp mark 1}{(10.155,3.726)}
\gppoint{gp mark 1}{(10.157,3.782)}
\gppoint{gp mark 1}{(10.159,3.795)}
\gppoint{gp mark 1}{(10.162,3.751)}
\gppoint{gp mark 1}{(10.164,3.672)}
\gppoint{gp mark 1}{(10.167,3.689)}
\gppoint{gp mark 1}{(10.169,3.733)}
\gppoint{gp mark 1}{(10.171,3.733)}
\gppoint{gp mark 1}{(10.174,3.751)}
\gppoint{gp mark 1}{(10.176,3.837)}
\gppoint{gp mark 1}{(10.179,3.766)}
\gppoint{gp mark 1}{(10.181,3.727)}
\gppoint{gp mark 1}{(10.184,3.797)}
\gppoint{gp mark 1}{(10.186,3.646)}
\gppoint{gp mark 1}{(10.188,3.805)}
\gppoint{gp mark 1}{(10.191,3.670)}
\gppoint{gp mark 1}{(10.193,3.812)}
\gppoint{gp mark 1}{(10.196,3.852)}
\gppoint{gp mark 1}{(10.198,3.807)}
\gppoint{gp mark 1}{(10.200,3.831)}
\gppoint{gp mark 1}{(10.203,3.682)}
\gppoint{gp mark 1}{(10.205,3.751)}
\gppoint{gp mark 1}{(10.208,3.770)}
\gppoint{gp mark 1}{(10.210,3.749)}
\gppoint{gp mark 1}{(10.212,3.833)}
\gppoint{gp mark 1}{(10.215,3.748)}
\gppoint{gp mark 1}{(10.217,3.731)}
\gppoint{gp mark 1}{(10.220,3.854)}
\gppoint{gp mark 1}{(10.222,3.796)}
\gppoint{gp mark 1}{(10.225,3.682)}
\gppoint{gp mark 1}{(10.227,3.748)}
\gppoint{gp mark 1}{(10.229,3.684)}
\gppoint{gp mark 1}{(10.232,3.769)}
\gppoint{gp mark 1}{(10.234,3.884)}
\gppoint{gp mark 1}{(10.237,3.794)}
\gppoint{gp mark 1}{(10.239,3.761)}
\gppoint{gp mark 1}{(10.241,3.699)}
\gppoint{gp mark 1}{(10.244,3.785)}
\gppoint{gp mark 1}{(10.246,3.711)}
\gppoint{gp mark 1}{(10.249,3.734)}
\gppoint{gp mark 1}{(10.251,3.717)}
\gppoint{gp mark 1}{(10.253,3.726)}
\gppoint{gp mark 1}{(10.256,3.700)}
\gppoint{gp mark 1}{(10.258,3.709)}
\gppoint{gp mark 1}{(10.261,3.830)}
\gppoint{gp mark 1}{(10.263,3.795)}
\gppoint{gp mark 1}{(10.266,3.835)}
\gppoint{gp mark 1}{(10.268,3.759)}
\gppoint{gp mark 1}{(10.270,3.779)}
\gppoint{gp mark 1}{(10.273,3.704)}
\gppoint{gp mark 1}{(10.275,3.767)}
\gppoint{gp mark 1}{(10.278,3.786)}
\gppoint{gp mark 1}{(10.280,3.744)}
\gppoint{gp mark 1}{(10.282,3.770)}
\gppoint{gp mark 1}{(10.285,3.751)}
\gppoint{gp mark 1}{(10.287,3.729)}
\gppoint{gp mark 1}{(10.290,3.797)}
\gppoint{gp mark 1}{(10.292,3.820)}
\gppoint{gp mark 1}{(10.294,3.816)}
\gppoint{gp mark 1}{(10.297,3.659)}
\gppoint{gp mark 1}{(10.299,3.792)}
\gppoint{gp mark 1}{(10.302,3.689)}
\gppoint{gp mark 1}{(10.304,3.789)}
\gppoint{gp mark 1}{(10.307,3.685)}
\gppoint{gp mark 1}{(10.309,3.597)}
\gppoint{gp mark 1}{(10.311,3.699)}
\gppoint{gp mark 1}{(10.314,3.657)}
\gppoint{gp mark 1}{(10.316,3.624)}
\gppoint{gp mark 1}{(10.319,3.726)}
\gppoint{gp mark 1}{(10.321,3.744)}
\gppoint{gp mark 1}{(10.323,3.745)}
\gppoint{gp mark 1}{(10.326,3.614)}
\gppoint{gp mark 1}{(10.328,3.593)}
\gppoint{gp mark 1}{(10.331,3.750)}
\gppoint{gp mark 1}{(10.333,3.728)}
\gppoint{gp mark 1}{(10.335,3.649)}
\gppoint{gp mark 1}{(10.338,3.758)}
\gppoint{gp mark 1}{(10.340,3.767)}
\gppoint{gp mark 1}{(10.343,3.686)}
\gppoint{gp mark 1}{(10.345,3.763)}
\gppoint{gp mark 1}{(10.347,3.825)}
\gppoint{gp mark 1}{(10.350,3.666)}
\gppoint{gp mark 1}{(10.352,3.746)}
\gppoint{gp mark 1}{(10.355,3.619)}
\gppoint{gp mark 1}{(10.357,3.747)}
\gppoint{gp mark 1}{(10.360,3.654)}
\gppoint{gp mark 1}{(10.362,3.766)}
\gppoint{gp mark 1}{(10.364,3.802)}
\gppoint{gp mark 1}{(10.367,3.749)}
\gppoint{gp mark 1}{(10.369,3.699)}
\gppoint{gp mark 1}{(10.372,3.616)}
\gppoint{gp mark 1}{(10.374,3.637)}
\gppoint{gp mark 1}{(10.376,3.633)}
\gppoint{gp mark 1}{(10.379,3.601)}
\gppoint{gp mark 1}{(10.381,3.650)}
\gppoint{gp mark 1}{(10.384,3.813)}
\gppoint{gp mark 1}{(10.386,3.699)}
\gppoint{gp mark 1}{(10.388,3.739)}
\gppoint{gp mark 1}{(10.391,3.712)}
\gppoint{gp mark 1}{(10.393,3.777)}
\gppoint{gp mark 1}{(10.396,3.697)}
\gppoint{gp mark 1}{(10.398,3.691)}
\gppoint{gp mark 1}{(10.401,3.670)}
\gppoint{gp mark 1}{(10.403,3.777)}
\gppoint{gp mark 1}{(10.405,3.657)}
\gppoint{gp mark 1}{(10.408,3.709)}
\gppoint{gp mark 1}{(10.410,3.641)}
\gppoint{gp mark 1}{(10.413,3.666)}
\gppoint{gp mark 1}{(10.415,3.623)}
\gppoint{gp mark 1}{(10.417,3.829)}
\gppoint{gp mark 1}{(10.420,3.707)}
\gppoint{gp mark 1}{(10.422,3.663)}
\gppoint{gp mark 1}{(10.425,3.680)}
\gppoint{gp mark 1}{(10.427,3.742)}
\gppoint{gp mark 1}{(10.429,3.746)}
\gppoint{gp mark 1}{(10.432,3.704)}
\gppoint{gp mark 1}{(10.434,3.742)}
\gppoint{gp mark 1}{(10.437,3.734)}
\gppoint{gp mark 1}{(10.439,3.733)}
\gppoint{gp mark 1}{(10.442,3.711)}
\gppoint{gp mark 1}{(10.444,3.654)}
\gppoint{gp mark 1}{(10.446,3.713)}
\gppoint{gp mark 1}{(10.449,3.879)}
\gppoint{gp mark 1}{(10.451,3.656)}
\gppoint{gp mark 1}{(10.454,3.691)}
\gppoint{gp mark 1}{(10.456,3.675)}
\gppoint{gp mark 1}{(10.458,3.793)}
\gppoint{gp mark 1}{(10.461,3.717)}
\gppoint{gp mark 1}{(10.463,3.646)}
\gppoint{gp mark 1}{(10.466,3.740)}
\gppoint{gp mark 1}{(10.468,3.694)}
\gppoint{gp mark 1}{(10.470,3.699)}
\gppoint{gp mark 1}{(10.473,3.712)}
\gppoint{gp mark 1}{(10.475,3.628)}
\gppoint{gp mark 1}{(10.478,3.786)}
\gppoint{gp mark 1}{(10.480,3.825)}
\gppoint{gp mark 1}{(10.483,3.740)}
\gppoint{gp mark 1}{(10.485,3.707)}
\gppoint{gp mark 1}{(10.487,3.710)}
\gppoint{gp mark 1}{(10.490,3.748)}
\gppoint{gp mark 1}{(10.492,3.726)}
\gppoint{gp mark 1}{(10.495,3.795)}
\gppoint{gp mark 1}{(10.497,3.712)}
\gppoint{gp mark 1}{(10.499,3.677)}
\gppoint{gp mark 1}{(10.502,3.776)}
\gppoint{gp mark 1}{(10.504,3.745)}
\gppoint{gp mark 1}{(10.507,3.638)}
\gppoint{gp mark 1}{(10.509,3.719)}
\gppoint{gp mark 1}{(10.511,3.614)}
\gppoint{gp mark 1}{(10.514,3.748)}
\gppoint{gp mark 1}{(10.516,3.697)}
\gppoint{gp mark 1}{(10.519,3.723)}
\gppoint{gp mark 1}{(10.521,3.686)}
\gppoint{gp mark 1}{(10.524,3.637)}
\gppoint{gp mark 1}{(10.526,3.691)}
\gppoint{gp mark 1}{(10.528,3.747)}
\gppoint{gp mark 1}{(10.531,3.740)}
\gppoint{gp mark 1}{(10.533,3.690)}
\gppoint{gp mark 1}{(10.536,3.651)}
\gppoint{gp mark 1}{(10.538,3.712)}
\gppoint{gp mark 1}{(10.540,3.707)}
\gppoint{gp mark 1}{(10.543,3.739)}
\gppoint{gp mark 1}{(10.545,3.675)}
\gppoint{gp mark 1}{(10.548,3.699)}
\gppoint{gp mark 1}{(10.550,3.614)}
\gppoint{gp mark 1}{(10.552,3.709)}
\gppoint{gp mark 1}{(10.555,3.678)}
\gppoint{gp mark 1}{(10.557,3.676)}
\gppoint{gp mark 1}{(10.560,3.722)}
\gppoint{gp mark 1}{(10.562,3.703)}
\gppoint{gp mark 1}{(10.565,3.556)}
\gppoint{gp mark 1}{(10.567,3.539)}
\gppoint{gp mark 1}{(10.569,3.699)}
\gppoint{gp mark 1}{(10.572,3.723)}
\gppoint{gp mark 1}{(10.574,3.656)}
\gppoint{gp mark 1}{(10.577,3.616)}
\gppoint{gp mark 1}{(10.579,3.620)}
\gppoint{gp mark 1}{(10.581,3.700)}
\gppoint{gp mark 1}{(10.584,3.588)}
\gppoint{gp mark 1}{(10.586,3.576)}
\gppoint{gp mark 1}{(10.589,3.572)}
\gppoint{gp mark 1}{(10.591,3.629)}
\gppoint{gp mark 1}{(10.593,3.550)}
\gppoint{gp mark 1}{(10.596,3.533)}
\gppoint{gp mark 1}{(10.598,3.367)}
\gppoint{gp mark 1}{(10.601,3.163)}
\gppoint{gp mark 1}{(10.603,3.041)}
\gppoint{gp mark 1}{(10.605,2.914)}
\gppoint{gp mark 1}{(10.608,2.701)}
\gppoint{gp mark 1}{(10.610,2.514)}
\gppoint{gp mark 1}{(10.613,2.495)}
\gppoint{gp mark 1}{(10.615,2.425)}
\gppoint{gp mark 1}{(10.618,2.545)}
\gppoint{gp mark 1}{(10.620,2.494)}
\gppoint{gp mark 1}{(10.622,2.390)}
\gppoint{gp mark 1}{(10.625,2.547)}
\gppoint{gp mark 1}{(10.627,2.421)}
\gppoint{gp mark 1}{(10.630,2.493)}
\gppoint{gp mark 1}{(10.632,2.488)}
\gppoint{gp mark 1}{(10.634,2.313)}
\gppoint{gp mark 1}{(10.637,2.250)}
\gppoint{gp mark 1}{(10.639,2.389)}
\gppoint{gp mark 1}{(10.642,2.373)}
\gppoint{gp mark 1}{(10.644,2.382)}
\gppoint{gp mark 1}{(10.646,2.271)}
\gppoint{gp mark 1}{(10.649,2.321)}
\gppoint{gp mark 1}{(10.651,2.346)}
\gppoint{gp mark 1}{(10.654,2.333)}
\gppoint{gp mark 1}{(10.656,2.329)}
\gppoint{gp mark 1}{(10.659,2.294)}
\gppoint{gp mark 1}{(10.661,2.378)}
\gppoint{gp mark 1}{(10.663,2.403)}
\gppoint{gp mark 1}{(10.666,2.411)}
\gppoint{gp mark 1}{(10.668,2.401)}
\gppoint{gp mark 1}{(10.671,2.278)}
\gppoint{gp mark 1}{(10.673,2.304)}
\gppoint{gp mark 1}{(10.675,2.344)}
\gppoint{gp mark 1}{(10.678,2.242)}
\gppoint{gp mark 1}{(10.680,2.246)}
\gppoint{gp mark 1}{(10.683,2.228)}
\gppoint{gp mark 1}{(10.685,2.194)}
\gppoint{gp mark 1}{(10.687,2.323)}
\gppoint{gp mark 1}{(10.690,2.337)}
\gppoint{gp mark 1}{(10.692,2.286)}
\gppoint{gp mark 1}{(10.695,2.311)}
\gppoint{gp mark 1}{(10.697,2.347)}
\gppoint{gp mark 1}{(10.700,2.379)}
\gppoint{gp mark 1}{(10.702,2.155)}
\gppoint{gp mark 1}{(10.704,2.088)}
\gppoint{gp mark 1}{(10.707,2.197)}
\gppoint{gp mark 1}{(10.709,2.317)}
\gppoint{gp mark 1}{(10.712,2.230)}
\gppoint{gp mark 1}{(10.714,2.220)}
\gppoint{gp mark 1}{(10.716,2.169)}
\gppoint{gp mark 1}{(10.719,2.163)}
\gppoint{gp mark 1}{(10.721,2.230)}
\gppoint{gp mark 1}{(10.724,2.210)}
\gppoint{gp mark 1}{(10.726,2.090)}
\gppoint{gp mark 1}{(10.728,2.108)}
\gppoint{gp mark 1}{(10.731,2.218)}
\gppoint{gp mark 1}{(10.733,2.264)}
\gppoint{gp mark 1}{(10.736,2.189)}
\gppoint{gp mark 1}{(10.738,2.242)}
\gppoint{gp mark 1}{(10.741,2.186)}
\gppoint{gp mark 1}{(10.743,2.315)}
\gppoint{gp mark 1}{(10.745,2.300)}
\gppoint{gp mark 1}{(10.748,2.186)}
\gppoint{gp mark 1}{(10.750,2.265)}
\gppoint{gp mark 1}{(10.753,2.162)}
\gppoint{gp mark 1}{(10.755,2.310)}
\gppoint{gp mark 1}{(10.757,2.074)}
\gppoint{gp mark 1}{(10.760,2.227)}
\gppoint{gp mark 1}{(10.762,2.229)}
\gppoint{gp mark 1}{(10.765,2.279)}
\gppoint{gp mark 1}{(10.767,2.354)}
\gppoint{gp mark 1}{(10.769,2.274)}
\gppoint{gp mark 1}{(10.772,2.245)}
\gppoint{gp mark 1}{(10.774,2.260)}
\gppoint{gp mark 1}{(10.777,2.227)}
\gppoint{gp mark 1}{(10.779,2.162)}
\gppoint{gp mark 1}{(10.782,2.050)}
\gppoint{gp mark 1}{(10.784,2.225)}
\gppoint{gp mark 1}{(10.786,2.265)}
\gppoint{gp mark 1}{(10.789,2.214)}
\gppoint{gp mark 1}{(10.791,2.241)}
\gppoint{gp mark 1}{(10.794,2.249)}
\gppoint{gp mark 1}{(10.796,2.188)}
\gppoint{gp mark 1}{(10.798,2.272)}
\gppoint{gp mark 1}{(10.801,2.181)}
\gppoint{gp mark 1}{(10.803,2.206)}
\gppoint{gp mark 1}{(10.806,2.173)}
\gppoint{gp mark 1}{(10.808,2.300)}
\gppoint{gp mark 1}{(10.810,2.330)}
\gppoint{gp mark 1}{(10.813,2.204)}
\gppoint{gp mark 1}{(10.815,2.095)}
\gppoint{gp mark 1}{(10.818,2.317)}
\gppoint{gp mark 1}{(10.820,2.238)}
\gppoint{gp mark 1}{(10.823,2.245)}
\gppoint{gp mark 1}{(10.825,2.213)}
\gppoint{gp mark 1}{(10.827,2.090)}
\gppoint{gp mark 1}{(10.830,2.299)}
\gppoint{gp mark 1}{(10.832,2.148)}
\gppoint{gp mark 1}{(10.835,2.211)}
\gppoint{gp mark 1}{(10.837,2.252)}
\gppoint{gp mark 1}{(10.839,2.187)}
\gppoint{gp mark 1}{(10.842,2.180)}
\gppoint{gp mark 1}{(10.844,2.100)}
\gppoint{gp mark 1}{(10.847,2.231)}
\gppoint{gp mark 1}{(10.849,2.260)}
\gppoint{gp mark 1}{(10.851,2.178)}
\gppoint{gp mark 1}{(10.854,2.230)}
\gppoint{gp mark 1}{(10.856,2.246)}
\gppoint{gp mark 1}{(10.859,2.148)}
\gppoint{gp mark 1}{(10.861,2.195)}
\gppoint{gp mark 1}{(10.863,2.148)}
\gppoint{gp mark 1}{(10.866,2.144)}
\gppoint{gp mark 1}{(10.868,2.197)}
\gppoint{gp mark 1}{(10.871,2.215)}
\gppoint{gp mark 1}{(10.873,2.181)}
\gppoint{gp mark 1}{(10.876,2.248)}
\gppoint{gp mark 1}{(10.878,2.288)}
\gppoint{gp mark 1}{(10.880,2.156)}
\gppoint{gp mark 1}{(10.883,2.224)}
\gppoint{gp mark 1}{(10.885,2.260)}
\gppoint{gp mark 1}{(10.888,2.209)}
\gppoint{gp mark 1}{(10.890,2.203)}
\gppoint{gp mark 1}{(10.892,2.098)}
\gppoint{gp mark 1}{(10.895,2.181)}
\gppoint{gp mark 1}{(10.897,2.189)}
\gppoint{gp mark 1}{(10.900,2.171)}
\gppoint{gp mark 1}{(10.902,2.177)}
\gppoint{gp mark 1}{(10.904,2.105)}
\gppoint{gp mark 1}{(10.907,2.159)}
\gppoint{gp mark 1}{(10.909,2.216)}
\gppoint{gp mark 1}{(10.912,2.223)}
\gppoint{gp mark 1}{(10.914,2.060)}
\gppoint{gp mark 1}{(10.917,2.188)}
\gppoint{gp mark 1}{(10.919,2.169)}
\gppoint{gp mark 1}{(10.921,2.088)}
\gppoint{gp mark 1}{(10.924,2.167)}
\gppoint{gp mark 1}{(10.926,2.271)}
\gppoint{gp mark 1}{(10.929,2.118)}
\gppoint{gp mark 1}{(10.931,2.130)}
\gppoint{gp mark 1}{(10.933,2.268)}
\gppoint{gp mark 1}{(10.936,2.056)}
\gppoint{gp mark 1}{(10.938,2.157)}
\gppoint{gp mark 1}{(10.941,2.215)}
\gppoint{gp mark 1}{(10.943,2.187)}
\gppoint{gp mark 1}{(10.945,2.129)}
\gppoint{gp mark 1}{(10.948,2.192)}
\gppoint{gp mark 1}{(10.950,2.239)}
\gppoint{gp mark 1}{(10.953,2.158)}
\gppoint{gp mark 1}{(10.955,2.198)}
\gppoint{gp mark 1}{(10.958,2.167)}
\gppoint{gp mark 1}{(10.960,2.087)}
\gppoint{gp mark 1}{(10.962,2.202)}
\gppoint{gp mark 1}{(10.965,2.048)}
\gppoint{gp mark 1}{(10.967,2.203)}
\gppoint{gp mark 1}{(10.970,2.161)}
\gppoint{gp mark 1}{(10.972,2.215)}
\gppoint{gp mark 1}{(10.974,2.114)}
\gppoint{gp mark 1}{(10.977,2.129)}
\gppoint{gp mark 1}{(10.979,2.155)}
\gppoint{gp mark 1}{(10.982,2.271)}
\gppoint{gp mark 1}{(10.984,2.124)}
\gppoint{gp mark 1}{(10.986,2.191)}
\gppoint{gp mark 1}{(10.989,2.201)}
\gppoint{gp mark 1}{(10.991,2.127)}
\gppoint{gp mark 1}{(10.994,2.201)}
\gppoint{gp mark 1}{(10.996,2.150)}
\gppoint{gp mark 1}{(10.999,2.137)}
\gppoint{gp mark 1}{(11.001,2.102)}
\gppoint{gp mark 1}{(11.003,2.196)}
\gppoint{gp mark 1}{(11.006,2.218)}
\gppoint{gp mark 1}{(11.008,2.182)}
\gppoint{gp mark 1}{(11.011,2.181)}
\gppoint{gp mark 1}{(11.013,2.090)}
\gppoint{gp mark 1}{(11.015,2.253)}
\gppoint{gp mark 1}{(11.018,2.123)}
\gppoint{gp mark 1}{(11.020,2.168)}
\gppoint{gp mark 1}{(11.023,2.099)}
\gppoint{gp mark 1}{(11.025,2.134)}
\gppoint{gp mark 1}{(11.027,2.083)}
\gppoint{gp mark 1}{(11.030,2.166)}
\gppoint{gp mark 1}{(11.032,2.142)}
\gppoint{gp mark 1}{(11.035,2.153)}
\gppoint{gp mark 1}{(11.037,2.178)}
\gppoint{gp mark 1}{(11.040,2.137)}
\gppoint{gp mark 1}{(11.042,2.106)}
\gppoint{gp mark 1}{(11.044,2.267)}
\gppoint{gp mark 1}{(11.047,2.209)}
\gppoint{gp mark 1}{(11.049,2.282)}
\gppoint{gp mark 1}{(11.052,2.146)}
\gppoint{gp mark 1}{(11.054,2.098)}
\gppoint{gp mark 1}{(11.056,2.173)}
\gppoint{gp mark 1}{(11.059,2.249)}
\gppoint{gp mark 1}{(11.061,2.201)}
\gppoint{gp mark 1}{(11.064,2.187)}
\gppoint{gp mark 1}{(11.066,2.211)}
\gppoint{gp mark 1}{(11.068,2.264)}
\gppoint{gp mark 1}{(11.071,2.143)}
\gppoint{gp mark 1}{(11.073,2.172)}
\gppoint{gp mark 1}{(11.076,2.208)}
\gppoint{gp mark 1}{(11.078,2.178)}
\gppoint{gp mark 1}{(11.080,2.179)}
\gppoint{gp mark 1}{(11.083,2.106)}
\gppoint{gp mark 1}{(11.085,2.253)}
\gppoint{gp mark 1}{(11.088,2.105)}
\gppoint{gp mark 1}{(11.090,2.084)}
\gppoint{gp mark 1}{(11.093,2.103)}
\gppoint{gp mark 1}{(11.095,2.136)}
\gppoint{gp mark 1}{(11.097,2.133)}
\gppoint{gp mark 1}{(11.100,2.211)}
\gppoint{gp mark 1}{(11.102,2.171)}
\gppoint{gp mark 1}{(11.105,2.267)}
\gppoint{gp mark 1}{(11.107,2.197)}
\gppoint{gp mark 1}{(11.109,2.209)}
\gppoint{gp mark 1}{(11.112,2.156)}
\gppoint{gp mark 1}{(11.114,2.116)}
\gppoint{gp mark 1}{(11.117,2.047)}
\gppoint{gp mark 1}{(11.119,2.130)}
\gppoint{gp mark 1}{(11.121,2.195)}
\gppoint{gp mark 1}{(11.124,2.214)}
\gppoint{gp mark 1}{(11.126,2.200)}
\gppoint{gp mark 1}{(11.129,2.154)}
\gppoint{gp mark 1}{(11.131,2.139)}
\gppoint{gp mark 1}{(11.134,2.169)}
\gppoint{gp mark 1}{(11.136,2.255)}
\gppoint{gp mark 1}{(11.138,2.150)}
\gppoint{gp mark 1}{(11.141,2.189)}
\gppoint{gp mark 1}{(11.143,2.127)}
\gppoint{gp mark 1}{(11.146,2.189)}
\gppoint{gp mark 1}{(11.148,2.261)}
\gppoint{gp mark 1}{(11.150,2.230)}
\gppoint{gp mark 1}{(11.153,2.097)}
\gppoint{gp mark 1}{(11.155,2.059)}
\gppoint{gp mark 1}{(11.158,2.149)}
\gppoint{gp mark 1}{(11.160,2.088)}
\gppoint{gp mark 1}{(11.162,2.083)}
\gppoint{gp mark 1}{(11.165,2.121)}
\gppoint{gp mark 1}{(11.167,2.079)}
\gppoint{gp mark 1}{(11.170,2.055)}
\gppoint{gp mark 1}{(11.172,2.154)}
\gppoint{gp mark 1}{(11.175,2.113)}
\gppoint{gp mark 1}{(11.177,2.126)}
\gppoint{gp mark 1}{(11.179,2.267)}
\gppoint{gp mark 1}{(11.182,2.112)}
\gppoint{gp mark 1}{(11.184,2.156)}
\gppoint{gp mark 1}{(11.187,2.174)}
\gppoint{gp mark 1}{(11.189,2.115)}
\gppoint{gp mark 1}{(11.191,2.171)}
\gppoint{gp mark 1}{(11.194,2.079)}
\gppoint{gp mark 1}{(11.196,2.154)}
\gppoint{gp mark 1}{(11.199,2.199)}
\gppoint{gp mark 1}{(11.201,2.145)}
\gppoint{gp mark 1}{(11.203,2.131)}
\gppoint{gp mark 1}{(11.206,2.146)}
\gppoint{gp mark 1}{(11.208,2.127)}
\gppoint{gp mark 1}{(11.211,2.050)}
\gppoint{gp mark 1}{(11.213,2.208)}
\gppoint{gp mark 1}{(11.216,2.228)}
\gppoint{gp mark 1}{(11.218,2.130)}
\gppoint{gp mark 1}{(11.220,2.152)}
\gppoint{gp mark 1}{(11.223,2.159)}
\gppoint{gp mark 1}{(11.225,2.233)}
\gppoint{gp mark 1}{(11.228,2.285)}
\gppoint{gp mark 1}{(11.230,2.214)}
\gppoint{gp mark 1}{(11.232,2.148)}
\gppoint{gp mark 1}{(11.235,2.064)}
\gppoint{gp mark 1}{(11.237,2.161)}
\gppoint{gp mark 1}{(11.240,2.260)}
\gppoint{gp mark 1}{(11.242,2.223)}
\gppoint{gp mark 1}{(11.244,2.137)}
\gppoint{gp mark 1}{(11.247,2.107)}
\gppoint{gp mark 1}{(11.249,2.222)}
\gppoint{gp mark 1}{(11.252,2.164)}
\gppoint{gp mark 1}{(11.254,2.178)}
\gppoint{gp mark 1}{(11.257,2.155)}
\gppoint{gp mark 1}{(11.259,2.104)}
\gppoint{gp mark 1}{(11.261,2.157)}
\gppoint{gp mark 1}{(11.264,2.294)}
\gppoint{gp mark 1}{(11.266,2.126)}
\gppoint{gp mark 1}{(11.269,1.995)}
\gppoint{gp mark 1}{(11.271,2.181)}
\gppoint{gp mark 1}{(11.273,2.122)}
\gppoint{gp mark 1}{(11.276,2.213)}
\gppoint{gp mark 1}{(11.278,2.119)}
\gppoint{gp mark 1}{(11.281,2.181)}
\gppoint{gp mark 1}{(11.283,2.171)}
\gppoint{gp mark 1}{(11.285,2.127)}
\gppoint{gp mark 1}{(11.288,2.122)}
\gppoint{gp mark 1}{(11.290,2.037)}
\gppoint{gp mark 1}{(11.293,2.114)}
\gppoint{gp mark 1}{(11.295,2.172)}
\gppoint{gp mark 1}{(11.298,2.173)}
\gppoint{gp mark 1}{(11.300,2.241)}
\gppoint{gp mark 1}{(11.302,2.140)}
\gppoint{gp mark 1}{(11.305,2.118)}
\gppoint{gp mark 1}{(11.307,2.157)}
\gppoint{gp mark 1}{(11.310,2.108)}
\gppoint{gp mark 1}{(11.312,2.230)}
\gppoint{gp mark 1}{(11.314,2.068)}
\gppoint{gp mark 1}{(11.317,2.130)}
\gppoint{gp mark 1}{(11.319,2.171)}
\gppoint{gp mark 1}{(11.322,2.116)}
\gppoint{gp mark 1}{(11.324,2.084)}
\gppoint{gp mark 1}{(11.326,2.178)}
\gppoint{gp mark 1}{(11.329,2.000)}
\gppoint{gp mark 1}{(11.331,2.227)}
\gppoint{gp mark 1}{(11.334,2.120)}
\gppoint{gp mark 1}{(11.336,2.098)}
\gppoint{gp mark 1}{(11.338,2.114)}
\gppoint{gp mark 1}{(11.341,2.152)}
\gppoint{gp mark 1}{(11.343,2.178)}
\gppoint{gp mark 1}{(11.346,2.078)}
\draw[gp path] (1.320,7.691)--(1.320,0.985)--(11.447,0.985)--(11.447,7.691)--cycle;
%% coordinates of the plot area
\gpdefrectangularnode{gp plot 1}{\pgfpoint{1.320cm}{0.985cm}}{\pgfpoint{11.447cm}{7.691cm}}
\end{tikzpicture}
%% gnuplot variables

\caption{お湯の温度変化に伴う信号電圧の変化}
\label{fig:hot_water}
\end{center}
\end{figure}