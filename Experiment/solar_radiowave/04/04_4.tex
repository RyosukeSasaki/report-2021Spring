\subsection{考察}
今回観測した$12\si{\giga\hertz}$帯は彩層からの放射が支配的であり,
その温度は$4200$-$10000\si{\celsius}$程度である.\cite{saiso:online}
したがって今回得られた結果は文献値からするとオーダーは一致するが$120\%$程度高いことがわかる.
この原因として太陽活動の変化,人工の電波源による外乱,放射の影響などが考えられる.

まず太陽活動の変化について考える.太陽活動が活発なとき太陽電波の強度が大きくなることが知られている.\cite{naze:online}
また太陽活動は黒点数が多いときに極大になる.しかし国立天文台の太陽活動データベースによれば当日のデータは存在しないものの,
測定が行われた2019年4月22日付近の黒点相対数は少ないことがわかる.\cite{nao:online}
したがって当日の太陽活動が著しく大きかったため輝度温度が大きく現れたとは考えにくい.

次に人工の電波源による外乱について考える.今回観測に用いた$12\si{\giga\hertz}$帯は電波法によって保護されている.\cite{12ghz:online}
したがってアマチュア無線や移動体通信などでの外乱が起こることは考えにくい.
また衛星通信の場合$12\si{\giga\hertz}$帯はBS放送などで用いられているが,これを提供する衛星は一般に静止衛星なので定常的な雑音となることはあっても
時間変化は比較的少ないと考えられる.また,アンテナも指向性のものであるので人工の電波源による外乱は起きていないと考えられる.

まず放射の影響について,文献値に比べて測定値が高く(\ref{equ:TB=TB0+T})から
太陽から地球までの光路上に温度の高い物質があればその影響で算出される輝度温度が高くなると考えられる.
その候補として,彩層の外側に存在するコロナが考えられる.
仮にコロナにより輻射の輝度温度が高められ,大気による吸収が十分少ないと仮定した場合コロナの温度を$T=1.0\times10^6\si{\kelvin}$,彩層の温度を$T_B(0)=1.0\times10^4\si{\kelvin}$とすると
(\ref{equ:TB=TB0+T})から
\begin{align}
  2.2\times10^4&=1.0\times10^4{\rm e}^{-\tau_\nu}+1.0\times10^6(1-{\rm e}^{-\tau_\nu})\nonumber\\
  \therefore\ \tau_\nu&=0.00529
\end{align}
が得られ,これがコロナの光学的厚みであると考えられる.コロナは希薄なガスであり吸収係数$\alpha_\nu$も小さいと考えられるので,もっともらしいと考えられる.
\mfig[width=8cm]{fig/h-T.png}{太陽の温度構造(実験テキストより引用)}