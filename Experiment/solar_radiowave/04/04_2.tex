\subsection{方法}
室外機を直射日光が当たる場所に設置し,続いて気温を測定した.
次に赤道儀の極軸合わせを行った.
極軸の高度は慶應義塾大学の緯度である北緯$35.5559\si{\degree}$に$\pm0.5\si{\degree}$の範囲で事前に設定済みである.
その後コンパスを用いて極軸を北に向けた.
次に緯度軸と経度軸を調整し大まかにアンテナの方向を太陽に向けた.
そして一方の軸を固定した状態でもう一方の軸を動かし,受信機の出力が最大になるように調整した.
同様の手順で両方の軸をアンテナが太陽の方向を向くように調整した.
次に赤道儀の緯度軸を西方向に動かし,太陽の動きを先回りするようにした.
この状態で1時間程度測定を行った.

上記の測定が完了した後アンテナを電波吸収体で覆い,常温の黒体放射を測定した.