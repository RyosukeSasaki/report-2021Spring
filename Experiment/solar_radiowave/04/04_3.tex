\subsection{結果}
図\ref{fig:solar_radio_raw}に測定データを示す.図\ref{fig:solar_radio_raw}の網掛け(青)部は太陽電波,網掛け(赤)部は黒体放射の測定である.
それぞれの経過時間は$t=583$-$3620\ \si{\second}$, $t=3622$-$4003\ \si{\second}$である.また気温$T_R=297.7\si{\kelvin}$
\begin{figure}[hptb]
\begin{center}
\begin{tikzpicture}[gnuplot]
%% generated with GNUPLOT 5.2p8 (Lua 5.3; terminal rev. Nov 2018, script rev. 108)
%% 2021年05月14日 08時08分20秒
\path (0.000,0.000) rectangle (12.000,8.000);
\gpfill{rgb color={0.000,1.000,1.000},color=.!30} (2.632,0.985)--(9.467,0.985)--(9.467,7.690)--(2.632,7.690)--cycle;
\gpfill{rgb color={1.000,0.000,0.000},color=.!30} (9.471,0.985)--(10.329,0.985)--(10.329,7.690)--(9.471,7.690)--cycle;
\gpcolor{color=gp lt color border}
\gpsetlinetype{gp lt border}
\gpsetdashtype{gp dt solid}
\gpsetlinewidth{1.00}
\draw[gp path] (1.320,0.985)--(1.500,0.985);
\draw[gp path] (11.447,0.985)--(11.267,0.985);
\node[gp node right] at (1.136,0.985) {$1$};
\draw[gp path] (1.320,1.432)--(1.410,1.432);
\draw[gp path] (11.447,1.432)--(11.357,1.432);
\draw[gp path] (1.320,1.879)--(1.500,1.879);
\draw[gp path] (11.447,1.879)--(11.267,1.879);
\node[gp node right] at (1.136,1.879) {$1.2$};
\draw[gp path] (1.320,2.326)--(1.410,2.326);
\draw[gp path] (11.447,2.326)--(11.357,2.326);
\draw[gp path] (1.320,2.773)--(1.500,2.773);
\draw[gp path] (11.447,2.773)--(11.267,2.773);
\node[gp node right] at (1.136,2.773) {$1.4$};
\draw[gp path] (1.320,3.220)--(1.410,3.220);
\draw[gp path] (11.447,3.220)--(11.357,3.220);
\draw[gp path] (1.320,3.667)--(1.500,3.667);
\draw[gp path] (11.447,3.667)--(11.267,3.667);
\node[gp node right] at (1.136,3.667) {$1.6$};
\draw[gp path] (1.320,4.114)--(1.410,4.114);
\draw[gp path] (11.447,4.114)--(11.357,4.114);
\draw[gp path] (1.320,4.562)--(1.500,4.562);
\draw[gp path] (11.447,4.562)--(11.267,4.562);
\node[gp node right] at (1.136,4.562) {$1.8$};
\draw[gp path] (1.320,5.009)--(1.410,5.009);
\draw[gp path] (11.447,5.009)--(11.357,5.009);
\draw[gp path] (1.320,5.456)--(1.500,5.456);
\draw[gp path] (11.447,5.456)--(11.267,5.456);
\node[gp node right] at (1.136,5.456) {$2$};
\draw[gp path] (1.320,5.903)--(1.410,5.903);
\draw[gp path] (11.447,5.903)--(11.357,5.903);
\draw[gp path] (1.320,6.350)--(1.500,6.350);
\draw[gp path] (11.447,6.350)--(11.267,6.350);
\node[gp node right] at (1.136,6.350) {$2.2$};
\draw[gp path] (1.320,6.797)--(1.410,6.797);
\draw[gp path] (11.447,6.797)--(11.357,6.797);
\draw[gp path] (1.320,7.244)--(1.500,7.244);
\draw[gp path] (11.447,7.244)--(11.267,7.244);
\node[gp node right] at (1.136,7.244) {$2.4$};
\draw[gp path] (1.320,7.691)--(1.410,7.691);
\draw[gp path] (11.447,7.691)--(11.357,7.691);
\draw[gp path] (1.320,0.985)--(1.320,1.165);
\draw[gp path] (1.320,7.691)--(1.320,7.511);
\node[gp node center] at (1.320,0.677) {$0$};
\draw[gp path] (2.445,0.985)--(2.445,1.075);
\draw[gp path] (2.445,7.691)--(2.445,7.601);
\draw[gp path] (3.570,0.985)--(3.570,1.165);
\draw[gp path] (3.570,7.691)--(3.570,7.511);
\node[gp node center] at (3.570,0.677) {$1000$};
\draw[gp path] (4.696,0.985)--(4.696,1.075);
\draw[gp path] (4.696,7.691)--(4.696,7.601);
\draw[gp path] (5.821,0.985)--(5.821,1.165);
\draw[gp path] (5.821,7.691)--(5.821,7.511);
\node[gp node center] at (5.821,0.677) {$2000$};
\draw[gp path] (6.946,0.985)--(6.946,1.075);
\draw[gp path] (6.946,7.691)--(6.946,7.601);
\draw[gp path] (8.071,0.985)--(8.071,1.165);
\draw[gp path] (8.071,7.691)--(8.071,7.511);
\node[gp node center] at (8.071,0.677) {$3000$};
\draw[gp path] (9.197,0.985)--(9.197,1.075);
\draw[gp path] (9.197,7.691)--(9.197,7.601);
\draw[gp path] (10.322,0.985)--(10.322,1.165);
\draw[gp path] (10.322,7.691)--(10.322,7.511);
\node[gp node center] at (10.322,0.677) {$4000$};
\draw[gp path] (11.447,0.985)--(11.447,1.075);
\draw[gp path] (11.447,7.691)--(11.447,7.601);
\draw[gp path] (1.320,7.691)--(1.320,0.985)--(11.447,0.985)--(11.447,7.691)--cycle;
\node[gp node center,rotate=-270] at (0.292,4.338) {電圧 / $\si{\volt}$};
\node[gp node center] at (6.383,0.215) {経過時間$t$ / $\si{\second}$};
\gpsetpointsize{4.00}
\gppoint{gp mark 1}{(1.322,1.478)}
\gppoint{gp mark 1}{(1.325,3.330)}
\gppoint{gp mark 1}{(1.327,3.353)}
\gppoint{gp mark 1}{(1.329,3.360)}
\gppoint{gp mark 1}{(1.331,3.350)}
\gppoint{gp mark 1}{(1.334,2.732)}
\gppoint{gp mark 1}{(1.336,2.256)}
\gppoint{gp mark 1}{(1.338,2.643)}
\gppoint{gp mark 1}{(1.340,1.774)}
\gppoint{gp mark 1}{(1.343,4.487)}
\gppoint{gp mark 1}{(1.345,3.161)}
\gppoint{gp mark 1}{(1.347,4.018)}
\gppoint{gp mark 1}{(1.349,3.622)}
\gppoint{gp mark 1}{(1.352,3.318)}
\gppoint{gp mark 1}{(1.354,3.340)}
\gppoint{gp mark 1}{(1.356,3.155)}
\gppoint{gp mark 1}{(1.358,1.747)}
\gppoint{gp mark 1}{(1.361,1.741)}
\gppoint{gp mark 1}{(1.363,1.961)}
\gppoint{gp mark 1}{(1.365,3.051)}
\gppoint{gp mark 1}{(1.367,2.912)}
\gppoint{gp mark 1}{(1.370,3.522)}
\gppoint{gp mark 1}{(1.372,4.010)}
\gppoint{gp mark 1}{(1.374,4.767)}
\gppoint{gp mark 1}{(1.376,2.364)}
\gppoint{gp mark 1}{(1.379,3.603)}
\gppoint{gp mark 1}{(1.381,3.866)}
\gppoint{gp mark 1}{(1.383,3.844)}
\gppoint{gp mark 1}{(1.385,3.868)}
\gppoint{gp mark 1}{(1.388,3.840)}
\gppoint{gp mark 1}{(1.390,3.848)}
\gppoint{gp mark 1}{(1.392,3.852)}
\gppoint{gp mark 1}{(1.394,3.866)}
\gppoint{gp mark 1}{(1.397,3.854)}
\gppoint{gp mark 1}{(1.399,3.846)}
\gppoint{gp mark 1}{(1.401,3.856)}
\gppoint{gp mark 1}{(1.403,3.843)}
\gppoint{gp mark 1}{(1.406,3.856)}
\gppoint{gp mark 1}{(1.408,3.818)}
\gppoint{gp mark 1}{(1.410,3.830)}
\gppoint{gp mark 1}{(1.412,3.799)}
\gppoint{gp mark 1}{(1.415,3.795)}
\gppoint{gp mark 1}{(1.417,3.804)}
\gppoint{gp mark 1}{(1.419,3.792)}
\gppoint{gp mark 1}{(1.421,3.745)}
\gppoint{gp mark 1}{(1.424,3.785)}
\gppoint{gp mark 1}{(1.426,3.767)}
\gppoint{gp mark 1}{(1.428,3.759)}
\gppoint{gp mark 1}{(1.430,3.754)}
\gppoint{gp mark 1}{(1.433,3.756)}
\gppoint{gp mark 1}{(1.435,3.742)}
\gppoint{gp mark 1}{(1.437,3.744)}
\gppoint{gp mark 1}{(1.439,3.751)}
\gppoint{gp mark 1}{(1.442,3.714)}
\gppoint{gp mark 1}{(1.444,3.728)}
\gppoint{gp mark 1}{(1.446,3.695)}
\gppoint{gp mark 1}{(1.448,3.681)}
\gppoint{gp mark 1}{(1.451,3.690)}
\gppoint{gp mark 1}{(1.453,3.678)}
\gppoint{gp mark 1}{(1.455,3.655)}
\gppoint{gp mark 1}{(1.457,3.635)}
\gppoint{gp mark 1}{(1.460,3.624)}
\gppoint{gp mark 1}{(1.462,3.638)}
\gppoint{gp mark 1}{(1.464,3.648)}
\gppoint{gp mark 1}{(1.466,3.652)}
\gppoint{gp mark 1}{(1.469,3.598)}
\gppoint{gp mark 1}{(1.471,3.620)}
\gppoint{gp mark 1}{(1.473,3.616)}
\gppoint{gp mark 1}{(1.475,3.591)}
\gppoint{gp mark 1}{(1.478,3.153)}
\gppoint{gp mark 1}{(1.480,2.152)}
\gppoint{gp mark 1}{(1.482,1.812)}
\gppoint{gp mark 1}{(1.484,1.760)}
\gppoint{gp mark 1}{(1.487,1.737)}
\gppoint{gp mark 1}{(1.489,1.756)}
\gppoint{gp mark 1}{(1.491,1.823)}
\gppoint{gp mark 1}{(1.493,3.374)}
\gppoint{gp mark 1}{(1.496,4.328)}
\gppoint{gp mark 1}{(1.498,5.324)}
\gppoint{gp mark 1}{(1.500,5.492)}
\gppoint{gp mark 1}{(1.502,5.703)}
\gppoint{gp mark 1}{(1.505,5.757)}
\gppoint{gp mark 1}{(1.507,5.762)}
\gppoint{gp mark 1}{(1.509,5.671)}
\gppoint{gp mark 1}{(1.511,5.475)}
\gppoint{gp mark 1}{(1.514,5.272)}
\gppoint{gp mark 1}{(1.516,5.197)}
\gppoint{gp mark 1}{(1.518,5.233)}
\gppoint{gp mark 1}{(1.520,5.638)}
\gppoint{gp mark 1}{(1.523,5.666)}
\gppoint{gp mark 1}{(1.525,5.682)}
\gppoint{gp mark 1}{(1.527,5.733)}
\gppoint{gp mark 1}{(1.529,5.747)}
\gppoint{gp mark 1}{(1.532,5.781)}
\gppoint{gp mark 1}{(1.534,5.779)}
\gppoint{gp mark 1}{(1.536,5.740)}
\gppoint{gp mark 1}{(1.538,5.771)}
\gppoint{gp mark 1}{(1.541,5.749)}
\gppoint{gp mark 1}{(1.543,5.771)}
\gppoint{gp mark 1}{(1.545,5.710)}
\gppoint{gp mark 1}{(1.547,5.725)}
\gppoint{gp mark 1}{(1.550,5.698)}
\gppoint{gp mark 1}{(1.552,5.676)}
\gppoint{gp mark 1}{(1.554,5.649)}
\gppoint{gp mark 1}{(1.556,5.655)}
\gppoint{gp mark 1}{(1.559,5.581)}
\gppoint{gp mark 1}{(1.561,5.609)}
\gppoint{gp mark 1}{(1.563,5.594)}
\gppoint{gp mark 1}{(1.565,5.768)}
\gppoint{gp mark 1}{(1.568,5.745)}
\gppoint{gp mark 1}{(1.570,5.742)}
\gppoint{gp mark 1}{(1.572,5.715)}
\gppoint{gp mark 1}{(1.574,5.745)}
\gppoint{gp mark 1}{(1.577,5.720)}
\gppoint{gp mark 1}{(1.579,5.738)}
\gppoint{gp mark 1}{(1.581,5.779)}
\gppoint{gp mark 1}{(1.583,5.758)}
\gppoint{gp mark 1}{(1.586,5.757)}
\gppoint{gp mark 1}{(1.588,5.746)}
\gppoint{gp mark 1}{(1.590,5.738)}
\gppoint{gp mark 1}{(1.592,5.699)}
\gppoint{gp mark 1}{(1.595,5.685)}
\gppoint{gp mark 1}{(1.597,5.653)}
\gppoint{gp mark 1}{(1.599,5.662)}
\gppoint{gp mark 1}{(1.601,5.627)}
\gppoint{gp mark 1}{(1.604,5.647)}
\gppoint{gp mark 1}{(1.606,5.691)}
\gppoint{gp mark 1}{(1.608,5.701)}
\gppoint{gp mark 1}{(1.610,5.686)}
\gppoint{gp mark 1}{(1.613,5.670)}
\gppoint{gp mark 1}{(1.615,5.682)}
\gppoint{gp mark 1}{(1.617,2.557)}
\gppoint{gp mark 1}{(1.619,3.456)}
\gppoint{gp mark 1}{(1.622,2.686)}
\gppoint{gp mark 1}{(1.624,1.830)}
\gppoint{gp mark 1}{(1.626,1.839)}
\gppoint{gp mark 1}{(1.628,1.927)}
\gppoint{gp mark 1}{(1.631,1.712)}
\gppoint{gp mark 1}{(1.633,1.677)}
\gppoint{gp mark 1}{(1.635,1.671)}
\gppoint{gp mark 1}{(1.637,1.903)}
\gppoint{gp mark 1}{(1.640,2.616)}
\gppoint{gp mark 1}{(1.642,3.818)}
\gppoint{gp mark 1}{(1.644,4.557)}
\gppoint{gp mark 1}{(1.646,5.087)}
\gppoint{gp mark 1}{(1.649,5.320)}
\gppoint{gp mark 1}{(1.651,5.624)}
\gppoint{gp mark 1}{(1.653,5.739)}
\gppoint{gp mark 1}{(1.655,5.669)}
\gppoint{gp mark 1}{(1.658,5.483)}
\gppoint{gp mark 1}{(1.660,5.312)}
\gppoint{gp mark 1}{(1.662,5.113)}
\gppoint{gp mark 1}{(1.664,4.717)}
\gppoint{gp mark 1}{(1.667,4.348)}
\gppoint{gp mark 1}{(1.669,4.089)}
\gppoint{gp mark 1}{(1.671,3.826)}
\gppoint{gp mark 1}{(1.673,3.099)}
\gppoint{gp mark 1}{(1.676,2.491)}
\gppoint{gp mark 1}{(1.678,2.230)}
\gppoint{gp mark 1}{(1.680,1.839)}
\gppoint{gp mark 1}{(1.682,1.670)}
\gppoint{gp mark 1}{(1.685,1.637)}
\gppoint{gp mark 1}{(1.687,1.654)}
\gppoint{gp mark 1}{(1.689,1.648)}
\gppoint{gp mark 1}{(1.691,1.646)}
\gppoint{gp mark 1}{(1.694,1.638)}
\gppoint{gp mark 1}{(1.696,1.634)}
\gppoint{gp mark 1}{(1.698,1.634)}
\gppoint{gp mark 1}{(1.700,1.643)}
\gppoint{gp mark 1}{(1.703,1.656)}
\gppoint{gp mark 1}{(1.705,1.645)}
\gppoint{gp mark 1}{(1.707,1.634)}
\gppoint{gp mark 1}{(1.709,1.695)}
\gppoint{gp mark 1}{(1.712,1.811)}
\gppoint{gp mark 1}{(1.714,1.920)}
\gppoint{gp mark 1}{(1.716,2.177)}
\gppoint{gp mark 1}{(1.718,2.333)}
\gppoint{gp mark 1}{(1.721,2.383)}
\gppoint{gp mark 1}{(1.723,2.477)}
\gppoint{gp mark 1}{(1.725,2.725)}
\gppoint{gp mark 1}{(1.727,3.144)}
\gppoint{gp mark 1}{(1.730,3.589)}
\gppoint{gp mark 1}{(1.732,3.971)}
\gppoint{gp mark 1}{(1.734,4.258)}
\gppoint{gp mark 1}{(1.736,4.529)}
\gppoint{gp mark 1}{(1.739,4.773)}
\gppoint{gp mark 1}{(1.741,5.181)}
\gppoint{gp mark 1}{(1.743,5.346)}
\gppoint{gp mark 1}{(1.745,5.530)}
\gppoint{gp mark 1}{(1.748,5.640)}
\gppoint{gp mark 1}{(1.750,5.623)}
\gppoint{gp mark 1}{(1.752,5.599)}
\gppoint{gp mark 1}{(1.754,5.580)}
\gppoint{gp mark 1}{(1.757,5.620)}
\gppoint{gp mark 1}{(1.759,5.610)}
\gppoint{gp mark 1}{(1.761,5.488)}
\gppoint{gp mark 1}{(1.763,5.337)}
\gppoint{gp mark 1}{(1.766,5.280)}
\gppoint{gp mark 1}{(1.768,5.246)}
\gppoint{gp mark 1}{(1.770,5.259)}
\gppoint{gp mark 1}{(1.772,5.348)}
\gppoint{gp mark 1}{(1.775,5.448)}
\gppoint{gp mark 1}{(1.777,5.512)}
\gppoint{gp mark 1}{(1.779,5.610)}
\gppoint{gp mark 1}{(1.781,5.582)}
\gppoint{gp mark 1}{(1.784,5.611)}
\gppoint{gp mark 1}{(1.786,5.566)}
\gppoint{gp mark 1}{(1.788,5.539)}
\gppoint{gp mark 1}{(1.790,5.555)}
\gppoint{gp mark 1}{(1.793,5.546)}
\gppoint{gp mark 1}{(1.795,5.454)}
\gppoint{gp mark 1}{(1.797,5.400)}
\gppoint{gp mark 1}{(1.799,5.204)}
\gppoint{gp mark 1}{(1.802,5.227)}
\gppoint{gp mark 1}{(1.804,5.269)}
\gppoint{gp mark 1}{(1.806,5.453)}
\gppoint{gp mark 1}{(1.808,5.466)}
\gppoint{gp mark 1}{(1.811,5.570)}
\gppoint{gp mark 1}{(1.813,5.554)}
\gppoint{gp mark 1}{(1.815,5.551)}
\gppoint{gp mark 1}{(1.817,5.582)}
\gppoint{gp mark 1}{(1.820,5.543)}
\gppoint{gp mark 1}{(1.822,5.566)}
\gppoint{gp mark 1}{(1.824,5.526)}
\gppoint{gp mark 1}{(1.826,5.269)}
\gppoint{gp mark 1}{(1.829,5.428)}
\gppoint{gp mark 1}{(1.831,5.596)}
\gppoint{gp mark 1}{(1.833,5.655)}
\gppoint{gp mark 1}{(1.835,5.502)}
\gppoint{gp mark 1}{(1.838,5.523)}
\gppoint{gp mark 1}{(1.840,5.531)}
\gppoint{gp mark 1}{(1.842,5.103)}
\gppoint{gp mark 1}{(1.844,4.861)}
\gppoint{gp mark 1}{(1.847,4.963)}
\gppoint{gp mark 1}{(1.849,3.602)}
\gppoint{gp mark 1}{(1.851,2.612)}
\gppoint{gp mark 1}{(1.853,2.626)}
\gppoint{gp mark 1}{(1.856,2.604)}
\gppoint{gp mark 1}{(1.858,2.598)}
\gppoint{gp mark 1}{(1.860,2.578)}
\gppoint{gp mark 1}{(1.862,2.572)}
\gppoint{gp mark 1}{(1.865,2.592)}
\gppoint{gp mark 1}{(1.867,2.596)}
\gppoint{gp mark 1}{(1.869,2.569)}
\gppoint{gp mark 1}{(1.871,2.556)}
\gppoint{gp mark 1}{(1.874,2.651)}
\gppoint{gp mark 1}{(1.876,2.576)}
\gppoint{gp mark 1}{(1.878,2.541)}
\gppoint{gp mark 1}{(1.880,1.699)}
\gppoint{gp mark 1}{(1.883,3.485)}
\gppoint{gp mark 1}{(1.885,4.835)}
\gppoint{gp mark 1}{(1.887,4.746)}
\gppoint{gp mark 1}{(1.889,4.313)}
\gppoint{gp mark 1}{(1.892,4.903)}
\gppoint{gp mark 1}{(1.894,4.927)}
\gppoint{gp mark 1}{(1.896,4.906)}
\gppoint{gp mark 1}{(1.898,4.915)}
\gppoint{gp mark 1}{(1.901,4.735)}
\gppoint{gp mark 1}{(1.903,2.669)}
\gppoint{gp mark 1}{(1.905,4.382)}
\gppoint{gp mark 1}{(1.907,4.561)}
\gppoint{gp mark 1}{(1.910,4.533)}
\gppoint{gp mark 1}{(1.912,4.569)}
\gppoint{gp mark 1}{(1.914,4.533)}
\gppoint{gp mark 1}{(1.916,4.621)}
\gppoint{gp mark 1}{(1.919,4.731)}
\gppoint{gp mark 1}{(1.921,4.798)}
\gppoint{gp mark 1}{(1.923,4.926)}
\gppoint{gp mark 1}{(1.925,4.838)}
\gppoint{gp mark 1}{(1.928,4.876)}
\gppoint{gp mark 1}{(1.930,4.685)}
\gppoint{gp mark 1}{(1.932,4.412)}
\gppoint{gp mark 1}{(1.934,4.309)}
\gppoint{gp mark 1}{(1.937,4.326)}
\gppoint{gp mark 1}{(1.939,4.664)}
\gppoint{gp mark 1}{(1.941,4.652)}
\gppoint{gp mark 1}{(1.943,4.628)}
\gppoint{gp mark 1}{(1.946,4.635)}
\gppoint{gp mark 1}{(1.948,4.577)}
\gppoint{gp mark 1}{(1.950,4.678)}
\gppoint{gp mark 1}{(1.952,4.738)}
\gppoint{gp mark 1}{(1.955,4.697)}
\gppoint{gp mark 1}{(1.957,4.521)}
\gppoint{gp mark 1}{(1.959,4.848)}
\gppoint{gp mark 1}{(1.961,5.210)}
\gppoint{gp mark 1}{(1.964,5.222)}
\gppoint{gp mark 1}{(1.966,5.207)}
\gppoint{gp mark 1}{(1.968,5.191)}
\gppoint{gp mark 1}{(1.970,5.169)}
\gppoint{gp mark 1}{(1.973,5.179)}
\gppoint{gp mark 1}{(1.975,5.197)}
\gppoint{gp mark 1}{(1.977,5.195)}
\gppoint{gp mark 1}{(1.979,5.199)}
\gppoint{gp mark 1}{(1.982,5.158)}
\gppoint{gp mark 1}{(1.984,5.160)}
\gppoint{gp mark 1}{(1.986,5.135)}
\gppoint{gp mark 1}{(1.988,5.167)}
\gppoint{gp mark 1}{(1.991,5.163)}
\gppoint{gp mark 1}{(1.993,5.126)}
\gppoint{gp mark 1}{(1.995,5.137)}
\gppoint{gp mark 1}{(1.997,5.065)}
\gppoint{gp mark 1}{(2.000,3.087)}
\gppoint{gp mark 1}{(2.002,2.456)}
\gppoint{gp mark 1}{(2.004,3.411)}
\gppoint{gp mark 1}{(2.006,4.182)}
\gppoint{gp mark 1}{(2.009,4.869)}
\gppoint{gp mark 1}{(2.011,4.895)}
\gppoint{gp mark 1}{(2.013,4.868)}
\gppoint{gp mark 1}{(2.015,4.876)}
\gppoint{gp mark 1}{(2.018,4.877)}
\gppoint{gp mark 1}{(2.020,4.858)}
\gppoint{gp mark 1}{(2.022,5.018)}
\gppoint{gp mark 1}{(2.024,5.046)}
\gppoint{gp mark 1}{(2.027,5.027)}
\gppoint{gp mark 1}{(2.029,5.080)}
\gppoint{gp mark 1}{(2.031,5.099)}
\gppoint{gp mark 1}{(2.033,5.098)}
\gppoint{gp mark 1}{(2.036,5.061)}
\gppoint{gp mark 1}{(2.038,5.083)}
\gppoint{gp mark 1}{(2.040,5.071)}
\gppoint{gp mark 1}{(2.042,4.841)}
\gppoint{gp mark 1}{(2.045,4.529)}
\gppoint{gp mark 1}{(2.047,4.559)}
\gppoint{gp mark 1}{(2.049,4.816)}
\gppoint{gp mark 1}{(2.051,4.958)}
\gppoint{gp mark 1}{(2.054,5.028)}
\gppoint{gp mark 1}{(2.056,5.022)}
\gppoint{gp mark 1}{(2.058,5.005)}
\gppoint{gp mark 1}{(2.060,4.957)}
\gppoint{gp mark 1}{(2.063,4.937)}
\gppoint{gp mark 1}{(2.065,4.914)}
\gppoint{gp mark 1}{(2.067,4.934)}
\gppoint{gp mark 1}{(2.069,4.934)}
\gppoint{gp mark 1}{(2.072,4.886)}
\gppoint{gp mark 1}{(2.074,4.392)}
\gppoint{gp mark 1}{(2.076,4.374)}
\gppoint{gp mark 1}{(2.078,4.565)}
\gppoint{gp mark 1}{(2.081,4.940)}
\gppoint{gp mark 1}{(2.083,4.945)}
\gppoint{gp mark 1}{(2.085,4.939)}
\gppoint{gp mark 1}{(2.087,4.932)}
\gppoint{gp mark 1}{(2.090,4.902)}
\gppoint{gp mark 1}{(2.092,4.909)}
\gppoint{gp mark 1}{(2.094,4.920)}
\gppoint{gp mark 1}{(2.096,4.860)}
\gppoint{gp mark 1}{(2.099,4.864)}
\gppoint{gp mark 1}{(2.101,4.842)}
\gppoint{gp mark 1}{(2.103,4.843)}
\gppoint{gp mark 1}{(2.105,4.864)}
\gppoint{gp mark 1}{(2.108,4.856)}
\gppoint{gp mark 1}{(2.110,4.877)}
\gppoint{gp mark 1}{(2.112,4.841)}
\gppoint{gp mark 1}{(2.114,4.856)}
\gppoint{gp mark 1}{(2.117,4.855)}
\gppoint{gp mark 1}{(2.119,4.865)}
\gppoint{gp mark 1}{(2.121,4.887)}
\gppoint{gp mark 1}{(2.123,4.906)}
\gppoint{gp mark 1}{(2.126,4.883)}
\gppoint{gp mark 1}{(2.128,4.886)}
\gppoint{gp mark 1}{(2.130,4.865)}
\gppoint{gp mark 1}{(2.132,4.842)}
\gppoint{gp mark 1}{(2.135,4.837)}
\gppoint{gp mark 1}{(2.137,4.839)}
\gppoint{gp mark 1}{(2.139,4.832)}
\gppoint{gp mark 1}{(2.141,4.862)}
\gppoint{gp mark 1}{(2.144,4.853)}
\gppoint{gp mark 1}{(2.146,4.828)}
\gppoint{gp mark 1}{(2.148,4.824)}
\gppoint{gp mark 1}{(2.150,4.835)}
\gppoint{gp mark 1}{(2.153,5.020)}
\gppoint{gp mark 1}{(2.155,5.239)}
\gppoint{gp mark 1}{(2.157,5.235)}
\gppoint{gp mark 1}{(2.159,5.360)}
\gppoint{gp mark 1}{(2.162,5.446)}
\gppoint{gp mark 1}{(2.164,5.572)}
\gppoint{gp mark 1}{(2.166,5.557)}
\gppoint{gp mark 1}{(2.168,5.545)}
\gppoint{gp mark 1}{(2.171,4.986)}
\gppoint{gp mark 1}{(2.173,4.481)}
\gppoint{gp mark 1}{(2.175,4.492)}
\gppoint{gp mark 1}{(2.177,4.500)}
\gppoint{gp mark 1}{(2.180,4.515)}
\gppoint{gp mark 1}{(2.182,4.455)}
\gppoint{gp mark 1}{(2.184,4.575)}
\gppoint{gp mark 1}{(2.186,5.150)}
\gppoint{gp mark 1}{(2.189,5.488)}
\gppoint{gp mark 1}{(2.191,5.579)}
\gppoint{gp mark 1}{(2.193,5.589)}
\gppoint{gp mark 1}{(2.195,5.589)}
\gppoint{gp mark 1}{(2.198,5.548)}
\gppoint{gp mark 1}{(2.200,5.508)}
\gppoint{gp mark 1}{(2.202,5.486)}
\gppoint{gp mark 1}{(2.204,5.474)}
\gppoint{gp mark 1}{(2.207,5.528)}
\gppoint{gp mark 1}{(2.209,5.564)}
\gppoint{gp mark 1}{(2.211,5.551)}
\gppoint{gp mark 1}{(2.213,5.566)}
\gppoint{gp mark 1}{(2.216,5.573)}
\gppoint{gp mark 1}{(2.218,5.566)}
\gppoint{gp mark 1}{(2.220,5.533)}
\gppoint{gp mark 1}{(2.222,5.582)}
\gppoint{gp mark 1}{(2.225,5.573)}
\gppoint{gp mark 1}{(2.227,5.583)}
\gppoint{gp mark 1}{(2.229,5.590)}
\gppoint{gp mark 1}{(2.231,5.611)}
\gppoint{gp mark 1}{(2.234,5.545)}
\gppoint{gp mark 1}{(2.236,5.540)}
\gppoint{gp mark 1}{(2.238,5.525)}
\gppoint{gp mark 1}{(2.240,5.501)}
\gppoint{gp mark 1}{(2.243,5.463)}
\gppoint{gp mark 1}{(2.245,5.453)}
\gppoint{gp mark 1}{(2.247,5.449)}
\gppoint{gp mark 1}{(2.249,5.444)}
\gppoint{gp mark 1}{(2.252,5.531)}
\gppoint{gp mark 1}{(2.254,5.583)}
\gppoint{gp mark 1}{(2.256,5.552)}
\gppoint{gp mark 1}{(2.258,5.535)}
\gppoint{gp mark 1}{(2.261,5.559)}
\gppoint{gp mark 1}{(2.263,5.578)}
\gppoint{gp mark 1}{(2.265,5.604)}
\gppoint{gp mark 1}{(2.267,5.565)}
\gppoint{gp mark 1}{(2.270,5.552)}
\gppoint{gp mark 1}{(2.272,5.564)}
\gppoint{gp mark 1}{(2.274,5.540)}
\gppoint{gp mark 1}{(2.276,5.567)}
\gppoint{gp mark 1}{(2.279,5.560)}
\gppoint{gp mark 1}{(2.281,5.607)}
\gppoint{gp mark 1}{(2.283,5.580)}
\gppoint{gp mark 1}{(2.285,5.550)}
\gppoint{gp mark 1}{(2.288,5.585)}
\gppoint{gp mark 1}{(2.290,5.603)}
\gppoint{gp mark 1}{(2.292,5.594)}
\gppoint{gp mark 1}{(2.294,5.611)}
\gppoint{gp mark 1}{(2.297,5.571)}
\gppoint{gp mark 1}{(2.299,5.557)}
\gppoint{gp mark 1}{(2.301,5.541)}
\gppoint{gp mark 1}{(2.303,5.547)}
\gppoint{gp mark 1}{(2.306,5.542)}
\gppoint{gp mark 1}{(2.308,5.545)}
\gppoint{gp mark 1}{(2.310,5.526)}
\gppoint{gp mark 1}{(2.312,5.564)}
\gppoint{gp mark 1}{(2.315,5.567)}
\gppoint{gp mark 1}{(2.317,5.552)}
\gppoint{gp mark 1}{(2.319,5.568)}
\gppoint{gp mark 1}{(2.321,5.569)}
\gppoint{gp mark 1}{(2.324,5.543)}
\gppoint{gp mark 1}{(2.326,5.593)}
\gppoint{gp mark 1}{(2.328,5.525)}
\gppoint{gp mark 1}{(2.330,5.545)}
\gppoint{gp mark 1}{(2.333,5.535)}
\gppoint{gp mark 1}{(2.335,5.523)}
\gppoint{gp mark 1}{(2.337,5.552)}
\gppoint{gp mark 1}{(2.339,5.535)}
\gppoint{gp mark 1}{(2.342,5.519)}
\gppoint{gp mark 1}{(2.344,5.549)}
\gppoint{gp mark 1}{(2.346,5.542)}
\gppoint{gp mark 1}{(2.348,5.554)}
\gppoint{gp mark 1}{(2.351,5.540)}
\gppoint{gp mark 1}{(2.353,5.564)}
\gppoint{gp mark 1}{(2.355,5.571)}
\gppoint{gp mark 1}{(2.357,5.534)}
\gppoint{gp mark 1}{(2.360,5.522)}
\gppoint{gp mark 1}{(2.362,5.550)}
\gppoint{gp mark 1}{(2.364,5.481)}
\gppoint{gp mark 1}{(2.366,5.479)}
\gppoint{gp mark 1}{(2.369,5.490)}
\gppoint{gp mark 1}{(2.371,5.491)}
\gppoint{gp mark 1}{(2.373,5.511)}
\gppoint{gp mark 1}{(2.375,5.536)}
\gppoint{gp mark 1}{(2.378,5.514)}
\gppoint{gp mark 1}{(2.380,5.523)}
\gppoint{gp mark 1}{(2.382,5.538)}
\gppoint{gp mark 1}{(2.384,5.518)}
\gppoint{gp mark 1}{(2.387,5.481)}
\gppoint{gp mark 1}{(2.389,5.520)}
\gppoint{gp mark 1}{(2.391,5.517)}
\gppoint{gp mark 1}{(2.393,5.513)}
\gppoint{gp mark 1}{(2.396,5.520)}
\gppoint{gp mark 1}{(2.398,5.499)}
\gppoint{gp mark 1}{(2.400,5.518)}
\gppoint{gp mark 1}{(2.402,5.537)}
\gppoint{gp mark 1}{(2.405,5.525)}
\gppoint{gp mark 1}{(2.407,5.502)}
\gppoint{gp mark 1}{(2.409,5.520)}
\gppoint{gp mark 1}{(2.411,5.520)}
\gppoint{gp mark 1}{(2.414,5.507)}
\gppoint{gp mark 1}{(2.416,5.485)}
\gppoint{gp mark 1}{(2.418,5.475)}
\gppoint{gp mark 1}{(2.420,5.512)}
\gppoint{gp mark 1}{(2.423,5.503)}
\gppoint{gp mark 1}{(2.425,5.512)}
\gppoint{gp mark 1}{(2.427,5.452)}
\gppoint{gp mark 1}{(2.429,5.476)}
\gppoint{gp mark 1}{(2.432,5.483)}
\gppoint{gp mark 1}{(2.434,5.474)}
\gppoint{gp mark 1}{(2.436,5.476)}
\gppoint{gp mark 1}{(2.438,5.452)}
\gppoint{gp mark 1}{(2.441,5.450)}
\gppoint{gp mark 1}{(2.443,5.522)}
\gppoint{gp mark 1}{(2.445,5.463)}
\gppoint{gp mark 1}{(2.447,5.475)}
\gppoint{gp mark 1}{(2.450,5.489)}
\gppoint{gp mark 1}{(2.452,5.457)}
\gppoint{gp mark 1}{(2.454,5.466)}
\gppoint{gp mark 1}{(2.456,5.431)}
\gppoint{gp mark 1}{(2.459,5.427)}
\gppoint{gp mark 1}{(2.461,5.426)}
\gppoint{gp mark 1}{(2.463,5.410)}
\gppoint{gp mark 1}{(2.465,5.392)}
\gppoint{gp mark 1}{(2.468,5.428)}
\gppoint{gp mark 1}{(2.470,5.400)}
\gppoint{gp mark 1}{(2.472,5.432)}
\gppoint{gp mark 1}{(2.474,5.443)}
\gppoint{gp mark 1}{(2.477,5.437)}
\gppoint{gp mark 1}{(2.479,5.450)}
\gppoint{gp mark 1}{(2.481,5.430)}
\gppoint{gp mark 1}{(2.483,2.977)}
\gppoint{gp mark 1}{(2.486,2.540)}
\gppoint{gp mark 1}{(2.488,4.763)}
\gppoint{gp mark 1}{(2.490,5.258)}
\gppoint{gp mark 1}{(2.492,5.278)}
\gppoint{gp mark 1}{(2.495,5.094)}
\gppoint{gp mark 1}{(2.497,1.717)}
\gppoint{gp mark 1}{(2.499,1.450)}
\gppoint{gp mark 1}{(2.501,1.458)}
\gppoint{gp mark 1}{(2.504,1.470)}
\gppoint{gp mark 1}{(2.506,1.456)}
\gppoint{gp mark 1}{(2.508,1.454)}
\gppoint{gp mark 1}{(2.510,1.455)}
\gppoint{gp mark 1}{(2.513,1.413)}
\gppoint{gp mark 1}{(2.515,1.418)}
\gppoint{gp mark 1}{(2.517,1.447)}
\gppoint{gp mark 1}{(2.519,1.442)}
\gppoint{gp mark 1}{(2.522,1.456)}
\gppoint{gp mark 1}{(2.524,1.454)}
\gppoint{gp mark 1}{(2.526,1.458)}
\gppoint{gp mark 1}{(2.528,1.435)}
\gppoint{gp mark 1}{(2.531,1.431)}
\gppoint{gp mark 1}{(2.533,1.434)}
\gppoint{gp mark 1}{(2.535,1.439)}
\gppoint{gp mark 1}{(2.537,1.433)}
\gppoint{gp mark 1}{(2.540,1.422)}
\gppoint{gp mark 1}{(2.542,1.440)}
\gppoint{gp mark 1}{(2.544,1.429)}
\gppoint{gp mark 1}{(2.546,1.408)}
\gppoint{gp mark 1}{(2.549,1.417)}
\gppoint{gp mark 1}{(2.551,1.420)}
\gppoint{gp mark 1}{(2.553,1.436)}
\gppoint{gp mark 1}{(2.555,1.448)}
\gppoint{gp mark 1}{(2.558,1.453)}
\gppoint{gp mark 1}{(2.560,1.455)}
\gppoint{gp mark 1}{(2.562,1.452)}
\gppoint{gp mark 1}{(2.564,1.456)}
\gppoint{gp mark 1}{(2.567,1.472)}
\gppoint{gp mark 1}{(2.569,1.489)}
\gppoint{gp mark 1}{(2.571,1.485)}
\gppoint{gp mark 1}{(2.573,1.490)}
\gppoint{gp mark 1}{(2.576,1.469)}
\gppoint{gp mark 1}{(2.578,1.495)}
\gppoint{gp mark 1}{(2.580,1.516)}
\gppoint{gp mark 1}{(2.582,1.712)}
\gppoint{gp mark 1}{(2.585,2.046)}
\gppoint{gp mark 1}{(2.587,2.177)}
\gppoint{gp mark 1}{(2.589,2.177)}
\gppoint{gp mark 1}{(2.592,2.167)}
\gppoint{gp mark 1}{(2.594,1.589)}
\gppoint{gp mark 1}{(2.596,1.492)}
\gppoint{gp mark 1}{(2.598,1.471)}
\gppoint{gp mark 1}{(2.601,1.469)}
\gppoint{gp mark 1}{(2.603,1.469)}
\gppoint{gp mark 1}{(2.605,1.454)}
\gppoint{gp mark 1}{(2.607,1.458)}
\gppoint{gp mark 1}{(2.610,1.464)}
\gppoint{gp mark 1}{(2.612,1.461)}
\gppoint{gp mark 1}{(2.614,1.452)}
\gppoint{gp mark 1}{(2.616,1.450)}
\gppoint{gp mark 1}{(2.619,1.449)}
\gppoint{gp mark 1}{(2.621,1.474)}
\gppoint{gp mark 1}{(2.623,1.453)}
\gppoint{gp mark 1}{(2.625,1.449)}
\gppoint{gp mark 1}{(2.628,1.448)}
\gppoint{gp mark 1}{(2.630,1.461)}
\gppoint{gp mark 1}{(2.632,1.455)}
\gppoint{gp mark 1}{(2.634,1.453)}
\gppoint{gp mark 1}{(2.637,1.463)}
\gppoint{gp mark 1}{(2.639,1.445)}
\gppoint{gp mark 1}{(2.641,1.459)}
\gppoint{gp mark 1}{(2.643,1.457)}
\gppoint{gp mark 1}{(2.646,1.451)}
\gppoint{gp mark 1}{(2.648,1.448)}
\gppoint{gp mark 1}{(2.650,1.463)}
\gppoint{gp mark 1}{(2.652,1.462)}
\gppoint{gp mark 1}{(2.655,1.457)}
\gppoint{gp mark 1}{(2.657,1.462)}
\gppoint{gp mark 1}{(2.659,1.456)}
\gppoint{gp mark 1}{(2.661,1.456)}
\gppoint{gp mark 1}{(2.664,1.458)}
\gppoint{gp mark 1}{(2.666,1.459)}
\gppoint{gp mark 1}{(2.668,1.472)}
\gppoint{gp mark 1}{(2.670,1.465)}
\gppoint{gp mark 1}{(2.673,1.456)}
\gppoint{gp mark 1}{(2.675,1.441)}
\gppoint{gp mark 1}{(2.677,1.450)}
\gppoint{gp mark 1}{(2.679,1.463)}
\gppoint{gp mark 1}{(2.682,1.445)}
\gppoint{gp mark 1}{(2.684,1.450)}
\gppoint{gp mark 1}{(2.686,1.439)}
\gppoint{gp mark 1}{(2.688,1.450)}
\gppoint{gp mark 1}{(2.691,1.447)}
\gppoint{gp mark 1}{(2.693,1.448)}
\gppoint{gp mark 1}{(2.695,1.451)}
\gppoint{gp mark 1}{(2.697,1.465)}
\gppoint{gp mark 1}{(2.700,1.452)}
\gppoint{gp mark 1}{(2.702,1.457)}
\gppoint{gp mark 1}{(2.704,1.442)}
\gppoint{gp mark 1}{(2.706,1.445)}
\gppoint{gp mark 1}{(2.709,1.457)}
\gppoint{gp mark 1}{(2.711,1.468)}
\gppoint{gp mark 1}{(2.713,1.459)}
\gppoint{gp mark 1}{(2.715,1.470)}
\gppoint{gp mark 1}{(2.718,1.439)}
\gppoint{gp mark 1}{(2.720,1.461)}
\gppoint{gp mark 1}{(2.722,1.447)}
\gppoint{gp mark 1}{(2.724,1.461)}
\gppoint{gp mark 1}{(2.727,1.448)}
\gppoint{gp mark 1}{(2.729,1.450)}
\gppoint{gp mark 1}{(2.731,1.449)}
\gppoint{gp mark 1}{(2.733,1.472)}
\gppoint{gp mark 1}{(2.736,1.471)}
\gppoint{gp mark 1}{(2.738,1.455)}
\gppoint{gp mark 1}{(2.740,1.459)}
\gppoint{gp mark 1}{(2.742,1.439)}
\gppoint{gp mark 1}{(2.745,1.438)}
\gppoint{gp mark 1}{(2.747,1.464)}
\gppoint{gp mark 1}{(2.749,1.460)}
\gppoint{gp mark 1}{(2.751,1.465)}
\gppoint{gp mark 1}{(2.754,1.450)}
\gppoint{gp mark 1}{(2.756,1.436)}
\gppoint{gp mark 1}{(2.758,1.462)}
\gppoint{gp mark 1}{(2.760,1.452)}
\gppoint{gp mark 1}{(2.763,1.453)}
\gppoint{gp mark 1}{(2.765,1.454)}
\gppoint{gp mark 1}{(2.767,1.459)}
\gppoint{gp mark 1}{(2.769,1.452)}
\gppoint{gp mark 1}{(2.772,1.450)}
\gppoint{gp mark 1}{(2.774,1.457)}
\gppoint{gp mark 1}{(2.776,1.451)}
\gppoint{gp mark 1}{(2.778,1.450)}
\gppoint{gp mark 1}{(2.781,1.446)}
\gppoint{gp mark 1}{(2.783,1.455)}
\gppoint{gp mark 1}{(2.785,1.458)}
\gppoint{gp mark 1}{(2.787,1.452)}
\gppoint{gp mark 1}{(2.790,1.469)}
\gppoint{gp mark 1}{(2.792,1.475)}
\gppoint{gp mark 1}{(2.794,1.440)}
\gppoint{gp mark 1}{(2.796,1.469)}
\gppoint{gp mark 1}{(2.799,1.467)}
\gppoint{gp mark 1}{(2.801,1.467)}
\gppoint{gp mark 1}{(2.803,1.468)}
\gppoint{gp mark 1}{(2.805,1.457)}
\gppoint{gp mark 1}{(2.808,1.462)}
\gppoint{gp mark 1}{(2.810,1.473)}
\gppoint{gp mark 1}{(2.812,1.446)}
\gppoint{gp mark 1}{(2.814,1.462)}
\gppoint{gp mark 1}{(2.817,1.462)}
\gppoint{gp mark 1}{(2.819,1.463)}
\gppoint{gp mark 1}{(2.821,1.451)}
\gppoint{gp mark 1}{(2.823,1.467)}
\gppoint{gp mark 1}{(2.826,1.457)}
\gppoint{gp mark 1}{(2.828,1.458)}
\gppoint{gp mark 1}{(2.830,1.460)}
\gppoint{gp mark 1}{(2.832,1.450)}
\gppoint{gp mark 1}{(2.835,1.448)}
\gppoint{gp mark 1}{(2.837,1.464)}
\gppoint{gp mark 1}{(2.839,1.456)}
\gppoint{gp mark 1}{(2.841,1.462)}
\gppoint{gp mark 1}{(2.844,1.447)}
\gppoint{gp mark 1}{(2.846,1.452)}
\gppoint{gp mark 1}{(2.848,1.448)}
\gppoint{gp mark 1}{(2.850,1.452)}
\gppoint{gp mark 1}{(2.853,1.455)}
\gppoint{gp mark 1}{(2.855,1.443)}
\gppoint{gp mark 1}{(2.857,1.462)}
\gppoint{gp mark 1}{(2.859,1.454)}
\gppoint{gp mark 1}{(2.862,1.453)}
\gppoint{gp mark 1}{(2.864,1.448)}
\gppoint{gp mark 1}{(2.866,1.473)}
\gppoint{gp mark 1}{(2.868,1.450)}
\gppoint{gp mark 1}{(2.871,1.440)}
\gppoint{gp mark 1}{(2.873,1.435)}
\gppoint{gp mark 1}{(2.875,1.446)}
\gppoint{gp mark 1}{(2.877,1.455)}
\gppoint{gp mark 1}{(2.880,1.445)}
\gppoint{gp mark 1}{(2.882,1.436)}
\gppoint{gp mark 1}{(2.884,1.461)}
\gppoint{gp mark 1}{(2.886,1.467)}
\gppoint{gp mark 1}{(2.889,1.461)}
\gppoint{gp mark 1}{(2.891,1.466)}
\gppoint{gp mark 1}{(2.893,1.462)}
\gppoint{gp mark 1}{(2.895,1.455)}
\gppoint{gp mark 1}{(2.898,1.446)}
\gppoint{gp mark 1}{(2.900,1.458)}
\gppoint{gp mark 1}{(2.902,1.451)}
\gppoint{gp mark 1}{(2.904,1.463)}
\gppoint{gp mark 1}{(2.907,1.470)}
\gppoint{gp mark 1}{(2.909,1.455)}
\gppoint{gp mark 1}{(2.911,1.450)}
\gppoint{gp mark 1}{(2.913,1.446)}
\gppoint{gp mark 1}{(2.916,1.458)}
\gppoint{gp mark 1}{(2.918,1.466)}
\gppoint{gp mark 1}{(2.920,1.463)}
\gppoint{gp mark 1}{(2.922,1.463)}
\gppoint{gp mark 1}{(2.925,1.451)}
\gppoint{gp mark 1}{(2.927,1.469)}
\gppoint{gp mark 1}{(2.929,1.467)}
\gppoint{gp mark 1}{(2.931,1.447)}
\gppoint{gp mark 1}{(2.934,1.453)}
\gppoint{gp mark 1}{(2.936,1.456)}
\gppoint{gp mark 1}{(2.938,1.465)}
\gppoint{gp mark 1}{(2.940,1.457)}
\gppoint{gp mark 1}{(2.943,1.461)}
\gppoint{gp mark 1}{(2.945,1.458)}
\gppoint{gp mark 1}{(2.947,1.459)}
\gppoint{gp mark 1}{(2.949,1.459)}
\gppoint{gp mark 1}{(2.952,1.460)}
\gppoint{gp mark 1}{(2.954,1.440)}
\gppoint{gp mark 1}{(2.956,1.460)}
\gppoint{gp mark 1}{(2.958,1.448)}
\gppoint{gp mark 1}{(2.961,1.450)}
\gppoint{gp mark 1}{(2.963,1.441)}
\gppoint{gp mark 1}{(2.965,1.454)}
\gppoint{gp mark 1}{(2.967,1.459)}
\gppoint{gp mark 1}{(2.970,1.438)}
\gppoint{gp mark 1}{(2.972,1.463)}
\gppoint{gp mark 1}{(2.974,1.469)}
\gppoint{gp mark 1}{(2.976,1.454)}
\gppoint{gp mark 1}{(2.979,1.445)}
\gppoint{gp mark 1}{(2.981,1.467)}
\gppoint{gp mark 1}{(2.983,1.465)}
\gppoint{gp mark 1}{(2.985,1.456)}
\gppoint{gp mark 1}{(2.988,1.461)}
\gppoint{gp mark 1}{(2.990,1.449)}
\gppoint{gp mark 1}{(2.992,1.462)}
\gppoint{gp mark 1}{(2.994,1.463)}
\gppoint{gp mark 1}{(2.997,1.454)}
\gppoint{gp mark 1}{(2.999,1.457)}
\gppoint{gp mark 1}{(3.001,1.471)}
\gppoint{gp mark 1}{(3.003,1.447)}
\gppoint{gp mark 1}{(3.006,1.459)}
\gppoint{gp mark 1}{(3.008,1.471)}
\gppoint{gp mark 1}{(3.010,1.451)}
\gppoint{gp mark 1}{(3.012,1.458)}
\gppoint{gp mark 1}{(3.015,1.451)}
\gppoint{gp mark 1}{(3.017,1.472)}
\gppoint{gp mark 1}{(3.019,1.463)}
\gppoint{gp mark 1}{(3.021,1.455)}
\gppoint{gp mark 1}{(3.024,1.476)}
\gppoint{gp mark 1}{(3.026,1.447)}
\gppoint{gp mark 1}{(3.028,1.459)}
\gppoint{gp mark 1}{(3.030,1.446)}
\gppoint{gp mark 1}{(3.033,1.462)}
\gppoint{gp mark 1}{(3.035,1.462)}
\gppoint{gp mark 1}{(3.037,1.467)}
\gppoint{gp mark 1}{(3.039,1.451)}
\gppoint{gp mark 1}{(3.042,1.450)}
\gppoint{gp mark 1}{(3.044,1.457)}
\gppoint{gp mark 1}{(3.046,1.460)}
\gppoint{gp mark 1}{(3.048,1.459)}
\gppoint{gp mark 1}{(3.051,1.455)}
\gppoint{gp mark 1}{(3.053,1.471)}
\gppoint{gp mark 1}{(3.055,1.458)}
\gppoint{gp mark 1}{(3.057,1.469)}
\gppoint{gp mark 1}{(3.060,1.470)}
\gppoint{gp mark 1}{(3.062,1.465)}
\gppoint{gp mark 1}{(3.064,1.464)}
\gppoint{gp mark 1}{(3.066,1.453)}
\gppoint{gp mark 1}{(3.069,1.442)}
\gppoint{gp mark 1}{(3.071,1.451)}
\gppoint{gp mark 1}{(3.073,1.461)}
\gppoint{gp mark 1}{(3.075,1.459)}
\gppoint{gp mark 1}{(3.078,1.445)}
\gppoint{gp mark 1}{(3.080,1.456)}
\gppoint{gp mark 1}{(3.082,1.454)}
\gppoint{gp mark 1}{(3.084,1.453)}
\gppoint{gp mark 1}{(3.087,1.447)}
\gppoint{gp mark 1}{(3.089,1.439)}
\gppoint{gp mark 1}{(3.091,1.457)}
\gppoint{gp mark 1}{(3.093,1.432)}
\gppoint{gp mark 1}{(3.096,1.467)}
\gppoint{gp mark 1}{(3.098,1.438)}
\gppoint{gp mark 1}{(3.100,1.464)}
\gppoint{gp mark 1}{(3.102,1.460)}
\gppoint{gp mark 1}{(3.105,1.454)}
\gppoint{gp mark 1}{(3.107,1.448)}
\gppoint{gp mark 1}{(3.109,1.449)}
\gppoint{gp mark 1}{(3.111,1.456)}
\gppoint{gp mark 1}{(3.114,1.469)}
\gppoint{gp mark 1}{(3.116,1.451)}
\gppoint{gp mark 1}{(3.118,1.445)}
\gppoint{gp mark 1}{(3.120,1.440)}
\gppoint{gp mark 1}{(3.123,1.469)}
\gppoint{gp mark 1}{(3.125,1.464)}
\gppoint{gp mark 1}{(3.127,1.436)}
\gppoint{gp mark 1}{(3.129,1.455)}
\gppoint{gp mark 1}{(3.132,1.427)}
\gppoint{gp mark 1}{(3.134,1.438)}
\gppoint{gp mark 1}{(3.136,1.461)}
\gppoint{gp mark 1}{(3.138,1.451)}
\gppoint{gp mark 1}{(3.141,1.459)}
\gppoint{gp mark 1}{(3.143,1.451)}
\gppoint{gp mark 1}{(3.145,1.454)}
\gppoint{gp mark 1}{(3.147,1.457)}
\gppoint{gp mark 1}{(3.150,1.450)}
\gppoint{gp mark 1}{(3.152,1.458)}
\gppoint{gp mark 1}{(3.154,1.448)}
\gppoint{gp mark 1}{(3.156,1.443)}
\gppoint{gp mark 1}{(3.159,1.443)}
\gppoint{gp mark 1}{(3.161,1.444)}
\gppoint{gp mark 1}{(3.163,1.454)}
\gppoint{gp mark 1}{(3.165,1.450)}
\gppoint{gp mark 1}{(3.168,1.446)}
\gppoint{gp mark 1}{(3.170,1.454)}
\gppoint{gp mark 1}{(3.172,1.435)}
\gppoint{gp mark 1}{(3.174,1.445)}
\gppoint{gp mark 1}{(3.177,1.464)}
\gppoint{gp mark 1}{(3.179,1.451)}
\gppoint{gp mark 1}{(3.181,1.447)}
\gppoint{gp mark 1}{(3.183,1.433)}
\gppoint{gp mark 1}{(3.186,1.448)}
\gppoint{gp mark 1}{(3.188,1.437)}
\gppoint{gp mark 1}{(3.190,1.464)}
\gppoint{gp mark 1}{(3.192,1.435)}
\gppoint{gp mark 1}{(3.195,1.456)}
\gppoint{gp mark 1}{(3.197,1.448)}
\gppoint{gp mark 1}{(3.199,1.447)}
\gppoint{gp mark 1}{(3.201,1.447)}
\gppoint{gp mark 1}{(3.204,1.438)}
\gppoint{gp mark 1}{(3.206,1.427)}
\gppoint{gp mark 1}{(3.208,1.435)}
\gppoint{gp mark 1}{(3.210,1.428)}
\gppoint{gp mark 1}{(3.213,1.453)}
\gppoint{gp mark 1}{(3.215,1.472)}
\gppoint{gp mark 1}{(3.217,1.432)}
\gppoint{gp mark 1}{(3.219,1.450)}
\gppoint{gp mark 1}{(3.222,1.439)}
\gppoint{gp mark 1}{(3.224,1.440)}
\gppoint{gp mark 1}{(3.226,1.442)}
\gppoint{gp mark 1}{(3.228,1.440)}
\gppoint{gp mark 1}{(3.231,1.435)}
\gppoint{gp mark 1}{(3.233,1.452)}
\gppoint{gp mark 1}{(3.235,1.424)}
\gppoint{gp mark 1}{(3.237,1.438)}
\gppoint{gp mark 1}{(3.240,1.430)}
\gppoint{gp mark 1}{(3.242,1.440)}
\gppoint{gp mark 1}{(3.244,1.446)}
\gppoint{gp mark 1}{(3.246,1.469)}
\gppoint{gp mark 1}{(3.249,1.437)}
\gppoint{gp mark 1}{(3.251,1.450)}
\gppoint{gp mark 1}{(3.253,1.440)}
\gppoint{gp mark 1}{(3.255,1.453)}
\gppoint{gp mark 1}{(3.258,1.444)}
\gppoint{gp mark 1}{(3.260,1.436)}
\gppoint{gp mark 1}{(3.262,1.438)}
\gppoint{gp mark 1}{(3.264,1.457)}
\gppoint{gp mark 1}{(3.267,1.436)}
\gppoint{gp mark 1}{(3.269,1.442)}
\gppoint{gp mark 1}{(3.271,1.439)}
\gppoint{gp mark 1}{(3.273,1.448)}
\gppoint{gp mark 1}{(3.276,1.429)}
\gppoint{gp mark 1}{(3.278,1.438)}
\gppoint{gp mark 1}{(3.280,1.435)}
\gppoint{gp mark 1}{(3.282,1.438)}
\gppoint{gp mark 1}{(3.285,1.448)}
\gppoint{gp mark 1}{(3.287,1.453)}
\gppoint{gp mark 1}{(3.289,1.452)}
\gppoint{gp mark 1}{(3.291,1.439)}
\gppoint{gp mark 1}{(3.294,1.441)}
\gppoint{gp mark 1}{(3.296,1.424)}
\gppoint{gp mark 1}{(3.298,1.433)}
\gppoint{gp mark 1}{(3.300,1.447)}
\gppoint{gp mark 1}{(3.303,1.438)}
\gppoint{gp mark 1}{(3.305,1.445)}
\gppoint{gp mark 1}{(3.307,1.441)}
\gppoint{gp mark 1}{(3.309,1.440)}
\gppoint{gp mark 1}{(3.312,1.433)}
\gppoint{gp mark 1}{(3.314,1.424)}
\gppoint{gp mark 1}{(3.316,1.426)}
\gppoint{gp mark 1}{(3.318,1.422)}
\gppoint{gp mark 1}{(3.321,1.442)}
\gppoint{gp mark 1}{(3.323,1.455)}
\gppoint{gp mark 1}{(3.325,1.451)}
\gppoint{gp mark 1}{(3.327,1.440)}
\gppoint{gp mark 1}{(3.330,1.439)}
\gppoint{gp mark 1}{(3.332,1.440)}
\gppoint{gp mark 1}{(3.334,1.442)}
\gppoint{gp mark 1}{(3.336,1.443)}
\gppoint{gp mark 1}{(3.339,1.448)}
\gppoint{gp mark 1}{(3.341,1.444)}
\gppoint{gp mark 1}{(3.343,1.419)}
\gppoint{gp mark 1}{(3.345,1.445)}
\gppoint{gp mark 1}{(3.348,1.431)}
\gppoint{gp mark 1}{(3.350,1.432)}
\gppoint{gp mark 1}{(3.352,1.442)}
\gppoint{gp mark 1}{(3.354,1.436)}
\gppoint{gp mark 1}{(3.357,1.441)}
\gppoint{gp mark 1}{(3.359,1.444)}
\gppoint{gp mark 1}{(3.361,1.422)}
\gppoint{gp mark 1}{(3.363,1.455)}
\gppoint{gp mark 1}{(3.366,1.455)}
\gppoint{gp mark 1}{(3.368,1.429)}
\gppoint{gp mark 1}{(3.370,1.448)}
\gppoint{gp mark 1}{(3.372,1.452)}
\gppoint{gp mark 1}{(3.375,1.436)}
\gppoint{gp mark 1}{(3.377,1.426)}
\gppoint{gp mark 1}{(3.379,1.425)}
\gppoint{gp mark 1}{(3.381,1.439)}
\gppoint{gp mark 1}{(3.384,1.425)}
\gppoint{gp mark 1}{(3.386,1.431)}
\gppoint{gp mark 1}{(3.388,1.434)}
\gppoint{gp mark 1}{(3.390,1.430)}
\gppoint{gp mark 1}{(3.393,1.434)}
\gppoint{gp mark 1}{(3.395,1.427)}
\gppoint{gp mark 1}{(3.397,1.421)}
\gppoint{gp mark 1}{(3.399,1.431)}
\gppoint{gp mark 1}{(3.402,1.426)}
\gppoint{gp mark 1}{(3.404,1.415)}
\gppoint{gp mark 1}{(3.406,1.430)}
\gppoint{gp mark 1}{(3.408,1.452)}
\gppoint{gp mark 1}{(3.411,1.436)}
\gppoint{gp mark 1}{(3.413,1.446)}
\gppoint{gp mark 1}{(3.415,1.435)}
\gppoint{gp mark 1}{(3.417,1.415)}
\gppoint{gp mark 1}{(3.420,1.452)}
\gppoint{gp mark 1}{(3.422,1.436)}
\gppoint{gp mark 1}{(3.424,1.433)}
\gppoint{gp mark 1}{(3.426,1.398)}
\gppoint{gp mark 1}{(3.429,1.428)}
\gppoint{gp mark 1}{(3.431,1.415)}
\gppoint{gp mark 1}{(3.433,1.414)}
\gppoint{gp mark 1}{(3.435,1.422)}
\gppoint{gp mark 1}{(3.438,1.429)}
\gppoint{gp mark 1}{(3.440,1.437)}
\gppoint{gp mark 1}{(3.442,1.412)}
\gppoint{gp mark 1}{(3.444,1.416)}
\gppoint{gp mark 1}{(3.447,1.417)}
\gppoint{gp mark 1}{(3.449,1.427)}
\gppoint{gp mark 1}{(3.451,1.439)}
\gppoint{gp mark 1}{(3.453,1.439)}
\gppoint{gp mark 1}{(3.456,1.418)}
\gppoint{gp mark 1}{(3.458,1.431)}
\gppoint{gp mark 1}{(3.460,1.402)}
\gppoint{gp mark 1}{(3.462,1.417)}
\gppoint{gp mark 1}{(3.465,1.439)}
\gppoint{gp mark 1}{(3.467,1.428)}
\gppoint{gp mark 1}{(3.469,1.432)}
\gppoint{gp mark 1}{(3.471,1.409)}
\gppoint{gp mark 1}{(3.474,1.413)}
\gppoint{gp mark 1}{(3.476,1.434)}
\gppoint{gp mark 1}{(3.478,1.434)}
\gppoint{gp mark 1}{(3.480,1.417)}
\gppoint{gp mark 1}{(3.483,1.420)}
\gppoint{gp mark 1}{(3.485,1.405)}
\gppoint{gp mark 1}{(3.487,1.403)}
\gppoint{gp mark 1}{(3.489,1.429)}
\gppoint{gp mark 1}{(3.492,1.424)}
\gppoint{gp mark 1}{(3.494,1.412)}
\gppoint{gp mark 1}{(3.496,1.427)}
\gppoint{gp mark 1}{(3.498,1.419)}
\gppoint{gp mark 1}{(3.501,1.429)}
\gppoint{gp mark 1}{(3.503,1.400)}
\gppoint{gp mark 1}{(3.505,1.415)}
\gppoint{gp mark 1}{(3.507,1.402)}
\gppoint{gp mark 1}{(3.510,1.426)}
\gppoint{gp mark 1}{(3.512,1.400)}
\gppoint{gp mark 1}{(3.514,1.427)}
\gppoint{gp mark 1}{(3.516,1.421)}
\gppoint{gp mark 1}{(3.519,1.409)}
\gppoint{gp mark 1}{(3.521,1.431)}
\gppoint{gp mark 1}{(3.523,1.413)}
\gppoint{gp mark 1}{(3.525,1.413)}
\gppoint{gp mark 1}{(3.528,1.422)}
\gppoint{gp mark 1}{(3.530,1.423)}
\gppoint{gp mark 1}{(3.532,1.403)}
\gppoint{gp mark 1}{(3.534,1.417)}
\gppoint{gp mark 1}{(3.537,1.458)}
\gppoint{gp mark 1}{(3.539,1.402)}
\gppoint{gp mark 1}{(3.541,1.420)}
\gppoint{gp mark 1}{(3.543,1.431)}
\gppoint{gp mark 1}{(3.546,1.410)}
\gppoint{gp mark 1}{(3.548,1.420)}
\gppoint{gp mark 1}{(3.550,1.395)}
\gppoint{gp mark 1}{(3.552,1.416)}
\gppoint{gp mark 1}{(3.555,1.409)}
\gppoint{gp mark 1}{(3.557,1.415)}
\gppoint{gp mark 1}{(3.559,1.425)}
\gppoint{gp mark 1}{(3.561,1.417)}
\gppoint{gp mark 1}{(3.564,1.435)}
\gppoint{gp mark 1}{(3.566,1.438)}
\gppoint{gp mark 1}{(3.568,1.419)}
\gppoint{gp mark 1}{(3.570,1.413)}
\gppoint{gp mark 1}{(3.573,1.410)}
\gppoint{gp mark 1}{(3.575,1.436)}
\gppoint{gp mark 1}{(3.577,1.408)}
\gppoint{gp mark 1}{(3.579,1.424)}
\gppoint{gp mark 1}{(3.582,1.416)}
\gppoint{gp mark 1}{(3.584,1.431)}
\gppoint{gp mark 1}{(3.586,1.442)}
\gppoint{gp mark 1}{(3.588,1.414)}
\gppoint{gp mark 1}{(3.591,1.423)}
\gppoint{gp mark 1}{(3.593,1.389)}
\gppoint{gp mark 1}{(3.595,1.413)}
\gppoint{gp mark 1}{(3.597,1.411)}
\gppoint{gp mark 1}{(3.600,1.419)}
\gppoint{gp mark 1}{(3.602,1.407)}
\gppoint{gp mark 1}{(3.604,1.409)}
\gppoint{gp mark 1}{(3.606,1.392)}
\gppoint{gp mark 1}{(3.609,1.409)}
\gppoint{gp mark 1}{(3.611,1.411)}
\gppoint{gp mark 1}{(3.613,1.405)}
\gppoint{gp mark 1}{(3.615,1.426)}
\gppoint{gp mark 1}{(3.618,1.417)}
\gppoint{gp mark 1}{(3.620,1.416)}
\gppoint{gp mark 1}{(3.622,1.420)}
\gppoint{gp mark 1}{(3.624,1.430)}
\gppoint{gp mark 1}{(3.627,1.422)}
\gppoint{gp mark 1}{(3.629,1.410)}
\gppoint{gp mark 1}{(3.631,1.405)}
\gppoint{gp mark 1}{(3.633,1.389)}
\gppoint{gp mark 1}{(3.636,1.413)}
\gppoint{gp mark 1}{(3.638,1.400)}
\gppoint{gp mark 1}{(3.640,1.427)}
\gppoint{gp mark 1}{(3.642,1.430)}
\gppoint{gp mark 1}{(3.645,1.441)}
\gppoint{gp mark 1}{(3.647,1.413)}
\gppoint{gp mark 1}{(3.649,1.418)}
\gppoint{gp mark 1}{(3.651,1.418)}
\gppoint{gp mark 1}{(3.654,1.405)}
\gppoint{gp mark 1}{(3.656,1.408)}
\gppoint{gp mark 1}{(3.658,1.436)}
\gppoint{gp mark 1}{(3.660,1.403)}
\gppoint{gp mark 1}{(3.663,1.410)}
\gppoint{gp mark 1}{(3.665,1.437)}
\gppoint{gp mark 1}{(3.667,1.407)}
\gppoint{gp mark 1}{(3.669,1.415)}
\gppoint{gp mark 1}{(3.672,1.389)}
\gppoint{gp mark 1}{(3.674,1.394)}
\gppoint{gp mark 1}{(3.676,1.398)}
\gppoint{gp mark 1}{(3.678,1.416)}
\gppoint{gp mark 1}{(3.681,1.422)}
\gppoint{gp mark 1}{(3.683,1.400)}
\gppoint{gp mark 1}{(3.685,1.430)}
\gppoint{gp mark 1}{(3.687,1.411)}
\gppoint{gp mark 1}{(3.690,1.426)}
\gppoint{gp mark 1}{(3.692,1.408)}
\gppoint{gp mark 1}{(3.694,1.420)}
\gppoint{gp mark 1}{(3.696,1.415)}
\gppoint{gp mark 1}{(3.699,1.413)}
\gppoint{gp mark 1}{(3.701,1.401)}
\gppoint{gp mark 1}{(3.703,1.396)}
\gppoint{gp mark 1}{(3.705,1.418)}
\gppoint{gp mark 1}{(3.708,1.413)}
\gppoint{gp mark 1}{(3.710,1.405)}
\gppoint{gp mark 1}{(3.712,1.413)}
\gppoint{gp mark 1}{(3.714,1.420)}
\gppoint{gp mark 1}{(3.717,1.413)}
\gppoint{gp mark 1}{(3.719,1.404)}
\gppoint{gp mark 1}{(3.721,1.406)}
\gppoint{gp mark 1}{(3.723,1.417)}
\gppoint{gp mark 1}{(3.726,1.420)}
\gppoint{gp mark 1}{(3.728,1.427)}
\gppoint{gp mark 1}{(3.730,1.423)}
\gppoint{gp mark 1}{(3.732,1.414)}
\gppoint{gp mark 1}{(3.735,1.423)}
\gppoint{gp mark 1}{(3.737,1.423)}
\gppoint{gp mark 1}{(3.739,1.415)}
\gppoint{gp mark 1}{(3.741,1.399)}
\gppoint{gp mark 1}{(3.744,1.411)}
\gppoint{gp mark 1}{(3.746,1.424)}
\gppoint{gp mark 1}{(3.748,1.409)}
\gppoint{gp mark 1}{(3.750,1.405)}
\gppoint{gp mark 1}{(3.753,1.405)}
\gppoint{gp mark 1}{(3.755,1.413)}
\gppoint{gp mark 1}{(3.757,1.427)}
\gppoint{gp mark 1}{(3.759,1.398)}
\gppoint{gp mark 1}{(3.762,1.412)}
\gppoint{gp mark 1}{(3.764,1.411)}
\gppoint{gp mark 1}{(3.766,1.412)}
\gppoint{gp mark 1}{(3.768,1.411)}
\gppoint{gp mark 1}{(3.771,1.400)}
\gppoint{gp mark 1}{(3.773,1.406)}
\gppoint{gp mark 1}{(3.775,1.405)}
\gppoint{gp mark 1}{(3.777,1.406)}
\gppoint{gp mark 1}{(3.780,1.398)}
\gppoint{gp mark 1}{(3.782,1.400)}
\gppoint{gp mark 1}{(3.784,1.410)}
\gppoint{gp mark 1}{(3.786,1.409)}
\gppoint{gp mark 1}{(3.789,1.411)}
\gppoint{gp mark 1}{(3.791,1.414)}
\gppoint{gp mark 1}{(3.793,1.405)}
\gppoint{gp mark 1}{(3.795,1.395)}
\gppoint{gp mark 1}{(3.798,1.388)}
\gppoint{gp mark 1}{(3.800,1.408)}
\gppoint{gp mark 1}{(3.802,1.398)}
\gppoint{gp mark 1}{(3.804,1.407)}
\gppoint{gp mark 1}{(3.807,1.419)}
\gppoint{gp mark 1}{(3.809,1.408)}
\gppoint{gp mark 1}{(3.811,1.394)}
\gppoint{gp mark 1}{(3.813,1.399)}
\gppoint{gp mark 1}{(3.816,1.404)}
\gppoint{gp mark 1}{(3.818,1.405)}
\gppoint{gp mark 1}{(3.820,1.409)}
\gppoint{gp mark 1}{(3.822,1.400)}
\gppoint{gp mark 1}{(3.825,1.393)}
\gppoint{gp mark 1}{(3.827,1.410)}
\gppoint{gp mark 1}{(3.829,1.418)}
\gppoint{gp mark 1}{(3.831,1.400)}
\gppoint{gp mark 1}{(3.834,1.403)}
\gppoint{gp mark 1}{(3.836,1.396)}
\gppoint{gp mark 1}{(3.838,1.400)}
\gppoint{gp mark 1}{(3.840,1.407)}
\gppoint{gp mark 1}{(3.843,1.388)}
\gppoint{gp mark 1}{(3.845,1.410)}
\gppoint{gp mark 1}{(3.847,1.389)}
\gppoint{gp mark 1}{(3.849,1.409)}
\gppoint{gp mark 1}{(3.852,1.410)}
\gppoint{gp mark 1}{(3.854,1.414)}
\gppoint{gp mark 1}{(3.856,1.394)}
\gppoint{gp mark 1}{(3.859,1.405)}
\gppoint{gp mark 1}{(3.861,1.400)}
\gppoint{gp mark 1}{(3.863,1.408)}
\gppoint{gp mark 1}{(3.865,1.398)}
\gppoint{gp mark 1}{(3.868,1.417)}
\gppoint{gp mark 1}{(3.870,1.420)}
\gppoint{gp mark 1}{(3.872,1.404)}
\gppoint{gp mark 1}{(3.874,1.410)}
\gppoint{gp mark 1}{(3.877,1.396)}
\gppoint{gp mark 1}{(3.879,1.402)}
\gppoint{gp mark 1}{(3.881,1.403)}
\gppoint{gp mark 1}{(3.883,1.421)}
\gppoint{gp mark 1}{(3.886,1.420)}
\gppoint{gp mark 1}{(3.888,1.397)}
\gppoint{gp mark 1}{(3.890,1.416)}
\gppoint{gp mark 1}{(3.892,1.412)}
\gppoint{gp mark 1}{(3.895,1.407)}
\gppoint{gp mark 1}{(3.897,1.405)}
\gppoint{gp mark 1}{(3.899,1.401)}
\gppoint{gp mark 1}{(3.901,1.401)}
\gppoint{gp mark 1}{(3.904,1.412)}
\gppoint{gp mark 1}{(3.906,1.420)}
\gppoint{gp mark 1}{(3.908,1.405)}
\gppoint{gp mark 1}{(3.910,1.418)}
\gppoint{gp mark 1}{(3.913,1.423)}
\gppoint{gp mark 1}{(3.915,1.423)}
\gppoint{gp mark 1}{(3.917,1.417)}
\gppoint{gp mark 1}{(3.919,1.401)}
\gppoint{gp mark 1}{(3.922,1.412)}
\gppoint{gp mark 1}{(3.924,1.417)}
\gppoint{gp mark 1}{(3.926,1.414)}
\gppoint{gp mark 1}{(3.928,1.419)}
\gppoint{gp mark 1}{(3.931,1.427)}
\gppoint{gp mark 1}{(3.933,1.400)}
\gppoint{gp mark 1}{(3.935,1.399)}
\gppoint{gp mark 1}{(3.937,1.418)}
\gppoint{gp mark 1}{(3.940,1.420)}
\gppoint{gp mark 1}{(3.942,1.422)}
\gppoint{gp mark 1}{(3.944,1.410)}
\gppoint{gp mark 1}{(3.946,1.422)}
\gppoint{gp mark 1}{(3.949,1.428)}
\gppoint{gp mark 1}{(3.951,1.399)}
\gppoint{gp mark 1}{(3.953,1.419)}
\gppoint{gp mark 1}{(3.955,1.419)}
\gppoint{gp mark 1}{(3.958,1.409)}
\gppoint{gp mark 1}{(3.960,1.402)}
\gppoint{gp mark 1}{(3.962,1.419)}
\gppoint{gp mark 1}{(3.964,1.400)}
\gppoint{gp mark 1}{(3.967,1.417)}
\gppoint{gp mark 1}{(3.969,1.410)}
\gppoint{gp mark 1}{(3.971,1.417)}
\gppoint{gp mark 1}{(3.973,1.415)}
\gppoint{gp mark 1}{(3.976,1.419)}
\gppoint{gp mark 1}{(3.978,1.424)}
\gppoint{gp mark 1}{(3.980,1.426)}
\gppoint{gp mark 1}{(3.982,1.420)}
\gppoint{gp mark 1}{(3.985,1.420)}
\gppoint{gp mark 1}{(3.987,1.417)}
\gppoint{gp mark 1}{(3.989,1.404)}
\gppoint{gp mark 1}{(3.991,1.434)}
\gppoint{gp mark 1}{(3.994,1.430)}
\gppoint{gp mark 1}{(3.996,1.425)}
\gppoint{gp mark 1}{(3.998,1.420)}
\gppoint{gp mark 1}{(4.000,1.425)}
\gppoint{gp mark 1}{(4.003,1.431)}
\gppoint{gp mark 1}{(4.005,1.434)}
\gppoint{gp mark 1}{(4.007,1.417)}
\gppoint{gp mark 1}{(4.009,1.446)}
\gppoint{gp mark 1}{(4.012,1.442)}
\gppoint{gp mark 1}{(4.014,1.414)}
\gppoint{gp mark 1}{(4.016,1.437)}
\gppoint{gp mark 1}{(4.018,1.416)}
\gppoint{gp mark 1}{(4.021,1.439)}
\gppoint{gp mark 1}{(4.023,1.456)}
\gppoint{gp mark 1}{(4.025,1.429)}
\gppoint{gp mark 1}{(4.027,1.427)}
\gppoint{gp mark 1}{(4.030,1.433)}
\gppoint{gp mark 1}{(4.032,1.428)}
\gppoint{gp mark 1}{(4.034,1.429)}
\gppoint{gp mark 1}{(4.036,1.432)}
\gppoint{gp mark 1}{(4.039,1.415)}
\gppoint{gp mark 1}{(4.041,1.428)}
\gppoint{gp mark 1}{(4.043,1.429)}
\gppoint{gp mark 1}{(4.045,1.426)}
\gppoint{gp mark 1}{(4.048,1.434)}
\gppoint{gp mark 1}{(4.050,1.441)}
\gppoint{gp mark 1}{(4.052,1.439)}
\gppoint{gp mark 1}{(4.054,1.427)}
\gppoint{gp mark 1}{(4.057,1.461)}
\gppoint{gp mark 1}{(4.059,1.447)}
\gppoint{gp mark 1}{(4.061,1.440)}
\gppoint{gp mark 1}{(4.063,1.447)}
\gppoint{gp mark 1}{(4.066,1.467)}
\gppoint{gp mark 1}{(4.068,1.447)}
\gppoint{gp mark 1}{(4.070,1.442)}
\gppoint{gp mark 1}{(4.072,1.446)}
\gppoint{gp mark 1}{(4.075,1.458)}
\gppoint{gp mark 1}{(4.077,1.455)}
\gppoint{gp mark 1}{(4.079,1.449)}
\gppoint{gp mark 1}{(4.081,1.439)}
\gppoint{gp mark 1}{(4.084,1.454)}
\gppoint{gp mark 1}{(4.086,1.454)}
\gppoint{gp mark 1}{(4.088,1.454)}
\gppoint{gp mark 1}{(4.090,1.447)}
\gppoint{gp mark 1}{(4.093,1.449)}
\gppoint{gp mark 1}{(4.095,1.465)}
\gppoint{gp mark 1}{(4.097,1.454)}
\gppoint{gp mark 1}{(4.099,1.465)}
\gppoint{gp mark 1}{(4.102,1.464)}
\gppoint{gp mark 1}{(4.104,1.464)}
\gppoint{gp mark 1}{(4.106,1.451)}
\gppoint{gp mark 1}{(4.108,1.451)}
\gppoint{gp mark 1}{(4.111,1.466)}
\gppoint{gp mark 1}{(4.113,1.464)}
\gppoint{gp mark 1}{(4.115,1.459)}
\gppoint{gp mark 1}{(4.117,1.482)}
\gppoint{gp mark 1}{(4.120,1.496)}
\gppoint{gp mark 1}{(4.122,1.469)}
\gppoint{gp mark 1}{(4.124,1.464)}
\gppoint{gp mark 1}{(4.126,1.467)}
\gppoint{gp mark 1}{(4.129,1.463)}
\gppoint{gp mark 1}{(4.131,1.482)}
\gppoint{gp mark 1}{(4.133,1.473)}
\gppoint{gp mark 1}{(4.135,1.484)}
\gppoint{gp mark 1}{(4.138,1.479)}
\gppoint{gp mark 1}{(4.140,1.485)}
\gppoint{gp mark 1}{(4.142,1.484)}
\gppoint{gp mark 1}{(4.144,1.488)}
\gppoint{gp mark 1}{(4.147,1.485)}
\gppoint{gp mark 1}{(4.149,1.476)}
\gppoint{gp mark 1}{(4.151,1.479)}
\gppoint{gp mark 1}{(4.153,1.494)}
\gppoint{gp mark 1}{(4.156,1.498)}
\gppoint{gp mark 1}{(4.158,1.469)}
\gppoint{gp mark 1}{(4.160,1.493)}
\gppoint{gp mark 1}{(4.162,1.492)}
\gppoint{gp mark 1}{(4.165,1.483)}
\gppoint{gp mark 1}{(4.167,1.494)}
\gppoint{gp mark 1}{(4.169,1.503)}
\gppoint{gp mark 1}{(4.171,1.480)}
\gppoint{gp mark 1}{(4.174,1.498)}
\gppoint{gp mark 1}{(4.176,1.485)}
\gppoint{gp mark 1}{(4.178,1.500)}
\gppoint{gp mark 1}{(4.180,1.493)}
\gppoint{gp mark 1}{(4.183,1.504)}
\gppoint{gp mark 1}{(4.185,1.533)}
\gppoint{gp mark 1}{(4.187,1.505)}
\gppoint{gp mark 1}{(4.189,1.489)}
\gppoint{gp mark 1}{(4.192,1.506)}
\gppoint{gp mark 1}{(4.194,1.512)}
\gppoint{gp mark 1}{(4.196,1.511)}
\gppoint{gp mark 1}{(4.198,1.493)}
\gppoint{gp mark 1}{(4.201,1.514)}
\gppoint{gp mark 1}{(4.203,1.519)}
\gppoint{gp mark 1}{(4.205,1.508)}
\gppoint{gp mark 1}{(4.207,1.523)}
\gppoint{gp mark 1}{(4.210,1.523)}
\gppoint{gp mark 1}{(4.212,1.507)}
\gppoint{gp mark 1}{(4.214,1.525)}
\gppoint{gp mark 1}{(4.216,1.504)}
\gppoint{gp mark 1}{(4.219,1.520)}
\gppoint{gp mark 1}{(4.221,1.545)}
\gppoint{gp mark 1}{(4.223,1.533)}
\gppoint{gp mark 1}{(4.225,1.521)}
\gppoint{gp mark 1}{(4.228,1.538)}
\gppoint{gp mark 1}{(4.230,1.541)}
\gppoint{gp mark 1}{(4.232,1.516)}
\gppoint{gp mark 1}{(4.234,1.538)}
\gppoint{gp mark 1}{(4.237,1.552)}
\gppoint{gp mark 1}{(4.239,1.546)}
\gppoint{gp mark 1}{(4.241,1.518)}
\gppoint{gp mark 1}{(4.243,1.538)}
\gppoint{gp mark 1}{(4.246,1.529)}
\gppoint{gp mark 1}{(4.248,1.543)}
\gppoint{gp mark 1}{(4.250,1.559)}
\gppoint{gp mark 1}{(4.252,1.560)}
\gppoint{gp mark 1}{(4.255,1.554)}
\gppoint{gp mark 1}{(4.257,1.564)}
\gppoint{gp mark 1}{(4.259,1.571)}
\gppoint{gp mark 1}{(4.261,1.561)}
\gppoint{gp mark 1}{(4.264,1.544)}
\gppoint{gp mark 1}{(4.266,1.566)}
\gppoint{gp mark 1}{(4.268,1.557)}
\gppoint{gp mark 1}{(4.270,1.566)}
\gppoint{gp mark 1}{(4.273,1.540)}
\gppoint{gp mark 1}{(4.275,1.544)}
\gppoint{gp mark 1}{(4.277,1.562)}
\gppoint{gp mark 1}{(4.279,1.564)}
\gppoint{gp mark 1}{(4.282,1.579)}
\gppoint{gp mark 1}{(4.284,1.552)}
\gppoint{gp mark 1}{(4.286,1.559)}
\gppoint{gp mark 1}{(4.288,1.585)}
\gppoint{gp mark 1}{(4.291,1.580)}
\gppoint{gp mark 1}{(4.293,1.587)}
\gppoint{gp mark 1}{(4.295,1.591)}
\gppoint{gp mark 1}{(4.297,1.593)}
\gppoint{gp mark 1}{(4.300,1.591)}
\gppoint{gp mark 1}{(4.302,1.611)}
\gppoint{gp mark 1}{(4.304,1.595)}
\gppoint{gp mark 1}{(4.306,1.600)}
\gppoint{gp mark 1}{(4.309,1.596)}
\gppoint{gp mark 1}{(4.311,1.580)}
\gppoint{gp mark 1}{(4.313,1.593)}
\gppoint{gp mark 1}{(4.315,1.600)}
\gppoint{gp mark 1}{(4.318,1.597)}
\gppoint{gp mark 1}{(4.320,1.607)}
\gppoint{gp mark 1}{(4.322,1.607)}
\gppoint{gp mark 1}{(4.324,1.614)}
\gppoint{gp mark 1}{(4.327,1.611)}
\gppoint{gp mark 1}{(4.329,1.632)}
\gppoint{gp mark 1}{(4.331,1.613)}
\gppoint{gp mark 1}{(4.333,1.614)}
\gppoint{gp mark 1}{(4.336,1.643)}
\gppoint{gp mark 1}{(4.338,1.615)}
\gppoint{gp mark 1}{(4.340,1.629)}
\gppoint{gp mark 1}{(4.342,1.624)}
\gppoint{gp mark 1}{(4.345,1.641)}
\gppoint{gp mark 1}{(4.347,1.643)}
\gppoint{gp mark 1}{(4.349,1.633)}
\gppoint{gp mark 1}{(4.351,1.636)}
\gppoint{gp mark 1}{(4.354,1.635)}
\gppoint{gp mark 1}{(4.356,1.647)}
\gppoint{gp mark 1}{(4.358,1.650)}
\gppoint{gp mark 1}{(4.360,1.652)}
\gppoint{gp mark 1}{(4.363,1.637)}
\gppoint{gp mark 1}{(4.365,1.632)}
\gppoint{gp mark 1}{(4.367,1.651)}
\gppoint{gp mark 1}{(4.369,1.653)}
\gppoint{gp mark 1}{(4.372,1.659)}
\gppoint{gp mark 1}{(4.374,1.665)}
\gppoint{gp mark 1}{(4.376,1.658)}
\gppoint{gp mark 1}{(4.378,1.670)}
\gppoint{gp mark 1}{(4.381,1.642)}
\gppoint{gp mark 1}{(4.383,1.655)}
\gppoint{gp mark 1}{(4.385,1.687)}
\gppoint{gp mark 1}{(4.387,1.685)}
\gppoint{gp mark 1}{(4.390,1.685)}
\gppoint{gp mark 1}{(4.392,1.689)}
\gppoint{gp mark 1}{(4.394,1.697)}
\gppoint{gp mark 1}{(4.396,1.670)}
\gppoint{gp mark 1}{(4.399,1.686)}
\gppoint{gp mark 1}{(4.401,1.696)}
\gppoint{gp mark 1}{(4.403,1.687)}
\gppoint{gp mark 1}{(4.405,1.687)}
\gppoint{gp mark 1}{(4.408,1.697)}
\gppoint{gp mark 1}{(4.410,1.696)}
\gppoint{gp mark 1}{(4.412,1.684)}
\gppoint{gp mark 1}{(4.414,1.701)}
\gppoint{gp mark 1}{(4.417,1.694)}
\gppoint{gp mark 1}{(4.419,1.701)}
\gppoint{gp mark 1}{(4.421,1.686)}
\gppoint{gp mark 1}{(4.423,1.729)}
\gppoint{gp mark 1}{(4.426,1.721)}
\gppoint{gp mark 1}{(4.428,1.719)}
\gppoint{gp mark 1}{(4.430,1.719)}
\gppoint{gp mark 1}{(4.432,1.724)}
\gppoint{gp mark 1}{(4.435,1.725)}
\gppoint{gp mark 1}{(4.437,1.720)}
\gppoint{gp mark 1}{(4.439,1.762)}
\gppoint{gp mark 1}{(4.441,1.733)}
\gppoint{gp mark 1}{(4.444,1.727)}
\gppoint{gp mark 1}{(4.446,1.755)}
\gppoint{gp mark 1}{(4.448,1.759)}
\gppoint{gp mark 1}{(4.450,1.754)}
\gppoint{gp mark 1}{(4.453,1.752)}
\gppoint{gp mark 1}{(4.455,1.757)}
\gppoint{gp mark 1}{(4.457,1.747)}
\gppoint{gp mark 1}{(4.459,1.771)}
\gppoint{gp mark 1}{(4.462,1.753)}
\gppoint{gp mark 1}{(4.464,1.773)}
\gppoint{gp mark 1}{(4.466,1.784)}
\gppoint{gp mark 1}{(4.468,1.763)}
\gppoint{gp mark 1}{(4.471,1.782)}
\gppoint{gp mark 1}{(4.473,1.767)}
\gppoint{gp mark 1}{(4.475,1.783)}
\gppoint{gp mark 1}{(4.477,1.792)}
\gppoint{gp mark 1}{(4.480,1.777)}
\gppoint{gp mark 1}{(4.482,1.780)}
\gppoint{gp mark 1}{(4.484,1.793)}
\gppoint{gp mark 1}{(4.486,1.808)}
\gppoint{gp mark 1}{(4.489,1.812)}
\gppoint{gp mark 1}{(4.491,1.814)}
\gppoint{gp mark 1}{(4.493,1.828)}
\gppoint{gp mark 1}{(4.495,1.824)}
\gppoint{gp mark 1}{(4.498,1.820)}
\gppoint{gp mark 1}{(4.500,1.817)}
\gppoint{gp mark 1}{(4.502,1.827)}
\gppoint{gp mark 1}{(4.504,1.842)}
\gppoint{gp mark 1}{(4.507,1.820)}
\gppoint{gp mark 1}{(4.509,1.835)}
\gppoint{gp mark 1}{(4.511,1.827)}
\gppoint{gp mark 1}{(4.513,1.837)}
\gppoint{gp mark 1}{(4.516,1.849)}
\gppoint{gp mark 1}{(4.518,1.862)}
\gppoint{gp mark 1}{(4.520,1.853)}
\gppoint{gp mark 1}{(4.522,1.863)}
\gppoint{gp mark 1}{(4.525,1.846)}
\gppoint{gp mark 1}{(4.527,1.848)}
\gppoint{gp mark 1}{(4.529,1.860)}
\gppoint{gp mark 1}{(4.531,1.859)}
\gppoint{gp mark 1}{(4.534,1.881)}
\gppoint{gp mark 1}{(4.536,1.878)}
\gppoint{gp mark 1}{(4.538,1.877)}
\gppoint{gp mark 1}{(4.540,1.886)}
\gppoint{gp mark 1}{(4.543,1.878)}
\gppoint{gp mark 1}{(4.545,1.872)}
\gppoint{gp mark 1}{(4.547,1.882)}
\gppoint{gp mark 1}{(4.549,1.878)}
\gppoint{gp mark 1}{(4.552,1.905)}
\gppoint{gp mark 1}{(4.554,1.902)}
\gppoint{gp mark 1}{(4.556,1.915)}
\gppoint{gp mark 1}{(4.558,1.920)}
\gppoint{gp mark 1}{(4.561,1.910)}
\gppoint{gp mark 1}{(4.563,1.920)}
\gppoint{gp mark 1}{(4.565,1.906)}
\gppoint{gp mark 1}{(4.567,1.926)}
\gppoint{gp mark 1}{(4.570,1.932)}
\gppoint{gp mark 1}{(4.572,1.918)}
\gppoint{gp mark 1}{(4.574,1.939)}
\gppoint{gp mark 1}{(4.576,1.939)}
\gppoint{gp mark 1}{(4.579,1.952)}
\gppoint{gp mark 1}{(4.581,1.936)}
\gppoint{gp mark 1}{(4.583,1.945)}
\gppoint{gp mark 1}{(4.585,1.944)}
\gppoint{gp mark 1}{(4.588,1.967)}
\gppoint{gp mark 1}{(4.590,1.977)}
\gppoint{gp mark 1}{(4.592,1.961)}
\gppoint{gp mark 1}{(4.594,1.978)}
\gppoint{gp mark 1}{(4.597,1.975)}
\gppoint{gp mark 1}{(4.599,1.985)}
\gppoint{gp mark 1}{(4.601,1.976)}
\gppoint{gp mark 1}{(4.603,1.978)}
\gppoint{gp mark 1}{(4.606,1.981)}
\gppoint{gp mark 1}{(4.608,1.971)}
\gppoint{gp mark 1}{(4.610,1.976)}
\gppoint{gp mark 1}{(4.612,1.987)}
\gppoint{gp mark 1}{(4.615,2.018)}
\gppoint{gp mark 1}{(4.617,1.986)}
\gppoint{gp mark 1}{(4.619,2.010)}
\gppoint{gp mark 1}{(4.621,2.004)}
\gppoint{gp mark 1}{(4.624,2.021)}
\gppoint{gp mark 1}{(4.626,2.027)}
\gppoint{gp mark 1}{(4.628,2.006)}
\gppoint{gp mark 1}{(4.630,2.035)}
\gppoint{gp mark 1}{(4.633,2.022)}
\gppoint{gp mark 1}{(4.635,2.021)}
\gppoint{gp mark 1}{(4.637,2.052)}
\gppoint{gp mark 1}{(4.639,2.026)}
\gppoint{gp mark 1}{(4.642,2.026)}
\gppoint{gp mark 1}{(4.644,2.048)}
\gppoint{gp mark 1}{(4.646,2.047)}
\gppoint{gp mark 1}{(4.648,2.043)}
\gppoint{gp mark 1}{(4.651,2.056)}
\gppoint{gp mark 1}{(4.653,2.060)}
\gppoint{gp mark 1}{(4.655,2.052)}
\gppoint{gp mark 1}{(4.657,2.059)}
\gppoint{gp mark 1}{(4.660,2.083)}
\gppoint{gp mark 1}{(4.662,2.074)}
\gppoint{gp mark 1}{(4.664,2.083)}
\gppoint{gp mark 1}{(4.666,2.097)}
\gppoint{gp mark 1}{(4.669,2.109)}
\gppoint{gp mark 1}{(4.671,2.091)}
\gppoint{gp mark 1}{(4.673,2.091)}
\gppoint{gp mark 1}{(4.675,2.105)}
\gppoint{gp mark 1}{(4.678,2.105)}
\gppoint{gp mark 1}{(4.680,2.124)}
\gppoint{gp mark 1}{(4.682,2.125)}
\gppoint{gp mark 1}{(4.684,2.143)}
\gppoint{gp mark 1}{(4.687,2.126)}
\gppoint{gp mark 1}{(4.689,2.136)}
\gppoint{gp mark 1}{(4.691,2.155)}
\gppoint{gp mark 1}{(4.693,2.141)}
\gppoint{gp mark 1}{(4.696,2.164)}
\gppoint{gp mark 1}{(4.698,2.170)}
\gppoint{gp mark 1}{(4.700,2.171)}
\gppoint{gp mark 1}{(4.702,2.168)}
\gppoint{gp mark 1}{(4.705,2.158)}
\gppoint{gp mark 1}{(4.707,2.184)}
\gppoint{gp mark 1}{(4.709,2.186)}
\gppoint{gp mark 1}{(4.711,2.193)}
\gppoint{gp mark 1}{(4.714,2.203)}
\gppoint{gp mark 1}{(4.716,2.187)}
\gppoint{gp mark 1}{(4.718,2.201)}
\gppoint{gp mark 1}{(4.720,2.213)}
\gppoint{gp mark 1}{(4.723,2.211)}
\gppoint{gp mark 1}{(4.725,2.213)}
\gppoint{gp mark 1}{(4.727,2.223)}
\gppoint{gp mark 1}{(4.729,2.221)}
\gppoint{gp mark 1}{(4.732,2.222)}
\gppoint{gp mark 1}{(4.734,2.225)}
\gppoint{gp mark 1}{(4.736,2.247)}
\gppoint{gp mark 1}{(4.738,2.242)}
\gppoint{gp mark 1}{(4.741,2.244)}
\gppoint{gp mark 1}{(4.743,2.248)}
\gppoint{gp mark 1}{(4.745,2.268)}
\gppoint{gp mark 1}{(4.747,2.255)}
\gppoint{gp mark 1}{(4.750,2.250)}
\gppoint{gp mark 1}{(4.752,2.261)}
\gppoint{gp mark 1}{(4.754,2.266)}
\gppoint{gp mark 1}{(4.756,2.298)}
\gppoint{gp mark 1}{(4.759,2.305)}
\gppoint{gp mark 1}{(4.761,2.301)}
\gppoint{gp mark 1}{(4.763,2.272)}
\gppoint{gp mark 1}{(4.765,2.290)}
\gppoint{gp mark 1}{(4.768,2.296)}
\gppoint{gp mark 1}{(4.770,2.315)}
\gppoint{gp mark 1}{(4.772,2.329)}
\gppoint{gp mark 1}{(4.774,2.313)}
\gppoint{gp mark 1}{(4.777,2.327)}
\gppoint{gp mark 1}{(4.779,2.317)}
\gppoint{gp mark 1}{(4.781,2.336)}
\gppoint{gp mark 1}{(4.783,2.373)}
\gppoint{gp mark 1}{(4.786,2.345)}
\gppoint{gp mark 1}{(4.788,2.364)}
\gppoint{gp mark 1}{(4.790,2.362)}
\gppoint{gp mark 1}{(4.792,2.369)}
\gppoint{gp mark 1}{(4.795,2.379)}
\gppoint{gp mark 1}{(4.797,2.378)}
\gppoint{gp mark 1}{(4.799,2.381)}
\gppoint{gp mark 1}{(4.801,2.376)}
\gppoint{gp mark 1}{(4.804,2.396)}
\gppoint{gp mark 1}{(4.806,2.394)}
\gppoint{gp mark 1}{(4.808,2.395)}
\gppoint{gp mark 1}{(4.810,2.420)}
\gppoint{gp mark 1}{(4.813,2.405)}
\gppoint{gp mark 1}{(4.815,2.425)}
\gppoint{gp mark 1}{(4.817,2.418)}
\gppoint{gp mark 1}{(4.819,2.443)}
\gppoint{gp mark 1}{(4.822,2.441)}
\gppoint{gp mark 1}{(4.824,2.440)}
\gppoint{gp mark 1}{(4.826,2.451)}
\gppoint{gp mark 1}{(4.828,2.474)}
\gppoint{gp mark 1}{(4.831,2.484)}
\gppoint{gp mark 1}{(4.833,2.487)}
\gppoint{gp mark 1}{(4.835,2.487)}
\gppoint{gp mark 1}{(4.837,2.463)}
\gppoint{gp mark 1}{(4.840,2.491)}
\gppoint{gp mark 1}{(4.842,2.477)}
\gppoint{gp mark 1}{(4.844,2.494)}
\gppoint{gp mark 1}{(4.846,2.504)}
\gppoint{gp mark 1}{(4.849,2.494)}
\gppoint{gp mark 1}{(4.851,2.497)}
\gppoint{gp mark 1}{(4.853,2.491)}
\gppoint{gp mark 1}{(4.855,2.506)}
\gppoint{gp mark 1}{(4.858,2.519)}
\gppoint{gp mark 1}{(4.860,2.533)}
\gppoint{gp mark 1}{(4.862,2.525)}
\gppoint{gp mark 1}{(4.864,2.523)}
\gppoint{gp mark 1}{(4.867,2.542)}
\gppoint{gp mark 1}{(4.869,2.546)}
\gppoint{gp mark 1}{(4.871,2.552)}
\gppoint{gp mark 1}{(4.873,2.588)}
\gppoint{gp mark 1}{(4.876,2.565)}
\gppoint{gp mark 1}{(4.878,2.581)}
\gppoint{gp mark 1}{(4.880,2.585)}
\gppoint{gp mark 1}{(4.882,2.560)}
\gppoint{gp mark 1}{(4.885,2.583)}
\gppoint{gp mark 1}{(4.887,2.592)}
\gppoint{gp mark 1}{(4.889,2.592)}
\gppoint{gp mark 1}{(4.891,2.595)}
\gppoint{gp mark 1}{(4.894,2.604)}
\gppoint{gp mark 1}{(4.896,2.612)}
\gppoint{gp mark 1}{(4.898,2.617)}
\gppoint{gp mark 1}{(4.900,2.605)}
\gppoint{gp mark 1}{(4.903,2.602)}
\gppoint{gp mark 1}{(4.905,2.629)}
\gppoint{gp mark 1}{(4.907,2.631)}
\gppoint{gp mark 1}{(4.909,2.644)}
\gppoint{gp mark 1}{(4.912,2.640)}
\gppoint{gp mark 1}{(4.914,2.645)}
\gppoint{gp mark 1}{(4.916,2.673)}
\gppoint{gp mark 1}{(4.918,2.653)}
\gppoint{gp mark 1}{(4.921,2.670)}
\gppoint{gp mark 1}{(4.923,2.667)}
\gppoint{gp mark 1}{(4.925,2.687)}
\gppoint{gp mark 1}{(4.927,2.685)}
\gppoint{gp mark 1}{(4.930,2.698)}
\gppoint{gp mark 1}{(4.932,2.680)}
\gppoint{gp mark 1}{(4.934,2.701)}
\gppoint{gp mark 1}{(4.936,2.699)}
\gppoint{gp mark 1}{(4.939,2.689)}
\gppoint{gp mark 1}{(4.941,2.704)}
\gppoint{gp mark 1}{(4.943,2.693)}
\gppoint{gp mark 1}{(4.945,2.712)}
\gppoint{gp mark 1}{(4.948,2.736)}
\gppoint{gp mark 1}{(4.950,2.732)}
\gppoint{gp mark 1}{(4.952,2.756)}
\gppoint{gp mark 1}{(4.954,2.759)}
\gppoint{gp mark 1}{(4.957,2.751)}
\gppoint{gp mark 1}{(4.959,2.759)}
\gppoint{gp mark 1}{(4.961,2.765)}
\gppoint{gp mark 1}{(4.963,2.796)}
\gppoint{gp mark 1}{(4.966,2.776)}
\gppoint{gp mark 1}{(4.968,2.792)}
\gppoint{gp mark 1}{(4.970,2.805)}
\gppoint{gp mark 1}{(4.972,2.778)}
\gppoint{gp mark 1}{(4.975,2.805)}
\gppoint{gp mark 1}{(4.977,2.792)}
\gppoint{gp mark 1}{(4.979,2.828)}
\gppoint{gp mark 1}{(4.981,2.804)}
\gppoint{gp mark 1}{(4.984,2.837)}
\gppoint{gp mark 1}{(4.986,2.818)}
\gppoint{gp mark 1}{(4.988,2.841)}
\gppoint{gp mark 1}{(4.990,2.824)}
\gppoint{gp mark 1}{(4.993,2.867)}
\gppoint{gp mark 1}{(4.995,2.862)}
\gppoint{gp mark 1}{(4.997,2.874)}
\gppoint{gp mark 1}{(4.999,2.865)}
\gppoint{gp mark 1}{(5.002,2.880)}
\gppoint{gp mark 1}{(5.004,2.881)}
\gppoint{gp mark 1}{(5.006,2.897)}
\gppoint{gp mark 1}{(5.008,2.897)}
\gppoint{gp mark 1}{(5.011,2.869)}
\gppoint{gp mark 1}{(5.013,2.881)}
\gppoint{gp mark 1}{(5.015,2.925)}
\gppoint{gp mark 1}{(5.017,2.928)}
\gppoint{gp mark 1}{(5.020,2.941)}
\gppoint{gp mark 1}{(5.022,2.929)}
\gppoint{gp mark 1}{(5.024,2.946)}
\gppoint{gp mark 1}{(5.026,2.946)}
\gppoint{gp mark 1}{(5.029,2.921)}
\gppoint{gp mark 1}{(5.031,2.953)}
\gppoint{gp mark 1}{(5.033,2.971)}
\gppoint{gp mark 1}{(5.035,2.969)}
\gppoint{gp mark 1}{(5.038,2.982)}
\gppoint{gp mark 1}{(5.040,2.989)}
\gppoint{gp mark 1}{(5.042,2.990)}
\gppoint{gp mark 1}{(5.044,2.991)}
\gppoint{gp mark 1}{(5.047,2.996)}
\gppoint{gp mark 1}{(5.049,3.019)}
\gppoint{gp mark 1}{(5.051,3.014)}
\gppoint{gp mark 1}{(5.053,3.007)}
\gppoint{gp mark 1}{(5.056,3.024)}
\gppoint{gp mark 1}{(5.058,3.030)}
\gppoint{gp mark 1}{(5.060,3.064)}
\gppoint{gp mark 1}{(5.062,3.057)}
\gppoint{gp mark 1}{(5.065,3.075)}
\gppoint{gp mark 1}{(5.067,3.060)}
\gppoint{gp mark 1}{(5.069,3.035)}
\gppoint{gp mark 1}{(5.071,3.057)}
\gppoint{gp mark 1}{(5.074,3.076)}
\gppoint{gp mark 1}{(5.076,3.076)}
\gppoint{gp mark 1}{(5.078,3.100)}
\gppoint{gp mark 1}{(5.080,3.073)}
\gppoint{gp mark 1}{(5.083,3.096)}
\gppoint{gp mark 1}{(5.085,3.092)}
\gppoint{gp mark 1}{(5.087,3.096)}
\gppoint{gp mark 1}{(5.089,3.122)}
\gppoint{gp mark 1}{(5.092,3.117)}
\gppoint{gp mark 1}{(5.094,3.127)}
\gppoint{gp mark 1}{(5.096,3.128)}
\gppoint{gp mark 1}{(5.098,3.133)}
\gppoint{gp mark 1}{(5.101,3.175)}
\gppoint{gp mark 1}{(5.103,3.142)}
\gppoint{gp mark 1}{(5.105,3.164)}
\gppoint{gp mark 1}{(5.107,3.153)}
\gppoint{gp mark 1}{(5.110,3.145)}
\gppoint{gp mark 1}{(5.112,3.172)}
\gppoint{gp mark 1}{(5.114,3.172)}
\gppoint{gp mark 1}{(5.116,3.178)}
\gppoint{gp mark 1}{(5.119,3.182)}
\gppoint{gp mark 1}{(5.121,3.205)}
\gppoint{gp mark 1}{(5.123,3.207)}
\gppoint{gp mark 1}{(5.126,3.189)}
\gppoint{gp mark 1}{(5.128,3.204)}
\gppoint{gp mark 1}{(5.130,3.230)}
\gppoint{gp mark 1}{(5.132,3.238)}
\gppoint{gp mark 1}{(5.135,3.250)}
\gppoint{gp mark 1}{(5.137,3.262)}
\gppoint{gp mark 1}{(5.139,3.273)}
\gppoint{gp mark 1}{(5.141,3.243)}
\gppoint{gp mark 1}{(5.144,3.266)}
\gppoint{gp mark 1}{(5.146,3.278)}
\gppoint{gp mark 1}{(5.148,3.294)}
\gppoint{gp mark 1}{(5.150,3.281)}
\gppoint{gp mark 1}{(5.153,3.301)}
\gppoint{gp mark 1}{(5.155,3.320)}
\gppoint{gp mark 1}{(5.157,3.313)}
\gppoint{gp mark 1}{(5.159,3.334)}
\gppoint{gp mark 1}{(5.162,3.325)}
\gppoint{gp mark 1}{(5.164,3.321)}
\gppoint{gp mark 1}{(5.166,3.328)}
\gppoint{gp mark 1}{(5.168,3.359)}
\gppoint{gp mark 1}{(5.171,3.349)}
\gppoint{gp mark 1}{(5.173,3.370)}
\gppoint{gp mark 1}{(5.175,3.391)}
\gppoint{gp mark 1}{(5.177,3.375)}
\gppoint{gp mark 1}{(5.180,3.388)}
\gppoint{gp mark 1}{(5.182,3.372)}
\gppoint{gp mark 1}{(5.184,3.395)}
\gppoint{gp mark 1}{(5.186,3.390)}
\gppoint{gp mark 1}{(5.189,3.405)}
\gppoint{gp mark 1}{(5.191,3.387)}
\gppoint{gp mark 1}{(5.193,3.413)}
\gppoint{gp mark 1}{(5.195,3.415)}
\gppoint{gp mark 1}{(5.198,3.429)}
\gppoint{gp mark 1}{(5.200,3.426)}
\gppoint{gp mark 1}{(5.202,3.443)}
\gppoint{gp mark 1}{(5.204,3.436)}
\gppoint{gp mark 1}{(5.207,3.449)}
\gppoint{gp mark 1}{(5.209,3.451)}
\gppoint{gp mark 1}{(5.211,3.472)}
\gppoint{gp mark 1}{(5.213,3.471)}
\gppoint{gp mark 1}{(5.216,3.494)}
\gppoint{gp mark 1}{(5.218,3.498)}
\gppoint{gp mark 1}{(5.220,3.506)}
\gppoint{gp mark 1}{(5.222,3.501)}
\gppoint{gp mark 1}{(5.225,3.509)}
\gppoint{gp mark 1}{(5.227,3.514)}
\gppoint{gp mark 1}{(5.229,3.539)}
\gppoint{gp mark 1}{(5.231,3.520)}
\gppoint{gp mark 1}{(5.234,3.519)}
\gppoint{gp mark 1}{(5.236,3.540)}
\gppoint{gp mark 1}{(5.238,3.520)}
\gppoint{gp mark 1}{(5.240,3.544)}
\gppoint{gp mark 1}{(5.243,3.567)}
\gppoint{gp mark 1}{(5.245,3.599)}
\gppoint{gp mark 1}{(5.247,3.575)}
\gppoint{gp mark 1}{(5.249,3.583)}
\gppoint{gp mark 1}{(5.252,3.571)}
\gppoint{gp mark 1}{(5.254,3.580)}
\gppoint{gp mark 1}{(5.256,3.598)}
\gppoint{gp mark 1}{(5.258,3.586)}
\gppoint{gp mark 1}{(5.261,3.614)}
\gppoint{gp mark 1}{(5.263,3.596)}
\gppoint{gp mark 1}{(5.265,3.609)}
\gppoint{gp mark 1}{(5.267,3.634)}
\gppoint{gp mark 1}{(5.270,3.634)}
\gppoint{gp mark 1}{(5.272,3.664)}
\gppoint{gp mark 1}{(5.274,3.642)}
\gppoint{gp mark 1}{(5.276,3.639)}
\gppoint{gp mark 1}{(5.279,3.635)}
\gppoint{gp mark 1}{(5.281,3.648)}
\gppoint{gp mark 1}{(5.283,3.672)}
\gppoint{gp mark 1}{(5.285,3.673)}
\gppoint{gp mark 1}{(5.288,3.672)}
\gppoint{gp mark 1}{(5.290,3.675)}
\gppoint{gp mark 1}{(5.292,3.709)}
\gppoint{gp mark 1}{(5.294,3.699)}
\gppoint{gp mark 1}{(5.297,3.694)}
\gppoint{gp mark 1}{(5.299,3.726)}
\gppoint{gp mark 1}{(5.301,3.710)}
\gppoint{gp mark 1}{(5.303,3.703)}
\gppoint{gp mark 1}{(5.306,3.721)}
\gppoint{gp mark 1}{(5.308,3.726)}
\gppoint{gp mark 1}{(5.310,3.725)}
\gppoint{gp mark 1}{(5.312,3.743)}
\gppoint{gp mark 1}{(5.315,3.753)}
\gppoint{gp mark 1}{(5.317,3.743)}
\gppoint{gp mark 1}{(5.319,3.765)}
\gppoint{gp mark 1}{(5.321,3.791)}
\gppoint{gp mark 1}{(5.324,3.787)}
\gppoint{gp mark 1}{(5.326,3.783)}
\gppoint{gp mark 1}{(5.328,3.801)}
\gppoint{gp mark 1}{(5.330,3.807)}
\gppoint{gp mark 1}{(5.333,3.809)}
\gppoint{gp mark 1}{(5.335,3.835)}
\gppoint{gp mark 1}{(5.337,3.832)}
\gppoint{gp mark 1}{(5.339,3.831)}
\gppoint{gp mark 1}{(5.342,3.856)}
\gppoint{gp mark 1}{(5.344,3.844)}
\gppoint{gp mark 1}{(5.346,3.844)}
\gppoint{gp mark 1}{(5.348,3.859)}
\gppoint{gp mark 1}{(5.351,3.859)}
\gppoint{gp mark 1}{(5.353,3.853)}
\gppoint{gp mark 1}{(5.355,3.875)}
\gppoint{gp mark 1}{(5.357,3.862)}
\gppoint{gp mark 1}{(5.360,3.885)}
\gppoint{gp mark 1}{(5.362,3.912)}
\gppoint{gp mark 1}{(5.364,3.864)}
\gppoint{gp mark 1}{(5.366,3.889)}
\gppoint{gp mark 1}{(5.369,3.905)}
\gppoint{gp mark 1}{(5.371,3.903)}
\gppoint{gp mark 1}{(5.373,3.892)}
\gppoint{gp mark 1}{(5.375,3.900)}
\gppoint{gp mark 1}{(5.378,3.922)}
\gppoint{gp mark 1}{(5.380,3.928)}
\gppoint{gp mark 1}{(5.382,3.940)}
\gppoint{gp mark 1}{(5.384,3.951)}
\gppoint{gp mark 1}{(5.387,3.936)}
\gppoint{gp mark 1}{(5.389,3.929)}
\gppoint{gp mark 1}{(5.391,3.925)}
\gppoint{gp mark 1}{(5.393,3.992)}
\gppoint{gp mark 1}{(5.396,3.974)}
\gppoint{gp mark 1}{(5.398,3.988)}
\gppoint{gp mark 1}{(5.400,4.018)}
\gppoint{gp mark 1}{(5.402,3.996)}
\gppoint{gp mark 1}{(5.405,4.001)}
\gppoint{gp mark 1}{(5.407,3.998)}
\gppoint{gp mark 1}{(5.409,4.019)}
\gppoint{gp mark 1}{(5.411,4.028)}
\gppoint{gp mark 1}{(5.414,4.014)}
\gppoint{gp mark 1}{(5.416,4.017)}
\gppoint{gp mark 1}{(5.418,4.050)}
\gppoint{gp mark 1}{(5.420,4.058)}
\gppoint{gp mark 1}{(5.423,4.050)}
\gppoint{gp mark 1}{(5.425,4.062)}
\gppoint{gp mark 1}{(5.427,4.082)}
\gppoint{gp mark 1}{(5.429,4.062)}
\gppoint{gp mark 1}{(5.432,4.085)}
\gppoint{gp mark 1}{(5.434,4.119)}
\gppoint{gp mark 1}{(5.436,4.095)}
\gppoint{gp mark 1}{(5.438,4.101)}
\gppoint{gp mark 1}{(5.441,4.109)}
\gppoint{gp mark 1}{(5.443,4.097)}
\gppoint{gp mark 1}{(5.445,4.137)}
\gppoint{gp mark 1}{(5.447,4.135)}
\gppoint{gp mark 1}{(5.450,4.135)}
\gppoint{gp mark 1}{(5.452,4.156)}
\gppoint{gp mark 1}{(5.454,4.137)}
\gppoint{gp mark 1}{(5.456,4.162)}
\gppoint{gp mark 1}{(5.459,4.139)}
\gppoint{gp mark 1}{(5.461,4.159)}
\gppoint{gp mark 1}{(5.463,4.183)}
\gppoint{gp mark 1}{(5.465,4.169)}
\gppoint{gp mark 1}{(5.468,4.189)}
\gppoint{gp mark 1}{(5.470,4.195)}
\gppoint{gp mark 1}{(5.472,4.214)}
\gppoint{gp mark 1}{(5.474,4.199)}
\gppoint{gp mark 1}{(5.477,4.218)}
\gppoint{gp mark 1}{(5.479,4.201)}
\gppoint{gp mark 1}{(5.481,4.218)}
\gppoint{gp mark 1}{(5.483,4.214)}
\gppoint{gp mark 1}{(5.486,4.219)}
\gppoint{gp mark 1}{(5.488,4.223)}
\gppoint{gp mark 1}{(5.490,4.246)}
\gppoint{gp mark 1}{(5.492,4.234)}
\gppoint{gp mark 1}{(5.495,4.258)}
\gppoint{gp mark 1}{(5.497,4.230)}
\gppoint{gp mark 1}{(5.499,4.280)}
\gppoint{gp mark 1}{(5.501,4.279)}
\gppoint{gp mark 1}{(5.504,4.284)}
\gppoint{gp mark 1}{(5.506,4.269)}
\gppoint{gp mark 1}{(5.508,4.303)}
\gppoint{gp mark 1}{(5.510,4.312)}
\gppoint{gp mark 1}{(5.513,4.336)}
\gppoint{gp mark 1}{(5.515,4.315)}
\gppoint{gp mark 1}{(5.517,4.333)}
\gppoint{gp mark 1}{(5.519,4.313)}
\gppoint{gp mark 1}{(5.522,4.318)}
\gppoint{gp mark 1}{(5.524,4.321)}
\gppoint{gp mark 1}{(5.526,4.341)}
\gppoint{gp mark 1}{(5.528,4.335)}
\gppoint{gp mark 1}{(5.531,4.333)}
\gppoint{gp mark 1}{(5.533,4.331)}
\gppoint{gp mark 1}{(5.535,4.345)}
\gppoint{gp mark 1}{(5.537,4.350)}
\gppoint{gp mark 1}{(5.540,4.329)}
\gppoint{gp mark 1}{(5.542,4.348)}
\gppoint{gp mark 1}{(5.544,4.382)}
\gppoint{gp mark 1}{(5.546,4.362)}
\gppoint{gp mark 1}{(5.549,4.362)}
\gppoint{gp mark 1}{(5.551,4.382)}
\gppoint{gp mark 1}{(5.553,4.415)}
\gppoint{gp mark 1}{(5.555,4.402)}
\gppoint{gp mark 1}{(5.558,4.402)}
\gppoint{gp mark 1}{(5.560,4.408)}
\gppoint{gp mark 1}{(5.562,4.413)}
\gppoint{gp mark 1}{(5.564,4.433)}
\gppoint{gp mark 1}{(5.567,4.423)}
\gppoint{gp mark 1}{(5.569,4.442)}
\gppoint{gp mark 1}{(5.571,4.449)}
\gppoint{gp mark 1}{(5.573,4.457)}
\gppoint{gp mark 1}{(5.576,4.428)}
\gppoint{gp mark 1}{(5.578,4.480)}
\gppoint{gp mark 1}{(5.580,4.439)}
\gppoint{gp mark 1}{(5.582,4.442)}
\gppoint{gp mark 1}{(5.585,4.449)}
\gppoint{gp mark 1}{(5.587,4.428)}
\gppoint{gp mark 1}{(5.589,4.454)}
\gppoint{gp mark 1}{(5.591,4.462)}
\gppoint{gp mark 1}{(5.594,4.452)}
\gppoint{gp mark 1}{(5.596,4.482)}
\gppoint{gp mark 1}{(5.598,4.513)}
\gppoint{gp mark 1}{(5.600,4.500)}
\gppoint{gp mark 1}{(5.603,4.547)}
\gppoint{gp mark 1}{(5.605,4.537)}
\gppoint{gp mark 1}{(5.607,4.546)}
\gppoint{gp mark 1}{(5.609,4.536)}
\gppoint{gp mark 1}{(5.612,4.505)}
\gppoint{gp mark 1}{(5.614,4.532)}
\gppoint{gp mark 1}{(5.616,4.562)}
\gppoint{gp mark 1}{(5.618,4.540)}
\gppoint{gp mark 1}{(5.621,4.561)}
\gppoint{gp mark 1}{(5.623,4.542)}
\gppoint{gp mark 1}{(5.625,4.601)}
\gppoint{gp mark 1}{(5.627,4.591)}
\gppoint{gp mark 1}{(5.630,4.590)}
\gppoint{gp mark 1}{(5.632,4.589)}
\gppoint{gp mark 1}{(5.634,4.612)}
\gppoint{gp mark 1}{(5.636,4.573)}
\gppoint{gp mark 1}{(5.639,4.601)}
\gppoint{gp mark 1}{(5.641,4.588)}
\gppoint{gp mark 1}{(5.643,4.569)}
\gppoint{gp mark 1}{(5.645,4.627)}
\gppoint{gp mark 1}{(5.648,4.635)}
\gppoint{gp mark 1}{(5.650,4.629)}
\gppoint{gp mark 1}{(5.652,4.609)}
\gppoint{gp mark 1}{(5.654,4.647)}
\gppoint{gp mark 1}{(5.657,4.600)}
\gppoint{gp mark 1}{(5.659,4.636)}
\gppoint{gp mark 1}{(5.661,4.621)}
\gppoint{gp mark 1}{(5.663,4.673)}
\gppoint{gp mark 1}{(5.666,4.680)}
\gppoint{gp mark 1}{(5.668,4.661)}
\gppoint{gp mark 1}{(5.670,4.657)}
\gppoint{gp mark 1}{(5.672,4.690)}
\gppoint{gp mark 1}{(5.675,4.675)}
\gppoint{gp mark 1}{(5.677,4.694)}
\gppoint{gp mark 1}{(5.679,4.686)}
\gppoint{gp mark 1}{(5.681,4.700)}
\gppoint{gp mark 1}{(5.684,4.696)}
\gppoint{gp mark 1}{(5.686,4.692)}
\gppoint{gp mark 1}{(5.688,4.743)}
\gppoint{gp mark 1}{(5.690,4.729)}
\gppoint{gp mark 1}{(5.693,4.713)}
\gppoint{gp mark 1}{(5.695,4.720)}
\gppoint{gp mark 1}{(5.697,4.735)}
\gppoint{gp mark 1}{(5.699,4.731)}
\gppoint{gp mark 1}{(5.702,4.727)}
\gppoint{gp mark 1}{(5.704,4.747)}
\gppoint{gp mark 1}{(5.706,4.771)}
\gppoint{gp mark 1}{(5.708,4.760)}
\gppoint{gp mark 1}{(5.711,4.795)}
\gppoint{gp mark 1}{(5.713,4.788)}
\gppoint{gp mark 1}{(5.715,4.763)}
\gppoint{gp mark 1}{(5.717,4.763)}
\gppoint{gp mark 1}{(5.720,4.765)}
\gppoint{gp mark 1}{(5.722,4.754)}
\gppoint{gp mark 1}{(5.724,4.765)}
\gppoint{gp mark 1}{(5.726,4.771)}
\gppoint{gp mark 1}{(5.729,4.768)}
\gppoint{gp mark 1}{(5.731,4.799)}
\gppoint{gp mark 1}{(5.733,4.794)}
\gppoint{gp mark 1}{(5.735,4.816)}
\gppoint{gp mark 1}{(5.738,4.825)}
\gppoint{gp mark 1}{(5.740,4.827)}
\gppoint{gp mark 1}{(5.742,4.832)}
\gppoint{gp mark 1}{(5.744,4.826)}
\gppoint{gp mark 1}{(5.747,4.821)}
\gppoint{gp mark 1}{(5.749,4.813)}
\gppoint{gp mark 1}{(5.751,4.822)}
\gppoint{gp mark 1}{(5.753,4.828)}
\gppoint{gp mark 1}{(5.756,4.831)}
\gppoint{gp mark 1}{(5.758,4.818)}
\gppoint{gp mark 1}{(5.760,4.860)}
\gppoint{gp mark 1}{(5.762,4.845)}
\gppoint{gp mark 1}{(5.765,4.850)}
\gppoint{gp mark 1}{(5.767,4.863)}
\gppoint{gp mark 1}{(5.769,4.860)}
\gppoint{gp mark 1}{(5.771,4.896)}
\gppoint{gp mark 1}{(5.774,4.880)}
\gppoint{gp mark 1}{(5.776,4.867)}
\gppoint{gp mark 1}{(5.778,4.875)}
\gppoint{gp mark 1}{(5.780,4.905)}
\gppoint{gp mark 1}{(5.783,4.907)}
\gppoint{gp mark 1}{(5.785,4.915)}
\gppoint{gp mark 1}{(5.787,4.914)}
\gppoint{gp mark 1}{(5.789,4.920)}
\gppoint{gp mark 1}{(5.792,4.941)}
\gppoint{gp mark 1}{(5.794,4.915)}
\gppoint{gp mark 1}{(5.796,4.921)}
\gppoint{gp mark 1}{(5.798,4.925)}
\gppoint{gp mark 1}{(5.801,4.934)}
\gppoint{gp mark 1}{(5.803,4.921)}
\gppoint{gp mark 1}{(5.805,4.895)}
\gppoint{gp mark 1}{(5.807,4.930)}
\gppoint{gp mark 1}{(5.810,4.933)}
\gppoint{gp mark 1}{(5.812,4.942)}
\gppoint{gp mark 1}{(5.814,4.958)}
\gppoint{gp mark 1}{(5.816,4.936)}
\gppoint{gp mark 1}{(5.819,4.934)}
\gppoint{gp mark 1}{(5.821,4.942)}
\gppoint{gp mark 1}{(5.823,4.942)}
\gppoint{gp mark 1}{(5.825,4.963)}
\gppoint{gp mark 1}{(5.828,4.961)}
\gppoint{gp mark 1}{(5.830,4.965)}
\gppoint{gp mark 1}{(5.832,4.994)}
\gppoint{gp mark 1}{(5.834,4.974)}
\gppoint{gp mark 1}{(5.837,4.981)}
\gppoint{gp mark 1}{(5.839,4.984)}
\gppoint{gp mark 1}{(5.841,4.980)}
\gppoint{gp mark 1}{(5.843,4.980)}
\gppoint{gp mark 1}{(5.846,4.978)}
\gppoint{gp mark 1}{(5.848,5.009)}
\gppoint{gp mark 1}{(5.850,4.987)}
\gppoint{gp mark 1}{(5.852,4.977)}
\gppoint{gp mark 1}{(5.855,4.969)}
\gppoint{gp mark 1}{(5.857,4.999)}
\gppoint{gp mark 1}{(5.859,4.988)}
\gppoint{gp mark 1}{(5.861,5.019)}
\gppoint{gp mark 1}{(5.864,5.055)}
\gppoint{gp mark 1}{(5.866,5.031)}
\gppoint{gp mark 1}{(5.868,5.028)}
\gppoint{gp mark 1}{(5.870,5.035)}
\gppoint{gp mark 1}{(5.873,5.028)}
\gppoint{gp mark 1}{(5.875,5.035)}
\gppoint{gp mark 1}{(5.877,5.044)}
\gppoint{gp mark 1}{(5.879,5.027)}
\gppoint{gp mark 1}{(5.882,5.057)}
\gppoint{gp mark 1}{(5.884,5.048)}
\gppoint{gp mark 1}{(5.886,5.014)}
\gppoint{gp mark 1}{(5.888,5.058)}
\gppoint{gp mark 1}{(5.891,5.063)}
\gppoint{gp mark 1}{(5.893,5.068)}
\gppoint{gp mark 1}{(5.895,5.090)}
\gppoint{gp mark 1}{(5.897,5.060)}
\gppoint{gp mark 1}{(5.900,5.055)}
\gppoint{gp mark 1}{(5.902,5.044)}
\gppoint{gp mark 1}{(5.904,5.072)}
\gppoint{gp mark 1}{(5.906,5.031)}
\gppoint{gp mark 1}{(5.909,5.082)}
\gppoint{gp mark 1}{(5.911,5.125)}
\gppoint{gp mark 1}{(5.913,5.092)}
\gppoint{gp mark 1}{(5.915,5.112)}
\gppoint{gp mark 1}{(5.918,5.113)}
\gppoint{gp mark 1}{(5.920,5.101)}
\gppoint{gp mark 1}{(5.922,5.070)}
\gppoint{gp mark 1}{(5.924,5.063)}
\gppoint{gp mark 1}{(5.927,5.107)}
\gppoint{gp mark 1}{(5.929,5.091)}
\gppoint{gp mark 1}{(5.931,5.108)}
\gppoint{gp mark 1}{(5.933,5.085)}
\gppoint{gp mark 1}{(5.936,5.131)}
\gppoint{gp mark 1}{(5.938,5.123)}
\gppoint{gp mark 1}{(5.940,5.094)}
\gppoint{gp mark 1}{(5.942,5.131)}
\gppoint{gp mark 1}{(5.945,5.118)}
\gppoint{gp mark 1}{(5.947,5.117)}
\gppoint{gp mark 1}{(5.949,5.092)}
\gppoint{gp mark 1}{(5.951,5.116)}
\gppoint{gp mark 1}{(5.954,5.095)}
\gppoint{gp mark 1}{(5.956,5.089)}
\gppoint{gp mark 1}{(5.958,5.106)}
\gppoint{gp mark 1}{(5.960,5.150)}
\gppoint{gp mark 1}{(5.963,5.165)}
\gppoint{gp mark 1}{(5.965,5.142)}
\gppoint{gp mark 1}{(5.967,5.133)}
\gppoint{gp mark 1}{(5.969,5.134)}
\gppoint{gp mark 1}{(5.972,5.135)}
\gppoint{gp mark 1}{(5.974,5.154)}
\gppoint{gp mark 1}{(5.976,5.161)}
\gppoint{gp mark 1}{(5.978,5.177)}
\gppoint{gp mark 1}{(5.981,5.211)}
\gppoint{gp mark 1}{(5.983,5.174)}
\gppoint{gp mark 1}{(5.985,5.197)}
\gppoint{gp mark 1}{(5.987,5.149)}
\gppoint{gp mark 1}{(5.990,5.147)}
\gppoint{gp mark 1}{(5.992,5.137)}
\gppoint{gp mark 1}{(5.994,5.155)}
\gppoint{gp mark 1}{(5.996,5.151)}
\gppoint{gp mark 1}{(5.999,5.142)}
\gppoint{gp mark 1}{(6.001,5.162)}
\gppoint{gp mark 1}{(6.003,5.160)}
\gppoint{gp mark 1}{(6.005,5.179)}
\gppoint{gp mark 1}{(6.008,5.165)}
\gppoint{gp mark 1}{(6.010,5.148)}
\gppoint{gp mark 1}{(6.012,5.141)}
\gppoint{gp mark 1}{(6.014,5.168)}
\gppoint{gp mark 1}{(6.017,5.160)}
\gppoint{gp mark 1}{(6.019,5.193)}
\gppoint{gp mark 1}{(6.021,5.148)}
\gppoint{gp mark 1}{(6.023,5.183)}
\gppoint{gp mark 1}{(6.026,5.211)}
\gppoint{gp mark 1}{(6.028,5.166)}
\gppoint{gp mark 1}{(6.030,5.230)}
\gppoint{gp mark 1}{(6.032,5.212)}
\gppoint{gp mark 1}{(6.035,5.228)}
\gppoint{gp mark 1}{(6.037,5.175)}
\gppoint{gp mark 1}{(6.039,5.182)}
\gppoint{gp mark 1}{(6.041,5.211)}
\gppoint{gp mark 1}{(6.044,5.187)}
\gppoint{gp mark 1}{(6.046,5.186)}
\gppoint{gp mark 1}{(6.048,5.184)}
\gppoint{gp mark 1}{(6.050,5.207)}
\gppoint{gp mark 1}{(6.053,5.200)}
\gppoint{gp mark 1}{(6.055,5.246)}
\gppoint{gp mark 1}{(6.057,5.215)}
\gppoint{gp mark 1}{(6.059,5.233)}
\gppoint{gp mark 1}{(6.062,5.186)}
\gppoint{gp mark 1}{(6.064,5.189)}
\gppoint{gp mark 1}{(6.066,5.191)}
\gppoint{gp mark 1}{(6.068,5.200)}
\gppoint{gp mark 1}{(6.071,5.209)}
\gppoint{gp mark 1}{(6.073,5.219)}
\gppoint{gp mark 1}{(6.075,5.235)}
\gppoint{gp mark 1}{(6.077,5.216)}
\gppoint{gp mark 1}{(6.080,5.202)}
\gppoint{gp mark 1}{(6.082,5.221)}
\gppoint{gp mark 1}{(6.084,5.222)}
\gppoint{gp mark 1}{(6.086,5.230)}
\gppoint{gp mark 1}{(6.089,5.242)}
\gppoint{gp mark 1}{(6.091,5.250)}
\gppoint{gp mark 1}{(6.093,5.216)}
\gppoint{gp mark 1}{(6.095,5.221)}
\gppoint{gp mark 1}{(6.098,5.238)}
\gppoint{gp mark 1}{(6.100,5.227)}
\gppoint{gp mark 1}{(6.102,5.241)}
\gppoint{gp mark 1}{(6.104,5.241)}
\gppoint{gp mark 1}{(6.107,5.265)}
\gppoint{gp mark 1}{(6.109,5.253)}
\gppoint{gp mark 1}{(6.111,5.231)}
\gppoint{gp mark 1}{(6.113,5.230)}
\gppoint{gp mark 1}{(6.116,5.224)}
\gppoint{gp mark 1}{(6.118,5.244)}
\gppoint{gp mark 1}{(6.120,5.222)}
\gppoint{gp mark 1}{(6.122,5.251)}
\gppoint{gp mark 1}{(6.125,5.240)}
\gppoint{gp mark 1}{(6.127,5.249)}
\gppoint{gp mark 1}{(6.129,5.228)}
\gppoint{gp mark 1}{(6.131,5.226)}
\gppoint{gp mark 1}{(6.134,5.282)}
\gppoint{gp mark 1}{(6.136,5.246)}
\gppoint{gp mark 1}{(6.138,5.250)}
\gppoint{gp mark 1}{(6.140,5.272)}
\gppoint{gp mark 1}{(6.143,5.253)}
\gppoint{gp mark 1}{(6.145,5.231)}
\gppoint{gp mark 1}{(6.147,5.230)}
\gppoint{gp mark 1}{(6.149,5.247)}
\gppoint{gp mark 1}{(6.152,5.244)}
\gppoint{gp mark 1}{(6.154,5.264)}
\gppoint{gp mark 1}{(6.156,5.276)}
\gppoint{gp mark 1}{(6.158,5.251)}
\gppoint{gp mark 1}{(6.161,5.239)}
\gppoint{gp mark 1}{(6.163,5.250)}
\gppoint{gp mark 1}{(6.165,5.212)}
\gppoint{gp mark 1}{(6.167,5.291)}
\gppoint{gp mark 1}{(6.170,5.282)}
\gppoint{gp mark 1}{(6.172,5.271)}
\gppoint{gp mark 1}{(6.174,5.261)}
\gppoint{gp mark 1}{(6.176,5.226)}
\gppoint{gp mark 1}{(6.179,5.288)}
\gppoint{gp mark 1}{(6.181,5.217)}
\gppoint{gp mark 1}{(6.183,5.234)}
\gppoint{gp mark 1}{(6.185,5.264)}
\gppoint{gp mark 1}{(6.188,5.260)}
\gppoint{gp mark 1}{(6.190,5.248)}
\gppoint{gp mark 1}{(6.192,5.271)}
\gppoint{gp mark 1}{(6.194,5.266)}
\gppoint{gp mark 1}{(6.197,5.250)}
\gppoint{gp mark 1}{(6.199,5.235)}
\gppoint{gp mark 1}{(6.201,5.239)}
\gppoint{gp mark 1}{(6.203,5.283)}
\gppoint{gp mark 1}{(6.206,5.271)}
\gppoint{gp mark 1}{(6.208,5.246)}
\gppoint{gp mark 1}{(6.210,5.259)}
\gppoint{gp mark 1}{(6.212,5.271)}
\gppoint{gp mark 1}{(6.215,5.251)}
\gppoint{gp mark 1}{(6.217,5.265)}
\gppoint{gp mark 1}{(6.219,5.250)}
\gppoint{gp mark 1}{(6.221,5.254)}
\gppoint{gp mark 1}{(6.224,5.264)}
\gppoint{gp mark 1}{(6.226,5.250)}
\gppoint{gp mark 1}{(6.228,5.232)}
\gppoint{gp mark 1}{(6.230,5.274)}
\gppoint{gp mark 1}{(6.233,5.231)}
\gppoint{gp mark 1}{(6.235,5.239)}
\gppoint{gp mark 1}{(6.237,5.233)}
\gppoint{gp mark 1}{(6.239,5.238)}
\gppoint{gp mark 1}{(6.242,5.248)}
\gppoint{gp mark 1}{(6.244,5.267)}
\gppoint{gp mark 1}{(6.246,5.275)}
\gppoint{gp mark 1}{(6.248,5.224)}
\gppoint{gp mark 1}{(6.251,5.258)}
\gppoint{gp mark 1}{(6.253,5.273)}
\gppoint{gp mark 1}{(6.255,5.279)}
\gppoint{gp mark 1}{(6.257,5.241)}
\gppoint{gp mark 1}{(6.260,5.241)}
\gppoint{gp mark 1}{(6.262,5.279)}
\gppoint{gp mark 1}{(6.264,5.243)}
\gppoint{gp mark 1}{(6.266,5.239)}
\gppoint{gp mark 1}{(6.269,5.241)}
\gppoint{gp mark 1}{(6.271,5.199)}
\gppoint{gp mark 1}{(6.273,5.223)}
\gppoint{gp mark 1}{(6.275,5.245)}
\gppoint{gp mark 1}{(6.278,5.229)}
\gppoint{gp mark 1}{(6.280,5.249)}
\gppoint{gp mark 1}{(6.282,5.264)}
\gppoint{gp mark 1}{(6.284,5.270)}
\gppoint{gp mark 1}{(6.287,5.219)}
\gppoint{gp mark 1}{(6.289,5.243)}
\gppoint{gp mark 1}{(6.291,5.254)}
\gppoint{gp mark 1}{(6.293,5.197)}
\gppoint{gp mark 1}{(6.296,5.224)}
\gppoint{gp mark 1}{(6.298,5.255)}
\gppoint{gp mark 1}{(6.300,5.242)}
\gppoint{gp mark 1}{(6.302,5.234)}
\gppoint{gp mark 1}{(6.305,5.236)}
\gppoint{gp mark 1}{(6.307,5.254)}
\gppoint{gp mark 1}{(6.309,5.275)}
\gppoint{gp mark 1}{(6.311,5.283)}
\gppoint{gp mark 1}{(6.314,5.248)}
\gppoint{gp mark 1}{(6.316,5.215)}
\gppoint{gp mark 1}{(6.318,5.206)}
\gppoint{gp mark 1}{(6.320,5.212)}
\gppoint{gp mark 1}{(6.323,5.240)}
\gppoint{gp mark 1}{(6.325,5.246)}
\gppoint{gp mark 1}{(6.327,5.231)}
\gppoint{gp mark 1}{(6.329,5.220)}
\gppoint{gp mark 1}{(6.332,5.205)}
\gppoint{gp mark 1}{(6.334,5.205)}
\gppoint{gp mark 1}{(6.336,5.188)}
\gppoint{gp mark 1}{(6.338,5.205)}
\gppoint{gp mark 1}{(6.341,5.213)}
\gppoint{gp mark 1}{(6.343,5.206)}
\gppoint{gp mark 1}{(6.345,5.190)}
\gppoint{gp mark 1}{(6.347,5.230)}
\gppoint{gp mark 1}{(6.350,5.228)}
\gppoint{gp mark 1}{(6.352,5.171)}
\gppoint{gp mark 1}{(6.354,5.183)}
\gppoint{gp mark 1}{(6.356,5.184)}
\gppoint{gp mark 1}{(6.359,5.209)}
\gppoint{gp mark 1}{(6.361,5.188)}
\gppoint{gp mark 1}{(6.363,5.180)}
\gppoint{gp mark 1}{(6.365,5.166)}
\gppoint{gp mark 1}{(6.368,5.211)}
\gppoint{gp mark 1}{(6.370,5.210)}
\gppoint{gp mark 1}{(6.372,5.164)}
\gppoint{gp mark 1}{(6.374,5.161)}
\gppoint{gp mark 1}{(6.377,5.201)}
\gppoint{gp mark 1}{(6.379,5.183)}
\gppoint{gp mark 1}{(6.381,5.170)}
\gppoint{gp mark 1}{(6.384,5.169)}
\gppoint{gp mark 1}{(6.386,5.161)}
\gppoint{gp mark 1}{(6.388,5.185)}
\gppoint{gp mark 1}{(6.390,5.191)}
\gppoint{gp mark 1}{(6.393,5.165)}
\gppoint{gp mark 1}{(6.395,5.214)}
\gppoint{gp mark 1}{(6.397,5.163)}
\gppoint{gp mark 1}{(6.399,5.180)}
\gppoint{gp mark 1}{(6.402,5.120)}
\gppoint{gp mark 1}{(6.404,5.130)}
\gppoint{gp mark 1}{(6.406,5.113)}
\gppoint{gp mark 1}{(6.408,5.154)}
\gppoint{gp mark 1}{(6.411,5.140)}
\gppoint{gp mark 1}{(6.413,5.112)}
\gppoint{gp mark 1}{(6.415,5.158)}
\gppoint{gp mark 1}{(6.417,5.160)}
\gppoint{gp mark 1}{(6.420,5.128)}
\gppoint{gp mark 1}{(6.422,5.135)}
\gppoint{gp mark 1}{(6.424,5.121)}
\gppoint{gp mark 1}{(6.426,5.122)}
\gppoint{gp mark 1}{(6.429,5.093)}
\gppoint{gp mark 1}{(6.431,5.129)}
\gppoint{gp mark 1}{(6.433,5.114)}
\gppoint{gp mark 1}{(6.435,5.099)}
\gppoint{gp mark 1}{(6.438,5.067)}
\gppoint{gp mark 1}{(6.440,5.067)}
\gppoint{gp mark 1}{(6.442,5.093)}
\gppoint{gp mark 1}{(6.444,5.067)}
\gppoint{gp mark 1}{(6.447,5.077)}
\gppoint{gp mark 1}{(6.449,5.083)}
\gppoint{gp mark 1}{(6.451,5.061)}
\gppoint{gp mark 1}{(6.453,5.104)}
\gppoint{gp mark 1}{(6.456,5.090)}
\gppoint{gp mark 1}{(6.458,5.098)}
\gppoint{gp mark 1}{(6.460,5.077)}
\gppoint{gp mark 1}{(6.462,5.105)}
\gppoint{gp mark 1}{(6.465,5.081)}
\gppoint{gp mark 1}{(6.467,5.061)}
\gppoint{gp mark 1}{(6.469,5.047)}
\gppoint{gp mark 1}{(6.471,5.058)}
\gppoint{gp mark 1}{(6.474,5.080)}
\gppoint{gp mark 1}{(6.476,5.087)}
\gppoint{gp mark 1}{(6.478,5.049)}
\gppoint{gp mark 1}{(6.480,5.035)}
\gppoint{gp mark 1}{(6.483,5.052)}
\gppoint{gp mark 1}{(6.485,5.046)}
\gppoint{gp mark 1}{(6.487,5.043)}
\gppoint{gp mark 1}{(6.489,5.050)}
\gppoint{gp mark 1}{(6.492,5.032)}
\gppoint{gp mark 1}{(6.494,5.031)}
\gppoint{gp mark 1}{(6.496,5.026)}
\gppoint{gp mark 1}{(6.498,5.009)}
\gppoint{gp mark 1}{(6.501,4.998)}
\gppoint{gp mark 1}{(6.503,5.061)}
\gppoint{gp mark 1}{(6.505,5.027)}
\gppoint{gp mark 1}{(6.507,5.015)}
\gppoint{gp mark 1}{(6.510,5.014)}
\gppoint{gp mark 1}{(6.512,5.027)}
\gppoint{gp mark 1}{(6.514,5.011)}
\gppoint{gp mark 1}{(6.516,4.952)}
\gppoint{gp mark 1}{(6.519,5.006)}
\gppoint{gp mark 1}{(6.521,5.003)}
\gppoint{gp mark 1}{(6.523,4.977)}
\gppoint{gp mark 1}{(6.525,4.957)}
\gppoint{gp mark 1}{(6.528,4.961)}
\gppoint{gp mark 1}{(6.530,4.989)}
\gppoint{gp mark 1}{(6.532,4.963)}
\gppoint{gp mark 1}{(6.534,4.979)}
\gppoint{gp mark 1}{(6.537,4.956)}
\gppoint{gp mark 1}{(6.539,4.997)}
\gppoint{gp mark 1}{(6.541,5.010)}
\gppoint{gp mark 1}{(6.543,4.971)}
\gppoint{gp mark 1}{(6.546,4.980)}
\gppoint{gp mark 1}{(6.548,4.987)}
\gppoint{gp mark 1}{(6.550,4.961)}
\gppoint{gp mark 1}{(6.552,4.972)}
\gppoint{gp mark 1}{(6.555,4.944)}
\gppoint{gp mark 1}{(6.557,4.956)}
\gppoint{gp mark 1}{(6.559,4.957)}
\gppoint{gp mark 1}{(6.561,4.942)}
\gppoint{gp mark 1}{(6.564,4.930)}
\gppoint{gp mark 1}{(6.566,4.911)}
\gppoint{gp mark 1}{(6.568,4.932)}
\gppoint{gp mark 1}{(6.570,4.936)}
\gppoint{gp mark 1}{(6.573,4.930)}
\gppoint{gp mark 1}{(6.575,4.915)}
\gppoint{gp mark 1}{(6.577,4.880)}
\gppoint{gp mark 1}{(6.579,4.895)}
\gppoint{gp mark 1}{(6.582,4.896)}
\gppoint{gp mark 1}{(6.584,4.900)}
\gppoint{gp mark 1}{(6.586,4.879)}
\gppoint{gp mark 1}{(6.588,4.910)}
\gppoint{gp mark 1}{(6.591,4.843)}
\gppoint{gp mark 1}{(6.593,4.902)}
\gppoint{gp mark 1}{(6.595,4.852)}
\gppoint{gp mark 1}{(6.597,4.890)}
\gppoint{gp mark 1}{(6.600,4.847)}
\gppoint{gp mark 1}{(6.602,4.834)}
\gppoint{gp mark 1}{(6.604,4.861)}
\gppoint{gp mark 1}{(6.606,4.851)}
\gppoint{gp mark 1}{(6.609,4.841)}
\gppoint{gp mark 1}{(6.611,4.861)}
\gppoint{gp mark 1}{(6.613,4.837)}
\gppoint{gp mark 1}{(6.615,4.851)}
\gppoint{gp mark 1}{(6.618,4.812)}
\gppoint{gp mark 1}{(6.620,4.846)}
\gppoint{gp mark 1}{(6.622,4.827)}
\gppoint{gp mark 1}{(6.624,4.829)}
\gppoint{gp mark 1}{(6.627,4.823)}
\gppoint{gp mark 1}{(6.629,4.821)}
\gppoint{gp mark 1}{(6.631,4.806)}
\gppoint{gp mark 1}{(6.633,4.841)}
\gppoint{gp mark 1}{(6.636,4.767)}
\gppoint{gp mark 1}{(6.638,4.812)}
\gppoint{gp mark 1}{(6.640,4.794)}
\gppoint{gp mark 1}{(6.642,4.751)}
\gppoint{gp mark 1}{(6.645,4.782)}
\gppoint{gp mark 1}{(6.647,4.760)}
\gppoint{gp mark 1}{(6.649,4.756)}
\gppoint{gp mark 1}{(6.651,4.769)}
\gppoint{gp mark 1}{(6.654,4.764)}
\gppoint{gp mark 1}{(6.656,4.769)}
\gppoint{gp mark 1}{(6.658,4.748)}
\gppoint{gp mark 1}{(6.660,4.755)}
\gppoint{gp mark 1}{(6.663,4.740)}
\gppoint{gp mark 1}{(6.665,4.742)}
\gppoint{gp mark 1}{(6.667,4.705)}
\gppoint{gp mark 1}{(6.669,4.730)}
\gppoint{gp mark 1}{(6.672,4.730)}
\gppoint{gp mark 1}{(6.674,4.737)}
\gppoint{gp mark 1}{(6.676,4.713)}
\gppoint{gp mark 1}{(6.678,4.718)}
\gppoint{gp mark 1}{(6.681,4.718)}
\gppoint{gp mark 1}{(6.683,4.686)}
\gppoint{gp mark 1}{(6.685,4.688)}
\gppoint{gp mark 1}{(6.687,4.709)}
\gppoint{gp mark 1}{(6.690,4.707)}
\gppoint{gp mark 1}{(6.692,4.686)}
\gppoint{gp mark 1}{(6.694,4.655)}
\gppoint{gp mark 1}{(6.696,4.638)}
\gppoint{gp mark 1}{(6.699,4.683)}
\gppoint{gp mark 1}{(6.701,4.677)}
\gppoint{gp mark 1}{(6.703,4.683)}
\gppoint{gp mark 1}{(6.705,4.652)}
\gppoint{gp mark 1}{(6.708,4.642)}
\gppoint{gp mark 1}{(6.710,4.636)}
\gppoint{gp mark 1}{(6.712,4.635)}
\gppoint{gp mark 1}{(6.714,4.650)}
\gppoint{gp mark 1}{(6.717,4.646)}
\gppoint{gp mark 1}{(6.719,4.632)}
\gppoint{gp mark 1}{(6.721,4.637)}
\gppoint{gp mark 1}{(6.723,4.617)}
\gppoint{gp mark 1}{(6.726,4.622)}
\gppoint{gp mark 1}{(6.728,4.617)}
\gppoint{gp mark 1}{(6.730,4.627)}
\gppoint{gp mark 1}{(6.732,4.597)}
\gppoint{gp mark 1}{(6.735,4.591)}
\gppoint{gp mark 1}{(6.737,4.594)}
\gppoint{gp mark 1}{(6.739,4.580)}
\gppoint{gp mark 1}{(6.741,4.565)}
\gppoint{gp mark 1}{(6.744,4.563)}
\gppoint{gp mark 1}{(6.746,4.572)}
\gppoint{gp mark 1}{(6.748,4.571)}
\gppoint{gp mark 1}{(6.750,4.567)}
\gppoint{gp mark 1}{(6.753,4.557)}
\gppoint{gp mark 1}{(6.755,4.569)}
\gppoint{gp mark 1}{(6.757,4.540)}
\gppoint{gp mark 1}{(6.759,4.545)}
\gppoint{gp mark 1}{(6.762,4.524)}
\gppoint{gp mark 1}{(6.764,4.528)}
\gppoint{gp mark 1}{(6.766,4.553)}
\gppoint{gp mark 1}{(6.768,4.537)}
\gppoint{gp mark 1}{(6.771,4.507)}
\gppoint{gp mark 1}{(6.773,4.497)}
\gppoint{gp mark 1}{(6.775,4.503)}
\gppoint{gp mark 1}{(6.777,4.513)}
\gppoint{gp mark 1}{(6.780,4.469)}
\gppoint{gp mark 1}{(6.782,4.479)}
\gppoint{gp mark 1}{(6.784,4.465)}
\gppoint{gp mark 1}{(6.786,4.483)}
\gppoint{gp mark 1}{(6.789,4.456)}
\gppoint{gp mark 1}{(6.791,4.449)}
\gppoint{gp mark 1}{(6.793,4.457)}
\gppoint{gp mark 1}{(6.795,4.498)}
\gppoint{gp mark 1}{(6.798,4.466)}
\gppoint{gp mark 1}{(6.800,4.464)}
\gppoint{gp mark 1}{(6.802,4.460)}
\gppoint{gp mark 1}{(6.804,4.485)}
\gppoint{gp mark 1}{(6.807,4.424)}
\gppoint{gp mark 1}{(6.809,4.432)}
\gppoint{gp mark 1}{(6.811,4.410)}
\gppoint{gp mark 1}{(6.813,4.422)}
\gppoint{gp mark 1}{(6.816,4.403)}
\gppoint{gp mark 1}{(6.818,4.387)}
\gppoint{gp mark 1}{(6.820,4.385)}
\gppoint{gp mark 1}{(6.822,4.399)}
\gppoint{gp mark 1}{(6.825,4.383)}
\gppoint{gp mark 1}{(6.827,4.349)}
\gppoint{gp mark 1}{(6.829,4.393)}
\gppoint{gp mark 1}{(6.831,4.393)}
\gppoint{gp mark 1}{(6.834,4.386)}
\gppoint{gp mark 1}{(6.836,4.373)}
\gppoint{gp mark 1}{(6.838,4.355)}
\gppoint{gp mark 1}{(6.840,4.355)}
\gppoint{gp mark 1}{(6.843,4.350)}
\gppoint{gp mark 1}{(6.845,4.364)}
\gppoint{gp mark 1}{(6.847,4.360)}
\gppoint{gp mark 1}{(6.849,4.340)}
\gppoint{gp mark 1}{(6.852,4.315)}
\gppoint{gp mark 1}{(6.854,4.306)}
\gppoint{gp mark 1}{(6.856,4.322)}
\gppoint{gp mark 1}{(6.858,4.299)}
\gppoint{gp mark 1}{(6.861,4.310)}
\gppoint{gp mark 1}{(6.863,4.316)}
\gppoint{gp mark 1}{(6.865,4.276)}
\gppoint{gp mark 1}{(6.867,4.255)}
\gppoint{gp mark 1}{(6.870,4.269)}
\gppoint{gp mark 1}{(6.872,4.268)}
\gppoint{gp mark 1}{(6.874,4.269)}
\gppoint{gp mark 1}{(6.876,4.255)}
\gppoint{gp mark 1}{(6.879,4.228)}
\gppoint{gp mark 1}{(6.881,4.237)}
\gppoint{gp mark 1}{(6.883,4.241)}
\gppoint{gp mark 1}{(6.885,4.222)}
\gppoint{gp mark 1}{(6.888,4.232)}
\gppoint{gp mark 1}{(6.890,4.229)}
\gppoint{gp mark 1}{(6.892,4.228)}
\gppoint{gp mark 1}{(6.894,4.219)}
\gppoint{gp mark 1}{(6.897,4.242)}
\gppoint{gp mark 1}{(6.899,4.198)}
\gppoint{gp mark 1}{(6.901,4.214)}
\gppoint{gp mark 1}{(6.903,4.173)}
\gppoint{gp mark 1}{(6.906,4.158)}
\gppoint{gp mark 1}{(6.908,4.165)}
\gppoint{gp mark 1}{(6.910,4.165)}
\gppoint{gp mark 1}{(6.912,4.167)}
\gppoint{gp mark 1}{(6.915,4.140)}
\gppoint{gp mark 1}{(6.917,4.162)}
\gppoint{gp mark 1}{(6.919,4.159)}
\gppoint{gp mark 1}{(6.921,4.123)}
\gppoint{gp mark 1}{(6.924,4.149)}
\gppoint{gp mark 1}{(6.926,4.128)}
\gppoint{gp mark 1}{(6.928,4.130)}
\gppoint{gp mark 1}{(6.930,4.122)}
\gppoint{gp mark 1}{(6.933,4.125)}
\gppoint{gp mark 1}{(6.935,4.153)}
\gppoint{gp mark 1}{(6.937,4.165)}
\gppoint{gp mark 1}{(6.939,4.129)}
\gppoint{gp mark 1}{(6.942,4.124)}
\gppoint{gp mark 1}{(6.944,4.112)}
\gppoint{gp mark 1}{(6.946,4.062)}
\gppoint{gp mark 1}{(6.948,4.067)}
\gppoint{gp mark 1}{(6.951,4.072)}
\gppoint{gp mark 1}{(6.953,4.052)}
\gppoint{gp mark 1}{(6.955,4.042)}
\gppoint{gp mark 1}{(6.957,4.052)}
\gppoint{gp mark 1}{(6.960,4.076)}
\gppoint{gp mark 1}{(6.962,4.072)}
\gppoint{gp mark 1}{(6.964,4.067)}
\gppoint{gp mark 1}{(6.966,4.029)}
\gppoint{gp mark 1}{(6.969,4.002)}
\gppoint{gp mark 1}{(6.971,4.011)}
\gppoint{gp mark 1}{(6.973,3.987)}
\gppoint{gp mark 1}{(6.975,4.020)}
\gppoint{gp mark 1}{(6.978,3.997)}
\gppoint{gp mark 1}{(6.980,3.996)}
\gppoint{gp mark 1}{(6.982,3.988)}
\gppoint{gp mark 1}{(6.984,3.986)}
\gppoint{gp mark 1}{(6.987,4.015)}
\gppoint{gp mark 1}{(6.989,4.000)}
\gppoint{gp mark 1}{(6.991,4.001)}
\gppoint{gp mark 1}{(6.993,3.966)}
\gppoint{gp mark 1}{(6.996,3.927)}
\gppoint{gp mark 1}{(6.998,3.917)}
\gppoint{gp mark 1}{(7.000,3.898)}
\gppoint{gp mark 1}{(7.002,3.910)}
\gppoint{gp mark 1}{(7.005,3.897)}
\gppoint{gp mark 1}{(7.007,3.928)}
\gppoint{gp mark 1}{(7.009,3.909)}
\gppoint{gp mark 1}{(7.011,3.905)}
\gppoint{gp mark 1}{(7.014,3.900)}
\gppoint{gp mark 1}{(7.016,3.888)}
\gppoint{gp mark 1}{(7.018,3.883)}
\gppoint{gp mark 1}{(7.020,3.835)}
\gppoint{gp mark 1}{(7.023,3.839)}
\gppoint{gp mark 1}{(7.025,3.840)}
\gppoint{gp mark 1}{(7.027,3.803)}
\gppoint{gp mark 1}{(7.029,3.833)}
\gppoint{gp mark 1}{(7.032,3.812)}
\gppoint{gp mark 1}{(7.034,3.835)}
\gppoint{gp mark 1}{(7.036,3.823)}
\gppoint{gp mark 1}{(7.038,3.820)}
\gppoint{gp mark 1}{(7.041,3.795)}
\gppoint{gp mark 1}{(7.043,3.802)}
\gppoint{gp mark 1}{(7.045,3.763)}
\gppoint{gp mark 1}{(7.047,3.782)}
\gppoint{gp mark 1}{(7.050,3.776)}
\gppoint{gp mark 1}{(7.052,3.759)}
\gppoint{gp mark 1}{(7.054,3.775)}
\gppoint{gp mark 1}{(7.056,3.783)}
\gppoint{gp mark 1}{(7.059,3.763)}
\gppoint{gp mark 1}{(7.061,3.738)}
\gppoint{gp mark 1}{(7.063,3.717)}
\gppoint{gp mark 1}{(7.065,3.749)}
\gppoint{gp mark 1}{(7.068,3.711)}
\gppoint{gp mark 1}{(7.070,3.720)}
\gppoint{gp mark 1}{(7.072,3.688)}
\gppoint{gp mark 1}{(7.074,3.674)}
\gppoint{gp mark 1}{(7.077,3.691)}
\gppoint{gp mark 1}{(7.079,3.676)}
\gppoint{gp mark 1}{(7.081,3.677)}
\gppoint{gp mark 1}{(7.083,3.677)}
\gppoint{gp mark 1}{(7.086,3.659)}
\gppoint{gp mark 1}{(7.088,3.654)}
\gppoint{gp mark 1}{(7.090,3.613)}
\gppoint{gp mark 1}{(7.092,3.626)}
\gppoint{gp mark 1}{(7.095,3.641)}
\gppoint{gp mark 1}{(7.097,3.631)}
\gppoint{gp mark 1}{(7.099,3.611)}
\gppoint{gp mark 1}{(7.101,3.611)}
\gppoint{gp mark 1}{(7.104,3.612)}
\gppoint{gp mark 1}{(7.106,3.612)}
\gppoint{gp mark 1}{(7.108,3.627)}
\gppoint{gp mark 1}{(7.110,3.592)}
\gppoint{gp mark 1}{(7.113,3.585)}
\gppoint{gp mark 1}{(7.115,3.590)}
\gppoint{gp mark 1}{(7.117,3.572)}
\gppoint{gp mark 1}{(7.119,3.582)}
\gppoint{gp mark 1}{(7.122,3.544)}
\gppoint{gp mark 1}{(7.124,3.548)}
\gppoint{gp mark 1}{(7.126,3.528)}
\gppoint{gp mark 1}{(7.128,3.546)}
\gppoint{gp mark 1}{(7.131,3.540)}
\gppoint{gp mark 1}{(7.133,3.517)}
\gppoint{gp mark 1}{(7.135,3.514)}
\gppoint{gp mark 1}{(7.137,3.502)}
\gppoint{gp mark 1}{(7.140,3.492)}
\gppoint{gp mark 1}{(7.142,3.479)}
\gppoint{gp mark 1}{(7.144,3.475)}
\gppoint{gp mark 1}{(7.146,3.509)}
\gppoint{gp mark 1}{(7.149,3.478)}
\gppoint{gp mark 1}{(7.151,3.445)}
\gppoint{gp mark 1}{(7.153,3.422)}
\gppoint{gp mark 1}{(7.155,3.422)}
\gppoint{gp mark 1}{(7.158,3.440)}
\gppoint{gp mark 1}{(7.160,3.448)}
\gppoint{gp mark 1}{(7.162,3.441)}
\gppoint{gp mark 1}{(7.164,3.418)}
\gppoint{gp mark 1}{(7.167,3.413)}
\gppoint{gp mark 1}{(7.169,3.391)}
\gppoint{gp mark 1}{(7.171,3.437)}
\gppoint{gp mark 1}{(7.173,3.410)}
\gppoint{gp mark 1}{(7.176,3.408)}
\gppoint{gp mark 1}{(7.178,3.377)}
\gppoint{gp mark 1}{(7.180,3.403)}
\gppoint{gp mark 1}{(7.182,3.399)}
\gppoint{gp mark 1}{(7.185,3.430)}
\gppoint{gp mark 1}{(7.187,3.391)}
\gppoint{gp mark 1}{(7.189,3.336)}
\gppoint{gp mark 1}{(7.191,3.353)}
\gppoint{gp mark 1}{(7.194,3.330)}
\gppoint{gp mark 1}{(7.196,3.346)}
\gppoint{gp mark 1}{(7.198,3.345)}
\gppoint{gp mark 1}{(7.200,3.329)}
\gppoint{gp mark 1}{(7.203,3.333)}
\gppoint{gp mark 1}{(7.205,3.325)}
\gppoint{gp mark 1}{(7.207,3.350)}
\gppoint{gp mark 1}{(7.209,3.332)}
\gppoint{gp mark 1}{(7.212,3.309)}
\gppoint{gp mark 1}{(7.214,3.312)}
\gppoint{gp mark 1}{(7.216,3.299)}
\gppoint{gp mark 1}{(7.218,3.280)}
\gppoint{gp mark 1}{(7.221,3.254)}
\gppoint{gp mark 1}{(7.223,3.271)}
\gppoint{gp mark 1}{(7.225,3.231)}
\gppoint{gp mark 1}{(7.227,3.264)}
\gppoint{gp mark 1}{(7.230,3.264)}
\gppoint{gp mark 1}{(7.232,3.270)}
\gppoint{gp mark 1}{(7.234,3.217)}
\gppoint{gp mark 1}{(7.236,3.212)}
\gppoint{gp mark 1}{(7.239,3.232)}
\gppoint{gp mark 1}{(7.241,3.206)}
\gppoint{gp mark 1}{(7.243,3.218)}
\gppoint{gp mark 1}{(7.245,3.226)}
\gppoint{gp mark 1}{(7.248,3.187)}
\gppoint{gp mark 1}{(7.250,3.193)}
\gppoint{gp mark 1}{(7.252,3.181)}
\gppoint{gp mark 1}{(7.254,3.150)}
\gppoint{gp mark 1}{(7.257,3.161)}
\gppoint{gp mark 1}{(7.259,3.124)}
\gppoint{gp mark 1}{(7.261,3.109)}
\gppoint{gp mark 1}{(7.263,3.100)}
\gppoint{gp mark 1}{(7.266,3.099)}
\gppoint{gp mark 1}{(7.268,3.136)}
\gppoint{gp mark 1}{(7.270,3.127)}
\gppoint{gp mark 1}{(7.272,3.112)}
\gppoint{gp mark 1}{(7.275,3.102)}
\gppoint{gp mark 1}{(7.277,3.080)}
\gppoint{gp mark 1}{(7.279,3.088)}
\gppoint{gp mark 1}{(7.281,3.084)}
\gppoint{gp mark 1}{(7.284,3.103)}
\gppoint{gp mark 1}{(7.286,3.058)}
\gppoint{gp mark 1}{(7.288,3.071)}
\gppoint{gp mark 1}{(7.290,3.049)}
\gppoint{gp mark 1}{(7.293,3.064)}
\gppoint{gp mark 1}{(7.295,3.039)}
\gppoint{gp mark 1}{(7.297,3.061)}
\gppoint{gp mark 1}{(7.299,3.053)}
\gppoint{gp mark 1}{(7.302,3.021)}
\gppoint{gp mark 1}{(7.304,2.993)}
\gppoint{gp mark 1}{(7.306,3.037)}
\gppoint{gp mark 1}{(7.308,3.027)}
\gppoint{gp mark 1}{(7.311,3.014)}
\gppoint{gp mark 1}{(7.313,2.970)}
\gppoint{gp mark 1}{(7.315,2.998)}
\gppoint{gp mark 1}{(7.317,2.990)}
\gppoint{gp mark 1}{(7.320,2.977)}
\gppoint{gp mark 1}{(7.322,2.969)}
\gppoint{gp mark 1}{(7.324,2.970)}
\gppoint{gp mark 1}{(7.326,2.965)}
\gppoint{gp mark 1}{(7.329,2.960)}
\gppoint{gp mark 1}{(7.331,2.964)}
\gppoint{gp mark 1}{(7.333,2.923)}
\gppoint{gp mark 1}{(7.335,2.906)}
\gppoint{gp mark 1}{(7.338,2.929)}
\gppoint{gp mark 1}{(7.340,2.937)}
\gppoint{gp mark 1}{(7.342,2.915)}
\gppoint{gp mark 1}{(7.344,2.903)}
\gppoint{gp mark 1}{(7.347,2.890)}
\gppoint{gp mark 1}{(7.349,2.905)}
\gppoint{gp mark 1}{(7.351,2.895)}
\gppoint{gp mark 1}{(7.353,2.877)}
\gppoint{gp mark 1}{(7.356,2.892)}
\gppoint{gp mark 1}{(7.358,2.846)}
\gppoint{gp mark 1}{(7.360,2.853)}
\gppoint{gp mark 1}{(7.362,2.866)}
\gppoint{gp mark 1}{(7.365,2.862)}
\gppoint{gp mark 1}{(7.367,2.839)}
\gppoint{gp mark 1}{(7.369,2.857)}
\gppoint{gp mark 1}{(7.371,2.846)}
\gppoint{gp mark 1}{(7.374,2.838)}
\gppoint{gp mark 1}{(7.376,2.811)}
\gppoint{gp mark 1}{(7.378,2.827)}
\gppoint{gp mark 1}{(7.380,2.820)}
\gppoint{gp mark 1}{(7.383,2.816)}
\gppoint{gp mark 1}{(7.385,2.779)}
\gppoint{gp mark 1}{(7.387,2.780)}
\gppoint{gp mark 1}{(7.389,2.793)}
\gppoint{gp mark 1}{(7.392,2.791)}
\gppoint{gp mark 1}{(7.394,2.757)}
\gppoint{gp mark 1}{(7.396,2.784)}
\gppoint{gp mark 1}{(7.398,2.772)}
\gppoint{gp mark 1}{(7.401,2.724)}
\gppoint{gp mark 1}{(7.403,2.739)}
\gppoint{gp mark 1}{(7.405,2.748)}
\gppoint{gp mark 1}{(7.407,2.748)}
\gppoint{gp mark 1}{(7.410,2.748)}
\gppoint{gp mark 1}{(7.412,2.732)}
\gppoint{gp mark 1}{(7.414,2.716)}
\gppoint{gp mark 1}{(7.416,2.713)}
\gppoint{gp mark 1}{(7.419,2.708)}
\gppoint{gp mark 1}{(7.421,2.707)}
\gppoint{gp mark 1}{(7.423,2.702)}
\gppoint{gp mark 1}{(7.425,2.682)}
\gppoint{gp mark 1}{(7.428,2.711)}
\gppoint{gp mark 1}{(7.430,2.695)}
\gppoint{gp mark 1}{(7.432,2.700)}
\gppoint{gp mark 1}{(7.434,2.702)}
\gppoint{gp mark 1}{(7.437,2.685)}
\gppoint{gp mark 1}{(7.439,2.655)}
\gppoint{gp mark 1}{(7.441,2.679)}
\gppoint{gp mark 1}{(7.443,2.669)}
\gppoint{gp mark 1}{(7.446,2.646)}
\gppoint{gp mark 1}{(7.448,2.640)}
\gppoint{gp mark 1}{(7.450,2.678)}
\gppoint{gp mark 1}{(7.452,2.654)}
\gppoint{gp mark 1}{(7.455,2.643)}
\gppoint{gp mark 1}{(7.457,2.635)}
\gppoint{gp mark 1}{(7.459,2.618)}
\gppoint{gp mark 1}{(7.461,2.646)}
\gppoint{gp mark 1}{(7.464,2.626)}
\gppoint{gp mark 1}{(7.466,2.604)}
\gppoint{gp mark 1}{(7.468,2.607)}
\gppoint{gp mark 1}{(7.470,2.617)}
\gppoint{gp mark 1}{(7.473,2.600)}
\gppoint{gp mark 1}{(7.475,2.581)}
\gppoint{gp mark 1}{(7.477,2.583)}
\gppoint{gp mark 1}{(7.479,2.580)}
\gppoint{gp mark 1}{(7.482,2.586)}
\gppoint{gp mark 1}{(7.484,2.588)}
\gppoint{gp mark 1}{(7.486,2.554)}
\gppoint{gp mark 1}{(7.488,2.546)}
\gppoint{gp mark 1}{(7.491,2.553)}
\gppoint{gp mark 1}{(7.493,2.541)}
\gppoint{gp mark 1}{(7.495,2.538)}
\gppoint{gp mark 1}{(7.497,2.550)}
\gppoint{gp mark 1}{(7.500,2.555)}
\gppoint{gp mark 1}{(7.502,2.561)}
\gppoint{gp mark 1}{(7.504,2.523)}
\gppoint{gp mark 1}{(7.506,2.540)}
\gppoint{gp mark 1}{(7.509,2.532)}
\gppoint{gp mark 1}{(7.511,2.517)}
\gppoint{gp mark 1}{(7.513,2.491)}
\gppoint{gp mark 1}{(7.515,2.506)}
\gppoint{gp mark 1}{(7.518,2.493)}
\gppoint{gp mark 1}{(7.520,2.490)}
\gppoint{gp mark 1}{(7.522,2.502)}
\gppoint{gp mark 1}{(7.524,2.444)}
\gppoint{gp mark 1}{(7.527,2.467)}
\gppoint{gp mark 1}{(7.529,2.460)}
\gppoint{gp mark 1}{(7.531,2.458)}
\gppoint{gp mark 1}{(7.533,2.436)}
\gppoint{gp mark 1}{(7.536,2.453)}
\gppoint{gp mark 1}{(7.538,2.446)}
\gppoint{gp mark 1}{(7.540,2.477)}
\gppoint{gp mark 1}{(7.542,2.456)}
\gppoint{gp mark 1}{(7.545,2.439)}
\gppoint{gp mark 1}{(7.547,2.433)}
\gppoint{gp mark 1}{(7.549,2.416)}
\gppoint{gp mark 1}{(7.551,2.420)}
\gppoint{gp mark 1}{(7.554,2.400)}
\gppoint{gp mark 1}{(7.556,2.408)}
\gppoint{gp mark 1}{(7.558,2.390)}
\gppoint{gp mark 1}{(7.560,2.386)}
\gppoint{gp mark 1}{(7.563,2.391)}
\gppoint{gp mark 1}{(7.565,2.364)}
\gppoint{gp mark 1}{(7.567,2.372)}
\gppoint{gp mark 1}{(7.569,2.365)}
\gppoint{gp mark 1}{(7.572,2.359)}
\gppoint{gp mark 1}{(7.574,2.364)}
\gppoint{gp mark 1}{(7.576,2.343)}
\gppoint{gp mark 1}{(7.578,2.332)}
\gppoint{gp mark 1}{(7.581,2.345)}
\gppoint{gp mark 1}{(7.583,2.336)}
\gppoint{gp mark 1}{(7.585,2.366)}
\gppoint{gp mark 1}{(7.587,2.323)}
\gppoint{gp mark 1}{(7.590,2.336)}
\gppoint{gp mark 1}{(7.592,2.341)}
\gppoint{gp mark 1}{(7.594,2.306)}
\gppoint{gp mark 1}{(7.596,2.293)}
\gppoint{gp mark 1}{(7.599,2.286)}
\gppoint{gp mark 1}{(7.601,2.296)}
\gppoint{gp mark 1}{(7.603,2.295)}
\gppoint{gp mark 1}{(7.605,2.319)}
\gppoint{gp mark 1}{(7.608,2.298)}
\gppoint{gp mark 1}{(7.610,2.289)}
\gppoint{gp mark 1}{(7.612,2.283)}
\gppoint{gp mark 1}{(7.614,2.264)}
\gppoint{gp mark 1}{(7.617,2.271)}
\gppoint{gp mark 1}{(7.619,2.285)}
\gppoint{gp mark 1}{(7.621,2.260)}
\gppoint{gp mark 1}{(7.623,2.258)}
\gppoint{gp mark 1}{(7.626,2.249)}
\gppoint{gp mark 1}{(7.628,2.234)}
\gppoint{gp mark 1}{(7.630,2.250)}
\gppoint{gp mark 1}{(7.632,2.236)}
\gppoint{gp mark 1}{(7.635,2.230)}
\gppoint{gp mark 1}{(7.637,2.230)}
\gppoint{gp mark 1}{(7.639,2.221)}
\gppoint{gp mark 1}{(7.641,2.210)}
\gppoint{gp mark 1}{(7.644,2.228)}
\gppoint{gp mark 1}{(7.646,2.204)}
\gppoint{gp mark 1}{(7.648,2.207)}
\gppoint{gp mark 1}{(7.651,2.203)}
\gppoint{gp mark 1}{(7.653,2.189)}
\gppoint{gp mark 1}{(7.655,2.183)}
\gppoint{gp mark 1}{(7.657,2.170)}
\gppoint{gp mark 1}{(7.660,2.191)}
\gppoint{gp mark 1}{(7.662,2.175)}
\gppoint{gp mark 1}{(7.664,2.163)}
\gppoint{gp mark 1}{(7.666,2.154)}
\gppoint{gp mark 1}{(7.669,2.155)}
\gppoint{gp mark 1}{(7.671,2.150)}
\gppoint{gp mark 1}{(7.673,2.158)}
\gppoint{gp mark 1}{(7.675,2.136)}
\gppoint{gp mark 1}{(7.678,2.143)}
\gppoint{gp mark 1}{(7.680,2.141)}
\gppoint{gp mark 1}{(7.682,2.136)}
\gppoint{gp mark 1}{(7.684,2.116)}
\gppoint{gp mark 1}{(7.687,2.097)}
\gppoint{gp mark 1}{(7.689,2.140)}
\gppoint{gp mark 1}{(7.691,2.100)}
\gppoint{gp mark 1}{(7.693,2.104)}
\gppoint{gp mark 1}{(7.696,2.118)}
\gppoint{gp mark 1}{(7.698,2.096)}
\gppoint{gp mark 1}{(7.700,2.076)}
\gppoint{gp mark 1}{(7.702,2.080)}
\gppoint{gp mark 1}{(7.705,2.060)}
\gppoint{gp mark 1}{(7.707,2.092)}
\gppoint{gp mark 1}{(7.709,2.062)}
\gppoint{gp mark 1}{(7.711,2.059)}
\gppoint{gp mark 1}{(7.714,2.062)}
\gppoint{gp mark 1}{(7.716,2.047)}
\gppoint{gp mark 1}{(7.718,2.063)}
\gppoint{gp mark 1}{(7.720,2.060)}
\gppoint{gp mark 1}{(7.723,2.047)}
\gppoint{gp mark 1}{(7.725,2.046)}
\gppoint{gp mark 1}{(7.727,2.030)}
\gppoint{gp mark 1}{(7.729,2.029)}
\gppoint{gp mark 1}{(7.732,2.029)}
\gppoint{gp mark 1}{(7.734,2.017)}
\gppoint{gp mark 1}{(7.736,2.030)}
\gppoint{gp mark 1}{(7.738,2.011)}
\gppoint{gp mark 1}{(7.741,2.022)}
\gppoint{gp mark 1}{(7.743,1.998)}
\gppoint{gp mark 1}{(7.745,2.016)}
\gppoint{gp mark 1}{(7.747,1.984)}
\gppoint{gp mark 1}{(7.750,1.991)}
\gppoint{gp mark 1}{(7.752,2.003)}
\gppoint{gp mark 1}{(7.754,1.991)}
\gppoint{gp mark 1}{(7.756,1.984)}
\gppoint{gp mark 1}{(7.759,1.963)}
\gppoint{gp mark 1}{(7.761,1.984)}
\gppoint{gp mark 1}{(7.763,1.978)}
\gppoint{gp mark 1}{(7.765,1.972)}
\gppoint{gp mark 1}{(7.768,1.950)}
\gppoint{gp mark 1}{(7.770,1.958)}
\gppoint{gp mark 1}{(7.772,1.951)}
\gppoint{gp mark 1}{(7.774,1.948)}
\gppoint{gp mark 1}{(7.777,1.960)}
\gppoint{gp mark 1}{(7.779,1.958)}
\gppoint{gp mark 1}{(7.781,1.960)}
\gppoint{gp mark 1}{(7.783,1.955)}
\gppoint{gp mark 1}{(7.786,1.958)}
\gppoint{gp mark 1}{(7.788,1.921)}
\gppoint{gp mark 1}{(7.790,1.930)}
\gppoint{gp mark 1}{(7.792,1.923)}
\gppoint{gp mark 1}{(7.795,1.906)}
\gppoint{gp mark 1}{(7.797,1.918)}
\gppoint{gp mark 1}{(7.799,1.920)}
\gppoint{gp mark 1}{(7.801,1.911)}
\gppoint{gp mark 1}{(7.804,1.912)}
\gppoint{gp mark 1}{(7.806,1.896)}
\gppoint{gp mark 1}{(7.808,1.898)}
\gppoint{gp mark 1}{(7.810,1.896)}
\gppoint{gp mark 1}{(7.813,1.878)}
\gppoint{gp mark 1}{(7.815,1.896)}
\gppoint{gp mark 1}{(7.817,1.880)}
\gppoint{gp mark 1}{(7.819,1.874)}
\gppoint{gp mark 1}{(7.822,1.878)}
\gppoint{gp mark 1}{(7.824,1.877)}
\gppoint{gp mark 1}{(7.826,1.856)}
\gppoint{gp mark 1}{(7.828,1.858)}
\gppoint{gp mark 1}{(7.831,1.873)}
\gppoint{gp mark 1}{(7.833,1.852)}
\gppoint{gp mark 1}{(7.835,1.828)}
\gppoint{gp mark 1}{(7.837,1.851)}
\gppoint{gp mark 1}{(7.840,1.853)}
\gppoint{gp mark 1}{(7.842,1.859)}
\gppoint{gp mark 1}{(7.844,1.856)}
\gppoint{gp mark 1}{(7.846,1.842)}
\gppoint{gp mark 1}{(7.849,1.821)}
\gppoint{gp mark 1}{(7.851,1.821)}
\gppoint{gp mark 1}{(7.853,1.824)}
\gppoint{gp mark 1}{(7.855,1.809)}
\gppoint{gp mark 1}{(7.858,1.829)}
\gppoint{gp mark 1}{(7.860,1.834)}
\gppoint{gp mark 1}{(7.862,1.811)}
\gppoint{gp mark 1}{(7.864,1.806)}
\gppoint{gp mark 1}{(7.867,1.816)}
\gppoint{gp mark 1}{(7.869,1.805)}
\gppoint{gp mark 1}{(7.871,1.807)}
\gppoint{gp mark 1}{(7.873,1.807)}
\gppoint{gp mark 1}{(7.876,1.778)}
\gppoint{gp mark 1}{(7.878,1.799)}
\gppoint{gp mark 1}{(7.880,1.804)}
\gppoint{gp mark 1}{(7.882,1.787)}
\gppoint{gp mark 1}{(7.885,1.790)}
\gppoint{gp mark 1}{(7.887,1.815)}
\gppoint{gp mark 1}{(7.889,1.781)}
\gppoint{gp mark 1}{(7.891,1.780)}
\gppoint{gp mark 1}{(7.894,1.779)}
\gppoint{gp mark 1}{(7.896,1.764)}
\gppoint{gp mark 1}{(7.898,1.769)}
\gppoint{gp mark 1}{(7.900,1.765)}
\gppoint{gp mark 1}{(7.903,1.777)}
\gppoint{gp mark 1}{(7.905,1.780)}
\gppoint{gp mark 1}{(7.907,1.788)}
\gppoint{gp mark 1}{(7.909,1.744)}
\gppoint{gp mark 1}{(7.912,1.769)}
\gppoint{gp mark 1}{(7.914,1.765)}
\gppoint{gp mark 1}{(7.916,1.756)}
\gppoint{gp mark 1}{(7.918,1.758)}
\gppoint{gp mark 1}{(7.921,1.754)}
\gppoint{gp mark 1}{(7.923,1.753)}
\gppoint{gp mark 1}{(7.925,1.762)}
\gppoint{gp mark 1}{(7.927,1.762)}
\gppoint{gp mark 1}{(7.930,1.754)}
\gppoint{gp mark 1}{(7.932,1.770)}
\gppoint{gp mark 1}{(7.934,1.753)}
\gppoint{gp mark 1}{(7.936,1.735)}
\gppoint{gp mark 1}{(7.939,1.735)}
\gppoint{gp mark 1}{(7.941,1.733)}
\gppoint{gp mark 1}{(7.943,1.734)}
\gppoint{gp mark 1}{(7.945,1.735)}
\gppoint{gp mark 1}{(7.948,1.737)}
\gppoint{gp mark 1}{(7.950,1.726)}
\gppoint{gp mark 1}{(7.952,1.726)}
\gppoint{gp mark 1}{(7.954,1.720)}
\gppoint{gp mark 1}{(7.957,1.726)}
\gppoint{gp mark 1}{(7.959,1.736)}
\gppoint{gp mark 1}{(7.961,1.708)}
\gppoint{gp mark 1}{(7.963,1.714)}
\gppoint{gp mark 1}{(7.966,1.715)}
\gppoint{gp mark 1}{(7.968,1.696)}
\gppoint{gp mark 1}{(7.970,1.707)}
\gppoint{gp mark 1}{(7.972,1.692)}
\gppoint{gp mark 1}{(7.975,1.708)}
\gppoint{gp mark 1}{(7.977,1.703)}
\gppoint{gp mark 1}{(7.979,1.698)}
\gppoint{gp mark 1}{(7.981,1.701)}
\gppoint{gp mark 1}{(7.984,1.687)}
\gppoint{gp mark 1}{(7.986,1.700)}
\gppoint{gp mark 1}{(7.988,1.675)}
\gppoint{gp mark 1}{(7.990,1.701)}
\gppoint{gp mark 1}{(7.993,1.670)}
\gppoint{gp mark 1}{(7.995,1.677)}
\gppoint{gp mark 1}{(7.997,1.683)}
\gppoint{gp mark 1}{(7.999,1.682)}
\gppoint{gp mark 1}{(8.002,1.670)}
\gppoint{gp mark 1}{(8.004,1.658)}
\gppoint{gp mark 1}{(8.006,1.672)}
\gppoint{gp mark 1}{(8.008,1.666)}
\gppoint{gp mark 1}{(8.011,1.649)}
\gppoint{gp mark 1}{(8.013,1.653)}
\gppoint{gp mark 1}{(8.015,1.666)}
\gppoint{gp mark 1}{(8.017,1.652)}
\gppoint{gp mark 1}{(8.020,1.663)}
\gppoint{gp mark 1}{(8.022,1.638)}
\gppoint{gp mark 1}{(8.024,1.648)}
\gppoint{gp mark 1}{(8.026,1.673)}
\gppoint{gp mark 1}{(8.029,1.657)}
\gppoint{gp mark 1}{(8.031,1.648)}
\gppoint{gp mark 1}{(8.033,1.633)}
\gppoint{gp mark 1}{(8.035,1.644)}
\gppoint{gp mark 1}{(8.038,1.646)}
\gppoint{gp mark 1}{(8.040,1.642)}
\gppoint{gp mark 1}{(8.042,1.637)}
\gppoint{gp mark 1}{(8.044,1.632)}
\gppoint{gp mark 1}{(8.047,1.638)}
\gppoint{gp mark 1}{(8.049,1.637)}
\gppoint{gp mark 1}{(8.051,1.614)}
\gppoint{gp mark 1}{(8.053,1.631)}
\gppoint{gp mark 1}{(8.056,1.630)}
\gppoint{gp mark 1}{(8.058,1.620)}
\gppoint{gp mark 1}{(8.060,1.640)}
\gppoint{gp mark 1}{(8.062,1.601)}
\gppoint{gp mark 1}{(8.065,1.611)}
\gppoint{gp mark 1}{(8.067,1.596)}
\gppoint{gp mark 1}{(8.069,1.602)}
\gppoint{gp mark 1}{(8.071,1.608)}
\gppoint{gp mark 1}{(8.074,1.628)}
\gppoint{gp mark 1}{(8.076,1.619)}
\gppoint{gp mark 1}{(8.078,1.598)}
\gppoint{gp mark 1}{(8.080,1.609)}
\gppoint{gp mark 1}{(8.083,1.592)}
\gppoint{gp mark 1}{(8.085,1.592)}
\gppoint{gp mark 1}{(8.087,1.593)}
\gppoint{gp mark 1}{(8.089,1.603)}
\gppoint{gp mark 1}{(8.092,1.589)}
\gppoint{gp mark 1}{(8.094,1.588)}
\gppoint{gp mark 1}{(8.096,1.593)}
\gppoint{gp mark 1}{(8.098,1.591)}
\gppoint{gp mark 1}{(8.101,1.590)}
\gppoint{gp mark 1}{(8.103,1.589)}
\gppoint{gp mark 1}{(8.105,1.594)}
\gppoint{gp mark 1}{(8.107,1.586)}
\gppoint{gp mark 1}{(8.110,1.568)}
\gppoint{gp mark 1}{(8.112,1.571)}
\gppoint{gp mark 1}{(8.114,1.559)}
\gppoint{gp mark 1}{(8.116,1.564)}
\gppoint{gp mark 1}{(8.119,1.563)}
\gppoint{gp mark 1}{(8.121,1.555)}
\gppoint{gp mark 1}{(8.123,1.559)}
\gppoint{gp mark 1}{(8.125,1.579)}
\gppoint{gp mark 1}{(8.128,1.552)}
\gppoint{gp mark 1}{(8.130,1.557)}
\gppoint{gp mark 1}{(8.132,1.584)}
\gppoint{gp mark 1}{(8.134,1.571)}
\gppoint{gp mark 1}{(8.137,1.550)}
\gppoint{gp mark 1}{(8.139,1.566)}
\gppoint{gp mark 1}{(8.141,1.567)}
\gppoint{gp mark 1}{(8.143,1.562)}
\gppoint{gp mark 1}{(8.146,1.562)}
\gppoint{gp mark 1}{(8.148,1.561)}
\gppoint{gp mark 1}{(8.150,1.534)}
\gppoint{gp mark 1}{(8.152,1.559)}
\gppoint{gp mark 1}{(8.155,1.528)}
\gppoint{gp mark 1}{(8.157,1.545)}
\gppoint{gp mark 1}{(8.159,1.543)}
\gppoint{gp mark 1}{(8.161,1.534)}
\gppoint{gp mark 1}{(8.164,1.552)}
\gppoint{gp mark 1}{(8.166,1.536)}
\gppoint{gp mark 1}{(8.168,1.550)}
\gppoint{gp mark 1}{(8.170,1.530)}
\gppoint{gp mark 1}{(8.173,1.518)}
\gppoint{gp mark 1}{(8.175,1.533)}
\gppoint{gp mark 1}{(8.177,1.526)}
\gppoint{gp mark 1}{(8.179,1.521)}
\gppoint{gp mark 1}{(8.182,1.509)}
\gppoint{gp mark 1}{(8.184,1.545)}
\gppoint{gp mark 1}{(8.186,1.502)}
\gppoint{gp mark 1}{(8.188,1.519)}
\gppoint{gp mark 1}{(8.191,1.510)}
\gppoint{gp mark 1}{(8.193,1.517)}
\gppoint{gp mark 1}{(8.195,1.522)}
\gppoint{gp mark 1}{(8.197,1.508)}
\gppoint{gp mark 1}{(8.200,1.510)}
\gppoint{gp mark 1}{(8.202,1.501)}
\gppoint{gp mark 1}{(8.204,1.500)}
\gppoint{gp mark 1}{(8.206,1.495)}
\gppoint{gp mark 1}{(8.209,1.499)}
\gppoint{gp mark 1}{(8.211,1.509)}
\gppoint{gp mark 1}{(8.213,1.488)}
\gppoint{gp mark 1}{(8.215,1.483)}
\gppoint{gp mark 1}{(8.218,1.484)}
\gppoint{gp mark 1}{(8.220,1.498)}
\gppoint{gp mark 1}{(8.222,1.502)}
\gppoint{gp mark 1}{(8.224,1.499)}
\gppoint{gp mark 1}{(8.227,1.476)}
\gppoint{gp mark 1}{(8.229,1.494)}
\gppoint{gp mark 1}{(8.231,1.500)}
\gppoint{gp mark 1}{(8.233,1.501)}
\gppoint{gp mark 1}{(8.236,1.496)}
\gppoint{gp mark 1}{(8.238,1.492)}
\gppoint{gp mark 1}{(8.240,1.483)}
\gppoint{gp mark 1}{(8.242,1.493)}
\gppoint{gp mark 1}{(8.245,1.486)}
\gppoint{gp mark 1}{(8.247,1.472)}
\gppoint{gp mark 1}{(8.249,1.484)}
\gppoint{gp mark 1}{(8.251,1.485)}
\gppoint{gp mark 1}{(8.254,1.482)}
\gppoint{gp mark 1}{(8.256,1.486)}
\gppoint{gp mark 1}{(8.258,1.473)}
\gppoint{gp mark 1}{(8.260,1.469)}
\gppoint{gp mark 1}{(8.263,1.459)}
\gppoint{gp mark 1}{(8.265,1.466)}
\gppoint{gp mark 1}{(8.267,1.471)}
\gppoint{gp mark 1}{(8.269,1.483)}
\gppoint{gp mark 1}{(8.272,1.464)}
\gppoint{gp mark 1}{(8.274,1.464)}
\gppoint{gp mark 1}{(8.276,1.450)}
\gppoint{gp mark 1}{(8.278,1.473)}
\gppoint{gp mark 1}{(8.281,1.452)}
\gppoint{gp mark 1}{(8.283,1.469)}
\gppoint{gp mark 1}{(8.285,1.460)}
\gppoint{gp mark 1}{(8.287,1.446)}
\gppoint{gp mark 1}{(8.290,1.464)}
\gppoint{gp mark 1}{(8.292,1.450)}
\gppoint{gp mark 1}{(8.294,1.446)}
\gppoint{gp mark 1}{(8.296,1.469)}
\gppoint{gp mark 1}{(8.299,1.459)}
\gppoint{gp mark 1}{(8.301,1.438)}
\gppoint{gp mark 1}{(8.303,1.475)}
\gppoint{gp mark 1}{(8.305,1.453)}
\gppoint{gp mark 1}{(8.308,1.453)}
\gppoint{gp mark 1}{(8.310,1.439)}
\gppoint{gp mark 1}{(8.312,1.456)}
\gppoint{gp mark 1}{(8.314,1.458)}
\gppoint{gp mark 1}{(8.317,1.454)}
\gppoint{gp mark 1}{(8.319,1.443)}
\gppoint{gp mark 1}{(8.321,1.450)}
\gppoint{gp mark 1}{(8.323,1.440)}
\gppoint{gp mark 1}{(8.326,1.437)}
\gppoint{gp mark 1}{(8.328,1.446)}
\gppoint{gp mark 1}{(8.330,1.431)}
\gppoint{gp mark 1}{(8.332,1.442)}
\gppoint{gp mark 1}{(8.335,1.436)}
\gppoint{gp mark 1}{(8.337,1.434)}
\gppoint{gp mark 1}{(8.339,1.445)}
\gppoint{gp mark 1}{(8.341,1.447)}
\gppoint{gp mark 1}{(8.344,1.427)}
\gppoint{gp mark 1}{(8.346,1.438)}
\gppoint{gp mark 1}{(8.348,1.463)}
\gppoint{gp mark 1}{(8.350,1.440)}
\gppoint{gp mark 1}{(8.353,1.428)}
\gppoint{gp mark 1}{(8.355,1.429)}
\gppoint{gp mark 1}{(8.357,1.420)}
\gppoint{gp mark 1}{(8.359,1.434)}
\gppoint{gp mark 1}{(8.362,1.430)}
\gppoint{gp mark 1}{(8.364,1.422)}
\gppoint{gp mark 1}{(8.366,1.432)}
\gppoint{gp mark 1}{(8.368,1.428)}
\gppoint{gp mark 1}{(8.371,1.426)}
\gppoint{gp mark 1}{(8.373,1.417)}
\gppoint{gp mark 1}{(8.375,1.446)}
\gppoint{gp mark 1}{(8.377,1.438)}
\gppoint{gp mark 1}{(8.380,1.434)}
\gppoint{gp mark 1}{(8.382,1.438)}
\gppoint{gp mark 1}{(8.384,1.433)}
\gppoint{gp mark 1}{(8.386,1.431)}
\gppoint{gp mark 1}{(8.389,1.426)}
\gppoint{gp mark 1}{(8.391,1.434)}
\gppoint{gp mark 1}{(8.393,1.444)}
\gppoint{gp mark 1}{(8.395,1.438)}
\gppoint{gp mark 1}{(8.398,1.424)}
\gppoint{gp mark 1}{(8.400,1.441)}
\gppoint{gp mark 1}{(8.402,1.426)}
\gppoint{gp mark 1}{(8.404,1.419)}
\gppoint{gp mark 1}{(8.407,1.439)}
\gppoint{gp mark 1}{(8.409,1.429)}
\gppoint{gp mark 1}{(8.411,1.419)}
\gppoint{gp mark 1}{(8.413,1.414)}
\gppoint{gp mark 1}{(8.416,1.432)}
\gppoint{gp mark 1}{(8.418,1.433)}
\gppoint{gp mark 1}{(8.420,1.433)}
\gppoint{gp mark 1}{(8.422,1.429)}
\gppoint{gp mark 1}{(8.425,1.414)}
\gppoint{gp mark 1}{(8.427,1.418)}
\gppoint{gp mark 1}{(8.429,1.408)}
\gppoint{gp mark 1}{(8.431,1.417)}
\gppoint{gp mark 1}{(8.434,1.416)}
\gppoint{gp mark 1}{(8.436,1.423)}
\gppoint{gp mark 1}{(8.438,1.404)}
\gppoint{gp mark 1}{(8.440,1.422)}
\gppoint{gp mark 1}{(8.443,1.422)}
\gppoint{gp mark 1}{(8.445,1.431)}
\gppoint{gp mark 1}{(8.447,1.415)}
\gppoint{gp mark 1}{(8.449,1.420)}
\gppoint{gp mark 1}{(8.452,1.403)}
\gppoint{gp mark 1}{(8.454,1.428)}
\gppoint{gp mark 1}{(8.456,1.414)}
\gppoint{gp mark 1}{(8.458,1.425)}
\gppoint{gp mark 1}{(8.461,1.423)}
\gppoint{gp mark 1}{(8.463,1.419)}
\gppoint{gp mark 1}{(8.465,1.422)}
\gppoint{gp mark 1}{(8.467,1.423)}
\gppoint{gp mark 1}{(8.470,1.414)}
\gppoint{gp mark 1}{(8.472,1.420)}
\gppoint{gp mark 1}{(8.474,1.435)}
\gppoint{gp mark 1}{(8.476,1.418)}
\gppoint{gp mark 1}{(8.479,1.401)}
\gppoint{gp mark 1}{(8.481,1.421)}
\gppoint{gp mark 1}{(8.483,1.413)}
\gppoint{gp mark 1}{(8.485,1.414)}
\gppoint{gp mark 1}{(8.488,1.422)}
\gppoint{gp mark 1}{(8.490,1.410)}
\gppoint{gp mark 1}{(8.492,1.412)}
\gppoint{gp mark 1}{(8.494,1.417)}
\gppoint{gp mark 1}{(8.497,1.411)}
\gppoint{gp mark 1}{(8.499,1.409)}
\gppoint{gp mark 1}{(8.501,1.398)}
\gppoint{gp mark 1}{(8.503,1.390)}
\gppoint{gp mark 1}{(8.506,1.420)}
\gppoint{gp mark 1}{(8.508,1.419)}
\gppoint{gp mark 1}{(8.510,1.430)}
\gppoint{gp mark 1}{(8.512,1.407)}
\gppoint{gp mark 1}{(8.515,1.420)}
\gppoint{gp mark 1}{(8.517,1.419)}
\gppoint{gp mark 1}{(8.519,1.394)}
\gppoint{gp mark 1}{(8.521,1.410)}
\gppoint{gp mark 1}{(8.524,1.417)}
\gppoint{gp mark 1}{(8.526,1.404)}
\gppoint{gp mark 1}{(8.528,1.404)}
\gppoint{gp mark 1}{(8.530,1.412)}
\gppoint{gp mark 1}{(8.533,1.400)}
\gppoint{gp mark 1}{(8.535,1.405)}
\gppoint{gp mark 1}{(8.537,1.411)}
\gppoint{gp mark 1}{(8.539,1.402)}
\gppoint{gp mark 1}{(8.542,1.410)}
\gppoint{gp mark 1}{(8.544,1.424)}
\gppoint{gp mark 1}{(8.546,1.402)}
\gppoint{gp mark 1}{(8.548,1.394)}
\gppoint{gp mark 1}{(8.551,1.411)}
\gppoint{gp mark 1}{(8.553,1.406)}
\gppoint{gp mark 1}{(8.555,1.426)}
\gppoint{gp mark 1}{(8.557,1.411)}
\gppoint{gp mark 1}{(8.560,1.422)}
\gppoint{gp mark 1}{(8.562,1.406)}
\gppoint{gp mark 1}{(8.564,1.395)}
\gppoint{gp mark 1}{(8.566,1.408)}
\gppoint{gp mark 1}{(8.569,1.417)}
\gppoint{gp mark 1}{(8.571,1.399)}
\gppoint{gp mark 1}{(8.573,1.410)}
\gppoint{gp mark 1}{(8.575,1.413)}
\gppoint{gp mark 1}{(8.578,1.409)}
\gppoint{gp mark 1}{(8.580,1.427)}
\gppoint{gp mark 1}{(8.582,1.424)}
\gppoint{gp mark 1}{(8.584,1.398)}
\gppoint{gp mark 1}{(8.587,1.404)}
\gppoint{gp mark 1}{(8.589,1.419)}
\gppoint{gp mark 1}{(8.591,1.411)}
\gppoint{gp mark 1}{(8.593,1.407)}
\gppoint{gp mark 1}{(8.596,1.411)}
\gppoint{gp mark 1}{(8.598,1.412)}
\gppoint{gp mark 1}{(8.600,1.408)}
\gppoint{gp mark 1}{(8.602,1.406)}
\gppoint{gp mark 1}{(8.605,1.393)}
\gppoint{gp mark 1}{(8.607,1.408)}
\gppoint{gp mark 1}{(8.609,1.424)}
\gppoint{gp mark 1}{(8.611,1.422)}
\gppoint{gp mark 1}{(8.614,1.393)}
\gppoint{gp mark 1}{(8.616,1.407)}
\gppoint{gp mark 1}{(8.618,1.403)}
\gppoint{gp mark 1}{(8.620,1.403)}
\gppoint{gp mark 1}{(8.623,1.408)}
\gppoint{gp mark 1}{(8.625,1.401)}
\gppoint{gp mark 1}{(8.627,1.398)}
\gppoint{gp mark 1}{(8.629,1.407)}
\gppoint{gp mark 1}{(8.632,1.393)}
\gppoint{gp mark 1}{(8.634,1.397)}
\gppoint{gp mark 1}{(8.636,1.401)}
\gppoint{gp mark 1}{(8.638,1.405)}
\gppoint{gp mark 1}{(8.641,1.406)}
\gppoint{gp mark 1}{(8.643,1.393)}
\gppoint{gp mark 1}{(8.645,1.417)}
\gppoint{gp mark 1}{(8.647,1.391)}
\gppoint{gp mark 1}{(8.650,1.405)}
\gppoint{gp mark 1}{(8.652,1.391)}
\gppoint{gp mark 1}{(8.654,1.403)}
\gppoint{gp mark 1}{(8.656,1.411)}
\gppoint{gp mark 1}{(8.659,1.415)}
\gppoint{gp mark 1}{(8.661,1.405)}
\gppoint{gp mark 1}{(8.663,1.402)}
\gppoint{gp mark 1}{(8.665,1.418)}
\gppoint{gp mark 1}{(8.668,1.423)}
\gppoint{gp mark 1}{(8.670,1.404)}
\gppoint{gp mark 1}{(8.672,1.382)}
\gppoint{gp mark 1}{(8.674,1.413)}
\gppoint{gp mark 1}{(8.677,1.405)}
\gppoint{gp mark 1}{(8.679,1.417)}
\gppoint{gp mark 1}{(8.681,1.411)}
\gppoint{gp mark 1}{(8.683,1.421)}
\gppoint{gp mark 1}{(8.686,1.399)}
\gppoint{gp mark 1}{(8.688,1.415)}
\gppoint{gp mark 1}{(8.690,1.396)}
\gppoint{gp mark 1}{(8.692,1.390)}
\gppoint{gp mark 1}{(8.695,1.396)}
\gppoint{gp mark 1}{(8.697,1.427)}
\gppoint{gp mark 1}{(8.699,1.403)}
\gppoint{gp mark 1}{(8.701,1.421)}
\gppoint{gp mark 1}{(8.704,1.399)}
\gppoint{gp mark 1}{(8.706,1.408)}
\gppoint{gp mark 1}{(8.708,1.398)}
\gppoint{gp mark 1}{(8.710,1.409)}
\gppoint{gp mark 1}{(8.713,1.401)}
\gppoint{gp mark 1}{(8.715,1.410)}
\gppoint{gp mark 1}{(8.717,1.410)}
\gppoint{gp mark 1}{(8.719,1.407)}
\gppoint{gp mark 1}{(8.722,1.411)}
\gppoint{gp mark 1}{(8.724,1.412)}
\gppoint{gp mark 1}{(8.726,1.409)}
\gppoint{gp mark 1}{(8.728,1.400)}
\gppoint{gp mark 1}{(8.731,1.426)}
\gppoint{gp mark 1}{(8.733,1.412)}
\gppoint{gp mark 1}{(8.735,1.401)}
\gppoint{gp mark 1}{(8.737,1.404)}
\gppoint{gp mark 1}{(8.740,1.401)}
\gppoint{gp mark 1}{(8.742,1.427)}
\gppoint{gp mark 1}{(8.744,1.409)}
\gppoint{gp mark 1}{(8.746,1.417)}
\gppoint{gp mark 1}{(8.749,1.410)}
\gppoint{gp mark 1}{(8.751,1.409)}
\gppoint{gp mark 1}{(8.753,1.412)}
\gppoint{gp mark 1}{(8.755,1.399)}
\gppoint{gp mark 1}{(8.758,1.429)}
\gppoint{gp mark 1}{(8.760,1.430)}
\gppoint{gp mark 1}{(8.762,1.416)}
\gppoint{gp mark 1}{(8.764,1.408)}
\gppoint{gp mark 1}{(8.767,1.407)}
\gppoint{gp mark 1}{(8.769,1.414)}
\gppoint{gp mark 1}{(8.771,1.433)}
\gppoint{gp mark 1}{(8.773,1.394)}
\gppoint{gp mark 1}{(8.776,1.415)}
\gppoint{gp mark 1}{(8.778,1.408)}
\gppoint{gp mark 1}{(8.780,1.404)}
\gppoint{gp mark 1}{(8.782,1.420)}
\gppoint{gp mark 1}{(8.785,1.393)}
\gppoint{gp mark 1}{(8.787,1.413)}
\gppoint{gp mark 1}{(8.789,1.413)}
\gppoint{gp mark 1}{(8.791,1.412)}
\gppoint{gp mark 1}{(8.794,1.405)}
\gppoint{gp mark 1}{(8.796,1.398)}
\gppoint{gp mark 1}{(8.798,1.428)}
\gppoint{gp mark 1}{(8.800,1.421)}
\gppoint{gp mark 1}{(8.803,1.420)}
\gppoint{gp mark 1}{(8.805,1.413)}
\gppoint{gp mark 1}{(8.807,1.409)}
\gppoint{gp mark 1}{(8.809,1.424)}
\gppoint{gp mark 1}{(8.812,1.428)}
\gppoint{gp mark 1}{(8.814,1.406)}
\gppoint{gp mark 1}{(8.816,1.399)}
\gppoint{gp mark 1}{(8.818,1.425)}
\gppoint{gp mark 1}{(8.821,1.417)}
\gppoint{gp mark 1}{(8.823,1.413)}
\gppoint{gp mark 1}{(8.825,1.414)}
\gppoint{gp mark 1}{(8.827,1.412)}
\gppoint{gp mark 1}{(8.830,1.415)}
\gppoint{gp mark 1}{(8.832,1.416)}
\gppoint{gp mark 1}{(8.834,1.402)}
\gppoint{gp mark 1}{(8.836,1.418)}
\gppoint{gp mark 1}{(8.839,1.427)}
\gppoint{gp mark 1}{(8.841,1.420)}
\gppoint{gp mark 1}{(8.843,1.426)}
\gppoint{gp mark 1}{(8.845,1.400)}
\gppoint{gp mark 1}{(8.848,1.410)}
\gppoint{gp mark 1}{(8.850,1.425)}
\gppoint{gp mark 1}{(8.852,1.425)}
\gppoint{gp mark 1}{(8.854,1.419)}
\gppoint{gp mark 1}{(8.857,1.431)}
\gppoint{gp mark 1}{(8.859,1.409)}
\gppoint{gp mark 1}{(8.861,1.431)}
\gppoint{gp mark 1}{(8.863,1.409)}
\gppoint{gp mark 1}{(8.866,1.413)}
\gppoint{gp mark 1}{(8.868,1.422)}
\gppoint{gp mark 1}{(8.870,1.412)}
\gppoint{gp mark 1}{(8.872,1.415)}
\gppoint{gp mark 1}{(8.875,1.433)}
\gppoint{gp mark 1}{(8.877,1.412)}
\gppoint{gp mark 1}{(8.879,1.425)}
\gppoint{gp mark 1}{(8.881,1.407)}
\gppoint{gp mark 1}{(8.884,1.398)}
\gppoint{gp mark 1}{(8.886,1.407)}
\gppoint{gp mark 1}{(8.888,1.414)}
\gppoint{gp mark 1}{(8.890,1.422)}
\gppoint{gp mark 1}{(8.893,1.412)}
\gppoint{gp mark 1}{(8.895,1.409)}
\gppoint{gp mark 1}{(8.897,1.415)}
\gppoint{gp mark 1}{(8.899,1.423)}
\gppoint{gp mark 1}{(8.902,1.417)}
\gppoint{gp mark 1}{(8.904,1.420)}
\gppoint{gp mark 1}{(8.906,1.416)}
\gppoint{gp mark 1}{(8.908,1.428)}
\gppoint{gp mark 1}{(8.911,1.413)}
\gppoint{gp mark 1}{(8.913,1.423)}
\gppoint{gp mark 1}{(8.915,1.421)}
\gppoint{gp mark 1}{(8.918,1.426)}
\gppoint{gp mark 1}{(8.920,1.423)}
\gppoint{gp mark 1}{(8.922,1.412)}
\gppoint{gp mark 1}{(8.924,1.403)}
\gppoint{gp mark 1}{(8.927,1.406)}
\gppoint{gp mark 1}{(8.929,1.422)}
\gppoint{gp mark 1}{(8.931,1.416)}
\gppoint{gp mark 1}{(8.933,1.436)}
\gppoint{gp mark 1}{(8.936,1.442)}
\gppoint{gp mark 1}{(8.938,1.426)}
\gppoint{gp mark 1}{(8.940,1.420)}
\gppoint{gp mark 1}{(8.942,1.441)}
\gppoint{gp mark 1}{(8.945,1.418)}
\gppoint{gp mark 1}{(8.947,1.426)}
\gppoint{gp mark 1}{(8.949,1.418)}
\gppoint{gp mark 1}{(8.951,1.413)}
\gppoint{gp mark 1}{(8.954,1.411)}
\gppoint{gp mark 1}{(8.956,1.424)}
\gppoint{gp mark 1}{(8.958,1.414)}
\gppoint{gp mark 1}{(8.960,1.424)}
\gppoint{gp mark 1}{(8.963,1.412)}
\gppoint{gp mark 1}{(8.965,1.416)}
\gppoint{gp mark 1}{(8.967,1.432)}
\gppoint{gp mark 1}{(8.969,1.440)}
\gppoint{gp mark 1}{(8.972,1.418)}
\gppoint{gp mark 1}{(8.974,1.423)}
\gppoint{gp mark 1}{(8.976,1.431)}
\gppoint{gp mark 1}{(8.978,1.436)}
\gppoint{gp mark 1}{(8.981,1.437)}
\gppoint{gp mark 1}{(8.983,1.423)}
\gppoint{gp mark 1}{(8.985,1.423)}
\gppoint{gp mark 1}{(8.987,1.436)}
\gppoint{gp mark 1}{(8.990,1.417)}
\gppoint{gp mark 1}{(8.992,1.419)}
\gppoint{gp mark 1}{(8.994,1.435)}
\gppoint{gp mark 1}{(8.996,1.412)}
\gppoint{gp mark 1}{(8.999,1.419)}
\gppoint{gp mark 1}{(9.001,1.416)}
\gppoint{gp mark 1}{(9.003,1.431)}
\gppoint{gp mark 1}{(9.005,1.418)}
\gppoint{gp mark 1}{(9.008,1.420)}
\gppoint{gp mark 1}{(9.010,1.421)}
\gppoint{gp mark 1}{(9.012,1.435)}
\gppoint{gp mark 1}{(9.014,1.421)}
\gppoint{gp mark 1}{(9.017,1.420)}
\gppoint{gp mark 1}{(9.019,1.427)}
\gppoint{gp mark 1}{(9.021,1.406)}
\gppoint{gp mark 1}{(9.023,1.423)}
\gppoint{gp mark 1}{(9.026,1.439)}
\gppoint{gp mark 1}{(9.028,1.412)}
\gppoint{gp mark 1}{(9.030,1.417)}
\gppoint{gp mark 1}{(9.032,1.429)}
\gppoint{gp mark 1}{(9.035,1.436)}
\gppoint{gp mark 1}{(9.037,1.433)}
\gppoint{gp mark 1}{(9.039,1.414)}
\gppoint{gp mark 1}{(9.041,1.390)}
\gppoint{gp mark 1}{(9.044,1.367)}
\gppoint{gp mark 1}{(9.046,1.356)}
\gppoint{gp mark 1}{(9.048,1.362)}
\gppoint{gp mark 1}{(9.050,1.381)}
\gppoint{gp mark 1}{(9.053,1.363)}
\gppoint{gp mark 1}{(9.055,1.370)}
\gppoint{gp mark 1}{(9.057,1.369)}
\gppoint{gp mark 1}{(9.059,1.362)}
\gppoint{gp mark 1}{(9.062,1.363)}
\gppoint{gp mark 1}{(9.064,1.367)}
\gppoint{gp mark 1}{(9.066,1.355)}
\gppoint{gp mark 1}{(9.068,1.352)}
\gppoint{gp mark 1}{(9.071,1.339)}
\gppoint{gp mark 1}{(9.073,1.354)}
\gppoint{gp mark 1}{(9.075,1.358)}
\gppoint{gp mark 1}{(9.077,1.353)}
\gppoint{gp mark 1}{(9.080,1.376)}
\gppoint{gp mark 1}{(9.082,1.360)}
\gppoint{gp mark 1}{(9.084,1.356)}
\gppoint{gp mark 1}{(9.086,1.346)}
\gppoint{gp mark 1}{(9.089,1.355)}
\gppoint{gp mark 1}{(9.091,1.362)}
\gppoint{gp mark 1}{(9.093,1.375)}
\gppoint{gp mark 1}{(9.095,1.357)}
\gppoint{gp mark 1}{(9.098,1.357)}
\gppoint{gp mark 1}{(9.100,1.368)}
\gppoint{gp mark 1}{(9.102,1.353)}
\gppoint{gp mark 1}{(9.104,1.379)}
\gppoint{gp mark 1}{(9.107,1.361)}
\gppoint{gp mark 1}{(9.109,1.370)}
\gppoint{gp mark 1}{(9.111,1.354)}
\gppoint{gp mark 1}{(9.113,1.360)}
\gppoint{gp mark 1}{(9.116,1.379)}
\gppoint{gp mark 1}{(9.118,1.356)}
\gppoint{gp mark 1}{(9.120,1.349)}
\gppoint{gp mark 1}{(9.122,1.354)}
\gppoint{gp mark 1}{(9.125,1.337)}
\gppoint{gp mark 1}{(9.127,1.361)}
\gppoint{gp mark 1}{(9.129,1.376)}
\gppoint{gp mark 1}{(9.131,1.361)}
\gppoint{gp mark 1}{(9.134,1.362)}
\gppoint{gp mark 1}{(9.136,1.358)}
\gppoint{gp mark 1}{(9.138,1.354)}
\gppoint{gp mark 1}{(9.140,1.348)}
\gppoint{gp mark 1}{(9.143,1.354)}
\gppoint{gp mark 1}{(9.145,1.354)}
\gppoint{gp mark 1}{(9.147,1.343)}
\gppoint{gp mark 1}{(9.149,1.365)}
\gppoint{gp mark 1}{(9.152,1.356)}
\gppoint{gp mark 1}{(9.154,1.354)}
\gppoint{gp mark 1}{(9.156,1.363)}
\gppoint{gp mark 1}{(9.158,1.368)}
\gppoint{gp mark 1}{(9.161,1.365)}
\gppoint{gp mark 1}{(9.163,1.358)}
\gppoint{gp mark 1}{(9.165,1.361)}
\gppoint{gp mark 1}{(9.167,1.365)}
\gppoint{gp mark 1}{(9.170,1.365)}
\gppoint{gp mark 1}{(9.172,1.362)}
\gppoint{gp mark 1}{(9.174,1.355)}
\gppoint{gp mark 1}{(9.176,1.354)}
\gppoint{gp mark 1}{(9.179,1.338)}
\gppoint{gp mark 1}{(9.181,1.342)}
\gppoint{gp mark 1}{(9.183,1.369)}
\gppoint{gp mark 1}{(9.185,1.351)}
\gppoint{gp mark 1}{(9.188,1.360)}
\gppoint{gp mark 1}{(9.190,1.360)}
\gppoint{gp mark 1}{(9.192,1.374)}
\gppoint{gp mark 1}{(9.194,1.336)}
\gppoint{gp mark 1}{(9.197,1.348)}
\gppoint{gp mark 1}{(9.199,1.365)}
\gppoint{gp mark 1}{(9.201,1.365)}
\gppoint{gp mark 1}{(9.203,1.360)}
\gppoint{gp mark 1}{(9.206,1.353)}
\gppoint{gp mark 1}{(9.208,1.390)}
\gppoint{gp mark 1}{(9.210,1.358)}
\gppoint{gp mark 1}{(9.212,1.360)}
\gppoint{gp mark 1}{(9.215,1.350)}
\gppoint{gp mark 1}{(9.217,1.352)}
\gppoint{gp mark 1}{(9.219,1.348)}
\gppoint{gp mark 1}{(9.221,1.362)}
\gppoint{gp mark 1}{(9.224,1.358)}
\gppoint{gp mark 1}{(9.226,1.355)}
\gppoint{gp mark 1}{(9.228,1.362)}
\gppoint{gp mark 1}{(9.230,1.367)}
\gppoint{gp mark 1}{(9.233,1.365)}
\gppoint{gp mark 1}{(9.235,1.344)}
\gppoint{gp mark 1}{(9.237,1.360)}
\gppoint{gp mark 1}{(9.239,1.376)}
\gppoint{gp mark 1}{(9.242,1.352)}
\gppoint{gp mark 1}{(9.244,1.355)}
\gppoint{gp mark 1}{(9.246,1.369)}
\gppoint{gp mark 1}{(9.248,1.369)}
\gppoint{gp mark 1}{(9.251,1.360)}
\gppoint{gp mark 1}{(9.253,1.363)}
\gppoint{gp mark 1}{(9.255,1.363)}
\gppoint{gp mark 1}{(9.257,1.375)}
\gppoint{gp mark 1}{(9.260,1.373)}
\gppoint{gp mark 1}{(9.262,1.375)}
\gppoint{gp mark 1}{(9.264,1.363)}
\gppoint{gp mark 1}{(9.266,1.370)}
\gppoint{gp mark 1}{(9.269,1.366)}
\gppoint{gp mark 1}{(9.271,1.359)}
\gppoint{gp mark 1}{(9.273,1.356)}
\gppoint{gp mark 1}{(9.275,1.371)}
\gppoint{gp mark 1}{(9.278,1.349)}
\gppoint{gp mark 1}{(9.280,1.376)}
\gppoint{gp mark 1}{(9.282,1.378)}
\gppoint{gp mark 1}{(9.284,1.381)}
\gppoint{gp mark 1}{(9.287,1.392)}
\gppoint{gp mark 1}{(9.289,1.385)}
\gppoint{gp mark 1}{(9.291,1.380)}
\gppoint{gp mark 1}{(9.293,1.367)}
\gppoint{gp mark 1}{(9.296,1.364)}
\gppoint{gp mark 1}{(9.298,1.382)}
\gppoint{gp mark 1}{(9.300,1.375)}
\gppoint{gp mark 1}{(9.302,1.386)}
\gppoint{gp mark 1}{(9.305,1.383)}
\gppoint{gp mark 1}{(9.307,1.368)}
\gppoint{gp mark 1}{(9.309,1.375)}
\gppoint{gp mark 1}{(9.311,1.366)}
\gppoint{gp mark 1}{(9.314,1.362)}
\gppoint{gp mark 1}{(9.316,1.381)}
\gppoint{gp mark 1}{(9.318,1.391)}
\gppoint{gp mark 1}{(9.320,1.399)}
\gppoint{gp mark 1}{(9.323,1.370)}
\gppoint{gp mark 1}{(9.325,1.382)}
\gppoint{gp mark 1}{(9.327,1.373)}
\gppoint{gp mark 1}{(9.329,1.369)}
\gppoint{gp mark 1}{(9.332,1.372)}
\gppoint{gp mark 1}{(9.334,1.395)}
\gppoint{gp mark 1}{(9.336,1.381)}
\gppoint{gp mark 1}{(9.338,1.385)}
\gppoint{gp mark 1}{(9.341,1.376)}
\gppoint{gp mark 1}{(9.343,1.384)}
\gppoint{gp mark 1}{(9.345,1.361)}
\gppoint{gp mark 1}{(9.347,1.396)}
\gppoint{gp mark 1}{(9.350,1.376)}
\gppoint{gp mark 1}{(9.352,1.398)}
\gppoint{gp mark 1}{(9.354,1.366)}
\gppoint{gp mark 1}{(9.356,1.389)}
\gppoint{gp mark 1}{(9.359,1.393)}
\gppoint{gp mark 1}{(9.361,1.382)}
\gppoint{gp mark 1}{(9.363,1.395)}
\gppoint{gp mark 1}{(9.365,1.379)}
\gppoint{gp mark 1}{(9.368,1.364)}
\gppoint{gp mark 1}{(9.370,1.371)}
\gppoint{gp mark 1}{(9.372,1.384)}
\gppoint{gp mark 1}{(9.374,1.372)}
\gppoint{gp mark 1}{(9.377,1.372)}
\gppoint{gp mark 1}{(9.379,1.376)}
\gppoint{gp mark 1}{(9.381,1.376)}
\gppoint{gp mark 1}{(9.383,1.361)}
\gppoint{gp mark 1}{(9.386,1.366)}
\gppoint{gp mark 1}{(9.388,1.374)}
\gppoint{gp mark 1}{(9.390,1.375)}
\gppoint{gp mark 1}{(9.392,1.367)}
\gppoint{gp mark 1}{(9.395,1.379)}
\gppoint{gp mark 1}{(9.397,1.372)}
\gppoint{gp mark 1}{(9.399,1.400)}
\gppoint{gp mark 1}{(9.401,1.384)}
\gppoint{gp mark 1}{(9.404,1.390)}
\gppoint{gp mark 1}{(9.406,1.410)}
\gppoint{gp mark 1}{(9.408,1.378)}
\gppoint{gp mark 1}{(9.410,1.398)}
\gppoint{gp mark 1}{(9.413,1.381)}
\gppoint{gp mark 1}{(9.415,1.393)}
\gppoint{gp mark 1}{(9.417,1.392)}
\gppoint{gp mark 1}{(9.419,1.392)}
\gppoint{gp mark 1}{(9.422,1.386)}
\gppoint{gp mark 1}{(9.424,1.381)}
\gppoint{gp mark 1}{(9.426,1.378)}
\gppoint{gp mark 1}{(9.428,1.379)}
\gppoint{gp mark 1}{(9.431,1.383)}
\gppoint{gp mark 1}{(9.433,1.382)}
\gppoint{gp mark 1}{(9.435,1.377)}
\gppoint{gp mark 1}{(9.437,1.386)}
\gppoint{gp mark 1}{(9.440,1.401)}
\gppoint{gp mark 1}{(9.442,1.385)}
\gppoint{gp mark 1}{(9.444,1.387)}
\gppoint{gp mark 1}{(9.446,1.376)}
\gppoint{gp mark 1}{(9.449,1.384)}
\gppoint{gp mark 1}{(9.451,1.385)}
\gppoint{gp mark 1}{(9.453,1.391)}
\gppoint{gp mark 1}{(9.455,1.401)}
\gppoint{gp mark 1}{(9.458,1.397)}
\gppoint{gp mark 1}{(9.460,1.394)}
\gppoint{gp mark 1}{(9.462,1.392)}
\gppoint{gp mark 1}{(9.464,1.375)}
\gppoint{gp mark 1}{(9.467,1.380)}
\gppoint{gp mark 1}{(9.469,3.457)}
\gppoint{gp mark 1}{(9.471,7.013)}
\gppoint{gp mark 1}{(9.473,7.031)}
\gppoint{gp mark 1}{(9.476,7.049)}
\gppoint{gp mark 1}{(9.478,7.049)}
\gppoint{gp mark 1}{(9.480,7.059)}
\gppoint{gp mark 1}{(9.482,7.030)}
\gppoint{gp mark 1}{(9.485,7.035)}
\gppoint{gp mark 1}{(9.487,7.044)}
\gppoint{gp mark 1}{(9.489,7.051)}
\gppoint{gp mark 1}{(9.491,7.047)}
\gppoint{gp mark 1}{(9.494,7.039)}
\gppoint{gp mark 1}{(9.496,7.061)}
\gppoint{gp mark 1}{(9.498,7.076)}
\gppoint{gp mark 1}{(9.500,7.064)}
\gppoint{gp mark 1}{(9.503,7.065)}
\gppoint{gp mark 1}{(9.505,7.069)}
\gppoint{gp mark 1}{(9.507,7.063)}
\gppoint{gp mark 1}{(9.509,7.053)}
\gppoint{gp mark 1}{(9.512,7.073)}
\gppoint{gp mark 1}{(9.514,7.050)}
\gppoint{gp mark 1}{(9.516,7.054)}
\gppoint{gp mark 1}{(9.518,7.052)}
\gppoint{gp mark 1}{(9.521,7.101)}
\gppoint{gp mark 1}{(9.523,7.086)}
\gppoint{gp mark 1}{(9.525,7.077)}
\gppoint{gp mark 1}{(9.527,7.056)}
\gppoint{gp mark 1}{(9.530,7.073)}
\gppoint{gp mark 1}{(9.532,7.086)}
\gppoint{gp mark 1}{(9.534,7.076)}
\gppoint{gp mark 1}{(9.536,7.037)}
\gppoint{gp mark 1}{(9.539,7.076)}
\gppoint{gp mark 1}{(9.541,7.078)}
\gppoint{gp mark 1}{(9.543,7.058)}
\gppoint{gp mark 1}{(9.545,7.049)}
\gppoint{gp mark 1}{(9.548,7.079)}
\gppoint{gp mark 1}{(9.550,7.069)}
\gppoint{gp mark 1}{(9.552,7.084)}
\gppoint{gp mark 1}{(9.554,7.096)}
\gppoint{gp mark 1}{(9.557,7.086)}
\gppoint{gp mark 1}{(9.559,7.072)}
\gppoint{gp mark 1}{(9.561,7.086)}
\gppoint{gp mark 1}{(9.563,7.092)}
\gppoint{gp mark 1}{(9.566,7.087)}
\gppoint{gp mark 1}{(9.568,7.105)}
\gppoint{gp mark 1}{(9.570,7.081)}
\gppoint{gp mark 1}{(9.572,7.082)}
\gppoint{gp mark 1}{(9.575,7.055)}
\gppoint{gp mark 1}{(9.577,7.089)}
\gppoint{gp mark 1}{(9.579,7.081)}
\gppoint{gp mark 1}{(9.581,7.081)}
\gppoint{gp mark 1}{(9.584,7.067)}
\gppoint{gp mark 1}{(9.586,7.087)}
\gppoint{gp mark 1}{(9.588,7.118)}
\gppoint{gp mark 1}{(9.590,7.093)}
\gppoint{gp mark 1}{(9.593,7.094)}
\gppoint{gp mark 1}{(9.595,7.042)}
\gppoint{gp mark 1}{(9.597,7.088)}
\gppoint{gp mark 1}{(9.599,7.094)}
\gppoint{gp mark 1}{(9.602,7.109)}
\gppoint{gp mark 1}{(9.604,7.034)}
\gppoint{gp mark 1}{(9.606,7.072)}
\gppoint{gp mark 1}{(9.608,7.084)}
\gppoint{gp mark 1}{(9.611,7.070)}
\gppoint{gp mark 1}{(9.613,7.099)}
\gppoint{gp mark 1}{(9.615,7.101)}
\gppoint{gp mark 1}{(9.617,7.103)}
\gppoint{gp mark 1}{(9.620,7.105)}
\gppoint{gp mark 1}{(9.622,7.102)}
\gppoint{gp mark 1}{(9.624,7.078)}
\gppoint{gp mark 1}{(9.626,7.088)}
\gppoint{gp mark 1}{(9.629,7.078)}
\gppoint{gp mark 1}{(9.631,7.106)}
\gppoint{gp mark 1}{(9.633,7.062)}
\gppoint{gp mark 1}{(9.635,7.067)}
\gppoint{gp mark 1}{(9.638,7.068)}
\gppoint{gp mark 1}{(9.640,7.093)}
\gppoint{gp mark 1}{(9.642,7.080)}
\gppoint{gp mark 1}{(9.644,7.081)}
\gppoint{gp mark 1}{(9.647,7.098)}
\gppoint{gp mark 1}{(9.649,7.091)}
\gppoint{gp mark 1}{(9.651,7.095)}
\gppoint{gp mark 1}{(9.653,7.052)}
\gppoint{gp mark 1}{(9.656,7.093)}
\gppoint{gp mark 1}{(9.658,7.100)}
\gppoint{gp mark 1}{(9.660,7.072)}
\gppoint{gp mark 1}{(9.662,7.100)}
\gppoint{gp mark 1}{(9.665,7.073)}
\gppoint{gp mark 1}{(9.667,7.097)}
\gppoint{gp mark 1}{(9.669,7.119)}
\gppoint{gp mark 1}{(9.671,7.093)}
\gppoint{gp mark 1}{(9.674,7.078)}
\gppoint{gp mark 1}{(9.676,7.083)}
\gppoint{gp mark 1}{(9.678,7.066)}
\gppoint{gp mark 1}{(9.680,7.125)}
\gppoint{gp mark 1}{(9.683,7.092)}
\gppoint{gp mark 1}{(9.685,7.082)}
\gppoint{gp mark 1}{(9.687,7.065)}
\gppoint{gp mark 1}{(9.689,7.103)}
\gppoint{gp mark 1}{(9.692,7.096)}
\gppoint{gp mark 1}{(9.694,7.052)}
\gppoint{gp mark 1}{(9.696,7.113)}
\gppoint{gp mark 1}{(9.698,7.100)}
\gppoint{gp mark 1}{(9.701,7.112)}
\gppoint{gp mark 1}{(9.703,7.076)}
\gppoint{gp mark 1}{(9.705,7.099)}
\gppoint{gp mark 1}{(9.707,7.113)}
\gppoint{gp mark 1}{(9.710,7.114)}
\gppoint{gp mark 1}{(9.712,7.133)}
\gppoint{gp mark 1}{(9.714,7.106)}
\gppoint{gp mark 1}{(9.716,7.099)}
\gppoint{gp mark 1}{(9.719,7.078)}
\gppoint{gp mark 1}{(9.721,7.078)}
\gppoint{gp mark 1}{(9.723,7.090)}
\gppoint{gp mark 1}{(9.725,7.098)}
\gppoint{gp mark 1}{(9.728,7.078)}
\gppoint{gp mark 1}{(9.730,7.095)}
\gppoint{gp mark 1}{(9.732,7.078)}
\gppoint{gp mark 1}{(9.734,7.119)}
\gppoint{gp mark 1}{(9.737,7.120)}
\gppoint{gp mark 1}{(9.739,7.114)}
\gppoint{gp mark 1}{(9.741,7.116)}
\gppoint{gp mark 1}{(9.743,7.104)}
\gppoint{gp mark 1}{(9.746,7.108)}
\gppoint{gp mark 1}{(9.748,7.089)}
\gppoint{gp mark 1}{(9.750,7.124)}
\gppoint{gp mark 1}{(9.752,7.085)}
\gppoint{gp mark 1}{(9.755,7.110)}
\gppoint{gp mark 1}{(9.757,7.094)}
\gppoint{gp mark 1}{(9.759,7.108)}
\gppoint{gp mark 1}{(9.761,7.115)}
\gppoint{gp mark 1}{(9.764,7.106)}
\gppoint{gp mark 1}{(9.766,7.081)}
\gppoint{gp mark 1}{(9.768,7.124)}
\gppoint{gp mark 1}{(9.770,7.096)}
\gppoint{gp mark 1}{(9.773,7.103)}
\gppoint{gp mark 1}{(9.775,7.089)}
\gppoint{gp mark 1}{(9.777,7.092)}
\gppoint{gp mark 1}{(9.779,7.099)}
\gppoint{gp mark 1}{(9.782,7.117)}
\gppoint{gp mark 1}{(9.784,7.124)}
\gppoint{gp mark 1}{(9.786,7.094)}
\gppoint{gp mark 1}{(9.788,7.112)}
\gppoint{gp mark 1}{(9.791,7.105)}
\gppoint{gp mark 1}{(9.793,7.104)}
\gppoint{gp mark 1}{(9.795,7.141)}
\gppoint{gp mark 1}{(9.797,7.107)}
\gppoint{gp mark 1}{(9.800,7.132)}
\gppoint{gp mark 1}{(9.802,7.119)}
\gppoint{gp mark 1}{(9.804,7.120)}
\gppoint{gp mark 1}{(9.806,7.122)}
\gppoint{gp mark 1}{(9.809,7.094)}
\gppoint{gp mark 1}{(9.811,7.123)}
\gppoint{gp mark 1}{(9.813,7.146)}
\gppoint{gp mark 1}{(9.815,7.130)}
\gppoint{gp mark 1}{(9.818,7.119)}
\gppoint{gp mark 1}{(9.820,7.079)}
\gppoint{gp mark 1}{(9.822,7.124)}
\gppoint{gp mark 1}{(9.824,7.090)}
\gppoint{gp mark 1}{(9.827,7.111)}
\gppoint{gp mark 1}{(9.829,7.120)}
\gppoint{gp mark 1}{(9.831,7.107)}
\gppoint{gp mark 1}{(9.833,7.129)}
\gppoint{gp mark 1}{(9.836,7.105)}
\gppoint{gp mark 1}{(9.838,7.099)}
\gppoint{gp mark 1}{(9.840,7.120)}
\gppoint{gp mark 1}{(9.842,7.087)}
\gppoint{gp mark 1}{(9.845,7.109)}
\gppoint{gp mark 1}{(9.847,7.106)}
\gppoint{gp mark 1}{(9.849,7.105)}
\gppoint{gp mark 1}{(9.851,7.130)}
\gppoint{gp mark 1}{(9.854,7.111)}
\gppoint{gp mark 1}{(9.856,7.152)}
\gppoint{gp mark 1}{(9.858,7.112)}
\gppoint{gp mark 1}{(9.860,7.127)}
\gppoint{gp mark 1}{(9.863,7.098)}
\gppoint{gp mark 1}{(9.865,7.139)}
\gppoint{gp mark 1}{(9.867,7.105)}
\gppoint{gp mark 1}{(9.869,7.130)}
\gppoint{gp mark 1}{(9.872,7.123)}
\gppoint{gp mark 1}{(9.874,7.127)}
\gppoint{gp mark 1}{(9.876,7.113)}
\gppoint{gp mark 1}{(9.878,7.093)}
\gppoint{gp mark 1}{(9.881,7.133)}
\gppoint{gp mark 1}{(9.883,7.103)}
\gppoint{gp mark 1}{(9.885,7.106)}
\gppoint{gp mark 1}{(9.887,7.123)}
\gppoint{gp mark 1}{(9.890,7.152)}
\gppoint{gp mark 1}{(9.892,7.132)}
\gppoint{gp mark 1}{(9.894,7.138)}
\gppoint{gp mark 1}{(9.896,7.141)}
\gppoint{gp mark 1}{(9.899,7.147)}
\gppoint{gp mark 1}{(9.901,7.145)}
\gppoint{gp mark 1}{(9.903,7.133)}
\gppoint{gp mark 1}{(9.905,7.145)}
\gppoint{gp mark 1}{(9.908,7.123)}
\gppoint{gp mark 1}{(9.910,7.135)}
\gppoint{gp mark 1}{(9.912,7.114)}
\gppoint{gp mark 1}{(9.914,7.117)}
\gppoint{gp mark 1}{(9.917,7.122)}
\gppoint{gp mark 1}{(9.919,7.133)}
\gppoint{gp mark 1}{(9.921,7.093)}
\gppoint{gp mark 1}{(9.923,7.096)}
\gppoint{gp mark 1}{(9.926,7.122)}
\gppoint{gp mark 1}{(9.928,7.118)}
\gppoint{gp mark 1}{(9.930,7.138)}
\gppoint{gp mark 1}{(9.932,7.146)}
\gppoint{gp mark 1}{(9.935,7.125)}
\gppoint{gp mark 1}{(9.937,7.108)}
\gppoint{gp mark 1}{(9.939,7.125)}
\gppoint{gp mark 1}{(9.941,7.138)}
\gppoint{gp mark 1}{(9.944,7.145)}
\gppoint{gp mark 1}{(9.946,7.123)}
\gppoint{gp mark 1}{(9.948,7.137)}
\gppoint{gp mark 1}{(9.950,7.126)}
\gppoint{gp mark 1}{(9.953,7.138)}
\gppoint{gp mark 1}{(9.955,7.139)}
\gppoint{gp mark 1}{(9.957,7.141)}
\gppoint{gp mark 1}{(9.959,7.123)}
\gppoint{gp mark 1}{(9.962,7.103)}
\gppoint{gp mark 1}{(9.964,7.124)}
\gppoint{gp mark 1}{(9.966,7.129)}
\gppoint{gp mark 1}{(9.968,7.133)}
\gppoint{gp mark 1}{(9.971,7.134)}
\gppoint{gp mark 1}{(9.973,7.098)}
\gppoint{gp mark 1}{(9.975,7.128)}
\gppoint{gp mark 1}{(9.977,7.096)}
\gppoint{gp mark 1}{(9.980,7.138)}
\gppoint{gp mark 1}{(9.982,7.107)}
\gppoint{gp mark 1}{(9.984,7.108)}
\gppoint{gp mark 1}{(9.986,7.154)}
\gppoint{gp mark 1}{(9.989,7.113)}
\gppoint{gp mark 1}{(9.991,7.139)}
\gppoint{gp mark 1}{(9.993,7.136)}
\gppoint{gp mark 1}{(9.995,7.142)}
\gppoint{gp mark 1}{(9.998,7.136)}
\gppoint{gp mark 1}{(10.000,7.108)}
\gppoint{gp mark 1}{(10.002,7.117)}
\gppoint{gp mark 1}{(10.004,7.157)}
\gppoint{gp mark 1}{(10.007,7.128)}
\gppoint{gp mark 1}{(10.009,7.102)}
\gppoint{gp mark 1}{(10.011,7.129)}
\gppoint{gp mark 1}{(10.013,7.123)}
\gppoint{gp mark 1}{(10.016,7.139)}
\gppoint{gp mark 1}{(10.018,7.124)}
\gppoint{gp mark 1}{(10.020,7.126)}
\gppoint{gp mark 1}{(10.022,7.127)}
\gppoint{gp mark 1}{(10.025,7.129)}
\gppoint{gp mark 1}{(10.027,7.134)}
\gppoint{gp mark 1}{(10.029,7.143)}
\gppoint{gp mark 1}{(10.031,7.145)}
\gppoint{gp mark 1}{(10.034,7.164)}
\gppoint{gp mark 1}{(10.036,7.142)}
\gppoint{gp mark 1}{(10.038,7.097)}
\gppoint{gp mark 1}{(10.040,7.132)}
\gppoint{gp mark 1}{(10.043,7.139)}
\gppoint{gp mark 1}{(10.045,7.116)}
\gppoint{gp mark 1}{(10.047,7.118)}
\gppoint{gp mark 1}{(10.049,7.136)}
\gppoint{gp mark 1}{(10.052,7.141)}
\gppoint{gp mark 1}{(10.054,7.124)}
\gppoint{gp mark 1}{(10.056,7.146)}
\gppoint{gp mark 1}{(10.058,7.141)}
\gppoint{gp mark 1}{(10.061,7.152)}
\gppoint{gp mark 1}{(10.063,7.150)}
\gppoint{gp mark 1}{(10.065,7.116)}
\gppoint{gp mark 1}{(10.067,7.145)}
\gppoint{gp mark 1}{(10.070,7.115)}
\gppoint{gp mark 1}{(10.072,7.150)}
\gppoint{gp mark 1}{(10.074,7.113)}
\gppoint{gp mark 1}{(10.076,7.153)}
\gppoint{gp mark 1}{(10.079,7.125)}
\gppoint{gp mark 1}{(10.081,7.128)}
\gppoint{gp mark 1}{(10.083,7.133)}
\gppoint{gp mark 1}{(10.085,7.154)}
\gppoint{gp mark 1}{(10.088,7.114)}
\gppoint{gp mark 1}{(10.090,7.140)}
\gppoint{gp mark 1}{(10.092,7.154)}
\gppoint{gp mark 1}{(10.094,7.119)}
\gppoint{gp mark 1}{(10.097,7.130)}
\gppoint{gp mark 1}{(10.099,7.166)}
\gppoint{gp mark 1}{(10.101,7.137)}
\gppoint{gp mark 1}{(10.103,7.168)}
\gppoint{gp mark 1}{(10.106,7.165)}
\gppoint{gp mark 1}{(10.108,7.153)}
\gppoint{gp mark 1}{(10.110,7.179)}
\gppoint{gp mark 1}{(10.112,7.158)}
\gppoint{gp mark 1}{(10.115,7.190)}
\gppoint{gp mark 1}{(10.117,7.151)}
\gppoint{gp mark 1}{(10.119,7.181)}
\gppoint{gp mark 1}{(10.121,7.159)}
\gppoint{gp mark 1}{(10.124,7.144)}
\gppoint{gp mark 1}{(10.126,7.154)}
\gppoint{gp mark 1}{(10.128,7.155)}
\gppoint{gp mark 1}{(10.130,7.142)}
\gppoint{gp mark 1}{(10.133,7.163)}
\gppoint{gp mark 1}{(10.135,7.175)}
\gppoint{gp mark 1}{(10.137,7.171)}
\gppoint{gp mark 1}{(10.139,7.202)}
\gppoint{gp mark 1}{(10.142,7.162)}
\gppoint{gp mark 1}{(10.144,7.143)}
\gppoint{gp mark 1}{(10.146,7.144)}
\gppoint{gp mark 1}{(10.148,7.166)}
\gppoint{gp mark 1}{(10.151,7.142)}
\gppoint{gp mark 1}{(10.153,7.162)}
\gppoint{gp mark 1}{(10.155,7.146)}
\gppoint{gp mark 1}{(10.157,7.149)}
\gppoint{gp mark 1}{(10.160,7.165)}
\gppoint{gp mark 1}{(10.162,7.137)}
\gppoint{gp mark 1}{(10.164,7.171)}
\gppoint{gp mark 1}{(10.166,7.150)}
\gppoint{gp mark 1}{(10.169,7.173)}
\gppoint{gp mark 1}{(10.171,7.131)}
\gppoint{gp mark 1}{(10.173,7.162)}
\gppoint{gp mark 1}{(10.175,7.164)}
\gppoint{gp mark 1}{(10.178,7.162)}
\gppoint{gp mark 1}{(10.180,7.118)}
\gppoint{gp mark 1}{(10.182,7.180)}
\gppoint{gp mark 1}{(10.185,7.163)}
\gppoint{gp mark 1}{(10.187,7.161)}
\gppoint{gp mark 1}{(10.189,7.181)}
\gppoint{gp mark 1}{(10.191,7.155)}
\gppoint{gp mark 1}{(10.194,7.181)}
\gppoint{gp mark 1}{(10.196,7.194)}
\gppoint{gp mark 1}{(10.198,7.180)}
\gppoint{gp mark 1}{(10.200,7.163)}
\gppoint{gp mark 1}{(10.203,7.136)}
\gppoint{gp mark 1}{(10.205,7.172)}
\gppoint{gp mark 1}{(10.207,7.152)}
\gppoint{gp mark 1}{(10.209,7.145)}
\gppoint{gp mark 1}{(10.212,7.173)}
\gppoint{gp mark 1}{(10.214,7.182)}
\gppoint{gp mark 1}{(10.216,7.161)}
\gppoint{gp mark 1}{(10.218,7.138)}
\gppoint{gp mark 1}{(10.221,7.176)}
\gppoint{gp mark 1}{(10.223,7.171)}
\gppoint{gp mark 1}{(10.225,7.182)}
\gppoint{gp mark 1}{(10.227,7.172)}
\gppoint{gp mark 1}{(10.230,7.181)}
\gppoint{gp mark 1}{(10.232,7.166)}
\gppoint{gp mark 1}{(10.234,7.178)}
\gppoint{gp mark 1}{(10.236,7.178)}
\gppoint{gp mark 1}{(10.239,7.181)}
\gppoint{gp mark 1}{(10.241,7.145)}
\gppoint{gp mark 1}{(10.243,7.167)}
\gppoint{gp mark 1}{(10.245,7.182)}
\gppoint{gp mark 1}{(10.248,7.148)}
\gppoint{gp mark 1}{(10.250,7.158)}
\gppoint{gp mark 1}{(10.252,7.148)}
\gppoint{gp mark 1}{(10.254,7.172)}
\gppoint{gp mark 1}{(10.257,7.146)}
\gppoint{gp mark 1}{(10.259,7.191)}
\gppoint{gp mark 1}{(10.261,7.160)}
\gppoint{gp mark 1}{(10.263,7.146)}
\gppoint{gp mark 1}{(10.266,7.181)}
\gppoint{gp mark 1}{(10.268,7.157)}
\gppoint{gp mark 1}{(10.270,7.174)}
\gppoint{gp mark 1}{(10.272,7.138)}
\gppoint{gp mark 1}{(10.275,7.201)}
\gppoint{gp mark 1}{(10.277,7.168)}
\gppoint{gp mark 1}{(10.279,7.155)}
\gppoint{gp mark 1}{(10.281,7.172)}
\gppoint{gp mark 1}{(10.284,7.145)}
\gppoint{gp mark 1}{(10.286,7.147)}
\gppoint{gp mark 1}{(10.288,7.195)}
\gppoint{gp mark 1}{(10.290,7.157)}
\gppoint{gp mark 1}{(10.293,7.178)}
\gppoint{gp mark 1}{(10.295,7.151)}
\gppoint{gp mark 1}{(10.297,7.152)}
\gppoint{gp mark 1}{(10.299,7.146)}
\gppoint{gp mark 1}{(10.302,7.171)}
\gppoint{gp mark 1}{(10.304,7.174)}
\gppoint{gp mark 1}{(10.306,7.150)}
\gppoint{gp mark 1}{(10.308,7.183)}
\gppoint{gp mark 1}{(10.311,7.171)}
\gppoint{gp mark 1}{(10.313,7.170)}
\gppoint{gp mark 1}{(10.315,7.169)}
\gppoint{gp mark 1}{(10.317,7.210)}
\gppoint{gp mark 1}{(10.320,7.179)}
\gppoint{gp mark 1}{(10.322,7.150)}
\gppoint{gp mark 1}{(10.324,7.170)}
\gppoint{gp mark 1}{(10.326,7.186)}
\gppoint{gp mark 1}{(10.329,7.188)}
\gppoint{gp mark 1}{(10.331,5.893)}
\gppoint{gp mark 1}{(10.333,1.426)}
\gppoint{gp mark 1}{(10.335,1.421)}
\gppoint{gp mark 1}{(10.338,1.400)}
\gppoint{gp mark 1}{(10.340,1.398)}
\gppoint{gp mark 1}{(10.342,1.420)}
\gppoint{gp mark 1}{(10.344,1.408)}
\gppoint{gp mark 1}{(10.347,1.403)}
\gppoint{gp mark 1}{(10.349,1.412)}
\gppoint{gp mark 1}{(10.351,1.407)}
\gppoint{gp mark 1}{(10.353,1.413)}
\gppoint{gp mark 1}{(10.356,1.422)}
\gppoint{gp mark 1}{(10.358,1.400)}
\gppoint{gp mark 1}{(10.360,1.412)}
\gppoint{gp mark 1}{(10.362,1.410)}
\gppoint{gp mark 1}{(10.365,1.409)}
\gppoint{gp mark 1}{(10.367,1.415)}
\gppoint{gp mark 1}{(10.369,1.418)}
\gppoint{gp mark 1}{(10.371,1.413)}
\gppoint{gp mark 1}{(10.374,1.419)}
\gppoint{gp mark 1}{(10.376,1.416)}
\gppoint{gp mark 1}{(10.378,1.407)}
\gppoint{gp mark 1}{(10.380,1.413)}
\gppoint{gp mark 1}{(10.383,1.420)}
\gppoint{gp mark 1}{(10.385,1.416)}
\gppoint{gp mark 1}{(10.387,1.409)}
\gppoint{gp mark 1}{(10.389,1.415)}
\gppoint{gp mark 1}{(10.392,1.425)}
\gppoint{gp mark 1}{(10.394,1.409)}
\gppoint{gp mark 1}{(10.396,1.411)}
\gppoint{gp mark 1}{(10.398,1.417)}
\gppoint{gp mark 1}{(10.401,1.411)}
\gppoint{gp mark 1}{(10.403,1.428)}
\gppoint{gp mark 1}{(10.405,1.417)}
\gppoint{gp mark 1}{(10.407,1.417)}
\gppoint{gp mark 1}{(10.410,1.417)}
\gppoint{gp mark 1}{(10.412,1.407)}
\gppoint{gp mark 1}{(10.414,1.424)}
\gppoint{gp mark 1}{(10.416,1.410)}
\gppoint{gp mark 1}{(10.419,1.421)}
\gppoint{gp mark 1}{(10.421,1.419)}
\gppoint{gp mark 1}{(10.423,1.420)}
\gppoint{gp mark 1}{(10.425,1.422)}
\gppoint{gp mark 1}{(10.428,1.406)}
\gppoint{gp mark 1}{(10.430,1.422)}
\gppoint{gp mark 1}{(10.432,1.415)}
\gppoint{gp mark 1}{(10.434,1.425)}
\gppoint{gp mark 1}{(10.437,1.407)}
\gppoint{gp mark 1}{(10.439,1.424)}
\gppoint{gp mark 1}{(10.441,1.403)}
\gppoint{gp mark 1}{(10.443,1.429)}
\gppoint{gp mark 1}{(10.446,1.410)}
\gppoint{gp mark 1}{(10.448,1.430)}
\gppoint{gp mark 1}{(10.450,1.418)}
\gppoint{gp mark 1}{(10.452,1.436)}
\gppoint{gp mark 1}{(10.455,1.420)}
\gppoint{gp mark 1}{(10.457,1.400)}
\gppoint{gp mark 1}{(10.459,1.424)}
\gppoint{gp mark 1}{(10.461,1.411)}
\gppoint{gp mark 1}{(10.464,1.408)}
\gppoint{gp mark 1}{(10.466,1.426)}
\gppoint{gp mark 1}{(10.468,1.426)}
\gppoint{gp mark 1}{(10.470,1.427)}
\gppoint{gp mark 1}{(10.473,1.414)}
\gppoint{gp mark 1}{(10.475,1.417)}
\gppoint{gp mark 1}{(10.477,1.424)}
\gppoint{gp mark 1}{(10.479,1.423)}
\gppoint{gp mark 1}{(10.482,1.432)}
\gppoint{gp mark 1}{(10.484,1.416)}
\gppoint{gp mark 1}{(10.486,1.415)}
\gppoint{gp mark 1}{(10.488,1.417)}
\gppoint{gp mark 1}{(10.491,1.409)}
\gppoint{gp mark 1}{(10.493,1.393)}
\gppoint{gp mark 1}{(10.495,1.407)}
\gppoint{gp mark 1}{(10.497,1.414)}
\gppoint{gp mark 1}{(10.500,1.419)}
\gppoint{gp mark 1}{(10.502,1.420)}
\gppoint{gp mark 1}{(10.504,1.410)}
\gppoint{gp mark 1}{(10.506,1.409)}
\gppoint{gp mark 1}{(10.509,1.419)}
\gppoint{gp mark 1}{(10.511,1.436)}
\gppoint{gp mark 1}{(10.513,1.415)}
\gppoint{gp mark 1}{(10.515,1.419)}
\gppoint{gp mark 1}{(10.518,1.421)}
\gppoint{gp mark 1}{(10.520,1.409)}
\gppoint{gp mark 1}{(10.522,1.411)}
\gppoint{gp mark 1}{(10.524,1.424)}
\gppoint{gp mark 1}{(10.527,1.422)}
\gppoint{gp mark 1}{(10.529,1.409)}
\gppoint{gp mark 1}{(10.531,1.427)}
\gppoint{gp mark 1}{(10.533,1.435)}
\gppoint{gp mark 1}{(10.536,1.446)}
\gppoint{gp mark 1}{(10.538,1.408)}
\gppoint{gp mark 1}{(10.540,1.421)}
\gppoint{gp mark 1}{(10.542,1.420)}
\gppoint{gp mark 1}{(10.545,1.427)}
\gppoint{gp mark 1}{(10.547,1.413)}
\gppoint{gp mark 1}{(10.549,1.420)}
\gppoint{gp mark 1}{(10.551,1.427)}
\gppoint{gp mark 1}{(10.554,1.410)}
\gppoint{gp mark 1}{(10.556,1.433)}
\gppoint{gp mark 1}{(10.558,1.415)}
\gppoint{gp mark 1}{(10.560,1.440)}
\gppoint{gp mark 1}{(10.563,1.426)}
\gppoint{gp mark 1}{(10.565,1.413)}
\gppoint{gp mark 1}{(10.567,1.437)}
\gppoint{gp mark 1}{(10.569,1.405)}
\gppoint{gp mark 1}{(10.572,1.411)}
\gppoint{gp mark 1}{(10.574,1.427)}
\gppoint{gp mark 1}{(10.576,1.410)}
\gppoint{gp mark 1}{(10.578,1.429)}
\gppoint{gp mark 1}{(10.581,1.434)}
\gppoint{gp mark 1}{(10.583,1.440)}
\gppoint{gp mark 1}{(10.585,1.421)}
\gppoint{gp mark 1}{(10.587,1.430)}
\gppoint{gp mark 1}{(10.590,1.442)}
\gppoint{gp mark 1}{(10.592,1.415)}
\gppoint{gp mark 1}{(10.594,1.425)}
\gppoint{gp mark 1}{(10.596,1.425)}
\gppoint{gp mark 1}{(10.599,1.428)}
\gppoint{gp mark 1}{(10.601,1.430)}
\gppoint{gp mark 1}{(10.603,1.422)}
\gppoint{gp mark 1}{(10.605,1.417)}
\gppoint{gp mark 1}{(10.608,1.417)}
\gppoint{gp mark 1}{(10.610,1.414)}
\gppoint{gp mark 1}{(10.612,1.426)}
\gppoint{gp mark 1}{(10.614,1.414)}
\gppoint{gp mark 1}{(10.617,1.419)}
\gppoint{gp mark 1}{(10.619,1.417)}
\gppoint{gp mark 1}{(10.621,1.415)}
\gppoint{gp mark 1}{(10.623,1.422)}
\gppoint{gp mark 1}{(10.626,1.428)}
\gppoint{gp mark 1}{(10.628,1.434)}
\gppoint{gp mark 1}{(10.630,1.446)}
\gppoint{gp mark 1}{(10.632,1.412)}
\gppoint{gp mark 1}{(10.635,1.440)}
\gppoint{gp mark 1}{(10.637,1.417)}
\gppoint{gp mark 1}{(10.639,1.429)}
\gppoint{gp mark 1}{(10.641,1.450)}
\gppoint{gp mark 1}{(10.644,1.424)}
\gppoint{gp mark 1}{(10.646,1.436)}
\gppoint{gp mark 1}{(10.648,1.425)}
\gppoint{gp mark 1}{(10.650,1.436)}
\draw[gp path] (1.320,7.691)--(1.320,0.985)--(11.447,0.985)--(11.447,7.691)--cycle;
%% coordinates of the plot area
\gpdefrectangularnode{gp plot 1}{\pgfpoint{1.320cm}{0.985cm}}{\pgfpoint{11.447cm}{7.691cm}}
\end{tikzpicture}
%% gnuplot variables

\caption{太陽電波の測定}
\label{fig:solar_radio_raw}
\end{center}
\end{figure}
\clearpage
\subsubsection{主ビーム幅・結合定数の算出}
図\ref{fig:solar_radio_fit}に太陽電波の測定値を(\ref{equ:v(t)})式の$v(t)$で最小自乗fitした曲線を示す.
最小自乗fitはgnuplotのfit機能を用いて行った.
ただしグラフの見易さのため,測定値のプロットは$1/10$に間引いている.
最小自乗fitの結果各定数は以下のように求まった.
\begin{align}
  A&=1.0858\pm0.0008\ \si{\volt}\\
  B&=0.8858\pm0.0014\ \si{\volt}\\
  t_0&=2161.6\pm0.6\ \si{\second}\\
  \sigma_t&=372.47\pm0.82\ \si{\second}
\end{align}
また太陽の日周運動の角速度$\mu$は測定日の太陽の赤緯が$\delta=12.025\si{\degree}$だったことから(\ref{equ:mu})式より
\begin{align}
  \mu=\frac{360\cos(12.025\si{\degree})}{24\times60\times60}=4.08\times10^{-3}\si{\degree.\second^{-1}}
\end{align}
となる,したがって$\sigma$は
\begin{align}
  \sigma=1.517\pm0.003\si{\degree}
\end{align}
である.したがってアンテナの主ビーム幅は(\ref{equ:HPBW})から
\begin{align}
  {\rm HPBW}=2\sqrt{2\ln2}\sigma=3.574\pm0.8\si{\degree}
\end{align}
である.またビーム占有率$\eta_{ff}$は測定日の太陽の視半径が$\theta_{\tiny\astrosun}=0.26283\si{\degree}$だったことから(\ref{equ:eta_ff})式より
\begin{align}
  \eta_{ff}&=1-\exp\left(-\frac{\theta_{\tiny\astrosun}^2}{2\sigma^2}\right)\nonumber\\
  &=1.488\times10^{-2}
\end{align}
また不確かさについては
\begin{align}
  \frac{\partial\eta_{ff}}{\partial\sigma}=-\frac{\theta_{\tiny\astrosun}^2}{\sigma^3}\exp\left(-\frac{\theta_{\tiny\astrosun}^2}{2\sigma^2}\right)
\end{align}
なので
\begin{align}
  \delta_{\eta_{ff}}&=\left|\frac{\theta_{\tiny\astrosun}^2}{\overline{\sigma}^3}\exp\left(-\frac{\theta_{\tiny\astrosun}^2}{2\overline{\sigma}^2}\right)\right|\delta_{\sigma}\\
  &=7\times10^{-5}
\end{align}
となり,以上から$\eta_{ff}$は
\begin{align}
  \eta_{ff}=(1.488\pm0.007)\times10^{-2}
\end{align}
となる.したがって主ビーム能率$\eta_{MB}=0.6\pm0.1$とすると結合定数$\eta_c$は
\begin{align}
  \overline{\eta_c}&=\overline{\eta_{MB}}\cdot\overline{\eta_{ff}}\\
  \delta_{\eta_c}&=\overline{\eta_c}\sqrt{\left(\frac{\delta_{\eta_{MB}}}{\eta_{MB}}\right)^2+\left(\frac{\delta_{\eta_{ff}}}{\eta_{ff}}\right)^2}
\end{align}
なので
\begin{align}
  \eta_c=(8.9\pm 1.4)\times 10^{-3}
\end{align}
となる.
\begin{figure}[hptb]
\begin{center}
\begin{tikzpicture}[gnuplot]
%% generated with GNUPLOT 5.2p8 (Lua 5.3; terminal rev. Nov 2018, script rev. 108)
%% 2021年05月14日 08時50分31秒
\path (0.000,0.000) rectangle (12.000,8.000);
\gpcolor{color=gp lt color border}
\gpsetlinetype{gp lt border}
\gpsetdashtype{gp dt solid}
\gpsetlinewidth{1.00}
\draw[gp path] (1.320,0.985)--(1.500,0.985);
\draw[gp path] (11.447,0.985)--(11.267,0.985);
\node[gp node right] at (1.136,0.985) {$1$};
\draw[gp path] (1.320,1.432)--(1.410,1.432);
\draw[gp path] (11.447,1.432)--(11.357,1.432);
\draw[gp path] (1.320,1.879)--(1.500,1.879);
\draw[gp path] (11.447,1.879)--(11.267,1.879);
\node[gp node right] at (1.136,1.879) {$1.2$};
\draw[gp path] (1.320,2.326)--(1.410,2.326);
\draw[gp path] (11.447,2.326)--(11.357,2.326);
\draw[gp path] (1.320,2.773)--(1.500,2.773);
\draw[gp path] (11.447,2.773)--(11.267,2.773);
\node[gp node right] at (1.136,2.773) {$1.4$};
\draw[gp path] (1.320,3.220)--(1.410,3.220);
\draw[gp path] (11.447,3.220)--(11.357,3.220);
\draw[gp path] (1.320,3.667)--(1.500,3.667);
\draw[gp path] (11.447,3.667)--(11.267,3.667);
\node[gp node right] at (1.136,3.667) {$1.6$};
\draw[gp path] (1.320,4.114)--(1.410,4.114);
\draw[gp path] (11.447,4.114)--(11.357,4.114);
\draw[gp path] (1.320,4.562)--(1.500,4.562);
\draw[gp path] (11.447,4.562)--(11.267,4.562);
\node[gp node right] at (1.136,4.562) {$1.8$};
\draw[gp path] (1.320,5.009)--(1.410,5.009);
\draw[gp path] (11.447,5.009)--(11.357,5.009);
\draw[gp path] (1.320,5.456)--(1.500,5.456);
\draw[gp path] (11.447,5.456)--(11.267,5.456);
\node[gp node right] at (1.136,5.456) {$2$};
\draw[gp path] (1.320,5.903)--(1.410,5.903);
\draw[gp path] (11.447,5.903)--(11.357,5.903);
\draw[gp path] (1.320,6.350)--(1.500,6.350);
\draw[gp path] (11.447,6.350)--(11.267,6.350);
\node[gp node right] at (1.136,6.350) {$2.2$};
\draw[gp path] (1.320,6.797)--(1.410,6.797);
\draw[gp path] (11.447,6.797)--(11.357,6.797);
\draw[gp path] (1.320,7.244)--(1.500,7.244);
\draw[gp path] (11.447,7.244)--(11.267,7.244);
\node[gp node right] at (1.136,7.244) {$2.4$};
\draw[gp path] (1.320,7.691)--(1.410,7.691);
\draw[gp path] (11.447,7.691)--(11.357,7.691);
\draw[gp path] (2.710,0.985)--(2.710,1.165);
\draw[gp path] (2.710,7.691)--(2.710,7.511);
\node[gp node center] at (2.710,0.677) {$1000$};
\draw[gp path] (4.376,0.985)--(4.376,1.075);
\draw[gp path] (4.376,7.691)--(4.376,7.601);
\draw[gp path] (6.042,0.985)--(6.042,1.165);
\draw[gp path] (6.042,7.691)--(6.042,7.511);
\node[gp node center] at (6.042,0.677) {$2000$};
\draw[gp path] (7.708,0.985)--(7.708,1.075);
\draw[gp path] (7.708,7.691)--(7.708,7.601);
\draw[gp path] (9.374,0.985)--(9.374,1.165);
\draw[gp path] (9.374,7.691)--(9.374,7.511);
\node[gp node center] at (9.374,0.677) {$3000$};
\draw[gp path] (11.040,0.985)--(11.040,1.075);
\draw[gp path] (11.040,7.691)--(11.040,7.601);
\draw[gp path] (1.320,7.691)--(1.320,0.985)--(11.447,0.985)--(11.447,7.691)--cycle;
\node[gp node center,rotate=-270] at (0.292,4.338) {電圧 / $\si{\volt}$};
\node[gp node center] at (6.383,0.215) {経過時間$t$ / $\si{\second}$};
\node[gp node right] at (9.979,7.357) {測定値};
\gpsetpointsize{4.00}
\gppoint{gp mark 1}{(1.347,1.463)}
\gppoint{gp mark 1}{(1.380,1.456)}
\gppoint{gp mark 1}{(1.413,1.451)}
\gppoint{gp mark 1}{(1.447,1.439)}
\gppoint{gp mark 1}{(1.480,1.459)}
\gppoint{gp mark 1}{(1.513,1.453)}
\gppoint{gp mark 1}{(1.547,1.458)}
\gppoint{gp mark 1}{(1.580,1.462)}
\gppoint{gp mark 1}{(1.613,1.460)}
\gppoint{gp mark 1}{(1.647,1.455)}
\gppoint{gp mark 1}{(1.680,1.446)}
\gppoint{gp mark 1}{(1.713,1.446)}
\gppoint{gp mark 1}{(1.747,1.463)}
\gppoint{gp mark 1}{(1.780,1.461)}
\gppoint{gp mark 1}{(1.813,1.454)}
\gppoint{gp mark 1}{(1.847,1.461)}
\gppoint{gp mark 1}{(1.880,1.451)}
\gppoint{gp mark 1}{(1.913,1.462)}
\gppoint{gp mark 1}{(1.946,1.458)}
\gppoint{gp mark 1}{(1.980,1.445)}
\gppoint{gp mark 1}{(2.013,1.464)}
\gppoint{gp mark 1}{(2.046,1.469)}
\gppoint{gp mark 1}{(2.080,1.454)}
\gppoint{gp mark 1}{(2.113,1.446)}
\gppoint{gp mark 1}{(2.146,1.464)}
\gppoint{gp mark 1}{(2.180,1.453)}
\gppoint{gp mark 1}{(2.213,1.424)}
\gppoint{gp mark 1}{(2.246,1.444)}
\gppoint{gp mark 1}{(2.280,1.435)}
\gppoint{gp mark 1}{(2.313,1.438)}
\gppoint{gp mark 1}{(2.346,1.451)}
\gppoint{gp mark 1}{(2.380,1.431)}
\gppoint{gp mark 1}{(2.413,1.448)}
\gppoint{gp mark 1}{(2.446,1.434)}
\gppoint{gp mark 1}{(2.480,1.435)}
\gppoint{gp mark 1}{(2.513,1.429)}
\gppoint{gp mark 1}{(2.546,1.402)}
\gppoint{gp mark 1}{(2.580,1.420)}
\gppoint{gp mark 1}{(2.613,1.415)}
\gppoint{gp mark 1}{(2.646,1.422)}
\gppoint{gp mark 1}{(2.680,1.395)}
\gppoint{gp mark 1}{(2.713,1.410)}
\gppoint{gp mark 1}{(2.746,1.413)}
\gppoint{gp mark 1}{(2.780,1.417)}
\gppoint{gp mark 1}{(2.813,1.427)}
\gppoint{gp mark 1}{(2.846,1.410)}
\gppoint{gp mark 1}{(2.880,1.430)}
\gppoint{gp mark 1}{(2.913,1.413)}
\gppoint{gp mark 1}{(2.946,1.423)}
\gppoint{gp mark 1}{(2.980,1.405)}
\gppoint{gp mark 1}{(3.013,1.405)}
\gppoint{gp mark 1}{(3.046,1.388)}
\gppoint{gp mark 1}{(3.079,1.409)}
\gppoint{gp mark 1}{(3.113,1.388)}
\gppoint{gp mark 1}{(3.146,1.398)}
\gppoint{gp mark 1}{(3.179,1.397)}
\gppoint{gp mark 1}{(3.213,1.418)}
\gppoint{gp mark 1}{(3.246,1.400)}
\gppoint{gp mark 1}{(3.279,1.419)}
\gppoint{gp mark 1}{(3.313,1.424)}
\gppoint{gp mark 1}{(3.346,1.425)}
\gppoint{gp mark 1}{(3.379,1.456)}
\gppoint{gp mark 1}{(3.413,1.426)}
\gppoint{gp mark 1}{(3.446,1.447)}
\gppoint{gp mark 1}{(3.479,1.447)}
\gppoint{gp mark 1}{(3.513,1.464)}
\gppoint{gp mark 1}{(3.546,1.484)}
\gppoint{gp mark 1}{(3.579,1.469)}
\gppoint{gp mark 1}{(3.613,1.493)}
\gppoint{gp mark 1}{(3.646,1.519)}
\gppoint{gp mark 1}{(3.679,1.521)}
\gppoint{gp mark 1}{(3.713,1.543)}
\gppoint{gp mark 1}{(3.746,1.566)}
\gppoint{gp mark 1}{(3.779,1.587)}
\gppoint{gp mark 1}{(3.813,1.600)}
\gppoint{gp mark 1}{(3.846,1.615)}
\gppoint{gp mark 1}{(3.879,1.652)}
\gppoint{gp mark 1}{(3.913,1.655)}
\gppoint{gp mark 1}{(3.946,1.687)}
\gppoint{gp mark 1}{(3.979,1.719)}
\gppoint{gp mark 1}{(4.013,1.754)}
\gppoint{gp mark 1}{(4.046,1.767)}
\gppoint{gp mark 1}{(4.079,1.824)}
\gppoint{gp mark 1}{(4.113,1.862)}
\gppoint{gp mark 1}{(4.146,1.886)}
\gppoint{gp mark 1}{(4.179,1.920)}
\gppoint{gp mark 1}{(4.212,1.944)}
\gppoint{gp mark 1}{(4.246,1.971)}
\gppoint{gp mark 1}{(4.279,2.035)}
\gppoint{gp mark 1}{(4.312,2.060)}
\gppoint{gp mark 1}{(4.346,2.105)}
\gppoint{gp mark 1}{(4.379,2.170)}
\gppoint{gp mark 1}{(4.412,2.213)}
\gppoint{gp mark 1}{(4.446,2.248)}
\gppoint{gp mark 1}{(4.479,2.290)}
\gppoint{gp mark 1}{(4.512,2.364)}
\gppoint{gp mark 1}{(4.546,2.420)}
\gppoint{gp mark 1}{(4.579,2.487)}
\gppoint{gp mark 1}{(4.612,2.506)}
\gppoint{gp mark 1}{(4.646,2.581)}
\gppoint{gp mark 1}{(4.679,2.605)}
\gppoint{gp mark 1}{(4.712,2.667)}
\gppoint{gp mark 1}{(4.746,2.712)}
\gppoint{gp mark 1}{(4.779,2.792)}
\gppoint{gp mark 1}{(4.812,2.824)}
\gppoint{gp mark 1}{(4.846,2.881)}
\gppoint{gp mark 1}{(4.879,2.969)}
\gppoint{gp mark 1}{(4.912,3.030)}
\gppoint{gp mark 1}{(4.946,3.073)}
\gppoint{gp mark 1}{(4.979,3.142)}
\gppoint{gp mark 1}{(5.012,3.189)}
\gppoint{gp mark 1}{(5.046,3.294)}
\gppoint{gp mark 1}{(5.079,3.349)}
\gppoint{gp mark 1}{(5.112,3.413)}
\gppoint{gp mark 1}{(5.146,3.494)}
\gppoint{gp mark 1}{(5.179,3.520)}
\gppoint{gp mark 1}{(5.212,3.614)}
\gppoint{gp mark 1}{(5.246,3.672)}
\gppoint{gp mark 1}{(5.279,3.721)}
\gppoint{gp mark 1}{(5.312,3.801)}
\gppoint{gp mark 1}{(5.345,3.859)}
\gppoint{gp mark 1}{(5.379,3.892)}
\gppoint{gp mark 1}{(5.412,3.974)}
\gppoint{gp mark 1}{(5.445,4.050)}
\gppoint{gp mark 1}{(5.479,4.109)}
\gppoint{gp mark 1}{(5.512,4.183)}
\gppoint{gp mark 1}{(5.545,4.219)}
\gppoint{gp mark 1}{(5.579,4.303)}
\gppoint{gp mark 1}{(5.612,4.333)}
\gppoint{gp mark 1}{(5.645,4.415)}
\gppoint{gp mark 1}{(5.679,4.428)}
\gppoint{gp mark 1}{(5.712,4.513)}
\gppoint{gp mark 1}{(5.745,4.561)}
\gppoint{gp mark 1}{(5.779,4.569)}
\gppoint{gp mark 1}{(5.812,4.680)}
\gppoint{gp mark 1}{(5.845,4.743)}
\gppoint{gp mark 1}{(5.879,4.795)}
\gppoint{gp mark 1}{(5.912,4.794)}
\gppoint{gp mark 1}{(5.945,4.831)}
\gppoint{gp mark 1}{(5.979,4.875)}
\gppoint{gp mark 1}{(6.012,4.934)}
\gppoint{gp mark 1}{(6.045,4.942)}
\gppoint{gp mark 1}{(6.079,4.978)}
\gppoint{gp mark 1}{(6.112,5.028)}
\gppoint{gp mark 1}{(6.145,5.063)}
\gppoint{gp mark 1}{(6.179,5.092)}
\gppoint{gp mark 1}{(6.212,5.131)}
\gppoint{gp mark 1}{(6.245,5.106)}
\gppoint{gp mark 1}{(6.279,5.211)}
\gppoint{gp mark 1}{(6.312,5.160)}
\gppoint{gp mark 1}{(6.345,5.211)}
\gppoint{gp mark 1}{(6.379,5.184)}
\gppoint{gp mark 1}{(6.412,5.209)}
\gppoint{gp mark 1}{(6.445,5.216)}
\gppoint{gp mark 1}{(6.478,5.224)}
\gppoint{gp mark 1}{(6.512,5.250)}
\gppoint{gp mark 1}{(6.545,5.239)}
\gppoint{gp mark 1}{(6.578,5.234)}
\gppoint{gp mark 1}{(6.612,5.271)}
\gppoint{gp mark 1}{(6.645,5.232)}
\gppoint{gp mark 1}{(6.678,5.258)}
\gppoint{gp mark 1}{(6.712,5.223)}
\gppoint{gp mark 1}{(6.745,5.224)}
\gppoint{gp mark 1}{(6.778,5.206)}
\gppoint{gp mark 1}{(6.812,5.213)}
\gppoint{gp mark 1}{(6.845,5.180)}
\gppoint{gp mark 1}{(6.878,5.161)}
\gppoint{gp mark 1}{(6.912,5.154)}
\gppoint{gp mark 1}{(6.945,5.129)}
\gppoint{gp mark 1}{(6.978,5.104)}
\gppoint{gp mark 1}{(7.012,5.087)}
\gppoint{gp mark 1}{(7.045,5.009)}
\gppoint{gp mark 1}{(7.078,5.003)}
\gppoint{gp mark 1}{(7.112,4.971)}
\gppoint{gp mark 1}{(7.145,4.911)}
\gppoint{gp mark 1}{(7.178,4.910)}
\gppoint{gp mark 1}{(7.212,4.861)}
\gppoint{gp mark 1}{(7.245,4.841)}
\gppoint{gp mark 1}{(7.278,4.769)}
\gppoint{gp mark 1}{(7.312,4.718)}
\gppoint{gp mark 1}{(7.345,4.677)}
\gppoint{gp mark 1}{(7.378,4.617)}
\gppoint{gp mark 1}{(7.412,4.572)}
\gppoint{gp mark 1}{(7.445,4.537)}
\gppoint{gp mark 1}{(7.478,4.449)}
\gppoint{gp mark 1}{(7.511,4.422)}
\gppoint{gp mark 1}{(7.545,4.373)}
\gppoint{gp mark 1}{(7.578,4.299)}
\gppoint{gp mark 1}{(7.611,4.237)}
\gppoint{gp mark 1}{(7.645,4.173)}
\gppoint{gp mark 1}{(7.678,4.128)}
\gppoint{gp mark 1}{(7.711,4.067)}
\gppoint{gp mark 1}{(7.745,4.011)}
\gppoint{gp mark 1}{(7.778,3.966)}
\gppoint{gp mark 1}{(7.811,3.888)}
\gppoint{gp mark 1}{(7.845,3.820)}
\gppoint{gp mark 1}{(7.878,3.738)}
\gppoint{gp mark 1}{(7.911,3.677)}
\gppoint{gp mark 1}{(7.945,3.612)}
\gppoint{gp mark 1}{(7.978,3.546)}
\gppoint{gp mark 1}{(8.011,3.445)}
\gppoint{gp mark 1}{(8.045,3.410)}
\gppoint{gp mark 1}{(8.078,3.346)}
\gppoint{gp mark 1}{(8.111,3.280)}
\gppoint{gp mark 1}{(8.145,3.206)}
\gppoint{gp mark 1}{(8.178,3.100)}
\gppoint{gp mark 1}{(8.211,3.058)}
\gppoint{gp mark 1}{(8.245,3.027)}
\gppoint{gp mark 1}{(8.278,2.964)}
\gppoint{gp mark 1}{(8.311,2.877)}
\gppoint{gp mark 1}{(8.345,2.811)}
\gppoint{gp mark 1}{(8.378,2.772)}
\gppoint{gp mark 1}{(8.411,2.707)}
\gppoint{gp mark 1}{(8.445,2.669)}
\gppoint{gp mark 1}{(8.478,2.604)}
\gppoint{gp mark 1}{(8.511,2.546)}
\gppoint{gp mark 1}{(8.545,2.517)}
\gppoint{gp mark 1}{(8.578,2.436)}
\gppoint{gp mark 1}{(8.611,2.408)}
\gppoint{gp mark 1}{(8.644,2.332)}
\gppoint{gp mark 1}{(8.678,2.296)}
\gppoint{gp mark 1}{(8.711,2.258)}
\gppoint{gp mark 1}{(8.744,2.204)}
\gppoint{gp mark 1}{(8.778,2.155)}
\gppoint{gp mark 1}{(8.811,2.100)}
\gppoint{gp mark 1}{(8.844,2.062)}
\gppoint{gp mark 1}{(8.878,2.030)}
\gppoint{gp mark 1}{(8.911,1.963)}
\gppoint{gp mark 1}{(8.944,1.960)}
\gppoint{gp mark 1}{(8.978,1.912)}
\gppoint{gp mark 1}{(9.011,1.856)}
\gppoint{gp mark 1}{(9.044,1.821)}
\gppoint{gp mark 1}{(9.078,1.807)}
\gppoint{gp mark 1}{(9.111,1.779)}
\gppoint{gp mark 1}{(9.144,1.756)}
\gppoint{gp mark 1}{(9.178,1.735)}
\gppoint{gp mark 1}{(9.211,1.708)}
\gppoint{gp mark 1}{(9.244,1.687)}
\gppoint{gp mark 1}{(9.278,1.672)}
\gppoint{gp mark 1}{(9.311,1.657)}
\gppoint{gp mark 1}{(9.344,1.614)}
\gppoint{gp mark 1}{(9.378,1.628)}
\gppoint{gp mark 1}{(9.411,1.593)}
\gppoint{gp mark 1}{(9.444,1.563)}
\gppoint{gp mark 1}{(9.478,1.567)}
\gppoint{gp mark 1}{(9.511,1.552)}
\gppoint{gp mark 1}{(9.544,1.502)}
\gppoint{gp mark 1}{(9.578,1.499)}
\gppoint{gp mark 1}{(9.611,1.500)}
\gppoint{gp mark 1}{(9.644,1.482)}
\gppoint{gp mark 1}{(9.678,1.450)}
\gppoint{gp mark 1}{(9.711,1.459)}
\gppoint{gp mark 1}{(9.744,1.450)}
\gppoint{gp mark 1}{(9.777,1.427)}
\gppoint{gp mark 1}{(9.811,1.432)}
\gppoint{gp mark 1}{(9.844,1.426)}
\gppoint{gp mark 1}{(9.877,1.419)}
\gppoint{gp mark 1}{(9.911,1.416)}
\gppoint{gp mark 1}{(9.944,1.414)}
\gppoint{gp mark 1}{(9.977,1.401)}
\gppoint{gp mark 1}{(10.011,1.398)}
\gppoint{gp mark 1}{(10.044,1.417)}
\gppoint{gp mark 1}{(10.077,1.402)}
\gppoint{gp mark 1}{(10.111,1.417)}
\gppoint{gp mark 1}{(10.144,1.411)}
\gppoint{gp mark 1}{(10.177,1.393)}
\gppoint{gp mark 1}{(10.211,1.401)}
\gppoint{gp mark 1}{(10.244,1.415)}
\gppoint{gp mark 1}{(10.277,1.411)}
\gppoint{gp mark 1}{(10.311,1.399)}
\gppoint{gp mark 1}{(10.344,1.409)}
\gppoint{gp mark 1}{(10.377,1.410)}
\gppoint{gp mark 1}{(10.411,1.433)}
\gppoint{gp mark 1}{(10.444,1.405)}
\gppoint{gp mark 1}{(10.477,1.399)}
\gppoint{gp mark 1}{(10.511,1.427)}
\gppoint{gp mark 1}{(10.544,1.431)}
\gppoint{gp mark 1}{(10.577,1.398)}
\gppoint{gp mark 1}{(10.611,1.416)}
\gppoint{gp mark 1}{(10.644,1.422)}
\gppoint{gp mark 1}{(10.677,1.413)}
\gppoint{gp mark 1}{(10.711,1.423)}
\gppoint{gp mark 1}{(10.744,1.412)}
\gppoint{gp mark 1}{(10.777,1.427)}
\gppoint{gp mark 1}{(10.811,1.390)}
\gppoint{gp mark 1}{(10.844,1.367)}
\gppoint{gp mark 1}{(10.877,1.346)}
\gppoint{gp mark 1}{(10.910,1.370)}
\gppoint{gp mark 1}{(10.944,1.361)}
\gppoint{gp mark 1}{(10.977,1.354)}
\gppoint{gp mark 1}{(11.010,1.354)}
\gppoint{gp mark 1}{(11.044,1.365)}
\gppoint{gp mark 1}{(11.077,1.362)}
\gppoint{gp mark 1}{(11.110,1.355)}
\gppoint{gp mark 1}{(11.144,1.370)}
\gppoint{gp mark 1}{(11.177,1.385)}
\gppoint{gp mark 1}{(11.210,1.366)}
\gppoint{gp mark 1}{(11.244,1.395)}
\gppoint{gp mark 1}{(11.277,1.389)}
\gppoint{gp mark 1}{(11.310,1.376)}
\gppoint{gp mark 1}{(11.344,1.384)}
\gppoint{gp mark 1}{(11.377,1.381)}
\gppoint{gp mark 1}{(11.410,1.376)}
\gppoint{gp mark 1}{(11.444,3.457)}
\gppoint{gp mark 1}{(10.621,7.357)}
\node[gp node right] at (9.979,7.049) {$v(t)$};
\gpcolor{rgb color={0.898,0.118,0.063}}
\gpsetlinewidth{2.00}
\draw[gp path] (10.163,7.049)--(11.079,7.049);
\draw[gp path] (1.320,1.369)--(1.422,1.369)--(1.525,1.370)--(1.627,1.370)--(1.729,1.371)%
  --(1.831,1.371)--(1.934,1.372)--(2.036,1.374)--(2.138,1.375)--(2.241,1.377)--(2.343,1.380)%
  --(2.445,1.384)--(2.548,1.389)--(2.650,1.395)--(2.752,1.403)--(2.854,1.412)--(2.957,1.425)%
  --(3.059,1.439)--(3.161,1.458)--(3.264,1.480)--(3.366,1.507)--(3.468,1.540)--(3.570,1.578)%
  --(3.673,1.623)--(3.775,1.677)--(3.877,1.738)--(3.980,1.810)--(4.082,1.891)--(4.184,1.983)%
  --(4.286,2.087)--(4.389,2.202)--(4.491,2.329)--(4.593,2.468)--(4.696,2.619)--(4.798,2.781)%
  --(4.900,2.953)--(5.003,3.134)--(5.105,3.323)--(5.207,3.516)--(5.309,3.713)--(5.412,3.911)%
  --(5.514,4.107)--(5.616,4.298)--(5.719,4.481)--(5.821,4.653)--(5.923,4.811)--(6.025,4.953)%
  --(6.128,5.074)--(6.230,5.175)--(6.332,5.251)--(6.435,5.302)--(6.537,5.327)--(6.639,5.325)%
  --(6.742,5.296)--(6.844,5.241)--(6.946,5.161)--(7.048,5.057)--(7.151,4.932)--(7.253,4.788)%
  --(7.355,4.628)--(7.458,4.454)--(7.560,4.269)--(7.662,4.077)--(7.764,3.881)--(7.867,3.683)%
  --(7.969,3.487)--(8.071,3.294)--(8.174,3.106)--(8.276,2.926)--(8.378,2.756)--(8.481,2.596)%
  --(8.583,2.447)--(8.685,2.309)--(8.787,2.184)--(8.890,2.070)--(8.992,1.968)--(9.094,1.878)%
  --(9.197,1.798)--(9.299,1.728)--(9.401,1.668)--(9.503,1.616)--(9.606,1.572)--(9.708,1.534)%
  --(9.810,1.503)--(9.913,1.477)--(10.015,1.455)--(10.117,1.437)--(10.219,1.423)--(10.322,1.411)%
  --(10.424,1.401)--(10.526,1.394)--(10.629,1.388)--(10.731,1.383)--(10.833,1.380)--(10.936,1.377)%
  --(11.038,1.375)--(11.140,1.373)--(11.242,1.372)--(11.345,1.371)--(11.447,1.371);
\gpcolor{color=gp lt color border}
\gpsetlinewidth{1.00}
\draw[gp path] (1.320,7.691)--(1.320,0.985)--(11.447,0.985)--(11.447,7.691)--cycle;
%% coordinates of the plot area
\gpdefrectangularnode{gp plot 1}{\pgfpoint{1.320cm}{0.985cm}}{\pgfpoint{11.447cm}{7.691cm}}
\end{tikzpicture}
%% gnuplot variables

\caption{太陽電波の測定値と$v(t)$}
\label{fig:solar_radio_fit}
\end{center}
\end{figure}
\clearpage
\subsubsection{太陽輝度温度算出}
図\ref{fig:solar_radio_Temp}の網掛け(赤),網掛け(青),網掛け(黄)領域で$V_{off}$, $V_{on}$, $V_R$を測定すると以下のようになる.
\begin{align}
  V_{off}=1.104\pm0.002\ \si{\volt}\\
  V_{on}=1.954\pm0.004\ \si{\volt}\\
  V_{R}=2.371\pm0.005\ \si{\volt}
\end{align}
太陽の平均の大気吸収補正済みアンテナ温度は
\begin{align}
  \overline{T_A^{*}}=\frac{\overline{V_{on}}-\overline{V_{off}}}{\overline{V_R}-\overline{V_{off}}}{T_R}=1.997\times10^2\si{\kelvin}\\
\end{align}
したがって太陽の平均の輝度温度は
\begin{align}
  \overline{T_{sun}}=\frac{\overline{T_A^{*}}}{\overline{\eta_c}}=2.2\times10^4\si{\kelvin}
\end{align}
また大気吸収補正済みアンテナ温度の標準偏差は
\begin{align}
  \delta_{T_A^{*}}&=\overline{T_A^{*}}\sqrt{\left(\frac{\delta_{V_{on}}-\delta_{V_{off}}}{\overline{V_{on}}-\overline{V_{off}}}\right)^2+\left(\frac{\delta_{V_{R}}-\delta_{V_{off}}}{\overline{V_{R}}-\overline{V_{off}}}\right)^2}\nonumber\\
  &=1.790\si{\kelvin}\nonumber
\end{align}
したがって太陽の輝度温度の標準偏差は
\begin{align}
  \delta_{T_{sun}}&=\overline{T_{sun}}\sqrt{\left(\frac{\delta_{T_A^{*}}}{\overline{T_A^{*}}}\right)^2+\left(\frac{\delta_{\eta_c}}{\overline{\eta_c}}\right)^2}\nonumber\\
  &=3.5\times10^3\si{\kelvin}
\end{align}
以上から太陽の輝度温度は以下のように求まる.
\begin{align}
  T_{sun}=(2.2\pm0.4)\times10^4\si{\kelvin}
\end{align}
\begin{figure}[hptb]
\begin{center}
\begin{tikzpicture}[gnuplot]
%% generated with GNUPLOT 5.2p8 (Lua 5.3; terminal rev. Nov 2018, script rev. 108)
%% 2021年05月14日 10時18分43秒
\path (0.000,0.000) rectangle (12.000,8.000);
\gpfill{rgb color={0.000,1.000,1.000},color=.!30} (6.116,0.985)--(6.251,0.985)--(6.251,7.690)--(6.116,7.690)--cycle;
\gpfill{rgb color={1.000,0.000,0.000},color=.!30} (2.895,0.985)--(3.345,0.985)--(3.345,7.690)--(2.895,7.690)--cycle;
\gpfill{rgb color={1.000,1.000,0.000},color=.!30} (9.647,0.985)--(10.097,0.985)--(10.097,7.690)--(9.647,7.690)--cycle;
\gpcolor{color=gp lt color border}
\gpsetlinetype{gp lt border}
\gpsetdashtype{gp dt solid}
\gpsetlinewidth{1.00}
\draw[gp path] (1.320,0.985)--(1.500,0.985);
\draw[gp path] (11.447,0.985)--(11.267,0.985);
\node[gp node right] at (1.136,0.985) {$1$};
\draw[gp path] (1.320,1.432)--(1.410,1.432);
\draw[gp path] (11.447,1.432)--(11.357,1.432);
\draw[gp path] (1.320,1.879)--(1.500,1.879);
\draw[gp path] (11.447,1.879)--(11.267,1.879);
\node[gp node right] at (1.136,1.879) {$1.2$};
\draw[gp path] (1.320,2.326)--(1.410,2.326);
\draw[gp path] (11.447,2.326)--(11.357,2.326);
\draw[gp path] (1.320,2.773)--(1.500,2.773);
\draw[gp path] (11.447,2.773)--(11.267,2.773);
\node[gp node right] at (1.136,2.773) {$1.4$};
\draw[gp path] (1.320,3.220)--(1.410,3.220);
\draw[gp path] (11.447,3.220)--(11.357,3.220);
\draw[gp path] (1.320,3.667)--(1.500,3.667);
\draw[gp path] (11.447,3.667)--(11.267,3.667);
\node[gp node right] at (1.136,3.667) {$1.6$};
\draw[gp path] (1.320,4.114)--(1.410,4.114);
\draw[gp path] (11.447,4.114)--(11.357,4.114);
\draw[gp path] (1.320,4.562)--(1.500,4.562);
\draw[gp path] (11.447,4.562)--(11.267,4.562);
\node[gp node right] at (1.136,4.562) {$1.8$};
\draw[gp path] (1.320,5.009)--(1.410,5.009);
\draw[gp path] (11.447,5.009)--(11.357,5.009);
\draw[gp path] (1.320,5.456)--(1.500,5.456);
\draw[gp path] (11.447,5.456)--(11.267,5.456);
\node[gp node right] at (1.136,5.456) {$2$};
\draw[gp path] (1.320,5.903)--(1.410,5.903);
\draw[gp path] (11.447,5.903)--(11.357,5.903);
\draw[gp path] (1.320,6.350)--(1.500,6.350);
\draw[gp path] (11.447,6.350)--(11.267,6.350);
\node[gp node right] at (1.136,6.350) {$2.2$};
\draw[gp path] (1.320,6.797)--(1.410,6.797);
\draw[gp path] (11.447,6.797)--(11.357,6.797);
\draw[gp path] (1.320,7.244)--(1.500,7.244);
\draw[gp path] (11.447,7.244)--(11.267,7.244);
\node[gp node right] at (1.136,7.244) {$2.4$};
\draw[gp path] (1.320,7.691)--(1.410,7.691);
\draw[gp path] (11.447,7.691)--(11.357,7.691);
\draw[gp path] (1.320,0.985)--(1.320,1.165);
\draw[gp path] (1.320,7.691)--(1.320,7.511);
\node[gp node center] at (1.320,0.677) {$0$};
\draw[gp path] (2.445,0.985)--(2.445,1.075);
\draw[gp path] (2.445,7.691)--(2.445,7.601);
\draw[gp path] (3.570,0.985)--(3.570,1.165);
\draw[gp path] (3.570,7.691)--(3.570,7.511);
\node[gp node center] at (3.570,0.677) {$1000$};
\draw[gp path] (4.696,0.985)--(4.696,1.075);
\draw[gp path] (4.696,7.691)--(4.696,7.601);
\draw[gp path] (5.821,0.985)--(5.821,1.165);
\draw[gp path] (5.821,7.691)--(5.821,7.511);
\node[gp node center] at (5.821,0.677) {$2000$};
\draw[gp path] (6.946,0.985)--(6.946,1.075);
\draw[gp path] (6.946,7.691)--(6.946,7.601);
\draw[gp path] (8.071,0.985)--(8.071,1.165);
\draw[gp path] (8.071,7.691)--(8.071,7.511);
\node[gp node center] at (8.071,0.677) {$3000$};
\draw[gp path] (9.197,0.985)--(9.197,1.075);
\draw[gp path] (9.197,7.691)--(9.197,7.601);
\draw[gp path] (10.322,0.985)--(10.322,1.165);
\draw[gp path] (10.322,7.691)--(10.322,7.511);
\node[gp node center] at (10.322,0.677) {$4000$};
\draw[gp path] (11.447,0.985)--(11.447,1.075);
\draw[gp path] (11.447,7.691)--(11.447,7.601);
\draw[gp path] (1.320,7.691)--(1.320,0.985)--(11.447,0.985)--(11.447,7.691)--cycle;
\node[gp node center,rotate=-270] at (0.292,4.338) {電圧 / $\si{\volt}$};
\node[gp node center] at (6.383,0.215) {経過時間$t$ / $\si{\second}$};
\gpsetpointsize{4.00}
\gppoint{gp mark 1}{(1.322,1.478)}
\gppoint{gp mark 1}{(1.325,3.330)}
\gppoint{gp mark 1}{(1.327,3.353)}
\gppoint{gp mark 1}{(1.329,3.360)}
\gppoint{gp mark 1}{(1.331,3.350)}
\gppoint{gp mark 1}{(1.334,2.732)}
\gppoint{gp mark 1}{(1.336,2.256)}
\gppoint{gp mark 1}{(1.338,2.643)}
\gppoint{gp mark 1}{(1.340,1.774)}
\gppoint{gp mark 1}{(1.343,4.487)}
\gppoint{gp mark 1}{(1.345,3.161)}
\gppoint{gp mark 1}{(1.347,4.018)}
\gppoint{gp mark 1}{(1.349,3.622)}
\gppoint{gp mark 1}{(1.352,3.318)}
\gppoint{gp mark 1}{(1.354,3.340)}
\gppoint{gp mark 1}{(1.356,3.155)}
\gppoint{gp mark 1}{(1.358,1.747)}
\gppoint{gp mark 1}{(1.361,1.741)}
\gppoint{gp mark 1}{(1.363,1.961)}
\gppoint{gp mark 1}{(1.365,3.051)}
\gppoint{gp mark 1}{(1.367,2.912)}
\gppoint{gp mark 1}{(1.370,3.522)}
\gppoint{gp mark 1}{(1.372,4.010)}
\gppoint{gp mark 1}{(1.374,4.767)}
\gppoint{gp mark 1}{(1.376,2.364)}
\gppoint{gp mark 1}{(1.379,3.603)}
\gppoint{gp mark 1}{(1.381,3.866)}
\gppoint{gp mark 1}{(1.383,3.844)}
\gppoint{gp mark 1}{(1.385,3.868)}
\gppoint{gp mark 1}{(1.388,3.840)}
\gppoint{gp mark 1}{(1.390,3.848)}
\gppoint{gp mark 1}{(1.392,3.852)}
\gppoint{gp mark 1}{(1.394,3.866)}
\gppoint{gp mark 1}{(1.397,3.854)}
\gppoint{gp mark 1}{(1.399,3.846)}
\gppoint{gp mark 1}{(1.401,3.856)}
\gppoint{gp mark 1}{(1.403,3.843)}
\gppoint{gp mark 1}{(1.406,3.856)}
\gppoint{gp mark 1}{(1.408,3.818)}
\gppoint{gp mark 1}{(1.410,3.830)}
\gppoint{gp mark 1}{(1.412,3.799)}
\gppoint{gp mark 1}{(1.415,3.795)}
\gppoint{gp mark 1}{(1.417,3.804)}
\gppoint{gp mark 1}{(1.419,3.792)}
\gppoint{gp mark 1}{(1.421,3.745)}
\gppoint{gp mark 1}{(1.424,3.785)}
\gppoint{gp mark 1}{(1.426,3.767)}
\gppoint{gp mark 1}{(1.428,3.759)}
\gppoint{gp mark 1}{(1.430,3.754)}
\gppoint{gp mark 1}{(1.433,3.756)}
\gppoint{gp mark 1}{(1.435,3.742)}
\gppoint{gp mark 1}{(1.437,3.744)}
\gppoint{gp mark 1}{(1.439,3.751)}
\gppoint{gp mark 1}{(1.442,3.714)}
\gppoint{gp mark 1}{(1.444,3.728)}
\gppoint{gp mark 1}{(1.446,3.695)}
\gppoint{gp mark 1}{(1.448,3.681)}
\gppoint{gp mark 1}{(1.451,3.690)}
\gppoint{gp mark 1}{(1.453,3.678)}
\gppoint{gp mark 1}{(1.455,3.655)}
\gppoint{gp mark 1}{(1.457,3.635)}
\gppoint{gp mark 1}{(1.460,3.624)}
\gppoint{gp mark 1}{(1.462,3.638)}
\gppoint{gp mark 1}{(1.464,3.648)}
\gppoint{gp mark 1}{(1.466,3.652)}
\gppoint{gp mark 1}{(1.469,3.598)}
\gppoint{gp mark 1}{(1.471,3.620)}
\gppoint{gp mark 1}{(1.473,3.616)}
\gppoint{gp mark 1}{(1.475,3.591)}
\gppoint{gp mark 1}{(1.478,3.153)}
\gppoint{gp mark 1}{(1.480,2.152)}
\gppoint{gp mark 1}{(1.482,1.812)}
\gppoint{gp mark 1}{(1.484,1.760)}
\gppoint{gp mark 1}{(1.487,1.737)}
\gppoint{gp mark 1}{(1.489,1.756)}
\gppoint{gp mark 1}{(1.491,1.823)}
\gppoint{gp mark 1}{(1.493,3.374)}
\gppoint{gp mark 1}{(1.496,4.328)}
\gppoint{gp mark 1}{(1.498,5.324)}
\gppoint{gp mark 1}{(1.500,5.492)}
\gppoint{gp mark 1}{(1.502,5.703)}
\gppoint{gp mark 1}{(1.505,5.757)}
\gppoint{gp mark 1}{(1.507,5.762)}
\gppoint{gp mark 1}{(1.509,5.671)}
\gppoint{gp mark 1}{(1.511,5.475)}
\gppoint{gp mark 1}{(1.514,5.272)}
\gppoint{gp mark 1}{(1.516,5.197)}
\gppoint{gp mark 1}{(1.518,5.233)}
\gppoint{gp mark 1}{(1.520,5.638)}
\gppoint{gp mark 1}{(1.523,5.666)}
\gppoint{gp mark 1}{(1.525,5.682)}
\gppoint{gp mark 1}{(1.527,5.733)}
\gppoint{gp mark 1}{(1.529,5.747)}
\gppoint{gp mark 1}{(1.532,5.781)}
\gppoint{gp mark 1}{(1.534,5.779)}
\gppoint{gp mark 1}{(1.536,5.740)}
\gppoint{gp mark 1}{(1.538,5.771)}
\gppoint{gp mark 1}{(1.541,5.749)}
\gppoint{gp mark 1}{(1.543,5.771)}
\gppoint{gp mark 1}{(1.545,5.710)}
\gppoint{gp mark 1}{(1.547,5.725)}
\gppoint{gp mark 1}{(1.550,5.698)}
\gppoint{gp mark 1}{(1.552,5.676)}
\gppoint{gp mark 1}{(1.554,5.649)}
\gppoint{gp mark 1}{(1.556,5.655)}
\gppoint{gp mark 1}{(1.559,5.581)}
\gppoint{gp mark 1}{(1.561,5.609)}
\gppoint{gp mark 1}{(1.563,5.594)}
\gppoint{gp mark 1}{(1.565,5.768)}
\gppoint{gp mark 1}{(1.568,5.745)}
\gppoint{gp mark 1}{(1.570,5.742)}
\gppoint{gp mark 1}{(1.572,5.715)}
\gppoint{gp mark 1}{(1.574,5.745)}
\gppoint{gp mark 1}{(1.577,5.720)}
\gppoint{gp mark 1}{(1.579,5.738)}
\gppoint{gp mark 1}{(1.581,5.779)}
\gppoint{gp mark 1}{(1.583,5.758)}
\gppoint{gp mark 1}{(1.586,5.757)}
\gppoint{gp mark 1}{(1.588,5.746)}
\gppoint{gp mark 1}{(1.590,5.738)}
\gppoint{gp mark 1}{(1.592,5.699)}
\gppoint{gp mark 1}{(1.595,5.685)}
\gppoint{gp mark 1}{(1.597,5.653)}
\gppoint{gp mark 1}{(1.599,5.662)}
\gppoint{gp mark 1}{(1.601,5.627)}
\gppoint{gp mark 1}{(1.604,5.647)}
\gppoint{gp mark 1}{(1.606,5.691)}
\gppoint{gp mark 1}{(1.608,5.701)}
\gppoint{gp mark 1}{(1.610,5.686)}
\gppoint{gp mark 1}{(1.613,5.670)}
\gppoint{gp mark 1}{(1.615,5.682)}
\gppoint{gp mark 1}{(1.617,2.557)}
\gppoint{gp mark 1}{(1.619,3.456)}
\gppoint{gp mark 1}{(1.622,2.686)}
\gppoint{gp mark 1}{(1.624,1.830)}
\gppoint{gp mark 1}{(1.626,1.839)}
\gppoint{gp mark 1}{(1.628,1.927)}
\gppoint{gp mark 1}{(1.631,1.712)}
\gppoint{gp mark 1}{(1.633,1.677)}
\gppoint{gp mark 1}{(1.635,1.671)}
\gppoint{gp mark 1}{(1.637,1.903)}
\gppoint{gp mark 1}{(1.640,2.616)}
\gppoint{gp mark 1}{(1.642,3.818)}
\gppoint{gp mark 1}{(1.644,4.557)}
\gppoint{gp mark 1}{(1.646,5.087)}
\gppoint{gp mark 1}{(1.649,5.320)}
\gppoint{gp mark 1}{(1.651,5.624)}
\gppoint{gp mark 1}{(1.653,5.739)}
\gppoint{gp mark 1}{(1.655,5.669)}
\gppoint{gp mark 1}{(1.658,5.483)}
\gppoint{gp mark 1}{(1.660,5.312)}
\gppoint{gp mark 1}{(1.662,5.113)}
\gppoint{gp mark 1}{(1.664,4.717)}
\gppoint{gp mark 1}{(1.667,4.348)}
\gppoint{gp mark 1}{(1.669,4.089)}
\gppoint{gp mark 1}{(1.671,3.826)}
\gppoint{gp mark 1}{(1.673,3.099)}
\gppoint{gp mark 1}{(1.676,2.491)}
\gppoint{gp mark 1}{(1.678,2.230)}
\gppoint{gp mark 1}{(1.680,1.839)}
\gppoint{gp mark 1}{(1.682,1.670)}
\gppoint{gp mark 1}{(1.685,1.637)}
\gppoint{gp mark 1}{(1.687,1.654)}
\gppoint{gp mark 1}{(1.689,1.648)}
\gppoint{gp mark 1}{(1.691,1.646)}
\gppoint{gp mark 1}{(1.694,1.638)}
\gppoint{gp mark 1}{(1.696,1.634)}
\gppoint{gp mark 1}{(1.698,1.634)}
\gppoint{gp mark 1}{(1.700,1.643)}
\gppoint{gp mark 1}{(1.703,1.656)}
\gppoint{gp mark 1}{(1.705,1.645)}
\gppoint{gp mark 1}{(1.707,1.634)}
\gppoint{gp mark 1}{(1.709,1.695)}
\gppoint{gp mark 1}{(1.712,1.811)}
\gppoint{gp mark 1}{(1.714,1.920)}
\gppoint{gp mark 1}{(1.716,2.177)}
\gppoint{gp mark 1}{(1.718,2.333)}
\gppoint{gp mark 1}{(1.721,2.383)}
\gppoint{gp mark 1}{(1.723,2.477)}
\gppoint{gp mark 1}{(1.725,2.725)}
\gppoint{gp mark 1}{(1.727,3.144)}
\gppoint{gp mark 1}{(1.730,3.589)}
\gppoint{gp mark 1}{(1.732,3.971)}
\gppoint{gp mark 1}{(1.734,4.258)}
\gppoint{gp mark 1}{(1.736,4.529)}
\gppoint{gp mark 1}{(1.739,4.773)}
\gppoint{gp mark 1}{(1.741,5.181)}
\gppoint{gp mark 1}{(1.743,5.346)}
\gppoint{gp mark 1}{(1.745,5.530)}
\gppoint{gp mark 1}{(1.748,5.640)}
\gppoint{gp mark 1}{(1.750,5.623)}
\gppoint{gp mark 1}{(1.752,5.599)}
\gppoint{gp mark 1}{(1.754,5.580)}
\gppoint{gp mark 1}{(1.757,5.620)}
\gppoint{gp mark 1}{(1.759,5.610)}
\gppoint{gp mark 1}{(1.761,5.488)}
\gppoint{gp mark 1}{(1.763,5.337)}
\gppoint{gp mark 1}{(1.766,5.280)}
\gppoint{gp mark 1}{(1.768,5.246)}
\gppoint{gp mark 1}{(1.770,5.259)}
\gppoint{gp mark 1}{(1.772,5.348)}
\gppoint{gp mark 1}{(1.775,5.448)}
\gppoint{gp mark 1}{(1.777,5.512)}
\gppoint{gp mark 1}{(1.779,5.610)}
\gppoint{gp mark 1}{(1.781,5.582)}
\gppoint{gp mark 1}{(1.784,5.611)}
\gppoint{gp mark 1}{(1.786,5.566)}
\gppoint{gp mark 1}{(1.788,5.539)}
\gppoint{gp mark 1}{(1.790,5.555)}
\gppoint{gp mark 1}{(1.793,5.546)}
\gppoint{gp mark 1}{(1.795,5.454)}
\gppoint{gp mark 1}{(1.797,5.400)}
\gppoint{gp mark 1}{(1.799,5.204)}
\gppoint{gp mark 1}{(1.802,5.227)}
\gppoint{gp mark 1}{(1.804,5.269)}
\gppoint{gp mark 1}{(1.806,5.453)}
\gppoint{gp mark 1}{(1.808,5.466)}
\gppoint{gp mark 1}{(1.811,5.570)}
\gppoint{gp mark 1}{(1.813,5.554)}
\gppoint{gp mark 1}{(1.815,5.551)}
\gppoint{gp mark 1}{(1.817,5.582)}
\gppoint{gp mark 1}{(1.820,5.543)}
\gppoint{gp mark 1}{(1.822,5.566)}
\gppoint{gp mark 1}{(1.824,5.526)}
\gppoint{gp mark 1}{(1.826,5.269)}
\gppoint{gp mark 1}{(1.829,5.428)}
\gppoint{gp mark 1}{(1.831,5.596)}
\gppoint{gp mark 1}{(1.833,5.655)}
\gppoint{gp mark 1}{(1.835,5.502)}
\gppoint{gp mark 1}{(1.838,5.523)}
\gppoint{gp mark 1}{(1.840,5.531)}
\gppoint{gp mark 1}{(1.842,5.103)}
\gppoint{gp mark 1}{(1.844,4.861)}
\gppoint{gp mark 1}{(1.847,4.963)}
\gppoint{gp mark 1}{(1.849,3.602)}
\gppoint{gp mark 1}{(1.851,2.612)}
\gppoint{gp mark 1}{(1.853,2.626)}
\gppoint{gp mark 1}{(1.856,2.604)}
\gppoint{gp mark 1}{(1.858,2.598)}
\gppoint{gp mark 1}{(1.860,2.578)}
\gppoint{gp mark 1}{(1.862,2.572)}
\gppoint{gp mark 1}{(1.865,2.592)}
\gppoint{gp mark 1}{(1.867,2.596)}
\gppoint{gp mark 1}{(1.869,2.569)}
\gppoint{gp mark 1}{(1.871,2.556)}
\gppoint{gp mark 1}{(1.874,2.651)}
\gppoint{gp mark 1}{(1.876,2.576)}
\gppoint{gp mark 1}{(1.878,2.541)}
\gppoint{gp mark 1}{(1.880,1.699)}
\gppoint{gp mark 1}{(1.883,3.485)}
\gppoint{gp mark 1}{(1.885,4.835)}
\gppoint{gp mark 1}{(1.887,4.746)}
\gppoint{gp mark 1}{(1.889,4.313)}
\gppoint{gp mark 1}{(1.892,4.903)}
\gppoint{gp mark 1}{(1.894,4.927)}
\gppoint{gp mark 1}{(1.896,4.906)}
\gppoint{gp mark 1}{(1.898,4.915)}
\gppoint{gp mark 1}{(1.901,4.735)}
\gppoint{gp mark 1}{(1.903,2.669)}
\gppoint{gp mark 1}{(1.905,4.382)}
\gppoint{gp mark 1}{(1.907,4.561)}
\gppoint{gp mark 1}{(1.910,4.533)}
\gppoint{gp mark 1}{(1.912,4.569)}
\gppoint{gp mark 1}{(1.914,4.533)}
\gppoint{gp mark 1}{(1.916,4.621)}
\gppoint{gp mark 1}{(1.919,4.731)}
\gppoint{gp mark 1}{(1.921,4.798)}
\gppoint{gp mark 1}{(1.923,4.926)}
\gppoint{gp mark 1}{(1.925,4.838)}
\gppoint{gp mark 1}{(1.928,4.876)}
\gppoint{gp mark 1}{(1.930,4.685)}
\gppoint{gp mark 1}{(1.932,4.412)}
\gppoint{gp mark 1}{(1.934,4.309)}
\gppoint{gp mark 1}{(1.937,4.326)}
\gppoint{gp mark 1}{(1.939,4.664)}
\gppoint{gp mark 1}{(1.941,4.652)}
\gppoint{gp mark 1}{(1.943,4.628)}
\gppoint{gp mark 1}{(1.946,4.635)}
\gppoint{gp mark 1}{(1.948,4.577)}
\gppoint{gp mark 1}{(1.950,4.678)}
\gppoint{gp mark 1}{(1.952,4.738)}
\gppoint{gp mark 1}{(1.955,4.697)}
\gppoint{gp mark 1}{(1.957,4.521)}
\gppoint{gp mark 1}{(1.959,4.848)}
\gppoint{gp mark 1}{(1.961,5.210)}
\gppoint{gp mark 1}{(1.964,5.222)}
\gppoint{gp mark 1}{(1.966,5.207)}
\gppoint{gp mark 1}{(1.968,5.191)}
\gppoint{gp mark 1}{(1.970,5.169)}
\gppoint{gp mark 1}{(1.973,5.179)}
\gppoint{gp mark 1}{(1.975,5.197)}
\gppoint{gp mark 1}{(1.977,5.195)}
\gppoint{gp mark 1}{(1.979,5.199)}
\gppoint{gp mark 1}{(1.982,5.158)}
\gppoint{gp mark 1}{(1.984,5.160)}
\gppoint{gp mark 1}{(1.986,5.135)}
\gppoint{gp mark 1}{(1.988,5.167)}
\gppoint{gp mark 1}{(1.991,5.163)}
\gppoint{gp mark 1}{(1.993,5.126)}
\gppoint{gp mark 1}{(1.995,5.137)}
\gppoint{gp mark 1}{(1.997,5.065)}
\gppoint{gp mark 1}{(2.000,3.087)}
\gppoint{gp mark 1}{(2.002,2.456)}
\gppoint{gp mark 1}{(2.004,3.411)}
\gppoint{gp mark 1}{(2.006,4.182)}
\gppoint{gp mark 1}{(2.009,4.869)}
\gppoint{gp mark 1}{(2.011,4.895)}
\gppoint{gp mark 1}{(2.013,4.868)}
\gppoint{gp mark 1}{(2.015,4.876)}
\gppoint{gp mark 1}{(2.018,4.877)}
\gppoint{gp mark 1}{(2.020,4.858)}
\gppoint{gp mark 1}{(2.022,5.018)}
\gppoint{gp mark 1}{(2.024,5.046)}
\gppoint{gp mark 1}{(2.027,5.027)}
\gppoint{gp mark 1}{(2.029,5.080)}
\gppoint{gp mark 1}{(2.031,5.099)}
\gppoint{gp mark 1}{(2.033,5.098)}
\gppoint{gp mark 1}{(2.036,5.061)}
\gppoint{gp mark 1}{(2.038,5.083)}
\gppoint{gp mark 1}{(2.040,5.071)}
\gppoint{gp mark 1}{(2.042,4.841)}
\gppoint{gp mark 1}{(2.045,4.529)}
\gppoint{gp mark 1}{(2.047,4.559)}
\gppoint{gp mark 1}{(2.049,4.816)}
\gppoint{gp mark 1}{(2.051,4.958)}
\gppoint{gp mark 1}{(2.054,5.028)}
\gppoint{gp mark 1}{(2.056,5.022)}
\gppoint{gp mark 1}{(2.058,5.005)}
\gppoint{gp mark 1}{(2.060,4.957)}
\gppoint{gp mark 1}{(2.063,4.937)}
\gppoint{gp mark 1}{(2.065,4.914)}
\gppoint{gp mark 1}{(2.067,4.934)}
\gppoint{gp mark 1}{(2.069,4.934)}
\gppoint{gp mark 1}{(2.072,4.886)}
\gppoint{gp mark 1}{(2.074,4.392)}
\gppoint{gp mark 1}{(2.076,4.374)}
\gppoint{gp mark 1}{(2.078,4.565)}
\gppoint{gp mark 1}{(2.081,4.940)}
\gppoint{gp mark 1}{(2.083,4.945)}
\gppoint{gp mark 1}{(2.085,4.939)}
\gppoint{gp mark 1}{(2.087,4.932)}
\gppoint{gp mark 1}{(2.090,4.902)}
\gppoint{gp mark 1}{(2.092,4.909)}
\gppoint{gp mark 1}{(2.094,4.920)}
\gppoint{gp mark 1}{(2.096,4.860)}
\gppoint{gp mark 1}{(2.099,4.864)}
\gppoint{gp mark 1}{(2.101,4.842)}
\gppoint{gp mark 1}{(2.103,4.843)}
\gppoint{gp mark 1}{(2.105,4.864)}
\gppoint{gp mark 1}{(2.108,4.856)}
\gppoint{gp mark 1}{(2.110,4.877)}
\gppoint{gp mark 1}{(2.112,4.841)}
\gppoint{gp mark 1}{(2.114,4.856)}
\gppoint{gp mark 1}{(2.117,4.855)}
\gppoint{gp mark 1}{(2.119,4.865)}
\gppoint{gp mark 1}{(2.121,4.887)}
\gppoint{gp mark 1}{(2.123,4.906)}
\gppoint{gp mark 1}{(2.126,4.883)}
\gppoint{gp mark 1}{(2.128,4.886)}
\gppoint{gp mark 1}{(2.130,4.865)}
\gppoint{gp mark 1}{(2.132,4.842)}
\gppoint{gp mark 1}{(2.135,4.837)}
\gppoint{gp mark 1}{(2.137,4.839)}
\gppoint{gp mark 1}{(2.139,4.832)}
\gppoint{gp mark 1}{(2.141,4.862)}
\gppoint{gp mark 1}{(2.144,4.853)}
\gppoint{gp mark 1}{(2.146,4.828)}
\gppoint{gp mark 1}{(2.148,4.824)}
\gppoint{gp mark 1}{(2.150,4.835)}
\gppoint{gp mark 1}{(2.153,5.020)}
\gppoint{gp mark 1}{(2.155,5.239)}
\gppoint{gp mark 1}{(2.157,5.235)}
\gppoint{gp mark 1}{(2.159,5.360)}
\gppoint{gp mark 1}{(2.162,5.446)}
\gppoint{gp mark 1}{(2.164,5.572)}
\gppoint{gp mark 1}{(2.166,5.557)}
\gppoint{gp mark 1}{(2.168,5.545)}
\gppoint{gp mark 1}{(2.171,4.986)}
\gppoint{gp mark 1}{(2.173,4.481)}
\gppoint{gp mark 1}{(2.175,4.492)}
\gppoint{gp mark 1}{(2.177,4.500)}
\gppoint{gp mark 1}{(2.180,4.515)}
\gppoint{gp mark 1}{(2.182,4.455)}
\gppoint{gp mark 1}{(2.184,4.575)}
\gppoint{gp mark 1}{(2.186,5.150)}
\gppoint{gp mark 1}{(2.189,5.488)}
\gppoint{gp mark 1}{(2.191,5.579)}
\gppoint{gp mark 1}{(2.193,5.589)}
\gppoint{gp mark 1}{(2.195,5.589)}
\gppoint{gp mark 1}{(2.198,5.548)}
\gppoint{gp mark 1}{(2.200,5.508)}
\gppoint{gp mark 1}{(2.202,5.486)}
\gppoint{gp mark 1}{(2.204,5.474)}
\gppoint{gp mark 1}{(2.207,5.528)}
\gppoint{gp mark 1}{(2.209,5.564)}
\gppoint{gp mark 1}{(2.211,5.551)}
\gppoint{gp mark 1}{(2.213,5.566)}
\gppoint{gp mark 1}{(2.216,5.573)}
\gppoint{gp mark 1}{(2.218,5.566)}
\gppoint{gp mark 1}{(2.220,5.533)}
\gppoint{gp mark 1}{(2.222,5.582)}
\gppoint{gp mark 1}{(2.225,5.573)}
\gppoint{gp mark 1}{(2.227,5.583)}
\gppoint{gp mark 1}{(2.229,5.590)}
\gppoint{gp mark 1}{(2.231,5.611)}
\gppoint{gp mark 1}{(2.234,5.545)}
\gppoint{gp mark 1}{(2.236,5.540)}
\gppoint{gp mark 1}{(2.238,5.525)}
\gppoint{gp mark 1}{(2.240,5.501)}
\gppoint{gp mark 1}{(2.243,5.463)}
\gppoint{gp mark 1}{(2.245,5.453)}
\gppoint{gp mark 1}{(2.247,5.449)}
\gppoint{gp mark 1}{(2.249,5.444)}
\gppoint{gp mark 1}{(2.252,5.531)}
\gppoint{gp mark 1}{(2.254,5.583)}
\gppoint{gp mark 1}{(2.256,5.552)}
\gppoint{gp mark 1}{(2.258,5.535)}
\gppoint{gp mark 1}{(2.261,5.559)}
\gppoint{gp mark 1}{(2.263,5.578)}
\gppoint{gp mark 1}{(2.265,5.604)}
\gppoint{gp mark 1}{(2.267,5.565)}
\gppoint{gp mark 1}{(2.270,5.552)}
\gppoint{gp mark 1}{(2.272,5.564)}
\gppoint{gp mark 1}{(2.274,5.540)}
\gppoint{gp mark 1}{(2.276,5.567)}
\gppoint{gp mark 1}{(2.279,5.560)}
\gppoint{gp mark 1}{(2.281,5.607)}
\gppoint{gp mark 1}{(2.283,5.580)}
\gppoint{gp mark 1}{(2.285,5.550)}
\gppoint{gp mark 1}{(2.288,5.585)}
\gppoint{gp mark 1}{(2.290,5.603)}
\gppoint{gp mark 1}{(2.292,5.594)}
\gppoint{gp mark 1}{(2.294,5.611)}
\gppoint{gp mark 1}{(2.297,5.571)}
\gppoint{gp mark 1}{(2.299,5.557)}
\gppoint{gp mark 1}{(2.301,5.541)}
\gppoint{gp mark 1}{(2.303,5.547)}
\gppoint{gp mark 1}{(2.306,5.542)}
\gppoint{gp mark 1}{(2.308,5.545)}
\gppoint{gp mark 1}{(2.310,5.526)}
\gppoint{gp mark 1}{(2.312,5.564)}
\gppoint{gp mark 1}{(2.315,5.567)}
\gppoint{gp mark 1}{(2.317,5.552)}
\gppoint{gp mark 1}{(2.319,5.568)}
\gppoint{gp mark 1}{(2.321,5.569)}
\gppoint{gp mark 1}{(2.324,5.543)}
\gppoint{gp mark 1}{(2.326,5.593)}
\gppoint{gp mark 1}{(2.328,5.525)}
\gppoint{gp mark 1}{(2.330,5.545)}
\gppoint{gp mark 1}{(2.333,5.535)}
\gppoint{gp mark 1}{(2.335,5.523)}
\gppoint{gp mark 1}{(2.337,5.552)}
\gppoint{gp mark 1}{(2.339,5.535)}
\gppoint{gp mark 1}{(2.342,5.519)}
\gppoint{gp mark 1}{(2.344,5.549)}
\gppoint{gp mark 1}{(2.346,5.542)}
\gppoint{gp mark 1}{(2.348,5.554)}
\gppoint{gp mark 1}{(2.351,5.540)}
\gppoint{gp mark 1}{(2.353,5.564)}
\gppoint{gp mark 1}{(2.355,5.571)}
\gppoint{gp mark 1}{(2.357,5.534)}
\gppoint{gp mark 1}{(2.360,5.522)}
\gppoint{gp mark 1}{(2.362,5.550)}
\gppoint{gp mark 1}{(2.364,5.481)}
\gppoint{gp mark 1}{(2.366,5.479)}
\gppoint{gp mark 1}{(2.369,5.490)}
\gppoint{gp mark 1}{(2.371,5.491)}
\gppoint{gp mark 1}{(2.373,5.511)}
\gppoint{gp mark 1}{(2.375,5.536)}
\gppoint{gp mark 1}{(2.378,5.514)}
\gppoint{gp mark 1}{(2.380,5.523)}
\gppoint{gp mark 1}{(2.382,5.538)}
\gppoint{gp mark 1}{(2.384,5.518)}
\gppoint{gp mark 1}{(2.387,5.481)}
\gppoint{gp mark 1}{(2.389,5.520)}
\gppoint{gp mark 1}{(2.391,5.517)}
\gppoint{gp mark 1}{(2.393,5.513)}
\gppoint{gp mark 1}{(2.396,5.520)}
\gppoint{gp mark 1}{(2.398,5.499)}
\gppoint{gp mark 1}{(2.400,5.518)}
\gppoint{gp mark 1}{(2.402,5.537)}
\gppoint{gp mark 1}{(2.405,5.525)}
\gppoint{gp mark 1}{(2.407,5.502)}
\gppoint{gp mark 1}{(2.409,5.520)}
\gppoint{gp mark 1}{(2.411,5.520)}
\gppoint{gp mark 1}{(2.414,5.507)}
\gppoint{gp mark 1}{(2.416,5.485)}
\gppoint{gp mark 1}{(2.418,5.475)}
\gppoint{gp mark 1}{(2.420,5.512)}
\gppoint{gp mark 1}{(2.423,5.503)}
\gppoint{gp mark 1}{(2.425,5.512)}
\gppoint{gp mark 1}{(2.427,5.452)}
\gppoint{gp mark 1}{(2.429,5.476)}
\gppoint{gp mark 1}{(2.432,5.483)}
\gppoint{gp mark 1}{(2.434,5.474)}
\gppoint{gp mark 1}{(2.436,5.476)}
\gppoint{gp mark 1}{(2.438,5.452)}
\gppoint{gp mark 1}{(2.441,5.450)}
\gppoint{gp mark 1}{(2.443,5.522)}
\gppoint{gp mark 1}{(2.445,5.463)}
\gppoint{gp mark 1}{(2.447,5.475)}
\gppoint{gp mark 1}{(2.450,5.489)}
\gppoint{gp mark 1}{(2.452,5.457)}
\gppoint{gp mark 1}{(2.454,5.466)}
\gppoint{gp mark 1}{(2.456,5.431)}
\gppoint{gp mark 1}{(2.459,5.427)}
\gppoint{gp mark 1}{(2.461,5.426)}
\gppoint{gp mark 1}{(2.463,5.410)}
\gppoint{gp mark 1}{(2.465,5.392)}
\gppoint{gp mark 1}{(2.468,5.428)}
\gppoint{gp mark 1}{(2.470,5.400)}
\gppoint{gp mark 1}{(2.472,5.432)}
\gppoint{gp mark 1}{(2.474,5.443)}
\gppoint{gp mark 1}{(2.477,5.437)}
\gppoint{gp mark 1}{(2.479,5.450)}
\gppoint{gp mark 1}{(2.481,5.430)}
\gppoint{gp mark 1}{(2.483,2.977)}
\gppoint{gp mark 1}{(2.486,2.540)}
\gppoint{gp mark 1}{(2.488,4.763)}
\gppoint{gp mark 1}{(2.490,5.258)}
\gppoint{gp mark 1}{(2.492,5.278)}
\gppoint{gp mark 1}{(2.495,5.094)}
\gppoint{gp mark 1}{(2.497,1.717)}
\gppoint{gp mark 1}{(2.499,1.450)}
\gppoint{gp mark 1}{(2.501,1.458)}
\gppoint{gp mark 1}{(2.504,1.470)}
\gppoint{gp mark 1}{(2.506,1.456)}
\gppoint{gp mark 1}{(2.508,1.454)}
\gppoint{gp mark 1}{(2.510,1.455)}
\gppoint{gp mark 1}{(2.513,1.413)}
\gppoint{gp mark 1}{(2.515,1.418)}
\gppoint{gp mark 1}{(2.517,1.447)}
\gppoint{gp mark 1}{(2.519,1.442)}
\gppoint{gp mark 1}{(2.522,1.456)}
\gppoint{gp mark 1}{(2.524,1.454)}
\gppoint{gp mark 1}{(2.526,1.458)}
\gppoint{gp mark 1}{(2.528,1.435)}
\gppoint{gp mark 1}{(2.531,1.431)}
\gppoint{gp mark 1}{(2.533,1.434)}
\gppoint{gp mark 1}{(2.535,1.439)}
\gppoint{gp mark 1}{(2.537,1.433)}
\gppoint{gp mark 1}{(2.540,1.422)}
\gppoint{gp mark 1}{(2.542,1.440)}
\gppoint{gp mark 1}{(2.544,1.429)}
\gppoint{gp mark 1}{(2.546,1.408)}
\gppoint{gp mark 1}{(2.549,1.417)}
\gppoint{gp mark 1}{(2.551,1.420)}
\gppoint{gp mark 1}{(2.553,1.436)}
\gppoint{gp mark 1}{(2.555,1.448)}
\gppoint{gp mark 1}{(2.558,1.453)}
\gppoint{gp mark 1}{(2.560,1.455)}
\gppoint{gp mark 1}{(2.562,1.452)}
\gppoint{gp mark 1}{(2.564,1.456)}
\gppoint{gp mark 1}{(2.567,1.472)}
\gppoint{gp mark 1}{(2.569,1.489)}
\gppoint{gp mark 1}{(2.571,1.485)}
\gppoint{gp mark 1}{(2.573,1.490)}
\gppoint{gp mark 1}{(2.576,1.469)}
\gppoint{gp mark 1}{(2.578,1.495)}
\gppoint{gp mark 1}{(2.580,1.516)}
\gppoint{gp mark 1}{(2.582,1.712)}
\gppoint{gp mark 1}{(2.585,2.046)}
\gppoint{gp mark 1}{(2.587,2.177)}
\gppoint{gp mark 1}{(2.589,2.177)}
\gppoint{gp mark 1}{(2.592,2.167)}
\gppoint{gp mark 1}{(2.594,1.589)}
\gppoint{gp mark 1}{(2.596,1.492)}
\gppoint{gp mark 1}{(2.598,1.471)}
\gppoint{gp mark 1}{(2.601,1.469)}
\gppoint{gp mark 1}{(2.603,1.469)}
\gppoint{gp mark 1}{(2.605,1.454)}
\gppoint{gp mark 1}{(2.607,1.458)}
\gppoint{gp mark 1}{(2.610,1.464)}
\gppoint{gp mark 1}{(2.612,1.461)}
\gppoint{gp mark 1}{(2.614,1.452)}
\gppoint{gp mark 1}{(2.616,1.450)}
\gppoint{gp mark 1}{(2.619,1.449)}
\gppoint{gp mark 1}{(2.621,1.474)}
\gppoint{gp mark 1}{(2.623,1.453)}
\gppoint{gp mark 1}{(2.625,1.449)}
\gppoint{gp mark 1}{(2.628,1.448)}
\gppoint{gp mark 1}{(2.630,1.461)}
\gppoint{gp mark 1}{(2.632,1.455)}
\gppoint{gp mark 1}{(2.634,1.453)}
\gppoint{gp mark 1}{(2.637,1.463)}
\gppoint{gp mark 1}{(2.639,1.445)}
\gppoint{gp mark 1}{(2.641,1.459)}
\gppoint{gp mark 1}{(2.643,1.457)}
\gppoint{gp mark 1}{(2.646,1.451)}
\gppoint{gp mark 1}{(2.648,1.448)}
\gppoint{gp mark 1}{(2.650,1.463)}
\gppoint{gp mark 1}{(2.652,1.462)}
\gppoint{gp mark 1}{(2.655,1.457)}
\gppoint{gp mark 1}{(2.657,1.462)}
\gppoint{gp mark 1}{(2.659,1.456)}
\gppoint{gp mark 1}{(2.661,1.456)}
\gppoint{gp mark 1}{(2.664,1.458)}
\gppoint{gp mark 1}{(2.666,1.459)}
\gppoint{gp mark 1}{(2.668,1.472)}
\gppoint{gp mark 1}{(2.670,1.465)}
\gppoint{gp mark 1}{(2.673,1.456)}
\gppoint{gp mark 1}{(2.675,1.441)}
\gppoint{gp mark 1}{(2.677,1.450)}
\gppoint{gp mark 1}{(2.679,1.463)}
\gppoint{gp mark 1}{(2.682,1.445)}
\gppoint{gp mark 1}{(2.684,1.450)}
\gppoint{gp mark 1}{(2.686,1.439)}
\gppoint{gp mark 1}{(2.688,1.450)}
\gppoint{gp mark 1}{(2.691,1.447)}
\gppoint{gp mark 1}{(2.693,1.448)}
\gppoint{gp mark 1}{(2.695,1.451)}
\gppoint{gp mark 1}{(2.697,1.465)}
\gppoint{gp mark 1}{(2.700,1.452)}
\gppoint{gp mark 1}{(2.702,1.457)}
\gppoint{gp mark 1}{(2.704,1.442)}
\gppoint{gp mark 1}{(2.706,1.445)}
\gppoint{gp mark 1}{(2.709,1.457)}
\gppoint{gp mark 1}{(2.711,1.468)}
\gppoint{gp mark 1}{(2.713,1.459)}
\gppoint{gp mark 1}{(2.715,1.470)}
\gppoint{gp mark 1}{(2.718,1.439)}
\gppoint{gp mark 1}{(2.720,1.461)}
\gppoint{gp mark 1}{(2.722,1.447)}
\gppoint{gp mark 1}{(2.724,1.461)}
\gppoint{gp mark 1}{(2.727,1.448)}
\gppoint{gp mark 1}{(2.729,1.450)}
\gppoint{gp mark 1}{(2.731,1.449)}
\gppoint{gp mark 1}{(2.733,1.472)}
\gppoint{gp mark 1}{(2.736,1.471)}
\gppoint{gp mark 1}{(2.738,1.455)}
\gppoint{gp mark 1}{(2.740,1.459)}
\gppoint{gp mark 1}{(2.742,1.439)}
\gppoint{gp mark 1}{(2.745,1.438)}
\gppoint{gp mark 1}{(2.747,1.464)}
\gppoint{gp mark 1}{(2.749,1.460)}
\gppoint{gp mark 1}{(2.751,1.465)}
\gppoint{gp mark 1}{(2.754,1.450)}
\gppoint{gp mark 1}{(2.756,1.436)}
\gppoint{gp mark 1}{(2.758,1.462)}
\gppoint{gp mark 1}{(2.760,1.452)}
\gppoint{gp mark 1}{(2.763,1.453)}
\gppoint{gp mark 1}{(2.765,1.454)}
\gppoint{gp mark 1}{(2.767,1.459)}
\gppoint{gp mark 1}{(2.769,1.452)}
\gppoint{gp mark 1}{(2.772,1.450)}
\gppoint{gp mark 1}{(2.774,1.457)}
\gppoint{gp mark 1}{(2.776,1.451)}
\gppoint{gp mark 1}{(2.778,1.450)}
\gppoint{gp mark 1}{(2.781,1.446)}
\gppoint{gp mark 1}{(2.783,1.455)}
\gppoint{gp mark 1}{(2.785,1.458)}
\gppoint{gp mark 1}{(2.787,1.452)}
\gppoint{gp mark 1}{(2.790,1.469)}
\gppoint{gp mark 1}{(2.792,1.475)}
\gppoint{gp mark 1}{(2.794,1.440)}
\gppoint{gp mark 1}{(2.796,1.469)}
\gppoint{gp mark 1}{(2.799,1.467)}
\gppoint{gp mark 1}{(2.801,1.467)}
\gppoint{gp mark 1}{(2.803,1.468)}
\gppoint{gp mark 1}{(2.805,1.457)}
\gppoint{gp mark 1}{(2.808,1.462)}
\gppoint{gp mark 1}{(2.810,1.473)}
\gppoint{gp mark 1}{(2.812,1.446)}
\gppoint{gp mark 1}{(2.814,1.462)}
\gppoint{gp mark 1}{(2.817,1.462)}
\gppoint{gp mark 1}{(2.819,1.463)}
\gppoint{gp mark 1}{(2.821,1.451)}
\gppoint{gp mark 1}{(2.823,1.467)}
\gppoint{gp mark 1}{(2.826,1.457)}
\gppoint{gp mark 1}{(2.828,1.458)}
\gppoint{gp mark 1}{(2.830,1.460)}
\gppoint{gp mark 1}{(2.832,1.450)}
\gppoint{gp mark 1}{(2.835,1.448)}
\gppoint{gp mark 1}{(2.837,1.464)}
\gppoint{gp mark 1}{(2.839,1.456)}
\gppoint{gp mark 1}{(2.841,1.462)}
\gppoint{gp mark 1}{(2.844,1.447)}
\gppoint{gp mark 1}{(2.846,1.452)}
\gppoint{gp mark 1}{(2.848,1.448)}
\gppoint{gp mark 1}{(2.850,1.452)}
\gppoint{gp mark 1}{(2.853,1.455)}
\gppoint{gp mark 1}{(2.855,1.443)}
\gppoint{gp mark 1}{(2.857,1.462)}
\gppoint{gp mark 1}{(2.859,1.454)}
\gppoint{gp mark 1}{(2.862,1.453)}
\gppoint{gp mark 1}{(2.864,1.448)}
\gppoint{gp mark 1}{(2.866,1.473)}
\gppoint{gp mark 1}{(2.868,1.450)}
\gppoint{gp mark 1}{(2.871,1.440)}
\gppoint{gp mark 1}{(2.873,1.435)}
\gppoint{gp mark 1}{(2.875,1.446)}
\gppoint{gp mark 1}{(2.877,1.455)}
\gppoint{gp mark 1}{(2.880,1.445)}
\gppoint{gp mark 1}{(2.882,1.436)}
\gppoint{gp mark 1}{(2.884,1.461)}
\gppoint{gp mark 1}{(2.886,1.467)}
\gppoint{gp mark 1}{(2.889,1.461)}
\gppoint{gp mark 1}{(2.891,1.466)}
\gppoint{gp mark 1}{(2.893,1.462)}
\gppoint{gp mark 1}{(2.895,1.455)}
\gppoint{gp mark 1}{(2.898,1.446)}
\gppoint{gp mark 1}{(2.900,1.458)}
\gppoint{gp mark 1}{(2.902,1.451)}
\gppoint{gp mark 1}{(2.904,1.463)}
\gppoint{gp mark 1}{(2.907,1.470)}
\gppoint{gp mark 1}{(2.909,1.455)}
\gppoint{gp mark 1}{(2.911,1.450)}
\gppoint{gp mark 1}{(2.913,1.446)}
\gppoint{gp mark 1}{(2.916,1.458)}
\gppoint{gp mark 1}{(2.918,1.466)}
\gppoint{gp mark 1}{(2.920,1.463)}
\gppoint{gp mark 1}{(2.922,1.463)}
\gppoint{gp mark 1}{(2.925,1.451)}
\gppoint{gp mark 1}{(2.927,1.469)}
\gppoint{gp mark 1}{(2.929,1.467)}
\gppoint{gp mark 1}{(2.931,1.447)}
\gppoint{gp mark 1}{(2.934,1.453)}
\gppoint{gp mark 1}{(2.936,1.456)}
\gppoint{gp mark 1}{(2.938,1.465)}
\gppoint{gp mark 1}{(2.940,1.457)}
\gppoint{gp mark 1}{(2.943,1.461)}
\gppoint{gp mark 1}{(2.945,1.458)}
\gppoint{gp mark 1}{(2.947,1.459)}
\gppoint{gp mark 1}{(2.949,1.459)}
\gppoint{gp mark 1}{(2.952,1.460)}
\gppoint{gp mark 1}{(2.954,1.440)}
\gppoint{gp mark 1}{(2.956,1.460)}
\gppoint{gp mark 1}{(2.958,1.448)}
\gppoint{gp mark 1}{(2.961,1.450)}
\gppoint{gp mark 1}{(2.963,1.441)}
\gppoint{gp mark 1}{(2.965,1.454)}
\gppoint{gp mark 1}{(2.967,1.459)}
\gppoint{gp mark 1}{(2.970,1.438)}
\gppoint{gp mark 1}{(2.972,1.463)}
\gppoint{gp mark 1}{(2.974,1.469)}
\gppoint{gp mark 1}{(2.976,1.454)}
\gppoint{gp mark 1}{(2.979,1.445)}
\gppoint{gp mark 1}{(2.981,1.467)}
\gppoint{gp mark 1}{(2.983,1.465)}
\gppoint{gp mark 1}{(2.985,1.456)}
\gppoint{gp mark 1}{(2.988,1.461)}
\gppoint{gp mark 1}{(2.990,1.449)}
\gppoint{gp mark 1}{(2.992,1.462)}
\gppoint{gp mark 1}{(2.994,1.463)}
\gppoint{gp mark 1}{(2.997,1.454)}
\gppoint{gp mark 1}{(2.999,1.457)}
\gppoint{gp mark 1}{(3.001,1.471)}
\gppoint{gp mark 1}{(3.003,1.447)}
\gppoint{gp mark 1}{(3.006,1.459)}
\gppoint{gp mark 1}{(3.008,1.471)}
\gppoint{gp mark 1}{(3.010,1.451)}
\gppoint{gp mark 1}{(3.012,1.458)}
\gppoint{gp mark 1}{(3.015,1.451)}
\gppoint{gp mark 1}{(3.017,1.472)}
\gppoint{gp mark 1}{(3.019,1.463)}
\gppoint{gp mark 1}{(3.021,1.455)}
\gppoint{gp mark 1}{(3.024,1.476)}
\gppoint{gp mark 1}{(3.026,1.447)}
\gppoint{gp mark 1}{(3.028,1.459)}
\gppoint{gp mark 1}{(3.030,1.446)}
\gppoint{gp mark 1}{(3.033,1.462)}
\gppoint{gp mark 1}{(3.035,1.462)}
\gppoint{gp mark 1}{(3.037,1.467)}
\gppoint{gp mark 1}{(3.039,1.451)}
\gppoint{gp mark 1}{(3.042,1.450)}
\gppoint{gp mark 1}{(3.044,1.457)}
\gppoint{gp mark 1}{(3.046,1.460)}
\gppoint{gp mark 1}{(3.048,1.459)}
\gppoint{gp mark 1}{(3.051,1.455)}
\gppoint{gp mark 1}{(3.053,1.471)}
\gppoint{gp mark 1}{(3.055,1.458)}
\gppoint{gp mark 1}{(3.057,1.469)}
\gppoint{gp mark 1}{(3.060,1.470)}
\gppoint{gp mark 1}{(3.062,1.465)}
\gppoint{gp mark 1}{(3.064,1.464)}
\gppoint{gp mark 1}{(3.066,1.453)}
\gppoint{gp mark 1}{(3.069,1.442)}
\gppoint{gp mark 1}{(3.071,1.451)}
\gppoint{gp mark 1}{(3.073,1.461)}
\gppoint{gp mark 1}{(3.075,1.459)}
\gppoint{gp mark 1}{(3.078,1.445)}
\gppoint{gp mark 1}{(3.080,1.456)}
\gppoint{gp mark 1}{(3.082,1.454)}
\gppoint{gp mark 1}{(3.084,1.453)}
\gppoint{gp mark 1}{(3.087,1.447)}
\gppoint{gp mark 1}{(3.089,1.439)}
\gppoint{gp mark 1}{(3.091,1.457)}
\gppoint{gp mark 1}{(3.093,1.432)}
\gppoint{gp mark 1}{(3.096,1.467)}
\gppoint{gp mark 1}{(3.098,1.438)}
\gppoint{gp mark 1}{(3.100,1.464)}
\gppoint{gp mark 1}{(3.102,1.460)}
\gppoint{gp mark 1}{(3.105,1.454)}
\gppoint{gp mark 1}{(3.107,1.448)}
\gppoint{gp mark 1}{(3.109,1.449)}
\gppoint{gp mark 1}{(3.111,1.456)}
\gppoint{gp mark 1}{(3.114,1.469)}
\gppoint{gp mark 1}{(3.116,1.451)}
\gppoint{gp mark 1}{(3.118,1.445)}
\gppoint{gp mark 1}{(3.120,1.440)}
\gppoint{gp mark 1}{(3.123,1.469)}
\gppoint{gp mark 1}{(3.125,1.464)}
\gppoint{gp mark 1}{(3.127,1.436)}
\gppoint{gp mark 1}{(3.129,1.455)}
\gppoint{gp mark 1}{(3.132,1.427)}
\gppoint{gp mark 1}{(3.134,1.438)}
\gppoint{gp mark 1}{(3.136,1.461)}
\gppoint{gp mark 1}{(3.138,1.451)}
\gppoint{gp mark 1}{(3.141,1.459)}
\gppoint{gp mark 1}{(3.143,1.451)}
\gppoint{gp mark 1}{(3.145,1.454)}
\gppoint{gp mark 1}{(3.147,1.457)}
\gppoint{gp mark 1}{(3.150,1.450)}
\gppoint{gp mark 1}{(3.152,1.458)}
\gppoint{gp mark 1}{(3.154,1.448)}
\gppoint{gp mark 1}{(3.156,1.443)}
\gppoint{gp mark 1}{(3.159,1.443)}
\gppoint{gp mark 1}{(3.161,1.444)}
\gppoint{gp mark 1}{(3.163,1.454)}
\gppoint{gp mark 1}{(3.165,1.450)}
\gppoint{gp mark 1}{(3.168,1.446)}
\gppoint{gp mark 1}{(3.170,1.454)}
\gppoint{gp mark 1}{(3.172,1.435)}
\gppoint{gp mark 1}{(3.174,1.445)}
\gppoint{gp mark 1}{(3.177,1.464)}
\gppoint{gp mark 1}{(3.179,1.451)}
\gppoint{gp mark 1}{(3.181,1.447)}
\gppoint{gp mark 1}{(3.183,1.433)}
\gppoint{gp mark 1}{(3.186,1.448)}
\gppoint{gp mark 1}{(3.188,1.437)}
\gppoint{gp mark 1}{(3.190,1.464)}
\gppoint{gp mark 1}{(3.192,1.435)}
\gppoint{gp mark 1}{(3.195,1.456)}
\gppoint{gp mark 1}{(3.197,1.448)}
\gppoint{gp mark 1}{(3.199,1.447)}
\gppoint{gp mark 1}{(3.201,1.447)}
\gppoint{gp mark 1}{(3.204,1.438)}
\gppoint{gp mark 1}{(3.206,1.427)}
\gppoint{gp mark 1}{(3.208,1.435)}
\gppoint{gp mark 1}{(3.210,1.428)}
\gppoint{gp mark 1}{(3.213,1.453)}
\gppoint{gp mark 1}{(3.215,1.472)}
\gppoint{gp mark 1}{(3.217,1.432)}
\gppoint{gp mark 1}{(3.219,1.450)}
\gppoint{gp mark 1}{(3.222,1.439)}
\gppoint{gp mark 1}{(3.224,1.440)}
\gppoint{gp mark 1}{(3.226,1.442)}
\gppoint{gp mark 1}{(3.228,1.440)}
\gppoint{gp mark 1}{(3.231,1.435)}
\gppoint{gp mark 1}{(3.233,1.452)}
\gppoint{gp mark 1}{(3.235,1.424)}
\gppoint{gp mark 1}{(3.237,1.438)}
\gppoint{gp mark 1}{(3.240,1.430)}
\gppoint{gp mark 1}{(3.242,1.440)}
\gppoint{gp mark 1}{(3.244,1.446)}
\gppoint{gp mark 1}{(3.246,1.469)}
\gppoint{gp mark 1}{(3.249,1.437)}
\gppoint{gp mark 1}{(3.251,1.450)}
\gppoint{gp mark 1}{(3.253,1.440)}
\gppoint{gp mark 1}{(3.255,1.453)}
\gppoint{gp mark 1}{(3.258,1.444)}
\gppoint{gp mark 1}{(3.260,1.436)}
\gppoint{gp mark 1}{(3.262,1.438)}
\gppoint{gp mark 1}{(3.264,1.457)}
\gppoint{gp mark 1}{(3.267,1.436)}
\gppoint{gp mark 1}{(3.269,1.442)}
\gppoint{gp mark 1}{(3.271,1.439)}
\gppoint{gp mark 1}{(3.273,1.448)}
\gppoint{gp mark 1}{(3.276,1.429)}
\gppoint{gp mark 1}{(3.278,1.438)}
\gppoint{gp mark 1}{(3.280,1.435)}
\gppoint{gp mark 1}{(3.282,1.438)}
\gppoint{gp mark 1}{(3.285,1.448)}
\gppoint{gp mark 1}{(3.287,1.453)}
\gppoint{gp mark 1}{(3.289,1.452)}
\gppoint{gp mark 1}{(3.291,1.439)}
\gppoint{gp mark 1}{(3.294,1.441)}
\gppoint{gp mark 1}{(3.296,1.424)}
\gppoint{gp mark 1}{(3.298,1.433)}
\gppoint{gp mark 1}{(3.300,1.447)}
\gppoint{gp mark 1}{(3.303,1.438)}
\gppoint{gp mark 1}{(3.305,1.445)}
\gppoint{gp mark 1}{(3.307,1.441)}
\gppoint{gp mark 1}{(3.309,1.440)}
\gppoint{gp mark 1}{(3.312,1.433)}
\gppoint{gp mark 1}{(3.314,1.424)}
\gppoint{gp mark 1}{(3.316,1.426)}
\gppoint{gp mark 1}{(3.318,1.422)}
\gppoint{gp mark 1}{(3.321,1.442)}
\gppoint{gp mark 1}{(3.323,1.455)}
\gppoint{gp mark 1}{(3.325,1.451)}
\gppoint{gp mark 1}{(3.327,1.440)}
\gppoint{gp mark 1}{(3.330,1.439)}
\gppoint{gp mark 1}{(3.332,1.440)}
\gppoint{gp mark 1}{(3.334,1.442)}
\gppoint{gp mark 1}{(3.336,1.443)}
\gppoint{gp mark 1}{(3.339,1.448)}
\gppoint{gp mark 1}{(3.341,1.444)}
\gppoint{gp mark 1}{(3.343,1.419)}
\gppoint{gp mark 1}{(3.345,1.445)}
\gppoint{gp mark 1}{(3.348,1.431)}
\gppoint{gp mark 1}{(3.350,1.432)}
\gppoint{gp mark 1}{(3.352,1.442)}
\gppoint{gp mark 1}{(3.354,1.436)}
\gppoint{gp mark 1}{(3.357,1.441)}
\gppoint{gp mark 1}{(3.359,1.444)}
\gppoint{gp mark 1}{(3.361,1.422)}
\gppoint{gp mark 1}{(3.363,1.455)}
\gppoint{gp mark 1}{(3.366,1.455)}
\gppoint{gp mark 1}{(3.368,1.429)}
\gppoint{gp mark 1}{(3.370,1.448)}
\gppoint{gp mark 1}{(3.372,1.452)}
\gppoint{gp mark 1}{(3.375,1.436)}
\gppoint{gp mark 1}{(3.377,1.426)}
\gppoint{gp mark 1}{(3.379,1.425)}
\gppoint{gp mark 1}{(3.381,1.439)}
\gppoint{gp mark 1}{(3.384,1.425)}
\gppoint{gp mark 1}{(3.386,1.431)}
\gppoint{gp mark 1}{(3.388,1.434)}
\gppoint{gp mark 1}{(3.390,1.430)}
\gppoint{gp mark 1}{(3.393,1.434)}
\gppoint{gp mark 1}{(3.395,1.427)}
\gppoint{gp mark 1}{(3.397,1.421)}
\gppoint{gp mark 1}{(3.399,1.431)}
\gppoint{gp mark 1}{(3.402,1.426)}
\gppoint{gp mark 1}{(3.404,1.415)}
\gppoint{gp mark 1}{(3.406,1.430)}
\gppoint{gp mark 1}{(3.408,1.452)}
\gppoint{gp mark 1}{(3.411,1.436)}
\gppoint{gp mark 1}{(3.413,1.446)}
\gppoint{gp mark 1}{(3.415,1.435)}
\gppoint{gp mark 1}{(3.417,1.415)}
\gppoint{gp mark 1}{(3.420,1.452)}
\gppoint{gp mark 1}{(3.422,1.436)}
\gppoint{gp mark 1}{(3.424,1.433)}
\gppoint{gp mark 1}{(3.426,1.398)}
\gppoint{gp mark 1}{(3.429,1.428)}
\gppoint{gp mark 1}{(3.431,1.415)}
\gppoint{gp mark 1}{(3.433,1.414)}
\gppoint{gp mark 1}{(3.435,1.422)}
\gppoint{gp mark 1}{(3.438,1.429)}
\gppoint{gp mark 1}{(3.440,1.437)}
\gppoint{gp mark 1}{(3.442,1.412)}
\gppoint{gp mark 1}{(3.444,1.416)}
\gppoint{gp mark 1}{(3.447,1.417)}
\gppoint{gp mark 1}{(3.449,1.427)}
\gppoint{gp mark 1}{(3.451,1.439)}
\gppoint{gp mark 1}{(3.453,1.439)}
\gppoint{gp mark 1}{(3.456,1.418)}
\gppoint{gp mark 1}{(3.458,1.431)}
\gppoint{gp mark 1}{(3.460,1.402)}
\gppoint{gp mark 1}{(3.462,1.417)}
\gppoint{gp mark 1}{(3.465,1.439)}
\gppoint{gp mark 1}{(3.467,1.428)}
\gppoint{gp mark 1}{(3.469,1.432)}
\gppoint{gp mark 1}{(3.471,1.409)}
\gppoint{gp mark 1}{(3.474,1.413)}
\gppoint{gp mark 1}{(3.476,1.434)}
\gppoint{gp mark 1}{(3.478,1.434)}
\gppoint{gp mark 1}{(3.480,1.417)}
\gppoint{gp mark 1}{(3.483,1.420)}
\gppoint{gp mark 1}{(3.485,1.405)}
\gppoint{gp mark 1}{(3.487,1.403)}
\gppoint{gp mark 1}{(3.489,1.429)}
\gppoint{gp mark 1}{(3.492,1.424)}
\gppoint{gp mark 1}{(3.494,1.412)}
\gppoint{gp mark 1}{(3.496,1.427)}
\gppoint{gp mark 1}{(3.498,1.419)}
\gppoint{gp mark 1}{(3.501,1.429)}
\gppoint{gp mark 1}{(3.503,1.400)}
\gppoint{gp mark 1}{(3.505,1.415)}
\gppoint{gp mark 1}{(3.507,1.402)}
\gppoint{gp mark 1}{(3.510,1.426)}
\gppoint{gp mark 1}{(3.512,1.400)}
\gppoint{gp mark 1}{(3.514,1.427)}
\gppoint{gp mark 1}{(3.516,1.421)}
\gppoint{gp mark 1}{(3.519,1.409)}
\gppoint{gp mark 1}{(3.521,1.431)}
\gppoint{gp mark 1}{(3.523,1.413)}
\gppoint{gp mark 1}{(3.525,1.413)}
\gppoint{gp mark 1}{(3.528,1.422)}
\gppoint{gp mark 1}{(3.530,1.423)}
\gppoint{gp mark 1}{(3.532,1.403)}
\gppoint{gp mark 1}{(3.534,1.417)}
\gppoint{gp mark 1}{(3.537,1.458)}
\gppoint{gp mark 1}{(3.539,1.402)}
\gppoint{gp mark 1}{(3.541,1.420)}
\gppoint{gp mark 1}{(3.543,1.431)}
\gppoint{gp mark 1}{(3.546,1.410)}
\gppoint{gp mark 1}{(3.548,1.420)}
\gppoint{gp mark 1}{(3.550,1.395)}
\gppoint{gp mark 1}{(3.552,1.416)}
\gppoint{gp mark 1}{(3.555,1.409)}
\gppoint{gp mark 1}{(3.557,1.415)}
\gppoint{gp mark 1}{(3.559,1.425)}
\gppoint{gp mark 1}{(3.561,1.417)}
\gppoint{gp mark 1}{(3.564,1.435)}
\gppoint{gp mark 1}{(3.566,1.438)}
\gppoint{gp mark 1}{(3.568,1.419)}
\gppoint{gp mark 1}{(3.570,1.413)}
\gppoint{gp mark 1}{(3.573,1.410)}
\gppoint{gp mark 1}{(3.575,1.436)}
\gppoint{gp mark 1}{(3.577,1.408)}
\gppoint{gp mark 1}{(3.579,1.424)}
\gppoint{gp mark 1}{(3.582,1.416)}
\gppoint{gp mark 1}{(3.584,1.431)}
\gppoint{gp mark 1}{(3.586,1.442)}
\gppoint{gp mark 1}{(3.588,1.414)}
\gppoint{gp mark 1}{(3.591,1.423)}
\gppoint{gp mark 1}{(3.593,1.389)}
\gppoint{gp mark 1}{(3.595,1.413)}
\gppoint{gp mark 1}{(3.597,1.411)}
\gppoint{gp mark 1}{(3.600,1.419)}
\gppoint{gp mark 1}{(3.602,1.407)}
\gppoint{gp mark 1}{(3.604,1.409)}
\gppoint{gp mark 1}{(3.606,1.392)}
\gppoint{gp mark 1}{(3.609,1.409)}
\gppoint{gp mark 1}{(3.611,1.411)}
\gppoint{gp mark 1}{(3.613,1.405)}
\gppoint{gp mark 1}{(3.615,1.426)}
\gppoint{gp mark 1}{(3.618,1.417)}
\gppoint{gp mark 1}{(3.620,1.416)}
\gppoint{gp mark 1}{(3.622,1.420)}
\gppoint{gp mark 1}{(3.624,1.430)}
\gppoint{gp mark 1}{(3.627,1.422)}
\gppoint{gp mark 1}{(3.629,1.410)}
\gppoint{gp mark 1}{(3.631,1.405)}
\gppoint{gp mark 1}{(3.633,1.389)}
\gppoint{gp mark 1}{(3.636,1.413)}
\gppoint{gp mark 1}{(3.638,1.400)}
\gppoint{gp mark 1}{(3.640,1.427)}
\gppoint{gp mark 1}{(3.642,1.430)}
\gppoint{gp mark 1}{(3.645,1.441)}
\gppoint{gp mark 1}{(3.647,1.413)}
\gppoint{gp mark 1}{(3.649,1.418)}
\gppoint{gp mark 1}{(3.651,1.418)}
\gppoint{gp mark 1}{(3.654,1.405)}
\gppoint{gp mark 1}{(3.656,1.408)}
\gppoint{gp mark 1}{(3.658,1.436)}
\gppoint{gp mark 1}{(3.660,1.403)}
\gppoint{gp mark 1}{(3.663,1.410)}
\gppoint{gp mark 1}{(3.665,1.437)}
\gppoint{gp mark 1}{(3.667,1.407)}
\gppoint{gp mark 1}{(3.669,1.415)}
\gppoint{gp mark 1}{(3.672,1.389)}
\gppoint{gp mark 1}{(3.674,1.394)}
\gppoint{gp mark 1}{(3.676,1.398)}
\gppoint{gp mark 1}{(3.678,1.416)}
\gppoint{gp mark 1}{(3.681,1.422)}
\gppoint{gp mark 1}{(3.683,1.400)}
\gppoint{gp mark 1}{(3.685,1.430)}
\gppoint{gp mark 1}{(3.687,1.411)}
\gppoint{gp mark 1}{(3.690,1.426)}
\gppoint{gp mark 1}{(3.692,1.408)}
\gppoint{gp mark 1}{(3.694,1.420)}
\gppoint{gp mark 1}{(3.696,1.415)}
\gppoint{gp mark 1}{(3.699,1.413)}
\gppoint{gp mark 1}{(3.701,1.401)}
\gppoint{gp mark 1}{(3.703,1.396)}
\gppoint{gp mark 1}{(3.705,1.418)}
\gppoint{gp mark 1}{(3.708,1.413)}
\gppoint{gp mark 1}{(3.710,1.405)}
\gppoint{gp mark 1}{(3.712,1.413)}
\gppoint{gp mark 1}{(3.714,1.420)}
\gppoint{gp mark 1}{(3.717,1.413)}
\gppoint{gp mark 1}{(3.719,1.404)}
\gppoint{gp mark 1}{(3.721,1.406)}
\gppoint{gp mark 1}{(3.723,1.417)}
\gppoint{gp mark 1}{(3.726,1.420)}
\gppoint{gp mark 1}{(3.728,1.427)}
\gppoint{gp mark 1}{(3.730,1.423)}
\gppoint{gp mark 1}{(3.732,1.414)}
\gppoint{gp mark 1}{(3.735,1.423)}
\gppoint{gp mark 1}{(3.737,1.423)}
\gppoint{gp mark 1}{(3.739,1.415)}
\gppoint{gp mark 1}{(3.741,1.399)}
\gppoint{gp mark 1}{(3.744,1.411)}
\gppoint{gp mark 1}{(3.746,1.424)}
\gppoint{gp mark 1}{(3.748,1.409)}
\gppoint{gp mark 1}{(3.750,1.405)}
\gppoint{gp mark 1}{(3.753,1.405)}
\gppoint{gp mark 1}{(3.755,1.413)}
\gppoint{gp mark 1}{(3.757,1.427)}
\gppoint{gp mark 1}{(3.759,1.398)}
\gppoint{gp mark 1}{(3.762,1.412)}
\gppoint{gp mark 1}{(3.764,1.411)}
\gppoint{gp mark 1}{(3.766,1.412)}
\gppoint{gp mark 1}{(3.768,1.411)}
\gppoint{gp mark 1}{(3.771,1.400)}
\gppoint{gp mark 1}{(3.773,1.406)}
\gppoint{gp mark 1}{(3.775,1.405)}
\gppoint{gp mark 1}{(3.777,1.406)}
\gppoint{gp mark 1}{(3.780,1.398)}
\gppoint{gp mark 1}{(3.782,1.400)}
\gppoint{gp mark 1}{(3.784,1.410)}
\gppoint{gp mark 1}{(3.786,1.409)}
\gppoint{gp mark 1}{(3.789,1.411)}
\gppoint{gp mark 1}{(3.791,1.414)}
\gppoint{gp mark 1}{(3.793,1.405)}
\gppoint{gp mark 1}{(3.795,1.395)}
\gppoint{gp mark 1}{(3.798,1.388)}
\gppoint{gp mark 1}{(3.800,1.408)}
\gppoint{gp mark 1}{(3.802,1.398)}
\gppoint{gp mark 1}{(3.804,1.407)}
\gppoint{gp mark 1}{(3.807,1.419)}
\gppoint{gp mark 1}{(3.809,1.408)}
\gppoint{gp mark 1}{(3.811,1.394)}
\gppoint{gp mark 1}{(3.813,1.399)}
\gppoint{gp mark 1}{(3.816,1.404)}
\gppoint{gp mark 1}{(3.818,1.405)}
\gppoint{gp mark 1}{(3.820,1.409)}
\gppoint{gp mark 1}{(3.822,1.400)}
\gppoint{gp mark 1}{(3.825,1.393)}
\gppoint{gp mark 1}{(3.827,1.410)}
\gppoint{gp mark 1}{(3.829,1.418)}
\gppoint{gp mark 1}{(3.831,1.400)}
\gppoint{gp mark 1}{(3.834,1.403)}
\gppoint{gp mark 1}{(3.836,1.396)}
\gppoint{gp mark 1}{(3.838,1.400)}
\gppoint{gp mark 1}{(3.840,1.407)}
\gppoint{gp mark 1}{(3.843,1.388)}
\gppoint{gp mark 1}{(3.845,1.410)}
\gppoint{gp mark 1}{(3.847,1.389)}
\gppoint{gp mark 1}{(3.849,1.409)}
\gppoint{gp mark 1}{(3.852,1.410)}
\gppoint{gp mark 1}{(3.854,1.414)}
\gppoint{gp mark 1}{(3.856,1.394)}
\gppoint{gp mark 1}{(3.859,1.405)}
\gppoint{gp mark 1}{(3.861,1.400)}
\gppoint{gp mark 1}{(3.863,1.408)}
\gppoint{gp mark 1}{(3.865,1.398)}
\gppoint{gp mark 1}{(3.868,1.417)}
\gppoint{gp mark 1}{(3.870,1.420)}
\gppoint{gp mark 1}{(3.872,1.404)}
\gppoint{gp mark 1}{(3.874,1.410)}
\gppoint{gp mark 1}{(3.877,1.396)}
\gppoint{gp mark 1}{(3.879,1.402)}
\gppoint{gp mark 1}{(3.881,1.403)}
\gppoint{gp mark 1}{(3.883,1.421)}
\gppoint{gp mark 1}{(3.886,1.420)}
\gppoint{gp mark 1}{(3.888,1.397)}
\gppoint{gp mark 1}{(3.890,1.416)}
\gppoint{gp mark 1}{(3.892,1.412)}
\gppoint{gp mark 1}{(3.895,1.407)}
\gppoint{gp mark 1}{(3.897,1.405)}
\gppoint{gp mark 1}{(3.899,1.401)}
\gppoint{gp mark 1}{(3.901,1.401)}
\gppoint{gp mark 1}{(3.904,1.412)}
\gppoint{gp mark 1}{(3.906,1.420)}
\gppoint{gp mark 1}{(3.908,1.405)}
\gppoint{gp mark 1}{(3.910,1.418)}
\gppoint{gp mark 1}{(3.913,1.423)}
\gppoint{gp mark 1}{(3.915,1.423)}
\gppoint{gp mark 1}{(3.917,1.417)}
\gppoint{gp mark 1}{(3.919,1.401)}
\gppoint{gp mark 1}{(3.922,1.412)}
\gppoint{gp mark 1}{(3.924,1.417)}
\gppoint{gp mark 1}{(3.926,1.414)}
\gppoint{gp mark 1}{(3.928,1.419)}
\gppoint{gp mark 1}{(3.931,1.427)}
\gppoint{gp mark 1}{(3.933,1.400)}
\gppoint{gp mark 1}{(3.935,1.399)}
\gppoint{gp mark 1}{(3.937,1.418)}
\gppoint{gp mark 1}{(3.940,1.420)}
\gppoint{gp mark 1}{(3.942,1.422)}
\gppoint{gp mark 1}{(3.944,1.410)}
\gppoint{gp mark 1}{(3.946,1.422)}
\gppoint{gp mark 1}{(3.949,1.428)}
\gppoint{gp mark 1}{(3.951,1.399)}
\gppoint{gp mark 1}{(3.953,1.419)}
\gppoint{gp mark 1}{(3.955,1.419)}
\gppoint{gp mark 1}{(3.958,1.409)}
\gppoint{gp mark 1}{(3.960,1.402)}
\gppoint{gp mark 1}{(3.962,1.419)}
\gppoint{gp mark 1}{(3.964,1.400)}
\gppoint{gp mark 1}{(3.967,1.417)}
\gppoint{gp mark 1}{(3.969,1.410)}
\gppoint{gp mark 1}{(3.971,1.417)}
\gppoint{gp mark 1}{(3.973,1.415)}
\gppoint{gp mark 1}{(3.976,1.419)}
\gppoint{gp mark 1}{(3.978,1.424)}
\gppoint{gp mark 1}{(3.980,1.426)}
\gppoint{gp mark 1}{(3.982,1.420)}
\gppoint{gp mark 1}{(3.985,1.420)}
\gppoint{gp mark 1}{(3.987,1.417)}
\gppoint{gp mark 1}{(3.989,1.404)}
\gppoint{gp mark 1}{(3.991,1.434)}
\gppoint{gp mark 1}{(3.994,1.430)}
\gppoint{gp mark 1}{(3.996,1.425)}
\gppoint{gp mark 1}{(3.998,1.420)}
\gppoint{gp mark 1}{(4.000,1.425)}
\gppoint{gp mark 1}{(4.003,1.431)}
\gppoint{gp mark 1}{(4.005,1.434)}
\gppoint{gp mark 1}{(4.007,1.417)}
\gppoint{gp mark 1}{(4.009,1.446)}
\gppoint{gp mark 1}{(4.012,1.442)}
\gppoint{gp mark 1}{(4.014,1.414)}
\gppoint{gp mark 1}{(4.016,1.437)}
\gppoint{gp mark 1}{(4.018,1.416)}
\gppoint{gp mark 1}{(4.021,1.439)}
\gppoint{gp mark 1}{(4.023,1.456)}
\gppoint{gp mark 1}{(4.025,1.429)}
\gppoint{gp mark 1}{(4.027,1.427)}
\gppoint{gp mark 1}{(4.030,1.433)}
\gppoint{gp mark 1}{(4.032,1.428)}
\gppoint{gp mark 1}{(4.034,1.429)}
\gppoint{gp mark 1}{(4.036,1.432)}
\gppoint{gp mark 1}{(4.039,1.415)}
\gppoint{gp mark 1}{(4.041,1.428)}
\gppoint{gp mark 1}{(4.043,1.429)}
\gppoint{gp mark 1}{(4.045,1.426)}
\gppoint{gp mark 1}{(4.048,1.434)}
\gppoint{gp mark 1}{(4.050,1.441)}
\gppoint{gp mark 1}{(4.052,1.439)}
\gppoint{gp mark 1}{(4.054,1.427)}
\gppoint{gp mark 1}{(4.057,1.461)}
\gppoint{gp mark 1}{(4.059,1.447)}
\gppoint{gp mark 1}{(4.061,1.440)}
\gppoint{gp mark 1}{(4.063,1.447)}
\gppoint{gp mark 1}{(4.066,1.467)}
\gppoint{gp mark 1}{(4.068,1.447)}
\gppoint{gp mark 1}{(4.070,1.442)}
\gppoint{gp mark 1}{(4.072,1.446)}
\gppoint{gp mark 1}{(4.075,1.458)}
\gppoint{gp mark 1}{(4.077,1.455)}
\gppoint{gp mark 1}{(4.079,1.449)}
\gppoint{gp mark 1}{(4.081,1.439)}
\gppoint{gp mark 1}{(4.084,1.454)}
\gppoint{gp mark 1}{(4.086,1.454)}
\gppoint{gp mark 1}{(4.088,1.454)}
\gppoint{gp mark 1}{(4.090,1.447)}
\gppoint{gp mark 1}{(4.093,1.449)}
\gppoint{gp mark 1}{(4.095,1.465)}
\gppoint{gp mark 1}{(4.097,1.454)}
\gppoint{gp mark 1}{(4.099,1.465)}
\gppoint{gp mark 1}{(4.102,1.464)}
\gppoint{gp mark 1}{(4.104,1.464)}
\gppoint{gp mark 1}{(4.106,1.451)}
\gppoint{gp mark 1}{(4.108,1.451)}
\gppoint{gp mark 1}{(4.111,1.466)}
\gppoint{gp mark 1}{(4.113,1.464)}
\gppoint{gp mark 1}{(4.115,1.459)}
\gppoint{gp mark 1}{(4.117,1.482)}
\gppoint{gp mark 1}{(4.120,1.496)}
\gppoint{gp mark 1}{(4.122,1.469)}
\gppoint{gp mark 1}{(4.124,1.464)}
\gppoint{gp mark 1}{(4.126,1.467)}
\gppoint{gp mark 1}{(4.129,1.463)}
\gppoint{gp mark 1}{(4.131,1.482)}
\gppoint{gp mark 1}{(4.133,1.473)}
\gppoint{gp mark 1}{(4.135,1.484)}
\gppoint{gp mark 1}{(4.138,1.479)}
\gppoint{gp mark 1}{(4.140,1.485)}
\gppoint{gp mark 1}{(4.142,1.484)}
\gppoint{gp mark 1}{(4.144,1.488)}
\gppoint{gp mark 1}{(4.147,1.485)}
\gppoint{gp mark 1}{(4.149,1.476)}
\gppoint{gp mark 1}{(4.151,1.479)}
\gppoint{gp mark 1}{(4.153,1.494)}
\gppoint{gp mark 1}{(4.156,1.498)}
\gppoint{gp mark 1}{(4.158,1.469)}
\gppoint{gp mark 1}{(4.160,1.493)}
\gppoint{gp mark 1}{(4.162,1.492)}
\gppoint{gp mark 1}{(4.165,1.483)}
\gppoint{gp mark 1}{(4.167,1.494)}
\gppoint{gp mark 1}{(4.169,1.503)}
\gppoint{gp mark 1}{(4.171,1.480)}
\gppoint{gp mark 1}{(4.174,1.498)}
\gppoint{gp mark 1}{(4.176,1.485)}
\gppoint{gp mark 1}{(4.178,1.500)}
\gppoint{gp mark 1}{(4.180,1.493)}
\gppoint{gp mark 1}{(4.183,1.504)}
\gppoint{gp mark 1}{(4.185,1.533)}
\gppoint{gp mark 1}{(4.187,1.505)}
\gppoint{gp mark 1}{(4.189,1.489)}
\gppoint{gp mark 1}{(4.192,1.506)}
\gppoint{gp mark 1}{(4.194,1.512)}
\gppoint{gp mark 1}{(4.196,1.511)}
\gppoint{gp mark 1}{(4.198,1.493)}
\gppoint{gp mark 1}{(4.201,1.514)}
\gppoint{gp mark 1}{(4.203,1.519)}
\gppoint{gp mark 1}{(4.205,1.508)}
\gppoint{gp mark 1}{(4.207,1.523)}
\gppoint{gp mark 1}{(4.210,1.523)}
\gppoint{gp mark 1}{(4.212,1.507)}
\gppoint{gp mark 1}{(4.214,1.525)}
\gppoint{gp mark 1}{(4.216,1.504)}
\gppoint{gp mark 1}{(4.219,1.520)}
\gppoint{gp mark 1}{(4.221,1.545)}
\gppoint{gp mark 1}{(4.223,1.533)}
\gppoint{gp mark 1}{(4.225,1.521)}
\gppoint{gp mark 1}{(4.228,1.538)}
\gppoint{gp mark 1}{(4.230,1.541)}
\gppoint{gp mark 1}{(4.232,1.516)}
\gppoint{gp mark 1}{(4.234,1.538)}
\gppoint{gp mark 1}{(4.237,1.552)}
\gppoint{gp mark 1}{(4.239,1.546)}
\gppoint{gp mark 1}{(4.241,1.518)}
\gppoint{gp mark 1}{(4.243,1.538)}
\gppoint{gp mark 1}{(4.246,1.529)}
\gppoint{gp mark 1}{(4.248,1.543)}
\gppoint{gp mark 1}{(4.250,1.559)}
\gppoint{gp mark 1}{(4.252,1.560)}
\gppoint{gp mark 1}{(4.255,1.554)}
\gppoint{gp mark 1}{(4.257,1.564)}
\gppoint{gp mark 1}{(4.259,1.571)}
\gppoint{gp mark 1}{(4.261,1.561)}
\gppoint{gp mark 1}{(4.264,1.544)}
\gppoint{gp mark 1}{(4.266,1.566)}
\gppoint{gp mark 1}{(4.268,1.557)}
\gppoint{gp mark 1}{(4.270,1.566)}
\gppoint{gp mark 1}{(4.273,1.540)}
\gppoint{gp mark 1}{(4.275,1.544)}
\gppoint{gp mark 1}{(4.277,1.562)}
\gppoint{gp mark 1}{(4.279,1.564)}
\gppoint{gp mark 1}{(4.282,1.579)}
\gppoint{gp mark 1}{(4.284,1.552)}
\gppoint{gp mark 1}{(4.286,1.559)}
\gppoint{gp mark 1}{(4.288,1.585)}
\gppoint{gp mark 1}{(4.291,1.580)}
\gppoint{gp mark 1}{(4.293,1.587)}
\gppoint{gp mark 1}{(4.295,1.591)}
\gppoint{gp mark 1}{(4.297,1.593)}
\gppoint{gp mark 1}{(4.300,1.591)}
\gppoint{gp mark 1}{(4.302,1.611)}
\gppoint{gp mark 1}{(4.304,1.595)}
\gppoint{gp mark 1}{(4.306,1.600)}
\gppoint{gp mark 1}{(4.309,1.596)}
\gppoint{gp mark 1}{(4.311,1.580)}
\gppoint{gp mark 1}{(4.313,1.593)}
\gppoint{gp mark 1}{(4.315,1.600)}
\gppoint{gp mark 1}{(4.318,1.597)}
\gppoint{gp mark 1}{(4.320,1.607)}
\gppoint{gp mark 1}{(4.322,1.607)}
\gppoint{gp mark 1}{(4.324,1.614)}
\gppoint{gp mark 1}{(4.327,1.611)}
\gppoint{gp mark 1}{(4.329,1.632)}
\gppoint{gp mark 1}{(4.331,1.613)}
\gppoint{gp mark 1}{(4.333,1.614)}
\gppoint{gp mark 1}{(4.336,1.643)}
\gppoint{gp mark 1}{(4.338,1.615)}
\gppoint{gp mark 1}{(4.340,1.629)}
\gppoint{gp mark 1}{(4.342,1.624)}
\gppoint{gp mark 1}{(4.345,1.641)}
\gppoint{gp mark 1}{(4.347,1.643)}
\gppoint{gp mark 1}{(4.349,1.633)}
\gppoint{gp mark 1}{(4.351,1.636)}
\gppoint{gp mark 1}{(4.354,1.635)}
\gppoint{gp mark 1}{(4.356,1.647)}
\gppoint{gp mark 1}{(4.358,1.650)}
\gppoint{gp mark 1}{(4.360,1.652)}
\gppoint{gp mark 1}{(4.363,1.637)}
\gppoint{gp mark 1}{(4.365,1.632)}
\gppoint{gp mark 1}{(4.367,1.651)}
\gppoint{gp mark 1}{(4.369,1.653)}
\gppoint{gp mark 1}{(4.372,1.659)}
\gppoint{gp mark 1}{(4.374,1.665)}
\gppoint{gp mark 1}{(4.376,1.658)}
\gppoint{gp mark 1}{(4.378,1.670)}
\gppoint{gp mark 1}{(4.381,1.642)}
\gppoint{gp mark 1}{(4.383,1.655)}
\gppoint{gp mark 1}{(4.385,1.687)}
\gppoint{gp mark 1}{(4.387,1.685)}
\gppoint{gp mark 1}{(4.390,1.685)}
\gppoint{gp mark 1}{(4.392,1.689)}
\gppoint{gp mark 1}{(4.394,1.697)}
\gppoint{gp mark 1}{(4.396,1.670)}
\gppoint{gp mark 1}{(4.399,1.686)}
\gppoint{gp mark 1}{(4.401,1.696)}
\gppoint{gp mark 1}{(4.403,1.687)}
\gppoint{gp mark 1}{(4.405,1.687)}
\gppoint{gp mark 1}{(4.408,1.697)}
\gppoint{gp mark 1}{(4.410,1.696)}
\gppoint{gp mark 1}{(4.412,1.684)}
\gppoint{gp mark 1}{(4.414,1.701)}
\gppoint{gp mark 1}{(4.417,1.694)}
\gppoint{gp mark 1}{(4.419,1.701)}
\gppoint{gp mark 1}{(4.421,1.686)}
\gppoint{gp mark 1}{(4.423,1.729)}
\gppoint{gp mark 1}{(4.426,1.721)}
\gppoint{gp mark 1}{(4.428,1.719)}
\gppoint{gp mark 1}{(4.430,1.719)}
\gppoint{gp mark 1}{(4.432,1.724)}
\gppoint{gp mark 1}{(4.435,1.725)}
\gppoint{gp mark 1}{(4.437,1.720)}
\gppoint{gp mark 1}{(4.439,1.762)}
\gppoint{gp mark 1}{(4.441,1.733)}
\gppoint{gp mark 1}{(4.444,1.727)}
\gppoint{gp mark 1}{(4.446,1.755)}
\gppoint{gp mark 1}{(4.448,1.759)}
\gppoint{gp mark 1}{(4.450,1.754)}
\gppoint{gp mark 1}{(4.453,1.752)}
\gppoint{gp mark 1}{(4.455,1.757)}
\gppoint{gp mark 1}{(4.457,1.747)}
\gppoint{gp mark 1}{(4.459,1.771)}
\gppoint{gp mark 1}{(4.462,1.753)}
\gppoint{gp mark 1}{(4.464,1.773)}
\gppoint{gp mark 1}{(4.466,1.784)}
\gppoint{gp mark 1}{(4.468,1.763)}
\gppoint{gp mark 1}{(4.471,1.782)}
\gppoint{gp mark 1}{(4.473,1.767)}
\gppoint{gp mark 1}{(4.475,1.783)}
\gppoint{gp mark 1}{(4.477,1.792)}
\gppoint{gp mark 1}{(4.480,1.777)}
\gppoint{gp mark 1}{(4.482,1.780)}
\gppoint{gp mark 1}{(4.484,1.793)}
\gppoint{gp mark 1}{(4.486,1.808)}
\gppoint{gp mark 1}{(4.489,1.812)}
\gppoint{gp mark 1}{(4.491,1.814)}
\gppoint{gp mark 1}{(4.493,1.828)}
\gppoint{gp mark 1}{(4.495,1.824)}
\gppoint{gp mark 1}{(4.498,1.820)}
\gppoint{gp mark 1}{(4.500,1.817)}
\gppoint{gp mark 1}{(4.502,1.827)}
\gppoint{gp mark 1}{(4.504,1.842)}
\gppoint{gp mark 1}{(4.507,1.820)}
\gppoint{gp mark 1}{(4.509,1.835)}
\gppoint{gp mark 1}{(4.511,1.827)}
\gppoint{gp mark 1}{(4.513,1.837)}
\gppoint{gp mark 1}{(4.516,1.849)}
\gppoint{gp mark 1}{(4.518,1.862)}
\gppoint{gp mark 1}{(4.520,1.853)}
\gppoint{gp mark 1}{(4.522,1.863)}
\gppoint{gp mark 1}{(4.525,1.846)}
\gppoint{gp mark 1}{(4.527,1.848)}
\gppoint{gp mark 1}{(4.529,1.860)}
\gppoint{gp mark 1}{(4.531,1.859)}
\gppoint{gp mark 1}{(4.534,1.881)}
\gppoint{gp mark 1}{(4.536,1.878)}
\gppoint{gp mark 1}{(4.538,1.877)}
\gppoint{gp mark 1}{(4.540,1.886)}
\gppoint{gp mark 1}{(4.543,1.878)}
\gppoint{gp mark 1}{(4.545,1.872)}
\gppoint{gp mark 1}{(4.547,1.882)}
\gppoint{gp mark 1}{(4.549,1.878)}
\gppoint{gp mark 1}{(4.552,1.905)}
\gppoint{gp mark 1}{(4.554,1.902)}
\gppoint{gp mark 1}{(4.556,1.915)}
\gppoint{gp mark 1}{(4.558,1.920)}
\gppoint{gp mark 1}{(4.561,1.910)}
\gppoint{gp mark 1}{(4.563,1.920)}
\gppoint{gp mark 1}{(4.565,1.906)}
\gppoint{gp mark 1}{(4.567,1.926)}
\gppoint{gp mark 1}{(4.570,1.932)}
\gppoint{gp mark 1}{(4.572,1.918)}
\gppoint{gp mark 1}{(4.574,1.939)}
\gppoint{gp mark 1}{(4.576,1.939)}
\gppoint{gp mark 1}{(4.579,1.952)}
\gppoint{gp mark 1}{(4.581,1.936)}
\gppoint{gp mark 1}{(4.583,1.945)}
\gppoint{gp mark 1}{(4.585,1.944)}
\gppoint{gp mark 1}{(4.588,1.967)}
\gppoint{gp mark 1}{(4.590,1.977)}
\gppoint{gp mark 1}{(4.592,1.961)}
\gppoint{gp mark 1}{(4.594,1.978)}
\gppoint{gp mark 1}{(4.597,1.975)}
\gppoint{gp mark 1}{(4.599,1.985)}
\gppoint{gp mark 1}{(4.601,1.976)}
\gppoint{gp mark 1}{(4.603,1.978)}
\gppoint{gp mark 1}{(4.606,1.981)}
\gppoint{gp mark 1}{(4.608,1.971)}
\gppoint{gp mark 1}{(4.610,1.976)}
\gppoint{gp mark 1}{(4.612,1.987)}
\gppoint{gp mark 1}{(4.615,2.018)}
\gppoint{gp mark 1}{(4.617,1.986)}
\gppoint{gp mark 1}{(4.619,2.010)}
\gppoint{gp mark 1}{(4.621,2.004)}
\gppoint{gp mark 1}{(4.624,2.021)}
\gppoint{gp mark 1}{(4.626,2.027)}
\gppoint{gp mark 1}{(4.628,2.006)}
\gppoint{gp mark 1}{(4.630,2.035)}
\gppoint{gp mark 1}{(4.633,2.022)}
\gppoint{gp mark 1}{(4.635,2.021)}
\gppoint{gp mark 1}{(4.637,2.052)}
\gppoint{gp mark 1}{(4.639,2.026)}
\gppoint{gp mark 1}{(4.642,2.026)}
\gppoint{gp mark 1}{(4.644,2.048)}
\gppoint{gp mark 1}{(4.646,2.047)}
\gppoint{gp mark 1}{(4.648,2.043)}
\gppoint{gp mark 1}{(4.651,2.056)}
\gppoint{gp mark 1}{(4.653,2.060)}
\gppoint{gp mark 1}{(4.655,2.052)}
\gppoint{gp mark 1}{(4.657,2.059)}
\gppoint{gp mark 1}{(4.660,2.083)}
\gppoint{gp mark 1}{(4.662,2.074)}
\gppoint{gp mark 1}{(4.664,2.083)}
\gppoint{gp mark 1}{(4.666,2.097)}
\gppoint{gp mark 1}{(4.669,2.109)}
\gppoint{gp mark 1}{(4.671,2.091)}
\gppoint{gp mark 1}{(4.673,2.091)}
\gppoint{gp mark 1}{(4.675,2.105)}
\gppoint{gp mark 1}{(4.678,2.105)}
\gppoint{gp mark 1}{(4.680,2.124)}
\gppoint{gp mark 1}{(4.682,2.125)}
\gppoint{gp mark 1}{(4.684,2.143)}
\gppoint{gp mark 1}{(4.687,2.126)}
\gppoint{gp mark 1}{(4.689,2.136)}
\gppoint{gp mark 1}{(4.691,2.155)}
\gppoint{gp mark 1}{(4.693,2.141)}
\gppoint{gp mark 1}{(4.696,2.164)}
\gppoint{gp mark 1}{(4.698,2.170)}
\gppoint{gp mark 1}{(4.700,2.171)}
\gppoint{gp mark 1}{(4.702,2.168)}
\gppoint{gp mark 1}{(4.705,2.158)}
\gppoint{gp mark 1}{(4.707,2.184)}
\gppoint{gp mark 1}{(4.709,2.186)}
\gppoint{gp mark 1}{(4.711,2.193)}
\gppoint{gp mark 1}{(4.714,2.203)}
\gppoint{gp mark 1}{(4.716,2.187)}
\gppoint{gp mark 1}{(4.718,2.201)}
\gppoint{gp mark 1}{(4.720,2.213)}
\gppoint{gp mark 1}{(4.723,2.211)}
\gppoint{gp mark 1}{(4.725,2.213)}
\gppoint{gp mark 1}{(4.727,2.223)}
\gppoint{gp mark 1}{(4.729,2.221)}
\gppoint{gp mark 1}{(4.732,2.222)}
\gppoint{gp mark 1}{(4.734,2.225)}
\gppoint{gp mark 1}{(4.736,2.247)}
\gppoint{gp mark 1}{(4.738,2.242)}
\gppoint{gp mark 1}{(4.741,2.244)}
\gppoint{gp mark 1}{(4.743,2.248)}
\gppoint{gp mark 1}{(4.745,2.268)}
\gppoint{gp mark 1}{(4.747,2.255)}
\gppoint{gp mark 1}{(4.750,2.250)}
\gppoint{gp mark 1}{(4.752,2.261)}
\gppoint{gp mark 1}{(4.754,2.266)}
\gppoint{gp mark 1}{(4.756,2.298)}
\gppoint{gp mark 1}{(4.759,2.305)}
\gppoint{gp mark 1}{(4.761,2.301)}
\gppoint{gp mark 1}{(4.763,2.272)}
\gppoint{gp mark 1}{(4.765,2.290)}
\gppoint{gp mark 1}{(4.768,2.296)}
\gppoint{gp mark 1}{(4.770,2.315)}
\gppoint{gp mark 1}{(4.772,2.329)}
\gppoint{gp mark 1}{(4.774,2.313)}
\gppoint{gp mark 1}{(4.777,2.327)}
\gppoint{gp mark 1}{(4.779,2.317)}
\gppoint{gp mark 1}{(4.781,2.336)}
\gppoint{gp mark 1}{(4.783,2.373)}
\gppoint{gp mark 1}{(4.786,2.345)}
\gppoint{gp mark 1}{(4.788,2.364)}
\gppoint{gp mark 1}{(4.790,2.362)}
\gppoint{gp mark 1}{(4.792,2.369)}
\gppoint{gp mark 1}{(4.795,2.379)}
\gppoint{gp mark 1}{(4.797,2.378)}
\gppoint{gp mark 1}{(4.799,2.381)}
\gppoint{gp mark 1}{(4.801,2.376)}
\gppoint{gp mark 1}{(4.804,2.396)}
\gppoint{gp mark 1}{(4.806,2.394)}
\gppoint{gp mark 1}{(4.808,2.395)}
\gppoint{gp mark 1}{(4.810,2.420)}
\gppoint{gp mark 1}{(4.813,2.405)}
\gppoint{gp mark 1}{(4.815,2.425)}
\gppoint{gp mark 1}{(4.817,2.418)}
\gppoint{gp mark 1}{(4.819,2.443)}
\gppoint{gp mark 1}{(4.822,2.441)}
\gppoint{gp mark 1}{(4.824,2.440)}
\gppoint{gp mark 1}{(4.826,2.451)}
\gppoint{gp mark 1}{(4.828,2.474)}
\gppoint{gp mark 1}{(4.831,2.484)}
\gppoint{gp mark 1}{(4.833,2.487)}
\gppoint{gp mark 1}{(4.835,2.487)}
\gppoint{gp mark 1}{(4.837,2.463)}
\gppoint{gp mark 1}{(4.840,2.491)}
\gppoint{gp mark 1}{(4.842,2.477)}
\gppoint{gp mark 1}{(4.844,2.494)}
\gppoint{gp mark 1}{(4.846,2.504)}
\gppoint{gp mark 1}{(4.849,2.494)}
\gppoint{gp mark 1}{(4.851,2.497)}
\gppoint{gp mark 1}{(4.853,2.491)}
\gppoint{gp mark 1}{(4.855,2.506)}
\gppoint{gp mark 1}{(4.858,2.519)}
\gppoint{gp mark 1}{(4.860,2.533)}
\gppoint{gp mark 1}{(4.862,2.525)}
\gppoint{gp mark 1}{(4.864,2.523)}
\gppoint{gp mark 1}{(4.867,2.542)}
\gppoint{gp mark 1}{(4.869,2.546)}
\gppoint{gp mark 1}{(4.871,2.552)}
\gppoint{gp mark 1}{(4.873,2.588)}
\gppoint{gp mark 1}{(4.876,2.565)}
\gppoint{gp mark 1}{(4.878,2.581)}
\gppoint{gp mark 1}{(4.880,2.585)}
\gppoint{gp mark 1}{(4.882,2.560)}
\gppoint{gp mark 1}{(4.885,2.583)}
\gppoint{gp mark 1}{(4.887,2.592)}
\gppoint{gp mark 1}{(4.889,2.592)}
\gppoint{gp mark 1}{(4.891,2.595)}
\gppoint{gp mark 1}{(4.894,2.604)}
\gppoint{gp mark 1}{(4.896,2.612)}
\gppoint{gp mark 1}{(4.898,2.617)}
\gppoint{gp mark 1}{(4.900,2.605)}
\gppoint{gp mark 1}{(4.903,2.602)}
\gppoint{gp mark 1}{(4.905,2.629)}
\gppoint{gp mark 1}{(4.907,2.631)}
\gppoint{gp mark 1}{(4.909,2.644)}
\gppoint{gp mark 1}{(4.912,2.640)}
\gppoint{gp mark 1}{(4.914,2.645)}
\gppoint{gp mark 1}{(4.916,2.673)}
\gppoint{gp mark 1}{(4.918,2.653)}
\gppoint{gp mark 1}{(4.921,2.670)}
\gppoint{gp mark 1}{(4.923,2.667)}
\gppoint{gp mark 1}{(4.925,2.687)}
\gppoint{gp mark 1}{(4.927,2.685)}
\gppoint{gp mark 1}{(4.930,2.698)}
\gppoint{gp mark 1}{(4.932,2.680)}
\gppoint{gp mark 1}{(4.934,2.701)}
\gppoint{gp mark 1}{(4.936,2.699)}
\gppoint{gp mark 1}{(4.939,2.689)}
\gppoint{gp mark 1}{(4.941,2.704)}
\gppoint{gp mark 1}{(4.943,2.693)}
\gppoint{gp mark 1}{(4.945,2.712)}
\gppoint{gp mark 1}{(4.948,2.736)}
\gppoint{gp mark 1}{(4.950,2.732)}
\gppoint{gp mark 1}{(4.952,2.756)}
\gppoint{gp mark 1}{(4.954,2.759)}
\gppoint{gp mark 1}{(4.957,2.751)}
\gppoint{gp mark 1}{(4.959,2.759)}
\gppoint{gp mark 1}{(4.961,2.765)}
\gppoint{gp mark 1}{(4.963,2.796)}
\gppoint{gp mark 1}{(4.966,2.776)}
\gppoint{gp mark 1}{(4.968,2.792)}
\gppoint{gp mark 1}{(4.970,2.805)}
\gppoint{gp mark 1}{(4.972,2.778)}
\gppoint{gp mark 1}{(4.975,2.805)}
\gppoint{gp mark 1}{(4.977,2.792)}
\gppoint{gp mark 1}{(4.979,2.828)}
\gppoint{gp mark 1}{(4.981,2.804)}
\gppoint{gp mark 1}{(4.984,2.837)}
\gppoint{gp mark 1}{(4.986,2.818)}
\gppoint{gp mark 1}{(4.988,2.841)}
\gppoint{gp mark 1}{(4.990,2.824)}
\gppoint{gp mark 1}{(4.993,2.867)}
\gppoint{gp mark 1}{(4.995,2.862)}
\gppoint{gp mark 1}{(4.997,2.874)}
\gppoint{gp mark 1}{(4.999,2.865)}
\gppoint{gp mark 1}{(5.002,2.880)}
\gppoint{gp mark 1}{(5.004,2.881)}
\gppoint{gp mark 1}{(5.006,2.897)}
\gppoint{gp mark 1}{(5.008,2.897)}
\gppoint{gp mark 1}{(5.011,2.869)}
\gppoint{gp mark 1}{(5.013,2.881)}
\gppoint{gp mark 1}{(5.015,2.925)}
\gppoint{gp mark 1}{(5.017,2.928)}
\gppoint{gp mark 1}{(5.020,2.941)}
\gppoint{gp mark 1}{(5.022,2.929)}
\gppoint{gp mark 1}{(5.024,2.946)}
\gppoint{gp mark 1}{(5.026,2.946)}
\gppoint{gp mark 1}{(5.029,2.921)}
\gppoint{gp mark 1}{(5.031,2.953)}
\gppoint{gp mark 1}{(5.033,2.971)}
\gppoint{gp mark 1}{(5.035,2.969)}
\gppoint{gp mark 1}{(5.038,2.982)}
\gppoint{gp mark 1}{(5.040,2.989)}
\gppoint{gp mark 1}{(5.042,2.990)}
\gppoint{gp mark 1}{(5.044,2.991)}
\gppoint{gp mark 1}{(5.047,2.996)}
\gppoint{gp mark 1}{(5.049,3.019)}
\gppoint{gp mark 1}{(5.051,3.014)}
\gppoint{gp mark 1}{(5.053,3.007)}
\gppoint{gp mark 1}{(5.056,3.024)}
\gppoint{gp mark 1}{(5.058,3.030)}
\gppoint{gp mark 1}{(5.060,3.064)}
\gppoint{gp mark 1}{(5.062,3.057)}
\gppoint{gp mark 1}{(5.065,3.075)}
\gppoint{gp mark 1}{(5.067,3.060)}
\gppoint{gp mark 1}{(5.069,3.035)}
\gppoint{gp mark 1}{(5.071,3.057)}
\gppoint{gp mark 1}{(5.074,3.076)}
\gppoint{gp mark 1}{(5.076,3.076)}
\gppoint{gp mark 1}{(5.078,3.100)}
\gppoint{gp mark 1}{(5.080,3.073)}
\gppoint{gp mark 1}{(5.083,3.096)}
\gppoint{gp mark 1}{(5.085,3.092)}
\gppoint{gp mark 1}{(5.087,3.096)}
\gppoint{gp mark 1}{(5.089,3.122)}
\gppoint{gp mark 1}{(5.092,3.117)}
\gppoint{gp mark 1}{(5.094,3.127)}
\gppoint{gp mark 1}{(5.096,3.128)}
\gppoint{gp mark 1}{(5.098,3.133)}
\gppoint{gp mark 1}{(5.101,3.175)}
\gppoint{gp mark 1}{(5.103,3.142)}
\gppoint{gp mark 1}{(5.105,3.164)}
\gppoint{gp mark 1}{(5.107,3.153)}
\gppoint{gp mark 1}{(5.110,3.145)}
\gppoint{gp mark 1}{(5.112,3.172)}
\gppoint{gp mark 1}{(5.114,3.172)}
\gppoint{gp mark 1}{(5.116,3.178)}
\gppoint{gp mark 1}{(5.119,3.182)}
\gppoint{gp mark 1}{(5.121,3.205)}
\gppoint{gp mark 1}{(5.123,3.207)}
\gppoint{gp mark 1}{(5.126,3.189)}
\gppoint{gp mark 1}{(5.128,3.204)}
\gppoint{gp mark 1}{(5.130,3.230)}
\gppoint{gp mark 1}{(5.132,3.238)}
\gppoint{gp mark 1}{(5.135,3.250)}
\gppoint{gp mark 1}{(5.137,3.262)}
\gppoint{gp mark 1}{(5.139,3.273)}
\gppoint{gp mark 1}{(5.141,3.243)}
\gppoint{gp mark 1}{(5.144,3.266)}
\gppoint{gp mark 1}{(5.146,3.278)}
\gppoint{gp mark 1}{(5.148,3.294)}
\gppoint{gp mark 1}{(5.150,3.281)}
\gppoint{gp mark 1}{(5.153,3.301)}
\gppoint{gp mark 1}{(5.155,3.320)}
\gppoint{gp mark 1}{(5.157,3.313)}
\gppoint{gp mark 1}{(5.159,3.334)}
\gppoint{gp mark 1}{(5.162,3.325)}
\gppoint{gp mark 1}{(5.164,3.321)}
\gppoint{gp mark 1}{(5.166,3.328)}
\gppoint{gp mark 1}{(5.168,3.359)}
\gppoint{gp mark 1}{(5.171,3.349)}
\gppoint{gp mark 1}{(5.173,3.370)}
\gppoint{gp mark 1}{(5.175,3.391)}
\gppoint{gp mark 1}{(5.177,3.375)}
\gppoint{gp mark 1}{(5.180,3.388)}
\gppoint{gp mark 1}{(5.182,3.372)}
\gppoint{gp mark 1}{(5.184,3.395)}
\gppoint{gp mark 1}{(5.186,3.390)}
\gppoint{gp mark 1}{(5.189,3.405)}
\gppoint{gp mark 1}{(5.191,3.387)}
\gppoint{gp mark 1}{(5.193,3.413)}
\gppoint{gp mark 1}{(5.195,3.415)}
\gppoint{gp mark 1}{(5.198,3.429)}
\gppoint{gp mark 1}{(5.200,3.426)}
\gppoint{gp mark 1}{(5.202,3.443)}
\gppoint{gp mark 1}{(5.204,3.436)}
\gppoint{gp mark 1}{(5.207,3.449)}
\gppoint{gp mark 1}{(5.209,3.451)}
\gppoint{gp mark 1}{(5.211,3.472)}
\gppoint{gp mark 1}{(5.213,3.471)}
\gppoint{gp mark 1}{(5.216,3.494)}
\gppoint{gp mark 1}{(5.218,3.498)}
\gppoint{gp mark 1}{(5.220,3.506)}
\gppoint{gp mark 1}{(5.222,3.501)}
\gppoint{gp mark 1}{(5.225,3.509)}
\gppoint{gp mark 1}{(5.227,3.514)}
\gppoint{gp mark 1}{(5.229,3.539)}
\gppoint{gp mark 1}{(5.231,3.520)}
\gppoint{gp mark 1}{(5.234,3.519)}
\gppoint{gp mark 1}{(5.236,3.540)}
\gppoint{gp mark 1}{(5.238,3.520)}
\gppoint{gp mark 1}{(5.240,3.544)}
\gppoint{gp mark 1}{(5.243,3.567)}
\gppoint{gp mark 1}{(5.245,3.599)}
\gppoint{gp mark 1}{(5.247,3.575)}
\gppoint{gp mark 1}{(5.249,3.583)}
\gppoint{gp mark 1}{(5.252,3.571)}
\gppoint{gp mark 1}{(5.254,3.580)}
\gppoint{gp mark 1}{(5.256,3.598)}
\gppoint{gp mark 1}{(5.258,3.586)}
\gppoint{gp mark 1}{(5.261,3.614)}
\gppoint{gp mark 1}{(5.263,3.596)}
\gppoint{gp mark 1}{(5.265,3.609)}
\gppoint{gp mark 1}{(5.267,3.634)}
\gppoint{gp mark 1}{(5.270,3.634)}
\gppoint{gp mark 1}{(5.272,3.664)}
\gppoint{gp mark 1}{(5.274,3.642)}
\gppoint{gp mark 1}{(5.276,3.639)}
\gppoint{gp mark 1}{(5.279,3.635)}
\gppoint{gp mark 1}{(5.281,3.648)}
\gppoint{gp mark 1}{(5.283,3.672)}
\gppoint{gp mark 1}{(5.285,3.673)}
\gppoint{gp mark 1}{(5.288,3.672)}
\gppoint{gp mark 1}{(5.290,3.675)}
\gppoint{gp mark 1}{(5.292,3.709)}
\gppoint{gp mark 1}{(5.294,3.699)}
\gppoint{gp mark 1}{(5.297,3.694)}
\gppoint{gp mark 1}{(5.299,3.726)}
\gppoint{gp mark 1}{(5.301,3.710)}
\gppoint{gp mark 1}{(5.303,3.703)}
\gppoint{gp mark 1}{(5.306,3.721)}
\gppoint{gp mark 1}{(5.308,3.726)}
\gppoint{gp mark 1}{(5.310,3.725)}
\gppoint{gp mark 1}{(5.312,3.743)}
\gppoint{gp mark 1}{(5.315,3.753)}
\gppoint{gp mark 1}{(5.317,3.743)}
\gppoint{gp mark 1}{(5.319,3.765)}
\gppoint{gp mark 1}{(5.321,3.791)}
\gppoint{gp mark 1}{(5.324,3.787)}
\gppoint{gp mark 1}{(5.326,3.783)}
\gppoint{gp mark 1}{(5.328,3.801)}
\gppoint{gp mark 1}{(5.330,3.807)}
\gppoint{gp mark 1}{(5.333,3.809)}
\gppoint{gp mark 1}{(5.335,3.835)}
\gppoint{gp mark 1}{(5.337,3.832)}
\gppoint{gp mark 1}{(5.339,3.831)}
\gppoint{gp mark 1}{(5.342,3.856)}
\gppoint{gp mark 1}{(5.344,3.844)}
\gppoint{gp mark 1}{(5.346,3.844)}
\gppoint{gp mark 1}{(5.348,3.859)}
\gppoint{gp mark 1}{(5.351,3.859)}
\gppoint{gp mark 1}{(5.353,3.853)}
\gppoint{gp mark 1}{(5.355,3.875)}
\gppoint{gp mark 1}{(5.357,3.862)}
\gppoint{gp mark 1}{(5.360,3.885)}
\gppoint{gp mark 1}{(5.362,3.912)}
\gppoint{gp mark 1}{(5.364,3.864)}
\gppoint{gp mark 1}{(5.366,3.889)}
\gppoint{gp mark 1}{(5.369,3.905)}
\gppoint{gp mark 1}{(5.371,3.903)}
\gppoint{gp mark 1}{(5.373,3.892)}
\gppoint{gp mark 1}{(5.375,3.900)}
\gppoint{gp mark 1}{(5.378,3.922)}
\gppoint{gp mark 1}{(5.380,3.928)}
\gppoint{gp mark 1}{(5.382,3.940)}
\gppoint{gp mark 1}{(5.384,3.951)}
\gppoint{gp mark 1}{(5.387,3.936)}
\gppoint{gp mark 1}{(5.389,3.929)}
\gppoint{gp mark 1}{(5.391,3.925)}
\gppoint{gp mark 1}{(5.393,3.992)}
\gppoint{gp mark 1}{(5.396,3.974)}
\gppoint{gp mark 1}{(5.398,3.988)}
\gppoint{gp mark 1}{(5.400,4.018)}
\gppoint{gp mark 1}{(5.402,3.996)}
\gppoint{gp mark 1}{(5.405,4.001)}
\gppoint{gp mark 1}{(5.407,3.998)}
\gppoint{gp mark 1}{(5.409,4.019)}
\gppoint{gp mark 1}{(5.411,4.028)}
\gppoint{gp mark 1}{(5.414,4.014)}
\gppoint{gp mark 1}{(5.416,4.017)}
\gppoint{gp mark 1}{(5.418,4.050)}
\gppoint{gp mark 1}{(5.420,4.058)}
\gppoint{gp mark 1}{(5.423,4.050)}
\gppoint{gp mark 1}{(5.425,4.062)}
\gppoint{gp mark 1}{(5.427,4.082)}
\gppoint{gp mark 1}{(5.429,4.062)}
\gppoint{gp mark 1}{(5.432,4.085)}
\gppoint{gp mark 1}{(5.434,4.119)}
\gppoint{gp mark 1}{(5.436,4.095)}
\gppoint{gp mark 1}{(5.438,4.101)}
\gppoint{gp mark 1}{(5.441,4.109)}
\gppoint{gp mark 1}{(5.443,4.097)}
\gppoint{gp mark 1}{(5.445,4.137)}
\gppoint{gp mark 1}{(5.447,4.135)}
\gppoint{gp mark 1}{(5.450,4.135)}
\gppoint{gp mark 1}{(5.452,4.156)}
\gppoint{gp mark 1}{(5.454,4.137)}
\gppoint{gp mark 1}{(5.456,4.162)}
\gppoint{gp mark 1}{(5.459,4.139)}
\gppoint{gp mark 1}{(5.461,4.159)}
\gppoint{gp mark 1}{(5.463,4.183)}
\gppoint{gp mark 1}{(5.465,4.169)}
\gppoint{gp mark 1}{(5.468,4.189)}
\gppoint{gp mark 1}{(5.470,4.195)}
\gppoint{gp mark 1}{(5.472,4.214)}
\gppoint{gp mark 1}{(5.474,4.199)}
\gppoint{gp mark 1}{(5.477,4.218)}
\gppoint{gp mark 1}{(5.479,4.201)}
\gppoint{gp mark 1}{(5.481,4.218)}
\gppoint{gp mark 1}{(5.483,4.214)}
\gppoint{gp mark 1}{(5.486,4.219)}
\gppoint{gp mark 1}{(5.488,4.223)}
\gppoint{gp mark 1}{(5.490,4.246)}
\gppoint{gp mark 1}{(5.492,4.234)}
\gppoint{gp mark 1}{(5.495,4.258)}
\gppoint{gp mark 1}{(5.497,4.230)}
\gppoint{gp mark 1}{(5.499,4.280)}
\gppoint{gp mark 1}{(5.501,4.279)}
\gppoint{gp mark 1}{(5.504,4.284)}
\gppoint{gp mark 1}{(5.506,4.269)}
\gppoint{gp mark 1}{(5.508,4.303)}
\gppoint{gp mark 1}{(5.510,4.312)}
\gppoint{gp mark 1}{(5.513,4.336)}
\gppoint{gp mark 1}{(5.515,4.315)}
\gppoint{gp mark 1}{(5.517,4.333)}
\gppoint{gp mark 1}{(5.519,4.313)}
\gppoint{gp mark 1}{(5.522,4.318)}
\gppoint{gp mark 1}{(5.524,4.321)}
\gppoint{gp mark 1}{(5.526,4.341)}
\gppoint{gp mark 1}{(5.528,4.335)}
\gppoint{gp mark 1}{(5.531,4.333)}
\gppoint{gp mark 1}{(5.533,4.331)}
\gppoint{gp mark 1}{(5.535,4.345)}
\gppoint{gp mark 1}{(5.537,4.350)}
\gppoint{gp mark 1}{(5.540,4.329)}
\gppoint{gp mark 1}{(5.542,4.348)}
\gppoint{gp mark 1}{(5.544,4.382)}
\gppoint{gp mark 1}{(5.546,4.362)}
\gppoint{gp mark 1}{(5.549,4.362)}
\gppoint{gp mark 1}{(5.551,4.382)}
\gppoint{gp mark 1}{(5.553,4.415)}
\gppoint{gp mark 1}{(5.555,4.402)}
\gppoint{gp mark 1}{(5.558,4.402)}
\gppoint{gp mark 1}{(5.560,4.408)}
\gppoint{gp mark 1}{(5.562,4.413)}
\gppoint{gp mark 1}{(5.564,4.433)}
\gppoint{gp mark 1}{(5.567,4.423)}
\gppoint{gp mark 1}{(5.569,4.442)}
\gppoint{gp mark 1}{(5.571,4.449)}
\gppoint{gp mark 1}{(5.573,4.457)}
\gppoint{gp mark 1}{(5.576,4.428)}
\gppoint{gp mark 1}{(5.578,4.480)}
\gppoint{gp mark 1}{(5.580,4.439)}
\gppoint{gp mark 1}{(5.582,4.442)}
\gppoint{gp mark 1}{(5.585,4.449)}
\gppoint{gp mark 1}{(5.587,4.428)}
\gppoint{gp mark 1}{(5.589,4.454)}
\gppoint{gp mark 1}{(5.591,4.462)}
\gppoint{gp mark 1}{(5.594,4.452)}
\gppoint{gp mark 1}{(5.596,4.482)}
\gppoint{gp mark 1}{(5.598,4.513)}
\gppoint{gp mark 1}{(5.600,4.500)}
\gppoint{gp mark 1}{(5.603,4.547)}
\gppoint{gp mark 1}{(5.605,4.537)}
\gppoint{gp mark 1}{(5.607,4.546)}
\gppoint{gp mark 1}{(5.609,4.536)}
\gppoint{gp mark 1}{(5.612,4.505)}
\gppoint{gp mark 1}{(5.614,4.532)}
\gppoint{gp mark 1}{(5.616,4.562)}
\gppoint{gp mark 1}{(5.618,4.540)}
\gppoint{gp mark 1}{(5.621,4.561)}
\gppoint{gp mark 1}{(5.623,4.542)}
\gppoint{gp mark 1}{(5.625,4.601)}
\gppoint{gp mark 1}{(5.627,4.591)}
\gppoint{gp mark 1}{(5.630,4.590)}
\gppoint{gp mark 1}{(5.632,4.589)}
\gppoint{gp mark 1}{(5.634,4.612)}
\gppoint{gp mark 1}{(5.636,4.573)}
\gppoint{gp mark 1}{(5.639,4.601)}
\gppoint{gp mark 1}{(5.641,4.588)}
\gppoint{gp mark 1}{(5.643,4.569)}
\gppoint{gp mark 1}{(5.645,4.627)}
\gppoint{gp mark 1}{(5.648,4.635)}
\gppoint{gp mark 1}{(5.650,4.629)}
\gppoint{gp mark 1}{(5.652,4.609)}
\gppoint{gp mark 1}{(5.654,4.647)}
\gppoint{gp mark 1}{(5.657,4.600)}
\gppoint{gp mark 1}{(5.659,4.636)}
\gppoint{gp mark 1}{(5.661,4.621)}
\gppoint{gp mark 1}{(5.663,4.673)}
\gppoint{gp mark 1}{(5.666,4.680)}
\gppoint{gp mark 1}{(5.668,4.661)}
\gppoint{gp mark 1}{(5.670,4.657)}
\gppoint{gp mark 1}{(5.672,4.690)}
\gppoint{gp mark 1}{(5.675,4.675)}
\gppoint{gp mark 1}{(5.677,4.694)}
\gppoint{gp mark 1}{(5.679,4.686)}
\gppoint{gp mark 1}{(5.681,4.700)}
\gppoint{gp mark 1}{(5.684,4.696)}
\gppoint{gp mark 1}{(5.686,4.692)}
\gppoint{gp mark 1}{(5.688,4.743)}
\gppoint{gp mark 1}{(5.690,4.729)}
\gppoint{gp mark 1}{(5.693,4.713)}
\gppoint{gp mark 1}{(5.695,4.720)}
\gppoint{gp mark 1}{(5.697,4.735)}
\gppoint{gp mark 1}{(5.699,4.731)}
\gppoint{gp mark 1}{(5.702,4.727)}
\gppoint{gp mark 1}{(5.704,4.747)}
\gppoint{gp mark 1}{(5.706,4.771)}
\gppoint{gp mark 1}{(5.708,4.760)}
\gppoint{gp mark 1}{(5.711,4.795)}
\gppoint{gp mark 1}{(5.713,4.788)}
\gppoint{gp mark 1}{(5.715,4.763)}
\gppoint{gp mark 1}{(5.717,4.763)}
\gppoint{gp mark 1}{(5.720,4.765)}
\gppoint{gp mark 1}{(5.722,4.754)}
\gppoint{gp mark 1}{(5.724,4.765)}
\gppoint{gp mark 1}{(5.726,4.771)}
\gppoint{gp mark 1}{(5.729,4.768)}
\gppoint{gp mark 1}{(5.731,4.799)}
\gppoint{gp mark 1}{(5.733,4.794)}
\gppoint{gp mark 1}{(5.735,4.816)}
\gppoint{gp mark 1}{(5.738,4.825)}
\gppoint{gp mark 1}{(5.740,4.827)}
\gppoint{gp mark 1}{(5.742,4.832)}
\gppoint{gp mark 1}{(5.744,4.826)}
\gppoint{gp mark 1}{(5.747,4.821)}
\gppoint{gp mark 1}{(5.749,4.813)}
\gppoint{gp mark 1}{(5.751,4.822)}
\gppoint{gp mark 1}{(5.753,4.828)}
\gppoint{gp mark 1}{(5.756,4.831)}
\gppoint{gp mark 1}{(5.758,4.818)}
\gppoint{gp mark 1}{(5.760,4.860)}
\gppoint{gp mark 1}{(5.762,4.845)}
\gppoint{gp mark 1}{(5.765,4.850)}
\gppoint{gp mark 1}{(5.767,4.863)}
\gppoint{gp mark 1}{(5.769,4.860)}
\gppoint{gp mark 1}{(5.771,4.896)}
\gppoint{gp mark 1}{(5.774,4.880)}
\gppoint{gp mark 1}{(5.776,4.867)}
\gppoint{gp mark 1}{(5.778,4.875)}
\gppoint{gp mark 1}{(5.780,4.905)}
\gppoint{gp mark 1}{(5.783,4.907)}
\gppoint{gp mark 1}{(5.785,4.915)}
\gppoint{gp mark 1}{(5.787,4.914)}
\gppoint{gp mark 1}{(5.789,4.920)}
\gppoint{gp mark 1}{(5.792,4.941)}
\gppoint{gp mark 1}{(5.794,4.915)}
\gppoint{gp mark 1}{(5.796,4.921)}
\gppoint{gp mark 1}{(5.798,4.925)}
\gppoint{gp mark 1}{(5.801,4.934)}
\gppoint{gp mark 1}{(5.803,4.921)}
\gppoint{gp mark 1}{(5.805,4.895)}
\gppoint{gp mark 1}{(5.807,4.930)}
\gppoint{gp mark 1}{(5.810,4.933)}
\gppoint{gp mark 1}{(5.812,4.942)}
\gppoint{gp mark 1}{(5.814,4.958)}
\gppoint{gp mark 1}{(5.816,4.936)}
\gppoint{gp mark 1}{(5.819,4.934)}
\gppoint{gp mark 1}{(5.821,4.942)}
\gppoint{gp mark 1}{(5.823,4.942)}
\gppoint{gp mark 1}{(5.825,4.963)}
\gppoint{gp mark 1}{(5.828,4.961)}
\gppoint{gp mark 1}{(5.830,4.965)}
\gppoint{gp mark 1}{(5.832,4.994)}
\gppoint{gp mark 1}{(5.834,4.974)}
\gppoint{gp mark 1}{(5.837,4.981)}
\gppoint{gp mark 1}{(5.839,4.984)}
\gppoint{gp mark 1}{(5.841,4.980)}
\gppoint{gp mark 1}{(5.843,4.980)}
\gppoint{gp mark 1}{(5.846,4.978)}
\gppoint{gp mark 1}{(5.848,5.009)}
\gppoint{gp mark 1}{(5.850,4.987)}
\gppoint{gp mark 1}{(5.852,4.977)}
\gppoint{gp mark 1}{(5.855,4.969)}
\gppoint{gp mark 1}{(5.857,4.999)}
\gppoint{gp mark 1}{(5.859,4.988)}
\gppoint{gp mark 1}{(5.861,5.019)}
\gppoint{gp mark 1}{(5.864,5.055)}
\gppoint{gp mark 1}{(5.866,5.031)}
\gppoint{gp mark 1}{(5.868,5.028)}
\gppoint{gp mark 1}{(5.870,5.035)}
\gppoint{gp mark 1}{(5.873,5.028)}
\gppoint{gp mark 1}{(5.875,5.035)}
\gppoint{gp mark 1}{(5.877,5.044)}
\gppoint{gp mark 1}{(5.879,5.027)}
\gppoint{gp mark 1}{(5.882,5.057)}
\gppoint{gp mark 1}{(5.884,5.048)}
\gppoint{gp mark 1}{(5.886,5.014)}
\gppoint{gp mark 1}{(5.888,5.058)}
\gppoint{gp mark 1}{(5.891,5.063)}
\gppoint{gp mark 1}{(5.893,5.068)}
\gppoint{gp mark 1}{(5.895,5.090)}
\gppoint{gp mark 1}{(5.897,5.060)}
\gppoint{gp mark 1}{(5.900,5.055)}
\gppoint{gp mark 1}{(5.902,5.044)}
\gppoint{gp mark 1}{(5.904,5.072)}
\gppoint{gp mark 1}{(5.906,5.031)}
\gppoint{gp mark 1}{(5.909,5.082)}
\gppoint{gp mark 1}{(5.911,5.125)}
\gppoint{gp mark 1}{(5.913,5.092)}
\gppoint{gp mark 1}{(5.915,5.112)}
\gppoint{gp mark 1}{(5.918,5.113)}
\gppoint{gp mark 1}{(5.920,5.101)}
\gppoint{gp mark 1}{(5.922,5.070)}
\gppoint{gp mark 1}{(5.924,5.063)}
\gppoint{gp mark 1}{(5.927,5.107)}
\gppoint{gp mark 1}{(5.929,5.091)}
\gppoint{gp mark 1}{(5.931,5.108)}
\gppoint{gp mark 1}{(5.933,5.085)}
\gppoint{gp mark 1}{(5.936,5.131)}
\gppoint{gp mark 1}{(5.938,5.123)}
\gppoint{gp mark 1}{(5.940,5.094)}
\gppoint{gp mark 1}{(5.942,5.131)}
\gppoint{gp mark 1}{(5.945,5.118)}
\gppoint{gp mark 1}{(5.947,5.117)}
\gppoint{gp mark 1}{(5.949,5.092)}
\gppoint{gp mark 1}{(5.951,5.116)}
\gppoint{gp mark 1}{(5.954,5.095)}
\gppoint{gp mark 1}{(5.956,5.089)}
\gppoint{gp mark 1}{(5.958,5.106)}
\gppoint{gp mark 1}{(5.960,5.150)}
\gppoint{gp mark 1}{(5.963,5.165)}
\gppoint{gp mark 1}{(5.965,5.142)}
\gppoint{gp mark 1}{(5.967,5.133)}
\gppoint{gp mark 1}{(5.969,5.134)}
\gppoint{gp mark 1}{(5.972,5.135)}
\gppoint{gp mark 1}{(5.974,5.154)}
\gppoint{gp mark 1}{(5.976,5.161)}
\gppoint{gp mark 1}{(5.978,5.177)}
\gppoint{gp mark 1}{(5.981,5.211)}
\gppoint{gp mark 1}{(5.983,5.174)}
\gppoint{gp mark 1}{(5.985,5.197)}
\gppoint{gp mark 1}{(5.987,5.149)}
\gppoint{gp mark 1}{(5.990,5.147)}
\gppoint{gp mark 1}{(5.992,5.137)}
\gppoint{gp mark 1}{(5.994,5.155)}
\gppoint{gp mark 1}{(5.996,5.151)}
\gppoint{gp mark 1}{(5.999,5.142)}
\gppoint{gp mark 1}{(6.001,5.162)}
\gppoint{gp mark 1}{(6.003,5.160)}
\gppoint{gp mark 1}{(6.005,5.179)}
\gppoint{gp mark 1}{(6.008,5.165)}
\gppoint{gp mark 1}{(6.010,5.148)}
\gppoint{gp mark 1}{(6.012,5.141)}
\gppoint{gp mark 1}{(6.014,5.168)}
\gppoint{gp mark 1}{(6.017,5.160)}
\gppoint{gp mark 1}{(6.019,5.193)}
\gppoint{gp mark 1}{(6.021,5.148)}
\gppoint{gp mark 1}{(6.023,5.183)}
\gppoint{gp mark 1}{(6.026,5.211)}
\gppoint{gp mark 1}{(6.028,5.166)}
\gppoint{gp mark 1}{(6.030,5.230)}
\gppoint{gp mark 1}{(6.032,5.212)}
\gppoint{gp mark 1}{(6.035,5.228)}
\gppoint{gp mark 1}{(6.037,5.175)}
\gppoint{gp mark 1}{(6.039,5.182)}
\gppoint{gp mark 1}{(6.041,5.211)}
\gppoint{gp mark 1}{(6.044,5.187)}
\gppoint{gp mark 1}{(6.046,5.186)}
\gppoint{gp mark 1}{(6.048,5.184)}
\gppoint{gp mark 1}{(6.050,5.207)}
\gppoint{gp mark 1}{(6.053,5.200)}
\gppoint{gp mark 1}{(6.055,5.246)}
\gppoint{gp mark 1}{(6.057,5.215)}
\gppoint{gp mark 1}{(6.059,5.233)}
\gppoint{gp mark 1}{(6.062,5.186)}
\gppoint{gp mark 1}{(6.064,5.189)}
\gppoint{gp mark 1}{(6.066,5.191)}
\gppoint{gp mark 1}{(6.068,5.200)}
\gppoint{gp mark 1}{(6.071,5.209)}
\gppoint{gp mark 1}{(6.073,5.219)}
\gppoint{gp mark 1}{(6.075,5.235)}
\gppoint{gp mark 1}{(6.077,5.216)}
\gppoint{gp mark 1}{(6.080,5.202)}
\gppoint{gp mark 1}{(6.082,5.221)}
\gppoint{gp mark 1}{(6.084,5.222)}
\gppoint{gp mark 1}{(6.086,5.230)}
\gppoint{gp mark 1}{(6.089,5.242)}
\gppoint{gp mark 1}{(6.091,5.250)}
\gppoint{gp mark 1}{(6.093,5.216)}
\gppoint{gp mark 1}{(6.095,5.221)}
\gppoint{gp mark 1}{(6.098,5.238)}
\gppoint{gp mark 1}{(6.100,5.227)}
\gppoint{gp mark 1}{(6.102,5.241)}
\gppoint{gp mark 1}{(6.104,5.241)}
\gppoint{gp mark 1}{(6.107,5.265)}
\gppoint{gp mark 1}{(6.109,5.253)}
\gppoint{gp mark 1}{(6.111,5.231)}
\gppoint{gp mark 1}{(6.113,5.230)}
\gppoint{gp mark 1}{(6.116,5.224)}
\gppoint{gp mark 1}{(6.118,5.244)}
\gppoint{gp mark 1}{(6.120,5.222)}
\gppoint{gp mark 1}{(6.122,5.251)}
\gppoint{gp mark 1}{(6.125,5.240)}
\gppoint{gp mark 1}{(6.127,5.249)}
\gppoint{gp mark 1}{(6.129,5.228)}
\gppoint{gp mark 1}{(6.131,5.226)}
\gppoint{gp mark 1}{(6.134,5.282)}
\gppoint{gp mark 1}{(6.136,5.246)}
\gppoint{gp mark 1}{(6.138,5.250)}
\gppoint{gp mark 1}{(6.140,5.272)}
\gppoint{gp mark 1}{(6.143,5.253)}
\gppoint{gp mark 1}{(6.145,5.231)}
\gppoint{gp mark 1}{(6.147,5.230)}
\gppoint{gp mark 1}{(6.149,5.247)}
\gppoint{gp mark 1}{(6.152,5.244)}
\gppoint{gp mark 1}{(6.154,5.264)}
\gppoint{gp mark 1}{(6.156,5.276)}
\gppoint{gp mark 1}{(6.158,5.251)}
\gppoint{gp mark 1}{(6.161,5.239)}
\gppoint{gp mark 1}{(6.163,5.250)}
\gppoint{gp mark 1}{(6.165,5.212)}
\gppoint{gp mark 1}{(6.167,5.291)}
\gppoint{gp mark 1}{(6.170,5.282)}
\gppoint{gp mark 1}{(6.172,5.271)}
\gppoint{gp mark 1}{(6.174,5.261)}
\gppoint{gp mark 1}{(6.176,5.226)}
\gppoint{gp mark 1}{(6.179,5.288)}
\gppoint{gp mark 1}{(6.181,5.217)}
\gppoint{gp mark 1}{(6.183,5.234)}
\gppoint{gp mark 1}{(6.185,5.264)}
\gppoint{gp mark 1}{(6.188,5.260)}
\gppoint{gp mark 1}{(6.190,5.248)}
\gppoint{gp mark 1}{(6.192,5.271)}
\gppoint{gp mark 1}{(6.194,5.266)}
\gppoint{gp mark 1}{(6.197,5.250)}
\gppoint{gp mark 1}{(6.199,5.235)}
\gppoint{gp mark 1}{(6.201,5.239)}
\gppoint{gp mark 1}{(6.203,5.283)}
\gppoint{gp mark 1}{(6.206,5.271)}
\gppoint{gp mark 1}{(6.208,5.246)}
\gppoint{gp mark 1}{(6.210,5.259)}
\gppoint{gp mark 1}{(6.212,5.271)}
\gppoint{gp mark 1}{(6.215,5.251)}
\gppoint{gp mark 1}{(6.217,5.265)}
\gppoint{gp mark 1}{(6.219,5.250)}
\gppoint{gp mark 1}{(6.221,5.254)}
\gppoint{gp mark 1}{(6.224,5.264)}
\gppoint{gp mark 1}{(6.226,5.250)}
\gppoint{gp mark 1}{(6.228,5.232)}
\gppoint{gp mark 1}{(6.230,5.274)}
\gppoint{gp mark 1}{(6.233,5.231)}
\gppoint{gp mark 1}{(6.235,5.239)}
\gppoint{gp mark 1}{(6.237,5.233)}
\gppoint{gp mark 1}{(6.239,5.238)}
\gppoint{gp mark 1}{(6.242,5.248)}
\gppoint{gp mark 1}{(6.244,5.267)}
\gppoint{gp mark 1}{(6.246,5.275)}
\gppoint{gp mark 1}{(6.248,5.224)}
\gppoint{gp mark 1}{(6.251,5.258)}
\gppoint{gp mark 1}{(6.253,5.273)}
\gppoint{gp mark 1}{(6.255,5.279)}
\gppoint{gp mark 1}{(6.257,5.241)}
\gppoint{gp mark 1}{(6.260,5.241)}
\gppoint{gp mark 1}{(6.262,5.279)}
\gppoint{gp mark 1}{(6.264,5.243)}
\gppoint{gp mark 1}{(6.266,5.239)}
\gppoint{gp mark 1}{(6.269,5.241)}
\gppoint{gp mark 1}{(6.271,5.199)}
\gppoint{gp mark 1}{(6.273,5.223)}
\gppoint{gp mark 1}{(6.275,5.245)}
\gppoint{gp mark 1}{(6.278,5.229)}
\gppoint{gp mark 1}{(6.280,5.249)}
\gppoint{gp mark 1}{(6.282,5.264)}
\gppoint{gp mark 1}{(6.284,5.270)}
\gppoint{gp mark 1}{(6.287,5.219)}
\gppoint{gp mark 1}{(6.289,5.243)}
\gppoint{gp mark 1}{(6.291,5.254)}
\gppoint{gp mark 1}{(6.293,5.197)}
\gppoint{gp mark 1}{(6.296,5.224)}
\gppoint{gp mark 1}{(6.298,5.255)}
\gppoint{gp mark 1}{(6.300,5.242)}
\gppoint{gp mark 1}{(6.302,5.234)}
\gppoint{gp mark 1}{(6.305,5.236)}
\gppoint{gp mark 1}{(6.307,5.254)}
\gppoint{gp mark 1}{(6.309,5.275)}
\gppoint{gp mark 1}{(6.311,5.283)}
\gppoint{gp mark 1}{(6.314,5.248)}
\gppoint{gp mark 1}{(6.316,5.215)}
\gppoint{gp mark 1}{(6.318,5.206)}
\gppoint{gp mark 1}{(6.320,5.212)}
\gppoint{gp mark 1}{(6.323,5.240)}
\gppoint{gp mark 1}{(6.325,5.246)}
\gppoint{gp mark 1}{(6.327,5.231)}
\gppoint{gp mark 1}{(6.329,5.220)}
\gppoint{gp mark 1}{(6.332,5.205)}
\gppoint{gp mark 1}{(6.334,5.205)}
\gppoint{gp mark 1}{(6.336,5.188)}
\gppoint{gp mark 1}{(6.338,5.205)}
\gppoint{gp mark 1}{(6.341,5.213)}
\gppoint{gp mark 1}{(6.343,5.206)}
\gppoint{gp mark 1}{(6.345,5.190)}
\gppoint{gp mark 1}{(6.347,5.230)}
\gppoint{gp mark 1}{(6.350,5.228)}
\gppoint{gp mark 1}{(6.352,5.171)}
\gppoint{gp mark 1}{(6.354,5.183)}
\gppoint{gp mark 1}{(6.356,5.184)}
\gppoint{gp mark 1}{(6.359,5.209)}
\gppoint{gp mark 1}{(6.361,5.188)}
\gppoint{gp mark 1}{(6.363,5.180)}
\gppoint{gp mark 1}{(6.365,5.166)}
\gppoint{gp mark 1}{(6.368,5.211)}
\gppoint{gp mark 1}{(6.370,5.210)}
\gppoint{gp mark 1}{(6.372,5.164)}
\gppoint{gp mark 1}{(6.374,5.161)}
\gppoint{gp mark 1}{(6.377,5.201)}
\gppoint{gp mark 1}{(6.379,5.183)}
\gppoint{gp mark 1}{(6.381,5.170)}
\gppoint{gp mark 1}{(6.384,5.169)}
\gppoint{gp mark 1}{(6.386,5.161)}
\gppoint{gp mark 1}{(6.388,5.185)}
\gppoint{gp mark 1}{(6.390,5.191)}
\gppoint{gp mark 1}{(6.393,5.165)}
\gppoint{gp mark 1}{(6.395,5.214)}
\gppoint{gp mark 1}{(6.397,5.163)}
\gppoint{gp mark 1}{(6.399,5.180)}
\gppoint{gp mark 1}{(6.402,5.120)}
\gppoint{gp mark 1}{(6.404,5.130)}
\gppoint{gp mark 1}{(6.406,5.113)}
\gppoint{gp mark 1}{(6.408,5.154)}
\gppoint{gp mark 1}{(6.411,5.140)}
\gppoint{gp mark 1}{(6.413,5.112)}
\gppoint{gp mark 1}{(6.415,5.158)}
\gppoint{gp mark 1}{(6.417,5.160)}
\gppoint{gp mark 1}{(6.420,5.128)}
\gppoint{gp mark 1}{(6.422,5.135)}
\gppoint{gp mark 1}{(6.424,5.121)}
\gppoint{gp mark 1}{(6.426,5.122)}
\gppoint{gp mark 1}{(6.429,5.093)}
\gppoint{gp mark 1}{(6.431,5.129)}
\gppoint{gp mark 1}{(6.433,5.114)}
\gppoint{gp mark 1}{(6.435,5.099)}
\gppoint{gp mark 1}{(6.438,5.067)}
\gppoint{gp mark 1}{(6.440,5.067)}
\gppoint{gp mark 1}{(6.442,5.093)}
\gppoint{gp mark 1}{(6.444,5.067)}
\gppoint{gp mark 1}{(6.447,5.077)}
\gppoint{gp mark 1}{(6.449,5.083)}
\gppoint{gp mark 1}{(6.451,5.061)}
\gppoint{gp mark 1}{(6.453,5.104)}
\gppoint{gp mark 1}{(6.456,5.090)}
\gppoint{gp mark 1}{(6.458,5.098)}
\gppoint{gp mark 1}{(6.460,5.077)}
\gppoint{gp mark 1}{(6.462,5.105)}
\gppoint{gp mark 1}{(6.465,5.081)}
\gppoint{gp mark 1}{(6.467,5.061)}
\gppoint{gp mark 1}{(6.469,5.047)}
\gppoint{gp mark 1}{(6.471,5.058)}
\gppoint{gp mark 1}{(6.474,5.080)}
\gppoint{gp mark 1}{(6.476,5.087)}
\gppoint{gp mark 1}{(6.478,5.049)}
\gppoint{gp mark 1}{(6.480,5.035)}
\gppoint{gp mark 1}{(6.483,5.052)}
\gppoint{gp mark 1}{(6.485,5.046)}
\gppoint{gp mark 1}{(6.487,5.043)}
\gppoint{gp mark 1}{(6.489,5.050)}
\gppoint{gp mark 1}{(6.492,5.032)}
\gppoint{gp mark 1}{(6.494,5.031)}
\gppoint{gp mark 1}{(6.496,5.026)}
\gppoint{gp mark 1}{(6.498,5.009)}
\gppoint{gp mark 1}{(6.501,4.998)}
\gppoint{gp mark 1}{(6.503,5.061)}
\gppoint{gp mark 1}{(6.505,5.027)}
\gppoint{gp mark 1}{(6.507,5.015)}
\gppoint{gp mark 1}{(6.510,5.014)}
\gppoint{gp mark 1}{(6.512,5.027)}
\gppoint{gp mark 1}{(6.514,5.011)}
\gppoint{gp mark 1}{(6.516,4.952)}
\gppoint{gp mark 1}{(6.519,5.006)}
\gppoint{gp mark 1}{(6.521,5.003)}
\gppoint{gp mark 1}{(6.523,4.977)}
\gppoint{gp mark 1}{(6.525,4.957)}
\gppoint{gp mark 1}{(6.528,4.961)}
\gppoint{gp mark 1}{(6.530,4.989)}
\gppoint{gp mark 1}{(6.532,4.963)}
\gppoint{gp mark 1}{(6.534,4.979)}
\gppoint{gp mark 1}{(6.537,4.956)}
\gppoint{gp mark 1}{(6.539,4.997)}
\gppoint{gp mark 1}{(6.541,5.010)}
\gppoint{gp mark 1}{(6.543,4.971)}
\gppoint{gp mark 1}{(6.546,4.980)}
\gppoint{gp mark 1}{(6.548,4.987)}
\gppoint{gp mark 1}{(6.550,4.961)}
\gppoint{gp mark 1}{(6.552,4.972)}
\gppoint{gp mark 1}{(6.555,4.944)}
\gppoint{gp mark 1}{(6.557,4.956)}
\gppoint{gp mark 1}{(6.559,4.957)}
\gppoint{gp mark 1}{(6.561,4.942)}
\gppoint{gp mark 1}{(6.564,4.930)}
\gppoint{gp mark 1}{(6.566,4.911)}
\gppoint{gp mark 1}{(6.568,4.932)}
\gppoint{gp mark 1}{(6.570,4.936)}
\gppoint{gp mark 1}{(6.573,4.930)}
\gppoint{gp mark 1}{(6.575,4.915)}
\gppoint{gp mark 1}{(6.577,4.880)}
\gppoint{gp mark 1}{(6.579,4.895)}
\gppoint{gp mark 1}{(6.582,4.896)}
\gppoint{gp mark 1}{(6.584,4.900)}
\gppoint{gp mark 1}{(6.586,4.879)}
\gppoint{gp mark 1}{(6.588,4.910)}
\gppoint{gp mark 1}{(6.591,4.843)}
\gppoint{gp mark 1}{(6.593,4.902)}
\gppoint{gp mark 1}{(6.595,4.852)}
\gppoint{gp mark 1}{(6.597,4.890)}
\gppoint{gp mark 1}{(6.600,4.847)}
\gppoint{gp mark 1}{(6.602,4.834)}
\gppoint{gp mark 1}{(6.604,4.861)}
\gppoint{gp mark 1}{(6.606,4.851)}
\gppoint{gp mark 1}{(6.609,4.841)}
\gppoint{gp mark 1}{(6.611,4.861)}
\gppoint{gp mark 1}{(6.613,4.837)}
\gppoint{gp mark 1}{(6.615,4.851)}
\gppoint{gp mark 1}{(6.618,4.812)}
\gppoint{gp mark 1}{(6.620,4.846)}
\gppoint{gp mark 1}{(6.622,4.827)}
\gppoint{gp mark 1}{(6.624,4.829)}
\gppoint{gp mark 1}{(6.627,4.823)}
\gppoint{gp mark 1}{(6.629,4.821)}
\gppoint{gp mark 1}{(6.631,4.806)}
\gppoint{gp mark 1}{(6.633,4.841)}
\gppoint{gp mark 1}{(6.636,4.767)}
\gppoint{gp mark 1}{(6.638,4.812)}
\gppoint{gp mark 1}{(6.640,4.794)}
\gppoint{gp mark 1}{(6.642,4.751)}
\gppoint{gp mark 1}{(6.645,4.782)}
\gppoint{gp mark 1}{(6.647,4.760)}
\gppoint{gp mark 1}{(6.649,4.756)}
\gppoint{gp mark 1}{(6.651,4.769)}
\gppoint{gp mark 1}{(6.654,4.764)}
\gppoint{gp mark 1}{(6.656,4.769)}
\gppoint{gp mark 1}{(6.658,4.748)}
\gppoint{gp mark 1}{(6.660,4.755)}
\gppoint{gp mark 1}{(6.663,4.740)}
\gppoint{gp mark 1}{(6.665,4.742)}
\gppoint{gp mark 1}{(6.667,4.705)}
\gppoint{gp mark 1}{(6.669,4.730)}
\gppoint{gp mark 1}{(6.672,4.730)}
\gppoint{gp mark 1}{(6.674,4.737)}
\gppoint{gp mark 1}{(6.676,4.713)}
\gppoint{gp mark 1}{(6.678,4.718)}
\gppoint{gp mark 1}{(6.681,4.718)}
\gppoint{gp mark 1}{(6.683,4.686)}
\gppoint{gp mark 1}{(6.685,4.688)}
\gppoint{gp mark 1}{(6.687,4.709)}
\gppoint{gp mark 1}{(6.690,4.707)}
\gppoint{gp mark 1}{(6.692,4.686)}
\gppoint{gp mark 1}{(6.694,4.655)}
\gppoint{gp mark 1}{(6.696,4.638)}
\gppoint{gp mark 1}{(6.699,4.683)}
\gppoint{gp mark 1}{(6.701,4.677)}
\gppoint{gp mark 1}{(6.703,4.683)}
\gppoint{gp mark 1}{(6.705,4.652)}
\gppoint{gp mark 1}{(6.708,4.642)}
\gppoint{gp mark 1}{(6.710,4.636)}
\gppoint{gp mark 1}{(6.712,4.635)}
\gppoint{gp mark 1}{(6.714,4.650)}
\gppoint{gp mark 1}{(6.717,4.646)}
\gppoint{gp mark 1}{(6.719,4.632)}
\gppoint{gp mark 1}{(6.721,4.637)}
\gppoint{gp mark 1}{(6.723,4.617)}
\gppoint{gp mark 1}{(6.726,4.622)}
\gppoint{gp mark 1}{(6.728,4.617)}
\gppoint{gp mark 1}{(6.730,4.627)}
\gppoint{gp mark 1}{(6.732,4.597)}
\gppoint{gp mark 1}{(6.735,4.591)}
\gppoint{gp mark 1}{(6.737,4.594)}
\gppoint{gp mark 1}{(6.739,4.580)}
\gppoint{gp mark 1}{(6.741,4.565)}
\gppoint{gp mark 1}{(6.744,4.563)}
\gppoint{gp mark 1}{(6.746,4.572)}
\gppoint{gp mark 1}{(6.748,4.571)}
\gppoint{gp mark 1}{(6.750,4.567)}
\gppoint{gp mark 1}{(6.753,4.557)}
\gppoint{gp mark 1}{(6.755,4.569)}
\gppoint{gp mark 1}{(6.757,4.540)}
\gppoint{gp mark 1}{(6.759,4.545)}
\gppoint{gp mark 1}{(6.762,4.524)}
\gppoint{gp mark 1}{(6.764,4.528)}
\gppoint{gp mark 1}{(6.766,4.553)}
\gppoint{gp mark 1}{(6.768,4.537)}
\gppoint{gp mark 1}{(6.771,4.507)}
\gppoint{gp mark 1}{(6.773,4.497)}
\gppoint{gp mark 1}{(6.775,4.503)}
\gppoint{gp mark 1}{(6.777,4.513)}
\gppoint{gp mark 1}{(6.780,4.469)}
\gppoint{gp mark 1}{(6.782,4.479)}
\gppoint{gp mark 1}{(6.784,4.465)}
\gppoint{gp mark 1}{(6.786,4.483)}
\gppoint{gp mark 1}{(6.789,4.456)}
\gppoint{gp mark 1}{(6.791,4.449)}
\gppoint{gp mark 1}{(6.793,4.457)}
\gppoint{gp mark 1}{(6.795,4.498)}
\gppoint{gp mark 1}{(6.798,4.466)}
\gppoint{gp mark 1}{(6.800,4.464)}
\gppoint{gp mark 1}{(6.802,4.460)}
\gppoint{gp mark 1}{(6.804,4.485)}
\gppoint{gp mark 1}{(6.807,4.424)}
\gppoint{gp mark 1}{(6.809,4.432)}
\gppoint{gp mark 1}{(6.811,4.410)}
\gppoint{gp mark 1}{(6.813,4.422)}
\gppoint{gp mark 1}{(6.816,4.403)}
\gppoint{gp mark 1}{(6.818,4.387)}
\gppoint{gp mark 1}{(6.820,4.385)}
\gppoint{gp mark 1}{(6.822,4.399)}
\gppoint{gp mark 1}{(6.825,4.383)}
\gppoint{gp mark 1}{(6.827,4.349)}
\gppoint{gp mark 1}{(6.829,4.393)}
\gppoint{gp mark 1}{(6.831,4.393)}
\gppoint{gp mark 1}{(6.834,4.386)}
\gppoint{gp mark 1}{(6.836,4.373)}
\gppoint{gp mark 1}{(6.838,4.355)}
\gppoint{gp mark 1}{(6.840,4.355)}
\gppoint{gp mark 1}{(6.843,4.350)}
\gppoint{gp mark 1}{(6.845,4.364)}
\gppoint{gp mark 1}{(6.847,4.360)}
\gppoint{gp mark 1}{(6.849,4.340)}
\gppoint{gp mark 1}{(6.852,4.315)}
\gppoint{gp mark 1}{(6.854,4.306)}
\gppoint{gp mark 1}{(6.856,4.322)}
\gppoint{gp mark 1}{(6.858,4.299)}
\gppoint{gp mark 1}{(6.861,4.310)}
\gppoint{gp mark 1}{(6.863,4.316)}
\gppoint{gp mark 1}{(6.865,4.276)}
\gppoint{gp mark 1}{(6.867,4.255)}
\gppoint{gp mark 1}{(6.870,4.269)}
\gppoint{gp mark 1}{(6.872,4.268)}
\gppoint{gp mark 1}{(6.874,4.269)}
\gppoint{gp mark 1}{(6.876,4.255)}
\gppoint{gp mark 1}{(6.879,4.228)}
\gppoint{gp mark 1}{(6.881,4.237)}
\gppoint{gp mark 1}{(6.883,4.241)}
\gppoint{gp mark 1}{(6.885,4.222)}
\gppoint{gp mark 1}{(6.888,4.232)}
\gppoint{gp mark 1}{(6.890,4.229)}
\gppoint{gp mark 1}{(6.892,4.228)}
\gppoint{gp mark 1}{(6.894,4.219)}
\gppoint{gp mark 1}{(6.897,4.242)}
\gppoint{gp mark 1}{(6.899,4.198)}
\gppoint{gp mark 1}{(6.901,4.214)}
\gppoint{gp mark 1}{(6.903,4.173)}
\gppoint{gp mark 1}{(6.906,4.158)}
\gppoint{gp mark 1}{(6.908,4.165)}
\gppoint{gp mark 1}{(6.910,4.165)}
\gppoint{gp mark 1}{(6.912,4.167)}
\gppoint{gp mark 1}{(6.915,4.140)}
\gppoint{gp mark 1}{(6.917,4.162)}
\gppoint{gp mark 1}{(6.919,4.159)}
\gppoint{gp mark 1}{(6.921,4.123)}
\gppoint{gp mark 1}{(6.924,4.149)}
\gppoint{gp mark 1}{(6.926,4.128)}
\gppoint{gp mark 1}{(6.928,4.130)}
\gppoint{gp mark 1}{(6.930,4.122)}
\gppoint{gp mark 1}{(6.933,4.125)}
\gppoint{gp mark 1}{(6.935,4.153)}
\gppoint{gp mark 1}{(6.937,4.165)}
\gppoint{gp mark 1}{(6.939,4.129)}
\gppoint{gp mark 1}{(6.942,4.124)}
\gppoint{gp mark 1}{(6.944,4.112)}
\gppoint{gp mark 1}{(6.946,4.062)}
\gppoint{gp mark 1}{(6.948,4.067)}
\gppoint{gp mark 1}{(6.951,4.072)}
\gppoint{gp mark 1}{(6.953,4.052)}
\gppoint{gp mark 1}{(6.955,4.042)}
\gppoint{gp mark 1}{(6.957,4.052)}
\gppoint{gp mark 1}{(6.960,4.076)}
\gppoint{gp mark 1}{(6.962,4.072)}
\gppoint{gp mark 1}{(6.964,4.067)}
\gppoint{gp mark 1}{(6.966,4.029)}
\gppoint{gp mark 1}{(6.969,4.002)}
\gppoint{gp mark 1}{(6.971,4.011)}
\gppoint{gp mark 1}{(6.973,3.987)}
\gppoint{gp mark 1}{(6.975,4.020)}
\gppoint{gp mark 1}{(6.978,3.997)}
\gppoint{gp mark 1}{(6.980,3.996)}
\gppoint{gp mark 1}{(6.982,3.988)}
\gppoint{gp mark 1}{(6.984,3.986)}
\gppoint{gp mark 1}{(6.987,4.015)}
\gppoint{gp mark 1}{(6.989,4.000)}
\gppoint{gp mark 1}{(6.991,4.001)}
\gppoint{gp mark 1}{(6.993,3.966)}
\gppoint{gp mark 1}{(6.996,3.927)}
\gppoint{gp mark 1}{(6.998,3.917)}
\gppoint{gp mark 1}{(7.000,3.898)}
\gppoint{gp mark 1}{(7.002,3.910)}
\gppoint{gp mark 1}{(7.005,3.897)}
\gppoint{gp mark 1}{(7.007,3.928)}
\gppoint{gp mark 1}{(7.009,3.909)}
\gppoint{gp mark 1}{(7.011,3.905)}
\gppoint{gp mark 1}{(7.014,3.900)}
\gppoint{gp mark 1}{(7.016,3.888)}
\gppoint{gp mark 1}{(7.018,3.883)}
\gppoint{gp mark 1}{(7.020,3.835)}
\gppoint{gp mark 1}{(7.023,3.839)}
\gppoint{gp mark 1}{(7.025,3.840)}
\gppoint{gp mark 1}{(7.027,3.803)}
\gppoint{gp mark 1}{(7.029,3.833)}
\gppoint{gp mark 1}{(7.032,3.812)}
\gppoint{gp mark 1}{(7.034,3.835)}
\gppoint{gp mark 1}{(7.036,3.823)}
\gppoint{gp mark 1}{(7.038,3.820)}
\gppoint{gp mark 1}{(7.041,3.795)}
\gppoint{gp mark 1}{(7.043,3.802)}
\gppoint{gp mark 1}{(7.045,3.763)}
\gppoint{gp mark 1}{(7.047,3.782)}
\gppoint{gp mark 1}{(7.050,3.776)}
\gppoint{gp mark 1}{(7.052,3.759)}
\gppoint{gp mark 1}{(7.054,3.775)}
\gppoint{gp mark 1}{(7.056,3.783)}
\gppoint{gp mark 1}{(7.059,3.763)}
\gppoint{gp mark 1}{(7.061,3.738)}
\gppoint{gp mark 1}{(7.063,3.717)}
\gppoint{gp mark 1}{(7.065,3.749)}
\gppoint{gp mark 1}{(7.068,3.711)}
\gppoint{gp mark 1}{(7.070,3.720)}
\gppoint{gp mark 1}{(7.072,3.688)}
\gppoint{gp mark 1}{(7.074,3.674)}
\gppoint{gp mark 1}{(7.077,3.691)}
\gppoint{gp mark 1}{(7.079,3.676)}
\gppoint{gp mark 1}{(7.081,3.677)}
\gppoint{gp mark 1}{(7.083,3.677)}
\gppoint{gp mark 1}{(7.086,3.659)}
\gppoint{gp mark 1}{(7.088,3.654)}
\gppoint{gp mark 1}{(7.090,3.613)}
\gppoint{gp mark 1}{(7.092,3.626)}
\gppoint{gp mark 1}{(7.095,3.641)}
\gppoint{gp mark 1}{(7.097,3.631)}
\gppoint{gp mark 1}{(7.099,3.611)}
\gppoint{gp mark 1}{(7.101,3.611)}
\gppoint{gp mark 1}{(7.104,3.612)}
\gppoint{gp mark 1}{(7.106,3.612)}
\gppoint{gp mark 1}{(7.108,3.627)}
\gppoint{gp mark 1}{(7.110,3.592)}
\gppoint{gp mark 1}{(7.113,3.585)}
\gppoint{gp mark 1}{(7.115,3.590)}
\gppoint{gp mark 1}{(7.117,3.572)}
\gppoint{gp mark 1}{(7.119,3.582)}
\gppoint{gp mark 1}{(7.122,3.544)}
\gppoint{gp mark 1}{(7.124,3.548)}
\gppoint{gp mark 1}{(7.126,3.528)}
\gppoint{gp mark 1}{(7.128,3.546)}
\gppoint{gp mark 1}{(7.131,3.540)}
\gppoint{gp mark 1}{(7.133,3.517)}
\gppoint{gp mark 1}{(7.135,3.514)}
\gppoint{gp mark 1}{(7.137,3.502)}
\gppoint{gp mark 1}{(7.140,3.492)}
\gppoint{gp mark 1}{(7.142,3.479)}
\gppoint{gp mark 1}{(7.144,3.475)}
\gppoint{gp mark 1}{(7.146,3.509)}
\gppoint{gp mark 1}{(7.149,3.478)}
\gppoint{gp mark 1}{(7.151,3.445)}
\gppoint{gp mark 1}{(7.153,3.422)}
\gppoint{gp mark 1}{(7.155,3.422)}
\gppoint{gp mark 1}{(7.158,3.440)}
\gppoint{gp mark 1}{(7.160,3.448)}
\gppoint{gp mark 1}{(7.162,3.441)}
\gppoint{gp mark 1}{(7.164,3.418)}
\gppoint{gp mark 1}{(7.167,3.413)}
\gppoint{gp mark 1}{(7.169,3.391)}
\gppoint{gp mark 1}{(7.171,3.437)}
\gppoint{gp mark 1}{(7.173,3.410)}
\gppoint{gp mark 1}{(7.176,3.408)}
\gppoint{gp mark 1}{(7.178,3.377)}
\gppoint{gp mark 1}{(7.180,3.403)}
\gppoint{gp mark 1}{(7.182,3.399)}
\gppoint{gp mark 1}{(7.185,3.430)}
\gppoint{gp mark 1}{(7.187,3.391)}
\gppoint{gp mark 1}{(7.189,3.336)}
\gppoint{gp mark 1}{(7.191,3.353)}
\gppoint{gp mark 1}{(7.194,3.330)}
\gppoint{gp mark 1}{(7.196,3.346)}
\gppoint{gp mark 1}{(7.198,3.345)}
\gppoint{gp mark 1}{(7.200,3.329)}
\gppoint{gp mark 1}{(7.203,3.333)}
\gppoint{gp mark 1}{(7.205,3.325)}
\gppoint{gp mark 1}{(7.207,3.350)}
\gppoint{gp mark 1}{(7.209,3.332)}
\gppoint{gp mark 1}{(7.212,3.309)}
\gppoint{gp mark 1}{(7.214,3.312)}
\gppoint{gp mark 1}{(7.216,3.299)}
\gppoint{gp mark 1}{(7.218,3.280)}
\gppoint{gp mark 1}{(7.221,3.254)}
\gppoint{gp mark 1}{(7.223,3.271)}
\gppoint{gp mark 1}{(7.225,3.231)}
\gppoint{gp mark 1}{(7.227,3.264)}
\gppoint{gp mark 1}{(7.230,3.264)}
\gppoint{gp mark 1}{(7.232,3.270)}
\gppoint{gp mark 1}{(7.234,3.217)}
\gppoint{gp mark 1}{(7.236,3.212)}
\gppoint{gp mark 1}{(7.239,3.232)}
\gppoint{gp mark 1}{(7.241,3.206)}
\gppoint{gp mark 1}{(7.243,3.218)}
\gppoint{gp mark 1}{(7.245,3.226)}
\gppoint{gp mark 1}{(7.248,3.187)}
\gppoint{gp mark 1}{(7.250,3.193)}
\gppoint{gp mark 1}{(7.252,3.181)}
\gppoint{gp mark 1}{(7.254,3.150)}
\gppoint{gp mark 1}{(7.257,3.161)}
\gppoint{gp mark 1}{(7.259,3.124)}
\gppoint{gp mark 1}{(7.261,3.109)}
\gppoint{gp mark 1}{(7.263,3.100)}
\gppoint{gp mark 1}{(7.266,3.099)}
\gppoint{gp mark 1}{(7.268,3.136)}
\gppoint{gp mark 1}{(7.270,3.127)}
\gppoint{gp mark 1}{(7.272,3.112)}
\gppoint{gp mark 1}{(7.275,3.102)}
\gppoint{gp mark 1}{(7.277,3.080)}
\gppoint{gp mark 1}{(7.279,3.088)}
\gppoint{gp mark 1}{(7.281,3.084)}
\gppoint{gp mark 1}{(7.284,3.103)}
\gppoint{gp mark 1}{(7.286,3.058)}
\gppoint{gp mark 1}{(7.288,3.071)}
\gppoint{gp mark 1}{(7.290,3.049)}
\gppoint{gp mark 1}{(7.293,3.064)}
\gppoint{gp mark 1}{(7.295,3.039)}
\gppoint{gp mark 1}{(7.297,3.061)}
\gppoint{gp mark 1}{(7.299,3.053)}
\gppoint{gp mark 1}{(7.302,3.021)}
\gppoint{gp mark 1}{(7.304,2.993)}
\gppoint{gp mark 1}{(7.306,3.037)}
\gppoint{gp mark 1}{(7.308,3.027)}
\gppoint{gp mark 1}{(7.311,3.014)}
\gppoint{gp mark 1}{(7.313,2.970)}
\gppoint{gp mark 1}{(7.315,2.998)}
\gppoint{gp mark 1}{(7.317,2.990)}
\gppoint{gp mark 1}{(7.320,2.977)}
\gppoint{gp mark 1}{(7.322,2.969)}
\gppoint{gp mark 1}{(7.324,2.970)}
\gppoint{gp mark 1}{(7.326,2.965)}
\gppoint{gp mark 1}{(7.329,2.960)}
\gppoint{gp mark 1}{(7.331,2.964)}
\gppoint{gp mark 1}{(7.333,2.923)}
\gppoint{gp mark 1}{(7.335,2.906)}
\gppoint{gp mark 1}{(7.338,2.929)}
\gppoint{gp mark 1}{(7.340,2.937)}
\gppoint{gp mark 1}{(7.342,2.915)}
\gppoint{gp mark 1}{(7.344,2.903)}
\gppoint{gp mark 1}{(7.347,2.890)}
\gppoint{gp mark 1}{(7.349,2.905)}
\gppoint{gp mark 1}{(7.351,2.895)}
\gppoint{gp mark 1}{(7.353,2.877)}
\gppoint{gp mark 1}{(7.356,2.892)}
\gppoint{gp mark 1}{(7.358,2.846)}
\gppoint{gp mark 1}{(7.360,2.853)}
\gppoint{gp mark 1}{(7.362,2.866)}
\gppoint{gp mark 1}{(7.365,2.862)}
\gppoint{gp mark 1}{(7.367,2.839)}
\gppoint{gp mark 1}{(7.369,2.857)}
\gppoint{gp mark 1}{(7.371,2.846)}
\gppoint{gp mark 1}{(7.374,2.838)}
\gppoint{gp mark 1}{(7.376,2.811)}
\gppoint{gp mark 1}{(7.378,2.827)}
\gppoint{gp mark 1}{(7.380,2.820)}
\gppoint{gp mark 1}{(7.383,2.816)}
\gppoint{gp mark 1}{(7.385,2.779)}
\gppoint{gp mark 1}{(7.387,2.780)}
\gppoint{gp mark 1}{(7.389,2.793)}
\gppoint{gp mark 1}{(7.392,2.791)}
\gppoint{gp mark 1}{(7.394,2.757)}
\gppoint{gp mark 1}{(7.396,2.784)}
\gppoint{gp mark 1}{(7.398,2.772)}
\gppoint{gp mark 1}{(7.401,2.724)}
\gppoint{gp mark 1}{(7.403,2.739)}
\gppoint{gp mark 1}{(7.405,2.748)}
\gppoint{gp mark 1}{(7.407,2.748)}
\gppoint{gp mark 1}{(7.410,2.748)}
\gppoint{gp mark 1}{(7.412,2.732)}
\gppoint{gp mark 1}{(7.414,2.716)}
\gppoint{gp mark 1}{(7.416,2.713)}
\gppoint{gp mark 1}{(7.419,2.708)}
\gppoint{gp mark 1}{(7.421,2.707)}
\gppoint{gp mark 1}{(7.423,2.702)}
\gppoint{gp mark 1}{(7.425,2.682)}
\gppoint{gp mark 1}{(7.428,2.711)}
\gppoint{gp mark 1}{(7.430,2.695)}
\gppoint{gp mark 1}{(7.432,2.700)}
\gppoint{gp mark 1}{(7.434,2.702)}
\gppoint{gp mark 1}{(7.437,2.685)}
\gppoint{gp mark 1}{(7.439,2.655)}
\gppoint{gp mark 1}{(7.441,2.679)}
\gppoint{gp mark 1}{(7.443,2.669)}
\gppoint{gp mark 1}{(7.446,2.646)}
\gppoint{gp mark 1}{(7.448,2.640)}
\gppoint{gp mark 1}{(7.450,2.678)}
\gppoint{gp mark 1}{(7.452,2.654)}
\gppoint{gp mark 1}{(7.455,2.643)}
\gppoint{gp mark 1}{(7.457,2.635)}
\gppoint{gp mark 1}{(7.459,2.618)}
\gppoint{gp mark 1}{(7.461,2.646)}
\gppoint{gp mark 1}{(7.464,2.626)}
\gppoint{gp mark 1}{(7.466,2.604)}
\gppoint{gp mark 1}{(7.468,2.607)}
\gppoint{gp mark 1}{(7.470,2.617)}
\gppoint{gp mark 1}{(7.473,2.600)}
\gppoint{gp mark 1}{(7.475,2.581)}
\gppoint{gp mark 1}{(7.477,2.583)}
\gppoint{gp mark 1}{(7.479,2.580)}
\gppoint{gp mark 1}{(7.482,2.586)}
\gppoint{gp mark 1}{(7.484,2.588)}
\gppoint{gp mark 1}{(7.486,2.554)}
\gppoint{gp mark 1}{(7.488,2.546)}
\gppoint{gp mark 1}{(7.491,2.553)}
\gppoint{gp mark 1}{(7.493,2.541)}
\gppoint{gp mark 1}{(7.495,2.538)}
\gppoint{gp mark 1}{(7.497,2.550)}
\gppoint{gp mark 1}{(7.500,2.555)}
\gppoint{gp mark 1}{(7.502,2.561)}
\gppoint{gp mark 1}{(7.504,2.523)}
\gppoint{gp mark 1}{(7.506,2.540)}
\gppoint{gp mark 1}{(7.509,2.532)}
\gppoint{gp mark 1}{(7.511,2.517)}
\gppoint{gp mark 1}{(7.513,2.491)}
\gppoint{gp mark 1}{(7.515,2.506)}
\gppoint{gp mark 1}{(7.518,2.493)}
\gppoint{gp mark 1}{(7.520,2.490)}
\gppoint{gp mark 1}{(7.522,2.502)}
\gppoint{gp mark 1}{(7.524,2.444)}
\gppoint{gp mark 1}{(7.527,2.467)}
\gppoint{gp mark 1}{(7.529,2.460)}
\gppoint{gp mark 1}{(7.531,2.458)}
\gppoint{gp mark 1}{(7.533,2.436)}
\gppoint{gp mark 1}{(7.536,2.453)}
\gppoint{gp mark 1}{(7.538,2.446)}
\gppoint{gp mark 1}{(7.540,2.477)}
\gppoint{gp mark 1}{(7.542,2.456)}
\gppoint{gp mark 1}{(7.545,2.439)}
\gppoint{gp mark 1}{(7.547,2.433)}
\gppoint{gp mark 1}{(7.549,2.416)}
\gppoint{gp mark 1}{(7.551,2.420)}
\gppoint{gp mark 1}{(7.554,2.400)}
\gppoint{gp mark 1}{(7.556,2.408)}
\gppoint{gp mark 1}{(7.558,2.390)}
\gppoint{gp mark 1}{(7.560,2.386)}
\gppoint{gp mark 1}{(7.563,2.391)}
\gppoint{gp mark 1}{(7.565,2.364)}
\gppoint{gp mark 1}{(7.567,2.372)}
\gppoint{gp mark 1}{(7.569,2.365)}
\gppoint{gp mark 1}{(7.572,2.359)}
\gppoint{gp mark 1}{(7.574,2.364)}
\gppoint{gp mark 1}{(7.576,2.343)}
\gppoint{gp mark 1}{(7.578,2.332)}
\gppoint{gp mark 1}{(7.581,2.345)}
\gppoint{gp mark 1}{(7.583,2.336)}
\gppoint{gp mark 1}{(7.585,2.366)}
\gppoint{gp mark 1}{(7.587,2.323)}
\gppoint{gp mark 1}{(7.590,2.336)}
\gppoint{gp mark 1}{(7.592,2.341)}
\gppoint{gp mark 1}{(7.594,2.306)}
\gppoint{gp mark 1}{(7.596,2.293)}
\gppoint{gp mark 1}{(7.599,2.286)}
\gppoint{gp mark 1}{(7.601,2.296)}
\gppoint{gp mark 1}{(7.603,2.295)}
\gppoint{gp mark 1}{(7.605,2.319)}
\gppoint{gp mark 1}{(7.608,2.298)}
\gppoint{gp mark 1}{(7.610,2.289)}
\gppoint{gp mark 1}{(7.612,2.283)}
\gppoint{gp mark 1}{(7.614,2.264)}
\gppoint{gp mark 1}{(7.617,2.271)}
\gppoint{gp mark 1}{(7.619,2.285)}
\gppoint{gp mark 1}{(7.621,2.260)}
\gppoint{gp mark 1}{(7.623,2.258)}
\gppoint{gp mark 1}{(7.626,2.249)}
\gppoint{gp mark 1}{(7.628,2.234)}
\gppoint{gp mark 1}{(7.630,2.250)}
\gppoint{gp mark 1}{(7.632,2.236)}
\gppoint{gp mark 1}{(7.635,2.230)}
\gppoint{gp mark 1}{(7.637,2.230)}
\gppoint{gp mark 1}{(7.639,2.221)}
\gppoint{gp mark 1}{(7.641,2.210)}
\gppoint{gp mark 1}{(7.644,2.228)}
\gppoint{gp mark 1}{(7.646,2.204)}
\gppoint{gp mark 1}{(7.648,2.207)}
\gppoint{gp mark 1}{(7.651,2.203)}
\gppoint{gp mark 1}{(7.653,2.189)}
\gppoint{gp mark 1}{(7.655,2.183)}
\gppoint{gp mark 1}{(7.657,2.170)}
\gppoint{gp mark 1}{(7.660,2.191)}
\gppoint{gp mark 1}{(7.662,2.175)}
\gppoint{gp mark 1}{(7.664,2.163)}
\gppoint{gp mark 1}{(7.666,2.154)}
\gppoint{gp mark 1}{(7.669,2.155)}
\gppoint{gp mark 1}{(7.671,2.150)}
\gppoint{gp mark 1}{(7.673,2.158)}
\gppoint{gp mark 1}{(7.675,2.136)}
\gppoint{gp mark 1}{(7.678,2.143)}
\gppoint{gp mark 1}{(7.680,2.141)}
\gppoint{gp mark 1}{(7.682,2.136)}
\gppoint{gp mark 1}{(7.684,2.116)}
\gppoint{gp mark 1}{(7.687,2.097)}
\gppoint{gp mark 1}{(7.689,2.140)}
\gppoint{gp mark 1}{(7.691,2.100)}
\gppoint{gp mark 1}{(7.693,2.104)}
\gppoint{gp mark 1}{(7.696,2.118)}
\gppoint{gp mark 1}{(7.698,2.096)}
\gppoint{gp mark 1}{(7.700,2.076)}
\gppoint{gp mark 1}{(7.702,2.080)}
\gppoint{gp mark 1}{(7.705,2.060)}
\gppoint{gp mark 1}{(7.707,2.092)}
\gppoint{gp mark 1}{(7.709,2.062)}
\gppoint{gp mark 1}{(7.711,2.059)}
\gppoint{gp mark 1}{(7.714,2.062)}
\gppoint{gp mark 1}{(7.716,2.047)}
\gppoint{gp mark 1}{(7.718,2.063)}
\gppoint{gp mark 1}{(7.720,2.060)}
\gppoint{gp mark 1}{(7.723,2.047)}
\gppoint{gp mark 1}{(7.725,2.046)}
\gppoint{gp mark 1}{(7.727,2.030)}
\gppoint{gp mark 1}{(7.729,2.029)}
\gppoint{gp mark 1}{(7.732,2.029)}
\gppoint{gp mark 1}{(7.734,2.017)}
\gppoint{gp mark 1}{(7.736,2.030)}
\gppoint{gp mark 1}{(7.738,2.011)}
\gppoint{gp mark 1}{(7.741,2.022)}
\gppoint{gp mark 1}{(7.743,1.998)}
\gppoint{gp mark 1}{(7.745,2.016)}
\gppoint{gp mark 1}{(7.747,1.984)}
\gppoint{gp mark 1}{(7.750,1.991)}
\gppoint{gp mark 1}{(7.752,2.003)}
\gppoint{gp mark 1}{(7.754,1.991)}
\gppoint{gp mark 1}{(7.756,1.984)}
\gppoint{gp mark 1}{(7.759,1.963)}
\gppoint{gp mark 1}{(7.761,1.984)}
\gppoint{gp mark 1}{(7.763,1.978)}
\gppoint{gp mark 1}{(7.765,1.972)}
\gppoint{gp mark 1}{(7.768,1.950)}
\gppoint{gp mark 1}{(7.770,1.958)}
\gppoint{gp mark 1}{(7.772,1.951)}
\gppoint{gp mark 1}{(7.774,1.948)}
\gppoint{gp mark 1}{(7.777,1.960)}
\gppoint{gp mark 1}{(7.779,1.958)}
\gppoint{gp mark 1}{(7.781,1.960)}
\gppoint{gp mark 1}{(7.783,1.955)}
\gppoint{gp mark 1}{(7.786,1.958)}
\gppoint{gp mark 1}{(7.788,1.921)}
\gppoint{gp mark 1}{(7.790,1.930)}
\gppoint{gp mark 1}{(7.792,1.923)}
\gppoint{gp mark 1}{(7.795,1.906)}
\gppoint{gp mark 1}{(7.797,1.918)}
\gppoint{gp mark 1}{(7.799,1.920)}
\gppoint{gp mark 1}{(7.801,1.911)}
\gppoint{gp mark 1}{(7.804,1.912)}
\gppoint{gp mark 1}{(7.806,1.896)}
\gppoint{gp mark 1}{(7.808,1.898)}
\gppoint{gp mark 1}{(7.810,1.896)}
\gppoint{gp mark 1}{(7.813,1.878)}
\gppoint{gp mark 1}{(7.815,1.896)}
\gppoint{gp mark 1}{(7.817,1.880)}
\gppoint{gp mark 1}{(7.819,1.874)}
\gppoint{gp mark 1}{(7.822,1.878)}
\gppoint{gp mark 1}{(7.824,1.877)}
\gppoint{gp mark 1}{(7.826,1.856)}
\gppoint{gp mark 1}{(7.828,1.858)}
\gppoint{gp mark 1}{(7.831,1.873)}
\gppoint{gp mark 1}{(7.833,1.852)}
\gppoint{gp mark 1}{(7.835,1.828)}
\gppoint{gp mark 1}{(7.837,1.851)}
\gppoint{gp mark 1}{(7.840,1.853)}
\gppoint{gp mark 1}{(7.842,1.859)}
\gppoint{gp mark 1}{(7.844,1.856)}
\gppoint{gp mark 1}{(7.846,1.842)}
\gppoint{gp mark 1}{(7.849,1.821)}
\gppoint{gp mark 1}{(7.851,1.821)}
\gppoint{gp mark 1}{(7.853,1.824)}
\gppoint{gp mark 1}{(7.855,1.809)}
\gppoint{gp mark 1}{(7.858,1.829)}
\gppoint{gp mark 1}{(7.860,1.834)}
\gppoint{gp mark 1}{(7.862,1.811)}
\gppoint{gp mark 1}{(7.864,1.806)}
\gppoint{gp mark 1}{(7.867,1.816)}
\gppoint{gp mark 1}{(7.869,1.805)}
\gppoint{gp mark 1}{(7.871,1.807)}
\gppoint{gp mark 1}{(7.873,1.807)}
\gppoint{gp mark 1}{(7.876,1.778)}
\gppoint{gp mark 1}{(7.878,1.799)}
\gppoint{gp mark 1}{(7.880,1.804)}
\gppoint{gp mark 1}{(7.882,1.787)}
\gppoint{gp mark 1}{(7.885,1.790)}
\gppoint{gp mark 1}{(7.887,1.815)}
\gppoint{gp mark 1}{(7.889,1.781)}
\gppoint{gp mark 1}{(7.891,1.780)}
\gppoint{gp mark 1}{(7.894,1.779)}
\gppoint{gp mark 1}{(7.896,1.764)}
\gppoint{gp mark 1}{(7.898,1.769)}
\gppoint{gp mark 1}{(7.900,1.765)}
\gppoint{gp mark 1}{(7.903,1.777)}
\gppoint{gp mark 1}{(7.905,1.780)}
\gppoint{gp mark 1}{(7.907,1.788)}
\gppoint{gp mark 1}{(7.909,1.744)}
\gppoint{gp mark 1}{(7.912,1.769)}
\gppoint{gp mark 1}{(7.914,1.765)}
\gppoint{gp mark 1}{(7.916,1.756)}
\gppoint{gp mark 1}{(7.918,1.758)}
\gppoint{gp mark 1}{(7.921,1.754)}
\gppoint{gp mark 1}{(7.923,1.753)}
\gppoint{gp mark 1}{(7.925,1.762)}
\gppoint{gp mark 1}{(7.927,1.762)}
\gppoint{gp mark 1}{(7.930,1.754)}
\gppoint{gp mark 1}{(7.932,1.770)}
\gppoint{gp mark 1}{(7.934,1.753)}
\gppoint{gp mark 1}{(7.936,1.735)}
\gppoint{gp mark 1}{(7.939,1.735)}
\gppoint{gp mark 1}{(7.941,1.733)}
\gppoint{gp mark 1}{(7.943,1.734)}
\gppoint{gp mark 1}{(7.945,1.735)}
\gppoint{gp mark 1}{(7.948,1.737)}
\gppoint{gp mark 1}{(7.950,1.726)}
\gppoint{gp mark 1}{(7.952,1.726)}
\gppoint{gp mark 1}{(7.954,1.720)}
\gppoint{gp mark 1}{(7.957,1.726)}
\gppoint{gp mark 1}{(7.959,1.736)}
\gppoint{gp mark 1}{(7.961,1.708)}
\gppoint{gp mark 1}{(7.963,1.714)}
\gppoint{gp mark 1}{(7.966,1.715)}
\gppoint{gp mark 1}{(7.968,1.696)}
\gppoint{gp mark 1}{(7.970,1.707)}
\gppoint{gp mark 1}{(7.972,1.692)}
\gppoint{gp mark 1}{(7.975,1.708)}
\gppoint{gp mark 1}{(7.977,1.703)}
\gppoint{gp mark 1}{(7.979,1.698)}
\gppoint{gp mark 1}{(7.981,1.701)}
\gppoint{gp mark 1}{(7.984,1.687)}
\gppoint{gp mark 1}{(7.986,1.700)}
\gppoint{gp mark 1}{(7.988,1.675)}
\gppoint{gp mark 1}{(7.990,1.701)}
\gppoint{gp mark 1}{(7.993,1.670)}
\gppoint{gp mark 1}{(7.995,1.677)}
\gppoint{gp mark 1}{(7.997,1.683)}
\gppoint{gp mark 1}{(7.999,1.682)}
\gppoint{gp mark 1}{(8.002,1.670)}
\gppoint{gp mark 1}{(8.004,1.658)}
\gppoint{gp mark 1}{(8.006,1.672)}
\gppoint{gp mark 1}{(8.008,1.666)}
\gppoint{gp mark 1}{(8.011,1.649)}
\gppoint{gp mark 1}{(8.013,1.653)}
\gppoint{gp mark 1}{(8.015,1.666)}
\gppoint{gp mark 1}{(8.017,1.652)}
\gppoint{gp mark 1}{(8.020,1.663)}
\gppoint{gp mark 1}{(8.022,1.638)}
\gppoint{gp mark 1}{(8.024,1.648)}
\gppoint{gp mark 1}{(8.026,1.673)}
\gppoint{gp mark 1}{(8.029,1.657)}
\gppoint{gp mark 1}{(8.031,1.648)}
\gppoint{gp mark 1}{(8.033,1.633)}
\gppoint{gp mark 1}{(8.035,1.644)}
\gppoint{gp mark 1}{(8.038,1.646)}
\gppoint{gp mark 1}{(8.040,1.642)}
\gppoint{gp mark 1}{(8.042,1.637)}
\gppoint{gp mark 1}{(8.044,1.632)}
\gppoint{gp mark 1}{(8.047,1.638)}
\gppoint{gp mark 1}{(8.049,1.637)}
\gppoint{gp mark 1}{(8.051,1.614)}
\gppoint{gp mark 1}{(8.053,1.631)}
\gppoint{gp mark 1}{(8.056,1.630)}
\gppoint{gp mark 1}{(8.058,1.620)}
\gppoint{gp mark 1}{(8.060,1.640)}
\gppoint{gp mark 1}{(8.062,1.601)}
\gppoint{gp mark 1}{(8.065,1.611)}
\gppoint{gp mark 1}{(8.067,1.596)}
\gppoint{gp mark 1}{(8.069,1.602)}
\gppoint{gp mark 1}{(8.071,1.608)}
\gppoint{gp mark 1}{(8.074,1.628)}
\gppoint{gp mark 1}{(8.076,1.619)}
\gppoint{gp mark 1}{(8.078,1.598)}
\gppoint{gp mark 1}{(8.080,1.609)}
\gppoint{gp mark 1}{(8.083,1.592)}
\gppoint{gp mark 1}{(8.085,1.592)}
\gppoint{gp mark 1}{(8.087,1.593)}
\gppoint{gp mark 1}{(8.089,1.603)}
\gppoint{gp mark 1}{(8.092,1.589)}
\gppoint{gp mark 1}{(8.094,1.588)}
\gppoint{gp mark 1}{(8.096,1.593)}
\gppoint{gp mark 1}{(8.098,1.591)}
\gppoint{gp mark 1}{(8.101,1.590)}
\gppoint{gp mark 1}{(8.103,1.589)}
\gppoint{gp mark 1}{(8.105,1.594)}
\gppoint{gp mark 1}{(8.107,1.586)}
\gppoint{gp mark 1}{(8.110,1.568)}
\gppoint{gp mark 1}{(8.112,1.571)}
\gppoint{gp mark 1}{(8.114,1.559)}
\gppoint{gp mark 1}{(8.116,1.564)}
\gppoint{gp mark 1}{(8.119,1.563)}
\gppoint{gp mark 1}{(8.121,1.555)}
\gppoint{gp mark 1}{(8.123,1.559)}
\gppoint{gp mark 1}{(8.125,1.579)}
\gppoint{gp mark 1}{(8.128,1.552)}
\gppoint{gp mark 1}{(8.130,1.557)}
\gppoint{gp mark 1}{(8.132,1.584)}
\gppoint{gp mark 1}{(8.134,1.571)}
\gppoint{gp mark 1}{(8.137,1.550)}
\gppoint{gp mark 1}{(8.139,1.566)}
\gppoint{gp mark 1}{(8.141,1.567)}
\gppoint{gp mark 1}{(8.143,1.562)}
\gppoint{gp mark 1}{(8.146,1.562)}
\gppoint{gp mark 1}{(8.148,1.561)}
\gppoint{gp mark 1}{(8.150,1.534)}
\gppoint{gp mark 1}{(8.152,1.559)}
\gppoint{gp mark 1}{(8.155,1.528)}
\gppoint{gp mark 1}{(8.157,1.545)}
\gppoint{gp mark 1}{(8.159,1.543)}
\gppoint{gp mark 1}{(8.161,1.534)}
\gppoint{gp mark 1}{(8.164,1.552)}
\gppoint{gp mark 1}{(8.166,1.536)}
\gppoint{gp mark 1}{(8.168,1.550)}
\gppoint{gp mark 1}{(8.170,1.530)}
\gppoint{gp mark 1}{(8.173,1.518)}
\gppoint{gp mark 1}{(8.175,1.533)}
\gppoint{gp mark 1}{(8.177,1.526)}
\gppoint{gp mark 1}{(8.179,1.521)}
\gppoint{gp mark 1}{(8.182,1.509)}
\gppoint{gp mark 1}{(8.184,1.545)}
\gppoint{gp mark 1}{(8.186,1.502)}
\gppoint{gp mark 1}{(8.188,1.519)}
\gppoint{gp mark 1}{(8.191,1.510)}
\gppoint{gp mark 1}{(8.193,1.517)}
\gppoint{gp mark 1}{(8.195,1.522)}
\gppoint{gp mark 1}{(8.197,1.508)}
\gppoint{gp mark 1}{(8.200,1.510)}
\gppoint{gp mark 1}{(8.202,1.501)}
\gppoint{gp mark 1}{(8.204,1.500)}
\gppoint{gp mark 1}{(8.206,1.495)}
\gppoint{gp mark 1}{(8.209,1.499)}
\gppoint{gp mark 1}{(8.211,1.509)}
\gppoint{gp mark 1}{(8.213,1.488)}
\gppoint{gp mark 1}{(8.215,1.483)}
\gppoint{gp mark 1}{(8.218,1.484)}
\gppoint{gp mark 1}{(8.220,1.498)}
\gppoint{gp mark 1}{(8.222,1.502)}
\gppoint{gp mark 1}{(8.224,1.499)}
\gppoint{gp mark 1}{(8.227,1.476)}
\gppoint{gp mark 1}{(8.229,1.494)}
\gppoint{gp mark 1}{(8.231,1.500)}
\gppoint{gp mark 1}{(8.233,1.501)}
\gppoint{gp mark 1}{(8.236,1.496)}
\gppoint{gp mark 1}{(8.238,1.492)}
\gppoint{gp mark 1}{(8.240,1.483)}
\gppoint{gp mark 1}{(8.242,1.493)}
\gppoint{gp mark 1}{(8.245,1.486)}
\gppoint{gp mark 1}{(8.247,1.472)}
\gppoint{gp mark 1}{(8.249,1.484)}
\gppoint{gp mark 1}{(8.251,1.485)}
\gppoint{gp mark 1}{(8.254,1.482)}
\gppoint{gp mark 1}{(8.256,1.486)}
\gppoint{gp mark 1}{(8.258,1.473)}
\gppoint{gp mark 1}{(8.260,1.469)}
\gppoint{gp mark 1}{(8.263,1.459)}
\gppoint{gp mark 1}{(8.265,1.466)}
\gppoint{gp mark 1}{(8.267,1.471)}
\gppoint{gp mark 1}{(8.269,1.483)}
\gppoint{gp mark 1}{(8.272,1.464)}
\gppoint{gp mark 1}{(8.274,1.464)}
\gppoint{gp mark 1}{(8.276,1.450)}
\gppoint{gp mark 1}{(8.278,1.473)}
\gppoint{gp mark 1}{(8.281,1.452)}
\gppoint{gp mark 1}{(8.283,1.469)}
\gppoint{gp mark 1}{(8.285,1.460)}
\gppoint{gp mark 1}{(8.287,1.446)}
\gppoint{gp mark 1}{(8.290,1.464)}
\gppoint{gp mark 1}{(8.292,1.450)}
\gppoint{gp mark 1}{(8.294,1.446)}
\gppoint{gp mark 1}{(8.296,1.469)}
\gppoint{gp mark 1}{(8.299,1.459)}
\gppoint{gp mark 1}{(8.301,1.438)}
\gppoint{gp mark 1}{(8.303,1.475)}
\gppoint{gp mark 1}{(8.305,1.453)}
\gppoint{gp mark 1}{(8.308,1.453)}
\gppoint{gp mark 1}{(8.310,1.439)}
\gppoint{gp mark 1}{(8.312,1.456)}
\gppoint{gp mark 1}{(8.314,1.458)}
\gppoint{gp mark 1}{(8.317,1.454)}
\gppoint{gp mark 1}{(8.319,1.443)}
\gppoint{gp mark 1}{(8.321,1.450)}
\gppoint{gp mark 1}{(8.323,1.440)}
\gppoint{gp mark 1}{(8.326,1.437)}
\gppoint{gp mark 1}{(8.328,1.446)}
\gppoint{gp mark 1}{(8.330,1.431)}
\gppoint{gp mark 1}{(8.332,1.442)}
\gppoint{gp mark 1}{(8.335,1.436)}
\gppoint{gp mark 1}{(8.337,1.434)}
\gppoint{gp mark 1}{(8.339,1.445)}
\gppoint{gp mark 1}{(8.341,1.447)}
\gppoint{gp mark 1}{(8.344,1.427)}
\gppoint{gp mark 1}{(8.346,1.438)}
\gppoint{gp mark 1}{(8.348,1.463)}
\gppoint{gp mark 1}{(8.350,1.440)}
\gppoint{gp mark 1}{(8.353,1.428)}
\gppoint{gp mark 1}{(8.355,1.429)}
\gppoint{gp mark 1}{(8.357,1.420)}
\gppoint{gp mark 1}{(8.359,1.434)}
\gppoint{gp mark 1}{(8.362,1.430)}
\gppoint{gp mark 1}{(8.364,1.422)}
\gppoint{gp mark 1}{(8.366,1.432)}
\gppoint{gp mark 1}{(8.368,1.428)}
\gppoint{gp mark 1}{(8.371,1.426)}
\gppoint{gp mark 1}{(8.373,1.417)}
\gppoint{gp mark 1}{(8.375,1.446)}
\gppoint{gp mark 1}{(8.377,1.438)}
\gppoint{gp mark 1}{(8.380,1.434)}
\gppoint{gp mark 1}{(8.382,1.438)}
\gppoint{gp mark 1}{(8.384,1.433)}
\gppoint{gp mark 1}{(8.386,1.431)}
\gppoint{gp mark 1}{(8.389,1.426)}
\gppoint{gp mark 1}{(8.391,1.434)}
\gppoint{gp mark 1}{(8.393,1.444)}
\gppoint{gp mark 1}{(8.395,1.438)}
\gppoint{gp mark 1}{(8.398,1.424)}
\gppoint{gp mark 1}{(8.400,1.441)}
\gppoint{gp mark 1}{(8.402,1.426)}
\gppoint{gp mark 1}{(8.404,1.419)}
\gppoint{gp mark 1}{(8.407,1.439)}
\gppoint{gp mark 1}{(8.409,1.429)}
\gppoint{gp mark 1}{(8.411,1.419)}
\gppoint{gp mark 1}{(8.413,1.414)}
\gppoint{gp mark 1}{(8.416,1.432)}
\gppoint{gp mark 1}{(8.418,1.433)}
\gppoint{gp mark 1}{(8.420,1.433)}
\gppoint{gp mark 1}{(8.422,1.429)}
\gppoint{gp mark 1}{(8.425,1.414)}
\gppoint{gp mark 1}{(8.427,1.418)}
\gppoint{gp mark 1}{(8.429,1.408)}
\gppoint{gp mark 1}{(8.431,1.417)}
\gppoint{gp mark 1}{(8.434,1.416)}
\gppoint{gp mark 1}{(8.436,1.423)}
\gppoint{gp mark 1}{(8.438,1.404)}
\gppoint{gp mark 1}{(8.440,1.422)}
\gppoint{gp mark 1}{(8.443,1.422)}
\gppoint{gp mark 1}{(8.445,1.431)}
\gppoint{gp mark 1}{(8.447,1.415)}
\gppoint{gp mark 1}{(8.449,1.420)}
\gppoint{gp mark 1}{(8.452,1.403)}
\gppoint{gp mark 1}{(8.454,1.428)}
\gppoint{gp mark 1}{(8.456,1.414)}
\gppoint{gp mark 1}{(8.458,1.425)}
\gppoint{gp mark 1}{(8.461,1.423)}
\gppoint{gp mark 1}{(8.463,1.419)}
\gppoint{gp mark 1}{(8.465,1.422)}
\gppoint{gp mark 1}{(8.467,1.423)}
\gppoint{gp mark 1}{(8.470,1.414)}
\gppoint{gp mark 1}{(8.472,1.420)}
\gppoint{gp mark 1}{(8.474,1.435)}
\gppoint{gp mark 1}{(8.476,1.418)}
\gppoint{gp mark 1}{(8.479,1.401)}
\gppoint{gp mark 1}{(8.481,1.421)}
\gppoint{gp mark 1}{(8.483,1.413)}
\gppoint{gp mark 1}{(8.485,1.414)}
\gppoint{gp mark 1}{(8.488,1.422)}
\gppoint{gp mark 1}{(8.490,1.410)}
\gppoint{gp mark 1}{(8.492,1.412)}
\gppoint{gp mark 1}{(8.494,1.417)}
\gppoint{gp mark 1}{(8.497,1.411)}
\gppoint{gp mark 1}{(8.499,1.409)}
\gppoint{gp mark 1}{(8.501,1.398)}
\gppoint{gp mark 1}{(8.503,1.390)}
\gppoint{gp mark 1}{(8.506,1.420)}
\gppoint{gp mark 1}{(8.508,1.419)}
\gppoint{gp mark 1}{(8.510,1.430)}
\gppoint{gp mark 1}{(8.512,1.407)}
\gppoint{gp mark 1}{(8.515,1.420)}
\gppoint{gp mark 1}{(8.517,1.419)}
\gppoint{gp mark 1}{(8.519,1.394)}
\gppoint{gp mark 1}{(8.521,1.410)}
\gppoint{gp mark 1}{(8.524,1.417)}
\gppoint{gp mark 1}{(8.526,1.404)}
\gppoint{gp mark 1}{(8.528,1.404)}
\gppoint{gp mark 1}{(8.530,1.412)}
\gppoint{gp mark 1}{(8.533,1.400)}
\gppoint{gp mark 1}{(8.535,1.405)}
\gppoint{gp mark 1}{(8.537,1.411)}
\gppoint{gp mark 1}{(8.539,1.402)}
\gppoint{gp mark 1}{(8.542,1.410)}
\gppoint{gp mark 1}{(8.544,1.424)}
\gppoint{gp mark 1}{(8.546,1.402)}
\gppoint{gp mark 1}{(8.548,1.394)}
\gppoint{gp mark 1}{(8.551,1.411)}
\gppoint{gp mark 1}{(8.553,1.406)}
\gppoint{gp mark 1}{(8.555,1.426)}
\gppoint{gp mark 1}{(8.557,1.411)}
\gppoint{gp mark 1}{(8.560,1.422)}
\gppoint{gp mark 1}{(8.562,1.406)}
\gppoint{gp mark 1}{(8.564,1.395)}
\gppoint{gp mark 1}{(8.566,1.408)}
\gppoint{gp mark 1}{(8.569,1.417)}
\gppoint{gp mark 1}{(8.571,1.399)}
\gppoint{gp mark 1}{(8.573,1.410)}
\gppoint{gp mark 1}{(8.575,1.413)}
\gppoint{gp mark 1}{(8.578,1.409)}
\gppoint{gp mark 1}{(8.580,1.427)}
\gppoint{gp mark 1}{(8.582,1.424)}
\gppoint{gp mark 1}{(8.584,1.398)}
\gppoint{gp mark 1}{(8.587,1.404)}
\gppoint{gp mark 1}{(8.589,1.419)}
\gppoint{gp mark 1}{(8.591,1.411)}
\gppoint{gp mark 1}{(8.593,1.407)}
\gppoint{gp mark 1}{(8.596,1.411)}
\gppoint{gp mark 1}{(8.598,1.412)}
\gppoint{gp mark 1}{(8.600,1.408)}
\gppoint{gp mark 1}{(8.602,1.406)}
\gppoint{gp mark 1}{(8.605,1.393)}
\gppoint{gp mark 1}{(8.607,1.408)}
\gppoint{gp mark 1}{(8.609,1.424)}
\gppoint{gp mark 1}{(8.611,1.422)}
\gppoint{gp mark 1}{(8.614,1.393)}
\gppoint{gp mark 1}{(8.616,1.407)}
\gppoint{gp mark 1}{(8.618,1.403)}
\gppoint{gp mark 1}{(8.620,1.403)}
\gppoint{gp mark 1}{(8.623,1.408)}
\gppoint{gp mark 1}{(8.625,1.401)}
\gppoint{gp mark 1}{(8.627,1.398)}
\gppoint{gp mark 1}{(8.629,1.407)}
\gppoint{gp mark 1}{(8.632,1.393)}
\gppoint{gp mark 1}{(8.634,1.397)}
\gppoint{gp mark 1}{(8.636,1.401)}
\gppoint{gp mark 1}{(8.638,1.405)}
\gppoint{gp mark 1}{(8.641,1.406)}
\gppoint{gp mark 1}{(8.643,1.393)}
\gppoint{gp mark 1}{(8.645,1.417)}
\gppoint{gp mark 1}{(8.647,1.391)}
\gppoint{gp mark 1}{(8.650,1.405)}
\gppoint{gp mark 1}{(8.652,1.391)}
\gppoint{gp mark 1}{(8.654,1.403)}
\gppoint{gp mark 1}{(8.656,1.411)}
\gppoint{gp mark 1}{(8.659,1.415)}
\gppoint{gp mark 1}{(8.661,1.405)}
\gppoint{gp mark 1}{(8.663,1.402)}
\gppoint{gp mark 1}{(8.665,1.418)}
\gppoint{gp mark 1}{(8.668,1.423)}
\gppoint{gp mark 1}{(8.670,1.404)}
\gppoint{gp mark 1}{(8.672,1.382)}
\gppoint{gp mark 1}{(8.674,1.413)}
\gppoint{gp mark 1}{(8.677,1.405)}
\gppoint{gp mark 1}{(8.679,1.417)}
\gppoint{gp mark 1}{(8.681,1.411)}
\gppoint{gp mark 1}{(8.683,1.421)}
\gppoint{gp mark 1}{(8.686,1.399)}
\gppoint{gp mark 1}{(8.688,1.415)}
\gppoint{gp mark 1}{(8.690,1.396)}
\gppoint{gp mark 1}{(8.692,1.390)}
\gppoint{gp mark 1}{(8.695,1.396)}
\gppoint{gp mark 1}{(8.697,1.427)}
\gppoint{gp mark 1}{(8.699,1.403)}
\gppoint{gp mark 1}{(8.701,1.421)}
\gppoint{gp mark 1}{(8.704,1.399)}
\gppoint{gp mark 1}{(8.706,1.408)}
\gppoint{gp mark 1}{(8.708,1.398)}
\gppoint{gp mark 1}{(8.710,1.409)}
\gppoint{gp mark 1}{(8.713,1.401)}
\gppoint{gp mark 1}{(8.715,1.410)}
\gppoint{gp mark 1}{(8.717,1.410)}
\gppoint{gp mark 1}{(8.719,1.407)}
\gppoint{gp mark 1}{(8.722,1.411)}
\gppoint{gp mark 1}{(8.724,1.412)}
\gppoint{gp mark 1}{(8.726,1.409)}
\gppoint{gp mark 1}{(8.728,1.400)}
\gppoint{gp mark 1}{(8.731,1.426)}
\gppoint{gp mark 1}{(8.733,1.412)}
\gppoint{gp mark 1}{(8.735,1.401)}
\gppoint{gp mark 1}{(8.737,1.404)}
\gppoint{gp mark 1}{(8.740,1.401)}
\gppoint{gp mark 1}{(8.742,1.427)}
\gppoint{gp mark 1}{(8.744,1.409)}
\gppoint{gp mark 1}{(8.746,1.417)}
\gppoint{gp mark 1}{(8.749,1.410)}
\gppoint{gp mark 1}{(8.751,1.409)}
\gppoint{gp mark 1}{(8.753,1.412)}
\gppoint{gp mark 1}{(8.755,1.399)}
\gppoint{gp mark 1}{(8.758,1.429)}
\gppoint{gp mark 1}{(8.760,1.430)}
\gppoint{gp mark 1}{(8.762,1.416)}
\gppoint{gp mark 1}{(8.764,1.408)}
\gppoint{gp mark 1}{(8.767,1.407)}
\gppoint{gp mark 1}{(8.769,1.414)}
\gppoint{gp mark 1}{(8.771,1.433)}
\gppoint{gp mark 1}{(8.773,1.394)}
\gppoint{gp mark 1}{(8.776,1.415)}
\gppoint{gp mark 1}{(8.778,1.408)}
\gppoint{gp mark 1}{(8.780,1.404)}
\gppoint{gp mark 1}{(8.782,1.420)}
\gppoint{gp mark 1}{(8.785,1.393)}
\gppoint{gp mark 1}{(8.787,1.413)}
\gppoint{gp mark 1}{(8.789,1.413)}
\gppoint{gp mark 1}{(8.791,1.412)}
\gppoint{gp mark 1}{(8.794,1.405)}
\gppoint{gp mark 1}{(8.796,1.398)}
\gppoint{gp mark 1}{(8.798,1.428)}
\gppoint{gp mark 1}{(8.800,1.421)}
\gppoint{gp mark 1}{(8.803,1.420)}
\gppoint{gp mark 1}{(8.805,1.413)}
\gppoint{gp mark 1}{(8.807,1.409)}
\gppoint{gp mark 1}{(8.809,1.424)}
\gppoint{gp mark 1}{(8.812,1.428)}
\gppoint{gp mark 1}{(8.814,1.406)}
\gppoint{gp mark 1}{(8.816,1.399)}
\gppoint{gp mark 1}{(8.818,1.425)}
\gppoint{gp mark 1}{(8.821,1.417)}
\gppoint{gp mark 1}{(8.823,1.413)}
\gppoint{gp mark 1}{(8.825,1.414)}
\gppoint{gp mark 1}{(8.827,1.412)}
\gppoint{gp mark 1}{(8.830,1.415)}
\gppoint{gp mark 1}{(8.832,1.416)}
\gppoint{gp mark 1}{(8.834,1.402)}
\gppoint{gp mark 1}{(8.836,1.418)}
\gppoint{gp mark 1}{(8.839,1.427)}
\gppoint{gp mark 1}{(8.841,1.420)}
\gppoint{gp mark 1}{(8.843,1.426)}
\gppoint{gp mark 1}{(8.845,1.400)}
\gppoint{gp mark 1}{(8.848,1.410)}
\gppoint{gp mark 1}{(8.850,1.425)}
\gppoint{gp mark 1}{(8.852,1.425)}
\gppoint{gp mark 1}{(8.854,1.419)}
\gppoint{gp mark 1}{(8.857,1.431)}
\gppoint{gp mark 1}{(8.859,1.409)}
\gppoint{gp mark 1}{(8.861,1.431)}
\gppoint{gp mark 1}{(8.863,1.409)}
\gppoint{gp mark 1}{(8.866,1.413)}
\gppoint{gp mark 1}{(8.868,1.422)}
\gppoint{gp mark 1}{(8.870,1.412)}
\gppoint{gp mark 1}{(8.872,1.415)}
\gppoint{gp mark 1}{(8.875,1.433)}
\gppoint{gp mark 1}{(8.877,1.412)}
\gppoint{gp mark 1}{(8.879,1.425)}
\gppoint{gp mark 1}{(8.881,1.407)}
\gppoint{gp mark 1}{(8.884,1.398)}
\gppoint{gp mark 1}{(8.886,1.407)}
\gppoint{gp mark 1}{(8.888,1.414)}
\gppoint{gp mark 1}{(8.890,1.422)}
\gppoint{gp mark 1}{(8.893,1.412)}
\gppoint{gp mark 1}{(8.895,1.409)}
\gppoint{gp mark 1}{(8.897,1.415)}
\gppoint{gp mark 1}{(8.899,1.423)}
\gppoint{gp mark 1}{(8.902,1.417)}
\gppoint{gp mark 1}{(8.904,1.420)}
\gppoint{gp mark 1}{(8.906,1.416)}
\gppoint{gp mark 1}{(8.908,1.428)}
\gppoint{gp mark 1}{(8.911,1.413)}
\gppoint{gp mark 1}{(8.913,1.423)}
\gppoint{gp mark 1}{(8.915,1.421)}
\gppoint{gp mark 1}{(8.918,1.426)}
\gppoint{gp mark 1}{(8.920,1.423)}
\gppoint{gp mark 1}{(8.922,1.412)}
\gppoint{gp mark 1}{(8.924,1.403)}
\gppoint{gp mark 1}{(8.927,1.406)}
\gppoint{gp mark 1}{(8.929,1.422)}
\gppoint{gp mark 1}{(8.931,1.416)}
\gppoint{gp mark 1}{(8.933,1.436)}
\gppoint{gp mark 1}{(8.936,1.442)}
\gppoint{gp mark 1}{(8.938,1.426)}
\gppoint{gp mark 1}{(8.940,1.420)}
\gppoint{gp mark 1}{(8.942,1.441)}
\gppoint{gp mark 1}{(8.945,1.418)}
\gppoint{gp mark 1}{(8.947,1.426)}
\gppoint{gp mark 1}{(8.949,1.418)}
\gppoint{gp mark 1}{(8.951,1.413)}
\gppoint{gp mark 1}{(8.954,1.411)}
\gppoint{gp mark 1}{(8.956,1.424)}
\gppoint{gp mark 1}{(8.958,1.414)}
\gppoint{gp mark 1}{(8.960,1.424)}
\gppoint{gp mark 1}{(8.963,1.412)}
\gppoint{gp mark 1}{(8.965,1.416)}
\gppoint{gp mark 1}{(8.967,1.432)}
\gppoint{gp mark 1}{(8.969,1.440)}
\gppoint{gp mark 1}{(8.972,1.418)}
\gppoint{gp mark 1}{(8.974,1.423)}
\gppoint{gp mark 1}{(8.976,1.431)}
\gppoint{gp mark 1}{(8.978,1.436)}
\gppoint{gp mark 1}{(8.981,1.437)}
\gppoint{gp mark 1}{(8.983,1.423)}
\gppoint{gp mark 1}{(8.985,1.423)}
\gppoint{gp mark 1}{(8.987,1.436)}
\gppoint{gp mark 1}{(8.990,1.417)}
\gppoint{gp mark 1}{(8.992,1.419)}
\gppoint{gp mark 1}{(8.994,1.435)}
\gppoint{gp mark 1}{(8.996,1.412)}
\gppoint{gp mark 1}{(8.999,1.419)}
\gppoint{gp mark 1}{(9.001,1.416)}
\gppoint{gp mark 1}{(9.003,1.431)}
\gppoint{gp mark 1}{(9.005,1.418)}
\gppoint{gp mark 1}{(9.008,1.420)}
\gppoint{gp mark 1}{(9.010,1.421)}
\gppoint{gp mark 1}{(9.012,1.435)}
\gppoint{gp mark 1}{(9.014,1.421)}
\gppoint{gp mark 1}{(9.017,1.420)}
\gppoint{gp mark 1}{(9.019,1.427)}
\gppoint{gp mark 1}{(9.021,1.406)}
\gppoint{gp mark 1}{(9.023,1.423)}
\gppoint{gp mark 1}{(9.026,1.439)}
\gppoint{gp mark 1}{(9.028,1.412)}
\gppoint{gp mark 1}{(9.030,1.417)}
\gppoint{gp mark 1}{(9.032,1.429)}
\gppoint{gp mark 1}{(9.035,1.436)}
\gppoint{gp mark 1}{(9.037,1.433)}
\gppoint{gp mark 1}{(9.039,1.414)}
\gppoint{gp mark 1}{(9.041,1.390)}
\gppoint{gp mark 1}{(9.044,1.367)}
\gppoint{gp mark 1}{(9.046,1.356)}
\gppoint{gp mark 1}{(9.048,1.362)}
\gppoint{gp mark 1}{(9.050,1.381)}
\gppoint{gp mark 1}{(9.053,1.363)}
\gppoint{gp mark 1}{(9.055,1.370)}
\gppoint{gp mark 1}{(9.057,1.369)}
\gppoint{gp mark 1}{(9.059,1.362)}
\gppoint{gp mark 1}{(9.062,1.363)}
\gppoint{gp mark 1}{(9.064,1.367)}
\gppoint{gp mark 1}{(9.066,1.355)}
\gppoint{gp mark 1}{(9.068,1.352)}
\gppoint{gp mark 1}{(9.071,1.339)}
\gppoint{gp mark 1}{(9.073,1.354)}
\gppoint{gp mark 1}{(9.075,1.358)}
\gppoint{gp mark 1}{(9.077,1.353)}
\gppoint{gp mark 1}{(9.080,1.376)}
\gppoint{gp mark 1}{(9.082,1.360)}
\gppoint{gp mark 1}{(9.084,1.356)}
\gppoint{gp mark 1}{(9.086,1.346)}
\gppoint{gp mark 1}{(9.089,1.355)}
\gppoint{gp mark 1}{(9.091,1.362)}
\gppoint{gp mark 1}{(9.093,1.375)}
\gppoint{gp mark 1}{(9.095,1.357)}
\gppoint{gp mark 1}{(9.098,1.357)}
\gppoint{gp mark 1}{(9.100,1.368)}
\gppoint{gp mark 1}{(9.102,1.353)}
\gppoint{gp mark 1}{(9.104,1.379)}
\gppoint{gp mark 1}{(9.107,1.361)}
\gppoint{gp mark 1}{(9.109,1.370)}
\gppoint{gp mark 1}{(9.111,1.354)}
\gppoint{gp mark 1}{(9.113,1.360)}
\gppoint{gp mark 1}{(9.116,1.379)}
\gppoint{gp mark 1}{(9.118,1.356)}
\gppoint{gp mark 1}{(9.120,1.349)}
\gppoint{gp mark 1}{(9.122,1.354)}
\gppoint{gp mark 1}{(9.125,1.337)}
\gppoint{gp mark 1}{(9.127,1.361)}
\gppoint{gp mark 1}{(9.129,1.376)}
\gppoint{gp mark 1}{(9.131,1.361)}
\gppoint{gp mark 1}{(9.134,1.362)}
\gppoint{gp mark 1}{(9.136,1.358)}
\gppoint{gp mark 1}{(9.138,1.354)}
\gppoint{gp mark 1}{(9.140,1.348)}
\gppoint{gp mark 1}{(9.143,1.354)}
\gppoint{gp mark 1}{(9.145,1.354)}
\gppoint{gp mark 1}{(9.147,1.343)}
\gppoint{gp mark 1}{(9.149,1.365)}
\gppoint{gp mark 1}{(9.152,1.356)}
\gppoint{gp mark 1}{(9.154,1.354)}
\gppoint{gp mark 1}{(9.156,1.363)}
\gppoint{gp mark 1}{(9.158,1.368)}
\gppoint{gp mark 1}{(9.161,1.365)}
\gppoint{gp mark 1}{(9.163,1.358)}
\gppoint{gp mark 1}{(9.165,1.361)}
\gppoint{gp mark 1}{(9.167,1.365)}
\gppoint{gp mark 1}{(9.170,1.365)}
\gppoint{gp mark 1}{(9.172,1.362)}
\gppoint{gp mark 1}{(9.174,1.355)}
\gppoint{gp mark 1}{(9.176,1.354)}
\gppoint{gp mark 1}{(9.179,1.338)}
\gppoint{gp mark 1}{(9.181,1.342)}
\gppoint{gp mark 1}{(9.183,1.369)}
\gppoint{gp mark 1}{(9.185,1.351)}
\gppoint{gp mark 1}{(9.188,1.360)}
\gppoint{gp mark 1}{(9.190,1.360)}
\gppoint{gp mark 1}{(9.192,1.374)}
\gppoint{gp mark 1}{(9.194,1.336)}
\gppoint{gp mark 1}{(9.197,1.348)}
\gppoint{gp mark 1}{(9.199,1.365)}
\gppoint{gp mark 1}{(9.201,1.365)}
\gppoint{gp mark 1}{(9.203,1.360)}
\gppoint{gp mark 1}{(9.206,1.353)}
\gppoint{gp mark 1}{(9.208,1.390)}
\gppoint{gp mark 1}{(9.210,1.358)}
\gppoint{gp mark 1}{(9.212,1.360)}
\gppoint{gp mark 1}{(9.215,1.350)}
\gppoint{gp mark 1}{(9.217,1.352)}
\gppoint{gp mark 1}{(9.219,1.348)}
\gppoint{gp mark 1}{(9.221,1.362)}
\gppoint{gp mark 1}{(9.224,1.358)}
\gppoint{gp mark 1}{(9.226,1.355)}
\gppoint{gp mark 1}{(9.228,1.362)}
\gppoint{gp mark 1}{(9.230,1.367)}
\gppoint{gp mark 1}{(9.233,1.365)}
\gppoint{gp mark 1}{(9.235,1.344)}
\gppoint{gp mark 1}{(9.237,1.360)}
\gppoint{gp mark 1}{(9.239,1.376)}
\gppoint{gp mark 1}{(9.242,1.352)}
\gppoint{gp mark 1}{(9.244,1.355)}
\gppoint{gp mark 1}{(9.246,1.369)}
\gppoint{gp mark 1}{(9.248,1.369)}
\gppoint{gp mark 1}{(9.251,1.360)}
\gppoint{gp mark 1}{(9.253,1.363)}
\gppoint{gp mark 1}{(9.255,1.363)}
\gppoint{gp mark 1}{(9.257,1.375)}
\gppoint{gp mark 1}{(9.260,1.373)}
\gppoint{gp mark 1}{(9.262,1.375)}
\gppoint{gp mark 1}{(9.264,1.363)}
\gppoint{gp mark 1}{(9.266,1.370)}
\gppoint{gp mark 1}{(9.269,1.366)}
\gppoint{gp mark 1}{(9.271,1.359)}
\gppoint{gp mark 1}{(9.273,1.356)}
\gppoint{gp mark 1}{(9.275,1.371)}
\gppoint{gp mark 1}{(9.278,1.349)}
\gppoint{gp mark 1}{(9.280,1.376)}
\gppoint{gp mark 1}{(9.282,1.378)}
\gppoint{gp mark 1}{(9.284,1.381)}
\gppoint{gp mark 1}{(9.287,1.392)}
\gppoint{gp mark 1}{(9.289,1.385)}
\gppoint{gp mark 1}{(9.291,1.380)}
\gppoint{gp mark 1}{(9.293,1.367)}
\gppoint{gp mark 1}{(9.296,1.364)}
\gppoint{gp mark 1}{(9.298,1.382)}
\gppoint{gp mark 1}{(9.300,1.375)}
\gppoint{gp mark 1}{(9.302,1.386)}
\gppoint{gp mark 1}{(9.305,1.383)}
\gppoint{gp mark 1}{(9.307,1.368)}
\gppoint{gp mark 1}{(9.309,1.375)}
\gppoint{gp mark 1}{(9.311,1.366)}
\gppoint{gp mark 1}{(9.314,1.362)}
\gppoint{gp mark 1}{(9.316,1.381)}
\gppoint{gp mark 1}{(9.318,1.391)}
\gppoint{gp mark 1}{(9.320,1.399)}
\gppoint{gp mark 1}{(9.323,1.370)}
\gppoint{gp mark 1}{(9.325,1.382)}
\gppoint{gp mark 1}{(9.327,1.373)}
\gppoint{gp mark 1}{(9.329,1.369)}
\gppoint{gp mark 1}{(9.332,1.372)}
\gppoint{gp mark 1}{(9.334,1.395)}
\gppoint{gp mark 1}{(9.336,1.381)}
\gppoint{gp mark 1}{(9.338,1.385)}
\gppoint{gp mark 1}{(9.341,1.376)}
\gppoint{gp mark 1}{(9.343,1.384)}
\gppoint{gp mark 1}{(9.345,1.361)}
\gppoint{gp mark 1}{(9.347,1.396)}
\gppoint{gp mark 1}{(9.350,1.376)}
\gppoint{gp mark 1}{(9.352,1.398)}
\gppoint{gp mark 1}{(9.354,1.366)}
\gppoint{gp mark 1}{(9.356,1.389)}
\gppoint{gp mark 1}{(9.359,1.393)}
\gppoint{gp mark 1}{(9.361,1.382)}
\gppoint{gp mark 1}{(9.363,1.395)}
\gppoint{gp mark 1}{(9.365,1.379)}
\gppoint{gp mark 1}{(9.368,1.364)}
\gppoint{gp mark 1}{(9.370,1.371)}
\gppoint{gp mark 1}{(9.372,1.384)}
\gppoint{gp mark 1}{(9.374,1.372)}
\gppoint{gp mark 1}{(9.377,1.372)}
\gppoint{gp mark 1}{(9.379,1.376)}
\gppoint{gp mark 1}{(9.381,1.376)}
\gppoint{gp mark 1}{(9.383,1.361)}
\gppoint{gp mark 1}{(9.386,1.366)}
\gppoint{gp mark 1}{(9.388,1.374)}
\gppoint{gp mark 1}{(9.390,1.375)}
\gppoint{gp mark 1}{(9.392,1.367)}
\gppoint{gp mark 1}{(9.395,1.379)}
\gppoint{gp mark 1}{(9.397,1.372)}
\gppoint{gp mark 1}{(9.399,1.400)}
\gppoint{gp mark 1}{(9.401,1.384)}
\gppoint{gp mark 1}{(9.404,1.390)}
\gppoint{gp mark 1}{(9.406,1.410)}
\gppoint{gp mark 1}{(9.408,1.378)}
\gppoint{gp mark 1}{(9.410,1.398)}
\gppoint{gp mark 1}{(9.413,1.381)}
\gppoint{gp mark 1}{(9.415,1.393)}
\gppoint{gp mark 1}{(9.417,1.392)}
\gppoint{gp mark 1}{(9.419,1.392)}
\gppoint{gp mark 1}{(9.422,1.386)}
\gppoint{gp mark 1}{(9.424,1.381)}
\gppoint{gp mark 1}{(9.426,1.378)}
\gppoint{gp mark 1}{(9.428,1.379)}
\gppoint{gp mark 1}{(9.431,1.383)}
\gppoint{gp mark 1}{(9.433,1.382)}
\gppoint{gp mark 1}{(9.435,1.377)}
\gppoint{gp mark 1}{(9.437,1.386)}
\gppoint{gp mark 1}{(9.440,1.401)}
\gppoint{gp mark 1}{(9.442,1.385)}
\gppoint{gp mark 1}{(9.444,1.387)}
\gppoint{gp mark 1}{(9.446,1.376)}
\gppoint{gp mark 1}{(9.449,1.384)}
\gppoint{gp mark 1}{(9.451,1.385)}
\gppoint{gp mark 1}{(9.453,1.391)}
\gppoint{gp mark 1}{(9.455,1.401)}
\gppoint{gp mark 1}{(9.458,1.397)}
\gppoint{gp mark 1}{(9.460,1.394)}
\gppoint{gp mark 1}{(9.462,1.392)}
\gppoint{gp mark 1}{(9.464,1.375)}
\gppoint{gp mark 1}{(9.467,1.380)}
\gppoint{gp mark 1}{(9.469,3.457)}
\gppoint{gp mark 1}{(9.471,7.013)}
\gppoint{gp mark 1}{(9.473,7.031)}
\gppoint{gp mark 1}{(9.476,7.049)}
\gppoint{gp mark 1}{(9.478,7.049)}
\gppoint{gp mark 1}{(9.480,7.059)}
\gppoint{gp mark 1}{(9.482,7.030)}
\gppoint{gp mark 1}{(9.485,7.035)}
\gppoint{gp mark 1}{(9.487,7.044)}
\gppoint{gp mark 1}{(9.489,7.051)}
\gppoint{gp mark 1}{(9.491,7.047)}
\gppoint{gp mark 1}{(9.494,7.039)}
\gppoint{gp mark 1}{(9.496,7.061)}
\gppoint{gp mark 1}{(9.498,7.076)}
\gppoint{gp mark 1}{(9.500,7.064)}
\gppoint{gp mark 1}{(9.503,7.065)}
\gppoint{gp mark 1}{(9.505,7.069)}
\gppoint{gp mark 1}{(9.507,7.063)}
\gppoint{gp mark 1}{(9.509,7.053)}
\gppoint{gp mark 1}{(9.512,7.073)}
\gppoint{gp mark 1}{(9.514,7.050)}
\gppoint{gp mark 1}{(9.516,7.054)}
\gppoint{gp mark 1}{(9.518,7.052)}
\gppoint{gp mark 1}{(9.521,7.101)}
\gppoint{gp mark 1}{(9.523,7.086)}
\gppoint{gp mark 1}{(9.525,7.077)}
\gppoint{gp mark 1}{(9.527,7.056)}
\gppoint{gp mark 1}{(9.530,7.073)}
\gppoint{gp mark 1}{(9.532,7.086)}
\gppoint{gp mark 1}{(9.534,7.076)}
\gppoint{gp mark 1}{(9.536,7.037)}
\gppoint{gp mark 1}{(9.539,7.076)}
\gppoint{gp mark 1}{(9.541,7.078)}
\gppoint{gp mark 1}{(9.543,7.058)}
\gppoint{gp mark 1}{(9.545,7.049)}
\gppoint{gp mark 1}{(9.548,7.079)}
\gppoint{gp mark 1}{(9.550,7.069)}
\gppoint{gp mark 1}{(9.552,7.084)}
\gppoint{gp mark 1}{(9.554,7.096)}
\gppoint{gp mark 1}{(9.557,7.086)}
\gppoint{gp mark 1}{(9.559,7.072)}
\gppoint{gp mark 1}{(9.561,7.086)}
\gppoint{gp mark 1}{(9.563,7.092)}
\gppoint{gp mark 1}{(9.566,7.087)}
\gppoint{gp mark 1}{(9.568,7.105)}
\gppoint{gp mark 1}{(9.570,7.081)}
\gppoint{gp mark 1}{(9.572,7.082)}
\gppoint{gp mark 1}{(9.575,7.055)}
\gppoint{gp mark 1}{(9.577,7.089)}
\gppoint{gp mark 1}{(9.579,7.081)}
\gppoint{gp mark 1}{(9.581,7.081)}
\gppoint{gp mark 1}{(9.584,7.067)}
\gppoint{gp mark 1}{(9.586,7.087)}
\gppoint{gp mark 1}{(9.588,7.118)}
\gppoint{gp mark 1}{(9.590,7.093)}
\gppoint{gp mark 1}{(9.593,7.094)}
\gppoint{gp mark 1}{(9.595,7.042)}
\gppoint{gp mark 1}{(9.597,7.088)}
\gppoint{gp mark 1}{(9.599,7.094)}
\gppoint{gp mark 1}{(9.602,7.109)}
\gppoint{gp mark 1}{(9.604,7.034)}
\gppoint{gp mark 1}{(9.606,7.072)}
\gppoint{gp mark 1}{(9.608,7.084)}
\gppoint{gp mark 1}{(9.611,7.070)}
\gppoint{gp mark 1}{(9.613,7.099)}
\gppoint{gp mark 1}{(9.615,7.101)}
\gppoint{gp mark 1}{(9.617,7.103)}
\gppoint{gp mark 1}{(9.620,7.105)}
\gppoint{gp mark 1}{(9.622,7.102)}
\gppoint{gp mark 1}{(9.624,7.078)}
\gppoint{gp mark 1}{(9.626,7.088)}
\gppoint{gp mark 1}{(9.629,7.078)}
\gppoint{gp mark 1}{(9.631,7.106)}
\gppoint{gp mark 1}{(9.633,7.062)}
\gppoint{gp mark 1}{(9.635,7.067)}
\gppoint{gp mark 1}{(9.638,7.068)}
\gppoint{gp mark 1}{(9.640,7.093)}
\gppoint{gp mark 1}{(9.642,7.080)}
\gppoint{gp mark 1}{(9.644,7.081)}
\gppoint{gp mark 1}{(9.647,7.098)}
\gppoint{gp mark 1}{(9.649,7.091)}
\gppoint{gp mark 1}{(9.651,7.095)}
\gppoint{gp mark 1}{(9.653,7.052)}
\gppoint{gp mark 1}{(9.656,7.093)}
\gppoint{gp mark 1}{(9.658,7.100)}
\gppoint{gp mark 1}{(9.660,7.072)}
\gppoint{gp mark 1}{(9.662,7.100)}
\gppoint{gp mark 1}{(9.665,7.073)}
\gppoint{gp mark 1}{(9.667,7.097)}
\gppoint{gp mark 1}{(9.669,7.119)}
\gppoint{gp mark 1}{(9.671,7.093)}
\gppoint{gp mark 1}{(9.674,7.078)}
\gppoint{gp mark 1}{(9.676,7.083)}
\gppoint{gp mark 1}{(9.678,7.066)}
\gppoint{gp mark 1}{(9.680,7.125)}
\gppoint{gp mark 1}{(9.683,7.092)}
\gppoint{gp mark 1}{(9.685,7.082)}
\gppoint{gp mark 1}{(9.687,7.065)}
\gppoint{gp mark 1}{(9.689,7.103)}
\gppoint{gp mark 1}{(9.692,7.096)}
\gppoint{gp mark 1}{(9.694,7.052)}
\gppoint{gp mark 1}{(9.696,7.113)}
\gppoint{gp mark 1}{(9.698,7.100)}
\gppoint{gp mark 1}{(9.701,7.112)}
\gppoint{gp mark 1}{(9.703,7.076)}
\gppoint{gp mark 1}{(9.705,7.099)}
\gppoint{gp mark 1}{(9.707,7.113)}
\gppoint{gp mark 1}{(9.710,7.114)}
\gppoint{gp mark 1}{(9.712,7.133)}
\gppoint{gp mark 1}{(9.714,7.106)}
\gppoint{gp mark 1}{(9.716,7.099)}
\gppoint{gp mark 1}{(9.719,7.078)}
\gppoint{gp mark 1}{(9.721,7.078)}
\gppoint{gp mark 1}{(9.723,7.090)}
\gppoint{gp mark 1}{(9.725,7.098)}
\gppoint{gp mark 1}{(9.728,7.078)}
\gppoint{gp mark 1}{(9.730,7.095)}
\gppoint{gp mark 1}{(9.732,7.078)}
\gppoint{gp mark 1}{(9.734,7.119)}
\gppoint{gp mark 1}{(9.737,7.120)}
\gppoint{gp mark 1}{(9.739,7.114)}
\gppoint{gp mark 1}{(9.741,7.116)}
\gppoint{gp mark 1}{(9.743,7.104)}
\gppoint{gp mark 1}{(9.746,7.108)}
\gppoint{gp mark 1}{(9.748,7.089)}
\gppoint{gp mark 1}{(9.750,7.124)}
\gppoint{gp mark 1}{(9.752,7.085)}
\gppoint{gp mark 1}{(9.755,7.110)}
\gppoint{gp mark 1}{(9.757,7.094)}
\gppoint{gp mark 1}{(9.759,7.108)}
\gppoint{gp mark 1}{(9.761,7.115)}
\gppoint{gp mark 1}{(9.764,7.106)}
\gppoint{gp mark 1}{(9.766,7.081)}
\gppoint{gp mark 1}{(9.768,7.124)}
\gppoint{gp mark 1}{(9.770,7.096)}
\gppoint{gp mark 1}{(9.773,7.103)}
\gppoint{gp mark 1}{(9.775,7.089)}
\gppoint{gp mark 1}{(9.777,7.092)}
\gppoint{gp mark 1}{(9.779,7.099)}
\gppoint{gp mark 1}{(9.782,7.117)}
\gppoint{gp mark 1}{(9.784,7.124)}
\gppoint{gp mark 1}{(9.786,7.094)}
\gppoint{gp mark 1}{(9.788,7.112)}
\gppoint{gp mark 1}{(9.791,7.105)}
\gppoint{gp mark 1}{(9.793,7.104)}
\gppoint{gp mark 1}{(9.795,7.141)}
\gppoint{gp mark 1}{(9.797,7.107)}
\gppoint{gp mark 1}{(9.800,7.132)}
\gppoint{gp mark 1}{(9.802,7.119)}
\gppoint{gp mark 1}{(9.804,7.120)}
\gppoint{gp mark 1}{(9.806,7.122)}
\gppoint{gp mark 1}{(9.809,7.094)}
\gppoint{gp mark 1}{(9.811,7.123)}
\gppoint{gp mark 1}{(9.813,7.146)}
\gppoint{gp mark 1}{(9.815,7.130)}
\gppoint{gp mark 1}{(9.818,7.119)}
\gppoint{gp mark 1}{(9.820,7.079)}
\gppoint{gp mark 1}{(9.822,7.124)}
\gppoint{gp mark 1}{(9.824,7.090)}
\gppoint{gp mark 1}{(9.827,7.111)}
\gppoint{gp mark 1}{(9.829,7.120)}
\gppoint{gp mark 1}{(9.831,7.107)}
\gppoint{gp mark 1}{(9.833,7.129)}
\gppoint{gp mark 1}{(9.836,7.105)}
\gppoint{gp mark 1}{(9.838,7.099)}
\gppoint{gp mark 1}{(9.840,7.120)}
\gppoint{gp mark 1}{(9.842,7.087)}
\gppoint{gp mark 1}{(9.845,7.109)}
\gppoint{gp mark 1}{(9.847,7.106)}
\gppoint{gp mark 1}{(9.849,7.105)}
\gppoint{gp mark 1}{(9.851,7.130)}
\gppoint{gp mark 1}{(9.854,7.111)}
\gppoint{gp mark 1}{(9.856,7.152)}
\gppoint{gp mark 1}{(9.858,7.112)}
\gppoint{gp mark 1}{(9.860,7.127)}
\gppoint{gp mark 1}{(9.863,7.098)}
\gppoint{gp mark 1}{(9.865,7.139)}
\gppoint{gp mark 1}{(9.867,7.105)}
\gppoint{gp mark 1}{(9.869,7.130)}
\gppoint{gp mark 1}{(9.872,7.123)}
\gppoint{gp mark 1}{(9.874,7.127)}
\gppoint{gp mark 1}{(9.876,7.113)}
\gppoint{gp mark 1}{(9.878,7.093)}
\gppoint{gp mark 1}{(9.881,7.133)}
\gppoint{gp mark 1}{(9.883,7.103)}
\gppoint{gp mark 1}{(9.885,7.106)}
\gppoint{gp mark 1}{(9.887,7.123)}
\gppoint{gp mark 1}{(9.890,7.152)}
\gppoint{gp mark 1}{(9.892,7.132)}
\gppoint{gp mark 1}{(9.894,7.138)}
\gppoint{gp mark 1}{(9.896,7.141)}
\gppoint{gp mark 1}{(9.899,7.147)}
\gppoint{gp mark 1}{(9.901,7.145)}
\gppoint{gp mark 1}{(9.903,7.133)}
\gppoint{gp mark 1}{(9.905,7.145)}
\gppoint{gp mark 1}{(9.908,7.123)}
\gppoint{gp mark 1}{(9.910,7.135)}
\gppoint{gp mark 1}{(9.912,7.114)}
\gppoint{gp mark 1}{(9.914,7.117)}
\gppoint{gp mark 1}{(9.917,7.122)}
\gppoint{gp mark 1}{(9.919,7.133)}
\gppoint{gp mark 1}{(9.921,7.093)}
\gppoint{gp mark 1}{(9.923,7.096)}
\gppoint{gp mark 1}{(9.926,7.122)}
\gppoint{gp mark 1}{(9.928,7.118)}
\gppoint{gp mark 1}{(9.930,7.138)}
\gppoint{gp mark 1}{(9.932,7.146)}
\gppoint{gp mark 1}{(9.935,7.125)}
\gppoint{gp mark 1}{(9.937,7.108)}
\gppoint{gp mark 1}{(9.939,7.125)}
\gppoint{gp mark 1}{(9.941,7.138)}
\gppoint{gp mark 1}{(9.944,7.145)}
\gppoint{gp mark 1}{(9.946,7.123)}
\gppoint{gp mark 1}{(9.948,7.137)}
\gppoint{gp mark 1}{(9.950,7.126)}
\gppoint{gp mark 1}{(9.953,7.138)}
\gppoint{gp mark 1}{(9.955,7.139)}
\gppoint{gp mark 1}{(9.957,7.141)}
\gppoint{gp mark 1}{(9.959,7.123)}
\gppoint{gp mark 1}{(9.962,7.103)}
\gppoint{gp mark 1}{(9.964,7.124)}
\gppoint{gp mark 1}{(9.966,7.129)}
\gppoint{gp mark 1}{(9.968,7.133)}
\gppoint{gp mark 1}{(9.971,7.134)}
\gppoint{gp mark 1}{(9.973,7.098)}
\gppoint{gp mark 1}{(9.975,7.128)}
\gppoint{gp mark 1}{(9.977,7.096)}
\gppoint{gp mark 1}{(9.980,7.138)}
\gppoint{gp mark 1}{(9.982,7.107)}
\gppoint{gp mark 1}{(9.984,7.108)}
\gppoint{gp mark 1}{(9.986,7.154)}
\gppoint{gp mark 1}{(9.989,7.113)}
\gppoint{gp mark 1}{(9.991,7.139)}
\gppoint{gp mark 1}{(9.993,7.136)}
\gppoint{gp mark 1}{(9.995,7.142)}
\gppoint{gp mark 1}{(9.998,7.136)}
\gppoint{gp mark 1}{(10.000,7.108)}
\gppoint{gp mark 1}{(10.002,7.117)}
\gppoint{gp mark 1}{(10.004,7.157)}
\gppoint{gp mark 1}{(10.007,7.128)}
\gppoint{gp mark 1}{(10.009,7.102)}
\gppoint{gp mark 1}{(10.011,7.129)}
\gppoint{gp mark 1}{(10.013,7.123)}
\gppoint{gp mark 1}{(10.016,7.139)}
\gppoint{gp mark 1}{(10.018,7.124)}
\gppoint{gp mark 1}{(10.020,7.126)}
\gppoint{gp mark 1}{(10.022,7.127)}
\gppoint{gp mark 1}{(10.025,7.129)}
\gppoint{gp mark 1}{(10.027,7.134)}
\gppoint{gp mark 1}{(10.029,7.143)}
\gppoint{gp mark 1}{(10.031,7.145)}
\gppoint{gp mark 1}{(10.034,7.164)}
\gppoint{gp mark 1}{(10.036,7.142)}
\gppoint{gp mark 1}{(10.038,7.097)}
\gppoint{gp mark 1}{(10.040,7.132)}
\gppoint{gp mark 1}{(10.043,7.139)}
\gppoint{gp mark 1}{(10.045,7.116)}
\gppoint{gp mark 1}{(10.047,7.118)}
\gppoint{gp mark 1}{(10.049,7.136)}
\gppoint{gp mark 1}{(10.052,7.141)}
\gppoint{gp mark 1}{(10.054,7.124)}
\gppoint{gp mark 1}{(10.056,7.146)}
\gppoint{gp mark 1}{(10.058,7.141)}
\gppoint{gp mark 1}{(10.061,7.152)}
\gppoint{gp mark 1}{(10.063,7.150)}
\gppoint{gp mark 1}{(10.065,7.116)}
\gppoint{gp mark 1}{(10.067,7.145)}
\gppoint{gp mark 1}{(10.070,7.115)}
\gppoint{gp mark 1}{(10.072,7.150)}
\gppoint{gp mark 1}{(10.074,7.113)}
\gppoint{gp mark 1}{(10.076,7.153)}
\gppoint{gp mark 1}{(10.079,7.125)}
\gppoint{gp mark 1}{(10.081,7.128)}
\gppoint{gp mark 1}{(10.083,7.133)}
\gppoint{gp mark 1}{(10.085,7.154)}
\gppoint{gp mark 1}{(10.088,7.114)}
\gppoint{gp mark 1}{(10.090,7.140)}
\gppoint{gp mark 1}{(10.092,7.154)}
\gppoint{gp mark 1}{(10.094,7.119)}
\gppoint{gp mark 1}{(10.097,7.130)}
\gppoint{gp mark 1}{(10.099,7.166)}
\gppoint{gp mark 1}{(10.101,7.137)}
\gppoint{gp mark 1}{(10.103,7.168)}
\gppoint{gp mark 1}{(10.106,7.165)}
\gppoint{gp mark 1}{(10.108,7.153)}
\gppoint{gp mark 1}{(10.110,7.179)}
\gppoint{gp mark 1}{(10.112,7.158)}
\gppoint{gp mark 1}{(10.115,7.190)}
\gppoint{gp mark 1}{(10.117,7.151)}
\gppoint{gp mark 1}{(10.119,7.181)}
\gppoint{gp mark 1}{(10.121,7.159)}
\gppoint{gp mark 1}{(10.124,7.144)}
\gppoint{gp mark 1}{(10.126,7.154)}
\gppoint{gp mark 1}{(10.128,7.155)}
\gppoint{gp mark 1}{(10.130,7.142)}
\gppoint{gp mark 1}{(10.133,7.163)}
\gppoint{gp mark 1}{(10.135,7.175)}
\gppoint{gp mark 1}{(10.137,7.171)}
\gppoint{gp mark 1}{(10.139,7.202)}
\gppoint{gp mark 1}{(10.142,7.162)}
\gppoint{gp mark 1}{(10.144,7.143)}
\gppoint{gp mark 1}{(10.146,7.144)}
\gppoint{gp mark 1}{(10.148,7.166)}
\gppoint{gp mark 1}{(10.151,7.142)}
\gppoint{gp mark 1}{(10.153,7.162)}
\gppoint{gp mark 1}{(10.155,7.146)}
\gppoint{gp mark 1}{(10.157,7.149)}
\gppoint{gp mark 1}{(10.160,7.165)}
\gppoint{gp mark 1}{(10.162,7.137)}
\gppoint{gp mark 1}{(10.164,7.171)}
\gppoint{gp mark 1}{(10.166,7.150)}
\gppoint{gp mark 1}{(10.169,7.173)}
\gppoint{gp mark 1}{(10.171,7.131)}
\gppoint{gp mark 1}{(10.173,7.162)}
\gppoint{gp mark 1}{(10.175,7.164)}
\gppoint{gp mark 1}{(10.178,7.162)}
\gppoint{gp mark 1}{(10.180,7.118)}
\gppoint{gp mark 1}{(10.182,7.180)}
\gppoint{gp mark 1}{(10.185,7.163)}
\gppoint{gp mark 1}{(10.187,7.161)}
\gppoint{gp mark 1}{(10.189,7.181)}
\gppoint{gp mark 1}{(10.191,7.155)}
\gppoint{gp mark 1}{(10.194,7.181)}
\gppoint{gp mark 1}{(10.196,7.194)}
\gppoint{gp mark 1}{(10.198,7.180)}
\gppoint{gp mark 1}{(10.200,7.163)}
\gppoint{gp mark 1}{(10.203,7.136)}
\gppoint{gp mark 1}{(10.205,7.172)}
\gppoint{gp mark 1}{(10.207,7.152)}
\gppoint{gp mark 1}{(10.209,7.145)}
\gppoint{gp mark 1}{(10.212,7.173)}
\gppoint{gp mark 1}{(10.214,7.182)}
\gppoint{gp mark 1}{(10.216,7.161)}
\gppoint{gp mark 1}{(10.218,7.138)}
\gppoint{gp mark 1}{(10.221,7.176)}
\gppoint{gp mark 1}{(10.223,7.171)}
\gppoint{gp mark 1}{(10.225,7.182)}
\gppoint{gp mark 1}{(10.227,7.172)}
\gppoint{gp mark 1}{(10.230,7.181)}
\gppoint{gp mark 1}{(10.232,7.166)}
\gppoint{gp mark 1}{(10.234,7.178)}
\gppoint{gp mark 1}{(10.236,7.178)}
\gppoint{gp mark 1}{(10.239,7.181)}
\gppoint{gp mark 1}{(10.241,7.145)}
\gppoint{gp mark 1}{(10.243,7.167)}
\gppoint{gp mark 1}{(10.245,7.182)}
\gppoint{gp mark 1}{(10.248,7.148)}
\gppoint{gp mark 1}{(10.250,7.158)}
\gppoint{gp mark 1}{(10.252,7.148)}
\gppoint{gp mark 1}{(10.254,7.172)}
\gppoint{gp mark 1}{(10.257,7.146)}
\gppoint{gp mark 1}{(10.259,7.191)}
\gppoint{gp mark 1}{(10.261,7.160)}
\gppoint{gp mark 1}{(10.263,7.146)}
\gppoint{gp mark 1}{(10.266,7.181)}
\gppoint{gp mark 1}{(10.268,7.157)}
\gppoint{gp mark 1}{(10.270,7.174)}
\gppoint{gp mark 1}{(10.272,7.138)}
\gppoint{gp mark 1}{(10.275,7.201)}
\gppoint{gp mark 1}{(10.277,7.168)}
\gppoint{gp mark 1}{(10.279,7.155)}
\gppoint{gp mark 1}{(10.281,7.172)}
\gppoint{gp mark 1}{(10.284,7.145)}
\gppoint{gp mark 1}{(10.286,7.147)}
\gppoint{gp mark 1}{(10.288,7.195)}
\gppoint{gp mark 1}{(10.290,7.157)}
\gppoint{gp mark 1}{(10.293,7.178)}
\gppoint{gp mark 1}{(10.295,7.151)}
\gppoint{gp mark 1}{(10.297,7.152)}
\gppoint{gp mark 1}{(10.299,7.146)}
\gppoint{gp mark 1}{(10.302,7.171)}
\gppoint{gp mark 1}{(10.304,7.174)}
\gppoint{gp mark 1}{(10.306,7.150)}
\gppoint{gp mark 1}{(10.308,7.183)}
\gppoint{gp mark 1}{(10.311,7.171)}
\gppoint{gp mark 1}{(10.313,7.170)}
\gppoint{gp mark 1}{(10.315,7.169)}
\gppoint{gp mark 1}{(10.317,7.210)}
\gppoint{gp mark 1}{(10.320,7.179)}
\gppoint{gp mark 1}{(10.322,7.150)}
\gppoint{gp mark 1}{(10.324,7.170)}
\gppoint{gp mark 1}{(10.326,7.186)}
\gppoint{gp mark 1}{(10.329,7.188)}
\gppoint{gp mark 1}{(10.331,5.893)}
\gppoint{gp mark 1}{(10.333,1.426)}
\gppoint{gp mark 1}{(10.335,1.421)}
\gppoint{gp mark 1}{(10.338,1.400)}
\gppoint{gp mark 1}{(10.340,1.398)}
\gppoint{gp mark 1}{(10.342,1.420)}
\gppoint{gp mark 1}{(10.344,1.408)}
\gppoint{gp mark 1}{(10.347,1.403)}
\gppoint{gp mark 1}{(10.349,1.412)}
\gppoint{gp mark 1}{(10.351,1.407)}
\gppoint{gp mark 1}{(10.353,1.413)}
\gppoint{gp mark 1}{(10.356,1.422)}
\gppoint{gp mark 1}{(10.358,1.400)}
\gppoint{gp mark 1}{(10.360,1.412)}
\gppoint{gp mark 1}{(10.362,1.410)}
\gppoint{gp mark 1}{(10.365,1.409)}
\gppoint{gp mark 1}{(10.367,1.415)}
\gppoint{gp mark 1}{(10.369,1.418)}
\gppoint{gp mark 1}{(10.371,1.413)}
\gppoint{gp mark 1}{(10.374,1.419)}
\gppoint{gp mark 1}{(10.376,1.416)}
\gppoint{gp mark 1}{(10.378,1.407)}
\gppoint{gp mark 1}{(10.380,1.413)}
\gppoint{gp mark 1}{(10.383,1.420)}
\gppoint{gp mark 1}{(10.385,1.416)}
\gppoint{gp mark 1}{(10.387,1.409)}
\gppoint{gp mark 1}{(10.389,1.415)}
\gppoint{gp mark 1}{(10.392,1.425)}
\gppoint{gp mark 1}{(10.394,1.409)}
\gppoint{gp mark 1}{(10.396,1.411)}
\gppoint{gp mark 1}{(10.398,1.417)}
\gppoint{gp mark 1}{(10.401,1.411)}
\gppoint{gp mark 1}{(10.403,1.428)}
\gppoint{gp mark 1}{(10.405,1.417)}
\gppoint{gp mark 1}{(10.407,1.417)}
\gppoint{gp mark 1}{(10.410,1.417)}
\gppoint{gp mark 1}{(10.412,1.407)}
\gppoint{gp mark 1}{(10.414,1.424)}
\gppoint{gp mark 1}{(10.416,1.410)}
\gppoint{gp mark 1}{(10.419,1.421)}
\gppoint{gp mark 1}{(10.421,1.419)}
\gppoint{gp mark 1}{(10.423,1.420)}
\gppoint{gp mark 1}{(10.425,1.422)}
\gppoint{gp mark 1}{(10.428,1.406)}
\gppoint{gp mark 1}{(10.430,1.422)}
\gppoint{gp mark 1}{(10.432,1.415)}
\gppoint{gp mark 1}{(10.434,1.425)}
\gppoint{gp mark 1}{(10.437,1.407)}
\gppoint{gp mark 1}{(10.439,1.424)}
\gppoint{gp mark 1}{(10.441,1.403)}
\gppoint{gp mark 1}{(10.443,1.429)}
\gppoint{gp mark 1}{(10.446,1.410)}
\gppoint{gp mark 1}{(10.448,1.430)}
\gppoint{gp mark 1}{(10.450,1.418)}
\gppoint{gp mark 1}{(10.452,1.436)}
\gppoint{gp mark 1}{(10.455,1.420)}
\gppoint{gp mark 1}{(10.457,1.400)}
\gppoint{gp mark 1}{(10.459,1.424)}
\gppoint{gp mark 1}{(10.461,1.411)}
\gppoint{gp mark 1}{(10.464,1.408)}
\gppoint{gp mark 1}{(10.466,1.426)}
\gppoint{gp mark 1}{(10.468,1.426)}
\gppoint{gp mark 1}{(10.470,1.427)}
\gppoint{gp mark 1}{(10.473,1.414)}
\gppoint{gp mark 1}{(10.475,1.417)}
\gppoint{gp mark 1}{(10.477,1.424)}
\gppoint{gp mark 1}{(10.479,1.423)}
\gppoint{gp mark 1}{(10.482,1.432)}
\gppoint{gp mark 1}{(10.484,1.416)}
\gppoint{gp mark 1}{(10.486,1.415)}
\gppoint{gp mark 1}{(10.488,1.417)}
\gppoint{gp mark 1}{(10.491,1.409)}
\gppoint{gp mark 1}{(10.493,1.393)}
\gppoint{gp mark 1}{(10.495,1.407)}
\gppoint{gp mark 1}{(10.497,1.414)}
\gppoint{gp mark 1}{(10.500,1.419)}
\gppoint{gp mark 1}{(10.502,1.420)}
\gppoint{gp mark 1}{(10.504,1.410)}
\gppoint{gp mark 1}{(10.506,1.409)}
\gppoint{gp mark 1}{(10.509,1.419)}
\gppoint{gp mark 1}{(10.511,1.436)}
\gppoint{gp mark 1}{(10.513,1.415)}
\gppoint{gp mark 1}{(10.515,1.419)}
\gppoint{gp mark 1}{(10.518,1.421)}
\gppoint{gp mark 1}{(10.520,1.409)}
\gppoint{gp mark 1}{(10.522,1.411)}
\gppoint{gp mark 1}{(10.524,1.424)}
\gppoint{gp mark 1}{(10.527,1.422)}
\gppoint{gp mark 1}{(10.529,1.409)}
\gppoint{gp mark 1}{(10.531,1.427)}
\gppoint{gp mark 1}{(10.533,1.435)}
\gppoint{gp mark 1}{(10.536,1.446)}
\gppoint{gp mark 1}{(10.538,1.408)}
\gppoint{gp mark 1}{(10.540,1.421)}
\gppoint{gp mark 1}{(10.542,1.420)}
\gppoint{gp mark 1}{(10.545,1.427)}
\gppoint{gp mark 1}{(10.547,1.413)}
\gppoint{gp mark 1}{(10.549,1.420)}
\gppoint{gp mark 1}{(10.551,1.427)}
\gppoint{gp mark 1}{(10.554,1.410)}
\gppoint{gp mark 1}{(10.556,1.433)}
\gppoint{gp mark 1}{(10.558,1.415)}
\gppoint{gp mark 1}{(10.560,1.440)}
\gppoint{gp mark 1}{(10.563,1.426)}
\gppoint{gp mark 1}{(10.565,1.413)}
\gppoint{gp mark 1}{(10.567,1.437)}
\gppoint{gp mark 1}{(10.569,1.405)}
\gppoint{gp mark 1}{(10.572,1.411)}
\gppoint{gp mark 1}{(10.574,1.427)}
\gppoint{gp mark 1}{(10.576,1.410)}
\gppoint{gp mark 1}{(10.578,1.429)}
\gppoint{gp mark 1}{(10.581,1.434)}
\gppoint{gp mark 1}{(10.583,1.440)}
\gppoint{gp mark 1}{(10.585,1.421)}
\gppoint{gp mark 1}{(10.587,1.430)}
\gppoint{gp mark 1}{(10.590,1.442)}
\gppoint{gp mark 1}{(10.592,1.415)}
\gppoint{gp mark 1}{(10.594,1.425)}
\gppoint{gp mark 1}{(10.596,1.425)}
\gppoint{gp mark 1}{(10.599,1.428)}
\gppoint{gp mark 1}{(10.601,1.430)}
\gppoint{gp mark 1}{(10.603,1.422)}
\gppoint{gp mark 1}{(10.605,1.417)}
\gppoint{gp mark 1}{(10.608,1.417)}
\gppoint{gp mark 1}{(10.610,1.414)}
\gppoint{gp mark 1}{(10.612,1.426)}
\gppoint{gp mark 1}{(10.614,1.414)}
\gppoint{gp mark 1}{(10.617,1.419)}
\gppoint{gp mark 1}{(10.619,1.417)}
\gppoint{gp mark 1}{(10.621,1.415)}
\gppoint{gp mark 1}{(10.623,1.422)}
\gppoint{gp mark 1}{(10.626,1.428)}
\gppoint{gp mark 1}{(10.628,1.434)}
\gppoint{gp mark 1}{(10.630,1.446)}
\gppoint{gp mark 1}{(10.632,1.412)}
\gppoint{gp mark 1}{(10.635,1.440)}
\gppoint{gp mark 1}{(10.637,1.417)}
\gppoint{gp mark 1}{(10.639,1.429)}
\gppoint{gp mark 1}{(10.641,1.450)}
\gppoint{gp mark 1}{(10.644,1.424)}
\gppoint{gp mark 1}{(10.646,1.436)}
\gppoint{gp mark 1}{(10.648,1.425)}
\gppoint{gp mark 1}{(10.650,1.436)}
\draw[gp path] (1.320,7.691)--(1.320,0.985)--(11.447,0.985)--(11.447,7.691)--cycle;
%% coordinates of the plot area
\gpdefrectangularnode{gp plot 1}{\pgfpoint{1.320cm}{0.985cm}}{\pgfpoint{11.447cm}{7.691cm}}
\end{tikzpicture}
%% gnuplot variables

\caption{平均を計算する領域}
\label{fig:solar_radio_Temp}
\end{center}
\end{figure}