\subsection{宇宙電波と太陽電波}
宇宙電波源には主に熱的電波源と非熱的電波源がある.
熱的電波は大きな運動エネルギーをもった自由電子が陽子に接近したときの急制動による制動放射であり,
プラズマから発せられる.
一方で非熱的電波は磁場の周りを荷電粒子が円運動することによるシンクロトロン放射である.
また熱的放射は定常的であるのに対し非熱的放射は爆発的なものである.

特に太陽は地球に最も近接した電波天体であり,輻射によって$3.85\times10^{26}\si{\watt}$ものエネルギーを放射し,
地球の大気表面に到達する単位面積あたりのエネルギー(太陽定数)は$1.37\si{\kilo\watt.\metre^{-2}}$になる.\cite{rikanenpyo}
そのエネルギー源は中心核での水素核融合反応である.
太陽からの熱的放射は可視光領域では$T=5800\si{\kelvin}$の光球から,
$10\si{\giga\hertz}$以上の領域では$T=7000\sim10000\si{\kelvin}$の彩層から,
$1\si{\giga\hertz}$以下の領域では$T\sim10^6\si{\kelvin}$のコロナからの放射が支配的である.
またフレアなどの爆発現象に伴い非熱的放射を行うこともあり,周波数は数$100\si{\kilo\hertz}$から数$10\si{\giga\hertz}$に渡る.