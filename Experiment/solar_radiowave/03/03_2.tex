\subsection{Allan分散}
\subsubsection{原理}
受信機の性能,雑音が定常的であるならば信号を時間平均することで受信機の感動を高めることができる.
しかし実際には受信機の性能は時間的に変動し,雑音もまた有限時間において完全にランダムではないことがわかっている.
このため有効に感度を向上できる積分時間の上限が存在する.この時間を考えるためにAllan分散が用いられる.

連続量$s(t)$を時間平均で測定することを考えると時刻$t$,積分時間$T$のとき観測される量は
\begin{align}
  x(T,t)=\frac{1}{T}\int^t_{t-T}s(t'){\rm d}t'
\end{align}
である.ここで時刻$t$での観測量$x_s=x(T,t)$と$T$秒後の観測量$x_r=x(T,t+T)$の差を取ると
\begin{align}
  d=x_s-x_r
\end{align}
この$d$の分散$\sigma_d^2$は
\begin{align}
  \sigma_d^2(T)&=\langle(d-\langle d\rangle)^2\rangle\nonumber\\
  &=\langle d^2\rangle-\langle d\rangle^2
\end{align}
となる.ここで$x_s$と$x_r$は同じ連続量を見ているので, $\langle d\rangle=0$が成り立つとすると
\begin{align}
  \sigma_d^2(T)=\langle d^2\rangle
\end{align}
となる.Allan分散はこれに$1/2$を掛けた値と定義される.
\begin{align}
  \sigma_A^2(T)=:\frac{1}{2}\langle d^2\rangle
\end{align}
ここで$\langle d^2\rangle$を考えると
\begin{align}
  \langle d^2\rangle&=\left\langle(x_s-x_r)^2\right\rangle\nonumber\\
  &=\langle x_s^2\rangle-2\langle x_sx_r\rangle+\langle x_r^2\rangle\nonumber
\end{align}
ここで$\langle x_s^2\rangle=\langle x_r^2\rangle$を仮定すると
\begin{align}
  \langle d^2\rangle=2(\langle x_s^2\rangle-\langle x_sx_r\rangle)\nonumber
\end{align}
となり,ここで自己相関関数$R(\tau)=\langle x(T,t)x(T,t+\tau)$を用いると
\begin{align}
  \langle d^2\rangle=2(R(0)-R(T))
\end{align}
となる.ここで$R(0)$は定数なので$R(T)$について考える.
Wiener-Khinchinの定理より自己相関関数のFourier変換はパワースペクトル$S(\omega)$で表される.\cite{ishikawa40:online}
\begin{align}
  \mathcal{F}(R(T))=S(\omega)
\end{align}
一般的に$S(\omega)=|\omega|^{-\alpha}$と表され,これを逆フーリエ変換すると
\begin{align}
  \mathcal{F}^{-1}(|\omega|^{-\alpha})=a(\alpha)|T|^{\alpha-1}
\end{align}
となる.以上から
\begin{align}
  \sigma_A^2(T)\sim -R(T)=-a(\alpha)|T|^{\alpha-1}
\end{align}
である.ここで$-a(\alpha)$は正である.
したがって$T$-$\sigma_A^2$グラフにおいて$\alpha=0$の成分は負の傾き,
$\alpha=1$の成分は定数, $\alpha=2$の成分は正の傾きを持つことになり,
それぞれをwhite noise, flicker noise, drift noiseと呼ぶ.
また$T$-$\sigma_A^2$グラフにおいて$\sigma_A^2$が最小となるのは全ノイズの影響が最小になる積分時間と言える.
\subsubsection{方法}
室外機を屋内に設置し,アンテナ全体を電波吸収体で覆った.
その状態で室温の黒体放射を1時間測定した.
測定中は出力結果の人為的な変動を防ぐため,観測装置を安定に保つよう配慮した.
次に測定用計算機上のAllan分散計算ツール allanを用いてAllan分散を計算した.
ただしAllan分散の計算は$t>1000$のデータで行った.
\subsubsection{結果}
図\ref{fig:allan_raw}に常温黒体の測定結果を示す.また,図\ref{fig:allan}にAllan分散の$T$依存性を示す.
図\ref{fig:allan}から$\sigma_A^2$は$T=59$で最小値を取るため, $t_{eff}=59$である.
\begin{figure}[hptb]
\begin{center}
\begin{tikzpicture}[gnuplot]
%% generated with GNUPLOT 5.2p8 (Lua 5.3; terminal rev. Nov 2018, script rev. 108)
%% 2021年05月13日 15時26分05秒
\path (0.000,0.000) rectangle (12.000,8.000);
\gpcolor{color=gp lt color border}
\gpsetlinetype{gp lt border}
\gpsetdashtype{gp dt solid}
\gpsetlinewidth{1.00}
\draw[gp path] (1.504,0.985)--(1.684,0.985);
\draw[gp path] (11.447,0.985)--(11.267,0.985);
\node[gp node right] at (1.320,0.985) {$3.1$};
\draw[gp path] (1.504,2.103)--(1.684,2.103);
\draw[gp path] (11.447,2.103)--(11.267,2.103);
\node[gp node right] at (1.320,2.103) {$3.11$};
\draw[gp path] (1.504,3.220)--(1.684,3.220);
\draw[gp path] (11.447,3.220)--(11.267,3.220);
\node[gp node right] at (1.320,3.220) {$3.12$};
\draw[gp path] (1.504,4.338)--(1.684,4.338);
\draw[gp path] (11.447,4.338)--(11.267,4.338);
\node[gp node right] at (1.320,4.338) {$3.13$};
\draw[gp path] (1.504,5.456)--(1.684,5.456);
\draw[gp path] (11.447,5.456)--(11.267,5.456);
\node[gp node right] at (1.320,5.456) {$3.14$};
\draw[gp path] (1.504,6.573)--(1.684,6.573);
\draw[gp path] (11.447,6.573)--(11.267,6.573);
\node[gp node right] at (1.320,6.573) {$3.15$};
\draw[gp path] (1.504,7.691)--(1.684,7.691);
\draw[gp path] (11.447,7.691)--(11.267,7.691);
\node[gp node right] at (1.320,7.691) {$3.16$};
\draw[gp path] (1.504,0.985)--(1.504,1.165);
\draw[gp path] (1.504,7.691)--(1.504,7.511);
\node[gp node center] at (1.504,0.677) {$1000$};
\draw[gp path] (2.747,0.985)--(2.747,1.165);
\draw[gp path] (2.747,7.691)--(2.747,7.511);
\node[gp node center] at (2.747,0.677) {$1500$};
\draw[gp path] (3.989,0.985)--(3.989,1.165);
\draw[gp path] (3.989,7.691)--(3.989,7.511);
\node[gp node center] at (3.989,0.677) {$2000$};
\draw[gp path] (5.232,0.985)--(5.232,1.165);
\draw[gp path] (5.232,7.691)--(5.232,7.511);
\node[gp node center] at (5.232,0.677) {$2500$};
\draw[gp path] (6.474,0.985)--(6.474,1.165);
\draw[gp path] (6.474,7.691)--(6.474,7.511);
\node[gp node center] at (6.474,0.677) {$3000$};
\draw[gp path] (7.717,0.985)--(7.717,1.165);
\draw[gp path] (7.717,7.691)--(7.717,7.511);
\node[gp node center] at (7.717,0.677) {$3500$};
\draw[gp path] (8.959,0.985)--(8.959,1.165);
\draw[gp path] (8.959,7.691)--(8.959,7.511);
\node[gp node center] at (8.959,0.677) {$4000$};
\draw[gp path] (10.202,0.985)--(10.202,1.165);
\draw[gp path] (10.202,7.691)--(10.202,7.511);
\node[gp node center] at (10.202,0.677) {$4500$};
\draw[gp path] (11.445,0.985)--(11.445,1.165);
\draw[gp path] (11.445,7.691)--(11.445,7.511);
\node[gp node center] at (11.445,0.677) {$5000$};
\draw[gp path] (1.504,7.691)--(1.504,0.985)--(11.447,0.985)--(11.447,7.691)--cycle;
\node[gp node center,rotate=-270] at (0.292,4.338) {電圧 / $\si{\volt}$};
\node[gp node center] at (6.475,0.215) {経過時間$t$ / $\si{\second}$};
\gpsetpointsize{4.00}
\gppoint{gp mark 1}{(1.504,5.546)}
\gppoint{gp mark 1}{(1.506,4.754)}
\gppoint{gp mark 1}{(1.509,6.360)}
\gppoint{gp mark 1}{(1.511,5.273)}
\gppoint{gp mark 1}{(1.514,5.792)}
\gppoint{gp mark 1}{(1.516,4.880)}
\gppoint{gp mark 1}{(1.519,5.257)}
\gppoint{gp mark 1}{(1.521,5.290)}
\gppoint{gp mark 1}{(1.524,5.989)}
\gppoint{gp mark 1}{(1.526,5.770)}
\gppoint{gp mark 1}{(1.529,5.945)}
\gppoint{gp mark 1}{(1.531,5.453)}
\gppoint{gp mark 1}{(1.534,5.951)}
\gppoint{gp mark 1}{(1.536,6.164)}
\gppoint{gp mark 1}{(1.539,5.006)}
\gppoint{gp mark 1}{(1.541,5.077)}
\gppoint{gp mark 1}{(1.544,5.754)}
\gppoint{gp mark 1}{(1.546,5.432)}
\gppoint{gp mark 1}{(1.549,5.617)}
\gppoint{gp mark 1}{(1.551,5.475)}
\gppoint{gp mark 1}{(1.554,5.393)}
\gppoint{gp mark 1}{(1.556,5.667)}
\gppoint{gp mark 1}{(1.559,5.699)}
\gppoint{gp mark 1}{(1.561,4.470)}
\gppoint{gp mark 1}{(1.564,4.765)}
\gppoint{gp mark 1}{(1.566,5.453)}
\gppoint{gp mark 1}{(1.569,4.869)}
\gppoint{gp mark 1}{(1.571,5.432)}
\gppoint{gp mark 1}{(1.574,4.569)}
\gppoint{gp mark 1}{(1.576,5.295)}
\gppoint{gp mark 1}{(1.579,5.388)}
\gppoint{gp mark 1}{(1.581,5.754)}
\gppoint{gp mark 1}{(1.584,5.077)}
\gppoint{gp mark 1}{(1.586,4.978)}
\gppoint{gp mark 1}{(1.588,5.453)}
\gppoint{gp mark 1}{(1.591,4.842)}
\gppoint{gp mark 1}{(1.593,5.372)}
\gppoint{gp mark 1}{(1.596,5.748)}
\gppoint{gp mark 1}{(1.598,5.071)}
\gppoint{gp mark 1}{(1.601,5.284)}
\gppoint{gp mark 1}{(1.603,5.650)}
\gppoint{gp mark 1}{(1.606,5.044)}
\gppoint{gp mark 1}{(1.608,5.208)}
\gppoint{gp mark 1}{(1.611,5.295)}
\gppoint{gp mark 1}{(1.613,6.016)}
\gppoint{gp mark 1}{(1.616,5.847)}
\gppoint{gp mark 1}{(1.618,4.907)}
\gppoint{gp mark 1}{(1.621,5.574)}
\gppoint{gp mark 1}{(1.623,5.907)}
\gppoint{gp mark 1}{(1.626,5.202)}
\gppoint{gp mark 1}{(1.628,5.574)}
\gppoint{gp mark 1}{(1.631,5.284)}
\gppoint{gp mark 1}{(1.633,5.087)}
\gppoint{gp mark 1}{(1.636,5.344)}
\gppoint{gp mark 1}{(1.638,4.809)}
\gppoint{gp mark 1}{(1.641,4.809)}
\gppoint{gp mark 1}{(1.643,5.617)}
\gppoint{gp mark 1}{(1.646,5.262)}
\gppoint{gp mark 1}{(1.648,4.552)}
\gppoint{gp mark 1}{(1.651,5.568)}
\gppoint{gp mark 1}{(1.653,4.629)}
\gppoint{gp mark 1}{(1.656,5.295)}
\gppoint{gp mark 1}{(1.658,5.213)}
\gppoint{gp mark 1}{(1.661,5.306)}
\gppoint{gp mark 1}{(1.663,4.727)}
\gppoint{gp mark 1}{(1.666,4.814)}
\gppoint{gp mark 1}{(1.668,4.399)}
\gppoint{gp mark 1}{(1.671,5.596)}
\gppoint{gp mark 1}{(1.673,5.328)}
\gppoint{gp mark 1}{(1.675,5.393)}
\gppoint{gp mark 1}{(1.678,4.858)}
\gppoint{gp mark 1}{(1.680,5.443)}
\gppoint{gp mark 1}{(1.683,5.628)}
\gppoint{gp mark 1}{(1.685,5.284)}
\gppoint{gp mark 1}{(1.688,5.628)}
\gppoint{gp mark 1}{(1.690,5.142)}
\gppoint{gp mark 1}{(1.693,5.344)}
\gppoint{gp mark 1}{(1.695,5.066)}
\gppoint{gp mark 1}{(1.698,5.339)}
\gppoint{gp mark 1}{(1.700,4.869)}
\gppoint{gp mark 1}{(1.703,5.727)}
\gppoint{gp mark 1}{(1.705,5.798)}
\gppoint{gp mark 1}{(1.708,5.486)}
\gppoint{gp mark 1}{(1.710,6.022)}
\gppoint{gp mark 1}{(1.713,4.754)}
\gppoint{gp mark 1}{(1.715,5.503)}
\gppoint{gp mark 1}{(1.718,5.415)}
\gppoint{gp mark 1}{(1.720,4.650)}
\gppoint{gp mark 1}{(1.723,5.120)}
\gppoint{gp mark 1}{(1.725,5.880)}
\gppoint{gp mark 1}{(1.728,5.262)}
\gppoint{gp mark 1}{(1.730,5.344)}
\gppoint{gp mark 1}{(1.733,5.792)}
\gppoint{gp mark 1}{(1.735,5.011)}
\gppoint{gp mark 1}{(1.738,5.748)}
\gppoint{gp mark 1}{(1.740,4.995)}
\gppoint{gp mark 1}{(1.743,5.087)}
\gppoint{gp mark 1}{(1.745,5.208)}
\gppoint{gp mark 1}{(1.748,5.038)}
\gppoint{gp mark 1}{(1.750,5.497)}
\gppoint{gp mark 1}{(1.753,4.667)}
\gppoint{gp mark 1}{(1.755,4.727)}
\gppoint{gp mark 1}{(1.757,5.169)}
\gppoint{gp mark 1}{(1.760,5.049)}
\gppoint{gp mark 1}{(1.762,4.967)}
\gppoint{gp mark 1}{(1.765,5.765)}
\gppoint{gp mark 1}{(1.767,4.803)}
\gppoint{gp mark 1}{(1.770,5.306)}
\gppoint{gp mark 1}{(1.772,5.453)}
\gppoint{gp mark 1}{(1.775,5.016)}
\gppoint{gp mark 1}{(1.777,5.148)}
\gppoint{gp mark 1}{(1.780,5.366)}
\gppoint{gp mark 1}{(1.782,5.033)}
\gppoint{gp mark 1}{(1.785,4.612)}
\gppoint{gp mark 1}{(1.787,5.142)}
\gppoint{gp mark 1}{(1.790,5.350)}
\gppoint{gp mark 1}{(1.792,4.552)}
\gppoint{gp mark 1}{(1.795,4.787)}
\gppoint{gp mark 1}{(1.797,4.995)}
\gppoint{gp mark 1}{(1.800,4.738)}
\gppoint{gp mark 1}{(1.802,5.153)}
\gppoint{gp mark 1}{(1.805,4.798)}
\gppoint{gp mark 1}{(1.807,5.694)}
\gppoint{gp mark 1}{(1.810,5.290)}
\gppoint{gp mark 1}{(1.812,5.721)}
\gppoint{gp mark 1}{(1.815,4.470)}
\gppoint{gp mark 1}{(1.817,5.694)}
\gppoint{gp mark 1}{(1.820,4.700)}
\gppoint{gp mark 1}{(1.822,5.596)}
\gppoint{gp mark 1}{(1.825,4.776)}
\gppoint{gp mark 1}{(1.827,5.273)}
\gppoint{gp mark 1}{(1.830,5.082)}
\gppoint{gp mark 1}{(1.832,5.186)}
\gppoint{gp mark 1}{(1.835,4.487)}
\gppoint{gp mark 1}{(1.837,5.284)}
\gppoint{gp mark 1}{(1.839,4.361)}
\gppoint{gp mark 1}{(1.842,4.924)}
\gppoint{gp mark 1}{(1.844,5.164)}
\gppoint{gp mark 1}{(1.847,5.415)}
\gppoint{gp mark 1}{(1.849,4.760)}
\gppoint{gp mark 1}{(1.852,5.896)}
\gppoint{gp mark 1}{(1.854,5.071)}
\gppoint{gp mark 1}{(1.857,4.776)}
\gppoint{gp mark 1}{(1.859,5.071)}
\gppoint{gp mark 1}{(1.862,4.918)}
\gppoint{gp mark 1}{(1.864,4.667)}
\gppoint{gp mark 1}{(1.867,5.251)}
\gppoint{gp mark 1}{(1.869,5.180)}
\gppoint{gp mark 1}{(1.872,4.956)}
\gppoint{gp mark 1}{(1.874,4.738)}
\gppoint{gp mark 1}{(1.877,5.382)}
\gppoint{gp mark 1}{(1.879,5.475)}
\gppoint{gp mark 1}{(1.882,4.339)}
\gppoint{gp mark 1}{(1.884,4.476)}
\gppoint{gp mark 1}{(1.887,5.421)}
\gppoint{gp mark 1}{(1.889,5.087)}
\gppoint{gp mark 1}{(1.892,4.935)}
\gppoint{gp mark 1}{(1.894,4.416)}
\gppoint{gp mark 1}{(1.897,4.874)}
\gppoint{gp mark 1}{(1.899,5.027)}
\gppoint{gp mark 1}{(1.902,4.645)}
\gppoint{gp mark 1}{(1.904,4.705)}
\gppoint{gp mark 1}{(1.907,5.006)}
\gppoint{gp mark 1}{(1.909,5.787)}
\gppoint{gp mark 1}{(1.912,4.399)}
\gppoint{gp mark 1}{(1.914,5.175)}
\gppoint{gp mark 1}{(1.917,5.656)}
\gppoint{gp mark 1}{(1.919,5.022)}
\gppoint{gp mark 1}{(1.922,4.809)}
\gppoint{gp mark 1}{(1.924,5.459)}
\gppoint{gp mark 1}{(1.926,4.525)}
\gppoint{gp mark 1}{(1.929,5.628)}
\gppoint{gp mark 1}{(1.931,5.262)}
\gppoint{gp mark 1}{(1.934,4.432)}
\gppoint{gp mark 1}{(1.936,5.066)}
\gppoint{gp mark 1}{(1.939,5.273)}
\gppoint{gp mark 1}{(1.941,5.770)}
\gppoint{gp mark 1}{(1.944,4.896)}
\gppoint{gp mark 1}{(1.946,5.830)}
\gppoint{gp mark 1}{(1.949,4.847)}
\gppoint{gp mark 1}{(1.951,5.301)}
\gppoint{gp mark 1}{(1.954,4.481)}
\gppoint{gp mark 1}{(1.956,5.301)}
\gppoint{gp mark 1}{(1.959,4.864)}
\gppoint{gp mark 1}{(1.961,4.416)}
\gppoint{gp mark 1}{(1.964,4.672)}
\gppoint{gp mark 1}{(1.966,5.175)}
\gppoint{gp mark 1}{(1.969,4.399)}
\gppoint{gp mark 1}{(1.971,5.033)}
\gppoint{gp mark 1}{(1.974,5.257)}
\gppoint{gp mark 1}{(1.976,4.711)}
\gppoint{gp mark 1}{(1.979,5.492)}
\gppoint{gp mark 1}{(1.981,4.552)}
\gppoint{gp mark 1}{(1.984,5.208)}
\gppoint{gp mark 1}{(1.986,4.896)}
\gppoint{gp mark 1}{(1.989,4.661)}
\gppoint{gp mark 1}{(1.991,4.924)}
\gppoint{gp mark 1}{(1.994,5.033)}
\gppoint{gp mark 1}{(1.996,4.814)}
\gppoint{gp mark 1}{(1.999,4.995)}
\gppoint{gp mark 1}{(2.001,4.902)}
\gppoint{gp mark 1}{(2.004,5.803)}
\gppoint{gp mark 1}{(2.006,5.339)}
\gppoint{gp mark 1}{(2.008,4.498)}
\gppoint{gp mark 1}{(2.011,5.104)}
\gppoint{gp mark 1}{(2.013,5.120)}
\gppoint{gp mark 1}{(2.016,4.514)}
\gppoint{gp mark 1}{(2.018,4.683)}
\gppoint{gp mark 1}{(2.021,4.967)}
\gppoint{gp mark 1}{(2.023,5.077)}
\gppoint{gp mark 1}{(2.026,5.503)}
\gppoint{gp mark 1}{(2.028,4.913)}
\gppoint{gp mark 1}{(2.031,5.437)}
\gppoint{gp mark 1}{(2.033,5.202)}
\gppoint{gp mark 1}{(2.036,5.224)}
\gppoint{gp mark 1}{(2.038,4.831)}
\gppoint{gp mark 1}{(2.041,4.989)}
\gppoint{gp mark 1}{(2.043,4.612)}
\gppoint{gp mark 1}{(2.046,5.224)}
\gppoint{gp mark 1}{(2.048,5.180)}
\gppoint{gp mark 1}{(2.051,4.596)}
\gppoint{gp mark 1}{(2.053,5.623)}
\gppoint{gp mark 1}{(2.056,5.027)}
\gppoint{gp mark 1}{(2.058,4.612)}
\gppoint{gp mark 1}{(2.061,4.470)}
\gppoint{gp mark 1}{(2.063,4.700)}
\gppoint{gp mark 1}{(2.066,5.350)}
\gppoint{gp mark 1}{(2.068,4.798)}
\gppoint{gp mark 1}{(2.071,4.809)}
\gppoint{gp mark 1}{(2.073,4.432)}
\gppoint{gp mark 1}{(2.076,4.771)}
\gppoint{gp mark 1}{(2.078,4.716)}
\gppoint{gp mark 1}{(2.081,4.503)}
\gppoint{gp mark 1}{(2.083,4.186)}
\gppoint{gp mark 1}{(2.086,5.240)}
\gppoint{gp mark 1}{(2.088,4.366)}
\gppoint{gp mark 1}{(2.090,4.514)}
\gppoint{gp mark 1}{(2.093,4.268)}
\gppoint{gp mark 1}{(2.095,4.820)}
\gppoint{gp mark 1}{(2.098,5.049)}
\gppoint{gp mark 1}{(2.100,4.945)}
\gppoint{gp mark 1}{(2.103,5.038)}
\gppoint{gp mark 1}{(2.105,4.541)}
\gppoint{gp mark 1}{(2.108,5.590)}
\gppoint{gp mark 1}{(2.110,4.803)}
\gppoint{gp mark 1}{(2.113,5.617)}
\gppoint{gp mark 1}{(2.115,5.257)}
\gppoint{gp mark 1}{(2.118,4.853)}
\gppoint{gp mark 1}{(2.120,4.705)}
\gppoint{gp mark 1}{(2.123,4.847)}
\gppoint{gp mark 1}{(2.125,4.973)}
\gppoint{gp mark 1}{(2.128,4.590)}
\gppoint{gp mark 1}{(2.130,4.623)}
\gppoint{gp mark 1}{(2.133,4.951)}
\gppoint{gp mark 1}{(2.135,5.262)}
\gppoint{gp mark 1}{(2.138,6.175)}
\gppoint{gp mark 1}{(2.140,5.306)}
\gppoint{gp mark 1}{(2.143,4.836)}
\gppoint{gp mark 1}{(2.145,5.077)}
\gppoint{gp mark 1}{(2.148,4.945)}
\gppoint{gp mark 1}{(2.150,3.782)}
\gppoint{gp mark 1}{(2.153,5.393)}
\gppoint{gp mark 1}{(2.155,5.656)}
\gppoint{gp mark 1}{(2.158,5.066)}
\gppoint{gp mark 1}{(2.160,5.481)}
\gppoint{gp mark 1}{(2.163,4.809)}
\gppoint{gp mark 1}{(2.165,4.809)}
\gppoint{gp mark 1}{(2.168,4.743)}
\gppoint{gp mark 1}{(2.170,5.060)}
\gppoint{gp mark 1}{(2.172,4.924)}
\gppoint{gp mark 1}{(2.175,4.924)}
\gppoint{gp mark 1}{(2.177,4.853)}
\gppoint{gp mark 1}{(2.180,5.634)}
\gppoint{gp mark 1}{(2.182,4.760)}
\gppoint{gp mark 1}{(2.185,4.623)}
\gppoint{gp mark 1}{(2.187,4.716)}
\gppoint{gp mark 1}{(2.190,4.519)}
\gppoint{gp mark 1}{(2.192,4.574)}
\gppoint{gp mark 1}{(2.195,4.864)}
\gppoint{gp mark 1}{(2.197,4.563)}
\gppoint{gp mark 1}{(2.200,3.869)}
\gppoint{gp mark 1}{(2.202,4.694)}
\gppoint{gp mark 1}{(2.205,4.842)}
\gppoint{gp mark 1}{(2.207,5.027)}
\gppoint{gp mark 1}{(2.210,4.902)}
\gppoint{gp mark 1}{(2.212,5.006)}
\gppoint{gp mark 1}{(2.215,5.197)}
\gppoint{gp mark 1}{(2.217,4.443)}
\gppoint{gp mark 1}{(2.220,4.760)}
\gppoint{gp mark 1}{(2.222,4.995)}
\gppoint{gp mark 1}{(2.225,5.470)}
\gppoint{gp mark 1}{(2.227,5.016)}
\gppoint{gp mark 1}{(2.230,5.350)}
\gppoint{gp mark 1}{(2.232,4.984)}
\gppoint{gp mark 1}{(2.235,4.787)}
\gppoint{gp mark 1}{(2.237,5.437)}
\gppoint{gp mark 1}{(2.240,5.246)}
\gppoint{gp mark 1}{(2.242,4.782)}
\gppoint{gp mark 1}{(2.245,4.874)}
\gppoint{gp mark 1}{(2.247,5.273)}
\gppoint{gp mark 1}{(2.250,5.290)}
\gppoint{gp mark 1}{(2.252,4.672)}
\gppoint{gp mark 1}{(2.255,4.809)}
\gppoint{gp mark 1}{(2.257,5.432)}
\gppoint{gp mark 1}{(2.259,4.727)}
\gppoint{gp mark 1}{(2.262,5.093)}
\gppoint{gp mark 1}{(2.264,4.809)}
\gppoint{gp mark 1}{(2.267,4.798)}
\gppoint{gp mark 1}{(2.269,4.470)}
\gppoint{gp mark 1}{(2.272,5.219)}
\gppoint{gp mark 1}{(2.274,4.678)}
\gppoint{gp mark 1}{(2.277,4.771)}
\gppoint{gp mark 1}{(2.279,5.033)}
\gppoint{gp mark 1}{(2.282,4.847)}
\gppoint{gp mark 1}{(2.284,4.667)}
\gppoint{gp mark 1}{(2.287,4.170)}
\gppoint{gp mark 1}{(2.289,5.224)}
\gppoint{gp mark 1}{(2.292,5.180)}
\gppoint{gp mark 1}{(2.294,5.006)}
\gppoint{gp mark 1}{(2.297,4.951)}
\gppoint{gp mark 1}{(2.299,4.847)}
\gppoint{gp mark 1}{(2.302,4.716)}
\gppoint{gp mark 1}{(2.304,4.596)}
\gppoint{gp mark 1}{(2.307,4.847)}
\gppoint{gp mark 1}{(2.309,5.186)}
\gppoint{gp mark 1}{(2.312,5.120)}
\gppoint{gp mark 1}{(2.314,5.535)}
\gppoint{gp mark 1}{(2.317,5.448)}
\gppoint{gp mark 1}{(2.319,5.841)}
\gppoint{gp mark 1}{(2.322,4.918)}
\gppoint{gp mark 1}{(2.324,5.066)}
\gppoint{gp mark 1}{(2.327,5.519)}
\gppoint{gp mark 1}{(2.329,5.186)}
\gppoint{gp mark 1}{(2.332,4.596)}
\gppoint{gp mark 1}{(2.334,4.487)}
\gppoint{gp mark 1}{(2.337,5.453)}
\gppoint{gp mark 1}{(2.339,5.754)}
\gppoint{gp mark 1}{(2.341,5.967)}
\gppoint{gp mark 1}{(2.344,5.972)}
\gppoint{gp mark 1}{(2.346,6.333)}
\gppoint{gp mark 1}{(2.349,5.885)}
\gppoint{gp mark 1}{(2.351,5.328)}
\gppoint{gp mark 1}{(2.354,5.530)}
\gppoint{gp mark 1}{(2.356,5.148)}
\gppoint{gp mark 1}{(2.359,5.765)}
\gppoint{gp mark 1}{(2.361,5.481)}
\gppoint{gp mark 1}{(2.364,5.754)}
\gppoint{gp mark 1}{(2.366,6.147)}
\gppoint{gp mark 1}{(2.369,6.459)}
\gppoint{gp mark 1}{(2.371,6.256)}
\gppoint{gp mark 1}{(2.374,6.819)}
\gppoint{gp mark 1}{(2.376,6.726)}
\gppoint{gp mark 1}{(2.379,6.633)}
\gppoint{gp mark 1}{(2.381,5.765)}
\gppoint{gp mark 1}{(2.384,7.420)}
\gppoint{gp mark 1}{(2.386,6.311)}
\gppoint{gp mark 1}{(2.389,5.694)}
\gppoint{gp mark 1}{(2.391,5.923)}
\gppoint{gp mark 1}{(2.394,6.306)}
\gppoint{gp mark 1}{(2.396,5.279)}
\gppoint{gp mark 1}{(2.399,5.748)}
\gppoint{gp mark 1}{(2.401,5.049)}
\gppoint{gp mark 1}{(2.404,5.727)}
\gppoint{gp mark 1}{(2.406,5.284)}
\gppoint{gp mark 1}{(2.409,4.547)}
\gppoint{gp mark 1}{(2.411,5.732)}
\gppoint{gp mark 1}{(2.414,5.710)}
\gppoint{gp mark 1}{(2.416,5.164)}
\gppoint{gp mark 1}{(2.419,4.732)}
\gppoint{gp mark 1}{(2.421,4.700)}
\gppoint{gp mark 1}{(2.423,6.218)}
\gppoint{gp mark 1}{(2.426,5.104)}
\gppoint{gp mark 1}{(2.428,5.093)}
\gppoint{gp mark 1}{(2.431,5.393)}
\gppoint{gp mark 1}{(2.433,4.995)}
\gppoint{gp mark 1}{(2.436,4.929)}
\gppoint{gp mark 1}{(2.438,5.503)}
\gppoint{gp mark 1}{(2.441,4.836)}
\gppoint{gp mark 1}{(2.443,5.328)}
\gppoint{gp mark 1}{(2.446,5.978)}
\gppoint{gp mark 1}{(2.448,5.514)}
\gppoint{gp mark 1}{(2.451,5.219)}
\gppoint{gp mark 1}{(2.453,4.869)}
\gppoint{gp mark 1}{(2.456,5.011)}
\gppoint{gp mark 1}{(2.458,5.137)}
\gppoint{gp mark 1}{(2.461,5.169)}
\gppoint{gp mark 1}{(2.463,5.393)}
\gppoint{gp mark 1}{(2.466,5.153)}
\gppoint{gp mark 1}{(2.468,5.432)}
\gppoint{gp mark 1}{(2.471,5.677)}
\gppoint{gp mark 1}{(2.473,4.945)}
\gppoint{gp mark 1}{(2.476,5.290)}
\gppoint{gp mark 1}{(2.478,4.831)}
\gppoint{gp mark 1}{(2.481,5.568)}
\gppoint{gp mark 1}{(2.483,5.142)}
\gppoint{gp mark 1}{(2.486,5.295)}
\gppoint{gp mark 1}{(2.488,4.896)}
\gppoint{gp mark 1}{(2.491,5.082)}
\gppoint{gp mark 1}{(2.493,4.634)}
\gppoint{gp mark 1}{(2.496,4.847)}
\gppoint{gp mark 1}{(2.498,4.366)}
\gppoint{gp mark 1}{(2.501,5.677)}
\gppoint{gp mark 1}{(2.503,5.164)}
\gppoint{gp mark 1}{(2.506,5.601)}
\gppoint{gp mark 1}{(2.508,4.716)}
\gppoint{gp mark 1}{(2.510,4.803)}
\gppoint{gp mark 1}{(2.513,4.929)}
\gppoint{gp mark 1}{(2.515,5.257)}
\gppoint{gp mark 1}{(2.518,4.989)}
\gppoint{gp mark 1}{(2.520,5.421)}
\gppoint{gp mark 1}{(2.523,5.404)}
\gppoint{gp mark 1}{(2.525,5.290)}
\gppoint{gp mark 1}{(2.528,5.574)}
\gppoint{gp mark 1}{(2.530,5.033)}
\gppoint{gp mark 1}{(2.533,5.038)}
\gppoint{gp mark 1}{(2.535,4.891)}
\gppoint{gp mark 1}{(2.538,4.792)}
\gppoint{gp mark 1}{(2.540,5.557)}
\gppoint{gp mark 1}{(2.543,4.924)}
\gppoint{gp mark 1}{(2.545,5.688)}
\gppoint{gp mark 1}{(2.548,5.393)}
\gppoint{gp mark 1}{(2.550,4.727)}
\gppoint{gp mark 1}{(2.553,4.650)}
\gppoint{gp mark 1}{(2.555,5.404)}
\gppoint{gp mark 1}{(2.558,5.432)}
\gppoint{gp mark 1}{(2.560,5.077)}
\gppoint{gp mark 1}{(2.563,5.164)}
\gppoint{gp mark 1}{(2.565,4.825)}
\gppoint{gp mark 1}{(2.568,5.169)}
\gppoint{gp mark 1}{(2.570,4.853)}
\gppoint{gp mark 1}{(2.573,5.186)}
\gppoint{gp mark 1}{(2.575,5.273)}
\gppoint{gp mark 1}{(2.578,5.650)}
\gppoint{gp mark 1}{(2.580,5.907)}
\gppoint{gp mark 1}{(2.583,4.667)}
\gppoint{gp mark 1}{(2.585,5.213)}
\gppoint{gp mark 1}{(2.588,4.978)}
\gppoint{gp mark 1}{(2.590,5.262)}
\gppoint{gp mark 1}{(2.592,5.016)}
\gppoint{gp mark 1}{(2.595,5.623)}
\gppoint{gp mark 1}{(2.597,4.705)}
\gppoint{gp mark 1}{(2.600,5.180)}
\gppoint{gp mark 1}{(2.602,5.415)}
\gppoint{gp mark 1}{(2.605,5.492)}
\gppoint{gp mark 1}{(2.607,5.087)}
\gppoint{gp mark 1}{(2.610,5.503)}
\gppoint{gp mark 1}{(2.612,5.006)}
\gppoint{gp mark 1}{(2.615,5.355)}
\gppoint{gp mark 1}{(2.617,5.202)}
\gppoint{gp mark 1}{(2.620,5.421)}
\gppoint{gp mark 1}{(2.622,5.197)}
\gppoint{gp mark 1}{(2.625,5.082)}
\gppoint{gp mark 1}{(2.627,4.514)}
\gppoint{gp mark 1}{(2.630,4.847)}
\gppoint{gp mark 1}{(2.632,5.322)}
\gppoint{gp mark 1}{(2.635,4.579)}
\gppoint{gp mark 1}{(2.637,5.311)}
\gppoint{gp mark 1}{(2.640,5.038)}
\gppoint{gp mark 1}{(2.642,4.284)}
\gppoint{gp mark 1}{(2.645,5.770)}
\gppoint{gp mark 1}{(2.647,5.382)}
\gppoint{gp mark 1}{(2.650,4.749)}
\gppoint{gp mark 1}{(2.652,5.464)}
\gppoint{gp mark 1}{(2.655,5.142)}
\gppoint{gp mark 1}{(2.657,5.109)}
\gppoint{gp mark 1}{(2.660,4.590)}
\gppoint{gp mark 1}{(2.662,4.864)}
\gppoint{gp mark 1}{(2.665,5.066)}
\gppoint{gp mark 1}{(2.667,4.246)}
\gppoint{gp mark 1}{(2.670,4.984)}
\gppoint{gp mark 1}{(2.672,5.213)}
\gppoint{gp mark 1}{(2.674,5.404)}
\gppoint{gp mark 1}{(2.677,4.579)}
\gppoint{gp mark 1}{(2.679,5.251)}
\gppoint{gp mark 1}{(2.682,5.677)}
\gppoint{gp mark 1}{(2.684,5.535)}
\gppoint{gp mark 1}{(2.687,6.131)}
\gppoint{gp mark 1}{(2.689,4.645)}
\gppoint{gp mark 1}{(2.692,4.951)}
\gppoint{gp mark 1}{(2.694,4.956)}
\gppoint{gp mark 1}{(2.697,4.907)}
\gppoint{gp mark 1}{(2.699,4.694)}
\gppoint{gp mark 1}{(2.702,4.334)}
\gppoint{gp mark 1}{(2.704,4.175)}
\gppoint{gp mark 1}{(2.707,5.464)}
\gppoint{gp mark 1}{(2.709,5.240)}
\gppoint{gp mark 1}{(2.712,5.093)}
\gppoint{gp mark 1}{(2.714,5.093)}
\gppoint{gp mark 1}{(2.717,5.279)}
\gppoint{gp mark 1}{(2.719,4.776)}
\gppoint{gp mark 1}{(2.722,5.180)}
\gppoint{gp mark 1}{(2.724,4.694)}
\gppoint{gp mark 1}{(2.727,4.164)}
\gppoint{gp mark 1}{(2.729,5.137)}
\gppoint{gp mark 1}{(2.732,5.475)}
\gppoint{gp mark 1}{(2.734,5.027)}
\gppoint{gp mark 1}{(2.737,5.257)}
\gppoint{gp mark 1}{(2.739,4.754)}
\gppoint{gp mark 1}{(2.742,5.169)}
\gppoint{gp mark 1}{(2.744,4.656)}
\gppoint{gp mark 1}{(2.747,5.169)}
\gppoint{gp mark 1}{(2.749,5.552)}
\gppoint{gp mark 1}{(2.752,4.880)}
\gppoint{gp mark 1}{(2.754,5.268)}
\gppoint{gp mark 1}{(2.757,5.399)}
\gppoint{gp mark 1}{(2.759,4.388)}
\gppoint{gp mark 1}{(2.761,4.607)}
\gppoint{gp mark 1}{(2.764,4.738)}
\gppoint{gp mark 1}{(2.766,5.596)}
\gppoint{gp mark 1}{(2.769,4.416)}
\gppoint{gp mark 1}{(2.771,5.410)}
\gppoint{gp mark 1}{(2.774,4.825)}
\gppoint{gp mark 1}{(2.776,5.568)}
\gppoint{gp mark 1}{(2.779,5.191)}
\gppoint{gp mark 1}{(2.781,5.109)}
\gppoint{gp mark 1}{(2.784,4.836)}
\gppoint{gp mark 1}{(2.786,4.995)}
\gppoint{gp mark 1}{(2.789,4.814)}
\gppoint{gp mark 1}{(2.791,4.978)}
\gppoint{gp mark 1}{(2.794,5.858)}
\gppoint{gp mark 1}{(2.796,5.284)}
\gppoint{gp mark 1}{(2.799,5.137)}
\gppoint{gp mark 1}{(2.801,5.301)}
\gppoint{gp mark 1}{(2.804,4.590)}
\gppoint{gp mark 1}{(2.806,5.333)}
\gppoint{gp mark 1}{(2.809,4.782)}
\gppoint{gp mark 1}{(2.811,5.082)}
\gppoint{gp mark 1}{(2.814,5.503)}
\gppoint{gp mark 1}{(2.816,5.208)}
\gppoint{gp mark 1}{(2.819,5.158)}
\gppoint{gp mark 1}{(2.821,5.142)}
\gppoint{gp mark 1}{(2.824,5.448)}
\gppoint{gp mark 1}{(2.826,5.126)}
\gppoint{gp mark 1}{(2.829,4.579)}
\gppoint{gp mark 1}{(2.831,5.437)}
\gppoint{gp mark 1}{(2.834,5.115)}
\gppoint{gp mark 1}{(2.836,5.055)}
\gppoint{gp mark 1}{(2.839,5.197)}
\gppoint{gp mark 1}{(2.841,4.973)}
\gppoint{gp mark 1}{(2.843,4.448)}
\gppoint{gp mark 1}{(2.846,4.711)}
\gppoint{gp mark 1}{(2.848,4.672)}
\gppoint{gp mark 1}{(2.851,4.465)}
\gppoint{gp mark 1}{(2.853,5.202)}
\gppoint{gp mark 1}{(2.856,4.312)}
\gppoint{gp mark 1}{(2.858,4.252)}
\gppoint{gp mark 1}{(2.861,5.006)}
\gppoint{gp mark 1}{(2.863,5.809)}
\gppoint{gp mark 1}{(2.866,5.503)}
\gppoint{gp mark 1}{(2.868,4.732)}
\gppoint{gp mark 1}{(2.871,4.437)}
\gppoint{gp mark 1}{(2.873,4.896)}
\gppoint{gp mark 1}{(2.876,4.579)}
\gppoint{gp mark 1}{(2.878,5.322)}
\gppoint{gp mark 1}{(2.881,4.148)}
\gppoint{gp mark 1}{(2.883,4.995)}
\gppoint{gp mark 1}{(2.886,4.798)}
\gppoint{gp mark 1}{(2.888,5.126)}
\gppoint{gp mark 1}{(2.891,4.836)}
\gppoint{gp mark 1}{(2.893,5.432)}
\gppoint{gp mark 1}{(2.896,5.180)}
\gppoint{gp mark 1}{(2.898,4.574)}
\gppoint{gp mark 1}{(2.901,5.601)}
\gppoint{gp mark 1}{(2.903,4.514)}
\gppoint{gp mark 1}{(2.906,5.836)}
\gppoint{gp mark 1}{(2.908,4.951)}
\gppoint{gp mark 1}{(2.911,5.350)}
\gppoint{gp mark 1}{(2.913,4.350)}
\gppoint{gp mark 1}{(2.916,4.711)}
\gppoint{gp mark 1}{(2.918,4.672)}
\gppoint{gp mark 1}{(2.921,4.689)}
\gppoint{gp mark 1}{(2.923,5.022)}
\gppoint{gp mark 1}{(2.925,4.596)}
\gppoint{gp mark 1}{(2.928,3.989)}
\gppoint{gp mark 1}{(2.930,4.891)}
\gppoint{gp mark 1}{(2.933,4.831)}
\gppoint{gp mark 1}{(2.935,5.311)}
\gppoint{gp mark 1}{(2.938,5.732)}
\gppoint{gp mark 1}{(2.940,4.891)}
\gppoint{gp mark 1}{(2.943,5.284)}
\gppoint{gp mark 1}{(2.945,4.809)}
\gppoint{gp mark 1}{(2.948,4.672)}
\gppoint{gp mark 1}{(2.950,4.727)}
\gppoint{gp mark 1}{(2.953,5.677)}
\gppoint{gp mark 1}{(2.955,4.547)}
\gppoint{gp mark 1}{(2.958,4.924)}
\gppoint{gp mark 1}{(2.960,5.120)}
\gppoint{gp mark 1}{(2.963,4.809)}
\gppoint{gp mark 1}{(2.965,4.558)}
\gppoint{gp mark 1}{(2.968,4.711)}
\gppoint{gp mark 1}{(2.970,5.273)}
\gppoint{gp mark 1}{(2.973,4.842)}
\gppoint{gp mark 1}{(2.975,4.803)}
\gppoint{gp mark 1}{(2.978,4.388)}
\gppoint{gp mark 1}{(2.980,5.339)}
\gppoint{gp mark 1}{(2.983,4.203)}
\gppoint{gp mark 1}{(2.985,5.093)}
\gppoint{gp mark 1}{(2.988,5.104)}
\gppoint{gp mark 1}{(2.990,5.284)}
\gppoint{gp mark 1}{(2.993,5.082)}
\gppoint{gp mark 1}{(2.995,5.158)}
\gppoint{gp mark 1}{(2.998,4.121)}
\gppoint{gp mark 1}{(3.000,5.497)}
\gppoint{gp mark 1}{(3.003,4.650)}
\gppoint{gp mark 1}{(3.005,5.344)}
\gppoint{gp mark 1}{(3.008,5.109)}
\gppoint{gp mark 1}{(3.010,5.257)}
\gppoint{gp mark 1}{(3.012,4.683)}
\gppoint{gp mark 1}{(3.015,4.689)}
\gppoint{gp mark 1}{(3.017,5.437)}
\gppoint{gp mark 1}{(3.020,4.159)}
\gppoint{gp mark 1}{(3.022,5.066)}
\gppoint{gp mark 1}{(3.025,4.973)}
\gppoint{gp mark 1}{(3.027,5.164)}
\gppoint{gp mark 1}{(3.030,5.077)}
\gppoint{gp mark 1}{(3.032,4.853)}
\gppoint{gp mark 1}{(3.035,4.820)}
\gppoint{gp mark 1}{(3.037,4.711)}
\gppoint{gp mark 1}{(3.040,4.465)}
\gppoint{gp mark 1}{(3.042,4.667)}
\gppoint{gp mark 1}{(3.045,5.082)}
\gppoint{gp mark 1}{(3.047,4.831)}
\gppoint{gp mark 1}{(3.050,4.853)}
\gppoint{gp mark 1}{(3.052,4.661)}
\gppoint{gp mark 1}{(3.055,4.536)}
\gppoint{gp mark 1}{(3.057,4.192)}
\gppoint{gp mark 1}{(3.060,5.087)}
\gppoint{gp mark 1}{(3.062,4.481)}
\gppoint{gp mark 1}{(3.065,4.792)}
\gppoint{gp mark 1}{(3.067,4.825)}
\gppoint{gp mark 1}{(3.070,4.727)}
\gppoint{gp mark 1}{(3.072,4.569)}
\gppoint{gp mark 1}{(3.075,4.935)}
\gppoint{gp mark 1}{(3.077,4.044)}
\gppoint{gp mark 1}{(3.080,4.612)}
\gppoint{gp mark 1}{(3.082,5.262)}
\gppoint{gp mark 1}{(3.085,4.345)}
\gppoint{gp mark 1}{(3.087,4.754)}
\gppoint{gp mark 1}{(3.090,4.798)}
\gppoint{gp mark 1}{(3.092,4.547)}
\gppoint{gp mark 1}{(3.094,5.322)}
\gppoint{gp mark 1}{(3.097,5.087)}
\gppoint{gp mark 1}{(3.099,4.361)}
\gppoint{gp mark 1}{(3.102,4.995)}
\gppoint{gp mark 1}{(3.104,5.273)}
\gppoint{gp mark 1}{(3.107,5.393)}
\gppoint{gp mark 1}{(3.109,4.547)}
\gppoint{gp mark 1}{(3.112,4.317)}
\gppoint{gp mark 1}{(3.114,5.027)}
\gppoint{gp mark 1}{(3.117,4.902)}
\gppoint{gp mark 1}{(3.119,4.951)}
\gppoint{gp mark 1}{(3.122,5.825)}
\gppoint{gp mark 1}{(3.124,4.880)}
\gppoint{gp mark 1}{(3.127,4.350)}
\gppoint{gp mark 1}{(3.129,4.738)}
\gppoint{gp mark 1}{(3.132,4.476)}
\gppoint{gp mark 1}{(3.134,5.049)}
\gppoint{gp mark 1}{(3.137,5.022)}
\gppoint{gp mark 1}{(3.139,4.246)}
\gppoint{gp mark 1}{(3.142,4.498)}
\gppoint{gp mark 1}{(3.144,4.612)}
\gppoint{gp mark 1}{(3.147,5.098)}
\gppoint{gp mark 1}{(3.149,4.678)}
\gppoint{gp mark 1}{(3.152,4.678)}
\gppoint{gp mark 1}{(3.154,4.683)}
\gppoint{gp mark 1}{(3.157,4.667)}
\gppoint{gp mark 1}{(3.159,4.727)}
\gppoint{gp mark 1}{(3.162,4.416)}
\gppoint{gp mark 1}{(3.164,5.481)}
\gppoint{gp mark 1}{(3.167,4.050)}
\gppoint{gp mark 1}{(3.169,5.284)}
\gppoint{gp mark 1}{(3.172,5.158)}
\gppoint{gp mark 1}{(3.174,4.956)}
\gppoint{gp mark 1}{(3.176,4.448)}
\gppoint{gp mark 1}{(3.179,5.202)}
\gppoint{gp mark 1}{(3.181,4.640)}
\gppoint{gp mark 1}{(3.184,4.749)}
\gppoint{gp mark 1}{(3.186,4.230)}
\gppoint{gp mark 1}{(3.189,5.410)}
\gppoint{gp mark 1}{(3.191,5.224)}
\gppoint{gp mark 1}{(3.194,4.792)}
\gppoint{gp mark 1}{(3.196,4.579)}
\gppoint{gp mark 1}{(3.199,5.311)}
\gppoint{gp mark 1}{(3.201,4.967)}
\gppoint{gp mark 1}{(3.204,4.798)}
\gppoint{gp mark 1}{(3.206,4.880)}
\gppoint{gp mark 1}{(3.209,5.202)}
\gppoint{gp mark 1}{(3.211,4.672)}
\gppoint{gp mark 1}{(3.214,4.973)}
\gppoint{gp mark 1}{(3.216,4.241)}
\gppoint{gp mark 1}{(3.219,4.989)}
\gppoint{gp mark 1}{(3.221,4.437)}
\gppoint{gp mark 1}{(3.224,5.066)}
\gppoint{gp mark 1}{(3.226,4.792)}
\gppoint{gp mark 1}{(3.229,4.601)}
\gppoint{gp mark 1}{(3.231,4.694)}
\gppoint{gp mark 1}{(3.234,4.771)}
\gppoint{gp mark 1}{(3.236,4.760)}
\gppoint{gp mark 1}{(3.239,4.760)}
\gppoint{gp mark 1}{(3.241,4.530)}
\gppoint{gp mark 1}{(3.244,4.661)}
\gppoint{gp mark 1}{(3.246,4.754)}
\gppoint{gp mark 1}{(3.249,4.853)}
\gppoint{gp mark 1}{(3.251,4.426)}
\gppoint{gp mark 1}{(3.254,4.601)}
\gppoint{gp mark 1}{(3.256,4.711)}
\gppoint{gp mark 1}{(3.259,4.405)}
\gppoint{gp mark 1}{(3.261,4.902)}
\gppoint{gp mark 1}{(3.263,4.694)}
\gppoint{gp mark 1}{(3.266,4.765)}
\gppoint{gp mark 1}{(3.268,4.448)}
\gppoint{gp mark 1}{(3.271,4.831)}
\gppoint{gp mark 1}{(3.273,5.361)}
\gppoint{gp mark 1}{(3.276,4.476)}
\gppoint{gp mark 1}{(3.278,4.426)}
\gppoint{gp mark 1}{(3.281,4.831)}
\gppoint{gp mark 1}{(3.283,4.913)}
\gppoint{gp mark 1}{(3.286,5.273)}
\gppoint{gp mark 1}{(3.288,5.049)}
\gppoint{gp mark 1}{(3.291,5.115)}
\gppoint{gp mark 1}{(3.293,4.312)}
\gppoint{gp mark 1}{(3.296,5.284)}
\gppoint{gp mark 1}{(3.298,5.361)}
\gppoint{gp mark 1}{(3.301,4.874)}
\gppoint{gp mark 1}{(3.303,4.629)}
\gppoint{gp mark 1}{(3.306,4.918)}
\gppoint{gp mark 1}{(3.308,4.590)}
\gppoint{gp mark 1}{(3.311,4.902)}
\gppoint{gp mark 1}{(3.313,4.142)}
\gppoint{gp mark 1}{(3.316,4.410)}
\gppoint{gp mark 1}{(3.318,4.410)}
\gppoint{gp mark 1}{(3.321,4.618)}
\gppoint{gp mark 1}{(3.323,4.776)}
\gppoint{gp mark 1}{(3.326,5.251)}
\gppoint{gp mark 1}{(3.328,3.793)}
\gppoint{gp mark 1}{(3.331,4.689)}
\gppoint{gp mark 1}{(3.333,4.798)}
\gppoint{gp mark 1}{(3.336,3.798)}
\gppoint{gp mark 1}{(3.338,4.667)}
\gppoint{gp mark 1}{(3.341,4.951)}
\gppoint{gp mark 1}{(3.343,4.454)}
\gppoint{gp mark 1}{(3.345,4.498)}
\gppoint{gp mark 1}{(3.348,4.787)}
\gppoint{gp mark 1}{(3.350,4.443)}
\gppoint{gp mark 1}{(3.353,4.650)}
\gppoint{gp mark 1}{(3.355,4.536)}
\gppoint{gp mark 1}{(3.358,4.995)}
\gppoint{gp mark 1}{(3.360,4.372)}
\gppoint{gp mark 1}{(3.363,4.650)}
\gppoint{gp mark 1}{(3.365,4.645)}
\gppoint{gp mark 1}{(3.368,5.164)}
\gppoint{gp mark 1}{(3.370,4.853)}
\gppoint{gp mark 1}{(3.373,4.749)}
\gppoint{gp mark 1}{(3.375,4.913)}
\gppoint{gp mark 1}{(3.378,4.050)}
\gppoint{gp mark 1}{(3.380,4.967)}
\gppoint{gp mark 1}{(3.383,5.492)}
\gppoint{gp mark 1}{(3.385,5.000)}
\gppoint{gp mark 1}{(3.388,4.443)}
\gppoint{gp mark 1}{(3.390,4.727)}
\gppoint{gp mark 1}{(3.393,5.066)}
\gppoint{gp mark 1}{(3.395,5.016)}
\gppoint{gp mark 1}{(3.398,4.645)}
\gppoint{gp mark 1}{(3.400,4.798)}
\gppoint{gp mark 1}{(3.403,5.006)}
\gppoint{gp mark 1}{(3.405,4.508)}
\gppoint{gp mark 1}{(3.408,4.448)}
\gppoint{gp mark 1}{(3.410,4.530)}
\gppoint{gp mark 1}{(3.413,4.618)}
\gppoint{gp mark 1}{(3.415,5.087)}
\gppoint{gp mark 1}{(3.418,4.711)}
\gppoint{gp mark 1}{(3.420,4.585)}
\gppoint{gp mark 1}{(3.423,4.869)}
\gppoint{gp mark 1}{(3.425,3.755)}
\gppoint{gp mark 1}{(3.427,5.082)}
\gppoint{gp mark 1}{(3.430,4.142)}
\gppoint{gp mark 1}{(3.432,5.437)}
\gppoint{gp mark 1}{(3.435,4.355)}
\gppoint{gp mark 1}{(3.437,4.432)}
\gppoint{gp mark 1}{(3.440,4.274)}
\gppoint{gp mark 1}{(3.442,5.453)}
\gppoint{gp mark 1}{(3.445,4.508)}
\gppoint{gp mark 1}{(3.447,5.115)}
\gppoint{gp mark 1}{(3.450,4.459)}
\gppoint{gp mark 1}{(3.452,5.497)}
\gppoint{gp mark 1}{(3.455,4.727)}
\gppoint{gp mark 1}{(3.457,4.317)}
\gppoint{gp mark 1}{(3.460,4.487)}
\gppoint{gp mark 1}{(3.462,4.448)}
\gppoint{gp mark 1}{(3.465,5.257)}
\gppoint{gp mark 1}{(3.467,5.202)}
\gppoint{gp mark 1}{(3.470,5.060)}
\gppoint{gp mark 1}{(3.472,4.596)}
\gppoint{gp mark 1}{(3.475,4.536)}
\gppoint{gp mark 1}{(3.477,4.694)}
\gppoint{gp mark 1}{(3.480,4.721)}
\gppoint{gp mark 1}{(3.482,5.164)}
\gppoint{gp mark 1}{(3.485,4.487)}
\gppoint{gp mark 1}{(3.487,4.896)}
\gppoint{gp mark 1}{(3.490,4.290)}
\gppoint{gp mark 1}{(3.492,4.399)}
\gppoint{gp mark 1}{(3.495,4.760)}
\gppoint{gp mark 1}{(3.497,5.601)}
\gppoint{gp mark 1}{(3.500,4.940)}
\gppoint{gp mark 1}{(3.502,4.383)}
\gppoint{gp mark 1}{(3.505,4.678)}
\gppoint{gp mark 1}{(3.507,4.831)}
\gppoint{gp mark 1}{(3.509,4.547)}
\gppoint{gp mark 1}{(3.512,5.344)}
\gppoint{gp mark 1}{(3.514,4.667)}
\gppoint{gp mark 1}{(3.517,4.891)}
\gppoint{gp mark 1}{(3.519,5.115)}
\gppoint{gp mark 1}{(3.522,4.508)}
\gppoint{gp mark 1}{(3.524,4.405)}
\gppoint{gp mark 1}{(3.527,4.328)}
\gppoint{gp mark 1}{(3.529,4.732)}
\gppoint{gp mark 1}{(3.532,4.208)}
\gppoint{gp mark 1}{(3.534,4.661)}
\gppoint{gp mark 1}{(3.537,4.803)}
\gppoint{gp mark 1}{(3.539,4.831)}
\gppoint{gp mark 1}{(3.542,5.115)}
\gppoint{gp mark 1}{(3.544,4.132)}
\gppoint{gp mark 1}{(3.547,4.519)}
\gppoint{gp mark 1}{(3.549,4.492)}
\gppoint{gp mark 1}{(3.552,4.574)}
\gppoint{gp mark 1}{(3.554,5.230)}
\gppoint{gp mark 1}{(3.557,3.989)}
\gppoint{gp mark 1}{(3.559,4.727)}
\gppoint{gp mark 1}{(3.562,4.700)}
\gppoint{gp mark 1}{(3.564,4.213)}
\gppoint{gp mark 1}{(3.567,4.732)}
\gppoint{gp mark 1}{(3.569,4.607)}
\gppoint{gp mark 1}{(3.572,4.437)}
\gppoint{gp mark 1}{(3.574,5.415)}
\gppoint{gp mark 1}{(3.577,4.142)}
\gppoint{gp mark 1}{(3.579,4.640)}
\gppoint{gp mark 1}{(3.582,4.749)}
\gppoint{gp mark 1}{(3.584,4.492)}
\gppoint{gp mark 1}{(3.587,4.596)}
\gppoint{gp mark 1}{(3.589,4.978)}
\gppoint{gp mark 1}{(3.592,4.541)}
\gppoint{gp mark 1}{(3.594,4.951)}
\gppoint{gp mark 1}{(3.596,4.913)}
\gppoint{gp mark 1}{(3.599,4.230)}
\gppoint{gp mark 1}{(3.601,4.498)}
\gppoint{gp mark 1}{(3.604,5.044)}
\gppoint{gp mark 1}{(3.606,4.437)}
\gppoint{gp mark 1}{(3.609,4.678)}
\gppoint{gp mark 1}{(3.611,4.055)}
\gppoint{gp mark 1}{(3.614,4.814)}
\gppoint{gp mark 1}{(3.616,4.306)}
\gppoint{gp mark 1}{(3.619,3.820)}
\gppoint{gp mark 1}{(3.621,4.350)}
\gppoint{gp mark 1}{(3.624,4.831)}
\gppoint{gp mark 1}{(3.626,4.803)}
\gppoint{gp mark 1}{(3.629,4.044)}
\gppoint{gp mark 1}{(3.631,3.908)}
\gppoint{gp mark 1}{(3.634,4.776)}
\gppoint{gp mark 1}{(3.636,5.284)}
\gppoint{gp mark 1}{(3.639,4.924)}
\gppoint{gp mark 1}{(3.641,5.033)}
\gppoint{gp mark 1}{(3.644,4.170)}
\gppoint{gp mark 1}{(3.646,4.213)}
\gppoint{gp mark 1}{(3.649,4.913)}
\gppoint{gp mark 1}{(3.651,4.110)}
\gppoint{gp mark 1}{(3.654,4.765)}
\gppoint{gp mark 1}{(3.656,5.158)}
\gppoint{gp mark 1}{(3.659,4.612)}
\gppoint{gp mark 1}{(3.661,4.738)}
\gppoint{gp mark 1}{(3.664,5.120)}
\gppoint{gp mark 1}{(3.666,4.432)}
\gppoint{gp mark 1}{(3.669,4.432)}
\gppoint{gp mark 1}{(3.671,4.033)}
\gppoint{gp mark 1}{(3.674,4.050)}
\gppoint{gp mark 1}{(3.676,4.301)}
\gppoint{gp mark 1}{(3.678,4.295)}
\gppoint{gp mark 1}{(3.681,3.902)}
\gppoint{gp mark 1}{(3.683,5.115)}
\gppoint{gp mark 1}{(3.686,3.247)}
\gppoint{gp mark 1}{(3.688,4.962)}
\gppoint{gp mark 1}{(3.691,4.339)}
\gppoint{gp mark 1}{(3.693,5.175)}
\gppoint{gp mark 1}{(3.696,4.399)}
\gppoint{gp mark 1}{(3.698,4.295)}
\gppoint{gp mark 1}{(3.701,4.377)}
\gppoint{gp mark 1}{(3.703,4.175)}
\gppoint{gp mark 1}{(3.706,4.394)}
\gppoint{gp mark 1}{(3.708,5.033)}
\gppoint{gp mark 1}{(3.711,4.148)}
\gppoint{gp mark 1}{(3.713,4.547)}
\gppoint{gp mark 1}{(3.716,4.339)}
\gppoint{gp mark 1}{(3.718,4.137)}
\gppoint{gp mark 1}{(3.721,3.962)}
\gppoint{gp mark 1}{(3.723,4.388)}
\gppoint{gp mark 1}{(3.726,4.208)}
\gppoint{gp mark 1}{(3.728,4.525)}
\gppoint{gp mark 1}{(3.731,4.437)}
\gppoint{gp mark 1}{(3.733,4.388)}
\gppoint{gp mark 1}{(3.736,4.328)}
\gppoint{gp mark 1}{(3.738,4.738)}
\gppoint{gp mark 1}{(3.741,4.749)}
\gppoint{gp mark 1}{(3.743,4.678)}
\gppoint{gp mark 1}{(3.746,4.175)}
\gppoint{gp mark 1}{(3.748,4.956)}
\gppoint{gp mark 1}{(3.751,4.765)}
\gppoint{gp mark 1}{(3.753,4.465)}
\gppoint{gp mark 1}{(3.756,4.437)}
\gppoint{gp mark 1}{(3.758,4.241)}
\gppoint{gp mark 1}{(3.760,4.711)}
\gppoint{gp mark 1}{(3.763,4.656)}
\gppoint{gp mark 1}{(3.765,4.858)}
\gppoint{gp mark 1}{(3.768,4.033)}
\gppoint{gp mark 1}{(3.770,5.000)}
\gppoint{gp mark 1}{(3.773,5.131)}
\gppoint{gp mark 1}{(3.775,3.908)}
\gppoint{gp mark 1}{(3.778,4.749)}
\gppoint{gp mark 1}{(3.780,4.547)}
\gppoint{gp mark 1}{(3.783,5.317)}
\gppoint{gp mark 1}{(3.785,3.722)}
\gppoint{gp mark 1}{(3.788,4.836)}
\gppoint{gp mark 1}{(3.790,4.525)}
\gppoint{gp mark 1}{(3.793,3.962)}
\gppoint{gp mark 1}{(3.795,4.864)}
\gppoint{gp mark 1}{(3.798,4.142)}
\gppoint{gp mark 1}{(3.800,3.372)}
\gppoint{gp mark 1}{(3.803,4.279)}
\gppoint{gp mark 1}{(3.805,5.344)}
\gppoint{gp mark 1}{(3.808,4.132)}
\gppoint{gp mark 1}{(3.810,4.192)}
\gppoint{gp mark 1}{(3.813,4.650)}
\gppoint{gp mark 1}{(3.815,4.443)}
\gppoint{gp mark 1}{(3.818,4.235)}
\gppoint{gp mark 1}{(3.820,4.585)}
\gppoint{gp mark 1}{(3.823,5.404)}
\gppoint{gp mark 1}{(3.825,4.694)}
\gppoint{gp mark 1}{(3.828,4.503)}
\gppoint{gp mark 1}{(3.830,4.093)}
\gppoint{gp mark 1}{(3.833,4.831)}
\gppoint{gp mark 1}{(3.835,4.323)}
\gppoint{gp mark 1}{(3.838,4.470)}
\gppoint{gp mark 1}{(3.840,4.700)}
\gppoint{gp mark 1}{(3.843,4.678)}
\gppoint{gp mark 1}{(3.845,4.618)}
\gppoint{gp mark 1}{(3.847,4.235)}
\gppoint{gp mark 1}{(3.850,5.006)}
\gppoint{gp mark 1}{(3.852,4.661)}
\gppoint{gp mark 1}{(3.855,5.049)}
\gppoint{gp mark 1}{(3.857,4.459)}
\gppoint{gp mark 1}{(3.860,4.224)}
\gppoint{gp mark 1}{(3.862,4.437)}
\gppoint{gp mark 1}{(3.865,4.727)}
\gppoint{gp mark 1}{(3.867,4.541)}
\gppoint{gp mark 1}{(3.870,4.328)}
\gppoint{gp mark 1}{(3.872,4.011)}
\gppoint{gp mark 1}{(3.875,4.000)}
\gppoint{gp mark 1}{(3.877,3.487)}
\gppoint{gp mark 1}{(3.880,4.612)}
\gppoint{gp mark 1}{(3.882,5.071)}
\gppoint{gp mark 1}{(3.885,3.591)}
\gppoint{gp mark 1}{(3.887,3.968)}
\gppoint{gp mark 1}{(3.890,4.235)}
\gppoint{gp mark 1}{(3.892,4.880)}
\gppoint{gp mark 1}{(3.895,4.667)}
\gppoint{gp mark 1}{(3.897,4.508)}
\gppoint{gp mark 1}{(3.900,4.962)}
\gppoint{gp mark 1}{(3.902,4.093)}
\gppoint{gp mark 1}{(3.905,4.650)}
\gppoint{gp mark 1}{(3.907,4.377)}
\gppoint{gp mark 1}{(3.910,4.645)}
\gppoint{gp mark 1}{(3.912,4.361)}
\gppoint{gp mark 1}{(3.915,4.295)}
\gppoint{gp mark 1}{(3.917,4.208)}
\gppoint{gp mark 1}{(3.920,4.443)}
\gppoint{gp mark 1}{(3.922,4.274)}
\gppoint{gp mark 1}{(3.925,3.864)}
\gppoint{gp mark 1}{(3.927,4.006)}
\gppoint{gp mark 1}{(3.929,4.476)}
\gppoint{gp mark 1}{(3.932,3.979)}
\gppoint{gp mark 1}{(3.934,4.487)}
\gppoint{gp mark 1}{(3.937,5.082)}
\gppoint{gp mark 1}{(3.939,4.525)}
\gppoint{gp mark 1}{(3.942,3.776)}
\gppoint{gp mark 1}{(3.944,4.487)}
\gppoint{gp mark 1}{(3.947,4.110)}
\gppoint{gp mark 1}{(3.949,4.377)}
\gppoint{gp mark 1}{(3.952,4.372)}
\gppoint{gp mark 1}{(3.954,4.339)}
\gppoint{gp mark 1}{(3.957,4.274)}
\gppoint{gp mark 1}{(3.959,4.814)}
\gppoint{gp mark 1}{(3.962,4.142)}
\gppoint{gp mark 1}{(3.964,4.711)}
\gppoint{gp mark 1}{(3.967,3.689)}
\gppoint{gp mark 1}{(3.969,4.110)}
\gppoint{gp mark 1}{(3.972,3.979)}
\gppoint{gp mark 1}{(3.974,4.110)}
\gppoint{gp mark 1}{(3.977,3.738)}
\gppoint{gp mark 1}{(3.979,4.317)}
\gppoint{gp mark 1}{(3.982,4.290)}
\gppoint{gp mark 1}{(3.984,4.372)}
\gppoint{gp mark 1}{(3.987,4.142)}
\gppoint{gp mark 1}{(3.989,3.700)}
\gppoint{gp mark 1}{(3.992,3.929)}
\gppoint{gp mark 1}{(3.994,4.525)}
\gppoint{gp mark 1}{(3.997,4.230)}
\gppoint{gp mark 1}{(3.999,3.547)}
\gppoint{gp mark 1}{(4.002,4.175)}
\gppoint{gp mark 1}{(4.004,4.235)}
\gppoint{gp mark 1}{(4.007,4.345)}
\gppoint{gp mark 1}{(4.009,3.678)}
\gppoint{gp mark 1}{(4.011,4.913)}
\gppoint{gp mark 1}{(4.014,3.716)}
\gppoint{gp mark 1}{(4.016,4.099)}
\gppoint{gp mark 1}{(4.019,4.765)}
\gppoint{gp mark 1}{(4.021,4.345)}
\gppoint{gp mark 1}{(4.024,4.366)}
\gppoint{gp mark 1}{(4.026,4.640)}
\gppoint{gp mark 1}{(4.029,4.121)}
\gppoint{gp mark 1}{(4.031,4.263)}
\gppoint{gp mark 1}{(4.034,4.481)}
\gppoint{gp mark 1}{(4.036,3.667)}
\gppoint{gp mark 1}{(4.039,4.355)}
\gppoint{gp mark 1}{(4.041,4.640)}
\gppoint{gp mark 1}{(4.044,4.508)}
\gppoint{gp mark 1}{(4.046,4.295)}
\gppoint{gp mark 1}{(4.049,5.148)}
\gppoint{gp mark 1}{(4.051,4.268)}
\gppoint{gp mark 1}{(4.054,4.623)}
\gppoint{gp mark 1}{(4.056,3.837)}
\gppoint{gp mark 1}{(4.059,4.945)}
\gppoint{gp mark 1}{(4.061,4.115)}
\gppoint{gp mark 1}{(4.064,4.000)}
\gppoint{gp mark 1}{(4.066,4.629)}
\gppoint{gp mark 1}{(4.069,4.607)}
\gppoint{gp mark 1}{(4.071,4.487)}
\gppoint{gp mark 1}{(4.074,3.968)}
\gppoint{gp mark 1}{(4.076,4.142)}
\gppoint{gp mark 1}{(4.079,4.099)}
\gppoint{gp mark 1}{(4.081,4.377)}
\gppoint{gp mark 1}{(4.084,3.569)}
\gppoint{gp mark 1}{(4.086,5.006)}
\gppoint{gp mark 1}{(4.089,4.230)}
\gppoint{gp mark 1}{(4.091,4.574)}
\gppoint{gp mark 1}{(4.094,3.913)}
\gppoint{gp mark 1}{(4.096,4.530)}
\gppoint{gp mark 1}{(4.098,4.323)}
\gppoint{gp mark 1}{(4.101,3.902)}
\gppoint{gp mark 1}{(4.103,4.814)}
\gppoint{gp mark 1}{(4.106,4.372)}
\gppoint{gp mark 1}{(4.108,4.711)}
\gppoint{gp mark 1}{(4.111,4.219)}
\gppoint{gp mark 1}{(4.113,4.257)}
\gppoint{gp mark 1}{(4.116,4.640)}
\gppoint{gp mark 1}{(4.118,4.831)}
\gppoint{gp mark 1}{(4.121,4.820)}
\gppoint{gp mark 1}{(4.123,4.312)}
\gppoint{gp mark 1}{(4.126,4.137)}
\gppoint{gp mark 1}{(4.128,4.164)}
\gppoint{gp mark 1}{(4.131,4.558)}
\gppoint{gp mark 1}{(4.133,4.213)}
\gppoint{gp mark 1}{(4.136,3.580)}
\gppoint{gp mark 1}{(4.138,4.148)}
\gppoint{gp mark 1}{(4.141,4.175)}
\gppoint{gp mark 1}{(4.143,4.629)}
\gppoint{gp mark 1}{(4.146,4.028)}
\gppoint{gp mark 1}{(4.148,3.826)}
\gppoint{gp mark 1}{(4.151,3.886)}
\gppoint{gp mark 1}{(4.153,4.924)}
\gppoint{gp mark 1}{(4.156,4.426)}
\gppoint{gp mark 1}{(4.158,4.579)}
\gppoint{gp mark 1}{(4.161,3.798)}
\gppoint{gp mark 1}{(4.163,4.743)}
\gppoint{gp mark 1}{(4.166,4.732)}
\gppoint{gp mark 1}{(4.168,3.782)}
\gppoint{gp mark 1}{(4.171,4.563)}
\gppoint{gp mark 1}{(4.173,4.066)}
\gppoint{gp mark 1}{(4.176,4.885)}
\gppoint{gp mark 1}{(4.178,4.197)}
\gppoint{gp mark 1}{(4.180,3.700)}
\gppoint{gp mark 1}{(4.183,3.946)}
\gppoint{gp mark 1}{(4.185,4.093)}
\gppoint{gp mark 1}{(4.188,4.601)}
\gppoint{gp mark 1}{(4.190,4.110)}
\gppoint{gp mark 1}{(4.193,4.060)}
\gppoint{gp mark 1}{(4.195,3.694)}
\gppoint{gp mark 1}{(4.198,4.399)}
\gppoint{gp mark 1}{(4.200,4.792)}
\gppoint{gp mark 1}{(4.203,4.175)}
\gppoint{gp mark 1}{(4.205,4.596)}
\gppoint{gp mark 1}{(4.208,4.798)}
\gppoint{gp mark 1}{(4.210,3.640)}
\gppoint{gp mark 1}{(4.213,4.328)}
\gppoint{gp mark 1}{(4.215,4.170)}
\gppoint{gp mark 1}{(4.218,3.656)}
\gppoint{gp mark 1}{(4.220,3.689)}
\gppoint{gp mark 1}{(4.223,4.252)}
\gppoint{gp mark 1}{(4.225,4.203)}
\gppoint{gp mark 1}{(4.228,3.826)}
\gppoint{gp mark 1}{(4.230,4.230)}
\gppoint{gp mark 1}{(4.233,3.776)}
\gppoint{gp mark 1}{(4.235,4.197)}
\gppoint{gp mark 1}{(4.238,3.957)}
\gppoint{gp mark 1}{(4.240,4.394)}
\gppoint{gp mark 1}{(4.243,4.607)}
\gppoint{gp mark 1}{(4.245,4.487)}
\gppoint{gp mark 1}{(4.248,4.099)}
\gppoint{gp mark 1}{(4.250,4.618)}
\gppoint{gp mark 1}{(4.253,3.099)}
\gppoint{gp mark 1}{(4.255,4.219)}
\gppoint{gp mark 1}{(4.258,3.995)}
\gppoint{gp mark 1}{(4.260,4.164)}
\gppoint{gp mark 1}{(4.262,4.159)}
\gppoint{gp mark 1}{(4.265,4.295)}
\gppoint{gp mark 1}{(4.267,3.449)}
\gppoint{gp mark 1}{(4.270,4.590)}
\gppoint{gp mark 1}{(4.272,4.541)}
\gppoint{gp mark 1}{(4.275,4.268)}
\gppoint{gp mark 1}{(4.277,4.044)}
\gppoint{gp mark 1}{(4.280,4.569)}
\gppoint{gp mark 1}{(4.282,4.601)}
\gppoint{gp mark 1}{(4.285,4.487)}
\gppoint{gp mark 1}{(4.287,4.530)}
\gppoint{gp mark 1}{(4.290,3.935)}
\gppoint{gp mark 1}{(4.292,4.039)}
\gppoint{gp mark 1}{(4.295,3.951)}
\gppoint{gp mark 1}{(4.297,3.689)}
\gppoint{gp mark 1}{(4.300,4.339)}
\gppoint{gp mark 1}{(4.302,3.421)}
\gppoint{gp mark 1}{(4.305,4.377)}
\gppoint{gp mark 1}{(4.307,4.383)}
\gppoint{gp mark 1}{(4.310,4.454)}
\gppoint{gp mark 1}{(4.312,4.339)}
\gppoint{gp mark 1}{(4.315,4.645)}
\gppoint{gp mark 1}{(4.317,4.257)}
\gppoint{gp mark 1}{(4.320,4.836)}
\gppoint{gp mark 1}{(4.322,4.148)}
\gppoint{gp mark 1}{(4.325,4.263)}
\gppoint{gp mark 1}{(4.327,4.476)}
\gppoint{gp mark 1}{(4.330,4.716)}
\gppoint{gp mark 1}{(4.332,4.443)}
\gppoint{gp mark 1}{(4.335,3.574)}
\gppoint{gp mark 1}{(4.337,4.066)}
\gppoint{gp mark 1}{(4.340,4.159)}
\gppoint{gp mark 1}{(4.342,4.448)}
\gppoint{gp mark 1}{(4.345,3.629)}
\gppoint{gp mark 1}{(4.347,4.350)}
\gppoint{gp mark 1}{(4.349,4.000)}
\gppoint{gp mark 1}{(4.352,4.148)}
\gppoint{gp mark 1}{(4.354,4.405)}
\gppoint{gp mark 1}{(4.357,4.197)}
\gppoint{gp mark 1}{(4.359,3.968)}
\gppoint{gp mark 1}{(4.362,4.192)}
\gppoint{gp mark 1}{(4.364,4.372)}
\gppoint{gp mark 1}{(4.367,4.137)}
\gppoint{gp mark 1}{(4.369,4.175)}
\gppoint{gp mark 1}{(4.372,3.847)}
\gppoint{gp mark 1}{(4.374,4.596)}
\gppoint{gp mark 1}{(4.377,3.804)}
\gppoint{gp mark 1}{(4.379,3.776)}
\gppoint{gp mark 1}{(4.382,3.591)}
\gppoint{gp mark 1}{(4.384,3.908)}
\gppoint{gp mark 1}{(4.387,4.082)}
\gppoint{gp mark 1}{(4.389,3.940)}
\gppoint{gp mark 1}{(4.392,3.509)}
\gppoint{gp mark 1}{(4.394,3.678)}
\gppoint{gp mark 1}{(4.397,3.858)}
\gppoint{gp mark 1}{(4.399,3.722)}
\gppoint{gp mark 1}{(4.402,3.558)}
\gppoint{gp mark 1}{(4.404,3.880)}
\gppoint{gp mark 1}{(4.407,4.257)}
\gppoint{gp mark 1}{(4.409,4.426)}
\gppoint{gp mark 1}{(4.412,3.410)}
\gppoint{gp mark 1}{(4.414,3.968)}
\gppoint{gp mark 1}{(4.417,4.372)}
\gppoint{gp mark 1}{(4.419,4.470)}
\gppoint{gp mark 1}{(4.422,4.110)}
\gppoint{gp mark 1}{(4.424,4.082)}
\gppoint{gp mark 1}{(4.427,4.388)}
\gppoint{gp mark 1}{(4.429,4.044)}
\gppoint{gp mark 1}{(4.431,4.274)}
\gppoint{gp mark 1}{(4.434,4.132)}
\gppoint{gp mark 1}{(4.436,4.377)}
\gppoint{gp mark 1}{(4.439,3.962)}
\gppoint{gp mark 1}{(4.441,3.640)}
\gppoint{gp mark 1}{(4.444,3.940)}
\gppoint{gp mark 1}{(4.446,3.236)}
\gppoint{gp mark 1}{(4.449,3.733)}
\gppoint{gp mark 1}{(4.451,3.558)}
\gppoint{gp mark 1}{(4.454,3.935)}
\gppoint{gp mark 1}{(4.456,4.071)}
\gppoint{gp mark 1}{(4.459,4.558)}
\gppoint{gp mark 1}{(4.461,4.203)}
\gppoint{gp mark 1}{(4.464,3.962)}
\gppoint{gp mark 1}{(4.466,3.645)}
\gppoint{gp mark 1}{(4.469,4.432)}
\gppoint{gp mark 1}{(4.471,3.809)}
\gppoint{gp mark 1}{(4.474,4.339)}
\gppoint{gp mark 1}{(4.476,4.323)}
\gppoint{gp mark 1}{(4.479,4.623)}
\gppoint{gp mark 1}{(4.481,4.661)}
\gppoint{gp mark 1}{(4.484,4.547)}
\gppoint{gp mark 1}{(4.486,4.694)}
\gppoint{gp mark 1}{(4.489,3.891)}
\gppoint{gp mark 1}{(4.491,4.279)}
\gppoint{gp mark 1}{(4.494,3.662)}
\gppoint{gp mark 1}{(4.496,3.968)}
\gppoint{gp mark 1}{(4.499,4.132)}
\gppoint{gp mark 1}{(4.501,3.815)}
\gppoint{gp mark 1}{(4.504,4.115)}
\gppoint{gp mark 1}{(4.506,3.427)}
\gppoint{gp mark 1}{(4.509,3.492)}
\gppoint{gp mark 1}{(4.511,4.060)}
\gppoint{gp mark 1}{(4.513,4.334)}
\gppoint{gp mark 1}{(4.516,4.044)}
\gppoint{gp mark 1}{(4.518,4.339)}
\gppoint{gp mark 1}{(4.521,4.721)}
\gppoint{gp mark 1}{(4.523,5.169)}
\gppoint{gp mark 1}{(4.526,4.224)}
\gppoint{gp mark 1}{(4.528,4.263)}
\gppoint{gp mark 1}{(4.531,4.437)}
\gppoint{gp mark 1}{(4.533,3.585)}
\gppoint{gp mark 1}{(4.536,3.640)}
\gppoint{gp mark 1}{(4.538,4.301)}
\gppoint{gp mark 1}{(4.541,4.334)}
\gppoint{gp mark 1}{(4.543,4.164)}
\gppoint{gp mark 1}{(4.546,3.700)}
\gppoint{gp mark 1}{(4.548,4.022)}
\gppoint{gp mark 1}{(4.551,3.886)}
\gppoint{gp mark 1}{(4.553,4.093)}
\gppoint{gp mark 1}{(4.556,4.295)}
\gppoint{gp mark 1}{(4.558,3.760)}
\gppoint{gp mark 1}{(4.561,3.509)}
\gppoint{gp mark 1}{(4.563,4.186)}
\gppoint{gp mark 1}{(4.566,3.858)}
\gppoint{gp mark 1}{(4.568,3.913)}
\gppoint{gp mark 1}{(4.571,4.306)}
\gppoint{gp mark 1}{(4.573,4.022)}
\gppoint{gp mark 1}{(4.576,3.749)}
\gppoint{gp mark 1}{(4.578,4.825)}
\gppoint{gp mark 1}{(4.581,3.776)}
\gppoint{gp mark 1}{(4.583,4.339)}
\gppoint{gp mark 1}{(4.586,4.279)}
\gppoint{gp mark 1}{(4.588,3.957)}
\gppoint{gp mark 1}{(4.591,3.760)}
\gppoint{gp mark 1}{(4.593,3.738)}
\gppoint{gp mark 1}{(4.596,4.192)}
\gppoint{gp mark 1}{(4.598,4.426)}
\gppoint{gp mark 1}{(4.600,3.929)}
\gppoint{gp mark 1}{(4.603,4.181)}
\gppoint{gp mark 1}{(4.605,3.984)}
\gppoint{gp mark 1}{(4.608,4.416)}
\gppoint{gp mark 1}{(4.610,4.328)}
\gppoint{gp mark 1}{(4.613,4.011)}
\gppoint{gp mark 1}{(4.615,3.771)}
\gppoint{gp mark 1}{(4.618,3.989)}
\gppoint{gp mark 1}{(4.620,4.558)}
\gppoint{gp mark 1}{(4.623,4.579)}
\gppoint{gp mark 1}{(4.625,3.951)}
\gppoint{gp mark 1}{(4.628,4.284)}
\gppoint{gp mark 1}{(4.630,3.602)}
\gppoint{gp mark 1}{(4.633,3.684)}
\gppoint{gp mark 1}{(4.635,4.022)}
\gppoint{gp mark 1}{(4.638,4.028)}
\gppoint{gp mark 1}{(4.640,3.820)}
\gppoint{gp mark 1}{(4.643,3.427)}
\gppoint{gp mark 1}{(4.645,4.372)}
\gppoint{gp mark 1}{(4.648,4.011)}
\gppoint{gp mark 1}{(4.650,4.148)}
\gppoint{gp mark 1}{(4.653,4.039)}
\gppoint{gp mark 1}{(4.655,4.399)}
\gppoint{gp mark 1}{(4.658,3.651)}
\gppoint{gp mark 1}{(4.660,3.787)}
\gppoint{gp mark 1}{(4.663,3.645)}
\gppoint{gp mark 1}{(4.665,4.650)}
\gppoint{gp mark 1}{(4.668,4.476)}
\gppoint{gp mark 1}{(4.670,3.410)}
\gppoint{gp mark 1}{(4.673,4.716)}
\gppoint{gp mark 1}{(4.675,3.837)}
\gppoint{gp mark 1}{(4.678,3.066)}
\gppoint{gp mark 1}{(4.680,3.602)}
\gppoint{gp mark 1}{(4.682,4.093)}
\gppoint{gp mark 1}{(4.685,3.667)}
\gppoint{gp mark 1}{(4.687,4.530)}
\gppoint{gp mark 1}{(4.690,3.645)}
\gppoint{gp mark 1}{(4.692,4.066)}
\gppoint{gp mark 1}{(4.695,3.525)}
\gppoint{gp mark 1}{(4.697,3.979)}
\gppoint{gp mark 1}{(4.700,3.908)}
\gppoint{gp mark 1}{(4.702,3.979)}
\gppoint{gp mark 1}{(4.705,2.979)}
\gppoint{gp mark 1}{(4.707,4.508)}
\gppoint{gp mark 1}{(4.710,4.355)}
\gppoint{gp mark 1}{(4.712,4.230)}
\gppoint{gp mark 1}{(4.715,4.416)}
\gppoint{gp mark 1}{(4.717,4.181)}
\gppoint{gp mark 1}{(4.720,5.044)}
\gppoint{gp mark 1}{(4.722,4.170)}
\gppoint{gp mark 1}{(4.725,4.104)}
\gppoint{gp mark 1}{(4.727,3.755)}
\gppoint{gp mark 1}{(4.730,4.006)}
\gppoint{gp mark 1}{(4.732,3.869)}
\gppoint{gp mark 1}{(4.735,3.946)}
\gppoint{gp mark 1}{(4.737,4.721)}
\gppoint{gp mark 1}{(4.740,4.481)}
\gppoint{gp mark 1}{(4.742,3.651)}
\gppoint{gp mark 1}{(4.745,3.946)}
\gppoint{gp mark 1}{(4.747,3.962)}
\gppoint{gp mark 1}{(4.750,3.629)}
\gppoint{gp mark 1}{(4.752,3.891)}
\gppoint{gp mark 1}{(4.755,3.372)}
\gppoint{gp mark 1}{(4.757,3.574)}
\gppoint{gp mark 1}{(4.760,4.776)}
\gppoint{gp mark 1}{(4.762,4.372)}
\gppoint{gp mark 1}{(4.764,4.164)}
\gppoint{gp mark 1}{(4.767,4.192)}
\gppoint{gp mark 1}{(4.769,3.192)}
\gppoint{gp mark 1}{(4.772,3.602)}
\gppoint{gp mark 1}{(4.774,3.585)}
\gppoint{gp mark 1}{(4.777,4.776)}
\gppoint{gp mark 1}{(4.779,4.033)}
\gppoint{gp mark 1}{(4.782,4.263)}
\gppoint{gp mark 1}{(4.784,3.853)}
\gppoint{gp mark 1}{(4.787,3.951)}
\gppoint{gp mark 1}{(4.789,3.481)}
\gppoint{gp mark 1}{(4.792,4.219)}
\gppoint{gp mark 1}{(4.794,3.432)}
\gppoint{gp mark 1}{(4.797,4.241)}
\gppoint{gp mark 1}{(4.799,4.006)}
\gppoint{gp mark 1}{(4.802,4.181)}
\gppoint{gp mark 1}{(4.804,4.536)}
\gppoint{gp mark 1}{(4.807,4.066)}
\gppoint{gp mark 1}{(4.809,4.487)}
\gppoint{gp mark 1}{(4.812,4.022)}
\gppoint{gp mark 1}{(4.814,3.864)}
\gppoint{gp mark 1}{(4.817,4.093)}
\gppoint{gp mark 1}{(4.819,4.252)}
\gppoint{gp mark 1}{(4.822,4.798)}
\gppoint{gp mark 1}{(4.824,2.728)}
\gppoint{gp mark 1}{(4.827,3.471)}
\gppoint{gp mark 1}{(4.829,3.481)}
\gppoint{gp mark 1}{(4.832,3.667)}
\gppoint{gp mark 1}{(4.834,3.405)}
\gppoint{gp mark 1}{(4.837,4.836)}
\gppoint{gp mark 1}{(4.839,4.104)}
\gppoint{gp mark 1}{(4.842,3.705)}
\gppoint{gp mark 1}{(4.844,4.459)}
\gppoint{gp mark 1}{(4.846,3.585)}
\gppoint{gp mark 1}{(4.849,3.481)}
\gppoint{gp mark 1}{(4.851,3.700)}
\gppoint{gp mark 1}{(4.854,3.410)}
\gppoint{gp mark 1}{(4.856,4.432)}
\gppoint{gp mark 1}{(4.859,4.219)}
\gppoint{gp mark 1}{(4.861,3.908)}
\gppoint{gp mark 1}{(4.864,3.858)}
\gppoint{gp mark 1}{(4.866,4.263)}
\gppoint{gp mark 1}{(4.869,4.104)}
\gppoint{gp mark 1}{(4.871,3.727)}
\gppoint{gp mark 1}{(4.874,4.186)}
\gppoint{gp mark 1}{(4.876,3.400)}
\gppoint{gp mark 1}{(4.879,3.908)}
\gppoint{gp mark 1}{(4.881,3.973)}
\gppoint{gp mark 1}{(4.884,4.317)}
\gppoint{gp mark 1}{(4.886,4.284)}
\gppoint{gp mark 1}{(4.889,3.613)}
\gppoint{gp mark 1}{(4.891,3.869)}
\gppoint{gp mark 1}{(4.894,4.607)}
\gppoint{gp mark 1}{(4.896,3.842)}
\gppoint{gp mark 1}{(4.899,3.716)}
\gppoint{gp mark 1}{(4.901,4.011)}
\gppoint{gp mark 1}{(4.904,3.733)}
\gppoint{gp mark 1}{(4.906,3.547)}
\gppoint{gp mark 1}{(4.909,3.585)}
\gppoint{gp mark 1}{(4.911,3.350)}
\gppoint{gp mark 1}{(4.914,3.716)}
\gppoint{gp mark 1}{(4.916,3.809)}
\gppoint{gp mark 1}{(4.919,3.798)}
\gppoint{gp mark 1}{(4.921,4.066)}
\gppoint{gp mark 1}{(4.924,3.973)}
\gppoint{gp mark 1}{(4.926,4.022)}
\gppoint{gp mark 1}{(4.929,3.640)}
\gppoint{gp mark 1}{(4.931,3.623)}
\gppoint{gp mark 1}{(4.933,3.662)}
\gppoint{gp mark 1}{(4.936,3.673)}
\gppoint{gp mark 1}{(4.938,3.831)}
\gppoint{gp mark 1}{(4.941,4.503)}
\gppoint{gp mark 1}{(4.943,3.716)}
\gppoint{gp mark 1}{(4.946,3.826)}
\gppoint{gp mark 1}{(4.948,3.858)}
\gppoint{gp mark 1}{(4.951,3.891)}
\gppoint{gp mark 1}{(4.953,3.744)}
\gppoint{gp mark 1}{(4.956,4.760)}
\gppoint{gp mark 1}{(4.958,3.929)}
\gppoint{gp mark 1}{(4.961,4.672)}
\gppoint{gp mark 1}{(4.963,3.853)}
\gppoint{gp mark 1}{(4.966,4.601)}
\gppoint{gp mark 1}{(4.968,3.509)}
\gppoint{gp mark 1}{(4.971,3.918)}
\gppoint{gp mark 1}{(4.973,3.897)}
\gppoint{gp mark 1}{(4.976,3.722)}
\gppoint{gp mark 1}{(4.978,4.241)}
\gppoint{gp mark 1}{(4.981,4.022)}
\gppoint{gp mark 1}{(4.983,3.962)}
\gppoint{gp mark 1}{(4.986,3.804)}
\gppoint{gp mark 1}{(4.988,4.164)}
\gppoint{gp mark 1}{(4.991,3.935)}
\gppoint{gp mark 1}{(4.993,4.328)}
\gppoint{gp mark 1}{(4.996,4.661)}
\gppoint{gp mark 1}{(4.998,4.426)}
\gppoint{gp mark 1}{(5.001,3.820)}
\gppoint{gp mark 1}{(5.003,4.328)}
\gppoint{gp mark 1}{(5.006,3.760)}
\gppoint{gp mark 1}{(5.008,4.754)}
\gppoint{gp mark 1}{(5.011,3.782)}
\gppoint{gp mark 1}{(5.013,3.793)}
\gppoint{gp mark 1}{(5.015,4.454)}
\gppoint{gp mark 1}{(5.018,3.995)}
\gppoint{gp mark 1}{(5.020,3.929)}
\gppoint{gp mark 1}{(5.023,4.317)}
\gppoint{gp mark 1}{(5.025,3.662)}
\gppoint{gp mark 1}{(5.028,4.110)}
\gppoint{gp mark 1}{(5.030,4.219)}
\gppoint{gp mark 1}{(5.033,4.044)}
\gppoint{gp mark 1}{(5.035,4.246)}
\gppoint{gp mark 1}{(5.038,3.705)}
\gppoint{gp mark 1}{(5.040,3.847)}
\gppoint{gp mark 1}{(5.043,3.454)}
\gppoint{gp mark 1}{(5.045,3.503)}
\gppoint{gp mark 1}{(5.048,3.940)}
\gppoint{gp mark 1}{(5.050,3.875)}
\gppoint{gp mark 1}{(5.053,4.011)}
\gppoint{gp mark 1}{(5.055,3.547)}
\gppoint{gp mark 1}{(5.058,4.612)}
\gppoint{gp mark 1}{(5.060,3.962)}
\gppoint{gp mark 1}{(5.063,4.503)}
\gppoint{gp mark 1}{(5.065,3.755)}
\gppoint{gp mark 1}{(5.068,3.356)}
\gppoint{gp mark 1}{(5.070,3.957)}
\gppoint{gp mark 1}{(5.073,4.558)}
\gppoint{gp mark 1}{(5.075,4.186)}
\gppoint{gp mark 1}{(5.078,3.924)}
\gppoint{gp mark 1}{(5.080,3.443)}
\gppoint{gp mark 1}{(5.083,4.110)}
\gppoint{gp mark 1}{(5.085,3.858)}
\gppoint{gp mark 1}{(5.088,4.465)}
\gppoint{gp mark 1}{(5.090,4.306)}
\gppoint{gp mark 1}{(5.093,4.022)}
\gppoint{gp mark 1}{(5.095,3.842)}
\gppoint{gp mark 1}{(5.097,3.957)}
\gppoint{gp mark 1}{(5.100,4.328)}
\gppoint{gp mark 1}{(5.102,3.946)}
\gppoint{gp mark 1}{(5.105,4.060)}
\gppoint{gp mark 1}{(5.107,3.492)}
\gppoint{gp mark 1}{(5.110,3.738)}
\gppoint{gp mark 1}{(5.112,3.219)}
\gppoint{gp mark 1}{(5.115,4.481)}
\gppoint{gp mark 1}{(5.117,3.694)}
\gppoint{gp mark 1}{(5.120,4.148)}
\gppoint{gp mark 1}{(5.122,4.017)}
\gppoint{gp mark 1}{(5.125,3.487)}
\gppoint{gp mark 1}{(5.127,3.902)}
\gppoint{gp mark 1}{(5.130,3.334)}
\gppoint{gp mark 1}{(5.132,3.618)}
\gppoint{gp mark 1}{(5.135,3.279)}
\gppoint{gp mark 1}{(5.137,4.345)}
\gppoint{gp mark 1}{(5.140,3.574)}
\gppoint{gp mark 1}{(5.142,4.508)}
\gppoint{gp mark 1}{(5.145,3.585)}
\gppoint{gp mark 1}{(5.147,3.400)}
\gppoint{gp mark 1}{(5.150,3.427)}
\gppoint{gp mark 1}{(5.152,3.897)}
\gppoint{gp mark 1}{(5.155,3.236)}
\gppoint{gp mark 1}{(5.157,3.389)}
\gppoint{gp mark 1}{(5.160,4.246)}
\gppoint{gp mark 1}{(5.162,3.793)}
\gppoint{gp mark 1}{(5.165,3.618)}
\gppoint{gp mark 1}{(5.167,3.908)}
\gppoint{gp mark 1}{(5.170,3.623)}
\gppoint{gp mark 1}{(5.172,3.820)}
\gppoint{gp mark 1}{(5.175,3.831)}
\gppoint{gp mark 1}{(5.177,3.274)}
\gppoint{gp mark 1}{(5.180,4.252)}
\gppoint{gp mark 1}{(5.182,4.792)}
\gppoint{gp mark 1}{(5.184,3.148)}
\gppoint{gp mark 1}{(5.187,3.984)}
\gppoint{gp mark 1}{(5.189,3.782)}
\gppoint{gp mark 1}{(5.192,3.525)}
\gppoint{gp mark 1}{(5.194,3.558)}
\gppoint{gp mark 1}{(5.197,4.050)}
\gppoint{gp mark 1}{(5.199,3.618)}
\gppoint{gp mark 1}{(5.202,3.913)}
\gppoint{gp mark 1}{(5.204,3.924)}
\gppoint{gp mark 1}{(5.207,3.328)}
\gppoint{gp mark 1}{(5.209,3.689)}
\gppoint{gp mark 1}{(5.212,4.126)}
\gppoint{gp mark 1}{(5.214,4.235)}
\gppoint{gp mark 1}{(5.217,3.148)}
\gppoint{gp mark 1}{(5.219,2.842)}
\gppoint{gp mark 1}{(5.222,3.334)}
\gppoint{gp mark 1}{(5.224,3.410)}
\gppoint{gp mark 1}{(5.227,4.050)}
\gppoint{gp mark 1}{(5.229,4.011)}
\gppoint{gp mark 1}{(5.232,3.498)}
\gppoint{gp mark 1}{(5.234,3.968)}
\gppoint{gp mark 1}{(5.237,4.230)}
\gppoint{gp mark 1}{(5.239,3.487)}
\gppoint{gp mark 1}{(5.242,3.634)}
\gppoint{gp mark 1}{(5.244,4.448)}
\gppoint{gp mark 1}{(5.247,4.328)}
\gppoint{gp mark 1}{(5.249,3.372)}
\gppoint{gp mark 1}{(5.252,3.946)}
\gppoint{gp mark 1}{(5.254,3.749)}
\gppoint{gp mark 1}{(5.257,3.908)}
\gppoint{gp mark 1}{(5.259,3.312)}
\gppoint{gp mark 1}{(5.262,3.208)}
\gppoint{gp mark 1}{(5.264,3.618)}
\gppoint{gp mark 1}{(5.266,3.088)}
\gppoint{gp mark 1}{(5.269,3.602)}
\gppoint{gp mark 1}{(5.271,3.263)}
\gppoint{gp mark 1}{(5.274,3.847)}
\gppoint{gp mark 1}{(5.276,4.246)}
\gppoint{gp mark 1}{(5.279,3.798)}
\gppoint{gp mark 1}{(5.281,4.055)}
\gppoint{gp mark 1}{(5.284,4.055)}
\gppoint{gp mark 1}{(5.286,3.531)}
\gppoint{gp mark 1}{(5.289,3.640)}
\gppoint{gp mark 1}{(5.291,3.498)}
\gppoint{gp mark 1}{(5.294,4.071)}
\gppoint{gp mark 1}{(5.296,3.842)}
\gppoint{gp mark 1}{(5.299,3.563)}
\gppoint{gp mark 1}{(5.301,4.339)}
\gppoint{gp mark 1}{(5.304,4.039)}
\gppoint{gp mark 1}{(5.306,3.722)}
\gppoint{gp mark 1}{(5.309,3.083)}
\gppoint{gp mark 1}{(5.311,3.793)}
\gppoint{gp mark 1}{(5.314,3.820)}
\gppoint{gp mark 1}{(5.316,4.050)}
\gppoint{gp mark 1}{(5.319,3.804)}
\gppoint{gp mark 1}{(5.321,3.170)}
\gppoint{gp mark 1}{(5.324,4.416)}
\gppoint{gp mark 1}{(5.326,3.400)}
\gppoint{gp mark 1}{(5.329,3.733)}
\gppoint{gp mark 1}{(5.331,3.733)}
\gppoint{gp mark 1}{(5.334,4.082)}
\gppoint{gp mark 1}{(5.336,3.651)}
\gppoint{gp mark 1}{(5.339,3.782)}
\gppoint{gp mark 1}{(5.341,3.744)}
\gppoint{gp mark 1}{(5.344,3.476)}
\gppoint{gp mark 1}{(5.346,3.596)}
\gppoint{gp mark 1}{(5.348,3.602)}
\gppoint{gp mark 1}{(5.351,3.345)}
\gppoint{gp mark 1}{(5.353,3.356)}
\gppoint{gp mark 1}{(5.356,3.585)}
\gppoint{gp mark 1}{(5.358,3.782)}
\gppoint{gp mark 1}{(5.361,3.542)}
\gppoint{gp mark 1}{(5.363,3.290)}
\gppoint{gp mark 1}{(5.366,4.448)}
\gppoint{gp mark 1}{(5.368,4.716)}
\gppoint{gp mark 1}{(5.371,3.634)}
\gppoint{gp mark 1}{(5.373,3.017)}
\gppoint{gp mark 1}{(5.376,3.563)}
\gppoint{gp mark 1}{(5.378,3.935)}
\gppoint{gp mark 1}{(5.381,3.640)}
\gppoint{gp mark 1}{(5.383,3.886)}
\gppoint{gp mark 1}{(5.386,3.946)}
\gppoint{gp mark 1}{(5.388,4.126)}
\gppoint{gp mark 1}{(5.391,4.093)}
\gppoint{gp mark 1}{(5.393,4.476)}
\gppoint{gp mark 1}{(5.396,3.711)}
\gppoint{gp mark 1}{(5.398,3.318)}
\gppoint{gp mark 1}{(5.401,3.296)}
\gppoint{gp mark 1}{(5.403,3.776)}
\gppoint{gp mark 1}{(5.406,3.121)}
\gppoint{gp mark 1}{(5.408,3.159)}
\gppoint{gp mark 1}{(5.411,3.809)}
\gppoint{gp mark 1}{(5.413,4.126)}
\gppoint{gp mark 1}{(5.416,3.536)}
\gppoint{gp mark 1}{(5.418,3.815)}
\gppoint{gp mark 1}{(5.421,3.853)}
\gppoint{gp mark 1}{(5.423,3.738)}
\gppoint{gp mark 1}{(5.426,3.034)}
\gppoint{gp mark 1}{(5.428,3.820)}
\gppoint{gp mark 1}{(5.431,3.252)}
\gppoint{gp mark 1}{(5.433,3.662)}
\gppoint{gp mark 1}{(5.435,3.574)}
\gppoint{gp mark 1}{(5.438,3.165)}
\gppoint{gp mark 1}{(5.440,3.738)}
\gppoint{gp mark 1}{(5.443,3.705)}
\gppoint{gp mark 1}{(5.445,4.454)}
\gppoint{gp mark 1}{(5.448,3.279)}
\gppoint{gp mark 1}{(5.450,4.334)}
\gppoint{gp mark 1}{(5.453,3.694)}
\gppoint{gp mark 1}{(5.455,3.596)}
\gppoint{gp mark 1}{(5.458,3.170)}
\gppoint{gp mark 1}{(5.460,3.837)}
\gppoint{gp mark 1}{(5.463,3.503)}
\gppoint{gp mark 1}{(5.465,3.809)}
\gppoint{gp mark 1}{(5.468,3.460)}
\gppoint{gp mark 1}{(5.470,4.252)}
\gppoint{gp mark 1}{(5.473,3.367)}
\gppoint{gp mark 1}{(5.475,3.815)}
\gppoint{gp mark 1}{(5.478,4.132)}
\gppoint{gp mark 1}{(5.480,2.941)}
\gppoint{gp mark 1}{(5.483,4.164)}
\gppoint{gp mark 1}{(5.485,3.476)}
\gppoint{gp mark 1}{(5.488,3.574)}
\gppoint{gp mark 1}{(5.490,3.935)}
\gppoint{gp mark 1}{(5.493,3.694)}
\gppoint{gp mark 1}{(5.495,3.782)}
\gppoint{gp mark 1}{(5.498,3.192)}
\gppoint{gp mark 1}{(5.500,3.465)}
\gppoint{gp mark 1}{(5.503,3.864)}
\gppoint{gp mark 1}{(5.505,3.471)}
\gppoint{gp mark 1}{(5.508,3.891)}
\gppoint{gp mark 1}{(5.510,3.176)}
\gppoint{gp mark 1}{(5.513,3.072)}
\gppoint{gp mark 1}{(5.515,3.072)}
\gppoint{gp mark 1}{(5.517,4.361)}
\gppoint{gp mark 1}{(5.520,3.968)}
\gppoint{gp mark 1}{(5.522,2.935)}
\gppoint{gp mark 1}{(5.525,3.427)}
\gppoint{gp mark 1}{(5.527,2.853)}
\gppoint{gp mark 1}{(5.530,3.864)}
\gppoint{gp mark 1}{(5.532,3.837)}
\gppoint{gp mark 1}{(5.535,3.940)}
\gppoint{gp mark 1}{(5.537,3.602)}
\gppoint{gp mark 1}{(5.540,3.847)}
\gppoint{gp mark 1}{(5.542,3.700)}
\gppoint{gp mark 1}{(5.545,3.591)}
\gppoint{gp mark 1}{(5.547,3.705)}
\gppoint{gp mark 1}{(5.550,3.842)}
\gppoint{gp mark 1}{(5.552,3.864)}
\gppoint{gp mark 1}{(5.555,3.700)}
\gppoint{gp mark 1}{(5.557,2.968)}
\gppoint{gp mark 1}{(5.560,3.634)}
\gppoint{gp mark 1}{(5.562,3.301)}
\gppoint{gp mark 1}{(5.565,3.268)}
\gppoint{gp mark 1}{(5.567,3.656)}
\gppoint{gp mark 1}{(5.570,3.449)}
\gppoint{gp mark 1}{(5.572,3.968)}
\gppoint{gp mark 1}{(5.575,3.733)}
\gppoint{gp mark 1}{(5.577,3.979)}
\gppoint{gp mark 1}{(5.580,3.913)}
\gppoint{gp mark 1}{(5.582,3.072)}
\gppoint{gp mark 1}{(5.585,3.837)}
\gppoint{gp mark 1}{(5.587,4.208)}
\gppoint{gp mark 1}{(5.590,3.684)}
\gppoint{gp mark 1}{(5.592,4.601)}
\gppoint{gp mark 1}{(5.595,4.082)}
\gppoint{gp mark 1}{(5.597,3.645)}
\gppoint{gp mark 1}{(5.599,2.629)}
\gppoint{gp mark 1}{(5.602,3.034)}
\gppoint{gp mark 1}{(5.604,3.760)}
\gppoint{gp mark 1}{(5.607,4.132)}
\gppoint{gp mark 1}{(5.609,3.580)}
\gppoint{gp mark 1}{(5.612,3.400)}
\gppoint{gp mark 1}{(5.614,3.274)}
\gppoint{gp mark 1}{(5.617,3.984)}
\gppoint{gp mark 1}{(5.619,3.591)}
\gppoint{gp mark 1}{(5.622,2.553)}
\gppoint{gp mark 1}{(5.624,3.678)}
\gppoint{gp mark 1}{(5.627,3.323)}
\gppoint{gp mark 1}{(5.629,3.023)}
\gppoint{gp mark 1}{(5.632,3.585)}
\gppoint{gp mark 1}{(5.634,3.837)}
\gppoint{gp mark 1}{(5.637,3.039)}
\gppoint{gp mark 1}{(5.639,4.508)}
\gppoint{gp mark 1}{(5.642,3.580)}
\gppoint{gp mark 1}{(5.644,3.569)}
\gppoint{gp mark 1}{(5.647,3.858)}
\gppoint{gp mark 1}{(5.649,4.257)}
\gppoint{gp mark 1}{(5.652,3.405)}
\gppoint{gp mark 1}{(5.654,4.039)}
\gppoint{gp mark 1}{(5.657,3.585)}
\gppoint{gp mark 1}{(5.659,3.350)}
\gppoint{gp mark 1}{(5.662,3.640)}
\gppoint{gp mark 1}{(5.664,2.870)}
\gppoint{gp mark 1}{(5.667,3.602)}
\gppoint{gp mark 1}{(5.669,3.410)}
\gppoint{gp mark 1}{(5.672,4.071)}
\gppoint{gp mark 1}{(5.674,4.099)}
\gppoint{gp mark 1}{(5.677,3.831)}
\gppoint{gp mark 1}{(5.679,3.345)}
\gppoint{gp mark 1}{(5.682,3.864)}
\gppoint{gp mark 1}{(5.684,3.536)}
\gppoint{gp mark 1}{(5.686,3.099)}
\gppoint{gp mark 1}{(5.689,3.083)}
\gppoint{gp mark 1}{(5.691,4.055)}
\gppoint{gp mark 1}{(5.694,4.011)}
\gppoint{gp mark 1}{(5.696,3.760)}
\gppoint{gp mark 1}{(5.699,3.345)}
\gppoint{gp mark 1}{(5.701,3.432)}
\gppoint{gp mark 1}{(5.704,3.935)}
\gppoint{gp mark 1}{(5.706,3.869)}
\gppoint{gp mark 1}{(5.709,4.066)}
\gppoint{gp mark 1}{(5.711,3.684)}
\gppoint{gp mark 1}{(5.714,4.104)}
\gppoint{gp mark 1}{(5.716,3.913)}
\gppoint{gp mark 1}{(5.719,3.137)}
\gppoint{gp mark 1}{(5.721,3.318)}
\gppoint{gp mark 1}{(5.724,3.738)}
\gppoint{gp mark 1}{(5.726,3.481)}
\gppoint{gp mark 1}{(5.729,3.165)}
\gppoint{gp mark 1}{(5.731,3.918)}
\gppoint{gp mark 1}{(5.734,3.673)}
\gppoint{gp mark 1}{(5.736,3.230)}
\gppoint{gp mark 1}{(5.739,3.700)}
\gppoint{gp mark 1}{(5.741,4.159)}
\gppoint{gp mark 1}{(5.744,3.312)}
\gppoint{gp mark 1}{(5.746,3.361)}
\gppoint{gp mark 1}{(5.749,3.908)}
\gppoint{gp mark 1}{(5.751,3.192)}
\gppoint{gp mark 1}{(5.754,3.886)}
\gppoint{gp mark 1}{(5.756,3.749)}
\gppoint{gp mark 1}{(5.759,3.804)}
\gppoint{gp mark 1}{(5.761,3.755)}
\gppoint{gp mark 1}{(5.764,3.738)}
\gppoint{gp mark 1}{(5.766,3.809)}
\gppoint{gp mark 1}{(5.768,3.465)}
\gppoint{gp mark 1}{(5.771,2.891)}
\gppoint{gp mark 1}{(5.773,3.492)}
\gppoint{gp mark 1}{(5.776,3.225)}
\gppoint{gp mark 1}{(5.778,3.465)}
\gppoint{gp mark 1}{(5.781,4.317)}
\gppoint{gp mark 1}{(5.783,3.203)}
\gppoint{gp mark 1}{(5.786,3.110)}
\gppoint{gp mark 1}{(5.788,3.853)}
\gppoint{gp mark 1}{(5.791,3.684)}
\gppoint{gp mark 1}{(5.793,4.159)}
\gppoint{gp mark 1}{(5.796,3.831)}
\gppoint{gp mark 1}{(5.798,3.847)}
\gppoint{gp mark 1}{(5.801,3.962)}
\gppoint{gp mark 1}{(5.803,3.864)}
\gppoint{gp mark 1}{(5.806,3.700)}
\gppoint{gp mark 1}{(5.808,3.902)}
\gppoint{gp mark 1}{(5.811,3.443)}
\gppoint{gp mark 1}{(5.813,4.399)}
\gppoint{gp mark 1}{(5.816,3.935)}
\gppoint{gp mark 1}{(5.818,3.968)}
\gppoint{gp mark 1}{(5.821,3.186)}
\gppoint{gp mark 1}{(5.823,3.290)}
\gppoint{gp mark 1}{(5.826,3.700)}
\gppoint{gp mark 1}{(5.828,3.236)}
\gppoint{gp mark 1}{(5.831,3.864)}
\gppoint{gp mark 1}{(5.833,3.875)}
\gppoint{gp mark 1}{(5.836,3.476)}
\gppoint{gp mark 1}{(5.838,3.416)}
\gppoint{gp mark 1}{(5.841,3.323)}
\gppoint{gp mark 1}{(5.843,3.301)}
\gppoint{gp mark 1}{(5.846,3.050)}
\gppoint{gp mark 1}{(5.848,3.318)}
\gppoint{gp mark 1}{(5.850,4.295)}
\gppoint{gp mark 1}{(5.853,3.684)}
\gppoint{gp mark 1}{(5.855,3.673)}
\gppoint{gp mark 1}{(5.858,3.613)}
\gppoint{gp mark 1}{(5.860,3.940)}
\gppoint{gp mark 1}{(5.863,3.132)}
\gppoint{gp mark 1}{(5.865,3.913)}
\gppoint{gp mark 1}{(5.868,3.421)}
\gppoint{gp mark 1}{(5.870,3.307)}
\gppoint{gp mark 1}{(5.873,3.427)}
\gppoint{gp mark 1}{(5.875,3.962)}
\gppoint{gp mark 1}{(5.878,3.700)}
\gppoint{gp mark 1}{(5.880,4.017)}
\gppoint{gp mark 1}{(5.883,3.438)}
\gppoint{gp mark 1}{(5.885,3.591)}
\gppoint{gp mark 1}{(5.888,4.033)}
\gppoint{gp mark 1}{(5.890,3.279)}
\gppoint{gp mark 1}{(5.893,3.345)}
\gppoint{gp mark 1}{(5.895,3.427)}
\gppoint{gp mark 1}{(5.898,3.596)}
\gppoint{gp mark 1}{(5.900,3.968)}
\gppoint{gp mark 1}{(5.903,3.126)}
\gppoint{gp mark 1}{(5.905,3.602)}
\gppoint{gp mark 1}{(5.908,3.279)}
\gppoint{gp mark 1}{(5.910,3.263)}
\gppoint{gp mark 1}{(5.913,3.760)}
\gppoint{gp mark 1}{(5.915,3.498)}
\gppoint{gp mark 1}{(5.918,3.438)}
\gppoint{gp mark 1}{(5.920,3.957)}
\gppoint{gp mark 1}{(5.923,4.115)}
\gppoint{gp mark 1}{(5.925,3.602)}
\gppoint{gp mark 1}{(5.928,3.667)}
\gppoint{gp mark 1}{(5.930,3.842)}
\gppoint{gp mark 1}{(5.932,3.487)}
\gppoint{gp mark 1}{(5.935,3.929)}
\gppoint{gp mark 1}{(5.937,3.394)}
\gppoint{gp mark 1}{(5.940,3.012)}
\gppoint{gp mark 1}{(5.942,3.787)}
\gppoint{gp mark 1}{(5.945,4.208)}
\gppoint{gp mark 1}{(5.947,3.481)}
\gppoint{gp mark 1}{(5.950,3.405)}
\gppoint{gp mark 1}{(5.952,3.558)}
\gppoint{gp mark 1}{(5.955,3.776)}
\gppoint{gp mark 1}{(5.957,3.115)}
\gppoint{gp mark 1}{(5.960,4.377)}
\gppoint{gp mark 1}{(5.962,2.979)}
\gppoint{gp mark 1}{(5.965,3.858)}
\gppoint{gp mark 1}{(5.967,3.531)}
\gppoint{gp mark 1}{(5.970,4.355)}
\gppoint{gp mark 1}{(5.972,3.438)}
\gppoint{gp mark 1}{(5.975,3.323)}
\gppoint{gp mark 1}{(5.977,3.307)}
\gppoint{gp mark 1}{(5.980,2.673)}
\gppoint{gp mark 1}{(5.982,3.820)}
\gppoint{gp mark 1}{(5.985,3.591)}
\gppoint{gp mark 1}{(5.987,3.897)}
\gppoint{gp mark 1}{(5.990,4.104)}
\gppoint{gp mark 1}{(5.992,4.623)}
\gppoint{gp mark 1}{(5.995,3.656)}
\gppoint{gp mark 1}{(5.997,3.520)}
\gppoint{gp mark 1}{(6.000,3.099)}
\gppoint{gp mark 1}{(6.002,3.236)}
\gppoint{gp mark 1}{(6.005,3.716)}
\gppoint{gp mark 1}{(6.007,3.421)}
\gppoint{gp mark 1}{(6.010,4.088)}
\gppoint{gp mark 1}{(6.012,3.776)}
\gppoint{gp mark 1}{(6.015,2.881)}
\gppoint{gp mark 1}{(6.017,3.629)}
\gppoint{gp mark 1}{(6.019,3.897)}
\gppoint{gp mark 1}{(6.022,3.973)}
\gppoint{gp mark 1}{(6.024,3.105)}
\gppoint{gp mark 1}{(6.027,2.886)}
\gppoint{gp mark 1}{(6.029,3.727)}
\gppoint{gp mark 1}{(6.032,3.449)}
\gppoint{gp mark 1}{(6.034,3.247)}
\gppoint{gp mark 1}{(6.037,2.881)}
\gppoint{gp mark 1}{(6.039,3.509)}
\gppoint{gp mark 1}{(6.042,3.509)}
\gppoint{gp mark 1}{(6.044,3.219)}
\gppoint{gp mark 1}{(6.047,3.542)}
\gppoint{gp mark 1}{(6.049,3.050)}
\gppoint{gp mark 1}{(6.052,2.815)}
\gppoint{gp mark 1}{(6.054,4.006)}
\gppoint{gp mark 1}{(6.057,3.705)}
\gppoint{gp mark 1}{(6.059,3.339)}
\gppoint{gp mark 1}{(6.062,3.432)}
\gppoint{gp mark 1}{(6.064,3.651)}
\gppoint{gp mark 1}{(6.067,3.432)}
\gppoint{gp mark 1}{(6.069,3.618)}
\gppoint{gp mark 1}{(6.072,4.088)}
\gppoint{gp mark 1}{(6.074,3.946)}
\gppoint{gp mark 1}{(6.077,3.804)}
\gppoint{gp mark 1}{(6.079,3.367)}
\gppoint{gp mark 1}{(6.082,3.716)}
\gppoint{gp mark 1}{(6.084,4.416)}
\gppoint{gp mark 1}{(6.087,3.580)}
\gppoint{gp mark 1}{(6.089,3.432)}
\gppoint{gp mark 1}{(6.092,3.066)}
\gppoint{gp mark 1}{(6.094,3.760)}
\gppoint{gp mark 1}{(6.097,3.776)}
\gppoint{gp mark 1}{(6.099,3.645)}
\gppoint{gp mark 1}{(6.101,2.733)}
\gppoint{gp mark 1}{(6.104,2.941)}
\gppoint{gp mark 1}{(6.106,4.186)}
\gppoint{gp mark 1}{(6.109,3.979)}
\gppoint{gp mark 1}{(6.111,3.525)}
\gppoint{gp mark 1}{(6.114,3.432)}
\gppoint{gp mark 1}{(6.116,3.307)}
\gppoint{gp mark 1}{(6.119,2.837)}
\gppoint{gp mark 1}{(6.121,3.186)}
\gppoint{gp mark 1}{(6.124,3.307)}
\gppoint{gp mark 1}{(6.126,3.689)}
\gppoint{gp mark 1}{(6.129,3.962)}
\gppoint{gp mark 1}{(6.131,3.760)}
\gppoint{gp mark 1}{(6.134,3.623)}
\gppoint{gp mark 1}{(6.136,3.498)}
\gppoint{gp mark 1}{(6.139,3.837)}
\gppoint{gp mark 1}{(6.141,3.880)}
\gppoint{gp mark 1}{(6.144,3.694)}
\gppoint{gp mark 1}{(6.146,4.306)}
\gppoint{gp mark 1}{(6.149,3.580)}
\gppoint{gp mark 1}{(6.151,3.170)}
\gppoint{gp mark 1}{(6.154,3.410)}
\gppoint{gp mark 1}{(6.156,3.307)}
\gppoint{gp mark 1}{(6.159,3.596)}
\gppoint{gp mark 1}{(6.161,3.274)}
\gppoint{gp mark 1}{(6.164,3.367)}
\gppoint{gp mark 1}{(6.166,4.093)}
\gppoint{gp mark 1}{(6.169,3.842)}
\gppoint{gp mark 1}{(6.171,3.328)}
\gppoint{gp mark 1}{(6.174,3.673)}
\gppoint{gp mark 1}{(6.176,3.383)}
\gppoint{gp mark 1}{(6.179,3.580)}
\gppoint{gp mark 1}{(6.181,3.891)}
\gppoint{gp mark 1}{(6.183,3.520)}
\gppoint{gp mark 1}{(6.186,4.366)}
\gppoint{gp mark 1}{(6.188,3.454)}
\gppoint{gp mark 1}{(6.191,3.476)}
\gppoint{gp mark 1}{(6.193,3.607)}
\gppoint{gp mark 1}{(6.196,3.705)}
\gppoint{gp mark 1}{(6.198,3.383)}
\gppoint{gp mark 1}{(6.201,3.514)}
\gppoint{gp mark 1}{(6.203,2.744)}
\gppoint{gp mark 1}{(6.206,3.716)}
\gppoint{gp mark 1}{(6.208,3.678)}
\gppoint{gp mark 1}{(6.211,3.547)}
\gppoint{gp mark 1}{(6.213,3.165)}
\gppoint{gp mark 1}{(6.216,4.028)}
\gppoint{gp mark 1}{(6.218,3.793)}
\gppoint{gp mark 1}{(6.221,3.563)}
\gppoint{gp mark 1}{(6.223,2.897)}
\gppoint{gp mark 1}{(6.226,3.350)}
\gppoint{gp mark 1}{(6.228,3.968)}
\gppoint{gp mark 1}{(6.231,3.176)}
\gppoint{gp mark 1}{(6.233,4.022)}
\gppoint{gp mark 1}{(6.236,3.367)}
\gppoint{gp mark 1}{(6.238,3.427)}
\gppoint{gp mark 1}{(6.241,4.629)}
\gppoint{gp mark 1}{(6.243,3.585)}
\gppoint{gp mark 1}{(6.246,3.766)}
\gppoint{gp mark 1}{(6.248,3.503)}
\gppoint{gp mark 1}{(6.251,4.077)}
\gppoint{gp mark 1}{(6.253,3.487)}
\gppoint{gp mark 1}{(6.256,3.569)}
\gppoint{gp mark 1}{(6.258,3.634)}
\gppoint{gp mark 1}{(6.261,2.995)}
\gppoint{gp mark 1}{(6.263,3.356)}
\gppoint{gp mark 1}{(6.266,3.460)}
\gppoint{gp mark 1}{(6.268,3.143)}
\gppoint{gp mark 1}{(6.270,3.487)}
\gppoint{gp mark 1}{(6.273,3.449)}
\gppoint{gp mark 1}{(6.275,2.962)}
\gppoint{gp mark 1}{(6.278,3.618)}
\gppoint{gp mark 1}{(6.280,3.279)}
\gppoint{gp mark 1}{(6.283,4.077)}
\gppoint{gp mark 1}{(6.285,3.793)}
\gppoint{gp mark 1}{(6.288,4.323)}
\gppoint{gp mark 1}{(6.290,4.011)}
\gppoint{gp mark 1}{(6.293,2.826)}
\gppoint{gp mark 1}{(6.295,3.684)}
\gppoint{gp mark 1}{(6.298,2.902)}
\gppoint{gp mark 1}{(6.300,3.241)}
\gppoint{gp mark 1}{(6.303,3.574)}
\gppoint{gp mark 1}{(6.305,2.657)}
\gppoint{gp mark 1}{(6.308,3.066)}
\gppoint{gp mark 1}{(6.310,2.782)}
\gppoint{gp mark 1}{(6.313,2.924)}
\gppoint{gp mark 1}{(6.315,3.334)}
\gppoint{gp mark 1}{(6.318,4.060)}
\gppoint{gp mark 1}{(6.320,3.820)}
\gppoint{gp mark 1}{(6.323,3.853)}
\gppoint{gp mark 1}{(6.325,3.563)}
\gppoint{gp mark 1}{(6.328,3.716)}
\gppoint{gp mark 1}{(6.330,3.460)}
\gppoint{gp mark 1}{(6.333,3.099)}
\gppoint{gp mark 1}{(6.335,2.952)}
\gppoint{gp mark 1}{(6.338,2.990)}
\gppoint{gp mark 1}{(6.340,3.176)}
\gppoint{gp mark 1}{(6.343,3.170)}
\gppoint{gp mark 1}{(6.345,4.372)}
\gppoint{gp mark 1}{(6.348,3.924)}
\gppoint{gp mark 1}{(6.350,3.274)}
\gppoint{gp mark 1}{(6.352,4.082)}
\gppoint{gp mark 1}{(6.355,2.815)}
\gppoint{gp mark 1}{(6.357,3.176)}
\gppoint{gp mark 1}{(6.360,3.383)}
\gppoint{gp mark 1}{(6.362,4.492)}
\gppoint{gp mark 1}{(6.365,3.465)}
\gppoint{gp mark 1}{(6.367,3.525)}
\gppoint{gp mark 1}{(6.370,3.897)}
\gppoint{gp mark 1}{(6.372,3.558)}
\gppoint{gp mark 1}{(6.375,3.318)}
\gppoint{gp mark 1}{(6.377,3.006)}
\gppoint{gp mark 1}{(6.380,3.613)}
\gppoint{gp mark 1}{(6.382,3.006)}
\gppoint{gp mark 1}{(6.385,3.531)}
\gppoint{gp mark 1}{(6.387,3.290)}
\gppoint{gp mark 1}{(6.390,3.438)}
\gppoint{gp mark 1}{(6.392,3.055)}
\gppoint{gp mark 1}{(6.395,3.099)}
\gppoint{gp mark 1}{(6.397,3.766)}
\gppoint{gp mark 1}{(6.400,3.804)}
\gppoint{gp mark 1}{(6.402,2.897)}
\gppoint{gp mark 1}{(6.405,3.170)}
\gppoint{gp mark 1}{(6.407,3.580)}
\gppoint{gp mark 1}{(6.410,3.427)}
\gppoint{gp mark 1}{(6.412,2.695)}
\gppoint{gp mark 1}{(6.415,3.913)}
\gppoint{gp mark 1}{(6.417,3.449)}
\gppoint{gp mark 1}{(6.420,3.394)}
\gppoint{gp mark 1}{(6.422,2.771)}
\gppoint{gp mark 1}{(6.425,3.257)}
\gppoint{gp mark 1}{(6.427,3.312)}
\gppoint{gp mark 1}{(6.430,3.705)}
\gppoint{gp mark 1}{(6.432,3.880)}
\gppoint{gp mark 1}{(6.434,3.744)}
\gppoint{gp mark 1}{(6.437,3.394)}
\gppoint{gp mark 1}{(6.439,3.580)}
\gppoint{gp mark 1}{(6.442,2.646)}
\gppoint{gp mark 1}{(6.444,3.809)}
\gppoint{gp mark 1}{(6.447,3.738)}
\gppoint{gp mark 1}{(6.449,3.257)}
\gppoint{gp mark 1}{(6.452,3.547)}
\gppoint{gp mark 1}{(6.454,2.957)}
\gppoint{gp mark 1}{(6.457,3.995)}
\gppoint{gp mark 1}{(6.459,3.197)}
\gppoint{gp mark 1}{(6.462,3.481)}
\gppoint{gp mark 1}{(6.464,3.585)}
\gppoint{gp mark 1}{(6.467,3.542)}
\gppoint{gp mark 1}{(6.469,3.438)}
\gppoint{gp mark 1}{(6.472,3.755)}
\gppoint{gp mark 1}{(6.474,3.481)}
\gppoint{gp mark 1}{(6.477,3.722)}
\gppoint{gp mark 1}{(6.479,3.356)}
\gppoint{gp mark 1}{(6.482,3.148)}
\gppoint{gp mark 1}{(6.484,3.236)}
\gppoint{gp mark 1}{(6.487,3.257)}
\gppoint{gp mark 1}{(6.489,3.460)}
\gppoint{gp mark 1}{(6.492,3.492)}
\gppoint{gp mark 1}{(6.494,3.361)}
\gppoint{gp mark 1}{(6.497,3.525)}
\gppoint{gp mark 1}{(6.499,3.694)}
\gppoint{gp mark 1}{(6.502,4.006)}
\gppoint{gp mark 1}{(6.504,4.208)}
\gppoint{gp mark 1}{(6.507,3.962)}
\gppoint{gp mark 1}{(6.509,3.651)}
\gppoint{gp mark 1}{(6.512,3.700)}
\gppoint{gp mark 1}{(6.514,3.110)}
\gppoint{gp mark 1}{(6.517,3.438)}
\gppoint{gp mark 1}{(6.519,3.214)}
\gppoint{gp mark 1}{(6.521,2.793)}
\gppoint{gp mark 1}{(6.524,3.099)}
\gppoint{gp mark 1}{(6.526,2.842)}
\gppoint{gp mark 1}{(6.529,3.897)}
\gppoint{gp mark 1}{(6.531,2.596)}
\gppoint{gp mark 1}{(6.534,3.443)}
\gppoint{gp mark 1}{(6.536,3.176)}
\gppoint{gp mark 1}{(6.539,3.290)}
\gppoint{gp mark 1}{(6.541,2.755)}
\gppoint{gp mark 1}{(6.544,3.389)}
\gppoint{gp mark 1}{(6.546,4.203)}
\gppoint{gp mark 1}{(6.549,3.017)}
\gppoint{gp mark 1}{(6.551,2.362)}
\gppoint{gp mark 1}{(6.554,3.858)}
\gppoint{gp mark 1}{(6.556,2.962)}
\gppoint{gp mark 1}{(6.559,3.569)}
\gppoint{gp mark 1}{(6.561,3.361)}
\gppoint{gp mark 1}{(6.564,3.733)}
\gppoint{gp mark 1}{(6.566,2.919)}
\gppoint{gp mark 1}{(6.569,3.760)}
\gppoint{gp mark 1}{(6.571,3.924)}
\gppoint{gp mark 1}{(6.574,3.569)}
\gppoint{gp mark 1}{(6.576,3.034)}
\gppoint{gp mark 1}{(6.579,3.673)}
\gppoint{gp mark 1}{(6.581,2.804)}
\gppoint{gp mark 1}{(6.584,3.618)}
\gppoint{gp mark 1}{(6.586,3.798)}
\gppoint{gp mark 1}{(6.589,3.061)}
\gppoint{gp mark 1}{(6.591,3.656)}
\gppoint{gp mark 1}{(6.594,3.416)}
\gppoint{gp mark 1}{(6.596,2.919)}
\gppoint{gp mark 1}{(6.599,3.749)}
\gppoint{gp mark 1}{(6.601,3.705)}
\gppoint{gp mark 1}{(6.603,2.935)}
\gppoint{gp mark 1}{(6.606,3.656)}
\gppoint{gp mark 1}{(6.608,3.094)}
\gppoint{gp mark 1}{(6.611,3.105)}
\gppoint{gp mark 1}{(6.613,3.361)}
\gppoint{gp mark 1}{(6.616,2.826)}
\gppoint{gp mark 1}{(6.618,3.766)}
\gppoint{gp mark 1}{(6.621,3.831)}
\gppoint{gp mark 1}{(6.623,3.880)}
\gppoint{gp mark 1}{(6.626,3.880)}
\gppoint{gp mark 1}{(6.628,2.482)}
\gppoint{gp mark 1}{(6.631,3.383)}
\gppoint{gp mark 1}{(6.633,3.787)}
\gppoint{gp mark 1}{(6.636,2.995)}
\gppoint{gp mark 1}{(6.638,3.230)}
\gppoint{gp mark 1}{(6.641,3.247)}
\gppoint{gp mark 1}{(6.643,3.547)}
\gppoint{gp mark 1}{(6.646,3.427)}
\gppoint{gp mark 1}{(6.648,3.225)}
\gppoint{gp mark 1}{(6.651,3.733)}
\gppoint{gp mark 1}{(6.653,3.137)}
\gppoint{gp mark 1}{(6.656,4.050)}
\gppoint{gp mark 1}{(6.658,3.563)}
\gppoint{gp mark 1}{(6.661,3.170)}
\gppoint{gp mark 1}{(6.663,3.558)}
\gppoint{gp mark 1}{(6.666,3.536)}
\gppoint{gp mark 1}{(6.668,2.881)}
\gppoint{gp mark 1}{(6.671,3.115)}
\gppoint{gp mark 1}{(6.673,3.531)}
\gppoint{gp mark 1}{(6.676,3.001)}
\gppoint{gp mark 1}{(6.678,3.389)}
\gppoint{gp mark 1}{(6.681,3.017)}
\gppoint{gp mark 1}{(6.683,3.301)}
\gppoint{gp mark 1}{(6.685,2.946)}
\gppoint{gp mark 1}{(6.688,4.088)}
\gppoint{gp mark 1}{(6.690,3.645)}
\gppoint{gp mark 1}{(6.693,3.121)}
\gppoint{gp mark 1}{(6.695,3.126)}
\gppoint{gp mark 1}{(6.698,3.219)}
\gppoint{gp mark 1}{(6.700,3.465)}
\gppoint{gp mark 1}{(6.703,3.438)}
\gppoint{gp mark 1}{(6.705,3.034)}
\gppoint{gp mark 1}{(6.708,3.192)}
\gppoint{gp mark 1}{(6.710,3.760)}
\gppoint{gp mark 1}{(6.713,2.728)}
\gppoint{gp mark 1}{(6.715,3.055)}
\gppoint{gp mark 1}{(6.718,2.908)}
\gppoint{gp mark 1}{(6.720,2.618)}
\gppoint{gp mark 1}{(6.723,3.121)}
\gppoint{gp mark 1}{(6.725,2.902)}
\gppoint{gp mark 1}{(6.728,2.984)}
\gppoint{gp mark 1}{(6.730,3.197)}
\gppoint{gp mark 1}{(6.733,2.919)}
\gppoint{gp mark 1}{(6.735,4.399)}
\gppoint{gp mark 1}{(6.738,3.132)}
\gppoint{gp mark 1}{(6.740,3.203)}
\gppoint{gp mark 1}{(6.743,3.066)}
\gppoint{gp mark 1}{(6.745,3.230)}
\gppoint{gp mark 1}{(6.748,3.247)}
\gppoint{gp mark 1}{(6.750,3.438)}
\gppoint{gp mark 1}{(6.753,3.722)}
\gppoint{gp mark 1}{(6.755,3.137)}
\gppoint{gp mark 1}{(6.758,3.542)}
\gppoint{gp mark 1}{(6.760,2.569)}
\gppoint{gp mark 1}{(6.763,3.656)}
\gppoint{gp mark 1}{(6.765,3.891)}
\gppoint{gp mark 1}{(6.768,3.716)}
\gppoint{gp mark 1}{(6.770,2.482)}
\gppoint{gp mark 1}{(6.772,3.591)}
\gppoint{gp mark 1}{(6.775,3.219)}
\gppoint{gp mark 1}{(6.777,3.694)}
\gppoint{gp mark 1}{(6.780,3.252)}
\gppoint{gp mark 1}{(6.782,3.793)}
\gppoint{gp mark 1}{(6.785,3.023)}
\gppoint{gp mark 1}{(6.787,3.514)}
\gppoint{gp mark 1}{(6.790,3.274)}
\gppoint{gp mark 1}{(6.792,2.766)}
\gppoint{gp mark 1}{(6.795,3.634)}
\gppoint{gp mark 1}{(6.797,3.536)}
\gppoint{gp mark 1}{(6.800,2.908)}
\gppoint{gp mark 1}{(6.802,3.613)}
\gppoint{gp mark 1}{(6.805,3.028)}
\gppoint{gp mark 1}{(6.807,3.077)}
\gppoint{gp mark 1}{(6.810,3.137)}
\gppoint{gp mark 1}{(6.812,3.285)}
\gppoint{gp mark 1}{(6.815,3.514)}
\gppoint{gp mark 1}{(6.817,3.711)}
\gppoint{gp mark 1}{(6.820,2.799)}
\gppoint{gp mark 1}{(6.822,3.252)}
\gppoint{gp mark 1}{(6.825,3.498)}
\gppoint{gp mark 1}{(6.827,3.350)}
\gppoint{gp mark 1}{(6.830,3.083)}
\gppoint{gp mark 1}{(6.832,3.268)}
\gppoint{gp mark 1}{(6.835,3.148)}
\gppoint{gp mark 1}{(6.837,3.236)}
\gppoint{gp mark 1}{(6.840,3.547)}
\gppoint{gp mark 1}{(6.842,3.099)}
\gppoint{gp mark 1}{(6.845,3.307)}
\gppoint{gp mark 1}{(6.847,3.673)}
\gppoint{gp mark 1}{(6.850,3.159)}
\gppoint{gp mark 1}{(6.852,3.951)}
\gppoint{gp mark 1}{(6.854,3.542)}
\gppoint{gp mark 1}{(6.857,3.531)}
\gppoint{gp mark 1}{(6.859,3.361)}
\gppoint{gp mark 1}{(6.862,3.563)}
\gppoint{gp mark 1}{(6.864,3.716)}
\gppoint{gp mark 1}{(6.867,2.875)}
\gppoint{gp mark 1}{(6.869,3.929)}
\gppoint{gp mark 1}{(6.872,3.771)}
\gppoint{gp mark 1}{(6.874,3.083)}
\gppoint{gp mark 1}{(6.877,2.771)}
\gppoint{gp mark 1}{(6.879,3.760)}
\gppoint{gp mark 1}{(6.882,3.509)}
\gppoint{gp mark 1}{(6.884,4.022)}
\gppoint{gp mark 1}{(6.887,3.170)}
\gppoint{gp mark 1}{(6.889,2.777)}
\gppoint{gp mark 1}{(6.892,3.170)}
\gppoint{gp mark 1}{(6.894,2.771)}
\gppoint{gp mark 1}{(6.897,3.061)}
\gppoint{gp mark 1}{(6.899,3.257)}
\gppoint{gp mark 1}{(6.902,3.143)}
\gppoint{gp mark 1}{(6.904,3.684)}
\gppoint{gp mark 1}{(6.907,3.629)}
\gppoint{gp mark 1}{(6.909,3.279)}
\gppoint{gp mark 1}{(6.912,2.941)}
\gppoint{gp mark 1}{(6.914,3.077)}
\gppoint{gp mark 1}{(6.917,3.121)}
\gppoint{gp mark 1}{(6.919,2.842)}
\gppoint{gp mark 1}{(6.922,3.913)}
\gppoint{gp mark 1}{(6.924,3.088)}
\gppoint{gp mark 1}{(6.927,4.050)}
\gppoint{gp mark 1}{(6.929,3.170)}
\gppoint{gp mark 1}{(6.932,3.460)}
\gppoint{gp mark 1}{(6.934,3.897)}
\gppoint{gp mark 1}{(6.936,3.410)}
\gppoint{gp mark 1}{(6.939,3.389)}
\gppoint{gp mark 1}{(6.941,3.339)}
\gppoint{gp mark 1}{(6.944,2.591)}
\gppoint{gp mark 1}{(6.946,4.132)}
\gppoint{gp mark 1}{(6.949,3.290)}
\gppoint{gp mark 1}{(6.951,3.034)}
\gppoint{gp mark 1}{(6.954,3.334)}
\gppoint{gp mark 1}{(6.956,4.017)}
\gppoint{gp mark 1}{(6.959,3.274)}
\gppoint{gp mark 1}{(6.961,3.055)}
\gppoint{gp mark 1}{(6.964,3.219)}
\gppoint{gp mark 1}{(6.966,3.247)}
\gppoint{gp mark 1}{(6.969,3.443)}
\gppoint{gp mark 1}{(6.971,3.804)}
\gppoint{gp mark 1}{(6.974,3.886)}
\gppoint{gp mark 1}{(6.976,3.356)}
\gppoint{gp mark 1}{(6.979,3.328)}
\gppoint{gp mark 1}{(6.981,3.356)}
\gppoint{gp mark 1}{(6.984,3.569)}
\gppoint{gp mark 1}{(6.986,3.673)}
\gppoint{gp mark 1}{(6.989,3.072)}
\gppoint{gp mark 1}{(6.991,3.252)}
\gppoint{gp mark 1}{(6.994,3.405)}
\gppoint{gp mark 1}{(6.996,3.328)}
\gppoint{gp mark 1}{(6.999,3.192)}
\gppoint{gp mark 1}{(7.001,3.471)}
\gppoint{gp mark 1}{(7.004,3.094)}
\gppoint{gp mark 1}{(7.006,3.345)}
\gppoint{gp mark 1}{(7.009,2.422)}
\gppoint{gp mark 1}{(7.011,3.640)}
\gppoint{gp mark 1}{(7.014,2.853)}
\gppoint{gp mark 1}{(7.016,3.170)}
\gppoint{gp mark 1}{(7.019,2.946)}
\gppoint{gp mark 1}{(7.021,2.520)}
\gppoint{gp mark 1}{(7.023,3.197)}
\gppoint{gp mark 1}{(7.026,3.197)}
\gppoint{gp mark 1}{(7.028,3.197)}
\gppoint{gp mark 1}{(7.031,2.744)}
\gppoint{gp mark 1}{(7.033,2.673)}
\gppoint{gp mark 1}{(7.036,3.394)}
\gppoint{gp mark 1}{(7.038,3.383)}
\gppoint{gp mark 1}{(7.041,3.208)}
\gppoint{gp mark 1}{(7.043,3.236)}
\gppoint{gp mark 1}{(7.046,2.624)}
\gppoint{gp mark 1}{(7.048,3.214)}
\gppoint{gp mark 1}{(7.051,3.203)}
\gppoint{gp mark 1}{(7.053,3.307)}
\gppoint{gp mark 1}{(7.056,3.367)}
\gppoint{gp mark 1}{(7.058,2.799)}
\gppoint{gp mark 1}{(7.061,3.476)}
\gppoint{gp mark 1}{(7.063,3.077)}
\gppoint{gp mark 1}{(7.066,3.421)}
\gppoint{gp mark 1}{(7.068,3.782)}
\gppoint{gp mark 1}{(7.071,4.066)}
\gppoint{gp mark 1}{(7.073,2.897)}
\gppoint{gp mark 1}{(7.076,2.973)}
\gppoint{gp mark 1}{(7.078,2.312)}
\gppoint{gp mark 1}{(7.081,3.449)}
\gppoint{gp mark 1}{(7.083,3.405)}
\gppoint{gp mark 1}{(7.086,2.607)}
\gppoint{gp mark 1}{(7.088,2.919)}
\gppoint{gp mark 1}{(7.091,3.460)}
\gppoint{gp mark 1}{(7.093,2.859)}
\gppoint{gp mark 1}{(7.096,3.449)}
\gppoint{gp mark 1}{(7.098,3.274)}
\gppoint{gp mark 1}{(7.101,2.870)}
\gppoint{gp mark 1}{(7.103,3.503)}
\gppoint{gp mark 1}{(7.105,3.716)}
\gppoint{gp mark 1}{(7.108,3.476)}
\gppoint{gp mark 1}{(7.110,3.339)}
\gppoint{gp mark 1}{(7.113,3.542)}
\gppoint{gp mark 1}{(7.115,2.935)}
\gppoint{gp mark 1}{(7.118,3.257)}
\gppoint{gp mark 1}{(7.120,2.859)}
\gppoint{gp mark 1}{(7.123,3.230)}
\gppoint{gp mark 1}{(7.125,2.935)}
\gppoint{gp mark 1}{(7.128,3.760)}
\gppoint{gp mark 1}{(7.130,3.110)}
\gppoint{gp mark 1}{(7.133,2.908)}
\gppoint{gp mark 1}{(7.135,3.989)}
\gppoint{gp mark 1}{(7.138,3.498)}
\gppoint{gp mark 1}{(7.140,3.591)}
\gppoint{gp mark 1}{(7.143,2.881)}
\gppoint{gp mark 1}{(7.145,3.105)}
\gppoint{gp mark 1}{(7.148,3.563)}
\gppoint{gp mark 1}{(7.150,2.990)}
\gppoint{gp mark 1}{(7.153,2.607)}
\gppoint{gp mark 1}{(7.155,3.219)}
\gppoint{gp mark 1}{(7.158,2.678)}
\gppoint{gp mark 1}{(7.160,3.531)}
\gppoint{gp mark 1}{(7.163,3.793)}
\gppoint{gp mark 1}{(7.165,2.668)}
\gppoint{gp mark 1}{(7.168,3.531)}
\gppoint{gp mark 1}{(7.170,3.028)}
\gppoint{gp mark 1}{(7.173,3.809)}
\gppoint{gp mark 1}{(7.175,3.197)}
\gppoint{gp mark 1}{(7.178,3.782)}
\gppoint{gp mark 1}{(7.180,2.962)}
\gppoint{gp mark 1}{(7.183,3.421)}
\gppoint{gp mark 1}{(7.185,3.318)}
\gppoint{gp mark 1}{(7.187,3.465)}
\gppoint{gp mark 1}{(7.190,2.378)}
\gppoint{gp mark 1}{(7.192,3.307)}
\gppoint{gp mark 1}{(7.195,3.203)}
\gppoint{gp mark 1}{(7.197,3.689)}
\gppoint{gp mark 1}{(7.200,2.979)}
\gppoint{gp mark 1}{(7.202,3.236)}
\gppoint{gp mark 1}{(7.205,3.400)}
\gppoint{gp mark 1}{(7.207,3.186)}
\gppoint{gp mark 1}{(7.210,3.372)}
\gppoint{gp mark 1}{(7.212,2.749)}
\gppoint{gp mark 1}{(7.215,4.421)}
\gppoint{gp mark 1}{(7.217,3.476)}
\gppoint{gp mark 1}{(7.220,3.356)}
\gppoint{gp mark 1}{(7.222,3.492)}
\gppoint{gp mark 1}{(7.225,2.717)}
\gppoint{gp mark 1}{(7.227,3.509)}
\gppoint{gp mark 1}{(7.230,3.613)}
\gppoint{gp mark 1}{(7.232,3.208)}
\gppoint{gp mark 1}{(7.235,3.268)}
\gppoint{gp mark 1}{(7.237,3.520)}
\gppoint{gp mark 1}{(7.240,3.454)}
\gppoint{gp mark 1}{(7.242,2.826)}
\gppoint{gp mark 1}{(7.245,3.716)}
\gppoint{gp mark 1}{(7.247,3.094)}
\gppoint{gp mark 1}{(7.250,3.378)}
\gppoint{gp mark 1}{(7.252,2.351)}
\gppoint{gp mark 1}{(7.255,2.935)}
\gppoint{gp mark 1}{(7.257,3.039)}
\gppoint{gp mark 1}{(7.260,3.815)}
\gppoint{gp mark 1}{(7.262,3.339)}
\gppoint{gp mark 1}{(7.265,3.361)}
\gppoint{gp mark 1}{(7.267,3.034)}
\gppoint{gp mark 1}{(7.269,2.919)}
\gppoint{gp mark 1}{(7.272,2.700)}
\gppoint{gp mark 1}{(7.274,3.023)}
\gppoint{gp mark 1}{(7.277,3.044)}
\gppoint{gp mark 1}{(7.279,2.695)}
\gppoint{gp mark 1}{(7.282,3.651)}
\gppoint{gp mark 1}{(7.284,3.214)}
\gppoint{gp mark 1}{(7.287,2.755)}
\gppoint{gp mark 1}{(7.289,3.121)}
\gppoint{gp mark 1}{(7.292,3.296)}
\gppoint{gp mark 1}{(7.294,2.995)}
\gppoint{gp mark 1}{(7.297,2.487)}
\gppoint{gp mark 1}{(7.299,4.170)}
\gppoint{gp mark 1}{(7.302,3.132)}
\gppoint{gp mark 1}{(7.304,3.312)}
\gppoint{gp mark 1}{(7.307,3.722)}
\gppoint{gp mark 1}{(7.309,3.023)}
\gppoint{gp mark 1}{(7.312,2.946)}
\gppoint{gp mark 1}{(7.314,3.361)}
\gppoint{gp mark 1}{(7.317,3.110)}
\gppoint{gp mark 1}{(7.319,3.186)}
\gppoint{gp mark 1}{(7.322,3.186)}
\gppoint{gp mark 1}{(7.324,4.104)}
\gppoint{gp mark 1}{(7.327,2.684)}
\gppoint{gp mark 1}{(7.329,3.574)}
\gppoint{gp mark 1}{(7.332,2.722)}
\gppoint{gp mark 1}{(7.334,3.110)}
\gppoint{gp mark 1}{(7.337,3.372)}
\gppoint{gp mark 1}{(7.339,2.613)}
\gppoint{gp mark 1}{(7.342,3.350)}
\gppoint{gp mark 1}{(7.344,2.192)}
\gppoint{gp mark 1}{(7.347,3.001)}
\gppoint{gp mark 1}{(7.349,2.542)}
\gppoint{gp mark 1}{(7.352,3.263)}
\gppoint{gp mark 1}{(7.354,3.143)}
\gppoint{gp mark 1}{(7.356,3.585)}
\gppoint{gp mark 1}{(7.359,3.285)}
\gppoint{gp mark 1}{(7.361,3.416)}
\gppoint{gp mark 1}{(7.364,3.099)}
\gppoint{gp mark 1}{(7.366,2.984)}
\gppoint{gp mark 1}{(7.369,2.285)}
\gppoint{gp mark 1}{(7.371,3.328)}
\gppoint{gp mark 1}{(7.374,2.875)}
\gppoint{gp mark 1}{(7.376,3.465)}
\gppoint{gp mark 1}{(7.379,2.586)}
\gppoint{gp mark 1}{(7.381,3.460)}
\gppoint{gp mark 1}{(7.384,3.023)}
\gppoint{gp mark 1}{(7.386,2.848)}
\gppoint{gp mark 1}{(7.389,2.782)}
\gppoint{gp mark 1}{(7.391,3.427)}
\gppoint{gp mark 1}{(7.394,3.137)}
\gppoint{gp mark 1}{(7.396,3.285)}
\gppoint{gp mark 1}{(7.399,3.001)}
\gppoint{gp mark 1}{(7.401,3.268)}
\gppoint{gp mark 1}{(7.404,2.689)}
\gppoint{gp mark 1}{(7.406,3.383)}
\gppoint{gp mark 1}{(7.409,3.339)}
\gppoint{gp mark 1}{(7.411,2.777)}
\gppoint{gp mark 1}{(7.414,3.236)}
\gppoint{gp mark 1}{(7.416,3.203)}
\gppoint{gp mark 1}{(7.419,3.416)}
\gppoint{gp mark 1}{(7.421,3.301)}
\gppoint{gp mark 1}{(7.424,3.094)}
\gppoint{gp mark 1}{(7.426,3.427)}
\gppoint{gp mark 1}{(7.429,3.438)}
\gppoint{gp mark 1}{(7.431,2.902)}
\gppoint{gp mark 1}{(7.434,3.350)}
\gppoint{gp mark 1}{(7.436,3.454)}
\gppoint{gp mark 1}{(7.438,3.995)}
\gppoint{gp mark 1}{(7.441,3.192)}
\gppoint{gp mark 1}{(7.443,3.186)}
\gppoint{gp mark 1}{(7.446,2.919)}
\gppoint{gp mark 1}{(7.448,3.077)}
\gppoint{gp mark 1}{(7.451,2.881)}
\gppoint{gp mark 1}{(7.453,3.563)}
\gppoint{gp mark 1}{(7.456,3.323)}
\gppoint{gp mark 1}{(7.458,3.247)}
\gppoint{gp mark 1}{(7.461,2.870)}
\gppoint{gp mark 1}{(7.463,3.809)}
\gppoint{gp mark 1}{(7.466,2.515)}
\gppoint{gp mark 1}{(7.468,3.143)}
\gppoint{gp mark 1}{(7.471,2.979)}
\gppoint{gp mark 1}{(7.473,3.449)}
\gppoint{gp mark 1}{(7.476,3.285)}
\gppoint{gp mark 1}{(7.478,3.328)}
\gppoint{gp mark 1}{(7.481,3.219)}
\gppoint{gp mark 1}{(7.483,2.826)}
\gppoint{gp mark 1}{(7.486,2.831)}
\gppoint{gp mark 1}{(7.488,2.733)}
\gppoint{gp mark 1}{(7.491,2.154)}
\gppoint{gp mark 1}{(7.493,2.285)}
\gppoint{gp mark 1}{(7.496,3.443)}
\gppoint{gp mark 1}{(7.498,2.990)}
\gppoint{gp mark 1}{(7.501,3.449)}
\gppoint{gp mark 1}{(7.503,2.897)}
\gppoint{gp mark 1}{(7.506,3.449)}
\gppoint{gp mark 1}{(7.508,2.930)}
\gppoint{gp mark 1}{(7.511,3.618)}
\gppoint{gp mark 1}{(7.513,3.039)}
\gppoint{gp mark 1}{(7.516,2.842)}
\gppoint{gp mark 1}{(7.518,3.290)}
\gppoint{gp mark 1}{(7.520,2.722)}
\gppoint{gp mark 1}{(7.523,3.208)}
\gppoint{gp mark 1}{(7.525,3.585)}
\gppoint{gp mark 1}{(7.528,2.678)}
\gppoint{gp mark 1}{(7.530,2.952)}
\gppoint{gp mark 1}{(7.533,2.733)}
\gppoint{gp mark 1}{(7.535,3.126)}
\gppoint{gp mark 1}{(7.538,2.908)}
\gppoint{gp mark 1}{(7.540,3.410)}
\gppoint{gp mark 1}{(7.543,3.154)}
\gppoint{gp mark 1}{(7.545,3.257)}
\gppoint{gp mark 1}{(7.548,3.574)}
\gppoint{gp mark 1}{(7.550,3.252)}
\gppoint{gp mark 1}{(7.553,2.826)}
\gppoint{gp mark 1}{(7.555,3.301)}
\gppoint{gp mark 1}{(7.558,2.717)}
\gppoint{gp mark 1}{(7.560,3.121)}
\gppoint{gp mark 1}{(7.563,3.285)}
\gppoint{gp mark 1}{(7.565,3.307)}
\gppoint{gp mark 1}{(7.568,2.913)}
\gppoint{gp mark 1}{(7.570,3.050)}
\gppoint{gp mark 1}{(7.573,3.296)}
\gppoint{gp mark 1}{(7.575,2.799)}
\gppoint{gp mark 1}{(7.578,3.137)}
\gppoint{gp mark 1}{(7.580,2.520)}
\gppoint{gp mark 1}{(7.583,2.760)}
\gppoint{gp mark 1}{(7.585,2.886)}
\gppoint{gp mark 1}{(7.588,3.443)}
\gppoint{gp mark 1}{(7.590,3.312)}
\gppoint{gp mark 1}{(7.593,2.902)}
\gppoint{gp mark 1}{(7.595,3.580)}
\gppoint{gp mark 1}{(7.598,3.252)}
\gppoint{gp mark 1}{(7.600,2.788)}
\gppoint{gp mark 1}{(7.603,3.236)}
\gppoint{gp mark 1}{(7.605,2.673)}
\gppoint{gp mark 1}{(7.607,3.869)}
\gppoint{gp mark 1}{(7.610,2.673)}
\gppoint{gp mark 1}{(7.612,3.055)}
\gppoint{gp mark 1}{(7.615,3.378)}
\gppoint{gp mark 1}{(7.617,2.908)}
\gppoint{gp mark 1}{(7.620,2.897)}
\gppoint{gp mark 1}{(7.622,3.290)}
\gppoint{gp mark 1}{(7.625,3.372)}
\gppoint{gp mark 1}{(7.627,3.716)}
\gppoint{gp mark 1}{(7.630,3.055)}
\gppoint{gp mark 1}{(7.632,2.848)}
\gppoint{gp mark 1}{(7.635,3.006)}
\gppoint{gp mark 1}{(7.637,3.416)}
\gppoint{gp mark 1}{(7.640,3.137)}
\gppoint{gp mark 1}{(7.642,2.487)}
\gppoint{gp mark 1}{(7.645,3.083)}
\gppoint{gp mark 1}{(7.647,3.285)}
\gppoint{gp mark 1}{(7.650,2.853)}
\gppoint{gp mark 1}{(7.652,2.520)}
\gppoint{gp mark 1}{(7.655,3.110)}
\gppoint{gp mark 1}{(7.657,2.881)}
\gppoint{gp mark 1}{(7.660,3.481)}
\gppoint{gp mark 1}{(7.662,2.941)}
\gppoint{gp mark 1}{(7.665,3.028)}
\gppoint{gp mark 1}{(7.667,3.585)}
\gppoint{gp mark 1}{(7.670,3.826)}
\gppoint{gp mark 1}{(7.672,3.476)}
\gppoint{gp mark 1}{(7.675,3.372)}
\gppoint{gp mark 1}{(7.677,3.279)}
\gppoint{gp mark 1}{(7.680,2.848)}
\gppoint{gp mark 1}{(7.682,3.296)}
\gppoint{gp mark 1}{(7.685,3.727)}
\gppoint{gp mark 1}{(7.687,3.394)}
\gppoint{gp mark 1}{(7.689,2.995)}
\gppoint{gp mark 1}{(7.692,3.350)}
\gppoint{gp mark 1}{(7.694,2.870)}
\gppoint{gp mark 1}{(7.697,2.640)}
\gppoint{gp mark 1}{(7.699,3.137)}
\gppoint{gp mark 1}{(7.702,2.962)}
\gppoint{gp mark 1}{(7.704,3.159)}
\gppoint{gp mark 1}{(7.707,3.356)}
\gppoint{gp mark 1}{(7.709,2.990)}
\gppoint{gp mark 1}{(7.712,2.378)}
\gppoint{gp mark 1}{(7.714,3.684)}
\gppoint{gp mark 1}{(7.717,3.651)}
\gppoint{gp mark 1}{(7.719,2.891)}
\gppoint{gp mark 1}{(7.722,2.662)}
\gppoint{gp mark 1}{(7.724,3.219)}
\gppoint{gp mark 1}{(7.727,3.547)}
\gppoint{gp mark 1}{(7.729,3.214)}
\gppoint{gp mark 1}{(7.732,3.815)}
\gppoint{gp mark 1}{(7.734,3.203)}
\gppoint{gp mark 1}{(7.737,2.973)}
\gppoint{gp mark 1}{(7.739,3.421)}
\gppoint{gp mark 1}{(7.742,2.482)}
\gppoint{gp mark 1}{(7.744,2.444)}
\gppoint{gp mark 1}{(7.747,3.023)}
\gppoint{gp mark 1}{(7.749,3.684)}
\gppoint{gp mark 1}{(7.752,2.924)}
\gppoint{gp mark 1}{(7.754,3.487)}
\gppoint{gp mark 1}{(7.757,2.416)}
\gppoint{gp mark 1}{(7.759,3.126)}
\gppoint{gp mark 1}{(7.762,2.810)}
\gppoint{gp mark 1}{(7.764,3.088)}
\gppoint{gp mark 1}{(7.767,3.115)}
\gppoint{gp mark 1}{(7.769,3.389)}
\gppoint{gp mark 1}{(7.771,2.449)}
\gppoint{gp mark 1}{(7.774,3.192)}
\gppoint{gp mark 1}{(7.776,2.525)}
\gppoint{gp mark 1}{(7.779,3.225)}
\gppoint{gp mark 1}{(7.781,2.771)}
\gppoint{gp mark 1}{(7.784,3.105)}
\gppoint{gp mark 1}{(7.786,3.247)}
\gppoint{gp mark 1}{(7.789,3.247)}
\gppoint{gp mark 1}{(7.791,3.394)}
\gppoint{gp mark 1}{(7.794,2.624)}
\gppoint{gp mark 1}{(7.796,3.629)}
\gppoint{gp mark 1}{(7.799,2.760)}
\gppoint{gp mark 1}{(7.801,2.930)}
\gppoint{gp mark 1}{(7.804,2.946)}
\gppoint{gp mark 1}{(7.806,3.061)}
\gppoint{gp mark 1}{(7.809,2.564)}
\gppoint{gp mark 1}{(7.811,2.984)}
\gppoint{gp mark 1}{(7.814,2.897)}
\gppoint{gp mark 1}{(7.816,2.968)}
\gppoint{gp mark 1}{(7.819,2.984)}
\gppoint{gp mark 1}{(7.821,2.711)}
\gppoint{gp mark 1}{(7.824,2.962)}
\gppoint{gp mark 1}{(7.826,3.023)}
\gppoint{gp mark 1}{(7.829,2.793)}
\gppoint{gp mark 1}{(7.831,3.110)}
\gppoint{gp mark 1}{(7.834,3.350)}
\gppoint{gp mark 1}{(7.836,3.307)}
\gppoint{gp mark 1}{(7.839,2.881)}
\gppoint{gp mark 1}{(7.841,3.891)}
\gppoint{gp mark 1}{(7.844,2.493)}
\gppoint{gp mark 1}{(7.846,2.733)}
\gppoint{gp mark 1}{(7.849,3.400)}
\gppoint{gp mark 1}{(7.851,2.607)}
\gppoint{gp mark 1}{(7.854,3.339)}
\gppoint{gp mark 1}{(7.856,3.181)}
\gppoint{gp mark 1}{(7.858,2.919)}
\gppoint{gp mark 1}{(7.861,2.744)}
\gppoint{gp mark 1}{(7.863,2.826)}
\gppoint{gp mark 1}{(7.866,2.777)}
\gppoint{gp mark 1}{(7.868,3.350)}
\gppoint{gp mark 1}{(7.871,2.826)}
\gppoint{gp mark 1}{(7.873,2.826)}
\gppoint{gp mark 1}{(7.876,3.105)}
\gppoint{gp mark 1}{(7.878,3.214)}
\gppoint{gp mark 1}{(7.881,2.799)}
\gppoint{gp mark 1}{(7.883,2.706)}
\gppoint{gp mark 1}{(7.886,3.186)}
\gppoint{gp mark 1}{(7.888,2.695)}
\gppoint{gp mark 1}{(7.891,3.088)}
\gppoint{gp mark 1}{(7.893,3.951)}
\gppoint{gp mark 1}{(7.896,3.307)}
\gppoint{gp mark 1}{(7.898,3.176)}
\gppoint{gp mark 1}{(7.901,3.061)}
\gppoint{gp mark 1}{(7.903,3.197)}
\gppoint{gp mark 1}{(7.906,3.061)}
\gppoint{gp mark 1}{(7.908,3.481)}
\gppoint{gp mark 1}{(7.911,2.755)}
\gppoint{gp mark 1}{(7.913,2.209)}
\gppoint{gp mark 1}{(7.916,3.148)}
\gppoint{gp mark 1}{(7.918,3.121)}
\gppoint{gp mark 1}{(7.921,2.717)}
\gppoint{gp mark 1}{(7.923,3.257)}
\gppoint{gp mark 1}{(7.926,2.367)}
\gppoint{gp mark 1}{(7.928,2.596)}
\gppoint{gp mark 1}{(7.931,3.208)}
\gppoint{gp mark 1}{(7.933,2.820)}
\gppoint{gp mark 1}{(7.936,3.208)}
\gppoint{gp mark 1}{(7.938,3.088)}
\gppoint{gp mark 1}{(7.940,2.826)}
\gppoint{gp mark 1}{(7.943,3.580)}
\gppoint{gp mark 1}{(7.945,3.689)}
\gppoint{gp mark 1}{(7.948,2.651)}
\gppoint{gp mark 1}{(7.950,2.023)}
\gppoint{gp mark 1}{(7.953,3.476)}
\gppoint{gp mark 1}{(7.955,2.881)}
\gppoint{gp mark 1}{(7.958,2.842)}
\gppoint{gp mark 1}{(7.960,2.728)}
\gppoint{gp mark 1}{(7.963,3.760)}
\gppoint{gp mark 1}{(7.965,2.782)}
\gppoint{gp mark 1}{(7.968,3.738)}
\gppoint{gp mark 1}{(7.970,2.717)}
\gppoint{gp mark 1}{(7.973,3.236)}
\gppoint{gp mark 1}{(7.975,3.132)}
\gppoint{gp mark 1}{(7.978,3.471)}
\gppoint{gp mark 1}{(7.980,3.176)}
\gppoint{gp mark 1}{(7.983,3.312)}
\gppoint{gp mark 1}{(7.985,3.137)}
\gppoint{gp mark 1}{(7.988,3.066)}
\gppoint{gp mark 1}{(7.990,2.586)}
\gppoint{gp mark 1}{(7.993,2.285)}
\gppoint{gp mark 1}{(7.995,3.356)}
\gppoint{gp mark 1}{(7.998,2.438)}
\gppoint{gp mark 1}{(8.000,3.312)}
\gppoint{gp mark 1}{(8.003,3.405)}
\gppoint{gp mark 1}{(8.005,2.733)}
\gppoint{gp mark 1}{(8.008,2.886)}
\gppoint{gp mark 1}{(8.010,2.891)}
\gppoint{gp mark 1}{(8.013,3.028)}
\gppoint{gp mark 1}{(8.015,3.241)}
\gppoint{gp mark 1}{(8.018,2.602)}
\gppoint{gp mark 1}{(8.020,3.498)}
\gppoint{gp mark 1}{(8.022,3.361)}
\gppoint{gp mark 1}{(8.025,3.640)}
\gppoint{gp mark 1}{(8.027,3.028)}
\gppoint{gp mark 1}{(8.030,2.760)}
\gppoint{gp mark 1}{(8.032,2.979)}
\gppoint{gp mark 1}{(8.035,2.400)}
\gppoint{gp mark 1}{(8.037,3.722)}
\gppoint{gp mark 1}{(8.040,2.930)}
\gppoint{gp mark 1}{(8.042,3.001)}
\gppoint{gp mark 1}{(8.045,2.962)}
\gppoint{gp mark 1}{(8.047,2.706)}
\gppoint{gp mark 1}{(8.050,2.946)}
\gppoint{gp mark 1}{(8.052,2.471)}
\gppoint{gp mark 1}{(8.055,2.749)}
\gppoint{gp mark 1}{(8.057,2.591)}
\gppoint{gp mark 1}{(8.060,3.132)}
\gppoint{gp mark 1}{(8.062,2.908)}
\gppoint{gp mark 1}{(8.065,2.913)}
\gppoint{gp mark 1}{(8.067,2.771)}
\gppoint{gp mark 1}{(8.070,2.569)}
\gppoint{gp mark 1}{(8.072,3.580)}
\gppoint{gp mark 1}{(8.075,3.405)}
\gppoint{gp mark 1}{(8.077,3.274)}
\gppoint{gp mark 1}{(8.080,2.454)}
\gppoint{gp mark 1}{(8.082,3.083)}
\gppoint{gp mark 1}{(8.085,2.919)}
\gppoint{gp mark 1}{(8.087,2.247)}
\gppoint{gp mark 1}{(8.090,2.897)}
\gppoint{gp mark 1}{(8.092,2.777)}
\gppoint{gp mark 1}{(8.095,3.072)}
\gppoint{gp mark 1}{(8.097,2.891)}
\gppoint{gp mark 1}{(8.100,2.405)}
\gppoint{gp mark 1}{(8.102,2.312)}
\gppoint{gp mark 1}{(8.105,2.820)}
\gppoint{gp mark 1}{(8.107,3.121)}
\gppoint{gp mark 1}{(8.109,2.525)}
\gppoint{gp mark 1}{(8.112,2.602)}
\gppoint{gp mark 1}{(8.114,2.678)}
\gppoint{gp mark 1}{(8.117,2.908)}
\gppoint{gp mark 1}{(8.119,3.055)}
\gppoint{gp mark 1}{(8.122,3.083)}
\gppoint{gp mark 1}{(8.124,3.285)}
\gppoint{gp mark 1}{(8.127,3.088)}
\gppoint{gp mark 1}{(8.129,3.602)}
\gppoint{gp mark 1}{(8.132,3.279)}
\gppoint{gp mark 1}{(8.134,3.334)}
\gppoint{gp mark 1}{(8.137,3.214)}
\gppoint{gp mark 1}{(8.139,2.744)}
\gppoint{gp mark 1}{(8.142,2.962)}
\gppoint{gp mark 1}{(8.144,2.575)}
\gppoint{gp mark 1}{(8.147,2.995)}
\gppoint{gp mark 1}{(8.149,2.995)}
\gppoint{gp mark 1}{(8.152,2.788)}
\gppoint{gp mark 1}{(8.154,3.028)}
\gppoint{gp mark 1}{(8.157,2.744)}
\gppoint{gp mark 1}{(8.159,2.542)}
\gppoint{gp mark 1}{(8.162,2.990)}
\gppoint{gp mark 1}{(8.164,2.957)}
\gppoint{gp mark 1}{(8.167,2.771)}
\gppoint{gp mark 1}{(8.169,2.482)}
\gppoint{gp mark 1}{(8.172,2.864)}
\gppoint{gp mark 1}{(8.174,3.236)}
\gppoint{gp mark 1}{(8.177,2.689)}
\gppoint{gp mark 1}{(8.179,2.646)}
\gppoint{gp mark 1}{(8.182,2.717)}
\gppoint{gp mark 1}{(8.184,3.083)}
\gppoint{gp mark 1}{(8.187,3.826)}
\gppoint{gp mark 1}{(8.189,2.553)}
\gppoint{gp mark 1}{(8.191,3.318)}
\gppoint{gp mark 1}{(8.194,3.159)}
\gppoint{gp mark 1}{(8.196,2.373)}
\gppoint{gp mark 1}{(8.199,2.323)}
\gppoint{gp mark 1}{(8.201,2.984)}
\gppoint{gp mark 1}{(8.204,3.066)}
\gppoint{gp mark 1}{(8.206,3.225)}
\gppoint{gp mark 1}{(8.209,3.143)}
\gppoint{gp mark 1}{(8.211,2.995)}
\gppoint{gp mark 1}{(8.214,2.919)}
\gppoint{gp mark 1}{(8.216,3.318)}
\gppoint{gp mark 1}{(8.219,2.564)}
\gppoint{gp mark 1}{(8.221,3.181)}
\gppoint{gp mark 1}{(8.224,2.618)}
\gppoint{gp mark 1}{(8.226,3.471)}
\gppoint{gp mark 1}{(8.229,3.094)}
\gppoint{gp mark 1}{(8.231,2.673)}
\gppoint{gp mark 1}{(8.234,2.760)}
\gppoint{gp mark 1}{(8.236,3.427)}
\gppoint{gp mark 1}{(8.239,3.481)}
\gppoint{gp mark 1}{(8.241,3.159)}
\gppoint{gp mark 1}{(8.244,2.607)}
\gppoint{gp mark 1}{(8.246,2.531)}
\gppoint{gp mark 1}{(8.249,2.837)}
\gppoint{gp mark 1}{(8.251,3.230)}
\gppoint{gp mark 1}{(8.254,3.077)}
\gppoint{gp mark 1}{(8.256,2.897)}
\gppoint{gp mark 1}{(8.259,3.268)}
\gppoint{gp mark 1}{(8.261,2.515)}
\gppoint{gp mark 1}{(8.264,2.815)}
\gppoint{gp mark 1}{(8.266,2.853)}
\gppoint{gp mark 1}{(8.269,3.072)}
\gppoint{gp mark 1}{(8.271,2.831)}
\gppoint{gp mark 1}{(8.273,3.061)}
\gppoint{gp mark 1}{(8.276,2.531)}
\gppoint{gp mark 1}{(8.278,2.509)}
\gppoint{gp mark 1}{(8.281,2.192)}
\gppoint{gp mark 1}{(8.283,2.383)}
\gppoint{gp mark 1}{(8.286,2.810)}
\gppoint{gp mark 1}{(8.288,2.788)}
\gppoint{gp mark 1}{(8.291,3.339)}
\gppoint{gp mark 1}{(8.293,2.607)}
\gppoint{gp mark 1}{(8.296,3.061)}
\gppoint{gp mark 1}{(8.298,3.339)}
\gppoint{gp mark 1}{(8.301,2.389)}
\gppoint{gp mark 1}{(8.303,2.826)}
\gppoint{gp mark 1}{(8.306,3.197)}
\gppoint{gp mark 1}{(8.308,2.427)}
\gppoint{gp mark 1}{(8.311,2.318)}
\gppoint{gp mark 1}{(8.313,3.083)}
\gppoint{gp mark 1}{(8.316,2.952)}
\gppoint{gp mark 1}{(8.318,3.312)}
\gppoint{gp mark 1}{(8.321,3.569)}
\gppoint{gp mark 1}{(8.323,2.252)}
\gppoint{gp mark 1}{(8.326,2.695)}
\gppoint{gp mark 1}{(8.328,2.842)}
\gppoint{gp mark 1}{(8.331,2.820)}
\gppoint{gp mark 1}{(8.333,3.126)}
\gppoint{gp mark 1}{(8.336,2.902)}
\gppoint{gp mark 1}{(8.338,3.503)}
\gppoint{gp mark 1}{(8.341,3.427)}
\gppoint{gp mark 1}{(8.343,3.110)}
\gppoint{gp mark 1}{(8.346,3.099)}
\gppoint{gp mark 1}{(8.348,3.072)}
\gppoint{gp mark 1}{(8.351,3.214)}
\gppoint{gp mark 1}{(8.353,2.804)}
\gppoint{gp mark 1}{(8.355,2.962)}
\gppoint{gp mark 1}{(8.358,3.083)}
\gppoint{gp mark 1}{(8.360,2.941)}
\gppoint{gp mark 1}{(8.363,3.006)}
\gppoint{gp mark 1}{(8.365,3.099)}
\gppoint{gp mark 1}{(8.368,2.853)}
\gppoint{gp mark 1}{(8.370,3.083)}
\gppoint{gp mark 1}{(8.373,2.962)}
\gppoint{gp mark 1}{(8.375,2.886)}
\gppoint{gp mark 1}{(8.378,2.995)}
\gppoint{gp mark 1}{(8.380,2.908)}
\gppoint{gp mark 1}{(8.383,3.023)}
\gppoint{gp mark 1}{(8.385,2.678)}
\gppoint{gp mark 1}{(8.388,3.176)}
\gppoint{gp mark 1}{(8.390,3.257)}
\gppoint{gp mark 1}{(8.393,2.820)}
\gppoint{gp mark 1}{(8.395,2.913)}
\gppoint{gp mark 1}{(8.398,2.984)}
\gppoint{gp mark 1}{(8.400,2.673)}
\gppoint{gp mark 1}{(8.403,3.028)}
\gppoint{gp mark 1}{(8.405,2.668)}
\gppoint{gp mark 1}{(8.408,2.744)}
\gppoint{gp mark 1}{(8.410,2.695)}
\gppoint{gp mark 1}{(8.413,3.421)}
\gppoint{gp mark 1}{(8.415,3.214)}
\gppoint{gp mark 1}{(8.418,3.236)}
\gppoint{gp mark 1}{(8.420,3.492)}
\gppoint{gp mark 1}{(8.423,3.028)}
\gppoint{gp mark 1}{(8.425,3.394)}
\gppoint{gp mark 1}{(8.428,3.023)}
\gppoint{gp mark 1}{(8.430,3.274)}
\gppoint{gp mark 1}{(8.433,3.208)}
\gppoint{gp mark 1}{(8.435,2.383)}
\gppoint{gp mark 1}{(8.438,3.099)}
\gppoint{gp mark 1}{(8.440,2.842)}
\gppoint{gp mark 1}{(8.442,2.766)}
\gppoint{gp mark 1}{(8.445,2.487)}
\gppoint{gp mark 1}{(8.447,3.186)}
\gppoint{gp mark 1}{(8.450,2.274)}
\gppoint{gp mark 1}{(8.452,2.935)}
\gppoint{gp mark 1}{(8.455,2.826)}
\gppoint{gp mark 1}{(8.457,2.842)}
\gppoint{gp mark 1}{(8.460,2.793)}
\gppoint{gp mark 1}{(8.462,3.061)}
\gppoint{gp mark 1}{(8.465,2.771)}
\gppoint{gp mark 1}{(8.467,2.629)}
\gppoint{gp mark 1}{(8.470,2.837)}
\gppoint{gp mark 1}{(8.472,2.509)}
\gppoint{gp mark 1}{(8.475,3.460)}
\gppoint{gp mark 1}{(8.477,2.312)}
\gppoint{gp mark 1}{(8.480,3.400)}
\gppoint{gp mark 1}{(8.482,2.744)}
\gppoint{gp mark 1}{(8.485,3.110)}
\gppoint{gp mark 1}{(8.487,2.755)}
\gppoint{gp mark 1}{(8.490,2.968)}
\gppoint{gp mark 1}{(8.492,2.460)}
\gppoint{gp mark 1}{(8.495,3.443)}
\gppoint{gp mark 1}{(8.497,2.897)}
\gppoint{gp mark 1}{(8.500,3.487)}
\gppoint{gp mark 1}{(8.502,3.602)}
\gppoint{gp mark 1}{(8.505,2.739)}
\gppoint{gp mark 1}{(8.507,2.629)}
\gppoint{gp mark 1}{(8.510,2.831)}
\gppoint{gp mark 1}{(8.512,3.176)}
\gppoint{gp mark 1}{(8.515,2.602)}
\gppoint{gp mark 1}{(8.517,2.061)}
\gppoint{gp mark 1}{(8.520,2.733)}
\gppoint{gp mark 1}{(8.522,2.449)}
\gppoint{gp mark 1}{(8.524,3.290)}
\gppoint{gp mark 1}{(8.527,3.569)}
\gppoint{gp mark 1}{(8.529,3.034)}
\gppoint{gp mark 1}{(8.532,3.219)}
\gppoint{gp mark 1}{(8.534,2.777)}
\gppoint{gp mark 1}{(8.537,2.957)}
\gppoint{gp mark 1}{(8.539,2.820)}
\gppoint{gp mark 1}{(8.542,3.268)}
\gppoint{gp mark 1}{(8.544,1.990)}
\gppoint{gp mark 1}{(8.547,4.055)}
\gppoint{gp mark 1}{(8.549,3.121)}
\gppoint{gp mark 1}{(8.552,2.438)}
\gppoint{gp mark 1}{(8.554,2.853)}
\gppoint{gp mark 1}{(8.557,3.285)}
\gppoint{gp mark 1}{(8.559,3.241)}
\gppoint{gp mark 1}{(8.562,2.646)}
\gppoint{gp mark 1}{(8.564,2.586)}
\gppoint{gp mark 1}{(8.567,2.149)}
\gppoint{gp mark 1}{(8.569,2.471)}
\gppoint{gp mark 1}{(8.572,2.668)}
\gppoint{gp mark 1}{(8.574,2.444)}
\gppoint{gp mark 1}{(8.577,2.826)}
\gppoint{gp mark 1}{(8.579,2.651)}
\gppoint{gp mark 1}{(8.582,2.596)}
\gppoint{gp mark 1}{(8.584,3.094)}
\gppoint{gp mark 1}{(8.587,3.055)}
\gppoint{gp mark 1}{(8.589,3.105)}
\gppoint{gp mark 1}{(8.592,2.782)}
\gppoint{gp mark 1}{(8.594,2.394)}
\gppoint{gp mark 1}{(8.597,2.400)}
\gppoint{gp mark 1}{(8.599,3.099)}
\gppoint{gp mark 1}{(8.602,2.946)}
\gppoint{gp mark 1}{(8.604,2.427)}
\gppoint{gp mark 1}{(8.606,2.962)}
\gppoint{gp mark 1}{(8.609,2.149)}
\gppoint{gp mark 1}{(8.611,2.973)}
\gppoint{gp mark 1}{(8.614,3.039)}
\gppoint{gp mark 1}{(8.616,2.826)}
\gppoint{gp mark 1}{(8.619,2.711)}
\gppoint{gp mark 1}{(8.621,2.924)}
\gppoint{gp mark 1}{(8.624,2.591)}
\gppoint{gp mark 1}{(8.626,2.252)}
\gppoint{gp mark 1}{(8.629,2.220)}
\gppoint{gp mark 1}{(8.631,2.192)}
\gppoint{gp mark 1}{(8.634,2.924)}
\gppoint{gp mark 1}{(8.636,2.815)}
\gppoint{gp mark 1}{(8.639,3.208)}
\gppoint{gp mark 1}{(8.641,2.859)}
\gppoint{gp mark 1}{(8.644,2.853)}
\gppoint{gp mark 1}{(8.646,3.247)}
\gppoint{gp mark 1}{(8.649,2.493)}
\gppoint{gp mark 1}{(8.651,2.427)}
\gppoint{gp mark 1}{(8.654,2.782)}
\gppoint{gp mark 1}{(8.656,2.067)}
\gppoint{gp mark 1}{(8.659,2.973)}
\gppoint{gp mark 1}{(8.661,1.837)}
\gppoint{gp mark 1}{(8.664,2.957)}
\gppoint{gp mark 1}{(8.666,3.088)}
\gppoint{gp mark 1}{(8.669,3.301)}
\gppoint{gp mark 1}{(8.671,2.766)}
\gppoint{gp mark 1}{(8.674,3.694)}
\gppoint{gp mark 1}{(8.676,3.684)}
\gppoint{gp mark 1}{(8.679,2.149)}
\gppoint{gp mark 1}{(8.681,2.962)}
\gppoint{gp mark 1}{(8.684,3.263)}
\gppoint{gp mark 1}{(8.686,2.678)}
\gppoint{gp mark 1}{(8.689,2.591)}
\gppoint{gp mark 1}{(8.691,2.766)}
\gppoint{gp mark 1}{(8.693,2.749)}
\gppoint{gp mark 1}{(8.696,3.088)}
\gppoint{gp mark 1}{(8.698,2.520)}
\gppoint{gp mark 1}{(8.701,2.886)}
\gppoint{gp mark 1}{(8.703,2.952)}
\gppoint{gp mark 1}{(8.706,2.383)}
\gppoint{gp mark 1}{(8.708,2.460)}
\gppoint{gp mark 1}{(8.711,2.935)}
\gppoint{gp mark 1}{(8.713,2.444)}
\gppoint{gp mark 1}{(8.716,2.602)}
\gppoint{gp mark 1}{(8.718,2.460)}
\gppoint{gp mark 1}{(8.721,2.542)}
\gppoint{gp mark 1}{(8.723,2.334)}
\gppoint{gp mark 1}{(8.726,3.115)}
\gppoint{gp mark 1}{(8.728,2.684)}
\gppoint{gp mark 1}{(8.731,2.760)}
\gppoint{gp mark 1}{(8.733,2.941)}
\gppoint{gp mark 1}{(8.736,3.034)}
\gppoint{gp mark 1}{(8.738,2.400)}
\gppoint{gp mark 1}{(8.741,3.170)}
\gppoint{gp mark 1}{(8.743,3.023)}
\gppoint{gp mark 1}{(8.746,2.629)}
\gppoint{gp mark 1}{(8.748,2.291)}
\gppoint{gp mark 1}{(8.751,2.203)}
\gppoint{gp mark 1}{(8.753,2.728)}
\gppoint{gp mark 1}{(8.756,3.012)}
\gppoint{gp mark 1}{(8.758,2.700)}
\gppoint{gp mark 1}{(8.761,2.646)}
\gppoint{gp mark 1}{(8.763,3.203)}
\gppoint{gp mark 1}{(8.766,2.586)}
\gppoint{gp mark 1}{(8.768,3.263)}
\gppoint{gp mark 1}{(8.771,2.695)}
\gppoint{gp mark 1}{(8.773,2.651)}
\gppoint{gp mark 1}{(8.775,2.575)}
\gppoint{gp mark 1}{(8.778,2.575)}
\gppoint{gp mark 1}{(8.780,2.575)}
\gppoint{gp mark 1}{(8.783,2.476)}
\gppoint{gp mark 1}{(8.785,2.799)}
\gppoint{gp mark 1}{(8.788,3.367)}
\gppoint{gp mark 1}{(8.790,2.596)}
\gppoint{gp mark 1}{(8.793,3.176)}
\gppoint{gp mark 1}{(8.795,2.547)}
\gppoint{gp mark 1}{(8.798,2.804)}
\gppoint{gp mark 1}{(8.800,2.602)}
\gppoint{gp mark 1}{(8.803,3.148)}
\gppoint{gp mark 1}{(8.805,2.471)}
\gppoint{gp mark 1}{(8.808,2.678)}
\gppoint{gp mark 1}{(8.810,3.487)}
\gppoint{gp mark 1}{(8.813,3.356)}
\gppoint{gp mark 1}{(8.815,2.799)}
\gppoint{gp mark 1}{(8.818,2.345)}
\gppoint{gp mark 1}{(8.820,1.957)}
\gppoint{gp mark 1}{(8.823,2.689)}
\gppoint{gp mark 1}{(8.825,3.503)}
\gppoint{gp mark 1}{(8.828,2.203)}
\gppoint{gp mark 1}{(8.830,3.072)}
\gppoint{gp mark 1}{(8.833,2.318)}
\gppoint{gp mark 1}{(8.835,2.799)}
\gppoint{gp mark 1}{(8.838,2.613)}
\gppoint{gp mark 1}{(8.840,3.181)}
\gppoint{gp mark 1}{(8.843,3.531)}
\gppoint{gp mark 1}{(8.845,1.662)}
\gppoint{gp mark 1}{(8.848,3.143)}
\gppoint{gp mark 1}{(8.850,2.706)}
\gppoint{gp mark 1}{(8.853,2.919)}
\gppoint{gp mark 1}{(8.855,2.728)}
\gppoint{gp mark 1}{(8.857,2.700)}
\gppoint{gp mark 1}{(8.860,2.700)}
\gppoint{gp mark 1}{(8.862,3.012)}
\gppoint{gp mark 1}{(8.865,2.755)}
\gppoint{gp mark 1}{(8.867,2.815)}
\gppoint{gp mark 1}{(8.870,3.252)}
\gppoint{gp mark 1}{(8.872,3.077)}
\gppoint{gp mark 1}{(8.875,3.268)}
\gppoint{gp mark 1}{(8.877,2.804)}
\gppoint{gp mark 1}{(8.880,3.023)}
\gppoint{gp mark 1}{(8.882,2.968)}
\gppoint{gp mark 1}{(8.885,3.050)}
\gppoint{gp mark 1}{(8.887,3.498)}
\gppoint{gp mark 1}{(8.890,3.034)}
\gppoint{gp mark 1}{(8.892,1.821)}
\gppoint{gp mark 1}{(8.895,3.099)}
\gppoint{gp mark 1}{(8.897,3.410)}
\gppoint{gp mark 1}{(8.900,2.476)}
\gppoint{gp mark 1}{(8.902,2.646)}
\gppoint{gp mark 1}{(8.905,2.378)}
\gppoint{gp mark 1}{(8.907,2.094)}
\gppoint{gp mark 1}{(8.910,2.886)}
\gppoint{gp mark 1}{(8.912,2.039)}
\gppoint{gp mark 1}{(8.915,3.126)}
\gppoint{gp mark 1}{(8.917,3.208)}
\gppoint{gp mark 1}{(8.920,2.717)}
\gppoint{gp mark 1}{(8.922,3.083)}
\gppoint{gp mark 1}{(8.925,3.618)}
\gppoint{gp mark 1}{(8.927,2.897)}
\gppoint{gp mark 1}{(8.930,2.831)}
\gppoint{gp mark 1}{(8.932,2.804)}
\gppoint{gp mark 1}{(8.935,2.635)}
\gppoint{gp mark 1}{(8.937,2.181)}
\gppoint{gp mark 1}{(8.940,3.181)}
\gppoint{gp mark 1}{(8.942,3.088)}
\gppoint{gp mark 1}{(8.944,2.034)}
\gppoint{gp mark 1}{(8.947,2.536)}
\gppoint{gp mark 1}{(8.949,2.924)}
\gppoint{gp mark 1}{(8.952,2.897)}
\gppoint{gp mark 1}{(8.954,2.580)}
\gppoint{gp mark 1}{(8.957,2.449)}
\gppoint{gp mark 1}{(8.959,2.362)}
\gppoint{gp mark 1}{(8.962,2.995)}
\gppoint{gp mark 1}{(8.964,2.881)}
\gppoint{gp mark 1}{(8.967,3.143)}
\gppoint{gp mark 1}{(8.969,2.498)}
\gppoint{gp mark 1}{(8.972,3.367)}
\gppoint{gp mark 1}{(8.974,2.454)}
\gppoint{gp mark 1}{(8.977,2.930)}
\gppoint{gp mark 1}{(8.979,3.126)}
\gppoint{gp mark 1}{(8.982,3.203)}
\gppoint{gp mark 1}{(8.984,2.957)}
\gppoint{gp mark 1}{(8.987,2.924)}
\gppoint{gp mark 1}{(8.989,3.312)}
\gppoint{gp mark 1}{(8.992,3.323)}
\gppoint{gp mark 1}{(8.994,2.952)}
\gppoint{gp mark 1}{(8.997,3.208)}
\gppoint{gp mark 1}{(8.999,2.138)}
\gppoint{gp mark 1}{(9.002,3.268)}
\gppoint{gp mark 1}{(9.004,2.460)}
\gppoint{gp mark 1}{(9.007,2.547)}
\gppoint{gp mark 1}{(9.009,2.810)}
\gppoint{gp mark 1}{(9.012,3.137)}
\gppoint{gp mark 1}{(9.014,2.515)}
\gppoint{gp mark 1}{(9.017,2.924)}
\gppoint{gp mark 1}{(9.019,2.334)}
\gppoint{gp mark 1}{(9.022,2.837)}
\gppoint{gp mark 1}{(9.024,2.427)}
\gppoint{gp mark 1}{(9.026,2.181)}
\gppoint{gp mark 1}{(9.029,2.684)}
\gppoint{gp mark 1}{(9.031,2.542)}
\gppoint{gp mark 1}{(9.034,3.034)}
\gppoint{gp mark 1}{(9.036,2.596)}
\gppoint{gp mark 1}{(9.039,2.853)}
\gppoint{gp mark 1}{(9.041,2.766)}
\gppoint{gp mark 1}{(9.044,2.859)}
\gppoint{gp mark 1}{(9.046,2.826)}
\gppoint{gp mark 1}{(9.049,2.804)}
\gppoint{gp mark 1}{(9.051,3.312)}
\gppoint{gp mark 1}{(9.054,2.034)}
\gppoint{gp mark 1}{(9.056,3.176)}
\gppoint{gp mark 1}{(9.059,2.045)}
\gppoint{gp mark 1}{(9.061,3.225)}
\gppoint{gp mark 1}{(9.064,3.039)}
\gppoint{gp mark 1}{(9.066,2.935)}
\gppoint{gp mark 1}{(9.069,2.864)}
\gppoint{gp mark 1}{(9.071,2.105)}
\gppoint{gp mark 1}{(9.074,2.875)}
\gppoint{gp mark 1}{(9.076,3.055)}
\gppoint{gp mark 1}{(9.079,3.006)}
\gppoint{gp mark 1}{(9.081,2.700)}
\gppoint{gp mark 1}{(9.084,3.176)}
\gppoint{gp mark 1}{(9.086,2.640)}
\gppoint{gp mark 1}{(9.089,2.476)}
\gppoint{gp mark 1}{(9.091,2.678)}
\gppoint{gp mark 1}{(9.094,3.378)}
\gppoint{gp mark 1}{(9.096,2.482)}
\gppoint{gp mark 1}{(9.099,2.848)}
\gppoint{gp mark 1}{(9.101,2.722)}
\gppoint{gp mark 1}{(9.104,3.115)}
\gppoint{gp mark 1}{(9.106,2.471)}
\gppoint{gp mark 1}{(9.108,2.924)}
\gppoint{gp mark 1}{(9.111,2.935)}
\gppoint{gp mark 1}{(9.113,2.711)}
\gppoint{gp mark 1}{(9.116,3.285)}
\gppoint{gp mark 1}{(9.118,2.444)}
\gppoint{gp mark 1}{(9.121,2.471)}
\gppoint{gp mark 1}{(9.123,3.099)}
\gppoint{gp mark 1}{(9.126,2.045)}
\gppoint{gp mark 1}{(9.128,2.946)}
\gppoint{gp mark 1}{(9.131,2.897)}
\gppoint{gp mark 1}{(9.133,2.657)}
\gppoint{gp mark 1}{(9.136,3.563)}
\gppoint{gp mark 1}{(9.138,3.547)}
\gppoint{gp mark 1}{(9.141,2.444)}
\gppoint{gp mark 1}{(9.143,2.520)}
\gppoint{gp mark 1}{(9.146,2.962)}
\gppoint{gp mark 1}{(9.148,2.192)}
\gppoint{gp mark 1}{(9.151,2.105)}
\gppoint{gp mark 1}{(9.153,3.176)}
\gppoint{gp mark 1}{(9.156,2.902)}
\gppoint{gp mark 1}{(9.158,2.525)}
\gppoint{gp mark 1}{(9.161,2.662)}
\gppoint{gp mark 1}{(9.163,2.525)}
\gppoint{gp mark 1}{(9.166,3.143)}
\gppoint{gp mark 1}{(9.168,2.618)}
\gppoint{gp mark 1}{(9.171,2.924)}
\gppoint{gp mark 1}{(9.173,2.383)}
\gppoint{gp mark 1}{(9.176,2.515)}
\gppoint{gp mark 1}{(9.178,2.088)}
\gppoint{gp mark 1}{(9.181,2.427)}
\gppoint{gp mark 1}{(9.183,3.017)}
\gppoint{gp mark 1}{(9.186,2.777)}
\gppoint{gp mark 1}{(9.188,2.984)}
\gppoint{gp mark 1}{(9.191,2.913)}
\gppoint{gp mark 1}{(9.193,3.077)}
\gppoint{gp mark 1}{(9.195,2.340)}
\gppoint{gp mark 1}{(9.198,2.831)}
\gppoint{gp mark 1}{(9.200,2.515)}
\gppoint{gp mark 1}{(9.203,2.717)}
\gppoint{gp mark 1}{(9.205,2.728)}
\gppoint{gp mark 1}{(9.208,2.810)}
\gppoint{gp mark 1}{(9.210,2.640)}
\gppoint{gp mark 1}{(9.213,2.515)}
\gppoint{gp mark 1}{(9.215,2.853)}
\gppoint{gp mark 1}{(9.218,2.935)}
\gppoint{gp mark 1}{(9.220,2.487)}
\gppoint{gp mark 1}{(9.223,3.296)}
\gppoint{gp mark 1}{(9.225,2.859)}
\gppoint{gp mark 1}{(9.228,2.728)}
\gppoint{gp mark 1}{(9.230,1.859)}
\gppoint{gp mark 1}{(9.233,3.099)}
\gppoint{gp mark 1}{(9.235,2.263)}
\gppoint{gp mark 1}{(9.238,2.771)}
\gppoint{gp mark 1}{(9.240,2.820)}
\gppoint{gp mark 1}{(9.243,2.771)}
\gppoint{gp mark 1}{(9.245,2.640)}
\gppoint{gp mark 1}{(9.248,2.586)}
\gppoint{gp mark 1}{(9.250,2.504)}
\gppoint{gp mark 1}{(9.253,2.689)}
\gppoint{gp mark 1}{(9.255,2.564)}
\gppoint{gp mark 1}{(9.258,2.487)}
\gppoint{gp mark 1}{(9.260,3.039)}
\gppoint{gp mark 1}{(9.263,2.678)}
\gppoint{gp mark 1}{(9.265,2.930)}
\gppoint{gp mark 1}{(9.268,2.596)}
\gppoint{gp mark 1}{(9.270,2.766)}
\gppoint{gp mark 1}{(9.273,2.624)}
\gppoint{gp mark 1}{(9.275,2.296)}
\gppoint{gp mark 1}{(9.277,2.979)}
\gppoint{gp mark 1}{(9.280,2.886)}
\gppoint{gp mark 1}{(9.282,1.777)}
\gppoint{gp mark 1}{(9.285,3.503)}
\gppoint{gp mark 1}{(9.287,2.613)}
\gppoint{gp mark 1}{(9.290,2.345)}
\gppoint{gp mark 1}{(9.292,2.515)}
\gppoint{gp mark 1}{(9.295,2.476)}
\gppoint{gp mark 1}{(9.297,2.220)}
\gppoint{gp mark 1}{(9.300,2.302)}
\gppoint{gp mark 1}{(9.302,2.875)}
\gppoint{gp mark 1}{(9.305,2.247)}
\gppoint{gp mark 1}{(9.307,2.170)}
\gppoint{gp mark 1}{(9.310,2.607)}
\gppoint{gp mark 1}{(9.312,2.782)}
\gppoint{gp mark 1}{(9.315,2.482)}
\gppoint{gp mark 1}{(9.317,2.515)}
\gppoint{gp mark 1}{(9.320,2.973)}
\gppoint{gp mark 1}{(9.322,2.291)}
\gppoint{gp mark 1}{(9.325,1.974)}
\gppoint{gp mark 1}{(9.327,2.454)}
\gppoint{gp mark 1}{(9.330,2.558)}
\gppoint{gp mark 1}{(9.332,2.340)}
\gppoint{gp mark 1}{(9.335,2.198)}
\gppoint{gp mark 1}{(9.337,2.591)}
\gppoint{gp mark 1}{(9.340,2.329)}
\gppoint{gp mark 1}{(9.342,2.908)}
\gppoint{gp mark 1}{(9.345,2.875)}
\gppoint{gp mark 1}{(9.347,2.400)}
\gppoint{gp mark 1}{(9.350,2.132)}
\gppoint{gp mark 1}{(9.352,2.760)}
\gppoint{gp mark 1}{(9.355,2.651)}
\gppoint{gp mark 1}{(9.357,2.919)}
\gppoint{gp mark 1}{(9.359,2.203)}
\gppoint{gp mark 1}{(9.362,2.984)}
\gppoint{gp mark 1}{(9.364,2.990)}
\gppoint{gp mark 1}{(9.367,2.815)}
\gppoint{gp mark 1}{(9.369,2.706)}
\gppoint{gp mark 1}{(9.372,2.919)}
\gppoint{gp mark 1}{(9.374,2.225)}
\gppoint{gp mark 1}{(9.377,2.848)}
\gppoint{gp mark 1}{(9.379,2.318)}
\gppoint{gp mark 1}{(9.382,3.088)}
\gppoint{gp mark 1}{(9.384,2.159)}
\gppoint{gp mark 1}{(9.387,2.711)}
\gppoint{gp mark 1}{(9.389,2.078)}
\gppoint{gp mark 1}{(9.392,2.689)}
\gppoint{gp mark 1}{(9.394,1.400)}
\gppoint{gp mark 1}{(9.397,2.498)}
\gppoint{gp mark 1}{(9.399,2.373)}
\gppoint{gp mark 1}{(9.402,2.471)}
\gppoint{gp mark 1}{(9.404,2.476)}
\gppoint{gp mark 1}{(9.407,2.154)}
\gppoint{gp mark 1}{(9.409,2.662)}
\gppoint{gp mark 1}{(9.412,3.197)}
\gppoint{gp mark 1}{(9.414,2.509)}
\gppoint{gp mark 1}{(9.417,2.629)}
\gppoint{gp mark 1}{(9.419,3.645)}
\gppoint{gp mark 1}{(9.422,3.214)}
\gppoint{gp mark 1}{(9.424,2.804)}
\gppoint{gp mark 1}{(9.427,2.624)}
\gppoint{gp mark 1}{(9.429,2.930)}
\gppoint{gp mark 1}{(9.432,2.471)}
\gppoint{gp mark 1}{(9.434,2.908)}
\gppoint{gp mark 1}{(9.437,3.378)}
\gppoint{gp mark 1}{(9.439,2.837)}
\gppoint{gp mark 1}{(9.442,3.296)}
\gppoint{gp mark 1}{(9.444,2.886)}
\gppoint{gp mark 1}{(9.446,2.788)}
\gppoint{gp mark 1}{(9.449,2.799)}
\gppoint{gp mark 1}{(9.451,2.837)}
\gppoint{gp mark 1}{(9.454,2.853)}
\gppoint{gp mark 1}{(9.456,3.651)}
\gppoint{gp mark 1}{(9.459,3.132)}
\gppoint{gp mark 1}{(9.461,3.039)}
\gppoint{gp mark 1}{(9.464,2.493)}
\gppoint{gp mark 1}{(9.466,2.302)}
\gppoint{gp mark 1}{(9.469,2.848)}
\gppoint{gp mark 1}{(9.471,2.356)}
\gppoint{gp mark 1}{(9.474,3.334)}
\gppoint{gp mark 1}{(9.476,1.897)}
\gppoint{gp mark 1}{(9.479,2.334)}
\gppoint{gp mark 1}{(9.481,2.225)}
\gppoint{gp mark 1}{(9.484,2.367)}
\gppoint{gp mark 1}{(9.486,2.334)}
\gppoint{gp mark 1}{(9.489,2.689)}
\gppoint{gp mark 1}{(9.491,2.760)}
\gppoint{gp mark 1}{(9.494,3.252)}
\gppoint{gp mark 1}{(9.496,2.132)}
\gppoint{gp mark 1}{(9.499,2.596)}
\gppoint{gp mark 1}{(9.501,2.422)}
\gppoint{gp mark 1}{(9.504,1.854)}
\gppoint{gp mark 1}{(9.506,2.717)}
\gppoint{gp mark 1}{(9.509,3.066)}
\gppoint{gp mark 1}{(9.511,3.350)}
\gppoint{gp mark 1}{(9.514,2.673)}
\gppoint{gp mark 1}{(9.516,2.099)}
\gppoint{gp mark 1}{(9.519,2.848)}
\gppoint{gp mark 1}{(9.521,2.580)}
\gppoint{gp mark 1}{(9.524,2.952)}
\gppoint{gp mark 1}{(9.526,2.132)}
\gppoint{gp mark 1}{(9.528,3.225)}
\gppoint{gp mark 1}{(9.531,1.936)}
\gppoint{gp mark 1}{(9.533,2.056)}
\gppoint{gp mark 1}{(9.536,3.290)}
\gppoint{gp mark 1}{(9.538,2.138)}
\gppoint{gp mark 1}{(9.541,2.553)}
\gppoint{gp mark 1}{(9.543,3.274)}
\gppoint{gp mark 1}{(9.546,2.449)}
\gppoint{gp mark 1}{(9.548,2.525)}
\gppoint{gp mark 1}{(9.551,2.646)}
\gppoint{gp mark 1}{(9.553,2.810)}
\gppoint{gp mark 1}{(9.556,2.454)}
\gppoint{gp mark 1}{(9.558,3.110)}
\gppoint{gp mark 1}{(9.561,2.695)}
\gppoint{gp mark 1}{(9.563,2.968)}
\gppoint{gp mark 1}{(9.566,2.717)}
\gppoint{gp mark 1}{(9.568,2.864)}
\gppoint{gp mark 1}{(9.571,2.067)}
\gppoint{gp mark 1}{(9.573,2.312)}
\gppoint{gp mark 1}{(9.576,2.433)}
\gppoint{gp mark 1}{(9.578,2.848)}
\gppoint{gp mark 1}{(9.581,2.383)}
\gppoint{gp mark 1}{(9.583,2.646)}
\gppoint{gp mark 1}{(9.586,2.875)}
\gppoint{gp mark 1}{(9.588,3.154)}
\gppoint{gp mark 1}{(9.591,2.263)}
\gppoint{gp mark 1}{(9.593,2.777)}
\gppoint{gp mark 1}{(9.596,2.542)}
\gppoint{gp mark 1}{(9.598,2.564)}
\gppoint{gp mark 1}{(9.601,2.684)}
\gppoint{gp mark 1}{(9.603,2.460)}
\gppoint{gp mark 1}{(9.606,3.263)}
\gppoint{gp mark 1}{(9.608,2.875)}
\gppoint{gp mark 1}{(9.610,3.410)}
\gppoint{gp mark 1}{(9.613,2.023)}
\gppoint{gp mark 1}{(9.615,2.596)}
\gppoint{gp mark 1}{(9.618,2.700)}
\gppoint{gp mark 1}{(9.620,2.913)}
\gppoint{gp mark 1}{(9.623,2.613)}
\gppoint{gp mark 1}{(9.625,2.312)}
\gppoint{gp mark 1}{(9.628,2.722)}
\gppoint{gp mark 1}{(9.630,2.034)}
\gppoint{gp mark 1}{(9.633,2.706)}
\gppoint{gp mark 1}{(9.635,2.941)}
\gppoint{gp mark 1}{(9.638,2.739)}
\gppoint{gp mark 1}{(9.640,3.345)}
\gppoint{gp mark 1}{(9.643,3.077)}
\gppoint{gp mark 1}{(9.645,2.438)}
\gppoint{gp mark 1}{(9.648,2.340)}
\gppoint{gp mark 1}{(9.650,3.296)}
\gppoint{gp mark 1}{(9.653,3.186)}
\gppoint{gp mark 1}{(9.655,2.771)}
\gppoint{gp mark 1}{(9.658,2.536)}
\gppoint{gp mark 1}{(9.660,2.154)}
\gppoint{gp mark 1}{(9.663,2.722)}
\gppoint{gp mark 1}{(9.665,2.607)}
\gppoint{gp mark 1}{(9.668,2.848)}
\gppoint{gp mark 1}{(9.670,2.711)}
\gppoint{gp mark 1}{(9.673,3.044)}
\gppoint{gp mark 1}{(9.675,3.165)}
\gppoint{gp mark 1}{(9.678,2.935)}
\gppoint{gp mark 1}{(9.680,2.400)}
\gppoint{gp mark 1}{(9.683,2.591)}
\gppoint{gp mark 1}{(9.685,2.662)}
\gppoint{gp mark 1}{(9.688,2.493)}
\gppoint{gp mark 1}{(9.690,2.323)}
\gppoint{gp mark 1}{(9.692,2.176)}
\gppoint{gp mark 1}{(9.695,2.542)}
\gppoint{gp mark 1}{(9.697,2.668)}
\gppoint{gp mark 1}{(9.700,2.848)}
\gppoint{gp mark 1}{(9.702,2.766)}
\gppoint{gp mark 1}{(9.705,2.831)}
\gppoint{gp mark 1}{(9.707,2.007)}
\gppoint{gp mark 1}{(9.710,2.984)}
\gppoint{gp mark 1}{(9.712,2.515)}
\gppoint{gp mark 1}{(9.715,2.143)}
\gppoint{gp mark 1}{(9.717,3.203)}
\gppoint{gp mark 1}{(9.720,3.028)}
\gppoint{gp mark 1}{(9.722,2.580)}
\gppoint{gp mark 1}{(9.725,2.935)}
\gppoint{gp mark 1}{(9.727,2.749)}
\gppoint{gp mark 1}{(9.730,3.110)}
\gppoint{gp mark 1}{(9.732,3.017)}
\gppoint{gp mark 1}{(9.735,2.766)}
\gppoint{gp mark 1}{(9.737,2.930)}
\gppoint{gp mark 1}{(9.740,2.525)}
\gppoint{gp mark 1}{(9.742,3.208)}
\gppoint{gp mark 1}{(9.745,2.810)}
\gppoint{gp mark 1}{(9.747,2.689)}
\gppoint{gp mark 1}{(9.750,2.678)}
\gppoint{gp mark 1}{(9.752,2.405)}
\gppoint{gp mark 1}{(9.755,2.689)}
\gppoint{gp mark 1}{(9.757,2.405)}
\gppoint{gp mark 1}{(9.760,2.383)}
\gppoint{gp mark 1}{(9.762,2.515)}
\gppoint{gp mark 1}{(9.765,3.034)}
\gppoint{gp mark 1}{(9.767,2.460)}
\gppoint{gp mark 1}{(9.770,2.810)}
\gppoint{gp mark 1}{(9.772,2.826)}
\gppoint{gp mark 1}{(9.775,2.203)}
\gppoint{gp mark 1}{(9.777,2.433)}
\gppoint{gp mark 1}{(9.779,2.646)}
\gppoint{gp mark 1}{(9.782,2.777)}
\gppoint{gp mark 1}{(9.784,2.842)}
\gppoint{gp mark 1}{(9.787,2.220)}
\gppoint{gp mark 1}{(9.789,2.542)}
\gppoint{gp mark 1}{(9.792,3.132)}
\gppoint{gp mark 1}{(9.794,2.536)}
\gppoint{gp mark 1}{(9.797,2.575)}
\gppoint{gp mark 1}{(9.799,2.476)}
\gppoint{gp mark 1}{(9.802,2.389)}
\gppoint{gp mark 1}{(9.804,2.968)}
\gppoint{gp mark 1}{(9.807,2.072)}
\gppoint{gp mark 1}{(9.809,2.444)}
\gppoint{gp mark 1}{(9.812,2.635)}
\gppoint{gp mark 1}{(9.814,2.433)}
\gppoint{gp mark 1}{(9.817,2.602)}
\gppoint{gp mark 1}{(9.819,2.804)}
\gppoint{gp mark 1}{(9.822,2.356)}
\gppoint{gp mark 1}{(9.824,2.504)}
\gppoint{gp mark 1}{(9.827,2.744)}
\gppoint{gp mark 1}{(9.829,2.449)}
\gppoint{gp mark 1}{(9.832,2.635)}
\gppoint{gp mark 1}{(9.834,2.728)}
\gppoint{gp mark 1}{(9.837,2.717)}
\gppoint{gp mark 1}{(9.839,3.225)}
\gppoint{gp mark 1}{(9.842,1.946)}
\gppoint{gp mark 1}{(9.844,2.744)}
\gppoint{gp mark 1}{(9.847,3.006)}
\gppoint{gp mark 1}{(9.849,2.973)}
\gppoint{gp mark 1}{(9.852,2.711)}
\gppoint{gp mark 1}{(9.854,2.662)}
\gppoint{gp mark 1}{(9.857,2.291)}
\gppoint{gp mark 1}{(9.859,2.782)}
\gppoint{gp mark 1}{(9.861,2.291)}
\gppoint{gp mark 1}{(9.864,2.744)}
\gppoint{gp mark 1}{(9.866,2.635)}
\gppoint{gp mark 1}{(9.869,2.427)}
\gppoint{gp mark 1}{(9.871,2.886)}
\gppoint{gp mark 1}{(9.874,2.170)}
\gppoint{gp mark 1}{(9.876,3.039)}
\gppoint{gp mark 1}{(9.879,2.542)}
\gppoint{gp mark 1}{(9.881,2.531)}
\gppoint{gp mark 1}{(9.884,3.367)}
\gppoint{gp mark 1}{(9.886,1.810)}
\gppoint{gp mark 1}{(9.889,2.520)}
\gppoint{gp mark 1}{(9.891,2.733)}
\gppoint{gp mark 1}{(9.894,3.400)}
\gppoint{gp mark 1}{(9.896,2.383)}
\gppoint{gp mark 1}{(9.899,2.028)}
\gppoint{gp mark 1}{(9.901,2.678)}
\gppoint{gp mark 1}{(9.904,2.668)}
\gppoint{gp mark 1}{(9.906,2.056)}
\gppoint{gp mark 1}{(9.909,2.564)}
\gppoint{gp mark 1}{(9.911,2.515)}
\gppoint{gp mark 1}{(9.914,2.416)}
\gppoint{gp mark 1}{(9.916,2.329)}
\gppoint{gp mark 1}{(9.919,1.963)}
\gppoint{gp mark 1}{(9.921,2.902)}
\gppoint{gp mark 1}{(9.924,2.422)}
\gppoint{gp mark 1}{(9.926,2.482)}
\gppoint{gp mark 1}{(9.929,2.400)}
\gppoint{gp mark 1}{(9.931,2.493)}
\gppoint{gp mark 1}{(9.934,2.176)}
\gppoint{gp mark 1}{(9.936,2.427)}
\gppoint{gp mark 1}{(9.939,2.149)}
\gppoint{gp mark 1}{(9.941,2.274)}
\gppoint{gp mark 1}{(9.943,2.007)}
\gppoint{gp mark 1}{(9.946,3.257)}
\gppoint{gp mark 1}{(9.948,2.782)}
\gppoint{gp mark 1}{(9.951,2.810)}
\gppoint{gp mark 1}{(9.953,2.668)}
\gppoint{gp mark 1}{(9.956,2.744)}
\gppoint{gp mark 1}{(9.958,2.198)}
\gppoint{gp mark 1}{(9.961,2.263)}
\gppoint{gp mark 1}{(9.963,2.952)}
\gppoint{gp mark 1}{(9.966,2.859)}
\gppoint{gp mark 1}{(9.968,2.618)}
\gppoint{gp mark 1}{(9.971,2.198)}
\gppoint{gp mark 1}{(9.973,2.728)}
\gppoint{gp mark 1}{(9.976,2.891)}
\gppoint{gp mark 1}{(9.978,2.771)}
\gppoint{gp mark 1}{(9.981,2.383)}
\gppoint{gp mark 1}{(9.983,2.842)}
\gppoint{gp mark 1}{(9.986,2.837)}
\gppoint{gp mark 1}{(9.988,2.241)}
\gppoint{gp mark 1}{(9.991,2.881)}
\gppoint{gp mark 1}{(9.993,1.832)}
\gppoint{gp mark 1}{(9.996,2.722)}
\gppoint{gp mark 1}{(9.998,2.777)}
\gppoint{gp mark 1}{(10.001,2.990)}
\gppoint{gp mark 1}{(10.003,2.777)}
\gppoint{gp mark 1}{(10.006,2.564)}
\gppoint{gp mark 1}{(10.008,2.618)}
\gppoint{gp mark 1}{(10.011,2.198)}
\gppoint{gp mark 1}{(10.013,2.575)}
\gppoint{gp mark 1}{(10.016,2.810)}
\gppoint{gp mark 1}{(10.018,2.624)}
\gppoint{gp mark 1}{(10.021,2.236)}
\gppoint{gp mark 1}{(10.023,2.542)}
\gppoint{gp mark 1}{(10.026,2.345)}
\gppoint{gp mark 1}{(10.028,2.394)}
\gppoint{gp mark 1}{(10.030,2.274)}
\gppoint{gp mark 1}{(10.033,2.405)}
\gppoint{gp mark 1}{(10.035,2.181)}
\gppoint{gp mark 1}{(10.038,2.744)}
\gppoint{gp mark 1}{(10.040,3.105)}
\gppoint{gp mark 1}{(10.043,2.624)}
\gppoint{gp mark 1}{(10.045,2.329)}
\gppoint{gp mark 1}{(10.048,2.400)}
\gppoint{gp mark 1}{(10.050,2.531)}
\gppoint{gp mark 1}{(10.053,2.405)}
\gppoint{gp mark 1}{(10.055,2.919)}
\gppoint{gp mark 1}{(10.058,2.673)}
\gppoint{gp mark 1}{(10.060,2.689)}
\gppoint{gp mark 1}{(10.063,2.416)}
\gppoint{gp mark 1}{(10.065,2.602)}
\gppoint{gp mark 1}{(10.068,2.536)}
\gppoint{gp mark 1}{(10.070,2.203)}
\gppoint{gp mark 1}{(10.073,2.378)}
\gppoint{gp mark 1}{(10.075,2.017)}
\gppoint{gp mark 1}{(10.078,2.744)}
\gppoint{gp mark 1}{(10.080,2.739)}
\gppoint{gp mark 1}{(10.083,2.444)}
\gppoint{gp mark 1}{(10.085,2.427)}
\gppoint{gp mark 1}{(10.088,3.006)}
\gppoint{gp mark 1}{(10.090,2.646)}
\gppoint{gp mark 1}{(10.093,2.329)}
\gppoint{gp mark 1}{(10.095,2.771)}
\gppoint{gp mark 1}{(10.098,2.493)}
\gppoint{gp mark 1}{(10.100,2.553)}
\gppoint{gp mark 1}{(10.103,2.542)}
\gppoint{gp mark 1}{(10.105,2.646)}
\gppoint{gp mark 1}{(10.108,2.760)}
\gppoint{gp mark 1}{(10.110,2.602)}
\gppoint{gp mark 1}{(10.112,3.148)}
\gppoint{gp mark 1}{(10.115,2.810)}
\gppoint{gp mark 1}{(10.117,3.072)}
\gppoint{gp mark 1}{(10.120,2.176)}
\gppoint{gp mark 1}{(10.122,2.760)}
\gppoint{gp mark 1}{(10.125,2.848)}
\gppoint{gp mark 1}{(10.127,2.891)}
\gppoint{gp mark 1}{(10.130,2.471)}
\gppoint{gp mark 1}{(10.132,3.678)}
\gppoint{gp mark 1}{(10.135,2.881)}
\gppoint{gp mark 1}{(10.137,2.646)}
\gppoint{gp mark 1}{(10.140,2.946)}
\gppoint{gp mark 1}{(10.142,2.848)}
\gppoint{gp mark 1}{(10.145,2.400)}
\gppoint{gp mark 1}{(10.147,2.689)}
\gppoint{gp mark 1}{(10.150,3.236)}
\gppoint{gp mark 1}{(10.152,2.465)}
\gppoint{gp mark 1}{(10.155,2.749)}
\gppoint{gp mark 1}{(10.157,3.356)}
\gppoint{gp mark 1}{(10.160,2.804)}
\gppoint{gp mark 1}{(10.162,3.077)}
\gppoint{gp mark 1}{(10.165,3.017)}
\gppoint{gp mark 1}{(10.167,3.372)}
\gppoint{gp mark 1}{(10.170,3.307)}
\gppoint{gp mark 1}{(10.172,2.400)}
\gppoint{gp mark 1}{(10.175,2.465)}
\gppoint{gp mark 1}{(10.177,2.471)}
\gppoint{gp mark 1}{(10.180,2.252)}
\gppoint{gp mark 1}{(10.182,2.864)}
\gppoint{gp mark 1}{(10.185,2.739)}
\gppoint{gp mark 1}{(10.187,2.717)}
\gppoint{gp mark 1}{(10.190,2.810)}
\gppoint{gp mark 1}{(10.192,2.394)}
\gppoint{gp mark 1}{(10.194,2.749)}
\gppoint{gp mark 1}{(10.197,1.919)}
\gppoint{gp mark 1}{(10.199,3.099)}
\gppoint{gp mark 1}{(10.202,2.416)}
\gppoint{gp mark 1}{(10.204,3.389)}
\gppoint{gp mark 1}{(10.207,3.110)}
\gppoint{gp mark 1}{(10.209,3.252)}
\gppoint{gp mark 1}{(10.212,2.902)}
\gppoint{gp mark 1}{(10.214,2.695)}
\gppoint{gp mark 1}{(10.217,2.678)}
\gppoint{gp mark 1}{(10.219,2.695)}
\gppoint{gp mark 1}{(10.222,2.875)}
\gppoint{gp mark 1}{(10.224,2.913)}
\gppoint{gp mark 1}{(10.227,2.509)}
\gppoint{gp mark 1}{(10.229,2.263)}
\gppoint{gp mark 1}{(10.232,2.837)}
\gppoint{gp mark 1}{(10.234,2.700)}
\gppoint{gp mark 1}{(10.237,2.471)}
\gppoint{gp mark 1}{(10.239,3.077)}
\gppoint{gp mark 1}{(10.242,2.739)}
\gppoint{gp mark 1}{(10.244,2.788)}
\gppoint{gp mark 1}{(10.247,2.460)}
\gppoint{gp mark 1}{(10.249,2.859)}
\gppoint{gp mark 1}{(10.252,2.662)}
\gppoint{gp mark 1}{(10.254,2.984)}
\gppoint{gp mark 1}{(10.257,2.651)}
\gppoint{gp mark 1}{(10.259,2.023)}
\gppoint{gp mark 1}{(10.262,2.837)}
\gppoint{gp mark 1}{(10.264,2.307)}
\gppoint{gp mark 1}{(10.267,2.340)}
\gppoint{gp mark 1}{(10.269,2.706)}
\gppoint{gp mark 1}{(10.272,3.055)}
\gppoint{gp mark 1}{(10.274,2.061)}
\gppoint{gp mark 1}{(10.277,2.766)}
\gppoint{gp mark 1}{(10.279,2.820)}
\gppoint{gp mark 1}{(10.281,2.782)}
\gppoint{gp mark 1}{(10.284,2.547)}
\gppoint{gp mark 1}{(10.286,3.176)}
\gppoint{gp mark 1}{(10.289,2.804)}
\gppoint{gp mark 1}{(10.291,3.137)}
\gppoint{gp mark 1}{(10.294,2.383)}
\gppoint{gp mark 1}{(10.296,2.717)}
\gppoint{gp mark 1}{(10.299,2.012)}
\gppoint{gp mark 1}{(10.301,2.285)}
\gppoint{gp mark 1}{(10.304,3.044)}
\gppoint{gp mark 1}{(10.306,2.908)}
\gppoint{gp mark 1}{(10.309,2.241)}
\gppoint{gp mark 1}{(10.311,2.258)}
\gppoint{gp mark 1}{(10.314,2.465)}
\gppoint{gp mark 1}{(10.316,2.777)}
\gppoint{gp mark 1}{(10.319,1.914)}
\gppoint{gp mark 1}{(10.321,2.804)}
\gppoint{gp mark 1}{(10.324,2.602)}
\gppoint{gp mark 1}{(10.326,3.328)}
\gppoint{gp mark 1}{(10.329,3.034)}
\gppoint{gp mark 1}{(10.331,3.083)}
\gppoint{gp mark 1}{(10.334,1.963)}
\gppoint{gp mark 1}{(10.336,3.001)}
\gppoint{gp mark 1}{(10.339,2.318)}
\gppoint{gp mark 1}{(10.341,2.586)}
\gppoint{gp mark 1}{(10.344,2.596)}
\gppoint{gp mark 1}{(10.346,2.618)}
\gppoint{gp mark 1}{(10.349,2.471)}
\gppoint{gp mark 1}{(10.351,2.482)}
\gppoint{gp mark 1}{(10.354,2.471)}
\gppoint{gp mark 1}{(10.356,2.460)}
\gppoint{gp mark 1}{(10.359,3.400)}
\gppoint{gp mark 1}{(10.361,2.760)}
\gppoint{gp mark 1}{(10.363,2.504)}
\gppoint{gp mark 1}{(10.366,3.372)}
\gppoint{gp mark 1}{(10.368,3.017)}
\gppoint{gp mark 1}{(10.371,3.137)}
\gppoint{gp mark 1}{(10.373,3.618)}
\gppoint{gp mark 1}{(10.376,2.334)}
\gppoint{gp mark 1}{(10.378,2.902)}
\gppoint{gp mark 1}{(10.381,2.788)}
\gppoint{gp mark 1}{(10.383,1.963)}
\gppoint{gp mark 1}{(10.386,2.941)}
\gppoint{gp mark 1}{(10.388,2.973)}
\gppoint{gp mark 1}{(10.391,2.411)}
\gppoint{gp mark 1}{(10.393,3.012)}
\gppoint{gp mark 1}{(10.396,2.149)}
\gppoint{gp mark 1}{(10.398,2.329)}
\gppoint{gp mark 1}{(10.401,3.356)}
\gppoint{gp mark 1}{(10.403,3.268)}
\gppoint{gp mark 1}{(10.406,2.547)}
\gppoint{gp mark 1}{(10.408,2.881)}
\gppoint{gp mark 1}{(10.411,3.034)}
\gppoint{gp mark 1}{(10.413,3.044)}
\gppoint{gp mark 1}{(10.416,3.230)}
\gppoint{gp mark 1}{(10.418,3.017)}
\gppoint{gp mark 1}{(10.421,2.433)}
\gppoint{gp mark 1}{(10.423,2.837)}
\gppoint{gp mark 1}{(10.426,2.525)}
\gppoint{gp mark 1}{(10.428,2.739)}
\gppoint{gp mark 1}{(10.431,3.318)}
\gppoint{gp mark 1}{(10.433,2.329)}
\gppoint{gp mark 1}{(10.436,2.984)}
\gppoint{gp mark 1}{(10.438,2.678)}
\gppoint{gp mark 1}{(10.441,3.192)}
\gppoint{gp mark 1}{(10.443,2.088)}
\gppoint{gp mark 1}{(10.445,3.099)}
\gppoint{gp mark 1}{(10.448,2.717)}
\gppoint{gp mark 1}{(10.450,2.722)}
\gppoint{gp mark 1}{(10.453,2.280)}
\gppoint{gp mark 1}{(10.455,2.323)}
\gppoint{gp mark 1}{(10.458,2.717)}
\gppoint{gp mark 1}{(10.460,2.607)}
\gppoint{gp mark 1}{(10.463,2.569)}
\gppoint{gp mark 1}{(10.465,2.739)}
\gppoint{gp mark 1}{(10.468,2.558)}
\gppoint{gp mark 1}{(10.470,3.077)}
\gppoint{gp mark 1}{(10.473,2.602)}
\gppoint{gp mark 1}{(10.475,2.515)}
\gppoint{gp mark 1}{(10.478,2.187)}
\gppoint{gp mark 1}{(10.480,3.247)}
\gppoint{gp mark 1}{(10.483,2.657)}
\gppoint{gp mark 1}{(10.485,2.946)}
\gppoint{gp mark 1}{(10.488,3.072)}
\gppoint{gp mark 1}{(10.490,2.564)}
\gppoint{gp mark 1}{(10.493,2.881)}
\gppoint{gp mark 1}{(10.495,2.684)}
\gppoint{gp mark 1}{(10.498,2.105)}
\gppoint{gp mark 1}{(10.500,2.536)}
\gppoint{gp mark 1}{(10.503,3.055)}
\gppoint{gp mark 1}{(10.505,3.099)}
\gppoint{gp mark 1}{(10.508,2.383)}
\gppoint{gp mark 1}{(10.510,2.345)}
\gppoint{gp mark 1}{(10.513,2.493)}
\gppoint{gp mark 1}{(10.515,2.629)}
\gppoint{gp mark 1}{(10.518,2.711)}
\gppoint{gp mark 1}{(10.520,2.383)}
\gppoint{gp mark 1}{(10.523,3.498)}
\gppoint{gp mark 1}{(10.525,3.208)}
\gppoint{gp mark 1}{(10.528,2.482)}
\gppoint{gp mark 1}{(10.530,2.973)}
\gppoint{gp mark 1}{(10.532,3.099)}
\gppoint{gp mark 1}{(10.535,2.804)}
\gppoint{gp mark 1}{(10.537,2.143)}
\gppoint{gp mark 1}{(10.540,2.941)}
\gppoint{gp mark 1}{(10.542,2.378)}
\gppoint{gp mark 1}{(10.545,2.252)}
\gppoint{gp mark 1}{(10.547,2.252)}
\gppoint{gp mark 1}{(10.550,2.640)}
\gppoint{gp mark 1}{(10.552,2.291)}
\gppoint{gp mark 1}{(10.555,2.657)}
\gppoint{gp mark 1}{(10.557,2.312)}
\gppoint{gp mark 1}{(10.560,3.077)}
\gppoint{gp mark 1}{(10.562,3.066)}
\gppoint{gp mark 1}{(10.565,2.820)}
\gppoint{gp mark 1}{(10.567,2.110)}
\gppoint{gp mark 1}{(10.570,2.181)}
\gppoint{gp mark 1}{(10.572,3.400)}
\gppoint{gp mark 1}{(10.575,2.668)}
\gppoint{gp mark 1}{(10.577,3.012)}
\gppoint{gp mark 1}{(10.580,3.263)}
\gppoint{gp mark 1}{(10.582,1.788)}
\gppoint{gp mark 1}{(10.585,2.744)}
\gppoint{gp mark 1}{(10.587,2.629)}
\gppoint{gp mark 1}{(10.590,2.056)}
\gppoint{gp mark 1}{(10.592,2.684)}
\gppoint{gp mark 1}{(10.595,2.433)}
\gppoint{gp mark 1}{(10.597,2.378)}
\gppoint{gp mark 1}{(10.600,2.733)}
\gppoint{gp mark 1}{(10.602,2.607)}
\gppoint{gp mark 1}{(10.605,3.006)}
\gppoint{gp mark 1}{(10.607,2.788)}
\gppoint{gp mark 1}{(10.610,3.290)}
\gppoint{gp mark 1}{(10.612,2.837)}
\gppoint{gp mark 1}{(10.614,2.853)}
\gppoint{gp mark 1}{(10.617,2.722)}
\gppoint{gp mark 1}{(10.619,2.979)}
\gppoint{gp mark 1}{(10.622,2.739)}
\gppoint{gp mark 1}{(10.624,2.678)}
\gppoint{gp mark 1}{(10.627,3.115)}
\gppoint{gp mark 1}{(10.629,3.323)}
\gppoint{gp mark 1}{(10.632,2.946)}
\gppoint{gp mark 1}{(10.634,2.121)}
\gppoint{gp mark 1}{(10.637,2.580)}
\gppoint{gp mark 1}{(10.639,3.481)}
\gppoint{gp mark 1}{(10.642,1.957)}
\gppoint{gp mark 1}{(10.644,3.443)}
\gppoint{gp mark 1}{(10.647,2.116)}
\gppoint{gp mark 1}{(10.649,2.564)}
\gppoint{gp mark 1}{(10.652,2.831)}
\gppoint{gp mark 1}{(10.654,2.220)}
\gppoint{gp mark 1}{(10.657,2.946)}
\gppoint{gp mark 1}{(10.659,2.886)}
\gppoint{gp mark 1}{(10.662,3.400)}
\gppoint{gp mark 1}{(10.664,2.121)}
\gppoint{gp mark 1}{(10.667,2.214)}
\gppoint{gp mark 1}{(10.669,2.591)}
\gppoint{gp mark 1}{(10.672,2.569)}
\gppoint{gp mark 1}{(10.674,2.345)}
\gppoint{gp mark 1}{(10.677,2.498)}
\gppoint{gp mark 1}{(10.679,2.962)}
\gppoint{gp mark 1}{(10.682,2.700)}
\gppoint{gp mark 1}{(10.684,2.564)}
\gppoint{gp mark 1}{(10.687,3.656)}
\gppoint{gp mark 1}{(10.689,2.307)}
\gppoint{gp mark 1}{(10.692,2.613)}
\gppoint{gp mark 1}{(10.694,3.274)}
\gppoint{gp mark 1}{(10.696,2.640)}
\gppoint{gp mark 1}{(10.699,2.973)}
\gppoint{gp mark 1}{(10.701,3.219)}
\gppoint{gp mark 1}{(10.704,2.657)}
\gppoint{gp mark 1}{(10.706,2.165)}
\gppoint{gp mark 1}{(10.709,2.504)}
\gppoint{gp mark 1}{(10.711,1.985)}
\gppoint{gp mark 1}{(10.714,2.362)}
\gppoint{gp mark 1}{(10.716,3.039)}
\gppoint{gp mark 1}{(10.719,2.668)}
\gppoint{gp mark 1}{(10.721,2.493)}
\gppoint{gp mark 1}{(10.724,2.542)}
\gppoint{gp mark 1}{(10.726,2.935)}
\gppoint{gp mark 1}{(10.729,3.006)}
\gppoint{gp mark 1}{(10.731,2.531)}
\gppoint{gp mark 1}{(10.734,2.329)}
\gppoint{gp mark 1}{(10.736,2.351)}
\gppoint{gp mark 1}{(10.739,2.187)}
\gppoint{gp mark 1}{(10.741,2.930)}
\gppoint{gp mark 1}{(10.744,2.121)}
\gppoint{gp mark 1}{(10.746,2.318)}
\gppoint{gp mark 1}{(10.749,1.936)}
\gppoint{gp mark 1}{(10.751,2.897)}
\gppoint{gp mark 1}{(10.754,2.897)}
\gppoint{gp mark 1}{(10.756,2.799)}
\gppoint{gp mark 1}{(10.759,2.902)}
\gppoint{gp mark 1}{(10.761,2.531)}
\gppoint{gp mark 1}{(10.764,2.596)}
\gppoint{gp mark 1}{(10.766,2.684)}
\gppoint{gp mark 1}{(10.769,2.411)}
\gppoint{gp mark 1}{(10.771,2.662)}
\gppoint{gp mark 1}{(10.774,2.302)}
\gppoint{gp mark 1}{(10.776,2.553)}
\gppoint{gp mark 1}{(10.779,2.979)}
\gppoint{gp mark 1}{(10.781,3.192)}
\gppoint{gp mark 1}{(10.783,2.684)}
\gppoint{gp mark 1}{(10.786,2.744)}
\gppoint{gp mark 1}{(10.788,2.127)}
\gppoint{gp mark 1}{(10.791,2.023)}
\gppoint{gp mark 1}{(10.793,2.498)}
\gppoint{gp mark 1}{(10.796,2.362)}
\gppoint{gp mark 1}{(10.798,2.695)}
\gppoint{gp mark 1}{(10.801,2.427)}
\gppoint{gp mark 1}{(10.803,2.569)}
\gppoint{gp mark 1}{(10.806,2.460)}
\gppoint{gp mark 1}{(10.808,2.362)}
\gppoint{gp mark 1}{(10.811,3.050)}
\gppoint{gp mark 1}{(10.813,2.842)}
\gppoint{gp mark 1}{(10.816,2.416)}
\gppoint{gp mark 1}{(10.818,3.170)}
\gppoint{gp mark 1}{(10.821,2.547)}
\gppoint{gp mark 1}{(10.823,2.635)}
\gppoint{gp mark 1}{(10.826,2.465)}
\gppoint{gp mark 1}{(10.828,2.793)}
\gppoint{gp mark 1}{(10.831,2.859)}
\gppoint{gp mark 1}{(10.833,2.416)}
\gppoint{gp mark 1}{(10.836,2.695)}
\gppoint{gp mark 1}{(10.838,3.236)}
\gppoint{gp mark 1}{(10.841,3.006)}
\gppoint{gp mark 1}{(10.843,2.553)}
\gppoint{gp mark 1}{(10.846,2.121)}
\gppoint{gp mark 1}{(10.848,2.487)}
\gppoint{gp mark 1}{(10.851,3.405)}
\gppoint{gp mark 1}{(10.853,2.646)}
\gppoint{gp mark 1}{(10.856,3.296)}
\gppoint{gp mark 1}{(10.858,2.465)}
\gppoint{gp mark 1}{(10.861,3.421)}
\gppoint{gp mark 1}{(10.863,2.842)}
\gppoint{gp mark 1}{(10.865,2.110)}
\gppoint{gp mark 1}{(10.868,2.662)}
\gppoint{gp mark 1}{(10.870,2.471)}
\gppoint{gp mark 1}{(10.873,2.717)}
\gppoint{gp mark 1}{(10.875,2.383)}
\gppoint{gp mark 1}{(10.878,2.575)}
\gppoint{gp mark 1}{(10.880,2.580)}
\gppoint{gp mark 1}{(10.883,2.039)}
\gppoint{gp mark 1}{(10.885,2.629)}
\gppoint{gp mark 1}{(10.888,2.405)}
\gppoint{gp mark 1}{(10.890,3.334)}
\gppoint{gp mark 1}{(10.893,2.487)}
\gppoint{gp mark 1}{(10.895,2.433)}
\gppoint{gp mark 1}{(10.898,3.055)}
\gppoint{gp mark 1}{(10.900,2.640)}
\gppoint{gp mark 1}{(10.903,2.919)}
\gppoint{gp mark 1}{(10.905,2.438)}
\gppoint{gp mark 1}{(10.908,2.886)}
\gppoint{gp mark 1}{(10.910,3.268)}
\gppoint{gp mark 1}{(10.913,2.482)}
\gppoint{gp mark 1}{(10.915,2.831)}
\gppoint{gp mark 1}{(10.918,2.045)}
\gppoint{gp mark 1}{(10.920,2.498)}
\gppoint{gp mark 1}{(10.923,3.279)}
\gppoint{gp mark 1}{(10.925,2.728)}
\gppoint{gp mark 1}{(10.928,2.269)}
\gppoint{gp mark 1}{(10.930,3.001)}
\gppoint{gp mark 1}{(10.933,2.373)}
\gppoint{gp mark 1}{(10.935,2.225)}
\gppoint{gp mark 1}{(10.938,2.318)}
\gppoint{gp mark 1}{(10.940,2.531)}
\gppoint{gp mark 1}{(10.943,2.564)}
\gppoint{gp mark 1}{(10.945,2.482)}
\gppoint{gp mark 1}{(10.947,2.460)}
\gppoint{gp mark 1}{(10.950,2.525)}
\gppoint{gp mark 1}{(10.952,2.564)}
\gppoint{gp mark 1}{(10.955,2.629)}
\gppoint{gp mark 1}{(10.957,2.810)}
\gppoint{gp mark 1}{(10.960,2.799)}
\gppoint{gp mark 1}{(10.962,2.143)}
\gppoint{gp mark 1}{(10.965,2.607)}
\gppoint{gp mark 1}{(10.967,2.837)}
\gppoint{gp mark 1}{(10.970,3.137)}
\gppoint{gp mark 1}{(10.972,3.416)}
\gppoint{gp mark 1}{(10.975,2.580)}
\gppoint{gp mark 1}{(10.977,2.629)}
\gppoint{gp mark 1}{(10.980,2.886)}
\gppoint{gp mark 1}{(10.982,1.750)}
\gppoint{gp mark 1}{(10.985,2.345)}
\gppoint{gp mark 1}{(10.987,2.760)}
\gppoint{gp mark 1}{(10.990,3.345)}
\gppoint{gp mark 1}{(10.992,2.941)}
\gppoint{gp mark 1}{(10.995,2.875)}
\gppoint{gp mark 1}{(10.997,2.908)}
\gppoint{gp mark 1}{(11.000,2.416)}
\gppoint{gp mark 1}{(11.002,2.826)}
\gppoint{gp mark 1}{(11.005,2.706)}
\gppoint{gp mark 1}{(11.007,3.312)}
\gppoint{gp mark 1}{(11.010,2.558)}
\gppoint{gp mark 1}{(11.012,2.034)}
\gppoint{gp mark 1}{(11.015,2.613)}
\gppoint{gp mark 1}{(11.017,2.881)}
\gppoint{gp mark 1}{(11.020,2.356)}
\gppoint{gp mark 1}{(11.022,2.968)}
\gppoint{gp mark 1}{(11.025,3.159)}
\gppoint{gp mark 1}{(11.027,2.214)}
\gppoint{gp mark 1}{(11.029,2.143)}
\gppoint{gp mark 1}{(11.032,3.154)}
\gppoint{gp mark 1}{(11.034,2.438)}
\gppoint{gp mark 1}{(11.037,2.733)}
\gppoint{gp mark 1}{(11.039,2.580)}
\gppoint{gp mark 1}{(11.042,2.886)}
\gppoint{gp mark 1}{(11.044,2.766)}
\gppoint{gp mark 1}{(11.047,2.668)}
\gppoint{gp mark 1}{(11.049,2.525)}
\gppoint{gp mark 1}{(11.052,2.542)}
\gppoint{gp mark 1}{(11.054,2.793)}
\gppoint{gp mark 1}{(11.057,2.034)}
\gppoint{gp mark 1}{(11.059,2.427)}
\gppoint{gp mark 1}{(11.062,2.433)}
\gppoint{gp mark 1}{(11.064,2.864)}
\gppoint{gp mark 1}{(11.067,2.586)}
\gppoint{gp mark 1}{(11.069,2.668)}
\gppoint{gp mark 1}{(11.072,2.203)}
\gppoint{gp mark 1}{(11.074,2.875)}
\gppoint{gp mark 1}{(11.077,2.749)}
\gppoint{gp mark 1}{(11.079,2.935)}
\gppoint{gp mark 1}{(11.082,3.083)}
\gppoint{gp mark 1}{(11.084,3.410)}
\gppoint{gp mark 1}{(11.087,2.607)}
\gppoint{gp mark 1}{(11.089,2.651)}
\gppoint{gp mark 1}{(11.092,1.793)}
\gppoint{gp mark 1}{(11.094,2.684)}
\gppoint{gp mark 1}{(11.097,2.416)}
\gppoint{gp mark 1}{(11.099,2.607)}
\gppoint{gp mark 1}{(11.102,2.323)}
\gppoint{gp mark 1}{(11.104,3.476)}
\gppoint{gp mark 1}{(11.107,2.389)}
\gppoint{gp mark 1}{(11.109,2.187)}
\gppoint{gp mark 1}{(11.112,2.684)}
\gppoint{gp mark 1}{(11.114,2.624)}
\gppoint{gp mark 1}{(11.116,2.651)}
\gppoint{gp mark 1}{(11.119,2.536)}
\gppoint{gp mark 1}{(11.121,2.460)}
\gppoint{gp mark 1}{(11.124,2.230)}
\gppoint{gp mark 1}{(11.126,2.580)}
\gppoint{gp mark 1}{(11.129,3.094)}
\gppoint{gp mark 1}{(11.131,1.684)}
\gppoint{gp mark 1}{(11.134,3.394)}
\gppoint{gp mark 1}{(11.136,2.252)}
\gppoint{gp mark 1}{(11.139,2.438)}
\gppoint{gp mark 1}{(11.141,2.334)}
\gppoint{gp mark 1}{(11.144,1.684)}
\gppoint{gp mark 1}{(11.146,2.837)}
\gppoint{gp mark 1}{(11.149,2.411)}
\gppoint{gp mark 1}{(11.151,3.296)}
\gppoint{gp mark 1}{(11.154,2.793)}
\gppoint{gp mark 1}{(11.156,2.340)}
\gppoint{gp mark 1}{(11.159,2.777)}
\gppoint{gp mark 1}{(11.161,2.476)}
\gppoint{gp mark 1}{(11.164,2.695)}
\gppoint{gp mark 1}{(11.166,3.203)}
\gppoint{gp mark 1}{(11.169,2.760)}
\gppoint{gp mark 1}{(11.171,2.444)}
\gppoint{gp mark 1}{(11.174,2.465)}
\gppoint{gp mark 1}{(11.176,2.176)}
\gppoint{gp mark 1}{(11.179,2.613)}
\gppoint{gp mark 1}{(11.181,2.127)}
\gppoint{gp mark 1}{(11.184,2.088)}
\gppoint{gp mark 1}{(11.186,2.946)}
\gppoint{gp mark 1}{(11.189,3.105)}
\gppoint{gp mark 1}{(11.191,2.422)}
\gppoint{gp mark 1}{(11.194,2.034)}
\gppoint{gp mark 1}{(11.196,2.886)}
\gppoint{gp mark 1}{(11.198,2.307)}
\gppoint{gp mark 1}{(11.201,2.886)}
\gppoint{gp mark 1}{(11.203,2.728)}
\gppoint{gp mark 1}{(11.206,2.329)}
\gppoint{gp mark 1}{(11.208,2.990)}
\gppoint{gp mark 1}{(11.211,2.438)}
\gppoint{gp mark 1}{(11.213,3.236)}
\gppoint{gp mark 1}{(11.216,2.793)}
\gppoint{gp mark 1}{(11.218,2.209)}
\gppoint{gp mark 1}{(11.221,3.465)}
\gppoint{gp mark 1}{(11.223,2.826)}
\gppoint{gp mark 1}{(11.226,2.340)}
\gppoint{gp mark 1}{(11.228,2.962)}
\gppoint{gp mark 1}{(11.231,2.088)}
\gppoint{gp mark 1}{(11.233,2.990)}
\gppoint{gp mark 1}{(11.236,2.973)}
\gppoint{gp mark 1}{(11.238,2.307)}
\gppoint{gp mark 1}{(11.241,1.979)}
\gppoint{gp mark 1}{(11.243,2.536)}
\gppoint{gp mark 1}{(11.246,2.362)}
\gppoint{gp mark 1}{(11.248,2.531)}
\gppoint{gp mark 1}{(11.251,1.471)}
\gppoint{gp mark 1}{(11.253,2.241)}
\gppoint{gp mark 1}{(11.256,2.394)}
\gppoint{gp mark 1}{(11.258,2.733)}
\gppoint{gp mark 1}{(11.261,3.077)}
\gppoint{gp mark 1}{(11.263,2.449)}
\gppoint{gp mark 1}{(11.266,2.575)}
\gppoint{gp mark 1}{(11.268,2.165)}
\gppoint{gp mark 1}{(11.271,2.498)}
\gppoint{gp mark 1}{(11.273,3.023)}
\gppoint{gp mark 1}{(11.276,2.444)}
\gppoint{gp mark 1}{(11.278,2.728)}
\gppoint{gp mark 1}{(11.280,2.853)}
\gppoint{gp mark 1}{(11.283,2.465)}
\gppoint{gp mark 1}{(11.285,3.520)}
\gppoint{gp mark 1}{(11.288,3.001)}
\gppoint{gp mark 1}{(11.290,2.558)}
\gppoint{gp mark 1}{(11.293,2.547)}
\gppoint{gp mark 1}{(11.295,2.919)}
\gppoint{gp mark 1}{(11.298,2.220)}
\gppoint{gp mark 1}{(11.300,2.099)}
\gppoint{gp mark 1}{(11.303,2.870)}
\gppoint{gp mark 1}{(11.305,2.094)}
\gppoint{gp mark 1}{(11.308,3.126)}
\gppoint{gp mark 1}{(11.310,3.241)}
\gppoint{gp mark 1}{(11.313,2.116)}
\gppoint{gp mark 1}{(11.315,2.591)}
\gppoint{gp mark 1}{(11.318,2.285)}
\gppoint{gp mark 1}{(11.320,2.902)}
\gppoint{gp mark 1}{(11.323,2.531)}
\gppoint{gp mark 1}{(11.325,2.498)}
\gppoint{gp mark 1}{(11.328,2.302)}
\gppoint{gp mark 1}{(11.330,1.804)}
\gppoint{gp mark 1}{(11.333,2.564)}
\gppoint{gp mark 1}{(11.335,2.760)}
\gppoint{gp mark 1}{(11.338,2.935)}
\gppoint{gp mark 1}{(11.340,2.689)}
\gppoint{gp mark 1}{(11.343,3.050)}
\gppoint{gp mark 1}{(11.345,2.504)}
\gppoint{gp mark 1}{(11.348,2.575)}
\gppoint{gp mark 1}{(11.350,2.187)}
\gppoint{gp mark 1}{(11.353,2.957)}
\gppoint{gp mark 1}{(11.355,1.886)}
\gppoint{gp mark 1}{(11.358,2.607)}
\gppoint{gp mark 1}{(11.360,2.263)}
\gppoint{gp mark 1}{(11.363,3.755)}
\gppoint{gp mark 1}{(11.365,2.728)}
\gppoint{gp mark 1}{(11.367,2.411)}
\gppoint{gp mark 1}{(11.370,2.493)}
\gppoint{gp mark 1}{(11.372,2.596)}
\gppoint{gp mark 1}{(11.375,3.197)}
\gppoint{gp mark 1}{(11.377,2.722)}
\gppoint{gp mark 1}{(11.380,3.023)}
\gppoint{gp mark 1}{(11.382,1.870)}
\gppoint{gp mark 1}{(11.385,2.646)}
\gppoint{gp mark 1}{(11.387,2.241)}
\gppoint{gp mark 1}{(11.390,2.487)}
\gppoint{gp mark 1}{(11.392,2.121)}
\gppoint{gp mark 1}{(11.395,1.957)}
\gppoint{gp mark 1}{(11.397,2.498)}
\gppoint{gp mark 1}{(11.400,2.061)}
\gppoint{gp mark 1}{(11.402,2.689)}
\gppoint{gp mark 1}{(11.405,2.586)}
\gppoint{gp mark 1}{(11.407,1.941)}
\gppoint{gp mark 1}{(11.410,2.875)}
\gppoint{gp mark 1}{(11.412,2.236)}
\gppoint{gp mark 1}{(11.415,3.247)}
\gppoint{gp mark 1}{(11.417,2.717)}
\gppoint{gp mark 1}{(11.420,2.411)}
\gppoint{gp mark 1}{(11.422,2.230)}
\gppoint{gp mark 1}{(11.425,2.815)}
\gppoint{gp mark 1}{(11.427,2.356)}
\gppoint{gp mark 1}{(11.430,2.542)}
\gppoint{gp mark 1}{(11.432,3.066)}
\gppoint{gp mark 1}{(11.435,2.881)}
\gppoint{gp mark 1}{(11.437,2.804)}
\gppoint{gp mark 1}{(11.440,2.181)}
\gppoint{gp mark 1}{(11.442,2.307)}
\gppoint{gp mark 1}{(11.445,2.454)}
\gppoint{gp mark 1}{(11.447,2.678)}
\draw[gp path] (1.504,7.691)--(1.504,0.985)--(11.447,0.985)--(11.447,7.691)--cycle;
%% coordinates of the plot area
\gpdefrectangularnode{gp plot 1}{\pgfpoint{1.504cm}{0.985cm}}{\pgfpoint{11.447cm}{7.691cm}}
\end{tikzpicture}
%% gnuplot variables

\caption{常温黒体の測定}
\label{fig:allan_raw}
\end{center}
\end{figure}
\begin{figure}[hptb]
\begin{center}
\begin{tikzpicture}[gnuplot]
%% generated with GNUPLOT 5.2p8 (Lua 5.3; terminal rev. Nov 2018, script rev. 108)
%% 2021年05月13日 16時30分31秒
\path (0.000,0.000) rectangle (12.000,8.000);
\gpcolor{color=gp lt color border}
\gpsetlinetype{gp lt border}
\gpsetdashtype{gp dt solid}
\gpsetlinewidth{1.00}
\draw[gp path] (1.504,0.985)--(1.684,0.985);
\draw[gp path] (11.447,0.985)--(11.267,0.985);
\node[gp node right] at (1.320,0.985) {$10^{-7}$};
\draw[gp path] (1.504,1.658)--(1.594,1.658);
\draw[gp path] (11.447,1.658)--(11.357,1.658);
\draw[gp path] (1.504,2.052)--(1.594,2.052);
\draw[gp path] (11.447,2.052)--(11.357,2.052);
\draw[gp path] (1.504,2.331)--(1.594,2.331);
\draw[gp path] (11.447,2.331)--(11.357,2.331);
\draw[gp path] (1.504,2.547)--(1.594,2.547);
\draw[gp path] (11.447,2.547)--(11.357,2.547);
\draw[gp path] (1.504,2.724)--(1.594,2.724);
\draw[gp path] (11.447,2.724)--(11.357,2.724);
\draw[gp path] (1.504,2.874)--(1.594,2.874);
\draw[gp path] (11.447,2.874)--(11.357,2.874);
\draw[gp path] (1.504,3.004)--(1.594,3.004);
\draw[gp path] (11.447,3.004)--(11.357,3.004);
\draw[gp path] (1.504,3.118)--(1.594,3.118);
\draw[gp path] (11.447,3.118)--(11.357,3.118);
\draw[gp path] (1.504,3.220)--(1.684,3.220);
\draw[gp path] (11.447,3.220)--(11.267,3.220);
\node[gp node right] at (1.320,3.220) {$10^{-6}$};
\draw[gp path] (1.504,3.893)--(1.594,3.893);
\draw[gp path] (11.447,3.893)--(11.357,3.893);
\draw[gp path] (1.504,4.287)--(1.594,4.287);
\draw[gp path] (11.447,4.287)--(11.357,4.287);
\draw[gp path] (1.504,4.566)--(1.594,4.566);
\draw[gp path] (11.447,4.566)--(11.357,4.566);
\draw[gp path] (1.504,4.783)--(1.594,4.783);
\draw[gp path] (11.447,4.783)--(11.357,4.783);
\draw[gp path] (1.504,4.960)--(1.594,4.960);
\draw[gp path] (11.447,4.960)--(11.357,4.960);
\draw[gp path] (1.504,5.109)--(1.594,5.109);
\draw[gp path] (11.447,5.109)--(11.357,5.109);
\draw[gp path] (1.504,5.239)--(1.594,5.239);
\draw[gp path] (11.447,5.239)--(11.357,5.239);
\draw[gp path] (1.504,5.353)--(1.594,5.353);
\draw[gp path] (11.447,5.353)--(11.357,5.353);
\draw[gp path] (1.504,5.456)--(1.684,5.456);
\draw[gp path] (11.447,5.456)--(11.267,5.456);
\node[gp node right] at (1.320,5.456) {$10^{-5}$};
\draw[gp path] (1.504,6.129)--(1.594,6.129);
\draw[gp path] (11.447,6.129)--(11.357,6.129);
\draw[gp path] (1.504,6.522)--(1.594,6.522);
\draw[gp path] (11.447,6.522)--(11.357,6.522);
\draw[gp path] (1.504,6.801)--(1.594,6.801);
\draw[gp path] (11.447,6.801)--(11.357,6.801);
\draw[gp path] (1.504,7.018)--(1.594,7.018);
\draw[gp path] (11.447,7.018)--(11.357,7.018);
\draw[gp path] (1.504,7.195)--(1.594,7.195);
\draw[gp path] (11.447,7.195)--(11.357,7.195);
\draw[gp path] (1.504,7.345)--(1.594,7.345);
\draw[gp path] (11.447,7.345)--(11.357,7.345);
\draw[gp path] (1.504,7.474)--(1.594,7.474);
\draw[gp path] (11.447,7.474)--(11.357,7.474);
\draw[gp path] (1.504,7.589)--(1.594,7.589);
\draw[gp path] (11.447,7.589)--(11.357,7.589);
\draw[gp path] (1.504,7.691)--(1.684,7.691);
\draw[gp path] (11.447,7.691)--(11.267,7.691);
\node[gp node right] at (1.320,7.691) {$10^{-4}$};
\draw[gp path] (1.504,0.985)--(1.504,1.165);
\draw[gp path] (1.504,7.691)--(1.504,7.511);
\node[gp node center] at (1.504,0.677) {$1$};
\draw[gp path] (2.502,0.985)--(2.502,1.075);
\draw[gp path] (2.502,7.691)--(2.502,7.601);
\draw[gp path] (3.085,0.985)--(3.085,1.075);
\draw[gp path] (3.085,7.691)--(3.085,7.601);
\draw[gp path] (3.499,0.985)--(3.499,1.075);
\draw[gp path] (3.499,7.691)--(3.499,7.601);
\draw[gp path] (3.821,0.985)--(3.821,1.075);
\draw[gp path] (3.821,7.691)--(3.821,7.601);
\draw[gp path] (4.083,0.985)--(4.083,1.075);
\draw[gp path] (4.083,7.691)--(4.083,7.601);
\draw[gp path] (4.305,0.985)--(4.305,1.075);
\draw[gp path] (4.305,7.691)--(4.305,7.601);
\draw[gp path] (4.497,0.985)--(4.497,1.075);
\draw[gp path] (4.497,7.691)--(4.497,7.601);
\draw[gp path] (4.667,0.985)--(4.667,1.075);
\draw[gp path] (4.667,7.691)--(4.667,7.601);
\draw[gp path] (4.818,0.985)--(4.818,1.165);
\draw[gp path] (4.818,7.691)--(4.818,7.511);
\node[gp node center] at (4.818,0.677) {$10$};
\draw[gp path] (5.816,0.985)--(5.816,1.075);
\draw[gp path] (5.816,7.691)--(5.816,7.601);
\draw[gp path] (6.400,0.985)--(6.400,1.075);
\draw[gp path] (6.400,7.691)--(6.400,7.601);
\draw[gp path] (6.814,0.985)--(6.814,1.075);
\draw[gp path] (6.814,7.691)--(6.814,7.601);
\draw[gp path] (7.135,0.985)--(7.135,1.075);
\draw[gp path] (7.135,7.691)--(7.135,7.601);
\draw[gp path] (7.397,0.985)--(7.397,1.075);
\draw[gp path] (7.397,7.691)--(7.397,7.601);
\draw[gp path] (7.619,0.985)--(7.619,1.075);
\draw[gp path] (7.619,7.691)--(7.619,7.601);
\draw[gp path] (7.811,0.985)--(7.811,1.075);
\draw[gp path] (7.811,7.691)--(7.811,7.601);
\draw[gp path] (7.981,0.985)--(7.981,1.075);
\draw[gp path] (7.981,7.691)--(7.981,7.601);
\draw[gp path] (8.133,0.985)--(8.133,1.165);
\draw[gp path] (8.133,7.691)--(8.133,7.511);
\node[gp node center] at (8.133,0.677) {$100$};
\draw[gp path] (9.130,0.985)--(9.130,1.075);
\draw[gp path] (9.130,7.691)--(9.130,7.601);
\draw[gp path] (9.714,0.985)--(9.714,1.075);
\draw[gp path] (9.714,7.691)--(9.714,7.601);
\draw[gp path] (10.128,0.985)--(10.128,1.075);
\draw[gp path] (10.128,7.691)--(10.128,7.601);
\draw[gp path] (10.449,0.985)--(10.449,1.075);
\draw[gp path] (10.449,7.691)--(10.449,7.601);
\draw[gp path] (10.712,0.985)--(10.712,1.075);
\draw[gp path] (10.712,7.691)--(10.712,7.601);
\draw[gp path] (10.934,0.985)--(10.934,1.075);
\draw[gp path] (10.934,7.691)--(10.934,7.601);
\draw[gp path] (11.126,0.985)--(11.126,1.075);
\draw[gp path] (11.126,7.691)--(11.126,7.601);
\draw[gp path] (11.295,0.985)--(11.295,1.075);
\draw[gp path] (11.295,7.691)--(11.295,7.601);
\draw[gp path] (11.447,0.985)--(11.447,1.165);
\draw[gp path] (11.447,7.691)--(11.447,7.511);
\node[gp node center] at (11.447,0.677) {$1000$};
\draw[gp path] (1.504,7.691)--(1.504,0.985)--(11.447,0.985)--(11.447,7.691)--cycle;
\node[gp node left] at (7.373,2.052) {$(59,3.60\times10^{-7})$};
\node[gp node center,rotate=-270] at (0.292,4.338) {$\sigma_A^2$ / $\si{\volt^2}$};
\node[gp node center] at (6.475,0.215) {$T$ / $\si{\second}$};
\gpsetpointsize{4.00}
\gppoint{gp mark 1}{(1.504,5.582)}
\gppoint{gp mark 1}{(2.502,4.883)}
\gppoint{gp mark 1}{(3.085,4.453)}
\gppoint{gp mark 1}{(3.499,4.235)}
\gppoint{gp mark 1}{(3.821,3.988)}
\gppoint{gp mark 1}{(4.083,3.883)}
\gppoint{gp mark 1}{(4.305,3.765)}
\gppoint{gp mark 1}{(4.497,3.620)}
\gppoint{gp mark 1}{(4.667,3.555)}
\gppoint{gp mark 1}{(4.818,3.519)}
\gppoint{gp mark 1}{(4.956,3.394)}
\gppoint{gp mark 1}{(5.081,3.345)}
\gppoint{gp mark 1}{(5.196,3.284)}
\gppoint{gp mark 1}{(5.303,3.325)}
\gppoint{gp mark 1}{(5.402,3.273)}
\gppoint{gp mark 1}{(5.495,3.173)}
\gppoint{gp mark 1}{(5.582,3.068)}
\gppoint{gp mark 1}{(5.664,3.234)}
\gppoint{gp mark 1}{(5.742,3.048)}
\gppoint{gp mark 1}{(5.816,3.041)}
\gppoint{gp mark 1}{(5.886,3.045)}
\gppoint{gp mark 1}{(5.953,2.804)}
\gppoint{gp mark 1}{(6.017,2.940)}
\gppoint{gp mark 1}{(6.078,3.101)}
\gppoint{gp mark 1}{(6.137,2.729)}
\gppoint{gp mark 1}{(6.194,2.888)}
\gppoint{gp mark 1}{(6.248,2.700)}
\gppoint{gp mark 1}{(6.300,3.014)}
\gppoint{gp mark 1}{(6.351,2.742)}
\gppoint{gp mark 1}{(6.400,2.934)}
\gppoint{gp mark 1}{(6.447,2.730)}
\gppoint{gp mark 1}{(6.493,2.610)}
\gppoint{gp mark 1}{(6.537,3.003)}
\gppoint{gp mark 1}{(6.580,2.765)}
\gppoint{gp mark 1}{(6.622,2.461)}
\gppoint{gp mark 1}{(6.662,2.957)}
\gppoint{gp mark 1}{(6.702,2.770)}
\gppoint{gp mark 1}{(6.740,2.619)}
\gppoint{gp mark 1}{(6.777,2.490)}
\gppoint{gp mark 1}{(6.814,2.682)}
\gppoint{gp mark 1}{(6.849,2.875)}
\gppoint{gp mark 1}{(6.884,2.692)}
\gppoint{gp mark 1}{(6.918,2.599)}
\gppoint{gp mark 1}{(6.951,2.409)}
\gppoint{gp mark 1}{(6.983,2.689)}
\gppoint{gp mark 1}{(7.015,2.877)}
\gppoint{gp mark 1}{(7.046,2.672)}
\gppoint{gp mark 1}{(7.076,2.834)}
\gppoint{gp mark 1}{(7.106,2.576)}
\gppoint{gp mark 1}{(7.135,2.426)}
\gppoint{gp mark 1}{(7.163,2.376)}
\gppoint{gp mark 1}{(7.191,2.650)}
\gppoint{gp mark 1}{(7.219,2.782)}
\gppoint{gp mark 1}{(7.246,2.917)}
\gppoint{gp mark 1}{(7.272,2.807)}
\gppoint{gp mark 1}{(7.298,2.865)}
\gppoint{gp mark 1}{(7.324,2.559)}
\gppoint{gp mark 1}{(7.349,2.543)}
\gppoint{gp mark 1}{(7.373,2.228)}
\gppoint{gp mark 1}{(7.397,2.571)}
\gppoint{gp mark 1}{(7.421,2.508)}
\gppoint{gp mark 1}{(7.445,2.524)}
\gppoint{gp mark 1}{(7.468,2.762)}
\gppoint{gp mark 1}{(7.490,2.766)}
\gppoint{gp mark 1}{(7.513,2.894)}
\gppoint{gp mark 1}{(7.535,2.853)}
\gppoint{gp mark 1}{(7.556,2.800)}
\gppoint{gp mark 1}{(7.578,2.602)}
\gppoint{gp mark 1}{(7.599,2.706)}
\gppoint{gp mark 1}{(7.619,2.347)}
\gppoint{gp mark 1}{(7.640,2.428)}
\gppoint{gp mark 1}{(7.660,2.536)}
\gppoint{gp mark 1}{(7.680,2.616)}
\gppoint{gp mark 1}{(7.699,2.619)}
\gppoint{gp mark 1}{(7.719,2.652)}
\gppoint{gp mark 1}{(7.738,2.644)}
\gppoint{gp mark 1}{(7.756,2.724)}
\gppoint{gp mark 1}{(7.775,2.714)}
\gppoint{gp mark 1}{(7.793,2.843)}
\gppoint{gp mark 1}{(7.811,2.735)}
\gppoint{gp mark 1}{(7.829,2.880)}
\gppoint{gp mark 1}{(7.847,2.822)}
\gppoint{gp mark 1}{(7.864,2.773)}
\gppoint{gp mark 1}{(7.882,2.880)}
\gppoint{gp mark 1}{(7.899,2.680)}
\gppoint{gp mark 1}{(7.916,2.729)}
\gppoint{gp mark 1}{(7.932,2.584)}
\gppoint{gp mark 1}{(7.949,2.472)}
\gppoint{gp mark 1}{(7.965,2.427)}
\gppoint{gp mark 1}{(7.981,2.513)}
\gppoint{gp mark 1}{(7.997,2.650)}
\gppoint{gp mark 1}{(8.013,2.649)}
\gppoint{gp mark 1}{(8.028,2.659)}
\gppoint{gp mark 1}{(8.044,2.708)}
\gppoint{gp mark 1}{(8.059,2.754)}
\gppoint{gp mark 1}{(8.074,2.857)}
\gppoint{gp mark 1}{(8.089,2.779)}
\gppoint{gp mark 1}{(8.104,2.905)}
\gppoint{gp mark 1}{(8.118,2.843)}
\gppoint{gp mark 1}{(8.133,2.763)}
\gppoint{gp mark 1}{(8.147,2.902)}
\gppoint{gp mark 1}{(8.161,2.907)}
\gppoint{gp mark 1}{(8.175,2.862)}
\gppoint{gp mark 1}{(8.189,2.942)}
\gppoint{gp mark 1}{(8.203,3.032)}
\gppoint{gp mark 1}{(8.217,3.042)}
\gppoint{gp mark 1}{(8.230,3.003)}
\gppoint{gp mark 1}{(8.243,3.078)}
\gppoint{gp mark 1}{(8.257,3.055)}
\gppoint{gp mark 1}{(8.270,3.051)}
\gppoint{gp mark 1}{(8.283,3.059)}
\gppoint{gp mark 1}{(8.296,3.151)}
\gppoint{gp mark 1}{(8.309,3.016)}
\gppoint{gp mark 1}{(8.321,2.958)}
\gppoint{gp mark 1}{(8.334,2.971)}
\gppoint{gp mark 1}{(8.346,2.887)}
\gppoint{gp mark 1}{(8.359,2.753)}
\gppoint{gp mark 1}{(8.371,2.778)}
\gppoint{gp mark 1}{(8.383,2.834)}
\gppoint{gp mark 1}{(8.395,2.776)}
\gppoint{gp mark 1}{(8.407,2.745)}
\gppoint{gp mark 1}{(8.419,2.794)}
\gppoint{gp mark 1}{(8.431,2.892)}
\gppoint{gp mark 1}{(8.442,2.873)}
\gppoint{gp mark 1}{(8.454,2.866)}
\gppoint{gp mark 1}{(8.465,2.864)}
\gppoint{gp mark 1}{(8.477,2.911)}
\gppoint{gp mark 1}{(8.488,2.956)}
\gppoint{gp mark 1}{(8.499,2.935)}
\gppoint{gp mark 1}{(8.510,2.896)}
\gppoint{gp mark 1}{(8.521,2.910)}
\gppoint{gp mark 1}{(8.532,2.953)}
\gppoint{gp mark 1}{(8.543,3.043)}
\gppoint{gp mark 1}{(8.554,3.009)}
\gppoint{gp mark 1}{(8.565,3.006)}
\gppoint{gp mark 1}{(8.575,2.971)}
\gppoint{gp mark 1}{(8.586,2.965)}
\gppoint{gp mark 1}{(8.596,3.054)}
\gppoint{gp mark 1}{(8.607,3.087)}
\gppoint{gp mark 1}{(8.617,3.114)}
\gppoint{gp mark 1}{(8.627,3.138)}
\gppoint{gp mark 1}{(8.637,3.172)}
\gppoint{gp mark 1}{(8.648,3.184)}
\gppoint{gp mark 1}{(8.658,3.251)}
\gppoint{gp mark 1}{(8.667,3.252)}
\gppoint{gp mark 1}{(8.677,3.198)}
\gppoint{gp mark 1}{(8.687,3.224)}
\gppoint{gp mark 1}{(8.697,3.242)}
\gppoint{gp mark 1}{(8.707,3.263)}
\gppoint{gp mark 1}{(8.716,3.277)}
\gppoint{gp mark 1}{(8.726,3.299)}
\gppoint{gp mark 1}{(8.735,3.248)}
\gppoint{gp mark 1}{(8.745,3.247)}
\gppoint{gp mark 1}{(8.754,3.311)}
\gppoint{gp mark 1}{(8.763,3.332)}
\gppoint{gp mark 1}{(8.773,3.331)}
\gppoint{gp mark 1}{(8.782,3.344)}
\gppoint{gp mark 1}{(8.791,3.380)}
\gppoint{gp mark 1}{(8.800,3.386)}
\gppoint{gp mark 1}{(8.809,3.367)}
\gppoint{gp mark 1}{(8.818,3.422)}
\gppoint{gp mark 1}{(8.827,3.421)}
\gppoint{gp mark 1}{(8.836,3.406)}
\gppoint{gp mark 1}{(8.845,3.417)}
\gppoint{gp mark 1}{(8.853,3.398)}
\gppoint{gp mark 1}{(8.862,3.375)}
\gppoint{gp mark 1}{(8.871,3.416)}
\gppoint{gp mark 1}{(8.879,3.440)}
\gppoint{gp mark 1}{(8.888,3.380)}
\gppoint{gp mark 1}{(8.896,3.333)}
\gppoint{gp mark 1}{(8.905,3.319)}
\gppoint{gp mark 1}{(8.913,3.334)}
\gppoint{gp mark 1}{(8.922,3.335)}
\gppoint{gp mark 1}{(8.930,3.343)}
\gppoint{gp mark 1}{(8.938,3.290)}
\gppoint{gp mark 1}{(8.946,3.227)}
\gppoint{gp mark 1}{(8.955,3.220)}
\gppoint{gp mark 1}{(8.963,3.176)}
\gppoint{gp mark 1}{(8.971,3.167)}
\gppoint{gp mark 1}{(8.979,3.154)}
\gppoint{gp mark 1}{(8.987,3.176)}
\gppoint{gp mark 1}{(8.995,3.227)}
\gppoint{gp mark 1}{(9.003,3.222)}
\gppoint{gp mark 1}{(9.010,3.207)}
\gppoint{gp mark 1}{(9.018,3.198)}
\gppoint{gp mark 1}{(9.026,3.184)}
\gppoint{gp mark 1}{(9.034,3.193)}
\gppoint{gp mark 1}{(9.041,3.186)}
\gppoint{gp mark 1}{(9.049,3.215)}
\gppoint{gp mark 1}{(9.057,3.201)}
\gppoint{gp mark 1}{(9.064,3.255)}
\gppoint{gp mark 1}{(9.072,3.278)}
\gppoint{gp mark 1}{(9.079,3.271)}
\gppoint{gp mark 1}{(9.087,3.274)}
\gppoint{gp mark 1}{(9.094,3.266)}
\gppoint{gp mark 1}{(9.101,3.264)}
\gppoint{gp mark 1}{(9.109,3.245)}
\gppoint{gp mark 1}{(9.116,3.249)}
\gppoint{gp mark 1}{(9.123,3.245)}
\gppoint{gp mark 1}{(9.130,3.253)}
\gppoint{gp mark 1}{(9.138,3.312)}
\gppoint{gp mark 1}{(9.145,3.332)}
\gppoint{gp mark 1}{(9.152,3.349)}
\gppoint{gp mark 1}{(9.159,3.361)}
\gppoint{gp mark 1}{(9.166,3.359)}
\gppoint{gp mark 1}{(9.173,3.356)}
\gppoint{gp mark 1}{(9.180,3.364)}
\gppoint{gp mark 1}{(9.187,3.374)}
\gppoint{gp mark 1}{(9.194,3.370)}
\gppoint{gp mark 1}{(9.201,3.365)}
\gppoint{gp mark 1}{(9.207,3.421)}
\gppoint{gp mark 1}{(9.214,3.427)}
\gppoint{gp mark 1}{(9.221,3.442)}
\gppoint{gp mark 1}{(9.228,3.447)}
\gppoint{gp mark 1}{(9.234,3.460)}
\gppoint{gp mark 1}{(9.241,3.471)}
\gppoint{gp mark 1}{(9.248,3.465)}
\gppoint{gp mark 1}{(9.254,3.458)}
\gppoint{gp mark 1}{(9.261,3.450)}
\gppoint{gp mark 1}{(9.268,3.448)}
\gppoint{gp mark 1}{(9.274,3.457)}
\gppoint{gp mark 1}{(9.281,3.463)}
\gppoint{gp mark 1}{(9.287,3.509)}
\gppoint{gp mark 1}{(9.294,3.516)}
\gppoint{gp mark 1}{(9.300,3.513)}
\gppoint{gp mark 1}{(9.306,3.519)}
\gppoint{gp mark 1}{(9.313,3.525)}
\gppoint{gp mark 1}{(9.319,3.536)}
\gppoint{gp mark 1}{(9.325,3.566)}
\gppoint{gp mark 1}{(9.332,3.580)}
\gppoint{gp mark 1}{(9.338,3.586)}
\gppoint{gp mark 1}{(9.344,3.581)}
\gppoint{gp mark 1}{(9.350,3.573)}
\gppoint{gp mark 1}{(9.356,3.578)}
\gppoint{gp mark 1}{(9.363,3.588)}
\gppoint{gp mark 1}{(9.369,3.657)}
\gppoint{gp mark 1}{(9.375,3.681)}
\gppoint{gp mark 1}{(9.381,3.693)}
\gppoint{gp mark 1}{(9.387,3.706)}
\gppoint{gp mark 1}{(9.393,3.694)}
\gppoint{gp mark 1}{(9.399,3.708)}
\gppoint{gp mark 1}{(9.405,3.708)}
\gppoint{gp mark 1}{(9.411,3.718)}
\gppoint{gp mark 1}{(9.417,3.710)}
\gppoint{gp mark 1}{(9.422,3.720)}
\gppoint{gp mark 1}{(9.428,3.734)}
\gppoint{gp mark 1}{(9.434,3.746)}
\gppoint{gp mark 1}{(9.440,3.734)}
\gppoint{gp mark 1}{(9.446,3.728)}
\gppoint{gp mark 1}{(9.452,3.724)}
\gppoint{gp mark 1}{(9.457,3.785)}
\gppoint{gp mark 1}{(9.463,3.794)}
\gppoint{gp mark 1}{(9.469,3.790)}
\gppoint{gp mark 1}{(9.474,3.792)}
\gppoint{gp mark 1}{(9.480,3.792)}
\gppoint{gp mark 1}{(9.486,3.792)}
\gppoint{gp mark 1}{(9.491,3.798)}
\gppoint{gp mark 1}{(9.497,3.804)}
\gppoint{gp mark 1}{(9.502,3.803)}
\gppoint{gp mark 1}{(9.508,3.810)}
\gppoint{gp mark 1}{(9.514,3.833)}
\gppoint{gp mark 1}{(9.519,3.837)}
\gppoint{gp mark 1}{(9.525,3.838)}
\gppoint{gp mark 1}{(9.530,3.848)}
\gppoint{gp mark 1}{(9.535,3.851)}
\gppoint{gp mark 1}{(9.541,3.864)}
\gppoint{gp mark 1}{(9.546,3.930)}
\gppoint{gp mark 1}{(9.552,3.928)}
\gppoint{gp mark 1}{(9.557,3.930)}
\gppoint{gp mark 1}{(9.562,3.929)}
\gppoint{gp mark 1}{(9.568,3.930)}
\gppoint{gp mark 1}{(9.573,3.935)}
\gppoint{gp mark 1}{(9.578,3.935)}
\gppoint{gp mark 1}{(9.584,3.944)}
\gppoint{gp mark 1}{(9.589,3.946)}
\gppoint{gp mark 1}{(9.594,3.946)}
\gppoint{gp mark 1}{(9.599,3.952)}
\gppoint{gp mark 1}{(9.604,3.956)}
\gppoint{gp mark 1}{(9.610,3.964)}
\gppoint{gp mark 1}{(9.615,3.973)}
\gppoint{gp mark 1}{(9.620,3.985)}
\gppoint{gp mark 1}{(9.625,3.993)}
\gppoint{gp mark 1}{(9.630,4.011)}
\gppoint{gp mark 1}{(9.635,4.015)}
\gppoint{gp mark 1}{(9.640,4.020)}
\gppoint{gp mark 1}{(9.645,4.089)}
\gppoint{gp mark 1}{(9.650,4.087)}
\gppoint{gp mark 1}{(9.655,4.098)}
\gppoint{gp mark 1}{(9.660,4.104)}
\gppoint{gp mark 1}{(9.665,4.106)}
\gppoint{gp mark 1}{(9.670,4.104)}
\gppoint{gp mark 1}{(9.675,4.107)}
\gppoint{gp mark 1}{(9.680,4.106)}
\gppoint{gp mark 1}{(9.685,4.106)}
\gppoint{gp mark 1}{(9.690,4.106)}
\gppoint{gp mark 1}{(9.695,4.110)}
\gppoint{gp mark 1}{(9.700,4.112)}
\gppoint{gp mark 1}{(9.704,4.108)}
\gppoint{gp mark 1}{(9.709,4.111)}
\gppoint{gp mark 1}{(9.714,4.111)}
\gppoint{gp mark 1}{(9.719,4.111)}
\gppoint{gp mark 1}{(9.724,4.118)}
\gppoint{gp mark 1}{(9.728,4.123)}
\gppoint{gp mark 1}{(9.733,4.126)}
\gppoint{gp mark 1}{(9.738,4.134)}
\gppoint{gp mark 1}{(9.743,4.139)}
\gppoint{gp mark 1}{(9.747,4.144)}
\gppoint{gp mark 1}{(9.752,4.235)}
\gppoint{gp mark 1}{(9.757,4.241)}
\gppoint{gp mark 1}{(9.761,4.242)}
\gppoint{gp mark 1}{(9.766,4.245)}
\gppoint{gp mark 1}{(9.770,4.251)}
\gppoint{gp mark 1}{(9.775,4.253)}
\gppoint{gp mark 1}{(9.780,4.257)}
\gppoint{gp mark 1}{(9.784,4.259)}
\gppoint{gp mark 1}{(9.789,4.261)}
\gppoint{gp mark 1}{(9.793,4.254)}
\gppoint{gp mark 1}{(9.798,4.256)}
\gppoint{gp mark 1}{(9.802,4.251)}
\gppoint{gp mark 1}{(9.807,4.252)}
\gppoint{gp mark 1}{(9.811,4.260)}
\gppoint{gp mark 1}{(9.816,4.264)}
\gppoint{gp mark 1}{(9.820,4.271)}
\gppoint{gp mark 1}{(9.825,4.270)}
\gppoint{gp mark 1}{(9.829,4.271)}
\gppoint{gp mark 1}{(9.834,4.276)}
\gppoint{gp mark 1}{(9.838,4.277)}
\gppoint{gp mark 1}{(9.842,4.274)}
\gppoint{gp mark 1}{(9.847,4.274)}
\gppoint{gp mark 1}{(9.851,4.275)}
\gppoint{gp mark 1}{(9.856,4.277)}
\gppoint{gp mark 1}{(9.860,4.276)}
\gppoint{gp mark 1}{(9.864,4.278)}
\gppoint{gp mark 1}{(9.869,4.371)}
\gppoint{gp mark 1}{(9.873,4.378)}
\gppoint{gp mark 1}{(9.877,4.382)}
\gppoint{gp mark 1}{(9.881,4.381)}
\gppoint{gp mark 1}{(9.886,4.381)}
\gppoint{gp mark 1}{(9.890,4.381)}
\gppoint{gp mark 1}{(9.894,4.387)}
\gppoint{gp mark 1}{(9.898,4.387)}
\gppoint{gp mark 1}{(9.903,4.393)}
\gppoint{gp mark 1}{(9.907,4.397)}
\gppoint{gp mark 1}{(9.911,4.403)}
\gppoint{gp mark 1}{(9.915,4.406)}
\gppoint{gp mark 1}{(9.919,4.405)}
\gppoint{gp mark 1}{(9.923,4.403)}
\gppoint{gp mark 1}{(9.928,4.398)}
\gppoint{gp mark 1}{(9.932,4.391)}
\gppoint{gp mark 1}{(9.936,4.390)}
\gppoint{gp mark 1}{(9.940,4.385)}
\gppoint{gp mark 1}{(9.944,4.384)}
\gppoint{gp mark 1}{(9.948,4.384)}
\gppoint{gp mark 1}{(9.952,4.385)}
\gppoint{gp mark 1}{(9.956,4.386)}
\gppoint{gp mark 1}{(9.960,4.383)}
\gppoint{gp mark 1}{(9.964,4.383)}
\gppoint{gp mark 1}{(9.968,4.385)}
\gppoint{gp mark 1}{(9.972,4.386)}
\gppoint{gp mark 1}{(9.976,4.388)}
\gppoint{gp mark 1}{(9.980,4.392)}
\gppoint{gp mark 1}{(9.984,4.396)}
\gppoint{gp mark 1}{(9.988,4.396)}
\gppoint{gp mark 1}{(9.992,4.498)}
\gppoint{gp mark 1}{(9.996,4.500)}
\gppoint{gp mark 1}{(10.000,4.502)}
\gppoint{gp mark 1}{(10.004,4.506)}
\gppoint{gp mark 1}{(10.008,4.504)}
\gppoint{gp mark 1}{(10.012,4.503)}
\gppoint{gp mark 1}{(10.016,4.505)}
\gppoint{gp mark 1}{(10.020,4.506)}
\gppoint{gp mark 1}{(10.024,4.512)}
\gppoint{gp mark 1}{(10.028,4.513)}
\gppoint{gp mark 1}{(10.031,4.514)}
\gppoint{gp mark 1}{(10.035,4.517)}
\gppoint{gp mark 1}{(10.039,4.521)}
\gppoint{gp mark 1}{(10.043,4.528)}
\gppoint{gp mark 1}{(10.047,4.531)}
\gppoint{gp mark 1}{(10.050,4.535)}
\gppoint{gp mark 1}{(10.054,4.537)}
\gppoint{gp mark 1}{(10.058,4.542)}
\gppoint{gp mark 1}{(10.062,4.547)}
\gppoint{gp mark 1}{(10.066,4.553)}
\gppoint{gp mark 1}{(10.069,4.558)}
\gppoint{gp mark 1}{(10.073,4.560)}
\gppoint{gp mark 1}{(10.077,4.558)}
\gppoint{gp mark 1}{(10.081,4.556)}
\gppoint{gp mark 1}{(10.084,4.555)}
\gppoint{gp mark 1}{(10.088,4.558)}
\gppoint{gp mark 1}{(10.092,4.555)}
\gppoint{gp mark 1}{(10.095,4.558)}
\gppoint{gp mark 1}{(10.099,4.560)}
\gppoint{gp mark 1}{(10.103,4.562)}
\gppoint{gp mark 1}{(10.106,4.564)}
\gppoint{gp mark 1}{(10.110,4.567)}
\gppoint{gp mark 1}{(10.114,4.573)}
\gppoint{gp mark 1}{(10.117,4.573)}
\gppoint{gp mark 1}{(10.121,4.578)}
\gppoint{gp mark 1}{(10.124,4.581)}
\gppoint{gp mark 1}{(10.128,4.580)}
\gppoint{gp mark 1}{(10.132,4.695)}
\gppoint{gp mark 1}{(10.135,4.701)}
\gppoint{gp mark 1}{(10.139,4.705)}
\gppoint{gp mark 1}{(10.142,4.708)}
\gppoint{gp mark 1}{(10.146,4.714)}
\gppoint{gp mark 1}{(10.150,4.718)}
\gppoint{gp mark 1}{(10.153,4.718)}
\gppoint{gp mark 1}{(10.157,4.719)}
\gppoint{gp mark 1}{(10.160,4.721)}
\gppoint{gp mark 1}{(10.164,4.720)}
\gppoint{gp mark 1}{(10.167,4.722)}
\gppoint{gp mark 1}{(10.171,4.723)}
\gppoint{gp mark 1}{(10.174,4.729)}
\gppoint{gp mark 1}{(10.178,4.729)}
\gppoint{gp mark 1}{(10.181,4.731)}
\gppoint{gp mark 1}{(10.185,4.734)}
\gppoint{gp mark 1}{(10.188,4.737)}
\gppoint{gp mark 1}{(10.191,4.739)}
\gppoint{gp mark 1}{(10.195,4.742)}
\gppoint{gp mark 1}{(10.198,4.744)}
\gppoint{gp mark 1}{(10.202,4.748)}
\gppoint{gp mark 1}{(10.205,4.750)}
\gppoint{gp mark 1}{(10.209,4.753)}
\gppoint{gp mark 1}{(10.212,4.757)}
\gppoint{gp mark 1}{(10.215,4.760)}
\gppoint{gp mark 1}{(10.219,4.761)}
\gppoint{gp mark 1}{(10.222,4.764)}
\gppoint{gp mark 1}{(10.225,4.765)}
\gppoint{gp mark 1}{(10.229,4.768)}
\gppoint{gp mark 1}{(10.232,4.775)}
\gppoint{gp mark 1}{(10.236,4.779)}
\gppoint{gp mark 1}{(10.239,4.782)}
\gppoint{gp mark 1}{(10.242,4.783)}
\gppoint{gp mark 1}{(10.246,4.785)}
\gppoint{gp mark 1}{(10.249,4.786)}
\gppoint{gp mark 1}{(10.252,4.785)}
\gppoint{gp mark 1}{(10.255,4.786)}
\gppoint{gp mark 1}{(10.259,4.786)}
\gppoint{gp mark 1}{(10.262,4.789)}
\gppoint{gp mark 1}{(10.265,4.788)}
\gppoint{gp mark 1}{(10.269,4.791)}
\gppoint{gp mark 1}{(10.272,4.792)}
\gppoint{gp mark 1}{(10.275,4.793)}
\gppoint{gp mark 1}{(10.278,4.796)}
\gppoint{gp mark 1}{(10.282,4.926)}
\gppoint{gp mark 1}{(10.285,4.925)}
\gppoint{gp mark 1}{(10.288,4.925)}
\gppoint{gp mark 1}{(10.291,4.925)}
\gppoint{gp mark 1}{(10.294,4.925)}
\gppoint{gp mark 1}{(10.298,4.926)}
\gppoint{gp mark 1}{(10.301,4.930)}
\gppoint{gp mark 1}{(10.304,4.932)}
\gppoint{gp mark 1}{(10.307,4.934)}
\gppoint{gp mark 1}{(10.310,4.935)}
\gppoint{gp mark 1}{(10.314,4.937)}
\gppoint{gp mark 1}{(10.317,4.939)}
\gppoint{gp mark 1}{(10.320,4.943)}
\gppoint{gp mark 1}{(10.323,4.945)}
\gppoint{gp mark 1}{(10.326,4.948)}
\gppoint{gp mark 1}{(10.329,4.950)}
\gppoint{gp mark 1}{(10.332,4.951)}
\gppoint{gp mark 1}{(10.336,4.954)}
\gppoint{gp mark 1}{(10.339,4.954)}
\gppoint{gp mark 1}{(10.342,4.958)}
\gppoint{gp mark 1}{(10.345,4.961)}
\gppoint{gp mark 1}{(10.348,4.961)}
\gppoint{gp mark 1}{(10.351,4.964)}
\gppoint{gp mark 1}{(10.354,4.967)}
\gppoint{gp mark 1}{(10.357,4.968)}
\gppoint{gp mark 1}{(10.360,4.969)}
\gppoint{gp mark 1}{(10.363,4.972)}
\gppoint{gp mark 1}{(10.366,4.974)}
\gppoint{gp mark 1}{(10.369,4.977)}
\gppoint{gp mark 1}{(10.372,4.978)}
\gppoint{gp mark 1}{(10.375,4.980)}
\gppoint{gp mark 1}{(10.378,4.983)}
\gppoint{gp mark 1}{(10.382,4.984)}
\gppoint{gp mark 1}{(10.385,4.983)}
\gppoint{gp mark 1}{(10.388,4.986)}
\gppoint{gp mark 1}{(10.391,4.985)}
\gppoint{gp mark 1}{(10.394,4.988)}
\gppoint{gp mark 1}{(10.397,4.990)}
\gppoint{gp mark 1}{(10.399,4.992)}
\gppoint{gp mark 1}{(10.402,4.995)}
\gppoint{gp mark 1}{(10.405,4.998)}
\gppoint{gp mark 1}{(10.408,4.999)}
\gppoint{gp mark 1}{(10.411,5.002)}
\gppoint{gp mark 1}{(10.414,5.001)}
\gppoint{gp mark 1}{(10.417,5.003)}
\gppoint{gp mark 1}{(10.420,5.007)}
\gppoint{gp mark 1}{(10.423,5.009)}
\gppoint{gp mark 1}{(10.426,5.011)}
\gppoint{gp mark 1}{(10.429,5.014)}
\gppoint{gp mark 1}{(10.432,5.016)}
\gppoint{gp mark 1}{(10.435,5.019)}
\gppoint{gp mark 1}{(10.438,5.020)}
\gppoint{gp mark 1}{(10.441,5.019)}
\gppoint{gp mark 1}{(10.444,5.019)}
\gppoint{gp mark 1}{(10.446,5.021)}
\gppoint{gp mark 1}{(10.449,5.021)}
\gppoint{gp mark 1}{(10.452,5.170)}
\gppoint{gp mark 1}{(10.455,5.173)}
\gppoint{gp mark 1}{(10.458,5.173)}
\gppoint{gp mark 1}{(10.461,5.175)}
\gppoint{gp mark 1}{(10.464,5.176)}
\gppoint{gp mark 1}{(10.466,5.177)}
\gppoint{gp mark 1}{(10.469,5.180)}
\gppoint{gp mark 1}{(10.472,5.181)}
\gppoint{gp mark 1}{(10.475,5.183)}
\gppoint{gp mark 1}{(10.478,5.181)}
\gppoint{gp mark 1}{(10.481,5.184)}
\gppoint{gp mark 1}{(10.483,5.185)}
\gppoint{gp mark 1}{(10.486,5.187)}
\gppoint{gp mark 1}{(10.489,5.188)}
\gppoint{gp mark 1}{(10.492,5.190)}
\gppoint{gp mark 1}{(10.495,5.192)}
\gppoint{gp mark 1}{(10.497,5.194)}
\gppoint{gp mark 1}{(10.500,5.195)}
\gppoint{gp mark 1}{(10.503,5.195)}
\gppoint{gp mark 1}{(10.506,5.198)}
\gppoint{gp mark 1}{(10.509,5.200)}
\gppoint{gp mark 1}{(10.511,5.202)}
\gppoint{gp mark 1}{(10.514,5.204)}
\gppoint{gp mark 1}{(10.517,5.205)}
\gppoint{gp mark 1}{(10.520,5.208)}
\gppoint{gp mark 1}{(10.522,5.210)}
\gppoint{gp mark 1}{(10.525,5.214)}
\gppoint{gp mark 1}{(10.528,5.215)}
\gppoint{gp mark 1}{(10.530,5.217)}
\gppoint{gp mark 1}{(10.533,5.218)}
\gppoint{gp mark 1}{(10.536,5.220)}
\gppoint{gp mark 1}{(10.539,5.221)}
\gppoint{gp mark 1}{(10.541,5.222)}
\gppoint{gp mark 1}{(10.544,5.225)}
\gppoint{gp mark 1}{(10.547,5.227)}
\gppoint{gp mark 1}{(10.549,5.230)}
\gppoint{gp mark 1}{(10.552,5.231)}
\gppoint{gp mark 1}{(10.555,5.234)}
\gppoint{gp mark 1}{(10.557,5.235)}
\gppoint{gp mark 1}{(10.560,5.237)}
\gppoint{gp mark 1}{(10.563,5.237)}
\gppoint{gp mark 1}{(10.565,5.238)}
\gppoint{gp mark 1}{(10.568,5.241)}
\gppoint{gp mark 1}{(10.571,5.243)}
\gppoint{gp mark 1}{(10.573,5.245)}
\gppoint{gp mark 1}{(10.576,5.244)}
\gppoint{gp mark 1}{(10.579,5.244)}
\gppoint{gp mark 1}{(10.581,5.246)}
\gppoint{gp mark 1}{(10.584,5.247)}
\gppoint{gp mark 1}{(10.586,5.249)}
\gppoint{gp mark 1}{(10.589,5.251)}
\gppoint{gp mark 1}{(10.592,5.253)}
\gppoint{gp mark 1}{(10.594,5.255)}
\gppoint{gp mark 1}{(10.597,5.257)}
\gppoint{gp mark 1}{(10.600,5.258)}
\gppoint{gp mark 1}{(10.602,5.259)}
\gppoint{gp mark 1}{(10.605,5.261)}
\gppoint{gp mark 1}{(10.607,5.264)}
\gppoint{gp mark 1}{(10.610,5.267)}
\gppoint{gp mark 1}{(10.612,5.268)}
\gppoint{gp mark 1}{(10.615,5.271)}
\gppoint{gp mark 1}{(10.618,5.272)}
\gppoint{gp mark 1}{(10.620,5.274)}
\gppoint{gp mark 1}{(10.623,5.276)}
\gppoint{gp mark 1}{(10.625,5.279)}
\gppoint{gp mark 1}{(10.628,5.283)}
\gppoint{gp mark 1}{(10.630,5.287)}
\gppoint{gp mark 1}{(10.633,5.288)}
\gppoint{gp mark 1}{(10.635,5.289)}
\gppoint{gp mark 1}{(10.638,5.290)}
\gppoint{gp mark 1}{(10.640,5.291)}
\gppoint{gp mark 1}{(10.643,5.468)}
\gppoint{gp mark 1}{(10.645,5.471)}
\gppoint{gp mark 1}{(10.648,5.470)}
\gppoint{gp mark 1}{(10.650,5.474)}
\gppoint{gp mark 1}{(10.653,5.475)}
\gppoint{gp mark 1}{(10.655,5.477)}
\gppoint{gp mark 1}{(10.658,5.480)}
\gppoint{gp mark 1}{(10.660,5.478)}
\gppoint{gp mark 1}{(10.663,5.478)}
\gppoint{gp mark 1}{(10.665,5.477)}
\gppoint{gp mark 1}{(10.668,5.477)}
\gppoint{gp mark 1}{(10.670,5.478)}
\gppoint{gp mark 1}{(10.673,5.478)}
\gppoint{gp mark 1}{(10.675,5.479)}
\gppoint{gp mark 1}{(10.678,5.480)}
\gppoint{gp mark 1}{(10.680,5.482)}
\gppoint{gp mark 1}{(10.683,5.483)}
\gppoint{gp mark 1}{(10.685,5.484)}
\gppoint{gp mark 1}{(10.688,5.485)}
\gppoint{gp mark 1}{(10.690,5.488)}
\gppoint{gp mark 1}{(10.692,5.488)}
\gppoint{gp mark 1}{(10.695,5.489)}
\gppoint{gp mark 1}{(10.697,5.489)}
\gppoint{gp mark 1}{(10.700,5.490)}
\gppoint{gp mark 1}{(10.702,5.490)}
\gppoint{gp mark 1}{(10.705,5.492)}
\gppoint{gp mark 1}{(10.707,5.492)}
\gppoint{gp mark 1}{(10.709,5.495)}
\gppoint{gp mark 1}{(10.712,5.496)}
\gppoint{gp mark 1}{(10.714,5.499)}
\gppoint{gp mark 1}{(10.717,5.499)}
\gppoint{gp mark 1}{(10.719,5.500)}
\gppoint{gp mark 1}{(10.721,5.501)}
\gppoint{gp mark 1}{(10.724,5.502)}
\gppoint{gp mark 1}{(10.726,5.503)}
\gppoint{gp mark 1}{(10.728,5.506)}
\gppoint{gp mark 1}{(10.731,5.508)}
\gppoint{gp mark 1}{(10.733,5.510)}
\gppoint{gp mark 1}{(10.736,5.513)}
\gppoint{gp mark 1}{(10.738,5.512)}
\gppoint{gp mark 1}{(10.740,5.513)}
\gppoint{gp mark 1}{(10.743,5.514)}
\gppoint{gp mark 1}{(10.745,5.515)}
\gppoint{gp mark 1}{(10.747,5.517)}
\gppoint{gp mark 1}{(10.750,5.518)}
\gppoint{gp mark 1}{(10.752,5.518)}
\gppoint{gp mark 1}{(10.754,5.520)}
\gppoint{gp mark 1}{(10.757,5.521)}
\gppoint{gp mark 1}{(10.759,5.523)}
\gppoint{gp mark 1}{(10.761,5.525)}
\gppoint{gp mark 1}{(10.764,5.526)}
\gppoint{gp mark 1}{(10.766,5.527)}
\gppoint{gp mark 1}{(10.768,5.529)}
\gppoint{gp mark 1}{(10.770,5.530)}
\gppoint{gp mark 1}{(10.773,5.530)}
\gppoint{gp mark 1}{(10.775,5.531)}
\gppoint{gp mark 1}{(10.777,5.532)}
\gppoint{gp mark 1}{(10.780,5.533)}
\gppoint{gp mark 1}{(10.782,5.534)}
\gppoint{gp mark 1}{(10.784,5.535)}
\gppoint{gp mark 1}{(10.787,5.535)}
\gppoint{gp mark 1}{(10.789,5.537)}
\gppoint{gp mark 1}{(10.791,5.538)}
\gppoint{gp mark 1}{(10.793,5.539)}
\gppoint{gp mark 1}{(10.796,5.540)}
\gppoint{gp mark 1}{(10.798,5.541)}
\gppoint{gp mark 1}{(10.800,5.542)}
\gppoint{gp mark 1}{(10.802,5.542)}
\gppoint{gp mark 1}{(10.805,5.543)}
\gppoint{gp mark 1}{(10.807,5.545)}
\gppoint{gp mark 1}{(10.809,5.546)}
\gppoint{gp mark 1}{(10.811,5.546)}
\gppoint{gp mark 1}{(10.814,5.549)}
\gppoint{gp mark 1}{(10.816,5.551)}
\gppoint{gp mark 1}{(10.818,5.554)}
\gppoint{gp mark 1}{(10.820,5.555)}
\gppoint{gp mark 1}{(10.822,5.556)}
\gppoint{gp mark 1}{(10.825,5.558)}
\gppoint{gp mark 1}{(10.827,5.560)}
\gppoint{gp mark 1}{(10.829,5.562)}
\gppoint{gp mark 1}{(10.831,5.565)}
\gppoint{gp mark 1}{(10.834,5.565)}
\gppoint{gp mark 1}{(10.836,5.567)}
\gppoint{gp mark 1}{(10.838,5.568)}
\gppoint{gp mark 1}{(10.840,5.569)}
\gppoint{gp mark 1}{(10.842,5.571)}
\gppoint{gp mark 1}{(10.845,5.572)}
\gppoint{gp mark 1}{(10.847,5.573)}
\gppoint{gp mark 1}{(10.849,5.574)}
\gppoint{gp mark 1}{(10.851,5.575)}
\gppoint{gp mark 1}{(10.853,5.576)}
\gppoint{gp mark 1}{(10.855,5.577)}
\gppoint{gp mark 1}{(10.858,5.578)}
\gppoint{gp mark 1}{(10.860,5.579)}
\gppoint{gp mark 1}{(10.862,5.580)}
\gppoint{gp mark 1}{(10.864,5.787)}
\gppoint{gp mark 1}{(10.866,5.789)}
\gppoint{gp mark 1}{(10.868,5.791)}
\gppoint{gp mark 1}{(10.871,5.791)}
\gppoint{gp mark 1}{(10.873,5.792)}
\gppoint{gp mark 1}{(10.875,5.794)}
\gppoint{gp mark 1}{(10.877,5.797)}
\gppoint{gp mark 1}{(10.879,5.797)}
\gppoint{gp mark 1}{(10.881,5.800)}
\gppoint{gp mark 1}{(10.883,5.802)}
\gppoint{gp mark 1}{(10.886,5.804)}
\gppoint{gp mark 1}{(10.888,5.807)}
\gppoint{gp mark 1}{(10.890,5.810)}
\gppoint{gp mark 1}{(10.892,5.812)}
\gppoint{gp mark 1}{(10.894,5.814)}
\gppoint{gp mark 1}{(10.896,5.814)}
\gppoint{gp mark 1}{(10.898,5.817)}
\gppoint{gp mark 1}{(10.900,5.818)}
\gppoint{gp mark 1}{(10.902,5.819)}
\gppoint{gp mark 1}{(10.905,5.822)}
\gppoint{gp mark 1}{(10.907,5.824)}
\gppoint{gp mark 1}{(10.909,5.826)}
\gppoint{gp mark 1}{(10.911,5.829)}
\gppoint{gp mark 1}{(10.913,5.831)}
\gppoint{gp mark 1}{(10.915,5.832)}
\gppoint{gp mark 1}{(10.917,5.834)}
\gppoint{gp mark 1}{(10.919,5.835)}
\gppoint{gp mark 1}{(10.921,5.836)}
\gppoint{gp mark 1}{(10.923,5.836)}
\gppoint{gp mark 1}{(10.925,5.838)}
\gppoint{gp mark 1}{(10.927,5.839)}
\gppoint{gp mark 1}{(10.929,5.839)}
\gppoint{gp mark 1}{(10.932,5.840)}
\gppoint{gp mark 1}{(10.934,5.842)}
\gppoint{gp mark 1}{(10.936,5.842)}
\gppoint{gp mark 1}{(10.938,5.842)}
\gppoint{gp mark 1}{(10.940,5.845)}
\gppoint{gp mark 1}{(10.942,5.846)}
\gppoint{gp mark 1}{(10.944,5.847)}
\gppoint{gp mark 1}{(10.946,5.848)}
\gppoint{gp mark 1}{(10.948,5.849)}
\gppoint{gp mark 1}{(10.950,5.851)}
\gppoint{gp mark 1}{(10.952,5.853)}
\gppoint{gp mark 1}{(10.954,5.855)}
\gppoint{gp mark 1}{(10.956,5.854)}
\gppoint{gp mark 1}{(10.958,5.855)}
\gppoint{gp mark 1}{(10.960,5.857)}
\gppoint{gp mark 1}{(10.962,5.856)}
\gppoint{gp mark 1}{(10.964,5.856)}
\gppoint{gp mark 1}{(10.966,5.858)}
\gppoint{gp mark 1}{(10.968,5.858)}
\gppoint{gp mark 1}{(10.970,5.858)}
\gppoint{gp mark 1}{(10.972,5.860)}
\gppoint{gp mark 1}{(10.974,5.861)}
\gppoint{gp mark 1}{(10.976,5.862)}
\gppoint{gp mark 1}{(10.978,5.864)}
\gppoint{gp mark 1}{(10.980,5.866)}
\gppoint{gp mark 1}{(10.982,5.867)}
\gppoint{gp mark 1}{(10.984,5.868)}
\gppoint{gp mark 1}{(10.986,5.869)}
\gppoint{gp mark 1}{(10.988,5.869)}
\gppoint{gp mark 1}{(10.990,5.871)}
\gppoint{gp mark 1}{(10.992,5.871)}
\gppoint{gp mark 1}{(10.994,5.871)}
\gppoint{gp mark 1}{(10.996,5.871)}
\gppoint{gp mark 1}{(10.998,5.873)}
\gppoint{gp mark 1}{(11.000,5.873)}
\gppoint{gp mark 1}{(11.002,5.875)}
\gppoint{gp mark 1}{(11.004,5.874)}
\gppoint{gp mark 1}{(11.006,5.875)}
\gppoint{gp mark 1}{(11.008,5.875)}
\gppoint{gp mark 1}{(11.010,5.875)}
\gppoint{gp mark 1}{(11.012,5.876)}
\gppoint{gp mark 1}{(11.014,5.877)}
\gppoint{gp mark 1}{(11.016,5.878)}
\gppoint{gp mark 1}{(11.017,5.879)}
\gppoint{gp mark 1}{(11.019,5.881)}
\gppoint{gp mark 1}{(11.021,5.882)}
\gppoint{gp mark 1}{(11.023,5.883)}
\gppoint{gp mark 1}{(11.025,5.884)}
\gppoint{gp mark 1}{(11.027,5.885)}
\gppoint{gp mark 1}{(11.029,5.886)}
\gppoint{gp mark 1}{(11.031,5.887)}
\gppoint{gp mark 1}{(11.033,5.888)}
\gppoint{gp mark 1}{(11.035,5.890)}
\gppoint{gp mark 1}{(11.037,5.891)}
\gppoint{gp mark 1}{(11.039,5.892)}
\gppoint{gp mark 1}{(11.041,5.893)}
\gppoint{gp mark 1}{(11.042,5.894)}
\gppoint{gp mark 1}{(11.044,5.895)}
\gppoint{gp mark 1}{(11.046,5.897)}
\gppoint{gp mark 1}{(11.048,5.898)}
\gppoint{gp mark 1}{(11.050,5.898)}
\gppoint{gp mark 1}{(11.052,5.900)}
\gppoint{gp mark 1}{(11.054,5.902)}
\gppoint{gp mark 1}{(11.056,5.903)}
\gppoint{gp mark 1}{(11.058,5.904)}
\gppoint{gp mark 1}{(11.060,5.905)}
\gppoint{gp mark 1}{(11.061,5.906)}
\gppoint{gp mark 1}{(11.063,5.907)}
\gppoint{gp mark 1}{(11.065,5.908)}
\gppoint{gp mark 1}{(11.067,5.909)}
\gppoint{gp mark 1}{(11.069,5.910)}
\gppoint{gp mark 1}{(11.071,5.912)}
\gppoint{gp mark 1}{(11.073,5.912)}
\gppoint{gp mark 1}{(11.075,5.913)}
\gppoint{gp mark 1}{(11.076,5.915)}
\gppoint{gp mark 1}{(11.078,5.914)}
\gppoint{gp mark 1}{(11.080,5.916)}
\gppoint{gp mark 1}{(11.082,5.917)}
\gppoint{gp mark 1}{(11.084,5.919)}
\gppoint{gp mark 1}{(11.086,5.919)}
\gppoint{gp mark 1}{(11.088,5.920)}
\gppoint{gp mark 1}{(11.089,5.921)}
\gppoint{gp mark 1}{(11.091,5.923)}
\gppoint{gp mark 1}{(11.093,5.924)}
\gppoint{gp mark 1}{(11.095,5.925)}
\gppoint{gp mark 1}{(11.097,5.926)}
\gppoint{gp mark 1}{(11.099,5.929)}
\gppoint{gp mark 1}{(11.100,5.930)}
\gppoint{gp mark 1}{(11.102,5.931)}
\gppoint{gp mark 1}{(11.104,5.931)}
\gppoint{gp mark 1}{(11.106,5.932)}
\gppoint{gp mark 1}{(11.108,5.934)}
\gppoint{gp mark 1}{(11.110,5.936)}
\gppoint{gp mark 1}{(11.111,5.938)}
\gppoint{gp mark 1}{(11.113,5.939)}
\gppoint{gp mark 1}{(11.115,5.940)}
\gppoint{gp mark 1}{(11.117,5.941)}
\gppoint{gp mark 1}{(11.119,5.943)}
\gppoint{gp mark 1}{(11.120,5.945)}
\gppoint{gp mark 1}{(11.122,5.946)}
\gppoint{gp mark 1}{(11.124,5.947)}
\gppoint{gp mark 1}{(11.126,5.948)}
\gppoint{gp mark 1}{(11.128,6.194)}
\gppoint{gp mark 1}{(11.129,6.196)}
\gppoint{gp mark 1}{(11.131,6.200)}
\gppoint{gp mark 1}{(11.133,6.201)}
\gppoint{gp mark 1}{(11.135,6.203)}
\gppoint{gp mark 1}{(11.137,6.204)}
\gppoint{gp mark 1}{(11.138,6.206)}
\gppoint{gp mark 1}{(11.140,6.209)}
\gppoint{gp mark 1}{(11.142,6.210)}
\gppoint{gp mark 1}{(11.144,6.212)}
\gppoint{gp mark 1}{(11.145,6.214)}
\gppoint{gp mark 1}{(11.147,6.217)}
\gppoint{gp mark 1}{(11.149,6.218)}
\gppoint{gp mark 1}{(11.151,6.219)}
\gppoint{gp mark 1}{(11.153,6.220)}
\gppoint{gp mark 1}{(11.154,6.221)}
\gppoint{gp mark 1}{(11.156,6.221)}
\gppoint{gp mark 1}{(11.158,6.224)}
\gppoint{gp mark 1}{(11.160,6.225)}
\gppoint{gp mark 1}{(11.161,6.226)}
\gppoint{gp mark 1}{(11.163,6.228)}
\gppoint{gp mark 1}{(11.165,6.229)}
\gppoint{gp mark 1}{(11.167,6.230)}
\gppoint{gp mark 1}{(11.168,6.231)}
\gppoint{gp mark 1}{(11.170,6.232)}
\gppoint{gp mark 1}{(11.172,6.235)}
\gppoint{gp mark 1}{(11.174,6.235)}
\gppoint{gp mark 1}{(11.175,6.237)}
\gppoint{gp mark 1}{(11.177,6.238)}
\gppoint{gp mark 1}{(11.179,6.239)}
\gppoint{gp mark 1}{(11.181,6.241)}
\gppoint{gp mark 1}{(11.182,6.242)}
\gppoint{gp mark 1}{(11.184,6.243)}
\gppoint{gp mark 1}{(11.186,6.246)}
\gppoint{gp mark 1}{(11.187,6.247)}
\gppoint{gp mark 1}{(11.189,6.249)}
\gppoint{gp mark 1}{(11.191,6.251)}
\gppoint{gp mark 1}{(11.193,6.253)}
\gppoint{gp mark 1}{(11.194,6.254)}
\gppoint{gp mark 1}{(11.196,6.255)}
\gppoint{gp mark 1}{(11.198,6.256)}
\gppoint{gp mark 1}{(11.199,6.258)}
\gppoint{gp mark 1}{(11.201,6.260)}
\gppoint{gp mark 1}{(11.203,6.261)}
\gppoint{gp mark 1}{(11.205,6.262)}
\gppoint{gp mark 1}{(11.206,6.264)}
\gppoint{gp mark 1}{(11.208,6.266)}
\gppoint{gp mark 1}{(11.210,6.267)}
\gppoint{gp mark 1}{(11.211,6.269)}
\gppoint{gp mark 1}{(11.213,6.270)}
\gppoint{gp mark 1}{(11.215,6.272)}
\gppoint{gp mark 1}{(11.216,6.272)}
\gppoint{gp mark 1}{(11.218,6.272)}
\gppoint{gp mark 1}{(11.220,6.274)}
\gppoint{gp mark 1}{(11.222,6.275)}
\gppoint{gp mark 1}{(11.223,6.276)}
\gppoint{gp mark 1}{(11.225,6.276)}
\gppoint{gp mark 1}{(11.227,6.279)}
\gppoint{gp mark 1}{(11.228,6.281)}
\gppoint{gp mark 1}{(11.230,6.283)}
\gppoint{gp mark 1}{(11.232,6.285)}
\gppoint{gp mark 1}{(11.233,6.286)}
\gppoint{gp mark 1}{(11.235,6.287)}
\gppoint{gp mark 1}{(11.237,6.289)}
\gppoint{gp mark 1}{(11.238,6.289)}
\gppoint{gp mark 1}{(11.240,6.290)}
\gppoint{gp mark 1}{(11.242,6.292)}
\gppoint{gp mark 1}{(11.243,6.293)}
\gppoint{gp mark 1}{(11.245,6.294)}
\gppoint{gp mark 1}{(11.247,6.295)}
\gppoint{gp mark 1}{(11.248,6.295)}
\gppoint{gp mark 1}{(11.250,6.295)}
\gppoint{gp mark 1}{(11.252,6.295)}
\gppoint{gp mark 1}{(11.253,6.295)}
\gppoint{gp mark 1}{(11.255,6.297)}
\gppoint{gp mark 1}{(11.256,6.296)}
\gppoint{gp mark 1}{(11.258,6.296)}
\gppoint{gp mark 1}{(11.260,6.298)}
\gppoint{gp mark 1}{(11.261,6.296)}
\gppoint{gp mark 1}{(11.263,6.298)}
\gppoint{gp mark 1}{(11.265,6.298)}
\gppoint{gp mark 1}{(11.266,6.301)}
\gppoint{gp mark 1}{(11.268,6.301)}
\gppoint{gp mark 1}{(11.270,6.301)}
\gppoint{gp mark 1}{(11.271,6.302)}
\gppoint{gp mark 1}{(11.273,6.302)}
\gppoint{gp mark 1}{(11.274,6.303)}
\gppoint{gp mark 1}{(11.276,6.304)}
\gppoint{gp mark 1}{(11.278,6.305)}
\gppoint{gp mark 1}{(11.279,6.306)}
\gppoint{gp mark 1}{(11.281,6.307)}
\gppoint{gp mark 1}{(11.282,6.306)}
\gppoint{gp mark 1}{(11.284,6.305)}
\gppoint{gp mark 1}{(11.286,6.306)}
\gppoint{gp mark 1}{(11.287,6.306)}
\gppoint{gp mark 1}{(11.289,6.306)}
\gppoint{gp mark 1}{(11.291,6.306)}
\gppoint{gp mark 1}{(11.292,6.307)}
\gppoint{gp mark 1}{(11.294,6.307)}
\gppoint{gp mark 1}{(11.295,6.308)}
\gppoint{gp mark 1}{(11.297,6.310)}
\gppoint{gp mark 1}{(11.299,6.311)}
\gppoint{gp mark 1}{(11.300,6.311)}
\gppoint{gp mark 1}{(11.302,6.312)}
\gppoint{gp mark 1}{(11.303,6.314)}
\gppoint{gp mark 1}{(11.305,6.314)}
\gppoint{gp mark 1}{(11.306,6.315)}
\gppoint{gp mark 1}{(11.308,6.315)}
\gppoint{gp mark 1}{(11.310,6.316)}
\gppoint{gp mark 1}{(11.311,6.317)}
\gppoint{gp mark 1}{(11.313,6.319)}
\gppoint{gp mark 1}{(11.314,6.319)}
\gppoint{gp mark 1}{(11.316,6.320)}
\gppoint{gp mark 1}{(11.318,6.322)}
\gppoint{gp mark 1}{(11.319,6.322)}
\gppoint{gp mark 1}{(11.321,6.323)}
\gppoint{gp mark 1}{(11.322,6.324)}
\gppoint{gp mark 1}{(11.324,6.325)}
\gppoint{gp mark 1}{(11.325,6.325)}
\gppoint{gp mark 1}{(11.327,6.326)}
\gppoint{gp mark 1}{(11.329,6.327)}
\gppoint{gp mark 1}{(11.330,6.327)}
\gppoint{gp mark 1}{(11.332,6.329)}
\gppoint{gp mark 1}{(11.333,6.328)}
\gppoint{gp mark 1}{(11.335,6.327)}
\gppoint{gp mark 1}{(11.336,6.328)}
\gppoint{gp mark 1}{(11.338,6.330)}
\gppoint{gp mark 1}{(11.339,6.331)}
\gppoint{gp mark 1}{(11.341,6.332)}
\gppoint{gp mark 1}{(11.343,6.333)}
\gppoint{gp mark 1}{(11.344,6.334)}
\gppoint{gp mark 1}{(11.346,6.334)}
\gppoint{gp mark 1}{(11.347,6.335)}
\gppoint{gp mark 1}{(11.349,6.337)}
\gppoint{gp mark 1}{(11.350,6.338)}
\gppoint{gp mark 1}{(11.352,6.340)}
\gppoint{gp mark 1}{(11.353,6.340)}
\gppoint{gp mark 1}{(11.355,6.341)}
\gppoint{gp mark 1}{(11.356,6.342)}
\gppoint{gp mark 1}{(11.358,6.342)}
\gppoint{gp mark 1}{(11.359,6.342)}
\gppoint{gp mark 1}{(11.361,6.343)}
\gppoint{gp mark 1}{(11.363,6.345)}
\gppoint{gp mark 1}{(11.364,6.345)}
\gppoint{gp mark 1}{(11.366,6.347)}
\gppoint{gp mark 1}{(11.367,6.348)}
\gppoint{gp mark 1}{(11.369,6.349)}
\gppoint{gp mark 1}{(11.370,6.350)}
\gppoint{gp mark 1}{(11.372,6.350)}
\gppoint{gp mark 1}{(11.373,6.351)}
\gppoint{gp mark 1}{(11.375,6.353)}
\gppoint{gp mark 1}{(11.376,6.353)}
\gppoint{gp mark 1}{(11.378,6.353)}
\gppoint{gp mark 1}{(11.379,6.353)}
\gppoint{gp mark 1}{(11.381,6.353)}
\gppoint{gp mark 1}{(11.382,6.352)}
\gppoint{gp mark 1}{(11.384,6.352)}
\gppoint{gp mark 1}{(11.385,6.354)}
\gppoint{gp mark 1}{(11.387,6.353)}
\gppoint{gp mark 1}{(11.388,6.353)}
\gppoint{gp mark 1}{(11.390,6.353)}
\gppoint{gp mark 1}{(11.391,6.355)}
\gppoint{gp mark 1}{(11.393,6.357)}
\gppoint{gp mark 1}{(11.394,6.357)}
\gppoint{gp mark 1}{(11.396,6.358)}
\gppoint{gp mark 1}{(11.397,6.358)}
\gppoint{gp mark 1}{(11.399,6.360)}
\gppoint{gp mark 1}{(11.400,6.361)}
\gppoint{gp mark 1}{(11.402,6.361)}
\gppoint{gp mark 1}{(11.403,6.363)}
\gppoint{gp mark 1}{(11.405,6.363)}
\gppoint{gp mark 1}{(11.406,6.363)}
\gppoint{gp mark 1}{(11.408,6.364)}
\gppoint{gp mark 1}{(11.409,6.365)}
\gppoint{gp mark 1}{(11.411,6.365)}
\gppoint{gp mark 1}{(11.412,6.365)}
\gppoint{gp mark 1}{(11.414,6.365)}
\gppoint{gp mark 1}{(11.415,6.365)}
\gppoint{gp mark 1}{(11.416,6.366)}
\gppoint{gp mark 1}{(11.418,6.368)}
\gppoint{gp mark 1}{(11.419,6.369)}
\gppoint{gp mark 1}{(11.421,6.369)}
\gppoint{gp mark 1}{(11.422,6.370)}
\gppoint{gp mark 1}{(11.424,6.370)}
\gppoint{gp mark 1}{(11.425,6.370)}
\gppoint{gp mark 1}{(11.427,6.371)}
\gppoint{gp mark 1}{(11.428,6.372)}
\gppoint{gp mark 1}{(11.430,6.372)}
\gppoint{gp mark 1}{(11.431,6.374)}
\gppoint{gp mark 1}{(11.433,6.374)}
\gppoint{gp mark 1}{(11.434,6.375)}
\gppoint{gp mark 1}{(11.435,6.375)}
\gppoint{gp mark 1}{(11.437,6.375)}
\gppoint{gp mark 1}{(11.438,6.375)}
\gppoint{gp mark 1}{(11.440,6.376)}
\gppoint{gp mark 1}{(11.441,6.376)}
\gppoint{gp mark 1}{(11.443,6.377)}
\gppoint{gp mark 1}{(11.444,6.377)}
\gppoint{gp mark 1}{(11.446,6.378)}
\gppoint{gp mark 1}{(11.447,6.378)}
\draw[gp path] (1.504,7.691)--(1.504,0.985)--(11.447,0.985)--(11.447,7.691)--cycle;
%% coordinates of the plot area
\gpdefrectangularnode{gp plot 1}{\pgfpoint{1.504cm}{0.985cm}}{\pgfpoint{11.447cm}{7.691cm}}
\end{tikzpicture}
%% gnuplot variables

\caption{$T$-$\sigma_A^2$グラフ}
\label{fig:allan}
\end{center}
\end{figure}