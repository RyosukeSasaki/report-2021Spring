\subsection{受信機雑音温度}
\subsubsection{原理}
\ref{subsec:hukusyayuso}節で示したように,輻射強度は温度の次元を持った輝度温度で表されたので,
望遠鏡の受信した電力を温度の次元で表示することを考える.
ここで
\begin{align}
  T_A=\frac{W_\nu}{k}\ \left[\frac{\si{\joule.\second^{-1}.\hertz^{-1}}}{\si{\joule.\kelvin^{-1}}}\right]
\end{align}
はアンテナ温度と呼ばれ,温度の次元を持つ.ただし$W_\nu$は周波数$\nu$の電波の単位周波数あたりの受信電力である.
これはある受信電力に相当する熱雑音を発生させるために必要な温度といえる.

受信した微弱な信号はコンバーターからADCまでの信号処理系全体で発生する雑音に埋もれるため,受信機の感度はその雑音によって制限される.
これらの雑音のアンテナ温度をまとめて受信機雑音温度$T_{RX}$で表す.
ここで受信機を高温$T_{hot}$,低温$T_{cold}$の黒体で覆ったとき,受信機出力電圧は
\begin{align}
  V_{hot}=a(T_{hot}+T_{RX})\\
  V_{cold=a(T_{cold}+T_{RX})}
\end{align}
と表される.ここで$a$は受信機の利得による比例定数である.上の2式と$Y=:V_{hot}/V_{cold}$を用いて$T_{RX}$を表すと
\begin{align}
  T_{RX}=\frac{T_{hot}-YT_{cold}}{Y-1}
\end{align}
となる.
\subsubsection{方法}
この実験では高温物体として室温の水,低温物体として液体窒素を用いた.
まず電波吸収体で覆ったデュワー瓶に室温の水を入れ,その温度を測定した.
次に室内機の集光部をデュワー瓶の水に浸し$V_{hot}$を測定した.
続いて上と同様のデュワー瓶に液体窒素を入れ,同様の手順で$V_{cold}$を測定した.
\subsubsection{結果}
図\ref{fig:Thot_Tcold}に測定結果を示す.図\ref{fig:Thot_Tcold}の網掛け(青)部は$T_{hot}$,
網掛け(赤)部は$T_{cold}$での測定である.それぞれの経過時間は$t=0$-$780$, $t=950$-$1230$である.
この範囲の値を平均すると$V_{hot}=3.10\si{\volt}$, $V_{cold}=2.02\si{\volt}$となった.
また$T_{hot}=298.15\si{\kelvin}$, $T_{cold}=77.0\si{\kelvin}$であった.以上から$T_{RX}$は
\begin{align}
  T_{RX}=3.37\times10^2\si{\kelvin}
\end{align}
となる.
\begin{figure}[hptb]
\begin{center}
\begin{tikzpicture}[gnuplot]
%% generated with GNUPLOT 5.2p8 (Lua 5.3; terminal rev. Nov 2018, script rev. 108)
%% 2021年05月12日 22時15分37秒
\path (0.000,0.000) rectangle (12.000,8.000);
\gpfill{rgb color={0.000,1.000,1.000},color=.!30} (1.320,0.985)--(7.396,0.985)--(7.396,7.690)--(1.320,7.690)--cycle;
\gpfill{rgb color={1.000,0.000,0.000},color=.!30} (8.721,0.985)--(10.902,0.985)--(10.902,7.690)--(8.721,7.690)--cycle;
\gpcolor{color=gp lt color border}
\gpsetlinetype{gp lt border}
\gpsetdashtype{gp dt solid}
\gpsetlinewidth{1.00}
\draw[gp path] (1.320,1.447)--(1.500,1.447);
\draw[gp path] (11.447,1.447)--(11.267,1.447);
\node[gp node right] at (1.136,1.447) {$2$};
\draw[gp path] (1.320,1.910)--(1.410,1.910);
\draw[gp path] (11.447,1.910)--(11.357,1.910);
\draw[gp path] (1.320,2.372)--(1.410,2.372);
\draw[gp path] (11.447,2.372)--(11.357,2.372);
\draw[gp path] (1.320,2.835)--(1.410,2.835);
\draw[gp path] (11.447,2.835)--(11.357,2.835);
\draw[gp path] (1.320,3.297)--(1.410,3.297);
\draw[gp path] (11.447,3.297)--(11.357,3.297);
\draw[gp path] (1.320,3.760)--(1.500,3.760);
\draw[gp path] (11.447,3.760)--(11.267,3.760);
\node[gp node right] at (1.136,3.760) {$2.5$};
\draw[gp path] (1.320,4.222)--(1.410,4.222);
\draw[gp path] (11.447,4.222)--(11.357,4.222);
\draw[gp path] (1.320,4.685)--(1.410,4.685);
\draw[gp path] (11.447,4.685)--(11.357,4.685);
\draw[gp path] (1.320,5.147)--(1.410,5.147);
\draw[gp path] (11.447,5.147)--(11.357,5.147);
\draw[gp path] (1.320,5.610)--(1.410,5.610);
\draw[gp path] (11.447,5.610)--(11.357,5.610);
\draw[gp path] (1.320,6.072)--(1.500,6.072);
\draw[gp path] (11.447,6.072)--(11.267,6.072);
\node[gp node right] at (1.136,6.072) {$3$};
\draw[gp path] (1.320,6.535)--(1.410,6.535);
\draw[gp path] (11.447,6.535)--(11.357,6.535);
\draw[gp path] (1.320,6.997)--(1.410,6.997);
\draw[gp path] (11.447,6.997)--(11.357,6.997);
\draw[gp path] (1.320,0.985)--(1.320,1.165);
\draw[gp path] (1.320,7.691)--(1.320,7.511);
\node[gp node center] at (1.320,0.677) {$0$};
\draw[gp path] (2.099,0.985)--(2.099,1.075);
\draw[gp path] (2.099,7.691)--(2.099,7.601);
\draw[gp path] (2.878,0.985)--(2.878,1.075);
\draw[gp path] (2.878,7.691)--(2.878,7.601);
\draw[gp path] (3.657,0.985)--(3.657,1.075);
\draw[gp path] (3.657,7.691)--(3.657,7.601);
\draw[gp path] (4.436,0.985)--(4.436,1.075);
\draw[gp path] (4.436,7.691)--(4.436,7.601);
\draw[gp path] (5.215,0.985)--(5.215,1.165);
\draw[gp path] (5.215,7.691)--(5.215,7.511);
\node[gp node center] at (5.215,0.677) {$500$};
\draw[gp path] (5.994,0.985)--(5.994,1.075);
\draw[gp path] (5.994,7.691)--(5.994,7.601);
\draw[gp path] (6.773,0.985)--(6.773,1.075);
\draw[gp path] (6.773,7.691)--(6.773,7.601);
\draw[gp path] (7.552,0.985)--(7.552,1.075);
\draw[gp path] (7.552,7.691)--(7.552,7.601);
\draw[gp path] (8.331,0.985)--(8.331,1.075);
\draw[gp path] (8.331,7.691)--(8.331,7.601);
\draw[gp path] (9.110,0.985)--(9.110,1.165);
\draw[gp path] (9.110,7.691)--(9.110,7.511);
\node[gp node center] at (9.110,0.677) {$1000$};
\draw[gp path] (9.889,0.985)--(9.889,1.075);
\draw[gp path] (9.889,7.691)--(9.889,7.601);
\draw[gp path] (10.668,0.985)--(10.668,1.075);
\draw[gp path] (10.668,7.691)--(10.668,7.601);
\draw[gp path] (11.447,0.985)--(11.447,1.075);
\draw[gp path] (11.447,7.691)--(11.447,7.601);
\draw[gp path] (1.320,7.691)--(1.320,0.985)--(11.447,0.985)--(11.447,7.691)--cycle;
\node[gp node center,rotate=-270] at (0.292,4.338) {電圧 / $\si{\volt}$};
\node[gp node center] at (6.383,0.215) {経過時間$T$ / $\si{\second}$};
\gpsetpointsize{4.00}
\gppoint{gp mark 1}{(1.328,6.657)}
\gppoint{gp mark 1}{(1.336,6.623)}
\gppoint{gp mark 1}{(1.343,6.618)}
\gppoint{gp mark 1}{(1.351,6.635)}
\gppoint{gp mark 1}{(1.359,6.650)}
\gppoint{gp mark 1}{(1.367,6.610)}
\gppoint{gp mark 1}{(1.375,6.669)}
\gppoint{gp mark 1}{(1.382,6.596)}
\gppoint{gp mark 1}{(1.390,6.606)}
\gppoint{gp mark 1}{(1.398,6.618)}
\gppoint{gp mark 1}{(1.406,6.627)}
\gppoint{gp mark 1}{(1.413,6.613)}
\gppoint{gp mark 1}{(1.421,6.630)}
\gppoint{gp mark 1}{(1.429,6.637)}
\gppoint{gp mark 1}{(1.437,6.616)}
\gppoint{gp mark 1}{(1.445,6.628)}
\gppoint{gp mark 1}{(1.452,6.606)}
\gppoint{gp mark 1}{(1.460,6.625)}
\gppoint{gp mark 1}{(1.468,6.610)}
\gppoint{gp mark 1}{(1.476,6.625)}
\gppoint{gp mark 1}{(1.484,6.607)}
\gppoint{gp mark 1}{(1.491,6.592)}
\gppoint{gp mark 1}{(1.499,6.623)}
\gppoint{gp mark 1}{(1.507,6.590)}
\gppoint{gp mark 1}{(1.515,6.627)}
\gppoint{gp mark 1}{(1.523,6.626)}
\gppoint{gp mark 1}{(1.530,6.559)}
\gppoint{gp mark 1}{(1.538,6.599)}
\gppoint{gp mark 1}{(1.546,6.608)}
\gppoint{gp mark 1}{(1.554,6.580)}
\gppoint{gp mark 1}{(1.561,6.572)}
\gppoint{gp mark 1}{(1.569,6.567)}
\gppoint{gp mark 1}{(1.577,6.585)}
\gppoint{gp mark 1}{(1.585,6.551)}
\gppoint{gp mark 1}{(1.593,6.588)}
\gppoint{gp mark 1}{(1.600,6.585)}
\gppoint{gp mark 1}{(1.608,6.567)}
\gppoint{gp mark 1}{(1.616,6.551)}
\gppoint{gp mark 1}{(1.624,6.550)}
\gppoint{gp mark 1}{(1.632,6.599)}
\gppoint{gp mark 1}{(1.639,6.577)}
\gppoint{gp mark 1}{(1.647,6.566)}
\gppoint{gp mark 1}{(1.655,6.601)}
\gppoint{gp mark 1}{(1.663,6.563)}
\gppoint{gp mark 1}{(1.671,6.576)}
\gppoint{gp mark 1}{(1.678,6.587)}
\gppoint{gp mark 1}{(1.686,6.559)}
\gppoint{gp mark 1}{(1.694,6.556)}
\gppoint{gp mark 1}{(1.702,6.561)}
\gppoint{gp mark 1}{(1.710,6.583)}
\gppoint{gp mark 1}{(1.717,6.574)}
\gppoint{gp mark 1}{(1.725,6.572)}
\gppoint{gp mark 1}{(1.733,6.588)}
\gppoint{gp mark 1}{(1.741,6.565)}
\gppoint{gp mark 1}{(1.748,6.565)}
\gppoint{gp mark 1}{(1.756,6.570)}
\gppoint{gp mark 1}{(1.764,6.597)}
\gppoint{gp mark 1}{(1.772,6.604)}
\gppoint{gp mark 1}{(1.780,6.583)}
\gppoint{gp mark 1}{(1.787,6.564)}
\gppoint{gp mark 1}{(1.795,6.550)}
\gppoint{gp mark 1}{(1.803,6.530)}
\gppoint{gp mark 1}{(1.811,6.573)}
\gppoint{gp mark 1}{(1.819,6.571)}
\gppoint{gp mark 1}{(1.826,6.587)}
\gppoint{gp mark 1}{(1.834,6.561)}
\gppoint{gp mark 1}{(1.842,6.606)}
\gppoint{gp mark 1}{(1.850,6.603)}
\gppoint{gp mark 1}{(1.858,6.570)}
\gppoint{gp mark 1}{(1.865,6.604)}
\gppoint{gp mark 1}{(1.873,6.605)}
\gppoint{gp mark 1}{(1.881,6.596)}
\gppoint{gp mark 1}{(1.889,6.590)}
\gppoint{gp mark 1}{(1.896,6.583)}
\gppoint{gp mark 1}{(1.904,6.569)}
\gppoint{gp mark 1}{(1.912,6.570)}
\gppoint{gp mark 1}{(1.920,6.577)}
\gppoint{gp mark 1}{(1.928,6.570)}
\gppoint{gp mark 1}{(1.935,6.564)}
\gppoint{gp mark 1}{(1.943,6.549)}
\gppoint{gp mark 1}{(1.951,6.590)}
\gppoint{gp mark 1}{(1.959,6.547)}
\gppoint{gp mark 1}{(1.967,6.556)}
\gppoint{gp mark 1}{(1.974,6.578)}
\gppoint{gp mark 1}{(1.982,6.594)}
\gppoint{gp mark 1}{(1.990,6.579)}
\gppoint{gp mark 1}{(1.998,6.580)}
\gppoint{gp mark 1}{(2.006,6.557)}
\gppoint{gp mark 1}{(2.013,6.566)}
\gppoint{gp mark 1}{(2.021,6.564)}
\gppoint{gp mark 1}{(2.029,6.567)}
\gppoint{gp mark 1}{(2.037,6.557)}
\gppoint{gp mark 1}{(2.044,6.542)}
\gppoint{gp mark 1}{(2.052,6.546)}
\gppoint{gp mark 1}{(2.060,6.550)}
\gppoint{gp mark 1}{(2.068,6.562)}
\gppoint{gp mark 1}{(2.076,6.554)}
\gppoint{gp mark 1}{(2.083,6.579)}
\gppoint{gp mark 1}{(2.091,6.567)}
\gppoint{gp mark 1}{(2.099,6.568)}
\gppoint{gp mark 1}{(2.107,6.560)}
\gppoint{gp mark 1}{(2.115,6.551)}
\gppoint{gp mark 1}{(2.122,6.541)}
\gppoint{gp mark 1}{(2.130,6.566)}
\gppoint{gp mark 1}{(2.138,6.561)}
\gppoint{gp mark 1}{(2.146,6.592)}
\gppoint{gp mark 1}{(2.154,6.560)}
\gppoint{gp mark 1}{(2.161,6.568)}
\gppoint{gp mark 1}{(2.169,6.564)}
\gppoint{gp mark 1}{(2.177,6.558)}
\gppoint{gp mark 1}{(2.185,6.542)}
\gppoint{gp mark 1}{(2.192,6.549)}
\gppoint{gp mark 1}{(2.200,6.584)}
\gppoint{gp mark 1}{(2.208,6.569)}
\gppoint{gp mark 1}{(2.216,6.615)}
\gppoint{gp mark 1}{(2.224,6.580)}
\gppoint{gp mark 1}{(2.231,6.579)}
\gppoint{gp mark 1}{(2.239,6.580)}
\gppoint{gp mark 1}{(2.247,6.560)}
\gppoint{gp mark 1}{(2.255,6.572)}
\gppoint{gp mark 1}{(2.263,6.540)}
\gppoint{gp mark 1}{(2.270,6.592)}
\gppoint{gp mark 1}{(2.278,6.552)}
\gppoint{gp mark 1}{(2.286,6.586)}
\gppoint{gp mark 1}{(2.294,6.586)}
\gppoint{gp mark 1}{(2.302,6.555)}
\gppoint{gp mark 1}{(2.309,6.537)}
\gppoint{gp mark 1}{(2.317,6.555)}
\gppoint{gp mark 1}{(2.325,6.542)}
\gppoint{gp mark 1}{(2.333,6.546)}
\gppoint{gp mark 1}{(2.340,6.573)}
\gppoint{gp mark 1}{(2.348,6.556)}
\gppoint{gp mark 1}{(2.356,6.564)}
\gppoint{gp mark 1}{(2.364,6.566)}
\gppoint{gp mark 1}{(2.372,6.555)}
\gppoint{gp mark 1}{(2.379,6.574)}
\gppoint{gp mark 1}{(2.387,6.558)}
\gppoint{gp mark 1}{(2.395,6.574)}
\gppoint{gp mark 1}{(2.403,6.580)}
\gppoint{gp mark 1}{(2.411,6.572)}
\gppoint{gp mark 1}{(2.418,6.552)}
\gppoint{gp mark 1}{(2.426,6.557)}
\gppoint{gp mark 1}{(2.434,6.570)}
\gppoint{gp mark 1}{(2.442,6.579)}
\gppoint{gp mark 1}{(2.450,6.549)}
\gppoint{gp mark 1}{(2.457,6.570)}
\gppoint{gp mark 1}{(2.465,6.556)}
\gppoint{gp mark 1}{(2.473,6.561)}
\gppoint{gp mark 1}{(2.481,6.580)}
\gppoint{gp mark 1}{(2.489,6.553)}
\gppoint{gp mark 1}{(2.496,6.539)}
\gppoint{gp mark 1}{(2.504,6.571)}
\gppoint{gp mark 1}{(2.512,6.554)}
\gppoint{gp mark 1}{(2.520,6.578)}
\gppoint{gp mark 1}{(2.527,6.573)}
\gppoint{gp mark 1}{(2.535,6.539)}
\gppoint{gp mark 1}{(2.543,6.539)}
\gppoint{gp mark 1}{(2.551,6.572)}
\gppoint{gp mark 1}{(2.559,6.585)}
\gppoint{gp mark 1}{(2.566,6.552)}
\gppoint{gp mark 1}{(2.574,6.569)}
\gppoint{gp mark 1}{(2.582,6.555)}
\gppoint{gp mark 1}{(2.590,6.541)}
\gppoint{gp mark 1}{(2.598,6.552)}
\gppoint{gp mark 1}{(2.605,6.532)}
\gppoint{gp mark 1}{(2.613,6.520)}
\gppoint{gp mark 1}{(2.621,6.547)}
\gppoint{gp mark 1}{(2.629,6.556)}
\gppoint{gp mark 1}{(2.637,6.550)}
\gppoint{gp mark 1}{(2.644,6.548)}
\gppoint{gp mark 1}{(2.652,6.554)}
\gppoint{gp mark 1}{(2.660,6.561)}
\gppoint{gp mark 1}{(2.668,6.574)}
\gppoint{gp mark 1}{(2.675,6.549)}
\gppoint{gp mark 1}{(2.683,6.549)}
\gppoint{gp mark 1}{(2.691,6.564)}
\gppoint{gp mark 1}{(2.699,6.542)}
\gppoint{gp mark 1}{(2.707,6.547)}
\gppoint{gp mark 1}{(2.714,6.560)}
\gppoint{gp mark 1}{(2.722,6.545)}
\gppoint{gp mark 1}{(2.730,6.560)}
\gppoint{gp mark 1}{(2.738,6.542)}
\gppoint{gp mark 1}{(2.746,6.558)}
\gppoint{gp mark 1}{(2.753,6.561)}
\gppoint{gp mark 1}{(2.761,6.550)}
\gppoint{gp mark 1}{(2.769,6.569)}
\gppoint{gp mark 1}{(2.777,6.571)}
\gppoint{gp mark 1}{(2.785,6.540)}
\gppoint{gp mark 1}{(2.792,6.558)}
\gppoint{gp mark 1}{(2.800,6.542)}
\gppoint{gp mark 1}{(2.808,6.590)}
\gppoint{gp mark 1}{(2.816,6.534)}
\gppoint{gp mark 1}{(2.823,6.543)}
\gppoint{gp mark 1}{(2.831,6.546)}
\gppoint{gp mark 1}{(2.839,6.554)}
\gppoint{gp mark 1}{(2.847,6.559)}
\gppoint{gp mark 1}{(2.855,6.557)}
\gppoint{gp mark 1}{(2.862,6.587)}
\gppoint{gp mark 1}{(2.870,6.552)}
\gppoint{gp mark 1}{(2.878,6.559)}
\gppoint{gp mark 1}{(2.886,6.556)}
\gppoint{gp mark 1}{(2.894,6.560)}
\gppoint{gp mark 1}{(2.901,6.555)}
\gppoint{gp mark 1}{(2.909,6.548)}
\gppoint{gp mark 1}{(2.917,6.536)}
\gppoint{gp mark 1}{(2.925,6.562)}
\gppoint{gp mark 1}{(2.933,6.536)}
\gppoint{gp mark 1}{(2.940,6.585)}
\gppoint{gp mark 1}{(2.948,6.562)}
\gppoint{gp mark 1}{(2.956,6.554)}
\gppoint{gp mark 1}{(2.964,6.560)}
\gppoint{gp mark 1}{(2.971,6.556)}
\gppoint{gp mark 1}{(2.979,6.552)}
\gppoint{gp mark 1}{(2.987,6.564)}
\gppoint{gp mark 1}{(2.995,6.566)}
\gppoint{gp mark 1}{(3.003,6.584)}
\gppoint{gp mark 1}{(3.010,6.556)}
\gppoint{gp mark 1}{(3.018,6.538)}
\gppoint{gp mark 1}{(3.026,6.553)}
\gppoint{gp mark 1}{(3.034,6.541)}
\gppoint{gp mark 1}{(3.042,6.558)}
\gppoint{gp mark 1}{(3.049,6.554)}
\gppoint{gp mark 1}{(3.057,6.559)}
\gppoint{gp mark 1}{(3.065,6.550)}
\gppoint{gp mark 1}{(3.073,6.563)}
\gppoint{gp mark 1}{(3.081,6.536)}
\gppoint{gp mark 1}{(3.088,6.543)}
\gppoint{gp mark 1}{(3.096,6.542)}
\gppoint{gp mark 1}{(3.104,6.557)}
\gppoint{gp mark 1}{(3.112,6.535)}
\gppoint{gp mark 1}{(3.119,6.587)}
\gppoint{gp mark 1}{(3.127,6.566)}
\gppoint{gp mark 1}{(3.135,6.549)}
\gppoint{gp mark 1}{(3.143,6.564)}
\gppoint{gp mark 1}{(3.151,6.544)}
\gppoint{gp mark 1}{(3.158,6.551)}
\gppoint{gp mark 1}{(3.166,6.585)}
\gppoint{gp mark 1}{(3.174,6.553)}
\gppoint{gp mark 1}{(3.182,6.551)}
\gppoint{gp mark 1}{(3.190,6.545)}
\gppoint{gp mark 1}{(3.197,6.535)}
\gppoint{gp mark 1}{(3.205,6.545)}
\gppoint{gp mark 1}{(3.213,6.562)}
\gppoint{gp mark 1}{(3.221,6.538)}
\gppoint{gp mark 1}{(3.229,6.557)}
\gppoint{gp mark 1}{(3.236,6.534)}
\gppoint{gp mark 1}{(3.244,6.542)}
\gppoint{gp mark 1}{(3.252,6.552)}
\gppoint{gp mark 1}{(3.260,6.523)}
\gppoint{gp mark 1}{(3.268,6.528)}
\gppoint{gp mark 1}{(3.275,6.557)}
\gppoint{gp mark 1}{(3.283,6.561)}
\gppoint{gp mark 1}{(3.291,6.520)}
\gppoint{gp mark 1}{(3.299,6.529)}
\gppoint{gp mark 1}{(3.306,6.526)}
\gppoint{gp mark 1}{(3.314,6.528)}
\gppoint{gp mark 1}{(3.322,6.546)}
\gppoint{gp mark 1}{(3.330,6.560)}
\gppoint{gp mark 1}{(3.338,6.535)}
\gppoint{gp mark 1}{(3.345,6.564)}
\gppoint{gp mark 1}{(3.353,6.546)}
\gppoint{gp mark 1}{(3.361,6.544)}
\gppoint{gp mark 1}{(3.369,6.551)}
\gppoint{gp mark 1}{(3.377,6.559)}
\gppoint{gp mark 1}{(3.384,6.542)}
\gppoint{gp mark 1}{(3.392,6.560)}
\gppoint{gp mark 1}{(3.400,6.585)}
\gppoint{gp mark 1}{(3.408,6.542)}
\gppoint{gp mark 1}{(3.416,6.551)}
\gppoint{gp mark 1}{(3.423,6.549)}
\gppoint{gp mark 1}{(3.431,6.567)}
\gppoint{gp mark 1}{(3.439,6.528)}
\gppoint{gp mark 1}{(3.447,6.551)}
\gppoint{gp mark 1}{(3.454,6.569)}
\gppoint{gp mark 1}{(3.462,6.553)}
\gppoint{gp mark 1}{(3.470,6.580)}
\gppoint{gp mark 1}{(3.478,6.540)}
\gppoint{gp mark 1}{(3.486,6.530)}
\gppoint{gp mark 1}{(3.493,6.574)}
\gppoint{gp mark 1}{(3.501,6.552)}
\gppoint{gp mark 1}{(3.509,6.545)}
\gppoint{gp mark 1}{(3.517,6.561)}
\gppoint{gp mark 1}{(3.525,6.542)}
\gppoint{gp mark 1}{(3.532,6.523)}
\gppoint{gp mark 1}{(3.540,6.544)}
\gppoint{gp mark 1}{(3.548,6.514)}
\gppoint{gp mark 1}{(3.556,6.556)}
\gppoint{gp mark 1}{(3.564,6.538)}
\gppoint{gp mark 1}{(3.571,6.554)}
\gppoint{gp mark 1}{(3.579,6.549)}
\gppoint{gp mark 1}{(3.587,6.532)}
\gppoint{gp mark 1}{(3.595,6.525)}
\gppoint{gp mark 1}{(3.602,6.566)}
\gppoint{gp mark 1}{(3.610,6.556)}
\gppoint{gp mark 1}{(3.618,6.550)}
\gppoint{gp mark 1}{(3.626,6.544)}
\gppoint{gp mark 1}{(3.634,6.563)}
\gppoint{gp mark 1}{(3.641,6.555)}
\gppoint{gp mark 1}{(3.649,6.538)}
\gppoint{gp mark 1}{(3.657,6.546)}
\gppoint{gp mark 1}{(3.665,6.555)}
\gppoint{gp mark 1}{(3.673,6.534)}
\gppoint{gp mark 1}{(3.680,6.533)}
\gppoint{gp mark 1}{(3.688,6.513)}
\gppoint{gp mark 1}{(3.696,6.567)}
\gppoint{gp mark 1}{(3.704,6.553)}
\gppoint{gp mark 1}{(3.712,6.562)}
\gppoint{gp mark 1}{(3.719,6.564)}
\gppoint{gp mark 1}{(3.727,6.545)}
\gppoint{gp mark 1}{(3.735,6.536)}
\gppoint{gp mark 1}{(3.743,6.523)}
\gppoint{gp mark 1}{(3.750,6.532)}
\gppoint{gp mark 1}{(3.758,6.544)}
\gppoint{gp mark 1}{(3.766,6.536)}
\gppoint{gp mark 1}{(3.774,6.561)}
\gppoint{gp mark 1}{(3.782,6.534)}
\gppoint{gp mark 1}{(3.789,6.526)}
\gppoint{gp mark 1}{(3.797,6.522)}
\gppoint{gp mark 1}{(3.805,6.531)}
\gppoint{gp mark 1}{(3.813,6.550)}
\gppoint{gp mark 1}{(3.821,6.529)}
\gppoint{gp mark 1}{(3.828,6.542)}
\gppoint{gp mark 1}{(3.836,6.535)}
\gppoint{gp mark 1}{(3.844,6.533)}
\gppoint{gp mark 1}{(3.852,6.523)}
\gppoint{gp mark 1}{(3.860,6.546)}
\gppoint{gp mark 1}{(3.867,6.552)}
\gppoint{gp mark 1}{(3.875,6.532)}
\gppoint{gp mark 1}{(3.883,6.548)}
\gppoint{gp mark 1}{(3.891,6.537)}
\gppoint{gp mark 1}{(3.898,6.548)}
\gppoint{gp mark 1}{(3.906,6.517)}
\gppoint{gp mark 1}{(3.914,6.542)}
\gppoint{gp mark 1}{(3.922,6.543)}
\gppoint{gp mark 1}{(3.930,6.551)}
\gppoint{gp mark 1}{(3.937,6.567)}
\gppoint{gp mark 1}{(3.945,6.512)}
\gppoint{gp mark 1}{(3.953,6.526)}
\gppoint{gp mark 1}{(3.961,6.534)}
\gppoint{gp mark 1}{(3.969,6.525)}
\gppoint{gp mark 1}{(3.976,6.538)}
\gppoint{gp mark 1}{(3.984,6.535)}
\gppoint{gp mark 1}{(3.992,6.555)}
\gppoint{gp mark 1}{(4.000,6.519)}
\gppoint{gp mark 1}{(4.008,6.510)}
\gppoint{gp mark 1}{(4.015,6.552)}
\gppoint{gp mark 1}{(4.023,6.541)}
\gppoint{gp mark 1}{(4.031,6.538)}
\gppoint{gp mark 1}{(4.039,6.540)}
\gppoint{gp mark 1}{(4.047,6.541)}
\gppoint{gp mark 1}{(4.054,6.533)}
\gppoint{gp mark 1}{(4.062,6.547)}
\gppoint{gp mark 1}{(4.070,6.543)}
\gppoint{gp mark 1}{(4.078,6.544)}
\gppoint{gp mark 1}{(4.085,6.532)}
\gppoint{gp mark 1}{(4.093,6.511)}
\gppoint{gp mark 1}{(4.101,6.517)}
\gppoint{gp mark 1}{(4.109,6.514)}
\gppoint{gp mark 1}{(4.117,6.521)}
\gppoint{gp mark 1}{(4.124,6.540)}
\gppoint{gp mark 1}{(4.132,6.521)}
\gppoint{gp mark 1}{(4.140,6.521)}
\gppoint{gp mark 1}{(4.148,6.548)}
\gppoint{gp mark 1}{(4.156,6.554)}
\gppoint{gp mark 1}{(4.163,6.558)}
\gppoint{gp mark 1}{(4.171,6.557)}
\gppoint{gp mark 1}{(4.179,6.543)}
\gppoint{gp mark 1}{(4.187,6.553)}
\gppoint{gp mark 1}{(4.195,6.541)}
\gppoint{gp mark 1}{(4.202,6.519)}
\gppoint{gp mark 1}{(4.210,6.520)}
\gppoint{gp mark 1}{(4.218,6.538)}
\gppoint{gp mark 1}{(4.226,6.520)}
\gppoint{gp mark 1}{(4.233,6.521)}
\gppoint{gp mark 1}{(4.241,6.530)}
\gppoint{gp mark 1}{(4.249,6.544)}
\gppoint{gp mark 1}{(4.257,6.524)}
\gppoint{gp mark 1}{(4.265,6.547)}
\gppoint{gp mark 1}{(4.272,6.555)}
\gppoint{gp mark 1}{(4.280,6.531)}
\gppoint{gp mark 1}{(4.288,6.519)}
\gppoint{gp mark 1}{(4.296,6.536)}
\gppoint{gp mark 1}{(4.304,6.536)}
\gppoint{gp mark 1}{(4.311,6.510)}
\gppoint{gp mark 1}{(4.319,6.542)}
\gppoint{gp mark 1}{(4.327,6.536)}
\gppoint{gp mark 1}{(4.335,6.535)}
\gppoint{gp mark 1}{(4.343,6.524)}
\gppoint{gp mark 1}{(4.350,6.518)}
\gppoint{gp mark 1}{(4.358,6.539)}
\gppoint{gp mark 1}{(4.366,6.527)}
\gppoint{gp mark 1}{(4.374,6.558)}
\gppoint{gp mark 1}{(4.381,6.520)}
\gppoint{gp mark 1}{(4.389,6.528)}
\gppoint{gp mark 1}{(4.397,6.547)}
\gppoint{gp mark 1}{(4.405,6.533)}
\gppoint{gp mark 1}{(4.413,6.517)}
\gppoint{gp mark 1}{(4.420,6.537)}
\gppoint{gp mark 1}{(4.428,6.531)}
\gppoint{gp mark 1}{(4.436,6.521)}
\gppoint{gp mark 1}{(4.444,6.512)}
\gppoint{gp mark 1}{(4.452,6.541)}
\gppoint{gp mark 1}{(4.459,6.542)}
\gppoint{gp mark 1}{(4.467,6.546)}
\gppoint{gp mark 1}{(4.475,6.534)}
\gppoint{gp mark 1}{(4.483,6.524)}
\gppoint{gp mark 1}{(4.491,6.521)}
\gppoint{gp mark 1}{(4.498,6.502)}
\gppoint{gp mark 1}{(4.506,6.516)}
\gppoint{gp mark 1}{(4.514,6.525)}
\gppoint{gp mark 1}{(4.522,6.537)}
\gppoint{gp mark 1}{(4.529,6.545)}
\gppoint{gp mark 1}{(4.537,6.549)}
\gppoint{gp mark 1}{(4.545,6.536)}
\gppoint{gp mark 1}{(4.553,6.518)}
\gppoint{gp mark 1}{(4.561,6.532)}
\gppoint{gp mark 1}{(4.568,6.524)}
\gppoint{gp mark 1}{(4.576,6.538)}
\gppoint{gp mark 1}{(4.584,6.557)}
\gppoint{gp mark 1}{(4.592,6.538)}
\gppoint{gp mark 1}{(4.600,6.490)}
\gppoint{gp mark 1}{(4.607,6.540)}
\gppoint{gp mark 1}{(4.615,6.539)}
\gppoint{gp mark 1}{(4.623,6.504)}
\gppoint{gp mark 1}{(4.631,6.554)}
\gppoint{gp mark 1}{(4.639,6.551)}
\gppoint{gp mark 1}{(4.646,6.528)}
\gppoint{gp mark 1}{(4.654,6.508)}
\gppoint{gp mark 1}{(4.662,6.545)}
\gppoint{gp mark 1}{(4.670,6.511)}
\gppoint{gp mark 1}{(4.677,6.533)}
\gppoint{gp mark 1}{(4.685,6.527)}
\gppoint{gp mark 1}{(4.693,6.534)}
\gppoint{gp mark 1}{(4.701,6.515)}
\gppoint{gp mark 1}{(4.709,6.541)}
\gppoint{gp mark 1}{(4.716,6.512)}
\gppoint{gp mark 1}{(4.724,6.514)}
\gppoint{gp mark 1}{(4.732,6.523)}
\gppoint{gp mark 1}{(4.740,6.524)}
\gppoint{gp mark 1}{(4.748,6.536)}
\gppoint{gp mark 1}{(4.755,6.533)}
\gppoint{gp mark 1}{(4.763,6.510)}
\gppoint{gp mark 1}{(4.771,6.529)}
\gppoint{gp mark 1}{(4.779,6.533)}
\gppoint{gp mark 1}{(4.787,6.503)}
\gppoint{gp mark 1}{(4.794,6.529)}
\gppoint{gp mark 1}{(4.802,6.521)}
\gppoint{gp mark 1}{(4.810,6.537)}
\gppoint{gp mark 1}{(4.818,6.532)}
\gppoint{gp mark 1}{(4.826,6.509)}
\gppoint{gp mark 1}{(4.833,6.534)}
\gppoint{gp mark 1}{(4.841,6.524)}
\gppoint{gp mark 1}{(4.849,6.533)}
\gppoint{gp mark 1}{(4.857,6.538)}
\gppoint{gp mark 1}{(4.864,6.551)}
\gppoint{gp mark 1}{(4.872,6.519)}
\gppoint{gp mark 1}{(4.880,6.537)}
\gppoint{gp mark 1}{(4.888,6.534)}
\gppoint{gp mark 1}{(4.896,6.527)}
\gppoint{gp mark 1}{(4.903,6.537)}
\gppoint{gp mark 1}{(4.911,6.524)}
\gppoint{gp mark 1}{(4.919,6.553)}
\gppoint{gp mark 1}{(4.927,6.522)}
\gppoint{gp mark 1}{(4.935,6.528)}
\gppoint{gp mark 1}{(4.942,6.560)}
\gppoint{gp mark 1}{(4.950,6.541)}
\gppoint{gp mark 1}{(4.958,6.516)}
\gppoint{gp mark 1}{(4.966,6.531)}
\gppoint{gp mark 1}{(4.974,6.537)}
\gppoint{gp mark 1}{(4.981,6.537)}
\gppoint{gp mark 1}{(4.989,6.538)}
\gppoint{gp mark 1}{(4.997,6.531)}
\gppoint{gp mark 1}{(5.005,6.515)}
\gppoint{gp mark 1}{(5.012,6.516)}
\gppoint{gp mark 1}{(5.020,6.523)}
\gppoint{gp mark 1}{(5.028,6.517)}
\gppoint{gp mark 1}{(5.036,6.503)}
\gppoint{gp mark 1}{(5.044,6.514)}
\gppoint{gp mark 1}{(5.051,6.521)}
\gppoint{gp mark 1}{(5.059,6.538)}
\gppoint{gp mark 1}{(5.067,6.540)}
\gppoint{gp mark 1}{(5.075,6.483)}
\gppoint{gp mark 1}{(5.083,6.507)}
\gppoint{gp mark 1}{(5.090,6.521)}
\gppoint{gp mark 1}{(5.098,6.506)}
\gppoint{gp mark 1}{(5.106,6.527)}
\gppoint{gp mark 1}{(5.114,6.505)}
\gppoint{gp mark 1}{(5.122,6.521)}
\gppoint{gp mark 1}{(5.129,6.518)}
\gppoint{gp mark 1}{(5.137,6.520)}
\gppoint{gp mark 1}{(5.145,6.513)}
\gppoint{gp mark 1}{(5.153,6.498)}
\gppoint{gp mark 1}{(5.160,6.538)}
\gppoint{gp mark 1}{(5.168,6.529)}
\gppoint{gp mark 1}{(5.176,6.481)}
\gppoint{gp mark 1}{(5.184,6.562)}
\gppoint{gp mark 1}{(5.192,6.523)}
\gppoint{gp mark 1}{(5.199,6.531)}
\gppoint{gp mark 1}{(5.207,6.535)}
\gppoint{gp mark 1}{(5.215,6.517)}
\gppoint{gp mark 1}{(5.223,6.532)}
\gppoint{gp mark 1}{(5.231,6.548)}
\gppoint{gp mark 1}{(5.238,6.532)}
\gppoint{gp mark 1}{(5.246,6.530)}
\gppoint{gp mark 1}{(5.254,6.559)}
\gppoint{gp mark 1}{(5.262,6.497)}
\gppoint{gp mark 1}{(5.270,6.547)}
\gppoint{gp mark 1}{(5.277,6.539)}
\gppoint{gp mark 1}{(5.285,6.525)}
\gppoint{gp mark 1}{(5.293,6.499)}
\gppoint{gp mark 1}{(5.301,6.507)}
\gppoint{gp mark 1}{(5.308,6.534)}
\gppoint{gp mark 1}{(5.316,6.548)}
\gppoint{gp mark 1}{(5.324,6.515)}
\gppoint{gp mark 1}{(5.332,6.511)}
\gppoint{gp mark 1}{(5.340,6.545)}
\gppoint{gp mark 1}{(5.347,6.505)}
\gppoint{gp mark 1}{(5.355,6.543)}
\gppoint{gp mark 1}{(5.363,6.498)}
\gppoint{gp mark 1}{(5.371,6.507)}
\gppoint{gp mark 1}{(5.379,6.507)}
\gppoint{gp mark 1}{(5.386,6.523)}
\gppoint{gp mark 1}{(5.394,6.514)}
\gppoint{gp mark 1}{(5.402,6.512)}
\gppoint{gp mark 1}{(5.410,6.512)}
\gppoint{gp mark 1}{(5.418,6.536)}
\gppoint{gp mark 1}{(5.425,6.512)}
\gppoint{gp mark 1}{(5.433,6.549)}
\gppoint{gp mark 1}{(5.441,6.478)}
\gppoint{gp mark 1}{(5.449,6.518)}
\gppoint{gp mark 1}{(5.456,6.526)}
\gppoint{gp mark 1}{(5.464,6.535)}
\gppoint{gp mark 1}{(5.472,6.512)}
\gppoint{gp mark 1}{(5.480,6.513)}
\gppoint{gp mark 1}{(5.488,6.516)}
\gppoint{gp mark 1}{(5.495,6.529)}
\gppoint{gp mark 1}{(5.503,6.512)}
\gppoint{gp mark 1}{(5.511,6.487)}
\gppoint{gp mark 1}{(5.519,6.517)}
\gppoint{gp mark 1}{(5.527,6.534)}
\gppoint{gp mark 1}{(5.534,6.526)}
\gppoint{gp mark 1}{(5.542,6.512)}
\gppoint{gp mark 1}{(5.550,6.530)}
\gppoint{gp mark 1}{(5.558,6.513)}
\gppoint{gp mark 1}{(5.566,6.519)}
\gppoint{gp mark 1}{(5.573,6.526)}
\gppoint{gp mark 1}{(5.581,6.511)}
\gppoint{gp mark 1}{(5.589,6.519)}
\gppoint{gp mark 1}{(5.597,6.530)}
\gppoint{gp mark 1}{(5.605,6.528)}
\gppoint{gp mark 1}{(5.612,6.528)}
\gppoint{gp mark 1}{(5.620,6.525)}
\gppoint{gp mark 1}{(5.628,6.494)}
\gppoint{gp mark 1}{(5.636,6.508)}
\gppoint{gp mark 1}{(5.643,6.516)}
\gppoint{gp mark 1}{(5.651,6.509)}
\gppoint{gp mark 1}{(5.659,6.486)}
\gppoint{gp mark 1}{(5.667,6.512)}
\gppoint{gp mark 1}{(5.675,6.515)}
\gppoint{gp mark 1}{(5.682,6.530)}
\gppoint{gp mark 1}{(5.690,6.556)}
\gppoint{gp mark 1}{(5.698,6.517)}
\gppoint{gp mark 1}{(5.706,6.533)}
\gppoint{gp mark 1}{(5.714,6.497)}
\gppoint{gp mark 1}{(5.721,6.510)}
\gppoint{gp mark 1}{(5.729,6.529)}
\gppoint{gp mark 1}{(5.737,6.512)}
\gppoint{gp mark 1}{(5.745,6.513)}
\gppoint{gp mark 1}{(5.753,6.517)}
\gppoint{gp mark 1}{(5.760,6.525)}
\gppoint{gp mark 1}{(5.768,6.530)}
\gppoint{gp mark 1}{(5.776,6.514)}
\gppoint{gp mark 1}{(5.784,6.507)}
\gppoint{gp mark 1}{(5.791,6.491)}
\gppoint{gp mark 1}{(5.799,6.532)}
\gppoint{gp mark 1}{(5.807,6.507)}
\gppoint{gp mark 1}{(5.815,6.503)}
\gppoint{gp mark 1}{(5.823,6.500)}
\gppoint{gp mark 1}{(5.830,6.498)}
\gppoint{gp mark 1}{(5.838,6.539)}
\gppoint{gp mark 1}{(5.846,6.511)}
\gppoint{gp mark 1}{(5.854,6.509)}
\gppoint{gp mark 1}{(5.862,6.489)}
\gppoint{gp mark 1}{(5.869,6.516)}
\gppoint{gp mark 1}{(5.877,6.511)}
\gppoint{gp mark 1}{(5.885,6.509)}
\gppoint{gp mark 1}{(5.893,6.496)}
\gppoint{gp mark 1}{(5.901,6.529)}
\gppoint{gp mark 1}{(5.908,6.529)}
\gppoint{gp mark 1}{(5.916,6.541)}
\gppoint{gp mark 1}{(5.924,6.492)}
\gppoint{gp mark 1}{(5.932,6.516)}
\gppoint{gp mark 1}{(5.939,6.525)}
\gppoint{gp mark 1}{(5.947,6.537)}
\gppoint{gp mark 1}{(5.955,6.505)}
\gppoint{gp mark 1}{(5.963,6.522)}
\gppoint{gp mark 1}{(5.971,6.492)}
\gppoint{gp mark 1}{(5.978,6.522)}
\gppoint{gp mark 1}{(5.986,6.529)}
\gppoint{gp mark 1}{(5.994,6.504)}
\gppoint{gp mark 1}{(6.002,6.509)}
\gppoint{gp mark 1}{(6.010,6.493)}
\gppoint{gp mark 1}{(6.017,6.497)}
\gppoint{gp mark 1}{(6.025,6.502)}
\gppoint{gp mark 1}{(6.033,6.479)}
\gppoint{gp mark 1}{(6.041,6.506)}
\gppoint{gp mark 1}{(6.049,6.500)}
\gppoint{gp mark 1}{(6.056,6.505)}
\gppoint{gp mark 1}{(6.064,6.483)}
\gppoint{gp mark 1}{(6.072,6.500)}
\gppoint{gp mark 1}{(6.080,6.489)}
\gppoint{gp mark 1}{(6.087,6.515)}
\gppoint{gp mark 1}{(6.095,6.492)}
\gppoint{gp mark 1}{(6.103,6.515)}
\gppoint{gp mark 1}{(6.111,6.523)}
\gppoint{gp mark 1}{(6.119,6.518)}
\gppoint{gp mark 1}{(6.126,6.516)}
\gppoint{gp mark 1}{(6.134,6.504)}
\gppoint{gp mark 1}{(6.142,6.528)}
\gppoint{gp mark 1}{(6.150,6.506)}
\gppoint{gp mark 1}{(6.158,6.499)}
\gppoint{gp mark 1}{(6.165,6.496)}
\gppoint{gp mark 1}{(6.173,6.494)}
\gppoint{gp mark 1}{(6.181,6.522)}
\gppoint{gp mark 1}{(6.189,6.521)}
\gppoint{gp mark 1}{(6.197,6.498)}
\gppoint{gp mark 1}{(6.204,6.519)}
\gppoint{gp mark 1}{(6.212,6.525)}
\gppoint{gp mark 1}{(6.220,6.553)}
\gppoint{gp mark 1}{(6.228,6.523)}
\gppoint{gp mark 1}{(6.235,6.504)}
\gppoint{gp mark 1}{(6.243,6.545)}
\gppoint{gp mark 1}{(6.251,6.502)}
\gppoint{gp mark 1}{(6.259,6.512)}
\gppoint{gp mark 1}{(6.267,6.474)}
\gppoint{gp mark 1}{(6.274,6.506)}
\gppoint{gp mark 1}{(6.282,6.519)}
\gppoint{gp mark 1}{(6.290,6.521)}
\gppoint{gp mark 1}{(6.298,6.524)}
\gppoint{gp mark 1}{(6.306,6.502)}
\gppoint{gp mark 1}{(6.313,6.526)}
\gppoint{gp mark 1}{(6.321,6.524)}
\gppoint{gp mark 1}{(6.329,6.505)}
\gppoint{gp mark 1}{(6.337,6.528)}
\gppoint{gp mark 1}{(6.345,6.494)}
\gppoint{gp mark 1}{(6.352,6.481)}
\gppoint{gp mark 1}{(6.360,6.531)}
\gppoint{gp mark 1}{(6.368,6.515)}
\gppoint{gp mark 1}{(6.376,6.502)}
\gppoint{gp mark 1}{(6.384,6.512)}
\gppoint{gp mark 1}{(6.391,6.495)}
\gppoint{gp mark 1}{(6.399,6.533)}
\gppoint{gp mark 1}{(6.407,6.533)}
\gppoint{gp mark 1}{(6.415,6.509)}
\gppoint{gp mark 1}{(6.422,6.534)}
\gppoint{gp mark 1}{(6.430,6.518)}
\gppoint{gp mark 1}{(6.438,6.519)}
\gppoint{gp mark 1}{(6.446,6.500)}
\gppoint{gp mark 1}{(6.454,6.481)}
\gppoint{gp mark 1}{(6.461,6.525)}
\gppoint{gp mark 1}{(6.469,6.493)}
\gppoint{gp mark 1}{(6.477,6.510)}
\gppoint{gp mark 1}{(6.485,6.472)}
\gppoint{gp mark 1}{(6.493,6.501)}
\gppoint{gp mark 1}{(6.500,6.510)}
\gppoint{gp mark 1}{(6.508,6.511)}
\gppoint{gp mark 1}{(6.516,6.494)}
\gppoint{gp mark 1}{(6.524,6.494)}
\gppoint{gp mark 1}{(6.532,6.508)}
\gppoint{gp mark 1}{(6.539,6.508)}
\gppoint{gp mark 1}{(6.547,6.502)}
\gppoint{gp mark 1}{(6.555,6.508)}
\gppoint{gp mark 1}{(6.563,6.524)}
\gppoint{gp mark 1}{(6.570,6.498)}
\gppoint{gp mark 1}{(6.578,6.529)}
\gppoint{gp mark 1}{(6.586,6.506)}
\gppoint{gp mark 1}{(6.594,6.508)}
\gppoint{gp mark 1}{(6.602,6.518)}
\gppoint{gp mark 1}{(6.609,6.524)}
\gppoint{gp mark 1}{(6.617,6.506)}
\gppoint{gp mark 1}{(6.625,6.517)}
\gppoint{gp mark 1}{(6.633,6.507)}
\gppoint{gp mark 1}{(6.641,6.516)}
\gppoint{gp mark 1}{(6.648,6.502)}
\gppoint{gp mark 1}{(6.656,6.485)}
\gppoint{gp mark 1}{(6.664,6.518)}
\gppoint{gp mark 1}{(6.672,6.499)}
\gppoint{gp mark 1}{(6.680,6.512)}
\gppoint{gp mark 1}{(6.687,6.502)}
\gppoint{gp mark 1}{(6.695,6.499)}
\gppoint{gp mark 1}{(6.703,6.522)}
\gppoint{gp mark 1}{(6.711,6.518)}
\gppoint{gp mark 1}{(6.718,6.503)}
\gppoint{gp mark 1}{(6.726,6.488)}
\gppoint{gp mark 1}{(6.734,6.500)}
\gppoint{gp mark 1}{(6.742,6.509)}
\gppoint{gp mark 1}{(6.750,6.494)}
\gppoint{gp mark 1}{(6.757,6.504)}
\gppoint{gp mark 1}{(6.765,6.481)}
\gppoint{gp mark 1}{(6.773,6.489)}
\gppoint{gp mark 1}{(6.781,6.508)}
\gppoint{gp mark 1}{(6.789,6.517)}
\gppoint{gp mark 1}{(6.796,6.527)}
\gppoint{gp mark 1}{(6.804,6.463)}
\gppoint{gp mark 1}{(6.812,6.489)}
\gppoint{gp mark 1}{(6.820,6.501)}
\gppoint{gp mark 1}{(6.828,6.508)}
\gppoint{gp mark 1}{(6.835,6.481)}
\gppoint{gp mark 1}{(6.843,6.504)}
\gppoint{gp mark 1}{(6.851,6.477)}
\gppoint{gp mark 1}{(6.859,6.503)}
\gppoint{gp mark 1}{(6.866,6.490)}
\gppoint{gp mark 1}{(6.874,6.521)}
\gppoint{gp mark 1}{(6.882,6.497)}
\gppoint{gp mark 1}{(6.890,6.529)}
\gppoint{gp mark 1}{(6.898,6.495)}
\gppoint{gp mark 1}{(6.905,6.482)}
\gppoint{gp mark 1}{(6.913,6.503)}
\gppoint{gp mark 1}{(6.921,6.498)}
\gppoint{gp mark 1}{(6.929,6.519)}
\gppoint{gp mark 1}{(6.937,6.493)}
\gppoint{gp mark 1}{(6.944,6.516)}
\gppoint{gp mark 1}{(6.952,6.475)}
\gppoint{gp mark 1}{(6.960,6.502)}
\gppoint{gp mark 1}{(6.968,6.511)}
\gppoint{gp mark 1}{(6.976,6.512)}
\gppoint{gp mark 1}{(6.983,6.474)}
\gppoint{gp mark 1}{(6.991,6.497)}
\gppoint{gp mark 1}{(6.999,6.510)}
\gppoint{gp mark 1}{(7.007,6.496)}
\gppoint{gp mark 1}{(7.014,6.479)}
\gppoint{gp mark 1}{(7.022,6.524)}
\gppoint{gp mark 1}{(7.030,6.475)}
\gppoint{gp mark 1}{(7.038,6.496)}
\gppoint{gp mark 1}{(7.046,6.481)}
\gppoint{gp mark 1}{(7.053,6.497)}
\gppoint{gp mark 1}{(7.061,6.500)}
\gppoint{gp mark 1}{(7.069,6.510)}
\gppoint{gp mark 1}{(7.077,6.496)}
\gppoint{gp mark 1}{(7.085,6.521)}
\gppoint{gp mark 1}{(7.092,6.509)}
\gppoint{gp mark 1}{(7.100,6.500)}
\gppoint{gp mark 1}{(7.108,6.502)}
\gppoint{gp mark 1}{(7.116,6.505)}
\gppoint{gp mark 1}{(7.124,6.535)}
\gppoint{gp mark 1}{(7.131,6.512)}
\gppoint{gp mark 1}{(7.139,6.502)}
\gppoint{gp mark 1}{(7.147,6.491)}
\gppoint{gp mark 1}{(7.155,6.488)}
\gppoint{gp mark 1}{(7.163,6.495)}
\gppoint{gp mark 1}{(7.170,6.481)}
\gppoint{gp mark 1}{(7.178,6.479)}
\gppoint{gp mark 1}{(7.186,6.493)}
\gppoint{gp mark 1}{(7.194,6.518)}
\gppoint{gp mark 1}{(7.201,6.502)}
\gppoint{gp mark 1}{(7.209,6.552)}
\gppoint{gp mark 1}{(7.217,6.499)}
\gppoint{gp mark 1}{(7.225,6.528)}
\gppoint{gp mark 1}{(7.233,6.552)}
\gppoint{gp mark 1}{(7.240,6.543)}
\gppoint{gp mark 1}{(7.248,6.519)}
\gppoint{gp mark 1}{(7.256,6.544)}
\gppoint{gp mark 1}{(7.264,6.521)}
\gppoint{gp mark 1}{(7.272,6.519)}
\gppoint{gp mark 1}{(7.279,6.570)}
\gppoint{gp mark 1}{(7.287,6.551)}
\gppoint{gp mark 1}{(7.295,6.519)}
\gppoint{gp mark 1}{(7.303,6.531)}
\gppoint{gp mark 1}{(7.311,6.532)}
\gppoint{gp mark 1}{(7.318,6.550)}
\gppoint{gp mark 1}{(7.326,6.563)}
\gppoint{gp mark 1}{(7.334,6.548)}
\gppoint{gp mark 1}{(7.342,6.560)}
\gppoint{gp mark 1}{(7.349,6.553)}
\gppoint{gp mark 1}{(7.357,6.545)}
\gppoint{gp mark 1}{(7.365,6.552)}
\gppoint{gp mark 1}{(7.373,6.525)}
\gppoint{gp mark 1}{(7.381,6.554)}
\gppoint{gp mark 1}{(7.388,6.556)}
\gppoint{gp mark 1}{(7.396,6.543)}
\gppoint{gp mark 1}{(7.404,6.555)}
\gppoint{gp mark 1}{(7.412,6.531)}
\gppoint{gp mark 1}{(7.420,6.550)}
\gppoint{gp mark 1}{(7.427,6.533)}
\gppoint{gp mark 1}{(7.435,6.566)}
\gppoint{gp mark 1}{(7.443,6.544)}
\gppoint{gp mark 1}{(7.451,6.574)}
\gppoint{gp mark 1}{(7.459,6.559)}
\gppoint{gp mark 1}{(7.466,6.532)}
\gppoint{gp mark 1}{(7.474,6.564)}
\gppoint{gp mark 1}{(7.482,6.532)}
\gppoint{gp mark 1}{(7.490,6.530)}
\gppoint{gp mark 1}{(7.497,6.499)}
\gppoint{gp mark 1}{(7.505,6.488)}
\gppoint{gp mark 1}{(7.513,6.486)}
\gppoint{gp mark 1}{(7.521,6.521)}
\gppoint{gp mark 1}{(7.529,6.502)}
\gppoint{gp mark 1}{(7.536,6.494)}
\gppoint{gp mark 1}{(7.544,6.449)}
\gppoint{gp mark 1}{(7.552,6.350)}
\gppoint{gp mark 1}{(7.560,5.819)}
\gppoint{gp mark 1}{(7.568,5.288)}
\gppoint{gp mark 1}{(7.575,5.220)}
\gppoint{gp mark 1}{(7.583,5.132)}
\gppoint{gp mark 1}{(7.591,4.855)}
\gppoint{gp mark 1}{(7.599,4.932)}
\gppoint{gp mark 1}{(7.607,4.950)}
\gppoint{gp mark 1}{(7.614,4.938)}
\gppoint{gp mark 1}{(7.622,4.962)}
\gppoint{gp mark 1}{(7.630,4.882)}
\gppoint{gp mark 1}{(7.638,4.843)}
\gppoint{gp mark 1}{(7.645,4.825)}
\gppoint{gp mark 1}{(7.653,4.801)}
\gppoint{gp mark 1}{(7.661,4.832)}
\gppoint{gp mark 1}{(7.669,4.789)}
\gppoint{gp mark 1}{(7.677,4.801)}
\gppoint{gp mark 1}{(7.684,4.818)}
\gppoint{gp mark 1}{(7.692,4.888)}
\gppoint{gp mark 1}{(7.700,4.869)}
\gppoint{gp mark 1}{(7.708,4.893)}
\gppoint{gp mark 1}{(7.716,5.030)}
\gppoint{gp mark 1}{(7.723,5.070)}
\gppoint{gp mark 1}{(7.731,5.034)}
\gppoint{gp mark 1}{(7.739,5.002)}
\gppoint{gp mark 1}{(7.747,4.888)}
\gppoint{gp mark 1}{(7.755,4.527)}
\gppoint{gp mark 1}{(7.762,4.183)}
\gppoint{gp mark 1}{(7.770,3.927)}
\gppoint{gp mark 1}{(7.778,3.524)}
\gppoint{gp mark 1}{(7.786,3.077)}
\gppoint{gp mark 1}{(7.793,2.811)}
\gppoint{gp mark 1}{(7.801,2.643)}
\gppoint{gp mark 1}{(7.809,2.437)}
\gppoint{gp mark 1}{(7.817,2.317)}
\gppoint{gp mark 1}{(7.825,2.218)}
\gppoint{gp mark 1}{(7.832,2.101)}
\gppoint{gp mark 1}{(7.840,2.034)}
\gppoint{gp mark 1}{(7.848,1.968)}
\gppoint{gp mark 1}{(7.856,1.873)}
\gppoint{gp mark 1}{(7.864,1.827)}
\gppoint{gp mark 1}{(7.871,1.791)}
\gppoint{gp mark 1}{(7.879,1.752)}
\gppoint{gp mark 1}{(7.887,1.729)}
\gppoint{gp mark 1}{(7.895,1.722)}
\gppoint{gp mark 1}{(7.903,1.724)}
\gppoint{gp mark 1}{(7.910,1.716)}
\gppoint{gp mark 1}{(7.918,1.667)}
\gppoint{gp mark 1}{(7.926,1.687)}
\gppoint{gp mark 1}{(7.934,1.701)}
\gppoint{gp mark 1}{(7.942,1.684)}
\gppoint{gp mark 1}{(7.949,1.671)}
\gppoint{gp mark 1}{(7.957,1.724)}
\gppoint{gp mark 1}{(7.965,1.662)}
\gppoint{gp mark 1}{(7.973,1.666)}
\gppoint{gp mark 1}{(7.980,1.674)}
\gppoint{gp mark 1}{(7.988,1.692)}
\gppoint{gp mark 1}{(7.996,1.697)}
\gppoint{gp mark 1}{(8.004,1.676)}
\gppoint{gp mark 1}{(8.012,1.672)}
\gppoint{gp mark 1}{(8.019,1.671)}
\gppoint{gp mark 1}{(8.027,1.640)}
\gppoint{gp mark 1}{(8.035,1.662)}
\gppoint{gp mark 1}{(8.043,1.666)}
\gppoint{gp mark 1}{(8.051,1.669)}
\gppoint{gp mark 1}{(8.058,1.665)}
\gppoint{gp mark 1}{(8.066,1.677)}
\gppoint{gp mark 1}{(8.074,1.664)}
\gppoint{gp mark 1}{(8.082,1.631)}
\gppoint{gp mark 1}{(8.090,1.658)}
\gppoint{gp mark 1}{(8.097,1.625)}
\gppoint{gp mark 1}{(8.105,1.619)}
\gppoint{gp mark 1}{(8.113,1.635)}
\gppoint{gp mark 1}{(8.121,1.641)}
\gppoint{gp mark 1}{(8.128,1.630)}
\gppoint{gp mark 1}{(8.136,1.630)}
\gppoint{gp mark 1}{(8.144,1.628)}
\gppoint{gp mark 1}{(8.152,1.615)}
\gppoint{gp mark 1}{(8.160,1.635)}
\gppoint{gp mark 1}{(8.167,1.612)}
\gppoint{gp mark 1}{(8.175,1.608)}
\gppoint{gp mark 1}{(8.183,1.601)}
\gppoint{gp mark 1}{(8.191,1.603)}
\gppoint{gp mark 1}{(8.199,1.622)}
\gppoint{gp mark 1}{(8.206,1.611)}
\gppoint{gp mark 1}{(8.214,1.607)}
\gppoint{gp mark 1}{(8.222,1.585)}
\gppoint{gp mark 1}{(8.230,1.597)}
\gppoint{gp mark 1}{(8.238,1.592)}
\gppoint{gp mark 1}{(8.245,1.597)}
\gppoint{gp mark 1}{(8.253,1.599)}
\gppoint{gp mark 1}{(8.261,1.600)}
\gppoint{gp mark 1}{(8.269,1.586)}
\gppoint{gp mark 1}{(8.276,1.579)}
\gppoint{gp mark 1}{(8.284,1.582)}
\gppoint{gp mark 1}{(8.292,1.599)}
\gppoint{gp mark 1}{(8.300,1.599)}
\gppoint{gp mark 1}{(8.308,1.593)}
\gppoint{gp mark 1}{(8.315,1.583)}
\gppoint{gp mark 1}{(8.323,1.588)}
\gppoint{gp mark 1}{(8.331,1.577)}
\gppoint{gp mark 1}{(8.339,1.593)}
\gppoint{gp mark 1}{(8.347,1.582)}
\gppoint{gp mark 1}{(8.354,1.583)}
\gppoint{gp mark 1}{(8.362,1.584)}
\gppoint{gp mark 1}{(8.370,1.587)}
\gppoint{gp mark 1}{(8.378,1.572)}
\gppoint{gp mark 1}{(8.386,1.566)}
\gppoint{gp mark 1}{(8.393,1.579)}
\gppoint{gp mark 1}{(8.401,1.585)}
\gppoint{gp mark 1}{(8.409,1.589)}
\gppoint{gp mark 1}{(8.417,1.561)}
\gppoint{gp mark 1}{(8.424,1.543)}
\gppoint{gp mark 1}{(8.432,1.594)}
\gppoint{gp mark 1}{(8.440,1.582)}
\gppoint{gp mark 1}{(8.448,1.567)}
\gppoint{gp mark 1}{(8.456,1.559)}
\gppoint{gp mark 1}{(8.463,1.571)}
\gppoint{gp mark 1}{(8.471,1.561)}
\gppoint{gp mark 1}{(8.479,1.567)}
\gppoint{gp mark 1}{(8.487,1.552)}
\gppoint{gp mark 1}{(8.495,1.560)}
\gppoint{gp mark 1}{(8.502,1.560)}
\gppoint{gp mark 1}{(8.510,1.570)}
\gppoint{gp mark 1}{(8.518,1.551)}
\gppoint{gp mark 1}{(8.526,1.553)}
\gppoint{gp mark 1}{(8.534,1.516)}
\gppoint{gp mark 1}{(8.541,1.562)}
\gppoint{gp mark 1}{(8.549,1.546)}
\gppoint{gp mark 1}{(8.557,1.566)}
\gppoint{gp mark 1}{(8.565,1.546)}
\gppoint{gp mark 1}{(8.572,1.557)}
\gppoint{gp mark 1}{(8.580,1.549)}
\gppoint{gp mark 1}{(8.588,1.538)}
\gppoint{gp mark 1}{(8.596,1.538)}
\gppoint{gp mark 1}{(8.604,1.561)}
\gppoint{gp mark 1}{(8.611,1.547)}
\gppoint{gp mark 1}{(8.619,1.534)}
\gppoint{gp mark 1}{(8.627,1.557)}
\gppoint{gp mark 1}{(8.635,1.558)}
\gppoint{gp mark 1}{(8.643,1.560)}
\gppoint{gp mark 1}{(8.650,1.538)}
\gppoint{gp mark 1}{(8.658,1.536)}
\gppoint{gp mark 1}{(8.666,1.511)}
\gppoint{gp mark 1}{(8.674,1.541)}
\gppoint{gp mark 1}{(8.682,1.539)}
\gppoint{gp mark 1}{(8.689,1.533)}
\gppoint{gp mark 1}{(8.697,1.544)}
\gppoint{gp mark 1}{(8.705,1.523)}
\gppoint{gp mark 1}{(8.713,1.546)}
\gppoint{gp mark 1}{(8.721,1.553)}
\gppoint{gp mark 1}{(8.728,1.543)}
\gppoint{gp mark 1}{(8.736,1.560)}
\gppoint{gp mark 1}{(8.744,1.534)}
\gppoint{gp mark 1}{(8.752,1.531)}
\gppoint{gp mark 1}{(8.759,1.550)}
\gppoint{gp mark 1}{(8.767,1.535)}
\gppoint{gp mark 1}{(8.775,1.535)}
\gppoint{gp mark 1}{(8.783,1.537)}
\gppoint{gp mark 1}{(8.791,1.568)}
\gppoint{gp mark 1}{(8.798,1.543)}
\gppoint{gp mark 1}{(8.806,1.527)}
\gppoint{gp mark 1}{(8.814,1.542)}
\gppoint{gp mark 1}{(8.822,1.558)}
\gppoint{gp mark 1}{(8.830,1.513)}
\gppoint{gp mark 1}{(8.837,1.539)}
\gppoint{gp mark 1}{(8.845,1.508)}
\gppoint{gp mark 1}{(8.853,1.527)}
\gppoint{gp mark 1}{(8.861,1.561)}
\gppoint{gp mark 1}{(8.869,1.527)}
\gppoint{gp mark 1}{(8.876,1.539)}
\gppoint{gp mark 1}{(8.884,1.563)}
\gppoint{gp mark 1}{(8.892,1.547)}
\gppoint{gp mark 1}{(8.900,1.539)}
\gppoint{gp mark 1}{(8.907,1.536)}
\gppoint{gp mark 1}{(8.915,1.524)}
\gppoint{gp mark 1}{(8.923,1.544)}
\gppoint{gp mark 1}{(8.931,1.533)}
\gppoint{gp mark 1}{(8.939,1.529)}
\gppoint{gp mark 1}{(8.946,1.549)}
\gppoint{gp mark 1}{(8.954,1.545)}
\gppoint{gp mark 1}{(8.962,1.536)}
\gppoint{gp mark 1}{(8.970,1.542)}
\gppoint{gp mark 1}{(8.978,1.560)}
\gppoint{gp mark 1}{(8.985,1.535)}
\gppoint{gp mark 1}{(8.993,1.532)}
\gppoint{gp mark 1}{(9.001,1.541)}
\gppoint{gp mark 1}{(9.009,1.551)}
\gppoint{gp mark 1}{(9.017,1.532)}
\gppoint{gp mark 1}{(9.024,1.552)}
\gppoint{gp mark 1}{(9.032,1.528)}
\gppoint{gp mark 1}{(9.040,1.540)}
\gppoint{gp mark 1}{(9.048,1.532)}
\gppoint{gp mark 1}{(9.055,1.566)}
\gppoint{gp mark 1}{(9.063,1.517)}
\gppoint{gp mark 1}{(9.071,1.543)}
\gppoint{gp mark 1}{(9.079,1.527)}
\gppoint{gp mark 1}{(9.087,1.548)}
\gppoint{gp mark 1}{(9.094,1.538)}
\gppoint{gp mark 1}{(9.102,1.533)}
\gppoint{gp mark 1}{(9.110,1.537)}
\gppoint{gp mark 1}{(9.118,1.534)}
\gppoint{gp mark 1}{(9.126,1.533)}
\gppoint{gp mark 1}{(9.133,1.556)}
\gppoint{gp mark 1}{(9.141,1.568)}
\gppoint{gp mark 1}{(9.149,1.526)}
\gppoint{gp mark 1}{(9.157,1.525)}
\gppoint{gp mark 1}{(9.165,1.524)}
\gppoint{gp mark 1}{(9.172,1.536)}
\gppoint{gp mark 1}{(9.180,1.524)}
\gppoint{gp mark 1}{(9.188,1.509)}
\gppoint{gp mark 1}{(9.196,1.520)}
\gppoint{gp mark 1}{(9.203,1.539)}
\gppoint{gp mark 1}{(9.211,1.546)}
\gppoint{gp mark 1}{(9.219,1.522)}
\gppoint{gp mark 1}{(9.227,1.535)}
\gppoint{gp mark 1}{(9.235,1.539)}
\gppoint{gp mark 1}{(9.242,1.551)}
\gppoint{gp mark 1}{(9.250,1.524)}
\gppoint{gp mark 1}{(9.258,1.536)}
\gppoint{gp mark 1}{(9.266,1.533)}
\gppoint{gp mark 1}{(9.274,1.537)}
\gppoint{gp mark 1}{(9.281,1.513)}
\gppoint{gp mark 1}{(9.289,1.555)}
\gppoint{gp mark 1}{(9.297,1.518)}
\gppoint{gp mark 1}{(9.305,1.528)}
\gppoint{gp mark 1}{(9.313,1.517)}
\gppoint{gp mark 1}{(9.320,1.551)}
\gppoint{gp mark 1}{(9.328,1.531)}
\gppoint{gp mark 1}{(9.336,1.546)}
\gppoint{gp mark 1}{(9.344,1.516)}
\gppoint{gp mark 1}{(9.351,1.523)}
\gppoint{gp mark 1}{(9.359,1.536)}
\gppoint{gp mark 1}{(9.367,1.533)}
\gppoint{gp mark 1}{(9.375,1.507)}
\gppoint{gp mark 1}{(9.383,1.517)}
\gppoint{gp mark 1}{(9.390,1.535)}
\gppoint{gp mark 1}{(9.398,1.544)}
\gppoint{gp mark 1}{(9.406,1.535)}
\gppoint{gp mark 1}{(9.414,1.521)}
\gppoint{gp mark 1}{(9.422,1.526)}
\gppoint{gp mark 1}{(9.429,1.540)}
\gppoint{gp mark 1}{(9.437,1.518)}
\gppoint{gp mark 1}{(9.445,1.541)}
\gppoint{gp mark 1}{(9.453,1.536)}
\gppoint{gp mark 1}{(9.461,1.523)}
\gppoint{gp mark 1}{(9.468,1.517)}
\gppoint{gp mark 1}{(9.476,1.548)}
\gppoint{gp mark 1}{(9.484,1.538)}
\gppoint{gp mark 1}{(9.492,1.523)}
\gppoint{gp mark 1}{(9.500,1.523)}
\gppoint{gp mark 1}{(9.507,1.533)}
\gppoint{gp mark 1}{(9.515,1.536)}
\gppoint{gp mark 1}{(9.523,1.534)}
\gppoint{gp mark 1}{(9.531,1.548)}
\gppoint{gp mark 1}{(9.538,1.541)}
\gppoint{gp mark 1}{(9.546,1.547)}
\gppoint{gp mark 1}{(9.554,1.531)}
\gppoint{gp mark 1}{(9.562,1.522)}
\gppoint{gp mark 1}{(9.570,1.518)}
\gppoint{gp mark 1}{(9.577,1.527)}
\gppoint{gp mark 1}{(9.585,1.545)}
\gppoint{gp mark 1}{(9.593,1.533)}
\gppoint{gp mark 1}{(9.601,1.506)}
\gppoint{gp mark 1}{(9.609,1.523)}
\gppoint{gp mark 1}{(9.616,1.532)}
\gppoint{gp mark 1}{(9.624,1.541)}
\gppoint{gp mark 1}{(9.632,1.543)}
\gppoint{gp mark 1}{(9.640,1.521)}
\gppoint{gp mark 1}{(9.648,1.551)}
\gppoint{gp mark 1}{(9.655,1.516)}
\gppoint{gp mark 1}{(9.663,1.545)}
\gppoint{gp mark 1}{(9.671,1.551)}
\gppoint{gp mark 1}{(9.679,1.534)}
\gppoint{gp mark 1}{(9.686,1.518)}
\gppoint{gp mark 1}{(9.694,1.520)}
\gppoint{gp mark 1}{(9.702,1.527)}
\gppoint{gp mark 1}{(9.710,1.529)}
\gppoint{gp mark 1}{(9.718,1.539)}
\gppoint{gp mark 1}{(9.725,1.531)}
\gppoint{gp mark 1}{(9.733,1.527)}
\gppoint{gp mark 1}{(9.741,1.541)}
\gppoint{gp mark 1}{(9.749,1.546)}
\gppoint{gp mark 1}{(9.757,1.496)}
\gppoint{gp mark 1}{(9.764,1.531)}
\gppoint{gp mark 1}{(9.772,1.525)}
\gppoint{gp mark 1}{(9.780,1.530)}
\gppoint{gp mark 1}{(9.788,1.566)}
\gppoint{gp mark 1}{(9.796,1.546)}
\gppoint{gp mark 1}{(9.803,1.516)}
\gppoint{gp mark 1}{(9.811,1.507)}
\gppoint{gp mark 1}{(9.819,1.530)}
\gppoint{gp mark 1}{(9.827,1.543)}
\gppoint{gp mark 1}{(9.834,1.538)}
\gppoint{gp mark 1}{(9.842,1.513)}
\gppoint{gp mark 1}{(9.850,1.521)}
\gppoint{gp mark 1}{(9.858,1.545)}
\gppoint{gp mark 1}{(9.866,1.549)}
\gppoint{gp mark 1}{(9.873,1.535)}
\gppoint{gp mark 1}{(9.881,1.547)}
\gppoint{gp mark 1}{(9.889,1.517)}
\gppoint{gp mark 1}{(9.897,1.523)}
\gppoint{gp mark 1}{(9.905,1.555)}
\gppoint{gp mark 1}{(9.912,1.521)}
\gppoint{gp mark 1}{(9.920,1.528)}
\gppoint{gp mark 1}{(9.928,1.529)}
\gppoint{gp mark 1}{(9.936,1.541)}
\gppoint{gp mark 1}{(9.944,1.538)}
\gppoint{gp mark 1}{(9.951,1.546)}
\gppoint{gp mark 1}{(9.959,1.542)}
\gppoint{gp mark 1}{(9.967,1.498)}
\gppoint{gp mark 1}{(9.975,1.527)}
\gppoint{gp mark 1}{(9.982,1.525)}
\gppoint{gp mark 1}{(9.990,1.541)}
\gppoint{gp mark 1}{(9.998,1.540)}
\gppoint{gp mark 1}{(10.006,1.563)}
\gppoint{gp mark 1}{(10.014,1.525)}
\gppoint{gp mark 1}{(10.021,1.539)}
\gppoint{gp mark 1}{(10.029,1.509)}
\gppoint{gp mark 1}{(10.037,1.518)}
\gppoint{gp mark 1}{(10.045,1.524)}
\gppoint{gp mark 1}{(10.053,1.525)}
\gppoint{gp mark 1}{(10.060,1.546)}
\gppoint{gp mark 1}{(10.068,1.528)}
\gppoint{gp mark 1}{(10.076,1.518)}
\gppoint{gp mark 1}{(10.084,1.538)}
\gppoint{gp mark 1}{(10.092,1.539)}
\gppoint{gp mark 1}{(10.099,1.535)}
\gppoint{gp mark 1}{(10.107,1.518)}
\gppoint{gp mark 1}{(10.115,1.534)}
\gppoint{gp mark 1}{(10.123,1.541)}
\gppoint{gp mark 1}{(10.130,1.517)}
\gppoint{gp mark 1}{(10.138,1.511)}
\gppoint{gp mark 1}{(10.146,1.538)}
\gppoint{gp mark 1}{(10.154,1.534)}
\gppoint{gp mark 1}{(10.162,1.530)}
\gppoint{gp mark 1}{(10.169,1.535)}
\gppoint{gp mark 1}{(10.177,1.537)}
\gppoint{gp mark 1}{(10.185,1.520)}
\gppoint{gp mark 1}{(10.193,1.509)}
\gppoint{gp mark 1}{(10.201,1.533)}
\gppoint{gp mark 1}{(10.208,1.520)}
\gppoint{gp mark 1}{(10.216,1.529)}
\gppoint{gp mark 1}{(10.224,1.544)}
\gppoint{gp mark 1}{(10.232,1.533)}
\gppoint{gp mark 1}{(10.240,1.536)}
\gppoint{gp mark 1}{(10.247,1.534)}
\gppoint{gp mark 1}{(10.255,1.525)}
\gppoint{gp mark 1}{(10.263,1.521)}
\gppoint{gp mark 1}{(10.271,1.530)}
\gppoint{gp mark 1}{(10.279,1.533)}
\gppoint{gp mark 1}{(10.286,1.534)}
\gppoint{gp mark 1}{(10.294,1.503)}
\gppoint{gp mark 1}{(10.302,1.503)}
\gppoint{gp mark 1}{(10.310,1.519)}
\gppoint{gp mark 1}{(10.317,1.530)}
\gppoint{gp mark 1}{(10.325,1.529)}
\gppoint{gp mark 1}{(10.333,1.528)}
\gppoint{gp mark 1}{(10.341,1.537)}
\gppoint{gp mark 1}{(10.349,1.505)}
\gppoint{gp mark 1}{(10.356,1.534)}
\gppoint{gp mark 1}{(10.364,1.503)}
\gppoint{gp mark 1}{(10.372,1.523)}
\gppoint{gp mark 1}{(10.380,1.513)}
\gppoint{gp mark 1}{(10.388,1.527)}
\gppoint{gp mark 1}{(10.395,1.509)}
\gppoint{gp mark 1}{(10.403,1.520)}
\gppoint{gp mark 1}{(10.411,1.510)}
\gppoint{gp mark 1}{(10.419,1.523)}
\gppoint{gp mark 1}{(10.427,1.536)}
\gppoint{gp mark 1}{(10.434,1.512)}
\gppoint{gp mark 1}{(10.442,1.535)}
\gppoint{gp mark 1}{(10.450,1.528)}
\gppoint{gp mark 1}{(10.458,1.533)}
\gppoint{gp mark 1}{(10.465,1.524)}
\gppoint{gp mark 1}{(10.473,1.516)}
\gppoint{gp mark 1}{(10.481,1.501)}
\gppoint{gp mark 1}{(10.489,1.519)}
\gppoint{gp mark 1}{(10.497,1.523)}
\gppoint{gp mark 1}{(10.504,1.507)}
\gppoint{gp mark 1}{(10.512,1.501)}
\gppoint{gp mark 1}{(10.520,1.524)}
\gppoint{gp mark 1}{(10.528,1.551)}
\gppoint{gp mark 1}{(10.536,1.517)}
\gppoint{gp mark 1}{(10.543,1.558)}
\gppoint{gp mark 1}{(10.551,1.523)}
\gppoint{gp mark 1}{(10.559,1.518)}
\gppoint{gp mark 1}{(10.567,1.543)}
\gppoint{gp mark 1}{(10.575,1.522)}
\gppoint{gp mark 1}{(10.582,1.538)}
\gppoint{gp mark 1}{(10.590,1.532)}
\gppoint{gp mark 1}{(10.598,1.528)}
\gppoint{gp mark 1}{(10.606,1.546)}
\gppoint{gp mark 1}{(10.613,1.530)}
\gppoint{gp mark 1}{(10.621,1.536)}
\gppoint{gp mark 1}{(10.629,1.520)}
\gppoint{gp mark 1}{(10.637,1.515)}
\gppoint{gp mark 1}{(10.645,1.513)}
\gppoint{gp mark 1}{(10.652,1.522)}
\gppoint{gp mark 1}{(10.660,1.544)}
\gppoint{gp mark 1}{(10.668,1.531)}
\gppoint{gp mark 1}{(10.676,1.533)}
\gppoint{gp mark 1}{(10.684,1.554)}
\gppoint{gp mark 1}{(10.691,1.552)}
\gppoint{gp mark 1}{(10.699,1.517)}
\gppoint{gp mark 1}{(10.707,1.527)}
\gppoint{gp mark 1}{(10.715,1.502)}
\gppoint{gp mark 1}{(10.723,1.516)}
\gppoint{gp mark 1}{(10.730,1.529)}
\gppoint{gp mark 1}{(10.738,1.518)}
\gppoint{gp mark 1}{(10.746,1.544)}
\gppoint{gp mark 1}{(10.754,1.545)}
\gppoint{gp mark 1}{(10.761,1.531)}
\gppoint{gp mark 1}{(10.769,1.528)}
\gppoint{gp mark 1}{(10.777,1.523)}
\gppoint{gp mark 1}{(10.785,1.527)}
\gppoint{gp mark 1}{(10.793,1.515)}
\gppoint{gp mark 1}{(10.800,1.505)}
\gppoint{gp mark 1}{(10.808,1.528)}
\gppoint{gp mark 1}{(10.816,1.514)}
\gppoint{gp mark 1}{(10.824,1.527)}
\gppoint{gp mark 1}{(10.832,1.516)}
\gppoint{gp mark 1}{(10.839,1.529)}
\gppoint{gp mark 1}{(10.847,1.545)}
\gppoint{gp mark 1}{(10.855,1.542)}
\gppoint{gp mark 1}{(10.863,1.528)}
\gppoint{gp mark 1}{(10.871,1.531)}
\gppoint{gp mark 1}{(10.878,1.523)}
\gppoint{gp mark 1}{(10.886,1.517)}
\gppoint{gp mark 1}{(10.894,1.539)}
\gppoint{gp mark 1}{(10.902,1.579)}
\gppoint{gp mark 1}{(10.909,1.657)}
\gppoint{gp mark 1}{(10.917,1.772)}
\gppoint{gp mark 1}{(10.925,1.968)}
\gppoint{gp mark 1}{(10.933,2.230)}
\gppoint{gp mark 1}{(10.941,2.826)}
\gppoint{gp mark 1}{(10.948,3.936)}
\gppoint{gp mark 1}{(10.956,4.733)}
\gppoint{gp mark 1}{(10.964,4.915)}
\gppoint{gp mark 1}{(10.972,4.837)}
\gppoint{gp mark 1}{(10.980,4.652)}
\gppoint{gp mark 1}{(10.987,4.662)}
\gppoint{gp mark 1}{(10.995,4.625)}
\gppoint{gp mark 1}{(11.003,4.619)}
\gppoint{gp mark 1}{(11.011,4.665)}
\gppoint{gp mark 1}{(11.019,4.653)}
\gppoint{gp mark 1}{(11.026,4.622)}
\gppoint{gp mark 1}{(11.034,4.664)}
\gppoint{gp mark 1}{(11.042,4.686)}
\gppoint{gp mark 1}{(11.050,4.917)}
\gppoint{gp mark 1}{(11.058,5.066)}
\gppoint{gp mark 1}{(11.065,5.565)}
\gppoint{gp mark 1}{(11.073,6.089)}
\gppoint{gp mark 1}{(11.081,6.431)}
\gppoint{gp mark 1}{(11.089,6.535)}
\gppoint{gp mark 1}{(11.096,6.482)}
\gppoint{gp mark 1}{(11.104,6.499)}
\gppoint{gp mark 1}{(11.112,6.497)}
\gppoint{gp mark 1}{(11.120,6.499)}
\gppoint{gp mark 1}{(11.128,6.479)}
\gppoint{gp mark 1}{(11.135,6.516)}
\gppoint{gp mark 1}{(11.143,6.481)}
\gppoint{gp mark 1}{(11.151,6.476)}
\gppoint{gp mark 1}{(11.159,6.486)}
\gppoint{gp mark 1}{(11.167,6.491)}
\gppoint{gp mark 1}{(11.174,6.538)}
\gppoint{gp mark 1}{(11.182,6.553)}
\gppoint{gp mark 1}{(11.190,6.514)}
\gppoint{gp mark 1}{(11.198,6.530)}
\gppoint{gp mark 1}{(11.206,6.522)}
\gppoint{gp mark 1}{(11.213,6.515)}
\gppoint{gp mark 1}{(11.221,6.499)}
\gppoint{gp mark 1}{(11.229,6.505)}
\gppoint{gp mark 1}{(11.237,6.507)}
\gppoint{gp mark 1}{(11.244,6.495)}
\gppoint{gp mark 1}{(11.252,6.494)}
\gppoint{gp mark 1}{(11.260,6.516)}
\gppoint{gp mark 1}{(11.268,6.486)}
\gppoint{gp mark 1}{(11.276,6.536)}
\gppoint{gp mark 1}{(11.283,6.530)}
\gppoint{gp mark 1}{(11.291,6.526)}
\gppoint{gp mark 1}{(11.299,6.534)}
\gppoint{gp mark 1}{(11.307,6.528)}
\gppoint{gp mark 1}{(11.315,6.532)}
\gppoint{gp mark 1}{(11.322,6.538)}
\gppoint{gp mark 1}{(11.330,6.518)}
\gppoint{gp mark 1}{(11.338,6.512)}
\gppoint{gp mark 1}{(11.346,6.503)}
\gppoint{gp mark 1}{(11.354,6.522)}
\gppoint{gp mark 1}{(11.361,6.522)}
\gppoint{gp mark 1}{(11.369,6.526)}
\gppoint{gp mark 1}{(11.377,6.508)}
\gppoint{gp mark 1}{(11.385,6.569)}
\gppoint{gp mark 1}{(11.392,6.532)}
\gppoint{gp mark 1}{(11.400,6.517)}
\draw[gp path] (1.320,7.691)--(1.320,0.985)--(11.447,0.985)--(11.447,7.691)--cycle;
%% coordinates of the plot area
\gpdefrectangularnode{gp plot 1}{\pgfpoint{1.320cm}{0.985cm}}{\pgfpoint{11.447cm}{7.691cm}}
\end{tikzpicture}
%% gnuplot variables

\caption{$T_{hot}$と$T_{cold}$の測定}
\label{fig:Thot_Tcold}
\end{center}
\end{figure}