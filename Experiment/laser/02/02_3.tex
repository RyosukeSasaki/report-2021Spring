\subsection{光電子増倍管}
光電子増倍管は応答性,感度,低雑音などに優れた光検出器である.
図\ref{fig:fig/fig4.png}に光電子増倍管の模式図を示す.
光電子増倍管はカソード,数段のダイノード,アノードから成る.
各ダイノードには分圧抵抗を用いて数十から数百$\si{\kilo\volt}$の電圧が掛けられている.
カソードに電子が衝突すると光電効果により電子が放出される.
これがダイノードに衝突するとダイノードからさらに複数の電子が放出される.
したがって数段のダイノードに衝突を繰り返すに従い$10^4$から$10^6$倍程度に増幅された電流がアノードから出力される.

光電子増倍管は熱電子放射や漏洩電流により光が入射していなくてもある程度の電流が流れており,これを暗電流と呼ぶ.
\mfig[width=10cm]{fig/fig4.png}{光電子増倍管の模式図}