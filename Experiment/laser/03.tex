\section{実験方法}
\subsection{測定1:半導体レーザーの発振閾値測定}
\subsubsection{準備}
\label{subsubsec:sokute1}
レーザーダイオード(LD)に定電流電源装置,電流計を接続し印加電流を測定できるようにした.
次に光電子増倍管にLDからのレーザー光を導入しADCの出力が$-0.7\si{\volt}$程度になるように高圧電源の電圧を調整した.
この際光電子増倍管に室内光のような強い光が入射しないように注意した.
\subsubsection{測定}
LDへの印加電流を$0$から$30\si{\milli\ampere}$程度まで$1.5\si{\milli\ampere}$ごとに変えながらADCの出力値を記録した.
記録した値からグラフを作成し,ゲインが変化する電流の閾値を読み取った.
\subsection{測定2:レーザー光の分光}
\subsubsection{準備}
\ref{subsubsec:sokute1}節と同様にLDをと定電流電源装置,電流計を接続した.
次にLDからのレーザー光を分光計に導入する.レンズを用いて分光計のスリット上にレーザー光が像結ぶように調整した.
その後分光計の波長ハンドルを回し,出力を中心波長に合わせた.
この状態で\ref{subsubsec:sokute1}と同様にADCの出力が$-0.7\si{\volt}$程度になるように高圧電源の電圧を調整した.
\subsubsection{測定}
表\ref{tab:measure2_condition}の各条件において分光計の掃引機能を用いて波長を掃引しながらADCの出力値を記録した.
その後各条件での測定結果をガウス関数$f(x)=a+c\exp(-(x-x_0)^2/2\sigma^2)$でfitし半値全幅と中心波長を求める.
fittingにはgnuplotのfitの機能を用いた.またB1からB4の測定では光電子増倍管による線形なドリフトが現れていたため
$f(x)=a+bx+cx\exp(-(x-x_0)^2/2\sigma^2)$という関数でfitした.
掃引の条件は表\ref{tab:sweep_condition}の通りである.
\begin{table}[h]
\caption{掃引条件}
\label{tab:sweep_condition}
\centering
\begin{tabular}{cc}
\hline
条件&内容\\
\hline \hline
掃引速度&$30\ \si{\nano\metre.\second^{-1}}$\\
掃引範囲&$665$〜$670\ \si{\nano\metre}$\\
\hline
\end{tabular}
\end{table}
\begin{table}[h]
\caption{測定条件}
\label{tab:measure2_condition}
\centering
\begin{tabular}{c|cc}
\hline
測定番号&スリット幅 / $\si{\micro\metre}$&印加電流 / $\si{\micro\metre}$\\
\hline \hline
A1&500&30\\
A2&400&30\\
A3&300&30\\
A4&200&30\\
A5&100&30\\
B1&500&25\\
B2&500&20\\
B3&500&10\\
B4&500&5\\
\hline
\end{tabular}
\end{table}
\subsection{測定3:FPIによるレーザ光の分光}
\subsubsection{準備}
この実験ではFPIによりHe-Neレーザーの多モード発振の発振周波数を調べた.
測定装置は図\ref{fig:fig/fig5.png}のような構成になっている.
まず検出器をアンプに,その出力をオシロスコープへ接続した.
また反射鏡とアンプ,アンプと鋸波発生器,鋸波発生器のモニター端子とオシロスコープを接続した.
次にフォトダイオードの受光面にレーザー光が当たるように検出器とレーザー光源の位置関係を調整した.
その後M2を設置し,光軸とM2の中心を合わせた.
そしてマイクロメーターを回して光軸とM2が垂直になるよう調整した.
その際に反射した光を光源と合わせるようにした.
同様にM1を設置し位置,角度を調整した.
最後にオシロスコープの出力を見ながら細いスペクトルが見えるように反射鏡の角度を更に追い込んだ.
\mfig[width=10cm]{fig/fig5.png}{測定3の構成}
\subsubsection{測定}
オシロスコープの測定モードを16回平均とし,スペクトルが見やすい部分を拡大して記録した.
また,その時の反射鏡間隔$D$をノギスで測定した.
反射鏡間隔を$5\ \si{\milli\metre}$ずつ変化させながら同様の実験を繰り返した.
\subsection{測定4:2台レーザー間のうなりの測定}
\subsubsection{準備}
この実験では2台のレーザーを混合させ,そのときに発生するうなりを測定する.
測定装置は図\ref{fig:fig/fig6.png}のような構成になっている.
まず光源1からのレーザー光がM2表面上で光源2の光点と一致するようにM1の角度を調整した.
次に光源1からのレーザ光とLD2からのレーザ光の光線が一致するようにM2の角度を調整した.
最後にレーザー光がフォトダイオードの受光面に当たるように調整した.
\mfig[width=10cm]{fig/fig6.png}{測定4}
\subsubsection{測定}
オシロスコープを調整しうなりの波形を記録した.
次に一方のLDからの光線に偏光板を通し,その状態でのうなりの波形を記録した.