\subsection{ガウスビームの拡散}
ガウスビームは回折によって伝搬に伴い広がりを持つようになり,やがて球面波となる.
任意の位置でのガウスビームのスポットサイズ$w(z)$と波面の曲率半径は以下で与えられる.
また図\ref{fig:w(z)}, 図\ref{fig:R(z)}に$w(z)$, $R(z)$
のプロットを示す.図\ref{fig:w(z)}は片対数である.
ただし$w_0=1\si{\milli\metre}$, $\lambda=600\si{\nano\metre}$とした.
図\ref{fig:w(z)}から$z\simeq10\si{\metre}$程度までは一定のビーム幅でその後急速に拡散しているのがわかる.
\begin{align}
  w(z)&=\sqrt{w_0^2+\left(\frac{\lambda z}{\pi w_0}\right)^2}\\
  R(z)&=z+\frac{\pi^2w_0^4}{\lambda^2z}
\end{align}
\clearpage
\begin{figure}[hptb]
\begin{center}
\begin{tikzpicture}[gnuplot]
%% generated with GNUPLOT 5.2p8 (Lua 5.3; terminal rev. Nov 2018, script rev. 108)
%% 2021年05月27日 21時56分20秒
\path (0.000,0.000) rectangle (10.000,10.000);
\gpcolor{color=gp lt color border}
\gpsetlinetype{gp lt border}
\gpsetdashtype{gp dt solid}
\gpsetlinewidth{1.00}
\draw[gp path] (1.504,1.366)--(1.684,1.366);
\draw[gp path] (9.447,1.366)--(9.267,1.366);
\node[gp node right] at (1.320,1.366) {$0$};
\draw[gp path] (1.504,2.160)--(1.594,2.160);
\draw[gp path] (9.447,2.160)--(9.357,2.160);
\draw[gp path] (1.504,2.955)--(1.594,2.955);
\draw[gp path] (9.447,2.955)--(9.357,2.955);
\draw[gp path] (1.504,3.749)--(1.594,3.749);
\draw[gp path] (9.447,3.749)--(9.357,3.749);
\draw[gp path] (1.504,4.544)--(1.594,4.544);
\draw[gp path] (9.447,4.544)--(9.357,4.544);
\draw[gp path] (1.504,5.338)--(1.684,5.338);
\draw[gp path] (9.447,5.338)--(9.267,5.338);
\node[gp node right] at (1.320,5.338) {$0.01$};
\draw[gp path] (1.504,6.132)--(1.594,6.132);
\draw[gp path] (9.447,6.132)--(9.357,6.132);
\draw[gp path] (1.504,6.927)--(1.594,6.927);
\draw[gp path] (9.447,6.927)--(9.357,6.927);
\draw[gp path] (1.504,7.721)--(1.594,7.721);
\draw[gp path] (9.447,7.721)--(9.357,7.721);
\draw[gp path] (1.504,8.516)--(1.594,8.516);
\draw[gp path] (9.447,8.516)--(9.357,8.516);
\draw[gp path] (1.504,9.310)--(1.684,9.310);
\draw[gp path] (9.447,9.310)--(9.267,9.310);
\node[gp node right] at (1.320,9.310) {$0.02$};
\draw[gp path] (1.504,1.366)--(1.504,1.546);
\draw[gp path] (1.504,9.310)--(1.504,9.130);
\node[gp node center] at (1.504,1.058) {$1$};
\draw[gp path] (2.700,1.366)--(2.700,1.456);
\draw[gp path] (2.700,9.310)--(2.700,9.220);
\draw[gp path] (3.399,1.366)--(3.399,1.456);
\draw[gp path] (3.399,9.310)--(3.399,9.220);
\draw[gp path] (3.895,1.366)--(3.895,1.456);
\draw[gp path] (3.895,9.310)--(3.895,9.220);
\draw[gp path] (4.280,1.366)--(4.280,1.456);
\draw[gp path] (4.280,9.310)--(4.280,9.220);
\draw[gp path] (4.594,1.366)--(4.594,1.456);
\draw[gp path] (4.594,9.310)--(4.594,9.220);
\draw[gp path] (4.860,1.366)--(4.860,1.456);
\draw[gp path] (4.860,9.310)--(4.860,9.220);
\draw[gp path] (5.091,1.366)--(5.091,1.456);
\draw[gp path] (5.091,9.310)--(5.091,9.220);
\draw[gp path] (5.294,1.366)--(5.294,1.456);
\draw[gp path] (5.294,9.310)--(5.294,9.220);
\draw[gp path] (5.476,1.366)--(5.476,1.546);
\draw[gp path] (5.476,9.310)--(5.476,9.130);
\node[gp node center] at (5.476,1.058) {$10$};
\draw[gp path] (6.671,1.366)--(6.671,1.456);
\draw[gp path] (6.671,9.310)--(6.671,9.220);
\draw[gp path] (7.370,1.366)--(7.370,1.456);
\draw[gp path] (7.370,9.310)--(7.370,9.220);
\draw[gp path] (7.867,1.366)--(7.867,1.456);
\draw[gp path] (7.867,9.310)--(7.867,9.220);
\draw[gp path] (8.251,1.366)--(8.251,1.456);
\draw[gp path] (8.251,9.310)--(8.251,9.220);
\draw[gp path] (8.566,1.366)--(8.566,1.456);
\draw[gp path] (8.566,9.310)--(8.566,9.220);
\draw[gp path] (8.832,1.366)--(8.832,1.456);
\draw[gp path] (8.832,9.310)--(8.832,9.220);
\draw[gp path] (9.062,1.366)--(9.062,1.456);
\draw[gp path] (9.062,9.310)--(9.062,9.220);
\draw[gp path] (9.265,1.366)--(9.265,1.456);
\draw[gp path] (9.265,9.310)--(9.265,9.220);
\draw[gp path] (9.447,1.366)--(9.447,1.546);
\draw[gp path] (9.447,9.310)--(9.447,9.130);
\node[gp node center] at (9.447,1.058) {$100$};
\draw[gp path] (1.504,9.310)--(1.504,1.366)--(9.447,1.366)--(9.447,9.310)--cycle;
\node[gp node center,rotate=-270] at (0.292,5.338) {$w(z)$};
\node[gp node center] at (5.475,0.596) {$z$ / $\si{\metre}$};
\draw[gp path] (1.504,1.770)--(1.584,1.771)--(1.664,1.772)--(1.745,1.773)--(1.825,1.774)%
  --(1.905,1.775)--(1.985,1.776)--(2.066,1.777)--(2.146,1.778)--(2.226,1.780)--(2.306,1.781)%
  --(2.387,1.783)--(2.467,1.785)--(2.547,1.787)--(2.627,1.789)--(2.707,1.791)--(2.788,1.794)%
  --(2.868,1.797)--(2.948,1.800)--(3.028,1.804)--(3.109,1.807)--(3.189,1.811)--(3.269,1.816)%
  --(3.349,1.821)--(3.430,1.826)--(3.510,1.831)--(3.590,1.838)--(3.670,1.844)--(3.751,1.851)%
  --(3.831,1.859)--(3.911,1.868)--(3.991,1.877)--(4.071,1.886)--(4.152,1.897)--(4.232,1.908)%
  --(4.312,1.920)--(4.392,1.933)--(4.473,1.947)--(4.553,1.962)--(4.633,1.978)--(4.713,1.995)%
  --(4.794,2.013)--(4.874,2.032)--(4.954,2.053)--(5.034,2.075)--(5.114,2.098)--(5.195,2.123)%
  --(5.275,2.149)--(5.355,2.177)--(5.435,2.207)--(5.516,2.238)--(5.596,2.271)--(5.676,2.306)%
  --(5.756,2.343)--(5.837,2.382)--(5.917,2.423)--(5.997,2.467)--(6.077,2.512)--(6.157,2.560)%
  --(6.238,2.611)--(6.318,2.665)--(6.398,2.721)--(6.478,2.780)--(6.559,2.842)--(6.639,2.907)%
  --(6.719,2.976)--(6.799,3.048)--(6.880,3.124)--(6.960,3.203)--(7.040,3.287)--(7.120,3.374)%
  --(7.200,3.466)--(7.281,3.563)--(7.361,3.664)--(7.441,3.770)--(7.521,3.882)--(7.602,3.998)%
  --(7.682,4.121)--(7.762,4.250)--(7.842,4.384)--(7.923,4.526)--(8.003,4.674)--(8.083,4.829)%
  --(8.163,4.992)--(8.244,5.162)--(8.324,5.341)--(8.404,5.529)--(8.484,5.725)--(8.564,5.931)%
  --(8.645,6.147)--(8.725,6.373)--(8.805,6.610)--(8.885,6.858)--(8.966,7.118)--(9.046,7.391)%
  --(9.126,7.677)--(9.206,7.976)--(9.287,8.289)--(9.367,8.618)--(9.447,8.962);
\draw[gp path] (1.504,9.310)--(1.504,1.366)--(9.447,1.366)--(9.447,9.310)--cycle;
%% coordinates of the plot area
\gpdefrectangularnode{gp plot 1}{\pgfpoint{1.504cm}{1.366cm}}{\pgfpoint{9.447cm}{9.310cm}}
\end{tikzpicture}
%% gnuplot variables

\caption{$z$-$w(z)$の関係}
\label{fig:w(z)}
\begin{tikzpicture}[gnuplot]
%% generated with GNUPLOT 5.2p8 (Lua 5.3; terminal rev. Nov 2018, script rev. 108)
%% 2021年05月27日 21時56分17秒
\path (0.000,0.000) rectangle (10.000,10.000);
\gpcolor{color=gp lt color border}
\gpsetlinetype{gp lt border}
\gpsetdashtype{gp dt solid}
\gpsetlinewidth{1.00}
\draw[gp path] (1.320,1.274)--(1.500,1.274);
\draw[gp path] (9.447,1.274)--(9.267,1.274);
\node[gp node right] at (1.136,1.274) {$0$};
\draw[gp path] (1.320,2.900)--(1.500,2.900);
\draw[gp path] (9.447,2.900)--(9.267,2.900);
\node[gp node right] at (1.136,2.900) {$20$};
\draw[gp path] (1.320,4.525)--(1.500,4.525);
\draw[gp path] (9.447,4.525)--(9.267,4.525);
\node[gp node right] at (1.136,4.525) {$40$};
\draw[gp path] (1.320,6.151)--(1.500,6.151);
\draw[gp path] (9.447,6.151)--(9.267,6.151);
\node[gp node right] at (1.136,6.151) {$60$};
\draw[gp path] (1.320,7.776)--(1.500,7.776);
\draw[gp path] (9.447,7.776)--(9.267,7.776);
\node[gp node right] at (1.136,7.776) {$80$};
\draw[gp path] (1.320,9.402)--(1.500,9.402);
\draw[gp path] (9.447,9.402)--(9.267,9.402);
\node[gp node right] at (1.136,9.402) {$100$};
\draw[gp path] (2.059,1.274)--(2.059,1.454);
\draw[gp path] (2.059,9.402)--(2.059,9.222);
\node[gp node center] at (2.059,0.966) {$10$};
\draw[gp path] (2.880,1.274)--(2.880,1.454);
\draw[gp path] (2.880,9.402)--(2.880,9.222);
\node[gp node center] at (2.880,0.966) {$20$};
\draw[gp path] (3.701,1.274)--(3.701,1.454);
\draw[gp path] (3.701,9.402)--(3.701,9.222);
\node[gp node center] at (3.701,0.966) {$30$};
\draw[gp path] (4.522,1.274)--(4.522,1.454);
\draw[gp path] (4.522,9.402)--(4.522,9.222);
\node[gp node center] at (4.522,0.966) {$40$};
\draw[gp path] (5.342,1.274)--(5.342,1.454);
\draw[gp path] (5.342,9.402)--(5.342,9.222);
\node[gp node center] at (5.342,0.966) {$50$};
\draw[gp path] (6.163,1.274)--(6.163,1.454);
\draw[gp path] (6.163,9.402)--(6.163,9.222);
\node[gp node center] at (6.163,0.966) {$60$};
\draw[gp path] (6.984,1.274)--(6.984,1.454);
\draw[gp path] (6.984,9.402)--(6.984,9.222);
\node[gp node center] at (6.984,0.966) {$70$};
\draw[gp path] (7.805,1.274)--(7.805,1.454);
\draw[gp path] (7.805,9.402)--(7.805,9.222);
\node[gp node center] at (7.805,0.966) {$80$};
\draw[gp path] (8.626,1.274)--(8.626,1.454);
\draw[gp path] (8.626,9.402)--(8.626,9.222);
\node[gp node center] at (8.626,0.966) {$90$};
\draw[gp path] (9.447,1.274)--(9.447,1.454);
\draw[gp path] (9.447,9.402)--(9.447,9.222);
\node[gp node center] at (9.447,0.966) {$100$};
\draw[gp path] (1.320,9.402)--(1.320,1.274)--(9.447,1.274)--(9.447,9.402)--cycle;
\node[gp node center,rotate=-270] at (0.292,5.338) {$R(z)$};
\node[gp node center] at (5.383,0.504) {$z$ / $\si{\metre}$};
\draw[gp path] (1.320,3.584)--(1.402,2.551)--(1.484,2.261)--(1.566,2.156)--(1.648,2.126)%
  --(1.730,2.133)--(1.813,2.161)--(1.895,2.203)--(1.977,2.253)--(2.059,2.310)--(2.141,2.371)%
  --(2.223,2.435)--(2.305,2.502)--(2.387,2.571)--(2.469,2.642)--(2.551,2.714)--(2.633,2.787)%
  --(2.716,2.861)--(2.798,2.936)--(2.880,3.011)--(2.962,3.087)--(3.044,3.163)--(3.126,3.240)%
  --(3.208,3.318)--(3.290,3.395)--(3.372,3.473)--(3.454,3.551)--(3.536,3.629)--(3.619,3.708)%
  --(3.701,3.787)--(3.783,3.866)--(3.865,3.945)--(3.947,4.024)--(4.029,4.103)--(4.111,4.182)%
  --(4.193,4.262)--(4.275,4.342)--(4.357,4.421)--(4.439,4.501)--(4.522,4.581)--(4.604,4.661)%
  --(4.686,4.741)--(4.768,4.821)--(4.850,4.901)--(4.932,4.981)--(5.014,5.061)--(5.096,5.142)%
  --(5.178,5.222)--(5.260,5.302)--(5.342,5.383)--(5.425,5.463)--(5.507,5.543)--(5.589,5.624)%
  --(5.671,5.704)--(5.753,5.785)--(5.835,5.865)--(5.917,5.946)--(5.999,6.027)--(6.081,6.107)%
  --(6.163,6.188)--(6.245,6.269)--(6.328,6.349)--(6.410,6.430)--(6.492,6.511)--(6.574,6.591)%
  --(6.656,6.672)--(6.738,6.753)--(6.820,6.834)--(6.902,6.915)--(6.984,6.995)--(7.066,7.076)%
  --(7.148,7.157)--(7.231,7.238)--(7.313,7.319)--(7.395,7.400)--(7.477,7.481)--(7.559,7.561)%
  --(7.641,7.642)--(7.723,7.723)--(7.805,7.804)--(7.887,7.885)--(7.969,7.966)--(8.051,8.047)%
  --(8.134,8.128)--(8.216,8.209)--(8.298,8.290)--(8.380,8.371)--(8.462,8.452)--(8.544,8.533)%
  --(8.626,8.614)--(8.708,8.695)--(8.790,8.776)--(8.872,8.857)--(8.954,8.938)--(9.037,9.019)%
  --(9.119,9.100)--(9.201,9.181)--(9.283,9.262)--(9.365,9.343)--(9.424,9.402);
\draw[gp path] (1.320,9.402)--(1.320,1.274)--(9.447,1.274)--(9.447,9.402)--cycle;
%% coordinates of the plot area
\gpdefrectangularnode{gp plot 1}{\pgfpoint{1.320cm}{1.274cm}}{\pgfpoint{9.447cm}{9.402cm}}
\end{tikzpicture}
%% gnuplot variables

\caption{$z$-$R(z)$の関係}
\label{fig:R(z)}
\end{center}
\end{figure}