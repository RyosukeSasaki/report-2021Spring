\subsection{ガウスビームの拡散}
ガウスビームは回折によって伝搬に伴い広がりを持つようになり,やがて球面波となる.
任意の位置でのガウスビームのスポットサイズ$w(z)$と波面の曲率半径は以下で与えられる.
また図\ref{fig:w(z)}, 図\ref{fig:R(z)}に$w(z)$, $R(z)$
のプロットを示す.図\ref{fig:w(z)}は片対数である.
ただし$w_0=1\si{\milli\metre}$, $\lambda=600\si{\nano\metre}$とした.
図\ref{fig:w(z)}から$z\simeq10\si{\metre}$程度までは一定のビーム幅でその後急速に拡散しているのがわかる.
\begin{align}
  w(z)&=\sqrt{w_0^2+\left(\frac{\lambda z}{\pi w_0}\right)^2}\\
  R(z)&=z+\frac{\pi^2w_0^4}{\lambda^2z}
\end{align}
\clearpage
\begin{figure}[hptb]
\begin{center}
\begin{tikzpicture}[gnuplot]
%% generated with GNUPLOT 5.2p8 (Lua 5.3; terminal rev. Nov 2018, script rev. 108)
%% 2021年05月27日 21時56分20秒
\path (0.000,0.000) rectangle (10.000,10.000);
\gpcolor{color=gp lt color border}
\gpsetlinetype{gp lt border}
\gpsetdashtype{gp dt solid}
\gpsetlinewidth{1.00}
\draw[gp path] (1.504,1.366)--(1.684,1.366);
\draw[gp path] (9.447,1.366)--(9.267,1.366);
\node[gp node right] at (1.320,1.366) {$0$};
\draw[gp path] (1.504,2.160)--(1.594,2.160);
\draw[gp path] (9.447,2.160)--(9.357,2.160);
\draw[gp path] (1.504,2.955)--(1.594,2.955);
\draw[gp path] (9.447,2.955)--(9.357,2.955);
\draw[gp path] (1.504,3.749)--(1.594,3.749);
\draw[gp path] (9.447,3.749)--(9.357,3.749);
\draw[gp path] (1.504,4.544)--(1.594,4.544);
\draw[gp path] (9.447,4.544)--(9.357,4.544);
\draw[gp path] (1.504,5.338)--(1.684,5.338);
\draw[gp path] (9.447,5.338)--(9.267,5.338);
\node[gp node right] at (1.320,5.338) {$0.01$};
\draw[gp path] (1.504,6.132)--(1.594,6.132);
\draw[gp path] (9.447,6.132)--(9.357,6.132);
\draw[gp path] (1.504,6.927)--(1.594,6.927);
\draw[gp path] (9.447,6.927)--(9.357,6.927);
\draw[gp path] (1.504,7.721)--(1.594,7.721);
\draw[gp path] (9.447,7.721)--(9.357,7.721);
\draw[gp path] (1.504,8.516)--(1.594,8.516);
\draw[gp path] (9.447,8.516)--(9.357,8.516);
\draw[gp path] (1.504,9.310)--(1.684,9.310);
\draw[gp path] (9.447,9.310)--(9.267,9.310);
\node[gp node right] at (1.320,9.310) {$0.02$};
\draw[gp path] (1.504,1.366)--(1.504,1.546);
\draw[gp path] (1.504,9.310)--(1.504,9.130);
\node[gp node center] at (1.504,1.058) {$1$};
\draw[gp path] (2.700,1.366)--(2.700,1.456);
\draw[gp path] (2.700,9.310)--(2.700,9.220);
\draw[gp path] (3.399,1.366)--(3.399,1.456);
\draw[gp path] (3.399,9.310)--(3.399,9.220);
\draw[gp path] (3.895,1.366)--(3.895,1.456);
\draw[gp path] (3.895,9.310)--(3.895,9.220);
\draw[gp path] (4.280,1.366)--(4.280,1.456);
\draw[gp path] (4.280,9.310)--(4.280,9.220);
\draw[gp path] (4.594,1.366)--(4.594,1.456);
\draw[gp path] (4.594,9.310)--(4.594,9.220);
\draw[gp path] (4.860,1.366)--(4.860,1.456);
\draw[gp path] (4.860,9.310)--(4.860,9.220);
\draw[gp path] (5.091,1.366)--(5.091,1.456);
\draw[gp path] (5.091,9.310)--(5.091,9.220);
\draw[gp path] (5.294,1.366)--(5.294,1.456);
\draw[gp path] (5.294,9.310)--(5.294,9.220);
\draw[gp path] (5.476,1.366)--(5.476,1.546);
\draw[gp path] (5.476,9.310)--(5.476,9.130);
\node[gp node center] at (5.476,1.058) {$10$};
\draw[gp path] (6.671,1.366)--(6.671,1.456);
\draw[gp path] (6.671,9.310)--(6.671,9.220);
\draw[gp path] (7.370,1.366)--(7.370,1.456);
\draw[gp path] (7.370,9.310)--(7.370,9.220);
\draw[gp path] (7.867,1.366)--(7.867,1.456);
\draw[gp path] (7.867,9.310)--(7.867,9.220);
\draw[gp path] (8.251,1.366)--(8.251,1.456);
\draw[gp path] (8.251,9.310)--(8.251,9.220);
\draw[gp path] (8.566,1.366)--(8.566,1.456);
\draw[gp path] (8.566,9.310)--(8.566,9.220);
\draw[gp path] (8.832,1.366)--(8.832,1.456);
\draw[gp path] (8.832,9.310)--(8.832,9.220);
\draw[gp path] (9.062,1.366)--(9.062,1.456);
\draw[gp path] (9.062,9.310)--(9.062,9.220);
\draw[gp path] (9.265,1.366)--(9.265,1.456);
\draw[gp path] (9.265,9.310)--(9.265,9.220);
\draw[gp path] (9.447,1.366)--(9.447,1.546);
\draw[gp path] (9.447,9.310)--(9.447,9.130);
\node[gp node center] at (9.447,1.058) {$100$};
\draw[gp path] (1.504,9.310)--(1.504,1.366)--(9.447,1.366)--(9.447,9.310)--cycle;
\node[gp node center,rotate=-270] at (0.292,5.338) {$w(z)$};
\node[gp node center] at (5.475,0.596) {$z$ / $\si{\metre}$};
\draw[gp path] (1.504,1.770)--(1.584,1.771)--(1.664,1.772)--(1.745,1.773)--(1.825,1.774)%
  --(1.905,1.775)--(1.985,1.776)--(2.066,1.777)--(2.146,1.778)--(2.226,1.780)--(2.306,1.781)%
  --(2.387,1.783)--(2.467,1.785)--(2.547,1.787)--(2.627,1.789)--(2.707,1.791)--(2.788,1.794)%
  --(2.868,1.797)--(2.948,1.800)--(3.028,1.804)--(3.109,1.807)--(3.189,1.811)--(3.269,1.816)%
  --(3.349,1.821)--(3.430,1.826)--(3.510,1.831)--(3.590,1.838)--(3.670,1.844)--(3.751,1.851)%
  --(3.831,1.859)--(3.911,1.868)--(3.991,1.877)--(4.071,1.886)--(4.152,1.897)--(4.232,1.908)%
  --(4.312,1.920)--(4.392,1.933)--(4.473,1.947)--(4.553,1.962)--(4.633,1.978)--(4.713,1.995)%
  --(4.794,2.013)--(4.874,2.032)--(4.954,2.053)--(5.034,2.075)--(5.114,2.098)--(5.195,2.123)%
  --(5.275,2.149)--(5.355,2.177)--(5.435,2.207)--(5.516,2.238)--(5.596,2.271)--(5.676,2.306)%
  --(5.756,2.343)--(5.837,2.382)--(5.917,2.423)--(5.997,2.467)--(6.077,2.512)--(6.157,2.560)%
  --(6.238,2.611)--(6.318,2.665)--(6.398,2.721)--(6.478,2.780)--(6.559,2.842)--(6.639,2.907)%
  --(6.719,2.976)--(6.799,3.048)--(6.880,3.124)--(6.960,3.203)--(7.040,3.287)--(7.120,3.374)%
  --(7.200,3.466)--(7.281,3.563)--(7.361,3.664)--(7.441,3.770)--(7.521,3.882)--(7.602,3.998)%
  --(7.682,4.121)--(7.762,4.250)--(7.842,4.384)--(7.923,4.526)--(8.003,4.674)--(8.083,4.829)%
  --(8.163,4.992)--(8.244,5.162)--(8.324,5.341)--(8.404,5.529)--(8.484,5.725)--(8.564,5.931)%
  --(8.645,6.147)--(8.725,6.373)--(8.805,6.610)--(8.885,6.858)--(8.966,7.118)--(9.046,7.391)%
  --(9.126,7.677)--(9.206,7.976)--(9.287,8.289)--(9.367,8.618)--(9.447,8.962);
\draw[gp path] (1.504,9.310)--(1.504,1.366)--(9.447,1.366)--(9.447,9.310)--cycle;
%% coordinates of the plot area
\gpdefrectangularnode{gp plot 1}{\pgfpoint{1.504cm}{1.366cm}}{\pgfpoint{9.447cm}{9.310cm}}
\end{tikzpicture}
%% gnuplot variables

\caption{$z$-$w(z)$の関係}
\label{fig:w(z)}
\begin{tikzpicture}[gnuplot]
%% generated with GNUPLOT 5.2p8 (Lua 5.3; terminal rev. Nov 2018, script rev. 108)
%% 2021年05月27日 23時03分28秒
\path (0.000,0.000) rectangle (9.000,9.000);
\gpcolor{color=gp lt color border}
\gpsetlinetype{gp lt border}
\gpsetdashtype{gp dt solid}
\gpsetlinewidth{1.00}
\draw[gp path] (1.320,1.274)--(1.500,1.274);
\draw[gp path] (8.447,1.274)--(8.267,1.274);
\node[gp node right] at (1.136,1.274) {$0$};
\draw[gp path] (1.320,2.700)--(1.500,2.700);
\draw[gp path] (8.447,2.700)--(8.267,2.700);
\node[gp node right] at (1.136,2.700) {$20$};
\draw[gp path] (1.320,4.125)--(1.500,4.125);
\draw[gp path] (8.447,4.125)--(8.267,4.125);
\node[gp node right] at (1.136,4.125) {$40$};
\draw[gp path] (1.320,5.551)--(1.500,5.551);
\draw[gp path] (8.447,5.551)--(8.267,5.551);
\node[gp node right] at (1.136,5.551) {$60$};
\draw[gp path] (1.320,6.976)--(1.500,6.976);
\draw[gp path] (8.447,6.976)--(8.267,6.976);
\node[gp node right] at (1.136,6.976) {$80$};
\draw[gp path] (1.320,8.402)--(1.500,8.402);
\draw[gp path] (8.447,8.402)--(8.267,8.402);
\node[gp node right] at (1.136,8.402) {$100$};
\draw[gp path] (1.968,1.274)--(1.968,1.454);
\draw[gp path] (1.968,8.402)--(1.968,8.222);
\node[gp node center] at (1.968,0.966) {$10$};
\draw[gp path] (2.688,1.274)--(2.688,1.454);
\draw[gp path] (2.688,8.402)--(2.688,8.222);
\node[gp node center] at (2.688,0.966) {$20$};
\draw[gp path] (3.408,1.274)--(3.408,1.454);
\draw[gp path] (3.408,8.402)--(3.408,8.222);
\node[gp node center] at (3.408,0.966) {$30$};
\draw[gp path] (4.128,1.274)--(4.128,1.454);
\draw[gp path] (4.128,8.402)--(4.128,8.222);
\node[gp node center] at (4.128,0.966) {$40$};
\draw[gp path] (4.848,1.274)--(4.848,1.454);
\draw[gp path] (4.848,8.402)--(4.848,8.222);
\node[gp node center] at (4.848,0.966) {$50$};
\draw[gp path] (5.567,1.274)--(5.567,1.454);
\draw[gp path] (5.567,8.402)--(5.567,8.222);
\node[gp node center] at (5.567,0.966) {$60$};
\draw[gp path] (6.287,1.274)--(6.287,1.454);
\draw[gp path] (6.287,8.402)--(6.287,8.222);
\node[gp node center] at (6.287,0.966) {$70$};
\draw[gp path] (7.007,1.274)--(7.007,1.454);
\draw[gp path] (7.007,8.402)--(7.007,8.222);
\node[gp node center] at (7.007,0.966) {$80$};
\draw[gp path] (7.727,1.274)--(7.727,1.454);
\draw[gp path] (7.727,8.402)--(7.727,8.222);
\node[gp node center] at (7.727,0.966) {$90$};
\draw[gp path] (8.447,1.274)--(8.447,1.454);
\draw[gp path] (8.447,8.402)--(8.447,8.222);
\node[gp node center] at (8.447,0.966) {$100$};
\draw[gp path] (1.320,8.402)--(1.320,1.274)--(8.447,1.274)--(8.447,8.402)--cycle;
\node[gp node center,rotate=-270] at (0.292,4.838) {$R(z)$};
\node[gp node center] at (4.883,0.504) {$z$ / $\si{\metre}$};
\draw[gp path] (1.320,3.299)--(1.392,2.394)--(1.464,2.139)--(1.536,2.048)--(1.608,2.021)%
  --(1.680,2.027)--(1.752,2.052)--(1.824,2.089)--(1.896,2.133)--(1.968,2.182)--(2.040,2.236)%
  --(2.112,2.292)--(2.184,2.351)--(2.256,2.412)--(2.328,2.473)--(2.400,2.537)--(2.472,2.601)%
  --(2.544,2.666)--(2.616,2.731)--(2.688,2.797)--(2.760,2.864)--(2.832,2.931)--(2.904,2.998)%
  --(2.976,3.066)--(3.048,3.134)--(3.120,3.202)--(3.192,3.271)--(3.264,3.340)--(3.336,3.409)%
  --(3.408,3.478)--(3.480,3.547)--(3.552,3.616)--(3.624,3.685)--(3.696,3.755)--(3.768,3.825)%
  --(3.840,3.894)--(3.912,3.964)--(3.984,4.034)--(4.056,4.104)--(4.128,4.174)--(4.200,4.244)%
  --(4.272,4.314)--(4.344,4.384)--(4.416,4.455)--(4.488,4.525)--(4.560,4.595)--(4.632,4.666)%
  --(4.704,4.736)--(4.776,4.807)--(4.848,4.877)--(4.919,4.948)--(4.991,5.018)--(5.063,5.089)%
  --(5.135,5.159)--(5.207,5.230)--(5.279,5.301)--(5.351,5.371)--(5.423,5.442)--(5.495,5.513)%
  --(5.567,5.583)--(5.639,5.654)--(5.711,5.725)--(5.783,5.796)--(5.855,5.866)--(5.927,5.937)%
  --(5.999,6.008)--(6.071,6.079)--(6.143,6.150)--(6.215,6.221)--(6.287,6.292)--(6.359,6.362)%
  --(6.431,6.433)--(6.503,6.504)--(6.575,6.575)--(6.647,6.646)--(6.719,6.717)--(6.791,6.788)%
  --(6.863,6.859)--(6.935,6.930)--(7.007,7.001)--(7.079,7.072)--(7.151,7.143)--(7.223,7.214)%
  --(7.295,7.285)--(7.367,7.356)--(7.439,7.427)--(7.511,7.498)--(7.583,7.569)--(7.655,7.640)%
  --(7.727,7.711)--(7.799,7.782)--(7.871,7.853)--(7.943,7.924)--(8.015,7.995)--(8.087,8.066)%
  --(8.159,8.137)--(8.231,8.208)--(8.303,8.279)--(8.375,8.350)--(8.427,8.402);
\draw[gp path] (1.320,8.402)--(1.320,1.274)--(8.447,1.274)--(8.447,8.402)--cycle;
%% coordinates of the plot area
\gpdefrectangularnode{gp plot 1}{\pgfpoint{1.320cm}{1.274cm}}{\pgfpoint{8.447cm}{8.402cm}}
\end{tikzpicture}
%% gnuplot variables

\caption{$z$-$R(z)$の関係}
\label{fig:R(z)}
\end{center}
\end{figure}