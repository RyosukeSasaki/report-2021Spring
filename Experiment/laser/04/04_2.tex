\subsection{測定2:レーザー光の分光}
表\ref{tab:sigma_center}に各測定での標準偏差$\sigma$,中心波長$\lambda_0$を示す.
スリット幅,印加電流と半値全幅の関係を図\ref{fig:graph/day2/width_FWHM.tex},図\ref{fig:graph/day2/current_FWHM.tex}示す.
また各測定で得られたスペクトル, fitting結果は補遺に残した.半値全幅は$2\sigma\sqrt{2\log2}$より求めた.
\begin{table}[h]
\caption{各測定での標準偏差と中心波長}
\label{tab:sigma_center}
\centering
\begin{tabular}{c|cc}
\hline
測定番号&標準偏差$\sigma$&中心波長$\lambda_0$\\
\hline \hline
A1&$0.7047\pm0.0006$& $658.8\pm0.0$\\
A2&$0.5675\pm0.0003$& $658.9\pm0.0$\\
A3&$0.4365\pm0.0002$& $658.8\pm0.0$\\
A4&$0.2783\pm0.0003$& $658.8\pm0.0$\\
A5&$0.1785\pm0.0018$& $658.8\pm0.0$\\
B1&$1.217\pm0.001$& $658.8\pm0.0$\\
B2&$5.704\pm0.006$& $657.8\pm0.0$\\
B3&$7.575\pm0.006$& $657.2\pm0.0$\\
B4&$8.613\pm0.011$& $656.7\pm0.0$\\
\hline
\end{tabular}
\end{table}
\gnu{スリット幅と半値全幅の関係}{graph/day2/width_FWHM.tex}
\gnu{印加電流と半値全幅の関係}{graph/day2/current_FWHM.tex}
%\gnu{test}{graph/day2/Aa.tex}
%\gnu{test}{graph/day2/Ab.tex}
%\gnu{test}{graph/day2/Ac.tex}
%\gnu{test}{graph/day2/Ad.tex}
%\gnu{test}{graph/day2/Ae.tex}
%\gnu{test}{graph/day2/Ba_slope.tex}
%\gnu{test}{graph/day2/Bb_slope.tex}
%\gnu{test}{graph/day2/Bc_slope.tex}
%\gnu{test}{graph/day2/Bd_slope.tex}