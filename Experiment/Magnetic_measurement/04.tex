\section{結果}
\subsection{Ni円盤の磁気ヒステリシス測定}
\begin{figure}[hptb]
  \begin{center}
    \begin{tikzpicture}[gnuplot]
%% generated with GNUPLOT 5.2p8 (Lua 5.3; terminal rev. Nov 2018, script rev. 108)
%% 2021年04月27日 14時36分09秒
\path (0.000,0.000) rectangle (12.500,8.750);
\gpcolor{color=gp lt color border}
\gpsetlinetype{gp lt border}
\gpsetdashtype{gp dt solid}
\gpsetlinewidth{1.00}
\draw[gp path] (2.997,0.985)--(3.177,0.985);
\draw[gp path] (10.454,0.985)--(10.274,0.985);
\node[gp node right] at (2.813,0.985) {$-0.6$};
\draw[gp path] (2.997,1.606)--(3.087,1.606);
\draw[gp path] (10.454,1.606)--(10.364,1.606);
\draw[gp path] (2.997,2.228)--(3.177,2.228);
\draw[gp path] (10.454,2.228)--(10.274,2.228);
\node[gp node right] at (2.813,2.228) {$-0.4$};
\draw[gp path] (2.997,2.849)--(3.087,2.849);
\draw[gp path] (10.454,2.849)--(10.364,2.849);
\draw[gp path] (2.997,3.470)--(3.177,3.470);
\draw[gp path] (10.454,3.470)--(10.274,3.470);
\node[gp node right] at (2.813,3.470) {$-0.2$};
\draw[gp path] (2.997,4.092)--(3.087,4.092);
\draw[gp path] (10.454,4.092)--(10.364,4.092);
\draw[gp path] (2.997,4.713)--(3.177,4.713);
\draw[gp path] (10.454,4.713)--(10.274,4.713);
\node[gp node right] at (2.813,4.713) {$0$};
\draw[gp path] (2.997,5.334)--(3.087,5.334);
\draw[gp path] (10.454,5.334)--(10.364,5.334);
\draw[gp path] (2.997,5.956)--(3.177,5.956);
\draw[gp path] (10.454,5.956)--(10.274,5.956);
\node[gp node right] at (2.813,5.956) {$0.2$};
\draw[gp path] (2.997,6.577)--(3.087,6.577);
\draw[gp path] (10.454,6.577)--(10.364,6.577);
\draw[gp path] (2.997,7.198)--(3.177,7.198);
\draw[gp path] (10.454,7.198)--(10.274,7.198);
\node[gp node right] at (2.813,7.198) {$0.4$};
\draw[gp path] (2.997,7.820)--(3.087,7.820);
\draw[gp path] (10.454,7.820)--(10.364,7.820);
\draw[gp path] (2.997,8.441)--(3.177,8.441);
\draw[gp path] (10.454,8.441)--(10.274,8.441);
\node[gp node right] at (2.813,8.441) {$0.6$};
\draw[gp path] (4.240,0.985)--(4.240,1.165);
\draw[gp path] (4.240,8.441)--(4.240,8.261);
\node[gp node center] at (4.240,0.677) {$-2.0$};
\draw[gp path] (5.483,0.985)--(5.483,1.075);
\draw[gp path] (5.483,8.441)--(5.483,8.351);
\draw[gp path] (6.726,0.985)--(6.726,1.165);
\draw[gp path] (6.726,8.441)--(6.726,8.261);
\node[gp node center] at (6.726,0.677) {$0.0$};
\draw[gp path] (7.968,0.985)--(7.968,1.075);
\draw[gp path] (7.968,8.441)--(7.968,8.351);
\draw[gp path] (9.211,0.985)--(9.211,1.165);
\draw[gp path] (9.211,8.441)--(9.211,8.261);
\node[gp node center] at (9.211,0.677) {$2.0$};
\gpsetlinetype{gp lt axes}
\gpsetdashtype{gp dt axes}
\gpsetlinewidth{1.50}
\draw[gp path] (2.997,4.713)--(10.454,4.713);
\draw[gp path] (6.726,0.985)--(6.726,8.441);
\gpsetlinetype{gp lt border}
\gpsetdashtype{gp dt solid}
\gpsetlinewidth{1.00}
\draw[gp path] (2.997,8.441)--(2.997,0.985)--(10.454,0.985)--(10.454,8.441)--cycle;
\node[gp node center,rotate=-270] at (1.785,4.713) {磁化$J$ / $\si{Wb.m^{-2}}$};
\node[gp node center] at (6.725,0.215) {磁場$H$ / $\times 10^{5}\ \si{A.m^{-1}}$};
\draw[gp path] (3.181,6.875)--(3.181,8.261)--(6.489,8.261)--(6.489,6.875)--cycle;
\node[gp node right] at (5.205,7.799) {面直磁場};
\gpcolor{rgb color={0.000,0.000,0.000}}
\gpsetpointsize{4.00}
\gppoint{gp mark 6}{(6.680,4.661)}
\gppoint{gp mark 6}{(7.046,4.966)}
\gppoint{gp mark 6}{(7.397,5.260)}
\gppoint{gp mark 6}{(7.747,5.550)}
\gppoint{gp mark 6}{(8.098,5.836)}
\gppoint{gp mark 6}{(8.448,6.125)}
\gppoint{gp mark 6}{(8.794,6.393)}
\gppoint{gp mark 6}{(9.134,6.660)}
\gppoint{gp mark 6}{(9.484,6.918)}
\gppoint{gp mark 6}{(9.820,7.155)}
\gppoint{gp mark 6}{(10.161,7.382)}
\gppoint{gp mark 6}{(9.864,7.196)}
\gppoint{gp mark 6}{(9.538,6.980)}
\gppoint{gp mark 6}{(9.217,6.743)}
\gppoint{gp mark 6}{(8.876,6.485)}
\gppoint{gp mark 6}{(8.526,6.217)}
\gppoint{gp mark 6}{(8.185,5.939)}
\gppoint{gp mark 6}{(7.830,5.655)}
\gppoint{gp mark 6}{(7.480,5.368)}
\gppoint{gp mark 6}{(7.128,5.077)}
\gppoint{gp mark 6}{(6.748,4.758)}
\gppoint{gp mark 6}{(6.389,4.457)}
\gppoint{gp mark 6}{(6.049,4.172)}
\gppoint{gp mark 6}{(5.694,3.880)}
\gppoint{gp mark 6}{(5.334,3.580)}
\gppoint{gp mark 6}{(4.998,3.312)}
\gppoint{gp mark 6}{(4.643,3.023)}
\gppoint{gp mark 6}{(4.302,2.766)}
\gppoint{gp mark 6}{(3.957,2.508)}
\gppoint{gp mark 6}{(3.621,2.271)}
\gppoint{gp mark 6}{(3.285,2.044)}
\gppoint{gp mark 6}{(3.572,2.230)}
\gppoint{gp mark 6}{(3.898,2.446)}
\gppoint{gp mark 6}{(4.234,2.693)}
\gppoint{gp mark 6}{(4.565,2.941)}
\gppoint{gp mark 6}{(4.910,3.209)}
\gppoint{gp mark 6}{(5.261,3.487)}
\gppoint{gp mark 6}{(5.616,3.774)}
\gppoint{gp mark 6}{(5.957,4.056)}
\gppoint{gp mark 6}{(6.311,4.349)}
\gppoint{gp mark 6}{(6.692,4.668)}
\gppoint{gp mark 6}{(7.043,4.962)}
\gppoint{gp mark 6}{(7.392,5.256)}
\gppoint{gp mark 6}{(7.747,5.551)}
\gppoint{gp mark 6}{(8.098,5.836)}
\gppoint{gp mark 6}{(8.448,6.125)}
\gppoint{gp mark 6}{(8.798,6.403)}
\gppoint{gp mark 6}{(9.144,6.671)}
\gppoint{gp mark 6}{(9.480,6.918)}
\gppoint{gp mark 6}{(9.820,7.155)}
\gppoint{gp mark 6}{(10.156,7.371)}
\gppoint{gp mark 6}{(5.847,7.799)}
\gpcolor{color=gp lt color border}
\node[gp node right] at (5.205,7.337) {面内磁場};
\gpcolor{rgb color={0.000,0.000,0.000}}
\gppoint{gp mark 7}{(6.726,4.713)}
\gppoint{gp mark 7}{(7.068,6.300)}
\gppoint{gp mark 7}{(7.407,7.361)}
\gppoint{gp mark 7}{(7.757,7.876)}
\gppoint{gp mark 7}{(8.103,8.062)}
\gppoint{gp mark 7}{(8.453,8.134)}
\gppoint{gp mark 7}{(8.803,8.165)}
\gppoint{gp mark 7}{(9.144,8.185)}
\gppoint{gp mark 7}{(9.489,8.206)}
\gppoint{gp mark 7}{(9.820,8.206)}
\gppoint{gp mark 7}{(10.151,8.206)}
\gppoint{gp mark 7}{(9.859,8.206)}
\gppoint{gp mark 7}{(9.548,8.196)}
\gppoint{gp mark 7}{(9.212,8.185)}
\gppoint{gp mark 7}{(8.871,8.165)}
\gppoint{gp mark 7}{(8.526,8.134)}
\gppoint{gp mark 7}{(8.176,8.072)}
\gppoint{gp mark 7}{(7.830,7.928)}
\gppoint{gp mark 7}{(7.480,7.536)}
\gppoint{gp mark 7}{(7.128,6.630)}
\gppoint{gp mark 7}{(6.751,4.970)}
\gppoint{gp mark 7}{(6.399,3.312)}
\gppoint{gp mark 7}{(6.049,2.116)}
\gppoint{gp mark 7}{(5.694,1.560)}
\gppoint{gp mark 7}{(5.348,1.374)}
\gppoint{gp mark 7}{(4.998,1.302)}
\gppoint{gp mark 7}{(4.648,1.271)}
\gppoint{gp mark 7}{(4.307,1.251)}
\gppoint{gp mark 7}{(3.962,1.230)}
\gppoint{gp mark 7}{(3.621,1.230)}
\gppoint{gp mark 7}{(3.285,1.230)}
\gppoint{gp mark 7}{(3.577,1.241)}
\gppoint{gp mark 7}{(3.894,1.241)}
\gppoint{gp mark 7}{(4.229,1.251)}
\gppoint{gp mark 7}{(4.565,1.271)}
\gppoint{gp mark 7}{(4.910,1.302)}
\gppoint{gp mark 7}{(5.256,1.364)}
\gppoint{gp mark 7}{(5.606,1.508)}
\gppoint{gp mark 7}{(5.962,1.900)}
\gppoint{gp mark 7}{(6.318,2.827)}
\gppoint{gp mark 7}{(6.689,4.467)}
\gppoint{gp mark 7}{(7.042,6.125)}
\gppoint{gp mark 7}{(7.392,7.310)}
\gppoint{gp mark 7}{(7.742,7.856)}
\gppoint{gp mark 7}{(8.093,8.041)}
\gppoint{gp mark 7}{(8.448,8.113)}
\gppoint{gp mark 7}{(8.798,8.144)}
\gppoint{gp mark 7}{(9.134,8.165)}
\gppoint{gp mark 7}{(9.489,8.175)}
\gppoint{gp mark 7}{(9.820,8.185)}
\gppoint{gp mark 7}{(10.156,8.196)}
\gppoint{gp mark 7}{(5.847,7.337)}
\gpcolor{color=gp lt color border}
\draw[gp path] (2.997,8.441)--(2.997,0.985)--(10.454,0.985)--(10.454,8.441)--cycle;
%% coordinates of the plot area
\gpdefrectangularnode{gp plot 1}{\pgfpoint{2.997cm}{0.985cm}}{\pgfpoint{10.454cm}{8.441cm}}
\end{tikzpicture}
%% gnuplot variables

    \caption{Niの磁気ヒステリシス曲線}
  \end{center}
\end{figure}

\begin{figure}[hptb]
  \begin{center}
    \begin{tikzpicture}[gnuplot]
%% generated with GNUPLOT 5.2p8 (Lua 5.3; terminal rev. Nov 2018, script rev. 108)
%% 2021年04月27日 14時36分57秒
\path (0.000,0.000) rectangle (12.500,8.750);
\gpcolor{color=gp lt color border}
\gpsetlinetype{gp lt border}
\gpsetdashtype{gp dt solid}
\gpsetlinewidth{1.00}
\draw[gp path] (2.905,1.163)--(3.085,1.163);
\draw[gp path] (10.362,1.163)--(10.182,1.163);
\node[gp node right] at (2.721,1.163) {$-20$};
\draw[gp path] (2.905,2.050)--(3.085,2.050);
\draw[gp path] (10.362,2.050)--(10.182,2.050);
\node[gp node right] at (2.721,2.050) {$-15$};
\draw[gp path] (2.905,2.938)--(3.085,2.938);
\draw[gp path] (10.362,2.938)--(10.182,2.938);
\node[gp node right] at (2.721,2.938) {$-10$};
\draw[gp path] (2.905,3.825)--(3.085,3.825);
\draw[gp path] (10.362,3.825)--(10.182,3.825);
\node[gp node right] at (2.721,3.825) {$-5$};
\draw[gp path] (2.905,4.713)--(3.085,4.713);
\draw[gp path] (10.362,4.713)--(10.182,4.713);
\node[gp node right] at (2.721,4.713) {$0$};
\draw[gp path] (2.905,5.601)--(3.085,5.601);
\draw[gp path] (10.362,5.601)--(10.182,5.601);
\node[gp node right] at (2.721,5.601) {$5$};
\draw[gp path] (2.905,6.488)--(3.085,6.488);
\draw[gp path] (10.362,6.488)--(10.182,6.488);
\node[gp node right] at (2.721,6.488) {$10$};
\draw[gp path] (2.905,7.376)--(3.085,7.376);
\draw[gp path] (10.362,7.376)--(10.182,7.376);
\node[gp node right] at (2.721,7.376) {$15$};
\draw[gp path] (2.905,8.263)--(3.085,8.263);
\draw[gp path] (10.362,8.263)--(10.182,8.263);
\node[gp node right] at (2.721,8.263) {$20$};
\draw[gp path] (2.905,0.985)--(2.905,1.165);
\draw[gp path] (2.905,8.441)--(2.905,8.261);
\node[gp node center] at (2.905,0.677) {$0$};
\draw[gp path] (3.734,0.985)--(3.734,1.165);
\draw[gp path] (3.734,8.441)--(3.734,8.261);
\node[gp node center] at (3.734,0.677) {$20$};
\draw[gp path] (4.562,0.985)--(4.562,1.165);
\draw[gp path] (4.562,8.441)--(4.562,8.261);
\node[gp node center] at (4.562,0.677) {$40$};
\draw[gp path] (5.391,0.985)--(5.391,1.165);
\draw[gp path] (5.391,8.441)--(5.391,8.261);
\node[gp node center] at (5.391,0.677) {$60$};
\draw[gp path] (6.219,0.985)--(6.219,1.165);
\draw[gp path] (6.219,8.441)--(6.219,8.261);
\node[gp node center] at (6.219,0.677) {$80$};
\draw[gp path] (7.048,0.985)--(7.048,1.165);
\draw[gp path] (7.048,8.441)--(7.048,8.261);
\node[gp node center] at (7.048,0.677) {$100$};
\draw[gp path] (7.876,0.985)--(7.876,1.165);
\draw[gp path] (7.876,8.441)--(7.876,8.261);
\node[gp node center] at (7.876,0.677) {$120$};
\draw[gp path] (8.705,0.985)--(8.705,1.165);
\draw[gp path] (8.705,8.441)--(8.705,8.261);
\node[gp node center] at (8.705,0.677) {$140$};
\draw[gp path] (9.533,0.985)--(9.533,1.165);
\draw[gp path] (9.533,8.441)--(9.533,8.261);
\node[gp node center] at (9.533,0.677) {$160$};
\draw[gp path] (10.362,0.985)--(10.362,1.165);
\draw[gp path] (10.362,8.441)--(10.362,8.261);
\node[gp node center] at (10.362,0.677) {$180$};
\gpsetlinetype{gp lt axes}
\gpsetdashtype{gp dt axes}
\gpsetlinewidth{1.50}
\draw[gp path] (2.905,4.713)--(10.362,4.713);
\gpsetlinetype{gp lt border}
\gpsetdashtype{gp dt solid}
\gpsetlinewidth{1.00}
\draw[gp path] (2.905,8.441)--(2.905,0.985)--(10.362,0.985)--(10.362,8.441)--cycle;
\node[gp node center,rotate=-270] at (1.877,4.713) {相殺電流$i$ / $\si{\milli\ampere}$};
\node[gp node center] at (6.633,0.215) {角度$\theta$ / $\si{\degree}$};
\gpcolor{rgb color={0.000,0.000,0.000}}
\gpsetpointsize{4.00}
\gppoint{gp mark 6}{(2.905,4.988)}
\gppoint{gp mark 6}{(3.029,4.713)}
\gppoint{gp mark 6}{(3.154,4.451)}
\gppoint{gp mark 6}{(3.278,4.182)}
\gppoint{gp mark 6}{(3.402,3.925)}
\gppoint{gp mark 6}{(3.526,3.670)}
\gppoint{gp mark 6}{(3.651,3.415)}
\gppoint{gp mark 6}{(3.775,3.166)}
\gppoint{gp mark 6}{(3.899,2.932)}
\gppoint{gp mark 6}{(4.024,2.692)}
\gppoint{gp mark 6}{(4.148,2.482)}
\gppoint{gp mark 6}{(4.272,2.273)}
\gppoint{gp mark 6}{(4.396,2.076)}
\gppoint{gp mark 6}{(4.521,1.890)}
\gppoint{gp mark 6}{(4.645,1.717)}
\gppoint{gp mark 6}{(4.769,1.569)}
\gppoint{gp mark 6}{(4.894,1.442)}
\gppoint{gp mark 6}{(5.018,1.336)}
\gppoint{gp mark 6}{(5.142,1.257)}
\gppoint{gp mark 6}{(5.266,1.212)}
\gppoint{gp mark 6}{(5.391,1.198)}
\gppoint{gp mark 6}{(5.515,1.228)}
\gppoint{gp mark 6}{(5.639,1.306)}
\gppoint{gp mark 6}{(5.764,1.431)}
\gppoint{gp mark 6}{(5.888,1.614)}
\gppoint{gp mark 6}{(6.012,1.866)}
\gppoint{gp mark 6}{(6.136,2.195)}
\gppoint{gp mark 6}{(6.261,2.594)}
\gppoint{gp mark 6}{(6.385,3.073)}
\gppoint{gp mark 6}{(6.509,3.610)}
\gppoint{gp mark 6}{(6.634,4.204)}
\gppoint{gp mark 6}{(6.758,4.849)}
\gppoint{gp mark 6}{(6.882,5.435)}
\gppoint{gp mark 6}{(7.006,6.038)}
\gppoint{gp mark 6}{(7.131,6.585)}
\gppoint{gp mark 6}{(7.255,6.995)}
\gppoint{gp mark 6}{(7.379,7.386)}
\gppoint{gp mark 6}{(7.503,7.715)}
\gppoint{gp mark 6}{(7.628,7.936)}
\gppoint{gp mark 6}{(7.752,8.095)}
\gppoint{gp mark 6}{(7.876,8.204)}
\gppoint{gp mark 6}{(8.001,8.261)}
\gppoint{gp mark 6}{(8.125,8.276)}
\gppoint{gp mark 6}{(8.249,8.258)}
\gppoint{gp mark 6}{(8.373,8.202)}
\gppoint{gp mark 6}{(8.498,8.120)}
\gppoint{gp mark 6}{(8.622,8.009)}
\gppoint{gp mark 6}{(8.746,7.876)}
\gppoint{gp mark 6}{(8.871,7.730)}
\gppoint{gp mark 6}{(8.995,7.549)}
\gppoint{gp mark 6}{(9.119,7.376)}
\gppoint{gp mark 6}{(9.243,7.161)}
\gppoint{gp mark 6}{(9.368,6.948)}
\gppoint{gp mark 6}{(9.492,6.731)}
\gppoint{gp mark 6}{(9.616,6.490)}
\gppoint{gp mark 6}{(9.741,6.253)}
\gppoint{gp mark 6}{(9.865,5.997)}
\gppoint{gp mark 6}{(9.989,5.745)}
\gppoint{gp mark 6}{(10.113,5.491)}
\gppoint{gp mark 6}{(10.238,5.229)}
\gppoint{gp mark 6}{(10.362,4.975)}
\gpcolor{color=gp lt color border}
\draw[gp path] (2.905,8.441)--(2.905,0.985)--(10.362,0.985)--(10.362,8.441)--cycle;
%% coordinates of the plot area
\gpdefrectangularnode{gp plot 1}{\pgfpoint{2.905cm}{0.985cm}}{\pgfpoint{10.362cm}{8.441cm}}
\end{tikzpicture}
%% gnuplot variables

    \caption{相殺電流の角度依存性}
  \end{center}
\end{figure}

\begin{figure}[hptb]
  \begin{center}
    \begin{tikzpicture}[gnuplot]
%% generated with GNUPLOT 5.2p8 (Lua 5.3; terminal rev. Nov 2018, script rev. 108)
%% 2021年04月19日 15時43分40秒
\path (0.000,0.000) rectangle (12.500,8.750);
\gpcolor{color=gp lt color border}
\gpsetlinetype{gp lt border}
\gpsetdashtype{gp dt solid}
\gpsetlinewidth{1.00}
\draw[gp path] (3.273,1.731)--(3.453,1.731);
\draw[gp path] (10.730,1.731)--(10.550,1.731);
\node[gp node right] at (3.089,1.731) {$-0.0004$};
\draw[gp path] (3.273,2.476)--(3.363,2.476);
\draw[gp path] (10.730,2.476)--(10.640,2.476);
\draw[gp path] (3.273,3.222)--(3.453,3.222);
\draw[gp path] (10.730,3.222)--(10.550,3.222);
\node[gp node right] at (3.089,3.222) {$-0.0002$};
\draw[gp path] (3.273,3.967)--(3.363,3.967);
\draw[gp path] (10.730,3.967)--(10.640,3.967);
\draw[gp path] (3.273,4.713)--(3.453,4.713);
\draw[gp path] (10.730,4.713)--(10.550,4.713);
\node[gp node right] at (3.089,4.713) {$0$};
\draw[gp path] (3.273,5.459)--(3.363,5.459);
\draw[gp path] (10.730,5.459)--(10.640,5.459);
\draw[gp path] (3.273,6.204)--(3.453,6.204);
\draw[gp path] (10.730,6.204)--(10.550,6.204);
\node[gp node right] at (3.089,6.204) {$0.0002$};
\draw[gp path] (3.273,6.950)--(3.363,6.950);
\draw[gp path] (10.730,6.950)--(10.640,6.950);
\draw[gp path] (3.273,7.695)--(3.453,7.695);
\draw[gp path] (10.730,7.695)--(10.550,7.695);
\node[gp node right] at (3.089,7.695) {$0.0004$};
\draw[gp path] (3.469,0.985)--(3.469,1.165);
\draw[gp path] (3.469,8.441)--(3.469,8.261);
\node[gp node center] at (3.469,0.677) {$0$};
\draw[gp path] (4.450,0.985)--(4.450,1.165);
\draw[gp path] (4.450,8.441)--(4.450,8.261);
\node[gp node center] at (4.450,0.677) {$50$};
\draw[gp path] (5.432,0.985)--(5.432,1.165);
\draw[gp path] (5.432,8.441)--(5.432,8.261);
\node[gp node center] at (5.432,0.677) {$100$};
\draw[gp path] (6.413,0.985)--(6.413,1.165);
\draw[gp path] (6.413,8.441)--(6.413,8.261);
\node[gp node center] at (6.413,0.677) {$150$};
\draw[gp path] (7.394,0.985)--(7.394,1.165);
\draw[gp path] (7.394,8.441)--(7.394,8.261);
\node[gp node center] at (7.394,0.677) {$200$};
\draw[gp path] (8.375,0.985)--(8.375,1.165);
\draw[gp path] (8.375,8.441)--(8.375,8.261);
\node[gp node center] at (8.375,0.677) {$250$};
\draw[gp path] (9.356,0.985)--(9.356,1.165);
\draw[gp path] (9.356,8.441)--(9.356,8.261);
\node[gp node center] at (9.356,0.677) {$300$};
\draw[gp path] (10.338,0.985)--(10.338,1.165);
\draw[gp path] (10.338,8.441)--(10.338,8.261);
\node[gp node center] at (10.338,0.677) {$350$};
\draw[gp path] (3.273,8.441)--(3.273,0.985)--(10.730,0.985)--(10.730,8.441)--cycle;
\node[gp node center,rotate=-270] at (1.509,4.713) {トルク$L$ / $\si{\newton.\meter}$};
\node[gp node center] at (7.001,0.215) {角度$\theta$ / $\si{\degree}$};
\gpcolor{rgb color={0.000,0.000,0.000}}
\gpsetpointsize{4.00}
\gppoint{gp mark 6}{(3.469,2.387)}
\gppoint{gp mark 6}{(3.528,2.811)}
\gppoint{gp mark 6}{(3.587,3.277)}
\gppoint{gp mark 6}{(3.646,3.753)}
\gppoint{gp mark 6}{(3.705,4.249)}
\gppoint{gp mark 6}{(3.764,4.693)}
\gppoint{gp mark 6}{(3.822,5.098)}
\gppoint{gp mark 6}{(3.881,5.426)}
\gppoint{gp mark 6}{(3.940,5.678)}
\gppoint{gp mark 6}{(3.999,5.849)}
\gppoint{gp mark 6}{(4.058,5.933)}
\gppoint{gp mark 6}{(4.117,5.928)}
\gppoint{gp mark 6}{(4.176,5.849)}
\gppoint{gp mark 6}{(4.235,5.706)}
\gppoint{gp mark 6}{(4.293,5.504)}
\gppoint{gp mark 6}{(4.352,5.269)}
\gppoint{gp mark 6}{(4.411,5.001)}
\gppoint{gp mark 6}{(4.470,4.714)}
\gppoint{gp mark 6}{(4.529,4.438)}
\gppoint{gp mark 6}{(4.588,4.182)}
\gppoint{gp mark 6}{(4.647,3.968)}
\gppoint{gp mark 6}{(4.706,3.791)}
\gppoint{gp mark 6}{(4.764,3.678)}
\gppoint{gp mark 6}{(4.823,3.636)}
\gppoint{gp mark 6}{(4.882,3.676)}
\gppoint{gp mark 6}{(4.941,3.809)}
\gppoint{gp mark 6}{(5.000,4.016)}
\gppoint{gp mark 6}{(5.059,4.316)}
\gppoint{gp mark 6}{(5.118,4.702)}
\gppoint{gp mark 6}{(5.176,5.149)}
\gppoint{gp mark 6}{(5.235,5.606)}
\gppoint{gp mark 6}{(5.294,6.128)}
\gppoint{gp mark 6}{(5.353,6.590)}
\gppoint{gp mark 6}{(5.412,7.033)}
\gppoint{gp mark 6}{(5.471,7.422)}
\gppoint{gp mark 6}{(5.530,7.725)}
\gppoint{gp mark 6}{(5.589,7.940)}
\gppoint{gp mark 6}{(5.647,8.063)}
\gppoint{gp mark 6}{(5.706,8.090)}
\gppoint{gp mark 6}{(5.765,8.005)}
\gppoint{gp mark 6}{(5.824,7.825)}
\gppoint{gp mark 6}{(5.883,7.546)}
\gppoint{gp mark 6}{(5.942,7.190)}
\gppoint{gp mark 6}{(6.001,6.751)}
\gppoint{gp mark 6}{(6.060,6.230)}
\gppoint{gp mark 6}{(6.118,5.755)}
\gppoint{gp mark 6}{(6.177,5.201)}
\gppoint{gp mark 6}{(6.236,4.615)}
\gppoint{gp mark 6}{(6.295,4.039)}
\gppoint{gp mark 6}{(6.354,3.523)}
\gppoint{gp mark 6}{(6.413,3.018)}
\gppoint{gp mark 6}{(6.472,2.541)}
\gppoint{gp mark 6}{(6.531,2.159)}
\gppoint{gp mark 6}{(6.589,1.835)}
\gppoint{gp mark 6}{(6.648,1.628)}
\gppoint{gp mark 6}{(6.707,1.518)}
\gppoint{gp mark 6}{(6.766,1.492)}
\gppoint{gp mark 6}{(6.825,1.583)}
\gppoint{gp mark 6}{(6.884,1.784)}
\gppoint{gp mark 6}{(6.943,2.054)}
\gppoint{gp mark 6}{(7.002,2.407)}
\gppoint{gp mark 6}{(7.060,2.853)}
\gppoint{gp mark 6}{(7.119,3.330)}
\gppoint{gp mark 6}{(7.178,3.797)}
\gppoint{gp mark 6}{(7.237,4.307)}
\gppoint{gp mark 6}{(7.296,4.666)}
\gppoint{gp mark 6}{(7.355,5.157)}
\gppoint{gp mark 6}{(7.414,5.477)}
\gppoint{gp mark 6}{(7.472,5.730)}
\gppoint{gp mark 6}{(7.531,5.894)}
\gppoint{gp mark 6}{(7.590,5.966)}
\gppoint{gp mark 6}{(7.649,5.980)}
\gppoint{gp mark 6}{(7.708,5.905)}
\gppoint{gp mark 6}{(7.767,5.752)}
\gppoint{gp mark 6}{(7.826,5.554)}
\gppoint{gp mark 6}{(7.885,5.316)}
\gppoint{gp mark 6}{(7.943,5.049)}
\gppoint{gp mark 6}{(8.002,4.773)}
\gppoint{gp mark 6}{(8.061,4.503)}
\gppoint{gp mark 6}{(8.120,4.247)}
\gppoint{gp mark 6}{(8.179,4.018)}
\gppoint{gp mark 6}{(8.238,3.840)}
\gppoint{gp mark 6}{(8.297,3.718)}
\gppoint{gp mark 6}{(8.356,3.682)}
\gppoint{gp mark 6}{(8.414,3.711)}
\gppoint{gp mark 6}{(8.473,3.826)}
\gppoint{gp mark 6}{(8.532,4.050)}
\gppoint{gp mark 6}{(8.591,4.342)}
\gppoint{gp mark 6}{(8.650,4.714)}
\gppoint{gp mark 6}{(8.709,5.149)}
\gppoint{gp mark 6}{(8.768,5.639)}
\gppoint{gp mark 6}{(8.827,6.118)}
\gppoint{gp mark 6}{(8.885,6.603)}
\gppoint{gp mark 6}{(8.944,7.054)}
\gppoint{gp mark 6}{(9.003,7.426)}
\gppoint{gp mark 6}{(9.062,7.731)}
\gppoint{gp mark 6}{(9.121,7.943)}
\gppoint{gp mark 6}{(9.180,8.071)}
\gppoint{gp mark 6}{(9.239,8.094)}
\gppoint{gp mark 6}{(9.297,8.005)}
\gppoint{gp mark 6}{(9.356,7.821)}
\gppoint{gp mark 6}{(9.415,7.535)}
\gppoint{gp mark 6}{(9.474,7.183)}
\gppoint{gp mark 6}{(9.533,6.743)}
\gppoint{gp mark 6}{(9.592,6.254)}
\gppoint{gp mark 6}{(9.651,5.720)}
\gppoint{gp mark 6}{(9.710,5.145)}
\gppoint{gp mark 6}{(9.768,4.610)}
\gppoint{gp mark 6}{(9.827,4.032)}
\gppoint{gp mark 6}{(9.886,3.472)}
\gppoint{gp mark 6}{(9.945,2.987)}
\gppoint{gp mark 6}{(10.004,2.532)}
\gppoint{gp mark 6}{(10.063,2.134)}
\gppoint{gp mark 6}{(10.122,1.832)}
\gppoint{gp mark 6}{(10.181,1.603)}
\gppoint{gp mark 6}{(10.239,1.477)}
\gppoint{gp mark 6}{(10.298,1.456)}
\gppoint{gp mark 6}{(10.357,1.545)}
\gppoint{gp mark 6}{(10.416,1.721)}
\gppoint{gp mark 6}{(10.475,2.021)}
\gppoint{gp mark 6}{(10.534,2.378)}
\gpcolor{color=gp lt color border}
\draw[gp path] (3.273,8.441)--(3.273,0.985)--(10.730,0.985)--(10.730,8.441)--cycle;
%% coordinates of the plot area
\gpdefrectangularnode{gp plot 1}{\pgfpoint{3.273cm}{0.985cm}}{\pgfpoint{10.730cm}{8.441cm}}
\end{tikzpicture}
%% gnuplot variables

    \caption{Fe-Si円盤の磁気トルク曲線}
  \end{center}
\end{figure}

\begin{figure}[hptb]
  \begin{center}
    \begin{tikzpicture}[gnuplot]
%% generated with GNUPLOT 5.2p8 (Lua 5.3; terminal rev. Nov 2018, script rev. 108)
%% 2021年04月27日 14時38分39秒
\path (0.000,0.000) rectangle (12.500,8.750);
\gpcolor{color=gp lt color border}
\gpsetlinetype{gp lt border}
\gpsetdashtype{gp dt solid}
\gpsetlinewidth{1.00}
\draw[gp path] (2.997,0.985)--(3.087,0.985);
\draw[gp path] (10.454,0.985)--(10.364,0.985);
\draw[gp path] (2.997,1.731)--(3.177,1.731);
\draw[gp path] (10.454,1.731)--(10.274,1.731);
\node[gp node right] at (2.813,1.731) {$-2.0$};
\draw[gp path] (2.997,2.476)--(3.087,2.476);
\draw[gp path] (10.454,2.476)--(10.364,2.476);
\draw[gp path] (2.997,3.222)--(3.177,3.222);
\draw[gp path] (10.454,3.222)--(10.274,3.222);
\node[gp node right] at (2.813,3.222) {$-1.0$};
\draw[gp path] (2.997,3.967)--(3.087,3.967);
\draw[gp path] (10.454,3.967)--(10.364,3.967);
\draw[gp path] (2.997,4.713)--(3.177,4.713);
\draw[gp path] (10.454,4.713)--(10.274,4.713);
\node[gp node right] at (2.813,4.713) {$0.0$};
\draw[gp path] (2.997,5.459)--(3.087,5.459);
\draw[gp path] (10.454,5.459)--(10.364,5.459);
\draw[gp path] (2.997,6.204)--(3.177,6.204);
\draw[gp path] (10.454,6.204)--(10.274,6.204);
\node[gp node right] at (2.813,6.204) {$1.0$};
\draw[gp path] (2.997,6.950)--(3.087,6.950);
\draw[gp path] (10.454,6.950)--(10.364,6.950);
\draw[gp path] (2.997,7.695)--(3.177,7.695);
\draw[gp path] (10.454,7.695)--(10.274,7.695);
\node[gp node right] at (2.813,7.695) {$2.0$};
\draw[gp path] (2.997,8.441)--(3.087,8.441);
\draw[gp path] (10.454,8.441)--(10.364,8.441);
\draw[gp path] (3.193,0.985)--(3.193,1.165);
\draw[gp path] (3.193,8.441)--(3.193,8.261);
\node[gp node center] at (3.193,0.677) {$0$};
\draw[gp path] (3.684,0.985)--(3.684,1.075);
\draw[gp path] (3.684,8.441)--(3.684,8.351);
\draw[gp path] (4.174,0.985)--(4.174,1.165);
\draw[gp path] (4.174,8.441)--(4.174,8.261);
\node[gp node center] at (4.174,0.677) {$50$};
\draw[gp path] (4.665,0.985)--(4.665,1.075);
\draw[gp path] (4.665,8.441)--(4.665,8.351);
\draw[gp path] (5.156,0.985)--(5.156,1.165);
\draw[gp path] (5.156,8.441)--(5.156,8.261);
\node[gp node center] at (5.156,0.677) {$100$};
\draw[gp path] (5.646,0.985)--(5.646,1.075);
\draw[gp path] (5.646,8.441)--(5.646,8.351);
\draw[gp path] (6.137,0.985)--(6.137,1.165);
\draw[gp path] (6.137,8.441)--(6.137,8.261);
\node[gp node center] at (6.137,0.677) {$150$};
\draw[gp path] (6.627,0.985)--(6.627,1.075);
\draw[gp path] (6.627,8.441)--(6.627,8.351);
\draw[gp path] (7.118,0.985)--(7.118,1.165);
\draw[gp path] (7.118,8.441)--(7.118,8.261);
\node[gp node center] at (7.118,0.677) {$200$};
\draw[gp path] (7.609,0.985)--(7.609,1.075);
\draw[gp path] (7.609,8.441)--(7.609,8.351);
\draw[gp path] (8.099,0.985)--(8.099,1.165);
\draw[gp path] (8.099,8.441)--(8.099,8.261);
\node[gp node center] at (8.099,0.677) {$250$};
\draw[gp path] (8.590,0.985)--(8.590,1.075);
\draw[gp path] (8.590,8.441)--(8.590,8.351);
\draw[gp path] (9.080,0.985)--(9.080,1.165);
\draw[gp path] (9.080,8.441)--(9.080,8.261);
\node[gp node center] at (9.080,0.677) {$300$};
\draw[gp path] (9.571,0.985)--(9.571,1.075);
\draw[gp path] (9.571,8.441)--(9.571,8.351);
\draw[gp path] (10.062,0.985)--(10.062,1.165);
\draw[gp path] (10.062,8.441)--(10.062,8.261);
\node[gp node center] at (10.062,0.677) {$350$};
\gpsetlinetype{gp lt axes}
\gpsetdashtype{gp dt axes}
\gpsetlinewidth{1.50}
\draw[gp path] (2.997,4.713)--(10.454,4.713);
\gpsetlinetype{gp lt border}
\gpsetdashtype{gp dt solid}
\gpsetlinewidth{1.00}
\draw[gp path] (2.997,8.441)--(2.997,0.985)--(10.454,0.985)--(10.454,8.441)--cycle;
\node[gp node center,rotate=-270] at (1.785,4.713) {単位体積あたりのトルク$L$ / $\times 10^{4}\ \si{\newton.\meter^{-2}}$};
\node[gp node center] at (6.725,0.215) {角度$\theta$ / $\si{\degree}$};
\draw[gp path] (3.181,1.165)--(3.181,2.551)--(5.937,2.551)--(5.937,1.165)--cycle;
\node[gp node right] at (4.653,2.089) {測定値};
\gpcolor{rgb color={0.000,0.000,0.000}}
\gpsetpointsize{4.00}
\gppoint{gp mark 6}{(3.193,2.598)}
\gppoint{gp mark 6}{(3.252,2.983)}
\gppoint{gp mark 6}{(3.311,3.407)}
\gppoint{gp mark 6}{(3.370,3.840)}
\gppoint{gp mark 6}{(3.429,4.291)}
\gppoint{gp mark 6}{(3.488,4.695)}
\gppoint{gp mark 6}{(3.546,5.064)}
\gppoint{gp mark 6}{(3.605,5.361)}
\gppoint{gp mark 6}{(3.664,5.591)}
\gppoint{gp mark 6}{(3.723,5.746)}
\gppoint{gp mark 6}{(3.782,5.823)}
\gppoint{gp mark 6}{(3.841,5.818)}
\gppoint{gp mark 6}{(3.900,5.746)}
\gppoint{gp mark 6}{(3.959,5.617)}
\gppoint{gp mark 6}{(4.017,5.433)}
\gppoint{gp mark 6}{(4.076,5.219)}
\gppoint{gp mark 6}{(4.135,4.975)}
\gppoint{gp mark 6}{(4.194,4.714)}
\gppoint{gp mark 6}{(4.253,4.463)}
\gppoint{gp mark 6}{(4.312,4.230)}
\gppoint{gp mark 6}{(4.371,4.036)}
\gppoint{gp mark 6}{(4.430,3.874)}
\gppoint{gp mark 6}{(4.488,3.772)}
\gppoint{gp mark 6}{(4.547,3.734)}
\gppoint{gp mark 6}{(4.606,3.770)}
\gppoint{gp mark 6}{(4.665,3.891)}
\gppoint{gp mark 6}{(4.724,4.079)}
\gppoint{gp mark 6}{(4.783,4.352)}
\gppoint{gp mark 6}{(4.842,4.703)}
\gppoint{gp mark 6}{(4.900,5.109)}
\gppoint{gp mark 6}{(4.959,5.525)}
\gppoint{gp mark 6}{(5.018,6.000)}
\gppoint{gp mark 6}{(5.077,6.420)}
\gppoint{gp mark 6}{(5.136,6.823)}
\gppoint{gp mark 6}{(5.195,7.177)}
\gppoint{gp mark 6}{(5.254,7.452)}
\gppoint{gp mark 6}{(5.313,7.648)}
\gppoint{gp mark 6}{(5.371,7.759)}
\gppoint{gp mark 6}{(5.430,7.784)}
\gppoint{gp mark 6}{(5.489,7.707)}
\gppoint{gp mark 6}{(5.548,7.544)}
\gppoint{gp mark 6}{(5.607,7.289)}
\gppoint{gp mark 6}{(5.666,6.966)}
\gppoint{gp mark 6}{(5.725,6.567)}
\gppoint{gp mark 6}{(5.784,6.093)}
\gppoint{gp mark 6}{(5.842,5.661)}
\gppoint{gp mark 6}{(5.901,5.156)}
\gppoint{gp mark 6}{(5.960,4.624)}
\gppoint{gp mark 6}{(6.019,4.100)}
\gppoint{gp mark 6}{(6.078,3.630)}
\gppoint{gp mark 6}{(6.137,3.172)}
\gppoint{gp mark 6}{(6.196,2.738)}
\gppoint{gp mark 6}{(6.255,2.391)}
\gppoint{gp mark 6}{(6.313,2.095)}
\gppoint{gp mark 6}{(6.372,1.908)}
\gppoint{gp mark 6}{(6.431,1.807)}
\gppoint{gp mark 6}{(6.490,1.783)}
\gppoint{gp mark 6}{(6.549,1.866)}
\gppoint{gp mark 6}{(6.608,2.049)}
\gppoint{gp mark 6}{(6.667,2.295)}
\gppoint{gp mark 6}{(6.726,2.616)}
\gppoint{gp mark 6}{(6.784,3.021)}
\gppoint{gp mark 6}{(6.843,3.455)}
\gppoint{gp mark 6}{(6.902,3.880)}
\gppoint{gp mark 6}{(6.961,4.343)}
\gppoint{gp mark 6}{(7.020,4.670)}
\gppoint{gp mark 6}{(7.079,5.116)}
\gppoint{gp mark 6}{(7.138,5.408)}
\gppoint{gp mark 6}{(7.196,5.638)}
\gppoint{gp mark 6}{(7.255,5.787)}
\gppoint{gp mark 6}{(7.314,5.852)}
\gppoint{gp mark 6}{(7.373,5.865)}
\gppoint{gp mark 6}{(7.432,5.797)}
\gppoint{gp mark 6}{(7.491,5.658)}
\gppoint{gp mark 6}{(7.550,5.478)}
\gppoint{gp mark 6}{(7.609,5.261)}
\gppoint{gp mark 6}{(7.667,5.019)}
\gppoint{gp mark 6}{(7.726,4.768)}
\gppoint{gp mark 6}{(7.785,4.522)}
\gppoint{gp mark 6}{(7.844,4.289)}
\gppoint{gp mark 6}{(7.903,4.081)}
\gppoint{gp mark 6}{(7.962,3.919)}
\gppoint{gp mark 6}{(8.021,3.808)}
\gppoint{gp mark 6}{(8.080,3.775)}
\gppoint{gp mark 6}{(8.138,3.802)}
\gppoint{gp mark 6}{(8.197,3.907)}
\gppoint{gp mark 6}{(8.256,4.110)}
\gppoint{gp mark 6}{(8.315,4.375)}
\gppoint{gp mark 6}{(8.374,4.714)}
\gppoint{gp mark 6}{(8.433,5.110)}
\gppoint{gp mark 6}{(8.492,5.555)}
\gppoint{gp mark 6}{(8.551,5.991)}
\gppoint{gp mark 6}{(8.609,6.432)}
\gppoint{gp mark 6}{(8.668,6.842)}
\gppoint{gp mark 6}{(8.727,7.180)}
\gppoint{gp mark 6}{(8.786,7.458)}
\gppoint{gp mark 6}{(8.845,7.651)}
\gppoint{gp mark 6}{(8.904,7.767)}
\gppoint{gp mark 6}{(8.963,7.788)}
\gppoint{gp mark 6}{(9.021,7.707)}
\gppoint{gp mark 6}{(9.080,7.539)}
\gppoint{gp mark 6}{(9.139,7.279)}
\gppoint{gp mark 6}{(9.198,6.959)}
\gppoint{gp mark 6}{(9.257,6.559)}
\gppoint{gp mark 6}{(9.316,6.114)}
\gppoint{gp mark 6}{(9.375,5.629)}
\gppoint{gp mark 6}{(9.434,5.106)}
\gppoint{gp mark 6}{(9.492,4.619)}
\gppoint{gp mark 6}{(9.551,4.094)}
\gppoint{gp mark 6}{(9.610,3.584)}
\gppoint{gp mark 6}{(9.669,3.143)}
\gppoint{gp mark 6}{(9.728,2.729)}
\gppoint{gp mark 6}{(9.787,2.368)}
\gppoint{gp mark 6}{(9.846,2.093)}
\gppoint{gp mark 6}{(9.905,1.885)}
\gppoint{gp mark 6}{(9.963,1.770)}
\gppoint{gp mark 6}{(10.022,1.751)}
\gppoint{gp mark 6}{(10.081,1.832)}
\gppoint{gp mark 6}{(10.140,1.992)}
\gppoint{gp mark 6}{(10.199,2.265)}
\gppoint{gp mark 6}{(10.258,2.590)}
\gppoint{gp mark 6}{(5.295,2.089)}
\gpcolor{color=gp lt color border}
\node[gp node right] at (4.653,1.627) {理論値};
\gpcolor{rgb color={0.000,0.000,0.000}}
\gpsetdashtype{gp dt 3}
\draw[gp path] (4.837,1.627)--(5.753,1.627);
\draw[gp path] (2.997,1.589)--(3.072,1.856)--(3.148,2.256)--(3.223,2.757)--(3.298,3.321)%
  --(3.374,3.908)--(3.449,4.477)--(3.524,4.988)--(3.600,5.409)--(3.675,5.715)--(3.750,5.889)%
  --(3.826,5.926)--(3.901,5.832)--(3.976,5.621)--(4.052,5.319)--(4.127,4.957)--(4.202,4.570)%
  --(4.277,4.198)--(4.353,3.876)--(4.428,3.639)--(4.503,3.512)--(4.579,3.514)--(4.654,3.652)%
  --(4.729,3.924)--(4.805,4.317)--(4.880,4.808)--(4.955,5.365)--(5.031,5.951)--(5.106,6.525)%
  --(5.181,7.046)--(5.257,7.476)--(5.332,7.781)--(5.407,7.935)--(5.483,7.922)--(5.558,7.736)%
  --(5.633,7.382)--(5.709,6.879)--(5.784,6.252)--(5.859,5.537)--(5.935,4.774)--(6.010,4.008)%
  --(6.085,3.282)--(6.161,2.639)--(6.236,2.114)--(6.311,1.735)--(6.387,1.522)--(6.462,1.482)%
  --(6.537,1.610)--(6.613,1.892)--(6.688,2.305)--(6.763,2.814)--(6.838,3.383)--(6.914,3.970)%
  --(6.989,4.534)--(7.064,5.037)--(7.140,5.447)--(7.215,5.740)--(7.290,5.900)--(7.366,5.923)%
  --(7.441,5.815)--(7.516,5.593)--(7.592,5.283)--(7.667,4.916)--(7.742,4.530)--(7.818,4.161)%
  --(7.893,3.847)--(7.968,3.620)--(8.044,3.506)--(8.119,3.522)--(8.194,3.675)--(8.270,3.960)%
  --(8.345,4.365)--(8.420,4.864)--(8.496,5.426)--(8.571,6.012)--(8.646,6.583)--(8.722,7.097)%
  --(8.797,7.515)--(8.872,7.805)--(8.948,7.942)--(9.023,7.910)--(9.098,7.706)--(9.174,7.336)%
  --(9.249,6.818)--(9.324,6.180)--(9.399,5.458)--(9.475,4.693)--(9.550,3.928)--(9.625,3.210)%
  --(9.701,2.577)--(9.776,2.066)--(9.851,1.705)--(9.927,1.510)--(10.002,1.487)--(10.077,1.633)%
  --(10.153,1.930)--(10.228,2.354)--(10.303,2.872)--(10.379,3.445)--(10.454,4.031);
\gpcolor{color=gp lt color border}
\gpsetdashtype{gp dt solid}
\draw[gp path] (2.997,8.441)--(2.997,0.985)--(10.454,0.985)--(10.454,8.441)--cycle;
%% coordinates of the plot area
\gpdefrectangularnode{gp plot 1}{\pgfpoint{2.997cm}{0.985cm}}{\pgfpoint{10.454cm}{8.441cm}}
\end{tikzpicture}
%% gnuplot variables

    \caption{Fe-Si円盤の磁気トルク曲線と理論曲線}
  \end{center}
\end{figure}

\begin{figure}[hptb]
  \begin{center}
    \begin{tikzpicture}[gnuplot]
%% generated with GNUPLOT 5.2p8 (Lua 5.3; terminal rev. Nov 2018, script rev. 108)
%% 2021年04月23日 15時55分18秒
\path (0.000,0.000) rectangle (12.500,8.750);
\gpcolor{color=gp lt color border}
\gpsetlinetype{gp lt border}
\gpsetdashtype{gp dt solid}
\gpsetlinewidth{1.00}
\draw[gp path] (2.905,1.163)--(3.085,1.163);
\draw[gp path] (10.362,1.163)--(10.182,1.163);
\node[gp node right] at (2.721,1.163) {$-20$};
\draw[gp path] (2.905,2.050)--(3.085,2.050);
\draw[gp path] (10.362,2.050)--(10.182,2.050);
\node[gp node right] at (2.721,2.050) {$-15$};
\draw[gp path] (2.905,2.938)--(3.085,2.938);
\draw[gp path] (10.362,2.938)--(10.182,2.938);
\node[gp node right] at (2.721,2.938) {$-10$};
\draw[gp path] (2.905,3.825)--(3.085,3.825);
\draw[gp path] (10.362,3.825)--(10.182,3.825);
\node[gp node right] at (2.721,3.825) {$-5$};
\draw[gp path] (2.905,4.713)--(3.085,4.713);
\draw[gp path] (10.362,4.713)--(10.182,4.713);
\node[gp node right] at (2.721,4.713) {$0$};
\draw[gp path] (2.905,5.601)--(3.085,5.601);
\draw[gp path] (10.362,5.601)--(10.182,5.601);
\node[gp node right] at (2.721,5.601) {$5$};
\draw[gp path] (2.905,6.488)--(3.085,6.488);
\draw[gp path] (10.362,6.488)--(10.182,6.488);
\node[gp node right] at (2.721,6.488) {$10$};
\draw[gp path] (2.905,7.376)--(3.085,7.376);
\draw[gp path] (10.362,7.376)--(10.182,7.376);
\node[gp node right] at (2.721,7.376) {$15$};
\draw[gp path] (2.905,8.263)--(3.085,8.263);
\draw[gp path] (10.362,8.263)--(10.182,8.263);
\node[gp node right] at (2.721,8.263) {$20$};
\draw[gp path] (2.905,0.985)--(2.905,1.165);
\draw[gp path] (2.905,8.441)--(2.905,8.261);
\node[gp node center] at (2.905,0.677) {$0$};
\draw[gp path] (3.734,0.985)--(3.734,1.165);
\draw[gp path] (3.734,8.441)--(3.734,8.261);
\node[gp node center] at (3.734,0.677) {$20$};
\draw[gp path] (4.562,0.985)--(4.562,1.165);
\draw[gp path] (4.562,8.441)--(4.562,8.261);
\node[gp node center] at (4.562,0.677) {$40$};
\draw[gp path] (5.391,0.985)--(5.391,1.165);
\draw[gp path] (5.391,8.441)--(5.391,8.261);
\node[gp node center] at (5.391,0.677) {$60$};
\draw[gp path] (6.219,0.985)--(6.219,1.165);
\draw[gp path] (6.219,8.441)--(6.219,8.261);
\node[gp node center] at (6.219,0.677) {$80$};
\draw[gp path] (7.048,0.985)--(7.048,1.165);
\draw[gp path] (7.048,8.441)--(7.048,8.261);
\node[gp node center] at (7.048,0.677) {$100$};
\draw[gp path] (7.876,0.985)--(7.876,1.165);
\draw[gp path] (7.876,8.441)--(7.876,8.261);
\node[gp node center] at (7.876,0.677) {$120$};
\draw[gp path] (8.705,0.985)--(8.705,1.165);
\draw[gp path] (8.705,8.441)--(8.705,8.261);
\node[gp node center] at (8.705,0.677) {$140$};
\draw[gp path] (9.533,0.985)--(9.533,1.165);
\draw[gp path] (9.533,8.441)--(9.533,8.261);
\node[gp node center] at (9.533,0.677) {$160$};
\draw[gp path] (10.362,0.985)--(10.362,1.165);
\draw[gp path] (10.362,8.441)--(10.362,8.261);
\node[gp node center] at (10.362,0.677) {$180$};
\draw[gp path] (2.905,8.441)--(2.905,0.985)--(10.362,0.985)--(10.362,8.441)--cycle;
\node[gp node center,rotate=-270] at (1.877,4.713) {相殺電流$i$ / $\si{\milli\ampere}$};
\node[gp node center] at (6.633,0.215) {角度$\theta$ / $\si{\degree}$};
\gpcolor{rgb color={0.000,0.000,0.000}}
\gpsetpointsize{4.00}
\gppoint{gp mark 6}{(2.905,4.846)}
\gppoint{gp mark 6}{(3.029,4.580)}
\gppoint{gp mark 6}{(3.154,4.314)}
\gppoint{gp mark 6}{(3.278,4.047)}
\gppoint{gp mark 6}{(3.402,3.781)}
\gppoint{gp mark 6}{(3.526,3.541)}
\gppoint{gp mark 6}{(3.651,3.275)}
\gppoint{gp mark 6}{(3.775,3.035)}
\gppoint{gp mark 6}{(3.899,2.796)}
\gppoint{gp mark 6}{(4.024,2.556)}
\gppoint{gp mark 6}{(4.148,2.343)}
\gppoint{gp mark 6}{(4.272,2.130)}
\gppoint{gp mark 6}{(4.396,1.944)}
\gppoint{gp mark 6}{(4.521,1.784)}
\gppoint{gp mark 6}{(4.645,1.624)}
\gppoint{gp mark 6}{(4.769,1.491)}
\gppoint{gp mark 6}{(4.894,1.384)}
\gppoint{gp mark 6}{(5.018,1.305)}
\gppoint{gp mark 6}{(5.142,1.251)}
\gppoint{gp mark 6}{(5.266,1.225)}
\gppoint{gp mark 6}{(5.391,1.225)}
\gppoint{gp mark 6}{(5.515,1.305)}
\gppoint{gp mark 6}{(5.639,1.384)}
\gppoint{gp mark 6}{(5.764,1.544)}
\gppoint{gp mark 6}{(5.888,1.784)}
\gppoint{gp mark 6}{(6.012,2.077)}
\gppoint{gp mark 6}{(6.136,2.450)}
\gppoint{gp mark 6}{(6.261,2.849)}
\gppoint{gp mark 6}{(6.385,3.328)}
\gppoint{gp mark 6}{(6.509,3.888)}
\gppoint{gp mark 6}{(6.634,4.473)}
\gppoint{gp mark 6}{(6.758,5.112)}
\gppoint{gp mark 6}{(6.882,5.698)}
\gppoint{gp mark 6}{(7.006,6.257)}
\gppoint{gp mark 6}{(7.131,6.737)}
\gppoint{gp mark 6}{(7.255,7.163)}
\gppoint{gp mark 6}{(7.379,7.509)}
\gppoint{gp mark 6}{(7.503,7.775)}
\gppoint{gp mark 6}{(7.628,8.015)}
\gppoint{gp mark 6}{(7.752,8.148)}
\gppoint{gp mark 6}{(7.876,8.228)}
\gppoint{gp mark 6}{(8.001,8.308)}
\gppoint{gp mark 6}{(8.125,8.308)}
\gppoint{gp mark 6}{(8.249,8.281)}
\gppoint{gp mark 6}{(8.373,8.201)}
\gppoint{gp mark 6}{(8.498,8.121)}
\gppoint{gp mark 6}{(8.622,7.988)}
\gppoint{gp mark 6}{(8.746,7.855)}
\gppoint{gp mark 6}{(8.871,7.695)}
\gppoint{gp mark 6}{(8.995,7.509)}
\gppoint{gp mark 6}{(9.119,7.323)}
\gppoint{gp mark 6}{(9.243,7.110)}
\gppoint{gp mark 6}{(9.368,6.897)}
\gppoint{gp mark 6}{(9.492,6.657)}
\gppoint{gp mark 6}{(9.616,6.417)}
\gppoint{gp mark 6}{(9.741,6.151)}
\gppoint{gp mark 6}{(9.865,5.911)}
\gppoint{gp mark 6}{(9.989,5.672)}
\gppoint{gp mark 6}{(10.113,5.379)}
\gppoint{gp mark 6}{(10.238,5.112)}
\gppoint{gp mark 6}{(10.362,4.873)}
\gpcolor{color=gp lt color border}
\draw[gp path] (2.905,8.441)--(2.905,0.985)--(10.362,0.985)--(10.362,8.441)--cycle;
%% coordinates of the plot area
\gpdefrectangularnode{gp plot 1}{\pgfpoint{2.905cm}{0.985cm}}{\pgfpoint{10.362cm}{8.441cm}}
\end{tikzpicture}
%% gnuplot variables

    \caption{相殺電流の角度依存性}
  \end{center}
\end{figure}

\begin{figure}[hptb]
  \begin{center}
    \begin{tikzpicture}[gnuplot]
%% generated with GNUPLOT 5.2p8 (Lua 5.3; terminal rev. Nov 2018, script rev. 108)
%% 2021年04月23日 16時10分25秒
\path (0.000,0.000) rectangle (12.500,8.750);
\gpcolor{color=gp lt color border}
\gpsetlinetype{gp lt border}
\gpsetdashtype{gp dt solid}
\gpsetlinewidth{1.00}
\draw[gp path] (2.997,1.358)--(3.177,1.358);
\draw[gp path] (10.454,1.358)--(10.274,1.358);
\node[gp node right] at (2.813,1.358) {$-9.0$};
\draw[gp path] (2.997,1.731)--(3.087,1.731);
\draw[gp path] (10.454,1.731)--(10.364,1.731);
\draw[gp path] (2.997,2.103)--(3.087,2.103);
\draw[gp path] (10.454,2.103)--(10.364,2.103);
\draw[gp path] (2.997,2.476)--(3.177,2.476);
\draw[gp path] (10.454,2.476)--(10.274,2.476);
\node[gp node right] at (2.813,2.476) {$-6.0$};
\draw[gp path] (2.997,2.849)--(3.087,2.849);
\draw[gp path] (10.454,2.849)--(10.364,2.849);
\draw[gp path] (2.997,3.222)--(3.087,3.222);
\draw[gp path] (10.454,3.222)--(10.364,3.222);
\draw[gp path] (2.997,3.595)--(3.177,3.595);
\draw[gp path] (10.454,3.595)--(10.274,3.595);
\node[gp node right] at (2.813,3.595) {$-3.0$};
\draw[gp path] (2.997,3.967)--(3.087,3.967);
\draw[gp path] (10.454,3.967)--(10.364,3.967);
\draw[gp path] (2.997,4.340)--(3.087,4.340);
\draw[gp path] (10.454,4.340)--(10.364,4.340);
\draw[gp path] (2.997,4.713)--(3.177,4.713);
\draw[gp path] (10.454,4.713)--(10.274,4.713);
\node[gp node right] at (2.813,4.713) {$0.0$};
\draw[gp path] (2.997,5.086)--(3.087,5.086);
\draw[gp path] (10.454,5.086)--(10.364,5.086);
\draw[gp path] (2.997,5.459)--(3.087,5.459);
\draw[gp path] (10.454,5.459)--(10.364,5.459);
\draw[gp path] (2.997,5.831)--(3.177,5.831);
\draw[gp path] (10.454,5.831)--(10.274,5.831);
\node[gp node right] at (2.813,5.831) {$3.0$};
\draw[gp path] (2.997,6.204)--(3.087,6.204);
\draw[gp path] (10.454,6.204)--(10.364,6.204);
\draw[gp path] (2.997,6.577)--(3.087,6.577);
\draw[gp path] (10.454,6.577)--(10.364,6.577);
\draw[gp path] (2.997,6.950)--(3.177,6.950);
\draw[gp path] (10.454,6.950)--(10.274,6.950);
\node[gp node right] at (2.813,6.950) {$6.0$};
\draw[gp path] (2.997,7.323)--(3.087,7.323);
\draw[gp path] (10.454,7.323)--(10.364,7.323);
\draw[gp path] (2.997,7.695)--(3.087,7.695);
\draw[gp path] (10.454,7.695)--(10.364,7.695);
\draw[gp path] (2.997,8.068)--(3.177,8.068);
\draw[gp path] (10.454,8.068)--(10.274,8.068);
\node[gp node right] at (2.813,8.068) {$9.0$};
\draw[gp path] (2.997,0.985)--(2.997,1.165);
\draw[gp path] (2.997,8.441)--(2.997,8.261);
\node[gp node center] at (2.997,0.677) {$0$};
\draw[gp path] (3.411,0.985)--(3.411,1.075);
\draw[gp path] (3.411,8.441)--(3.411,8.351);
\draw[gp path] (3.826,0.985)--(3.826,1.075);
\draw[gp path] (3.826,8.441)--(3.826,8.351);
\draw[gp path] (4.240,0.985)--(4.240,1.165);
\draw[gp path] (4.240,8.441)--(4.240,8.261);
\node[gp node center] at (4.240,0.677) {$30$};
\draw[gp path] (4.654,0.985)--(4.654,1.075);
\draw[gp path] (4.654,8.441)--(4.654,8.351);
\draw[gp path] (5.068,0.985)--(5.068,1.075);
\draw[gp path] (5.068,8.441)--(5.068,8.351);
\draw[gp path] (5.483,0.985)--(5.483,1.165);
\draw[gp path] (5.483,8.441)--(5.483,8.261);
\node[gp node center] at (5.483,0.677) {$60$};
\draw[gp path] (5.897,0.985)--(5.897,1.075);
\draw[gp path] (5.897,8.441)--(5.897,8.351);
\draw[gp path] (6.311,0.985)--(6.311,1.075);
\draw[gp path] (6.311,8.441)--(6.311,8.351);
\draw[gp path] (6.726,0.985)--(6.726,1.165);
\draw[gp path] (6.726,8.441)--(6.726,8.261);
\node[gp node center] at (6.726,0.677) {$90$};
\draw[gp path] (7.140,0.985)--(7.140,1.075);
\draw[gp path] (7.140,8.441)--(7.140,8.351);
\draw[gp path] (7.554,0.985)--(7.554,1.075);
\draw[gp path] (7.554,8.441)--(7.554,8.351);
\draw[gp path] (7.968,0.985)--(7.968,1.165);
\draw[gp path] (7.968,8.441)--(7.968,8.261);
\node[gp node center] at (7.968,0.677) {$120$};
\draw[gp path] (8.383,0.985)--(8.383,1.075);
\draw[gp path] (8.383,8.441)--(8.383,8.351);
\draw[gp path] (8.797,0.985)--(8.797,1.075);
\draw[gp path] (8.797,8.441)--(8.797,8.351);
\draw[gp path] (9.211,0.985)--(9.211,1.165);
\draw[gp path] (9.211,8.441)--(9.211,8.261);
\node[gp node center] at (9.211,0.677) {$150$};
\draw[gp path] (9.625,0.985)--(9.625,1.075);
\draw[gp path] (9.625,8.441)--(9.625,8.351);
\draw[gp path] (10.040,0.985)--(10.040,1.075);
\draw[gp path] (10.040,8.441)--(10.040,8.351);
\draw[gp path] (10.454,0.985)--(10.454,1.165);
\draw[gp path] (10.454,8.441)--(10.454,8.261);
\node[gp node center] at (10.454,0.677) {$180$};
\draw[gp path] (2.997,8.441)--(2.997,0.985)--(10.454,0.985)--(10.454,8.441)--cycle;
\node[gp node center,rotate=-270] at (1.785,4.713) {単位体積あたりのトルク$L$ / $\times 10^{4}\ \si{\newton.\meter^{-2}}$};
\node[gp node center] at (6.725,0.215) {角度$\theta$ / $\si{\degree}$};
\gpcolor{rgb color={0.000,0.000,0.000}}
\gpsetpointsize{4.00}
\gppoint{gp mark 6}{(2.997,4.625)}
\gppoint{gp mark 6}{(3.121,4.403)}
\gppoint{gp mark 6}{(3.246,4.182)}
\gppoint{gp mark 6}{(3.370,3.983)}
\gppoint{gp mark 6}{(3.494,3.784)}
\gppoint{gp mark 6}{(3.618,3.585)}
\gppoint{gp mark 6}{(3.743,3.386)}
\gppoint{gp mark 6}{(3.867,3.187)}
\gppoint{gp mark 6}{(3.991,2.988)}
\gppoint{gp mark 6}{(4.116,2.811)}
\gppoint{gp mark 6}{(4.240,2.634)}
\gppoint{gp mark 6}{(4.364,2.435)}
\gppoint{gp mark 6}{(4.488,2.280)}
\gppoint{gp mark 6}{(4.613,2.126)}
\gppoint{gp mark 6}{(4.737,1.971)}
\gppoint{gp mark 6}{(4.861,1.838)}
\gppoint{gp mark 6}{(4.986,1.705)}
\gppoint{gp mark 6}{(5.110,1.595)}
\gppoint{gp mark 6}{(5.234,1.506)}
\gppoint{gp mark 6}{(5.358,1.440)}
\gppoint{gp mark 6}{(5.483,1.374)}
\gppoint{gp mark 6}{(5.607,1.352)}
\gppoint{gp mark 6}{(5.731,1.374)}
\gppoint{gp mark 6}{(5.856,1.440)}
\gppoint{gp mark 6}{(5.980,1.573)}
\gppoint{gp mark 6}{(6.104,1.794)}
\gppoint{gp mark 6}{(6.228,2.126)}
\gppoint{gp mark 6}{(6.353,2.656)}
\gppoint{gp mark 6}{(6.477,3.386)}
\gppoint{gp mark 6}{(6.601,4.293)}
\gppoint{gp mark 6}{(6.726,5.310)}
\gppoint{gp mark 6}{(6.850,6.217)}
\gppoint{gp mark 6}{(6.974,6.924)}
\gppoint{gp mark 6}{(7.098,7.411)}
\gppoint{gp mark 6}{(7.223,7.632)}
\gppoint{gp mark 6}{(7.347,7.942)}
\gppoint{gp mark 6}{(7.471,8.052)}
\gppoint{gp mark 6}{(7.595,8.119)}
\gppoint{gp mark 6}{(7.720,8.119)}
\gppoint{gp mark 6}{(7.844,8.096)}
\gppoint{gp mark 6}{(7.968,8.030)}
\gppoint{gp mark 6}{(8.093,7.964)}
\gppoint{gp mark 6}{(8.217,7.853)}
\gppoint{gp mark 6}{(8.341,7.743)}
\gppoint{gp mark 6}{(8.465,7.610)}
\gppoint{gp mark 6}{(8.590,7.477)}
\gppoint{gp mark 6}{(8.714,7.322)}
\gppoint{gp mark 6}{(8.838,7.146)}
\gppoint{gp mark 6}{(8.963,6.991)}
\gppoint{gp mark 6}{(9.087,6.814)}
\gppoint{gp mark 6}{(9.211,6.637)}
\gppoint{gp mark 6}{(9.335,6.438)}
\gppoint{gp mark 6}{(9.460,6.239)}
\gppoint{gp mark 6}{(9.584,6.040)}
\gppoint{gp mark 6}{(9.708,5.841)}
\gppoint{gp mark 6}{(9.833,5.642)}
\gppoint{gp mark 6}{(9.957,5.443)}
\gppoint{gp mark 6}{(10.081,5.222)}
\gppoint{gp mark 6}{(10.205,5.023)}
\gppoint{gp mark 6}{(10.330,4.824)}
\gppoint{gp mark 6}{(10.454,4.602)}
\gpcolor{color=gp lt color border}
\draw[gp path] (2.997,8.441)--(2.997,0.985)--(10.454,0.985)--(10.454,8.441)--cycle;
%% coordinates of the plot area
\gpdefrectangularnode{gp plot 1}{\pgfpoint{2.997cm}{0.985cm}}{\pgfpoint{10.454cm}{8.441cm}}
\end{tikzpicture}
%% gnuplot variables

    \caption{Ni円盤の磁気トルク曲線}
  \end{center}
\end{figure}

\begin{figure}[hptb]
  \begin{center}
    \begin{tikzpicture}[gnuplot]
%% generated with GNUPLOT 5.2p8 (Lua 5.3; terminal rev. Nov 2018, script rev. 108)
%% 2021年04月23日 17時07分28秒
\path (0.000,0.000) rectangle (12.500,8.750);
\gpcolor{color=gp lt color border}
\gpsetlinetype{gp lt border}
\gpsetdashtype{gp dt solid}
\gpsetlinewidth{1.00}
\draw[gp path] (2.997,0.985)--(3.177,0.985);
\draw[gp path] (10.454,0.985)--(10.274,0.985);
\node[gp node right] at (2.813,0.985) {$-0.6$};
\draw[gp path] (2.997,1.606)--(3.087,1.606);
\draw[gp path] (10.454,1.606)--(10.364,1.606);
\draw[gp path] (2.997,2.228)--(3.177,2.228);
\draw[gp path] (10.454,2.228)--(10.274,2.228);
\node[gp node right] at (2.813,2.228) {$-0.4$};
\draw[gp path] (2.997,2.849)--(3.087,2.849);
\draw[gp path] (10.454,2.849)--(10.364,2.849);
\draw[gp path] (2.997,3.470)--(3.177,3.470);
\draw[gp path] (10.454,3.470)--(10.274,3.470);
\node[gp node right] at (2.813,3.470) {$-0.2$};
\draw[gp path] (2.997,4.092)--(3.087,4.092);
\draw[gp path] (10.454,4.092)--(10.364,4.092);
\draw[gp path] (2.997,4.713)--(3.177,4.713);
\draw[gp path] (10.454,4.713)--(10.274,4.713);
\node[gp node right] at (2.813,4.713) {$0$};
\draw[gp path] (2.997,5.334)--(3.087,5.334);
\draw[gp path] (10.454,5.334)--(10.364,5.334);
\draw[gp path] (2.997,5.956)--(3.177,5.956);
\draw[gp path] (10.454,5.956)--(10.274,5.956);
\node[gp node right] at (2.813,5.956) {$0.2$};
\draw[gp path] (2.997,6.577)--(3.087,6.577);
\draw[gp path] (10.454,6.577)--(10.364,6.577);
\draw[gp path] (2.997,7.198)--(3.177,7.198);
\draw[gp path] (10.454,7.198)--(10.274,7.198);
\node[gp node right] at (2.813,7.198) {$0.4$};
\draw[gp path] (2.997,7.820)--(3.087,7.820);
\draw[gp path] (10.454,7.820)--(10.364,7.820);
\draw[gp path] (2.997,8.441)--(3.177,8.441);
\draw[gp path] (10.454,8.441)--(10.274,8.441);
\node[gp node right] at (2.813,8.441) {$0.6$};
\draw[gp path] (4.240,0.985)--(4.240,1.165);
\draw[gp path] (4.240,8.441)--(4.240,8.261);
\node[gp node center] at (4.240,0.677) {$-2.0$};
\draw[gp path] (5.483,0.985)--(5.483,1.075);
\draw[gp path] (5.483,8.441)--(5.483,8.351);
\draw[gp path] (6.726,0.985)--(6.726,1.165);
\draw[gp path] (6.726,8.441)--(6.726,8.261);
\node[gp node center] at (6.726,0.677) {$0.0$};
\draw[gp path] (7.968,0.985)--(7.968,1.075);
\draw[gp path] (7.968,8.441)--(7.968,8.351);
\draw[gp path] (9.211,0.985)--(9.211,1.165);
\draw[gp path] (9.211,8.441)--(9.211,8.261);
\node[gp node center] at (9.211,0.677) {$2.0$};
\draw[gp path] (2.997,8.441)--(2.997,0.985)--(10.454,0.985)--(10.454,8.441)--cycle;
\node[gp node center,rotate=-270] at (1.785,4.713) {磁化$J$ / $\si{Wb.m^{-2}}$};
\node[gp node center] at (6.725,0.215) {磁場$H$ / $\times 10^{5}\ \si{A.m^{-1}}$};
\draw[gp path] (3.181,6.875)--(3.181,8.261)--(6.489,8.261)--(6.489,6.875)--cycle;
\node[gp node right] at (5.205,7.799) {面直磁場};
\gpcolor{rgb color={0.000,0.000,0.000}}
\gpsetpointsize{4.00}
\gppoint{gp mark 6}{(6.731,4.661)}
\gppoint{gp mark 6}{(6.799,4.966)}
\gppoint{gp mark 6}{(6.865,5.260)}
\gppoint{gp mark 6}{(6.933,5.550)}
\gppoint{gp mark 6}{(7.005,5.836)}
\gppoint{gp mark 6}{(7.075,6.125)}
\gppoint{gp mark 6}{(7.159,6.393)}
\gppoint{gp mark 6}{(7.239,6.660)}
\gppoint{gp mark 6}{(7.339,6.918)}
\gppoint{gp mark 6}{(7.444,7.155)}
\gppoint{gp mark 6}{(7.564,7.382)}
\gppoint{gp mark 6}{(7.448,7.196)}
\gppoint{gp mark 6}{(7.332,6.980)}
\gppoint{gp mark 6}{(7.242,6.743)}
\gppoint{gp mark 6}{(7.152,6.485)}
\gppoint{gp mark 6}{(7.062,6.217)}
\gppoint{gp mark 6}{(6.992,5.939)}
\gppoint{gp mark 6}{(6.914,5.655)}
\gppoint{gp mark 6}{(6.842,5.368)}
\gppoint{gp mark 6}{(6.775,5.077)}
\gppoint{gp mark 6}{(6.704,4.758)}
\gppoint{gp mark 6}{(6.638,4.457)}
\gppoint{gp mark 6}{(6.575,4.172)}
\gppoint{gp mark 6}{(6.504,3.880)}
\gppoint{gp mark 6}{(6.437,3.580)}
\gppoint{gp mark 6}{(6.362,3.312)}
\gppoint{gp mark 6}{(6.287,3.023)}
\gppoint{gp mark 6}{(6.197,2.766)}
\gppoint{gp mark 6}{(6.102,2.508)}
\gppoint{gp mark 6}{(5.997,2.271)}
\gppoint{gp mark 6}{(5.882,2.044)}
\gppoint{gp mark 6}{(5.988,2.230)}
\gppoint{gp mark 6}{(6.104,2.446)}
\gppoint{gp mark 6}{(6.199,2.693)}
\gppoint{gp mark 6}{(6.289,2.941)}
\gppoint{gp mark 6}{(6.374,3.209)}
\gppoint{gp mark 6}{(6.454,3.487)}
\gppoint{gp mark 6}{(6.529,3.774)}
\gppoint{gp mark 6}{(6.596,4.056)}
\gppoint{gp mark 6}{(6.665,4.349)}
\gppoint{gp mark 6}{(6.735,4.668)}
\gppoint{gp mark 6}{(6.800,4.962)}
\gppoint{gp mark 6}{(6.864,5.256)}
\gppoint{gp mark 6}{(6.932,5.551)}
\gppoint{gp mark 6}{(7.005,5.836)}
\gppoint{gp mark 6}{(7.075,6.125)}
\gppoint{gp mark 6}{(7.154,6.403)}
\gppoint{gp mark 6}{(7.239,6.671)}
\gppoint{gp mark 6}{(7.334,6.918)}
\gppoint{gp mark 6}{(7.444,7.155)}
\gppoint{gp mark 6}{(7.570,7.371)}
\gppoint{gp mark 6}{(5.847,7.799)}
\gpcolor{color=gp lt color border}
\node[gp node right] at (5.205,7.337) {面内磁場};
\gpcolor{rgb color={0.000,0.000,0.000}}
\gppoint{gp mark 7}{(6.726,4.713)}
\gppoint{gp mark 7}{(6.577,6.300)}
\gppoint{gp mark 7}{(6.588,7.361)}
\gppoint{gp mark 7}{(6.779,7.876)}
\gppoint{gp mark 7}{(7.067,8.062)}
\gppoint{gp mark 7}{(7.395,8.134)}
\gppoint{gp mark 7}{(7.736,8.165)}
\gppoint{gp mark 7}{(8.070,8.185)}
\gppoint{gp mark 7}{(8.409,8.206)}
\gppoint{gp mark 7}{(8.740,8.206)}
\gppoint{gp mark 7}{(9.071,8.206)}
\gppoint{gp mark 7}{(8.779,8.206)}
\gppoint{gp mark 7}{(8.471,8.196)}
\gppoint{gp mark 7}{(8.138,8.185)}
\gppoint{gp mark 7}{(7.804,8.165)}
\gppoint{gp mark 7}{(7.468,8.134)}
\gppoint{gp mark 7}{(7.137,8.072)}
\gppoint{gp mark 7}{(6.836,7.928)}
\gppoint{gp mark 7}{(6.607,7.536)}
\gppoint{gp mark 7}{(6.535,6.630)}
\gppoint{gp mark 7}{(6.671,4.970)}
\gppoint{gp mark 7}{(6.833,3.312)}
\gppoint{gp mark 7}{(6.852,2.116)}
\gppoint{gp mark 7}{(6.669,1.560)}
\gppoint{gp mark 7}{(6.381,1.374)}
\gppoint{gp mark 7}{(6.053,1.302)}
\gppoint{gp mark 7}{(5.712,1.271)}
\gppoint{gp mark 7}{(5.378,1.251)}
\gppoint{gp mark 7}{(5.038,1.230)}
\gppoint{gp mark 7}{(4.698,1.230)}
\gppoint{gp mark 7}{(4.362,1.230)}
\gppoint{gp mark 7}{(4.651,1.241)}
\gppoint{gp mark 7}{(4.967,1.241)}
\gppoint{gp mark 7}{(5.300,1.251)}
\gppoint{gp mark 7}{(5.629,1.271)}
\gppoint{gp mark 7}{(5.965,1.302)}
\gppoint{gp mark 7}{(6.291,1.364)}
\gppoint{gp mark 7}{(6.597,1.508)}
\gppoint{gp mark 7}{(6.831,1.900)}
\gppoint{gp mark 7}{(6.901,2.827)}
\gppoint{gp mark 7}{(6.765,4.467)}
\gppoint{gp mark 7}{(6.605,6.125)}
\gppoint{gp mark 7}{(6.589,7.310)}
\gppoint{gp mark 7}{(6.771,7.856)}
\gppoint{gp mark 7}{(7.064,8.041)}
\gppoint{gp mark 7}{(7.397,8.113)}
\gppoint{gp mark 7}{(7.737,8.144)}
\gppoint{gp mark 7}{(8.067,8.165)}
\gppoint{gp mark 7}{(8.419,8.175)}
\gppoint{gp mark 7}{(8.747,8.185)}
\gppoint{gp mark 7}{(9.079,8.196)}
\gppoint{gp mark 7}{(5.847,7.337)}
\gpcolor{color=gp lt color border}
\draw[gp path] (2.997,8.441)--(2.997,0.985)--(10.454,0.985)--(10.454,8.441)--cycle;
%% coordinates of the plot area
\gpdefrectangularnode{gp plot 1}{\pgfpoint{2.997cm}{0.985cm}}{\pgfpoint{10.454cm}{8.441cm}}
\end{tikzpicture}
%% gnuplot variables

    \caption{形状から算出した反磁場係数による反磁場補正}
  \end{center}
\end{figure}

\begin{figure}[hptb]
  \begin{center}
    \begin{tikzpicture}[gnuplot]
%% generated with GNUPLOT 5.2p8 (Lua 5.3; terminal rev. Nov 2018, script rev. 108)
%% 2021年04月23日 17時08分14秒
\path (0.000,0.000) rectangle (12.500,8.750);
\gpcolor{color=gp lt color border}
\gpsetlinetype{gp lt border}
\gpsetdashtype{gp dt solid}
\gpsetlinewidth{1.00}
\draw[gp path] (2.997,0.985)--(3.177,0.985);
\draw[gp path] (10.454,0.985)--(10.274,0.985);
\node[gp node right] at (2.813,0.985) {$-0.6$};
\draw[gp path] (2.997,1.606)--(3.087,1.606);
\draw[gp path] (10.454,1.606)--(10.364,1.606);
\draw[gp path] (2.997,2.228)--(3.177,2.228);
\draw[gp path] (10.454,2.228)--(10.274,2.228);
\node[gp node right] at (2.813,2.228) {$-0.4$};
\draw[gp path] (2.997,2.849)--(3.087,2.849);
\draw[gp path] (10.454,2.849)--(10.364,2.849);
\draw[gp path] (2.997,3.470)--(3.177,3.470);
\draw[gp path] (10.454,3.470)--(10.274,3.470);
\node[gp node right] at (2.813,3.470) {$-0.2$};
\draw[gp path] (2.997,4.092)--(3.087,4.092);
\draw[gp path] (10.454,4.092)--(10.364,4.092);
\draw[gp path] (2.997,4.713)--(3.177,4.713);
\draw[gp path] (10.454,4.713)--(10.274,4.713);
\node[gp node right] at (2.813,4.713) {$0$};
\draw[gp path] (2.997,5.334)--(3.087,5.334);
\draw[gp path] (10.454,5.334)--(10.364,5.334);
\draw[gp path] (2.997,5.956)--(3.177,5.956);
\draw[gp path] (10.454,5.956)--(10.274,5.956);
\node[gp node right] at (2.813,5.956) {$0.2$};
\draw[gp path] (2.997,6.577)--(3.087,6.577);
\draw[gp path] (10.454,6.577)--(10.364,6.577);
\draw[gp path] (2.997,7.198)--(3.177,7.198);
\draw[gp path] (10.454,7.198)--(10.274,7.198);
\node[gp node right] at (2.813,7.198) {$0.4$};
\draw[gp path] (2.997,7.820)--(3.087,7.820);
\draw[gp path] (10.454,7.820)--(10.364,7.820);
\draw[gp path] (2.997,8.441)--(3.177,8.441);
\draw[gp path] (10.454,8.441)--(10.274,8.441);
\node[gp node right] at (2.813,8.441) {$0.6$};
\draw[gp path] (4.240,0.985)--(4.240,1.165);
\draw[gp path] (4.240,8.441)--(4.240,8.261);
\node[gp node center] at (4.240,0.677) {$-2.0$};
\draw[gp path] (5.483,0.985)--(5.483,1.075);
\draw[gp path] (5.483,8.441)--(5.483,8.351);
\draw[gp path] (6.726,0.985)--(6.726,1.165);
\draw[gp path] (6.726,8.441)--(6.726,8.261);
\node[gp node center] at (6.726,0.677) {$0.0$};
\draw[gp path] (7.968,0.985)--(7.968,1.075);
\draw[gp path] (7.968,8.441)--(7.968,8.351);
\draw[gp path] (9.211,0.985)--(9.211,1.165);
\draw[gp path] (9.211,8.441)--(9.211,8.261);
\node[gp node center] at (9.211,0.677) {$2.0$};
\draw[gp path] (2.997,8.441)--(2.997,0.985)--(10.454,0.985)--(10.454,8.441)--cycle;
\node[gp node center,rotate=-270] at (1.785,4.713) {磁化$J$ / $\si{Wb.m^{-2}}$};
\node[gp node center] at (6.725,0.215) {磁場$H$ / $\times 10^{5}\ \si{A.m^{-1}}$};
\draw[gp path] (3.181,6.875)--(3.181,8.261)--(6.489,8.261)--(6.489,6.875)--cycle;
\node[gp node right] at (5.205,7.799) {面直磁場};
\gpcolor{rgb color={0.000,0.000,0.000}}
\gpsetpointsize{4.00}
\gppoint{gp mark 6}{(6.741,4.661)}
\gppoint{gp mark 6}{(6.748,4.966)}
\gppoint{gp mark 6}{(6.755,5.260)}
\gppoint{gp mark 6}{(6.766,5.550)}
\gppoint{gp mark 6}{(6.781,5.836)}
\gppoint{gp mark 6}{(6.793,6.125)}
\gppoint{gp mark 6}{(6.824,6.393)}
\gppoint{gp mark 6}{(6.851,6.660)}
\gppoint{gp mark 6}{(6.899,6.918)}
\gppoint{gp mark 6}{(6.957,7.155)}
\gppoint{gp mark 6}{(7.032,7.382)}
\gppoint{gp mark 6}{(6.952,7.196)}
\gppoint{gp mark 6}{(6.880,6.980)}
\gppoint{gp mark 6}{(6.837,6.743)}
\gppoint{gp mark 6}{(6.798,6.485)}
\gppoint{gp mark 6}{(6.762,6.217)}
\gppoint{gp mark 6}{(6.748,5.939)}
\gppoint{gp mark 6}{(6.726,5.655)}
\gppoint{gp mark 6}{(6.711,5.368)}
\gppoint{gp mark 6}{(6.702,5.077)}
\gppoint{gp mark 6}{(6.695,4.758)}
\gppoint{gp mark 6}{(6.689,4.457)}
\gppoint{gp mark 6}{(6.683,4.172)}
\gppoint{gp mark 6}{(6.670,3.880)}
\gppoint{gp mark 6}{(6.663,3.580)}
\gppoint{gp mark 6}{(6.641,3.312)}
\gppoint{gp mark 6}{(6.624,3.023)}
\gppoint{gp mark 6}{(6.586,2.766)}
\gppoint{gp mark 6}{(6.542,2.508)}
\gppoint{gp mark 6}{(6.484,2.271)}
\gppoint{gp mark 6}{(6.414,2.044)}
\gppoint{gp mark 6}{(6.484,2.230)}
\gppoint{gp mark 6}{(6.556,2.446)}
\gppoint{gp mark 6}{(6.602,2.693)}
\gppoint{gp mark 6}{(6.643,2.941)}
\gppoint{gp mark 6}{(6.674,3.209)}
\gppoint{gp mark 6}{(6.699,3.487)}
\gppoint{gp mark 6}{(6.717,3.774)}
\gppoint{gp mark 6}{(6.727,4.056)}
\gppoint{gp mark 6}{(6.737,4.349)}
\gppoint{gp mark 6}{(6.744,4.668)}
\gppoint{gp mark 6}{(6.750,4.962)}
\gppoint{gp mark 6}{(6.755,5.256)}
\gppoint{gp mark 6}{(6.765,5.551)}
\gppoint{gp mark 6}{(6.781,5.836)}
\gppoint{gp mark 6}{(6.793,6.125)}
\gppoint{gp mark 6}{(6.817,6.403)}
\gppoint{gp mark 6}{(6.848,6.671)}
\gppoint{gp mark 6}{(6.894,6.918)}
\gppoint{gp mark 6}{(6.957,7.155)}
\gppoint{gp mark 6}{(7.039,7.371)}
\gppoint{gp mark 6}{(5.847,7.799)}
\gpcolor{color=gp lt color border}
\node[gp node right] at (5.205,7.337) {面内磁場};
\gpcolor{rgb color={0.000,0.000,0.000}}
\gppoint{gp mark 7}{(6.726,4.713)}
\gppoint{gp mark 7}{(6.736,6.300)}
\gppoint{gp mark 7}{(6.852,7.361)}
\gppoint{gp mark 7}{(7.095,7.876)}
\gppoint{gp mark 7}{(7.401,8.062)}
\gppoint{gp mark 7}{(7.737,8.134)}
\gppoint{gp mark 7}{(8.080,8.165)}
\gppoint{gp mark 7}{(8.417,8.185)}
\gppoint{gp mark 7}{(8.758,8.206)}
\gppoint{gp mark 7}{(9.089,8.206)}
\gppoint{gp mark 7}{(9.420,8.206)}
\gppoint{gp mark 7}{(9.128,8.206)}
\gppoint{gp mark 7}{(8.818,8.196)}
\gppoint{gp mark 7}{(8.485,8.185)}
\gppoint{gp mark 7}{(8.149,8.165)}
\gppoint{gp mark 7}{(7.809,8.134)}
\gppoint{gp mark 7}{(7.472,8.072)}
\gppoint{gp mark 7}{(7.157,7.928)}
\gppoint{gp mark 7}{(6.888,7.536)}
\gppoint{gp mark 7}{(6.727,6.630)}
\gppoint{gp mark 7}{(6.697,4.970)}
\gppoint{gp mark 7}{(6.693,3.312)}
\gppoint{gp mark 7}{(6.593,2.116)}
\gppoint{gp mark 7}{(6.354,1.560)}
\gppoint{gp mark 7}{(6.048,1.374)}
\gppoint{gp mark 7}{(5.712,1.302)}
\gppoint{gp mark 7}{(5.368,1.271)}
\gppoint{gp mark 7}{(5.032,1.251)}
\gppoint{gp mark 7}{(4.691,1.230)}
\gppoint{gp mark 7}{(4.350,1.230)}
\gppoint{gp mark 7}{(4.015,1.230)}
\gppoint{gp mark 7}{(4.304,1.241)}
\gppoint{gp mark 7}{(4.621,1.241)}
\gppoint{gp mark 7}{(4.954,1.251)}
\gppoint{gp mark 7}{(5.286,1.271)}
\gppoint{gp mark 7}{(5.625,1.302)}
\gppoint{gp mark 7}{(5.957,1.364)}
\gppoint{gp mark 7}{(6.277,1.508)}
\gppoint{gp mark 7}{(6.551,1.900)}
\gppoint{gp mark 7}{(6.713,2.827)}
\gppoint{gp mark 7}{(6.741,4.467)}
\gppoint{gp mark 7}{(6.746,6.125)}
\gppoint{gp mark 7}{(6.848,7.310)}
\gppoint{gp mark 7}{(7.084,7.856)}
\gppoint{gp mark 7}{(7.396,8.041)}
\gppoint{gp mark 7}{(7.736,8.113)}
\gppoint{gp mark 7}{(8.080,8.144)}
\gppoint{gp mark 7}{(8.411,8.165)}
\gppoint{gp mark 7}{(8.764,8.175)}
\gppoint{gp mark 7}{(9.093,8.185)}
\gppoint{gp mark 7}{(9.427,8.196)}
\gppoint{gp mark 7}{(5.847,7.337)}
\gpcolor{color=gp lt color border}
\draw[gp path] (2.997,8.441)--(2.997,0.985)--(10.454,0.985)--(10.454,8.441)--cycle;
%% coordinates of the plot area
\gpdefrectangularnode{gp plot 1}{\pgfpoint{2.997cm}{0.985cm}}{\pgfpoint{10.454cm}{8.441cm}}
\end{tikzpicture}
%% gnuplot variables

    \caption{トルク曲線から算出した反磁場係数による反磁場補正}
  \end{center}
\end{figure}

\begin{figure}[hptb]
  \begin{center}
    \begin{tikzpicture}[gnuplot]
%% generated with GNUPLOT 5.2p8 (Lua 5.3; terminal rev. Nov 2018, script rev. 108)
%% 2021年04月27日 14時42分33秒
\path (0.000,0.000) rectangle (12.500,8.750);
\gpcolor{color=gp lt color border}
\gpsetlinetype{gp lt border}
\gpsetdashtype{gp dt solid}
\gpsetlinewidth{1.00}
\draw[gp path] (2.997,0.985)--(3.177,0.985);
\draw[gp path] (10.454,0.985)--(10.274,0.985);
\node[gp node right] at (2.813,0.985) {$-0.6$};
\draw[gp path] (2.997,1.606)--(3.087,1.606);
\draw[gp path] (10.454,1.606)--(10.364,1.606);
\draw[gp path] (2.997,2.228)--(3.177,2.228);
\draw[gp path] (10.454,2.228)--(10.274,2.228);
\node[gp node right] at (2.813,2.228) {$-0.4$};
\draw[gp path] (2.997,2.849)--(3.087,2.849);
\draw[gp path] (10.454,2.849)--(10.364,2.849);
\draw[gp path] (2.997,3.470)--(3.177,3.470);
\draw[gp path] (10.454,3.470)--(10.274,3.470);
\node[gp node right] at (2.813,3.470) {$-0.2$};
\draw[gp path] (2.997,4.092)--(3.087,4.092);
\draw[gp path] (10.454,4.092)--(10.364,4.092);
\draw[gp path] (2.997,4.713)--(3.177,4.713);
\draw[gp path] (10.454,4.713)--(10.274,4.713);
\node[gp node right] at (2.813,4.713) {$0$};
\draw[gp path] (2.997,5.334)--(3.087,5.334);
\draw[gp path] (10.454,5.334)--(10.364,5.334);
\draw[gp path] (2.997,5.956)--(3.177,5.956);
\draw[gp path] (10.454,5.956)--(10.274,5.956);
\node[gp node right] at (2.813,5.956) {$0.2$};
\draw[gp path] (2.997,6.577)--(3.087,6.577);
\draw[gp path] (10.454,6.577)--(10.364,6.577);
\draw[gp path] (2.997,7.198)--(3.177,7.198);
\draw[gp path] (10.454,7.198)--(10.274,7.198);
\node[gp node right] at (2.813,7.198) {$0.4$};
\draw[gp path] (2.997,7.820)--(3.087,7.820);
\draw[gp path] (10.454,7.820)--(10.364,7.820);
\draw[gp path] (2.997,8.441)--(3.177,8.441);
\draw[gp path] (10.454,8.441)--(10.274,8.441);
\node[gp node right] at (2.813,8.441) {$0.6$};
\draw[gp path] (4.240,0.985)--(4.240,1.165);
\draw[gp path] (4.240,8.441)--(4.240,8.261);
\node[gp node center] at (4.240,0.677) {$-2.0$};
\draw[gp path] (5.483,0.985)--(5.483,1.075);
\draw[gp path] (5.483,8.441)--(5.483,8.351);
\draw[gp path] (6.726,0.985)--(6.726,1.165);
\draw[gp path] (6.726,8.441)--(6.726,8.261);
\node[gp node center] at (6.726,0.677) {$0.0$};
\draw[gp path] (7.968,0.985)--(7.968,1.075);
\draw[gp path] (7.968,8.441)--(7.968,8.351);
\draw[gp path] (9.211,0.985)--(9.211,1.165);
\draw[gp path] (9.211,8.441)--(9.211,8.261);
\node[gp node center] at (9.211,0.677) {$2.0$};
\gpsetlinetype{gp lt axes}
\gpsetdashtype{gp dt axes}
\gpsetlinewidth{1.50}
\draw[gp path] (2.997,4.713)--(10.454,4.713);
\draw[gp path] (6.726,0.985)--(6.726,8.441);
\gpsetlinetype{gp lt border}
\gpsetdashtype{gp dt solid}
\gpsetlinewidth{1.00}
\draw[gp path] (2.997,8.441)--(2.997,0.985)--(10.454,0.985)--(10.454,8.441)--cycle;
\node[gp node center,rotate=-270] at (1.785,4.713) {磁化$J$ / $\si{Wb.m^{-2}}$};
\node[gp node center] at (6.725,0.215) {磁場$H$ / $\times 10^{5}\ \si{A.m^{-1}}$};
\draw[gp path] (3.181,6.875)--(3.181,8.261)--(6.489,8.261)--(6.489,6.875)--cycle;
\node[gp node right] at (5.205,7.799) {面直磁場};
\gpcolor{rgb color={0.000,0.000,0.000}}
\gpsetpointsize{4.00}
\gppoint{gp mark 6}{(6.742,4.661)}
\gppoint{gp mark 6}{(6.742,4.966)}
\gppoint{gp mark 6}{(6.743,5.260)}
\gppoint{gp mark 6}{(6.747,5.550)}
\gppoint{gp mark 6}{(6.754,5.836)}
\gppoint{gp mark 6}{(6.759,6.125)}
\gppoint{gp mark 6}{(6.784,6.393)}
\gppoint{gp mark 6}{(6.805,6.660)}
\gppoint{gp mark 6}{(6.847,6.918)}
\gppoint{gp mark 6}{(6.899,7.155)}
\gppoint{gp mark 6}{(6.969,7.382)}
\gppoint{gp mark 6}{(6.894,7.196)}
\gppoint{gp mark 6}{(6.826,6.980)}
\gppoint{gp mark 6}{(6.789,6.743)}
\gppoint{gp mark 6}{(6.756,6.485)}
\gppoint{gp mark 6}{(6.726,6.217)}
\gppoint{gp mark 6}{(6.719,5.939)}
\gppoint{gp mark 6}{(6.704,5.655)}
\gppoint{gp mark 6}{(6.696,5.368)}
\gppoint{gp mark 6}{(6.693,5.077)}
\gppoint{gp mark 6}{(6.694,4.758)}
\gppoint{gp mark 6}{(6.695,4.457)}
\gppoint{gp mark 6}{(6.696,4.172)}
\gppoint{gp mark 6}{(6.690,3.880)}
\gppoint{gp mark 6}{(6.690,3.580)}
\gppoint{gp mark 6}{(6.674,3.312)}
\gppoint{gp mark 6}{(6.664,3.023)}
\gppoint{gp mark 6}{(6.632,2.766)}
\gppoint{gp mark 6}{(6.594,2.508)}
\gppoint{gp mark 6}{(6.542,2.271)}
\gppoint{gp mark 6}{(6.478,2.044)}
\gppoint{gp mark 6}{(6.543,2.230)}
\gppoint{gp mark 6}{(6.610,2.446)}
\gppoint{gp mark 6}{(6.650,2.693)}
\gppoint{gp mark 6}{(6.685,2.941)}
\gppoint{gp mark 6}{(6.710,3.209)}
\gppoint{gp mark 6}{(6.728,3.487)}
\gppoint{gp mark 6}{(6.739,3.774)}
\gppoint{gp mark 6}{(6.743,4.056)}
\gppoint{gp mark 6}{(6.746,4.349)}
\gppoint{gp mark 6}{(6.745,4.668)}
\gppoint{gp mark 6}{(6.744,4.962)}
\gppoint{gp mark 6}{(6.743,5.256)}
\gppoint{gp mark 6}{(6.745,5.551)}
\gppoint{gp mark 6}{(6.754,5.836)}
\gppoint{gp mark 6}{(6.759,6.125)}
\gppoint{gp mark 6}{(6.777,6.403)}
\gppoint{gp mark 6}{(6.802,6.671)}
\gppoint{gp mark 6}{(6.842,6.918)}
\gppoint{gp mark 6}{(6.899,7.155)}
\gppoint{gp mark 6}{(6.976,7.371)}
\gppoint{gp mark 6}{(5.847,7.799)}
\gpcolor{color=gp lt color border}
\node[gp node right] at (5.205,7.337) {面内磁場};
\gpcolor{rgb color={0.000,0.000,0.000}}
\gppoint{gp mark 7}{(6.726,4.713)}
\gppoint{gp mark 7}{(6.726,6.300)}
\gppoint{gp mark 7}{(6.835,7.361)}
\gppoint{gp mark 7}{(7.074,7.876)}
\gppoint{gp mark 7}{(7.380,8.062)}
\gppoint{gp mark 7}{(7.715,8.134)}
\gppoint{gp mark 7}{(8.058,8.165)}
\gppoint{gp mark 7}{(8.395,8.185)}
\gppoint{gp mark 7}{(8.736,8.206)}
\gppoint{gp mark 7}{(9.067,8.206)}
\gppoint{gp mark 7}{(9.397,8.206)}
\gppoint{gp mark 7}{(9.105,8.206)}
\gppoint{gp mark 7}{(8.796,8.196)}
\gppoint{gp mark 7}{(8.463,8.185)}
\gppoint{gp mark 7}{(8.127,8.165)}
\gppoint{gp mark 7}{(7.788,8.134)}
\gppoint{gp mark 7}{(7.451,8.072)}
\gppoint{gp mark 7}{(7.136,7.928)}
\gppoint{gp mark 7}{(6.871,7.536)}
\gppoint{gp mark 7}{(6.714,6.630)}
\gppoint{gp mark 7}{(6.695,4.970)}
\gppoint{gp mark 7}{(6.702,3.312)}
\gppoint{gp mark 7}{(6.609,2.116)}
\gppoint{gp mark 7}{(6.374,1.560)}
\gppoint{gp mark 7}{(6.069,1.374)}
\gppoint{gp mark 7}{(5.734,1.302)}
\gppoint{gp mark 7}{(5.390,1.271)}
\gppoint{gp mark 7}{(5.054,1.251)}
\gppoint{gp mark 7}{(4.713,1.230)}
\gppoint{gp mark 7}{(4.373,1.230)}
\gppoint{gp mark 7}{(4.037,1.230)}
\gppoint{gp mark 7}{(4.327,1.241)}
\gppoint{gp mark 7}{(4.643,1.241)}
\gppoint{gp mark 7}{(4.976,1.251)}
\gppoint{gp mark 7}{(5.308,1.271)}
\gppoint{gp mark 7}{(5.646,1.302)}
\gppoint{gp mark 7}{(5.979,1.364)}
\gppoint{gp mark 7}{(6.298,1.508)}
\gppoint{gp mark 7}{(6.569,1.900)}
\gppoint{gp mark 7}{(6.725,2.827)}
\gppoint{gp mark 7}{(6.742,4.467)}
\gppoint{gp mark 7}{(6.737,6.125)}
\gppoint{gp mark 7}{(6.832,7.310)}
\gppoint{gp mark 7}{(7.064,7.856)}
\gppoint{gp mark 7}{(7.375,8.041)}
\gppoint{gp mark 7}{(7.714,8.113)}
\gppoint{gp mark 7}{(8.058,8.144)}
\gppoint{gp mark 7}{(8.389,8.165)}
\gppoint{gp mark 7}{(8.742,8.175)}
\gppoint{gp mark 7}{(9.071,8.185)}
\gppoint{gp mark 7}{(9.404,8.196)}
\gppoint{gp mark 7}{(5.847,7.337)}
\gpcolor{color=gp lt color border}
\draw[gp path] (2.997,8.441)--(2.997,0.985)--(10.454,0.985)--(10.454,8.441)--cycle;
%% coordinates of the plot area
\gpdefrectangularnode{gp plot 1}{\pgfpoint{2.997cm}{0.985cm}}{\pgfpoint{10.454cm}{8.441cm}}
\end{tikzpicture}
%% gnuplot variables

    \caption{磁気ヒステリシス曲線の傾きから算出した反磁場係数による反磁場補正}
  \end{center}
\end{figure}