\subsection{キュリー温度と分子磁場の関係式}
キュリー温度と分子磁場の関係式である(\ref{equ:bunshijiba-kyuriondo})を導出する.
\begin{align}
  \label{equ:bunshijiba-kyuriondo}
  \Theta=\frac{2zS(S+1)}{3k}J_e
\end{align}
\subsubsection{常磁性の場合}
スピンによる磁気モーメント${\bm M}$が${\bm H}$中に存在するときそのエネルギー$U$は
\begin{align}
  \label{equ:U}
  U=-\bm{M\cdot H}=-MH\cos\theta
\end{align}
ただし図\ref{fig:fig/def_MH.jpg}のように原点から単位球面表面に向かって引いたベクトルを用いて角度分布を表す.
このときスピンが熱雑音に抗って磁場方向を向く確率は以下のボルツマン係数に比例することが知られている\cite{alma990008542450204034}.
\begin{align}
  \exp\left(-\frac{U}{kT}\right)=\exp\left(\frac{MH}{kT}\cos\theta\right)
\end{align}
図\ref{fig:fig/def_MH.jpg}の網掛け部分の表面積は$2\pi\sin\theta{\rm d}\theta$なので,スピンが$\theta$と$\theta+{\rm d}\theta$方向を向く確率$p(\theta){\rm d}\theta$は.
\begin{align}
  p(\theta){\rm d}\theta=\frac{\displaystyle \exp\left(\frac{MH}{kT}\cos\theta\right)\sin\theta{\rm d}\theta}{\displaystyle \int_0^\pi\exp\left(\frac{MH}{kT}\cos\theta\right)\sin\theta{\rm d}\theta}
\end{align}
と表される.ここで単位体積中の磁性原子の数を$N$とすると,スピンによる磁化$\bm{I}$の大きさは$\cos\theta$の期待値$\overline{\cos\theta}$を用いて
\begin{align}
  \label{equ:I_keisan1}
  I&=NM\overline{\cos\theta}
  =NM\int_0^\pi \cos\theta\ p(\theta){\rm d}\theta\nonumber\\
  &=NM\frac{\displaystyle \int_0^\pi\exp\left(\frac{MH}{kT}\cos\theta\right)\cos\theta\sin\theta{\rm d}\theta}{\displaystyle \int_0^\pi\exp\left(\frac{MH}{kT}\cos\theta\right)\sin\theta{\rm d}\theta}\nonumber\\
  &=NM\frac{\displaystyle \int_{-1}^1{\rm e}^{\alpha x}x{\rm d}x}{\displaystyle \int_{-1}^{1}{\rm e}^{\alpha x}{\rm d}x}
\end{align}
となる,ただし$\alpha=MH/kT$, $x=\cos\theta$とした.ここで(\ref{equ:I_keisan1})の分母を計算する.
\begin{align}
  \label{equ:I_keisan2}
  \int_{-1}^1{\rm e}^{\alpha x}{\rm d}x=\frac{1}{\alpha}\left[{\rm e}^{\alpha x}\right]_{-1}^1=\frac{1}{\alpha}\left({\rm e}^\alpha-{\rm e}^{-\alpha}\right)
\end{align}
また(\ref{equ:I_keisan2})の両辺を$\alpha$で微分すれば分子を計算できる.
\begin{align}
  \int_{-1}^1{\rm e}^{\alpha x}x{\rm d}x=\frac{1}{\alpha}\left({\rm e}^\alpha+{\rm e}^{-\alpha}\right)-\frac{1}{\alpha^2}\left({\rm e}^\alpha-{\rm e}^{-\alpha}\right)
\end{align}
以上から(\ref{equ:I_keisan1})は
\begin{align}
  I&=NM\left(\frac{{\rm e}^\alpha+{\rm e}^{-\alpha}}{{\rm e}^\alpha-{\rm e}^{-\alpha}}-\frac{1}{\alpha}\right)\nonumber\\
  &=NM\left(\coth\alpha-\frac{1}{\alpha}\right)=:NML(\alpha)
\end{align}
ここで$L(\alpha)$はランジュバン関数と呼ばれる.
%ランジュバン関数は$\alpha\ll 1$のとき(\ref{equ:L_tenkai})のように展開される.
%\begin{align}
%  \label{equ:L_tenkai}
%  L(\alpha)=\frac{\alpha}{3}-\frac{\alpha^3}{45}-\cdots
%\end{align}
%ここで第1項のみを拾うと以下が成り立つ.
%\begin{align}
%  \label{equ:Curie_law}
%  I&=\frac{NM}{3}\alpha=\frac{NM^2}{3kT}H\nonumber\\
%  \therefore \chi&=\frac{I}{H}=\frac{NM^2}{3kT}
%\end{align}
%(\ref{equ:Curie_law})はキュリーの法則と呼ばれる.実験室系では温度$T$の影響は磁場$H$に比べて大きく\cite{alma990008542450204034}, $\alpha\ll 1$と言えるので,この近似は実用的なものである.
\mfig[width=6cm]{fig/def_MH.jpg}{座標の定義}
これまでの計算ではスピンは空間のどの方向にも向きうる古典的なモデルを用いたが,実際にはスピンの向く方向は量子化されている.
$H$の方向を$z$方向に取ると, $M$の$H$成分は
\begin{align}
  M_z=g\mu_BM_J
\end{align}
と表される.ここで$M_J=-J,\ -(J-1),\ ...,\ J$は磁気量子数, $g$は$g$因子, $\mu_B$はボーア磁子である.
このとき$H$方向の磁化の期待値は
\begin{align}
  I&=Ng\mu_B\frac{\sum_{M_J}\exp\left(\displaystyle\frac{g\mu_B}{kT}HM_J\right)}{\sum_{M_J}\exp\left(\displaystyle\frac{g\mu_B}{kT}HM_J\right)}\nonumber\\
  &=Ng\mu_B\frac{\sum_{b}b{\rm e}^{by}}{\sum_{b}{\rm e}^{by}}
\end{align}
ここで$y=g\mu_BH/kT$, $b=M_J$とした.また分母は等比級数の和を用いて
\begin{align}
  \sum_{b}{\rm e}^{by}&={\rm e}^{-Jy}+{\rm e}^{-Jy}{\rm e}^{y}+{\rm e}^{-Jy}{\rm e}^{2y}+\cdots+{\rm e}^{Jy}\nonumber\\
  &=\frac{{\rm e}^{-Jy}-{\rm e}^{Jy}{\rm e}^y}{1-{\rm e}^y}\nonumber\\
  &=\frac{{\rm e}^{(J+\frac{1}{2})y}-{\rm e}^{-(J+\frac{1}{2})y}}{{\rm e}^{\frac{y}{2}}-{\rm e}^{-\frac{y}{2}}}\nonumber\\
  &=\frac{\sinh\left(J+\frac{1}{2}\right)y}{\sinh\frac{y}{2}}
\end{align}
となる.また
\begin{align}
  \frac{\rm d}{{\rm d}y}\left(\ln\sum_b{\rm e}^{by}\right)=\frac{\sum_bb{\rm e}^{by}}{\sum_b{\rm e}^{by}}
\end{align}
の関係を用いると以下のように表される.
\begin{align}
  \frac{\sum_bb{\rm e}^{by}}{\sum_b{\rm e}^{by}}&=\frac{\rm d}{{\rm d}y}\left(\ln\frac{\sinh\left(J+\frac{1}{2}\right)y}{\sinh\frac{y}{2}}\right)\nonumber\\
  &=\frac{\rm d}{{\rm d}y}\left(\ln\sinh\left(J+\frac{1}{2}\right)y-\ln\sinh\frac{y}{2}\right)\nonumber\\
  &=\left(J+\frac{1}{2}\right)\frac{\cosh\left(J+\frac{1}{2}\right)y}{\sinh\left(J+\frac{1}{2}\right)y}-\frac{1}{2}\frac{\cosh\frac{y}{2}}{\sinh\frac{y}{2}}\nonumber\\
  &=\left(\frac{2J+1}{2}\right)\coth\left(\frac{2J+1}{2J}\right)a-\frac{1}{2}\coth\frac{a}{2J}
\end{align}
最後に$a=g\mu_BJH/kT$とした.したがって$I$は
\begin{align}
  \label{equ:Brillouin}
  I&=Ng\mu_BJ\left(\left(\frac{2J+1}{2J}\right)\coth\left(\frac{2J+1}{2J}\right)a-\frac{1}{2J}\coth\frac{a}{2J}\right)\nonumber\\
  &=:Ng\mu_BJB_J(a)
\end{align}
ここで$B_J(a)$はブリュアン関数と呼ばれる.
\subsubsection{強磁性の場合}
強磁性体は原子の磁気モーメントがある方向に揃っている磁性体であり,
ワイスはその原因が分子内に存在する磁場である分子磁場$\bm{H}_m$であると仮定した.
さらに分子磁場の大きさ$H_m$はその磁性体の磁化の大きさ$I$に比例すると考え,以下の式を仮定した.
\begin{align}
  H_m&=\omega I
\end{align}
常磁性での磁化は(\ref{equ:Brillouin})で与えられたが,強磁性においては$H\rightarrow H+\omega I$であると考え
\begin{align}
  \label{equ:I_strong}
  I=Ng\mu_BJB_J(x)
\end{align}
とする.ここで$x=g\mu_BJ(H+\omega I)/kT$である.

自発磁化は外部磁場が$0$のときに発生している磁化であるので(\ref{equ:I_strong})において$H=0$とすると
\begin{align}
  I=\frac{kTx}{g\mu_BJ\omega}
\end{align}
さらに$I_0=Ng\mu_BJ$と置くと
\begin{align}
  \frac{I}{I_0}=\frac{kTx}{Ng^2\mu_B^2J^2\omega}=B_J(x)
\end{align}
となるので,中辺と最右辺の交点が$I/I_0$を与える.
特に図\ref{equ:Brillouin}に示すようにある温度$T=\Theta$で$B_J(x)$と$kTx/Ng^2\mu_B^2J^2\omega$が接し,それ以上の温度では$B_J(x)=kTx/Ng^2\mu_B^2J^2\omega$は解を持たず自発磁化は消失する.
この温度$\Theta$をキュリー温度と呼ぶ.

図\ref{equ:Brillouin}のように$T=\Theta$において$B_J(x)$と$kTx/Ng^2\mu_B^2J^2\omega$は原点付近で接するため$x\ll 1$とする.
ここで$B_J(x)$をマクローリン展開すると
\begin{align}
  B_J(x)=\frac{J+1}{3J}x+O(x^3)
\end{align}
となる.これが$kTx/Ng^2\mu_B^2J^2\omega$と一致するので
\begin{align}
  \label{equ:Theta}
  \frac{J+1}{3J}x&=\frac{k\Theta x}{Ng^2\mu_B^2J^2\omega}\nonumber\\
  \therefore \Theta&=\frac{Ng^2\mu_B^2J(J+1)\omega}{3k}=\frac{Nm^2}{3k}\omega
\end{align}
ここで$m=g\mu_B\sqrt{J(J+1)}$とした.

また量子力学では分子磁場の起源はスピン間の交換相互作用であると考えられている.交換相互作用によるエネルギー$E_e$は
\begin{align}
  E_e=-2J_e\bm{S}_i\bm{S}_j
\end{align}
で与えられる.ただし$J_e$は交換積分, $\bm{S}_i$は$i$番目の原子が持つ全体のスピンである.今$\bm{S}_i$の周りに$z$個のスピンがあると
仮定し,その平均を$\overline{\bm{S}}_j$とすると
\begin{align}
  E_e=-2J_e\bm{S}_iz\overline{\bm{S}}_j
\end{align}
となる.またスピン$\overline{\bm{S}}_j$による分子磁場は
\begin{align}
  \bm{H}_m=\omega Ng\mu_B\overline{{\bm{S}}}_j
\end{align}
であり,この分子磁場中にスピン$\bm{S}_i$が存在するときのエネルギー$E_m$は
\begin{align}
  E_m&=-(g\mu_B\bm{S}_i)\bm{H}_m\nonumber\\
  &=-\omega Ng^2\mu_B^2\bm{S_i}\overline{\bm{S}}_j
\end{align}
となり,これと$E_e$が等しいはずなので
\begin{align}
  \omega Ng^2\mu_B^2\bm{S_i}\overline{\bm{S}}_j&=-2J_e\bm{S}_i\bm{S}_j\\
  \therefore \omega=\frac{2zJ_e}{Ng^2\mu_B^2}
\end{align}
これを(\ref{equ:Theta})に代入し
\begin{align}
  \Theta=\frac{2zS(S+1)}{3k}J_e
\end{align}
を得る.

\begin{figure}[hptb]
  \begin{center}
    \begin{tikzpicture}[gnuplot]
%% generated with GNUPLOT 5.2p8 (Lua 5.3; terminal rev. Nov 2018, script rev. 108)
%% 2021年04月27日 18時21分18秒
\path (0.000,0.000) rectangle (8.000,8.000);
\gpcolor{color=gp lt color border}
\gpsetlinetype{gp lt border}
\gpsetdashtype{gp dt solid}
\gpsetlinewidth{1.00}
\draw[gp path] (1.320,1.274)--(1.500,1.274);
\draw[gp path] (7.447,1.274)--(7.267,1.274);
\node[gp node right] at (1.136,1.274) {$0$};
\draw[gp path] (1.320,3.631)--(1.500,3.631);
\draw[gp path] (7.447,3.631)--(7.267,3.631);
\node[gp node right] at (1.136,3.631) {$0.5$};
\draw[gp path] (1.320,5.988)--(1.500,5.988);
\draw[gp path] (7.447,5.988)--(7.267,5.988);
\node[gp node right] at (1.136,5.988) {$1$};
\draw[gp path] (1.320,1.274)--(1.320,1.454);
\draw[gp path] (1.320,7.402)--(1.320,7.222);
\node[gp node center] at (1.320,0.966) {$0$};
\draw[gp path] (2.545,1.274)--(2.545,1.454);
\draw[gp path] (2.545,7.402)--(2.545,7.222);
\node[gp node center] at (2.545,0.966) {$1$};
\draw[gp path] (3.771,1.274)--(3.771,1.454);
\draw[gp path] (3.771,7.402)--(3.771,7.222);
\node[gp node center] at (3.771,0.966) {$2$};
\draw[gp path] (4.996,1.274)--(4.996,1.454);
\draw[gp path] (4.996,7.402)--(4.996,7.222);
\node[gp node center] at (4.996,0.966) {$3$};
\draw[gp path] (6.222,1.274)--(6.222,1.454);
\draw[gp path] (6.222,7.402)--(6.222,7.222);
\node[gp node center] at (6.222,0.966) {$4$};
\draw[gp path] (7.447,1.274)--(7.447,1.454);
\draw[gp path] (7.447,7.402)--(7.447,7.222);
\node[gp node center] at (7.447,0.966) {$5$};
\draw[gp path] (1.320,7.402)--(1.320,1.274)--(7.447,1.274)--(7.447,7.402)--cycle;
\node[gp node center,rotate=-270] at (0.292,4.338) {$B_J(x)$};
\node[gp node center] at (4.383,0.504) {$x$};
\draw[gp path] (2.483,1.454)--(2.483,2.840)--(7.263,2.840)--(7.263,1.454)--cycle;
\node[gp node right] at (5.979,2.378) {$B_{J=1}(x)$};
\gpcolor{rgb color={0.000,0.000,0.000}}
\gpsetlinewidth{2.00}
\draw[gp path] (6.163,2.378)--(7.079,2.378);
\draw[gp path] (1.320,1.274)--(1.320,1.275)--(1.321,1.275)--(1.321,1.276)--(1.321,1.277)%
  --(1.321,1.278)--(1.322,1.278)--(1.322,1.279)--(1.322,1.280)--(1.323,1.280)--(1.323,1.281)%
  --(1.323,1.282)--(1.323,1.283)--(1.324,1.283)--(1.324,1.284)--(1.324,1.285)--(1.325,1.286)%
  --(1.325,1.287)--(1.325,1.288)--(1.326,1.288)--(1.326,1.289)--(1.326,1.290)--(1.326,1.291)%
  --(1.327,1.291)--(1.327,1.292)--(1.327,1.293)--(1.328,1.293)--(1.328,1.294)--(1.328,1.295)%
  --(1.328,1.296)--(1.329,1.296)--(1.329,1.297)--(1.329,1.298)--(1.330,1.299)--(1.330,1.300)%
  --(1.330,1.301)--(1.331,1.301)--(1.331,1.302)--(1.331,1.303)--(1.332,1.304)--(1.332,1.305)%
  --(1.332,1.306)--(1.333,1.306)--(1.333,1.307)--(1.333,1.308)--(1.333,1.309)--(1.334,1.309)%
  --(1.334,1.310)--(1.334,1.311)--(1.335,1.311)--(1.335,1.312)--(1.335,1.313)--(1.335,1.314)%
  --(1.336,1.314)--(1.336,1.315)--(1.336,1.316)--(1.337,1.316)--(1.337,1.317)--(1.337,1.318)%
  --(1.337,1.319)--(1.338,1.319)--(1.338,1.320)--(1.338,1.321)--(1.339,1.321)--(1.339,1.322)%
  --(1.339,1.323)--(1.339,1.324)--(1.340,1.324)--(1.340,1.325)--(1.340,1.326)--(1.341,1.327)%
  --(1.341,1.328)--(1.341,1.329)--(1.342,1.329)--(1.342,1.330)--(1.342,1.331)--(1.342,1.332)%
  --(1.343,1.332)--(1.343,1.333)--(1.343,1.334)--(1.344,1.334)--(1.344,1.335)--(1.344,1.336)%
  --(1.344,1.337)--(1.345,1.337)--(1.345,1.338)--(1.345,1.339)--(1.346,1.340)--(1.346,1.341)%
  --(1.346,1.342)--(1.347,1.342)--(1.347,1.343)--(1.347,1.344)--(1.348,1.345)--(1.348,1.346)%
  --(1.348,1.347)--(1.349,1.347)--(1.349,1.348)--(1.349,1.349)--(1.349,1.350)--(1.350,1.350)%
  --(1.350,1.351)--(1.350,1.352)--(1.351,1.352)--(1.351,1.353)--(1.351,1.354)--(1.351,1.355)%
  --(1.352,1.355)--(1.352,1.356)--(1.352,1.357)--(1.353,1.357)--(1.353,1.358)--(1.353,1.359)%
  --(1.353,1.360)--(1.354,1.360)--(1.354,1.361)--(1.354,1.362)--(1.355,1.363)--(1.355,1.364)%
  --(1.355,1.365)--(1.356,1.365)--(1.356,1.366)--(1.356,1.367)--(1.357,1.368)--(1.357,1.369)%
  --(1.357,1.370)--(1.358,1.370)--(1.358,1.371)--(1.358,1.372)--(1.358,1.373)--(1.359,1.373)%
  --(1.359,1.374)--(1.359,1.375)--(1.360,1.375)--(1.360,1.376)--(1.360,1.377)--(1.360,1.378)%
  --(1.361,1.378)--(1.361,1.379)--(1.361,1.380)--(1.362,1.381)--(1.362,1.382)--(1.362,1.383)%
  --(1.363,1.383)--(1.363,1.384)--(1.363,1.385)--(1.364,1.386)--(1.364,1.387)--(1.364,1.388)%
  --(1.365,1.388)--(1.365,1.389)--(1.365,1.390)--(1.365,1.391)--(1.366,1.391)--(1.366,1.392)%
  --(1.366,1.393)--(1.367,1.393)--(1.367,1.394)--(1.367,1.395)--(1.367,1.396)--(1.368,1.396)%
  --(1.368,1.397)--(1.368,1.398)--(1.369,1.398)--(1.369,1.399)--(1.369,1.400)--(1.369,1.401)%
  --(1.370,1.401)--(1.370,1.402)--(1.370,1.403)--(1.371,1.404)--(1.371,1.405)--(1.371,1.406)%
  --(1.372,1.406)--(1.372,1.407)--(1.372,1.408)--(1.373,1.409)--(1.373,1.410)--(1.373,1.411)%
  --(1.374,1.411)--(1.374,1.412)--(1.374,1.413)--(1.374,1.414)--(1.375,1.414)--(1.375,1.415)%
  --(1.375,1.416)--(1.376,1.416)--(1.376,1.417)--(1.376,1.418)--(1.376,1.419)--(1.377,1.419)%
  --(1.377,1.420)--(1.377,1.421)--(1.378,1.421)--(1.378,1.422)--(1.378,1.423)--(1.378,1.424)%
  --(1.379,1.424)--(1.379,1.425)--(1.379,1.426)--(1.379,1.427)--(1.380,1.427)--(1.380,1.428)%
  --(1.380,1.429)--(1.381,1.429)--(1.381,1.430)--(1.381,1.431)--(1.381,1.432)--(1.382,1.432)%
  --(1.382,1.433)--(1.382,1.434)--(1.383,1.434)--(1.383,1.435)--(1.383,1.436)--(1.383,1.437)%
  --(1.384,1.437)--(1.384,1.438)--(1.384,1.439)--(1.385,1.439)--(1.385,1.440)--(1.385,1.441)%
  --(1.385,1.442)--(1.386,1.442)--(1.386,1.443)--(1.386,1.444)--(1.387,1.445)--(1.387,1.446)%
  --(1.387,1.447)--(1.388,1.447)--(1.388,1.448)--(1.388,1.449)--(1.389,1.450)--(1.389,1.451)%
  --(1.389,1.452)--(1.390,1.452)--(1.390,1.453)--(1.390,1.454)--(1.390,1.455)--(1.391,1.455)%
  --(1.391,1.456)--(1.391,1.457)--(1.392,1.457)--(1.392,1.458)--(1.392,1.459)--(1.392,1.460)%
  --(1.393,1.460)--(1.393,1.461)--(1.393,1.462)--(1.394,1.462)--(1.394,1.463)--(1.394,1.464)%
  --(1.394,1.465)--(1.395,1.465)--(1.395,1.466)--(1.395,1.467)--(1.396,1.468)--(1.396,1.469)%
  --(1.396,1.470)--(1.397,1.470)--(1.397,1.471)--(1.397,1.472)--(1.398,1.473)--(1.398,1.474)%
  --(1.398,1.475)--(1.399,1.475)--(1.399,1.476)--(1.399,1.477)--(1.399,1.478)--(1.400,1.478)%
  --(1.400,1.479)--(1.400,1.480)--(1.401,1.480)--(1.401,1.481)--(1.401,1.482)--(1.401,1.483)%
  --(1.402,1.483)--(1.402,1.484)--(1.402,1.485)--(1.403,1.485)--(1.403,1.486)--(1.403,1.487)%
  --(1.403,1.488)--(1.404,1.488)--(1.404,1.489)--(1.404,1.490)--(1.404,1.491)--(1.405,1.491)%
  --(1.405,1.492)--(1.405,1.493)--(1.406,1.493)--(1.406,1.494)--(1.406,1.495)--(1.406,1.496)%
  --(1.407,1.496)--(1.407,1.497)--(1.407,1.498)--(1.408,1.498)--(1.408,1.499)--(1.408,1.500)%
  --(1.408,1.501)--(1.409,1.501)--(1.409,1.502)--(1.409,1.503)--(1.410,1.503)--(1.410,1.504)%
  --(1.410,1.505)--(1.410,1.506)--(1.411,1.506)--(1.411,1.507)--(1.411,1.508)--(1.412,1.509)%
  --(1.412,1.510)--(1.412,1.511)--(1.413,1.511)--(1.413,1.512)--(1.413,1.513)--(1.413,1.514)%
  --(1.414,1.514)--(1.414,1.515)--(1.414,1.516)--(1.415,1.516)--(1.415,1.517)--(1.415,1.518)%
  --(1.415,1.519)--(1.416,1.519)--(1.416,1.520)--(1.416,1.521)--(1.417,1.521)--(1.417,1.522)%
  --(1.417,1.523)--(1.417,1.524)--(1.418,1.524)--(1.418,1.525)--(1.418,1.526)--(1.419,1.526)%
  --(1.419,1.527)--(1.419,1.528)--(1.419,1.529)--(1.420,1.529)--(1.420,1.530)--(1.420,1.531)%
  --(1.421,1.532)--(1.421,1.533)--(1.421,1.534)--(1.422,1.534)--(1.422,1.535)--(1.422,1.536)%
  --(1.423,1.537)--(1.423,1.538)--(1.423,1.539)--(1.424,1.539)--(1.424,1.540)--(1.424,1.541)%
  --(1.424,1.542)--(1.425,1.542)--(1.425,1.543)--(1.425,1.544)--(1.426,1.544)--(1.426,1.545)%
  --(1.426,1.546)--(1.426,1.547)--(1.427,1.547)--(1.427,1.548)--(1.427,1.549)--(1.428,1.549)%
  --(1.428,1.550)--(1.428,1.551)--(1.428,1.552)--(1.429,1.552)--(1.429,1.553)--(1.429,1.554)%
  --(1.430,1.555)--(1.430,1.556)--(1.430,1.557)--(1.431,1.557)--(1.431,1.558)--(1.431,1.559)%
  --(1.432,1.560)--(1.432,1.561)--(1.432,1.562)--(1.433,1.562)--(1.433,1.563)--(1.433,1.564)%
  --(1.433,1.565)--(1.434,1.565)--(1.434,1.566)--(1.434,1.567)--(1.435,1.567)--(1.435,1.568)%
  --(1.435,1.569)--(1.435,1.570)--(1.436,1.570)--(1.436,1.571)--(1.436,1.572)--(1.437,1.572)%
  --(1.437,1.573)--(1.437,1.574)--(1.437,1.575)--(1.438,1.575)--(1.438,1.576)--(1.438,1.577)%
  --(1.439,1.578)--(1.439,1.579)--(1.439,1.580)--(1.440,1.580)--(1.440,1.581)--(1.440,1.582)%
  --(1.441,1.583)--(1.441,1.584)--(1.441,1.585)--(1.442,1.585)--(1.442,1.586)--(1.442,1.587)%
  --(1.442,1.588)--(1.443,1.588)--(1.443,1.589)--(1.443,1.590)--(1.444,1.590)--(1.444,1.591)%
  --(1.444,1.592)--(1.444,1.593)--(1.445,1.593)--(1.445,1.594)--(1.445,1.595)--(1.446,1.595)%
  --(1.446,1.596)--(1.446,1.597)--(1.446,1.598)--(1.447,1.598)--(1.447,1.599)--(1.447,1.600)%
  --(1.448,1.600)--(1.448,1.601)--(1.448,1.602)--(1.448,1.603)--(1.449,1.603)--(1.449,1.604)%
  --(1.449,1.605)--(1.450,1.606)--(1.450,1.607)--(1.450,1.608)--(1.451,1.608)--(1.451,1.609)%
  --(1.451,1.610)--(1.451,1.611)--(1.452,1.611)--(1.452,1.612)--(1.452,1.613)--(1.453,1.613)%
  --(1.453,1.614)--(1.453,1.615)--(1.453,1.616)--(1.454,1.616)--(1.454,1.617)--(1.454,1.618)%
  --(1.455,1.618)--(1.455,1.619)--(1.455,1.620)--(1.455,1.621)--(1.456,1.621)--(1.456,1.622)%
  --(1.456,1.623)--(1.457,1.623)--(1.457,1.624)--(1.457,1.625)--(1.457,1.626)--(1.458,1.626)%
  --(1.458,1.627)--(1.458,1.628)--(1.459,1.629)--(1.459,1.630)--(1.459,1.631)--(1.460,1.631)%
  --(1.460,1.632)--(1.460,1.633)--(1.460,1.634)--(1.461,1.634)--(1.461,1.635)--(1.461,1.636)%
  --(1.462,1.636)--(1.462,1.637)--(1.462,1.638)--(1.462,1.639)--(1.463,1.639)--(1.463,1.640)%
  --(1.463,1.641)--(1.464,1.641)--(1.464,1.642)--(1.464,1.643)--(1.464,1.644)--(1.465,1.644)%
  --(1.465,1.645)--(1.465,1.646)--(1.466,1.646)--(1.466,1.647)--(1.466,1.648)--(1.466,1.649)%
  --(1.467,1.649)--(1.467,1.650)--(1.467,1.651)--(1.468,1.651)--(1.468,1.652)--(1.468,1.653)%
  --(1.468,1.654)--(1.469,1.654)--(1.469,1.655)--(1.469,1.656)--(1.470,1.656)--(1.470,1.657)%
  --(1.470,1.658)--(1.470,1.659)--(1.471,1.659)--(1.471,1.660)--(1.471,1.661)--(1.472,1.662)%
  --(1.472,1.663)--(1.472,1.664)--(1.473,1.664)--(1.473,1.665)--(1.473,1.666)--(1.473,1.667)%
  --(1.474,1.667)--(1.474,1.668)--(1.474,1.669)--(1.475,1.669)--(1.475,1.670)--(1.475,1.671)%
  --(1.475,1.672)--(1.476,1.672)--(1.476,1.673)--(1.476,1.674)--(1.477,1.674)--(1.477,1.675)%
  --(1.477,1.676)--(1.477,1.677)--(1.478,1.677)--(1.478,1.678)--(1.478,1.679)--(1.479,1.679)%
  --(1.479,1.680)--(1.479,1.681)--(1.479,1.682)--(1.480,1.682)--(1.480,1.683)--(1.480,1.684)%
  --(1.481,1.685)--(1.481,1.686)--(1.481,1.687)--(1.482,1.687)--(1.482,1.688)--(1.482,1.689)%
  --(1.483,1.690)--(1.483,1.691)--(1.483,1.692)--(1.484,1.692)--(1.484,1.693)--(1.484,1.694)%
  --(1.485,1.695)--(1.485,1.696)--(1.485,1.697)--(1.486,1.697)--(1.486,1.698)--(1.486,1.699)%
  --(1.486,1.700)--(1.487,1.700)--(1.487,1.701)--(1.487,1.702)--(1.488,1.702)--(1.488,1.703)%
  --(1.488,1.704)--(1.488,1.705)--(1.489,1.705)--(1.489,1.706)--(1.489,1.707)--(1.490,1.707)%
  --(1.490,1.708)--(1.490,1.709)--(1.490,1.710)--(1.491,1.710)--(1.491,1.711)--(1.491,1.712)%
  --(1.492,1.713)--(1.492,1.714)--(1.492,1.715)--(1.493,1.715)--(1.493,1.716)--(1.493,1.717)%
  --(1.494,1.718)--(1.494,1.719)--(1.494,1.720)--(1.495,1.720)--(1.495,1.721)--(1.495,1.722)%
  --(1.496,1.723)--(1.496,1.724)--(1.496,1.725)--(1.497,1.725)--(1.497,1.726)--(1.497,1.727)%
  --(1.498,1.728)--(1.498,1.729)--(1.498,1.730)--(1.499,1.730)--(1.499,1.731)--(1.499,1.732)%
  --(1.499,1.733)--(1.500,1.733)--(1.500,1.734)--(1.500,1.735)--(1.501,1.735)--(1.501,1.736)%
  --(1.501,1.737)--(1.501,1.738)--(1.502,1.738)--(1.502,1.739)--(1.502,1.740)--(1.503,1.740)%
  --(1.503,1.741)--(1.503,1.742)--(1.503,1.743)--(1.504,1.743)--(1.504,1.744)--(1.504,1.745)%
  --(1.505,1.745)--(1.505,1.746)--(1.505,1.747)--(1.505,1.748)--(1.506,1.748)--(1.506,1.749)%
  --(1.506,1.750)--(1.507,1.750)--(1.507,1.751)--(1.507,1.752)--(1.507,1.753)--(1.508,1.753)%
  --(1.508,1.754)--(1.508,1.755)--(1.509,1.756)--(1.509,1.757)--(1.509,1.758)--(1.510,1.758)%
  --(1.510,1.759)--(1.510,1.760)--(1.510,1.761)--(1.511,1.761)--(1.511,1.762)--(1.511,1.763)%
  --(1.512,1.763)--(1.512,1.764)--(1.512,1.765)--(1.512,1.766)--(1.513,1.766)--(1.513,1.767)%
  --(1.513,1.768)--(1.514,1.768)--(1.514,1.769)--(1.514,1.770)--(1.514,1.771)--(1.515,1.771)%
  --(1.515,1.772)--(1.515,1.773)--(1.516,1.773)--(1.516,1.774)--(1.516,1.775)--(1.516,1.776)%
  --(1.517,1.776)--(1.517,1.777)--(1.517,1.778)--(1.518,1.778)--(1.518,1.779)--(1.518,1.780)%
  --(1.518,1.781)--(1.519,1.781)--(1.519,1.782)--(1.519,1.783)--(1.520,1.784)--(1.520,1.785)%
  --(1.520,1.786)--(1.521,1.786)--(1.521,1.787)--(1.521,1.788)--(1.522,1.788)--(1.522,1.789)%
  --(1.522,1.790)--(1.522,1.791)--(1.523,1.791)--(1.523,1.792)--(1.523,1.793)--(1.524,1.794)%
  --(1.524,1.795)--(1.524,1.796)--(1.525,1.796)--(1.525,1.797)--(1.525,1.798)--(1.526,1.799)%
  --(1.526,1.800)--(1.526,1.801)--(1.527,1.801)--(1.527,1.802)--(1.527,1.803)--(1.527,1.804)%
  --(1.528,1.804)--(1.528,1.805)--(1.528,1.806)--(1.529,1.806)--(1.529,1.807)--(1.529,1.808)%
  --(1.529,1.809)--(1.530,1.809)--(1.530,1.810)--(1.530,1.811)--(1.531,1.811)--(1.531,1.812)%
  --(1.531,1.813)--(1.531,1.814)--(1.532,1.814)--(1.532,1.815)--(1.532,1.816)--(1.533,1.816)%
  --(1.533,1.817)--(1.533,1.818)--(1.533,1.819)--(1.534,1.819)--(1.534,1.820)--(1.534,1.821)%
  --(1.535,1.821)--(1.535,1.822)--(1.535,1.823)--(1.535,1.824)--(1.536,1.824)--(1.536,1.825)%
  --(1.536,1.826)--(1.537,1.826)--(1.537,1.827)--(1.537,1.828)--(1.537,1.829)--(1.538,1.829)%
  --(1.538,1.830)--(1.538,1.831)--(1.539,1.832)--(1.539,1.833)--(1.539,1.834)--(1.540,1.834)%
  --(1.540,1.835)--(1.540,1.836)--(1.541,1.836)--(1.541,1.837)--(1.541,1.838)--(1.541,1.839)%
  --(1.542,1.839)--(1.542,1.840)--(1.542,1.841)--(1.543,1.842)--(1.543,1.843)--(1.543,1.844)%
  --(1.544,1.844)--(1.544,1.845)--(1.544,1.846)--(1.544,1.847)--(1.545,1.847)--(1.545,1.848)%
  --(1.545,1.849)--(1.546,1.849)--(1.546,1.850)--(1.546,1.851)--(1.547,1.852)--(1.547,1.853)%
  --(1.547,1.854)--(1.548,1.854)--(1.548,1.855)--(1.548,1.856)--(1.548,1.857)--(1.549,1.857)%
  --(1.549,1.858)--(1.549,1.859)--(1.550,1.859)--(1.550,1.860)--(1.550,1.861)--(1.550,1.862)%
  --(1.551,1.862)--(1.551,1.863)--(1.551,1.864)--(1.552,1.864)--(1.552,1.865)--(1.552,1.866)%
  --(1.552,1.867)--(1.553,1.867)--(1.553,1.868)--(1.553,1.869)--(1.554,1.869)--(1.554,1.870)%
  --(1.554,1.871)--(1.554,1.872)--(1.555,1.872)--(1.555,1.873)--(1.555,1.874)--(1.556,1.874)%
  --(1.556,1.875)--(1.556,1.876)--(1.556,1.877)--(1.557,1.877)--(1.557,1.878)--(1.557,1.879)%
  --(1.558,1.879)--(1.558,1.880)--(1.558,1.881)--(1.558,1.882)--(1.559,1.882)--(1.559,1.883)%
  --(1.559,1.884)--(1.560,1.884)--(1.560,1.885)--(1.560,1.886)--(1.560,1.887)--(1.561,1.887)%
  --(1.561,1.888)--(1.561,1.889)--(1.562,1.889)--(1.562,1.890)--(1.562,1.891)--(1.562,1.892)%
  --(1.563,1.892)--(1.563,1.893)--(1.563,1.894)--(1.564,1.895)--(1.564,1.896)--(1.564,1.897)%
  --(1.565,1.897)--(1.565,1.898)--(1.565,1.899)--(1.566,1.899)--(1.566,1.900)--(1.566,1.901)%
  --(1.566,1.902)--(1.567,1.902)--(1.567,1.903)--(1.567,1.904)--(1.568,1.905)--(1.568,1.906)%
  --(1.568,1.907)--(1.569,1.907)--(1.569,1.908)--(1.569,1.909)--(1.570,1.910)--(1.570,1.911)%
  --(1.570,1.912)--(1.571,1.912)--(1.571,1.913)--(1.571,1.914)--(1.572,1.915)--(1.572,1.916)%
  --(1.572,1.917)--(1.573,1.917)--(1.573,1.918)--(1.573,1.919)--(1.574,1.920)--(1.574,1.921)%
  --(1.574,1.922)--(1.575,1.922)--(1.575,1.923)--(1.575,1.924)--(1.575,1.925)--(1.576,1.925)%
  --(1.576,1.926)--(1.576,1.927)--(1.577,1.927)--(1.577,1.928)--(1.577,1.929)--(1.578,1.930)%
  --(1.578,1.931)--(1.578,1.932)--(1.579,1.932)--(1.579,1.933)--(1.579,1.934)--(1.579,1.935)%
  --(1.580,1.935)--(1.580,1.936)--(1.580,1.937)--(1.581,1.937)--(1.581,1.938)--(1.581,1.939)%
  --(1.582,1.940)--(1.582,1.941)--(1.582,1.942)--(1.583,1.942)--(1.583,1.943)--(1.583,1.944)%
  --(1.583,1.945)--(1.584,1.945)--(1.584,1.946)--(1.584,1.947)--(1.585,1.947)--(1.585,1.948)%
  --(1.585,1.949)--(1.585,1.950)--(1.586,1.950)--(1.586,1.951)--(1.586,1.952)--(1.587,1.952)%
  --(1.587,1.953)--(1.587,1.954)--(1.588,1.955)--(1.588,1.956)--(1.588,1.957)--(1.589,1.957)%
  --(1.589,1.958)--(1.589,1.959)--(1.589,1.960)--(1.590,1.960)--(1.590,1.961)--(1.590,1.962)%
  --(1.591,1.962)--(1.591,1.963)--(1.591,1.964)--(1.591,1.965)--(1.592,1.965)--(1.592,1.966)%
  --(1.592,1.967)--(1.593,1.967)--(1.593,1.968)--(1.593,1.969)--(1.593,1.970)--(1.594,1.970)%
  --(1.594,1.971)--(1.594,1.972)--(1.595,1.972)--(1.595,1.973)--(1.595,1.974)--(1.595,1.975)%
  --(1.596,1.975)--(1.596,1.976)--(1.596,1.977)--(1.597,1.977)--(1.597,1.978)--(1.597,1.979)%
  --(1.597,1.980)--(1.598,1.980)--(1.598,1.981)--(1.598,1.982)--(1.599,1.982)--(1.599,1.983)%
  --(1.599,1.984)--(1.599,1.985)--(1.600,1.985)--(1.600,1.986)--(1.600,1.987)--(1.601,1.987)%
  --(1.601,1.988)--(1.601,1.989)--(1.601,1.990)--(1.602,1.990)--(1.602,1.991)--(1.602,1.992)%
  --(1.603,1.992)--(1.603,1.993)--(1.603,1.994)--(1.603,1.995)--(1.604,1.995)--(1.604,1.996)%
  --(1.604,1.997)--(1.605,1.997)--(1.605,1.998)--(1.605,1.999)--(1.605,2.000)--(1.606,2.000)%
  --(1.606,2.001)--(1.606,2.002)--(1.607,2.002)--(1.607,2.003)--(1.607,2.004)--(1.607,2.005)%
  --(1.608,2.005)--(1.608,2.006)--(1.608,2.007)--(1.609,2.007)--(1.609,2.008)--(1.609,2.009)%
  --(1.610,2.010)--(1.610,2.011)--(1.610,2.012)--(1.611,2.012)--(1.611,2.013)--(1.611,2.014)%
  --(1.611,2.015)--(1.612,2.015)--(1.612,2.016)--(1.612,2.017)--(1.613,2.017)--(1.613,2.018)%
  --(1.613,2.019)--(1.613,2.020)--(1.614,2.020)--(1.614,2.021)--(1.614,2.022)--(1.615,2.022)%
  --(1.615,2.023)--(1.615,2.024)--(1.616,2.025)--(1.616,2.026)--(1.616,2.027)--(1.617,2.027)%
  --(1.617,2.028)--(1.617,2.029)--(1.618,2.030)--(1.618,2.031)--(1.618,2.032)--(1.619,2.032)%
  --(1.619,2.033)--(1.619,2.034)--(1.619,2.035)--(1.620,2.035)--(1.620,2.036)--(1.620,2.037)%
  --(1.621,2.037)--(1.621,2.038)--(1.621,2.039)--(1.622,2.040)--(1.622,2.041)--(1.622,2.042)%
  --(1.623,2.042)--(1.623,2.043)--(1.623,2.044)--(1.624,2.045)--(1.624,2.046)--(1.624,2.047)%
  --(1.625,2.047)--(1.625,2.048)--(1.625,2.049)--(1.626,2.050)--(1.626,2.051)--(1.626,2.052)%
  --(1.627,2.052)--(1.627,2.053)--(1.627,2.054)--(1.628,2.054)--(1.628,2.055)--(1.628,2.056)%
  --(1.628,2.057)--(1.629,2.057)--(1.629,2.058)--(1.629,2.059)--(1.630,2.059)--(1.630,2.060)%
  --(1.630,2.061)--(1.630,2.062)--(1.631,2.062)--(1.631,2.063)--(1.631,2.064)--(1.632,2.065)%
  --(1.632,2.066)--(1.632,2.067)--(1.633,2.067)--(1.633,2.068)--(1.633,2.069)--(1.634,2.069)%
  --(1.634,2.070)--(1.634,2.071)--(1.634,2.072)--(1.635,2.072)--(1.635,2.073)--(1.635,2.074)%
  --(1.636,2.074)--(1.636,2.075)--(1.636,2.076)--(1.636,2.077)--(1.637,2.077)--(1.637,2.078)%
  --(1.637,2.079)--(1.638,2.079)--(1.638,2.080)--(1.638,2.081)--(1.638,2.082)--(1.639,2.082)%
  --(1.639,2.083)--(1.639,2.084)--(1.640,2.084)--(1.640,2.085)--(1.640,2.086)--(1.640,2.087)%
  --(1.641,2.087)--(1.641,2.088)--(1.641,2.089)--(1.642,2.089)--(1.642,2.090)--(1.642,2.091)%
  --(1.642,2.092)--(1.643,2.092)--(1.643,2.093)--(1.643,2.094)--(1.644,2.094)--(1.644,2.095)%
  --(1.644,2.096)--(1.644,2.097)--(1.645,2.097)--(1.645,2.098)--(1.645,2.099)--(1.646,2.099)%
  --(1.646,2.100)--(1.646,2.101)--(1.647,2.102)--(1.647,2.103)--(1.647,2.104)--(1.648,2.104)%
  --(1.648,2.105)--(1.648,2.106)--(1.649,2.107)--(1.649,2.108)--(1.649,2.109)--(1.650,2.109)%
  --(1.650,2.110)--(1.650,2.111)--(1.651,2.112)--(1.651,2.113)--(1.651,2.114)--(1.652,2.114)%
  --(1.652,2.115)--(1.652,2.116)--(1.653,2.116)--(1.653,2.117)--(1.653,2.118)--(1.653,2.119)%
  --(1.654,2.119)--(1.654,2.120)--(1.654,2.121)--(1.655,2.121)--(1.655,2.122)--(1.655,2.123)%
  --(1.655,2.124)--(1.656,2.124)--(1.656,2.125)--(1.656,2.126)--(1.657,2.126)--(1.657,2.127)%
  --(1.657,2.128)--(1.657,2.129)--(1.658,2.129)--(1.658,2.130)--(1.658,2.131)--(1.659,2.131)%
  --(1.659,2.132)--(1.659,2.133)--(1.659,2.134)--(1.660,2.134)--(1.660,2.135)--(1.660,2.136)%
  --(1.661,2.136)--(1.661,2.137)--(1.661,2.138)--(1.661,2.139)--(1.662,2.139)--(1.662,2.140)%
  --(1.662,2.141)--(1.663,2.141)--(1.663,2.142)--(1.663,2.143)--(1.663,2.144)--(1.664,2.144)%
  --(1.664,2.145)--(1.664,2.146)--(1.665,2.146)--(1.665,2.147)--(1.665,2.148)--(1.666,2.149)%
  --(1.666,2.150)--(1.666,2.151)--(1.667,2.151)--(1.667,2.152)--(1.667,2.153)--(1.668,2.153)%
  --(1.668,2.154)--(1.668,2.155)--(1.668,2.156)--(1.669,2.156)--(1.669,2.157)--(1.669,2.158)%
  --(1.670,2.158)--(1.670,2.159)--(1.670,2.160)--(1.670,2.161)--(1.671,2.161)--(1.671,2.162)%
  --(1.671,2.163)--(1.672,2.163)--(1.672,2.164)--(1.672,2.165)--(1.672,2.166)--(1.673,2.166)%
  --(1.673,2.167)--(1.673,2.168)--(1.674,2.168)--(1.674,2.169)--(1.674,2.170)--(1.674,2.171)%
  --(1.675,2.171)--(1.675,2.172)--(1.675,2.173)--(1.676,2.173)--(1.676,2.174)--(1.676,2.175)%
  --(1.676,2.176)--(1.677,2.176)--(1.677,2.177)--(1.677,2.178)--(1.678,2.178)--(1.678,2.179)%
  --(1.678,2.180)--(1.679,2.181)--(1.679,2.182)--(1.679,2.183)--(1.680,2.183)--(1.680,2.184)%
  --(1.680,2.185)--(1.681,2.185)--(1.681,2.186)--(1.681,2.187)--(1.681,2.188)--(1.682,2.188)%
  --(1.682,2.189)--(1.682,2.190)--(1.683,2.190)--(1.683,2.191)--(1.683,2.192)--(1.683,2.193)%
  --(1.684,2.193)--(1.684,2.194)--(1.684,2.195)--(1.685,2.195)--(1.685,2.196)--(1.685,2.197)%
  --(1.685,2.198)--(1.686,2.198)--(1.686,2.199)--(1.686,2.200)--(1.687,2.200)--(1.687,2.201)%
  --(1.687,2.202)--(1.688,2.203)--(1.688,2.204)--(1.688,2.205)--(1.689,2.205)--(1.689,2.206)%
  --(1.689,2.207)--(1.690,2.208)--(1.690,2.209)--(1.690,2.210)--(1.691,2.210)--(1.691,2.211)%
  --(1.691,2.212)--(1.692,2.213)--(1.692,2.214)--(1.692,2.215)--(1.693,2.215)--(1.693,2.216)%
  --(1.693,2.217)--(1.694,2.217)--(1.694,2.218)--(1.694,2.219)--(1.694,2.220)--(1.695,2.220)%
  --(1.695,2.221)--(1.695,2.222)--(1.696,2.222)--(1.696,2.223)--(1.696,2.224)--(1.696,2.225)%
  --(1.697,2.225)--(1.697,2.226)--(1.697,2.227)--(1.698,2.227)--(1.698,2.228)--(1.698,2.229)%
  --(1.699,2.230)--(1.699,2.231)--(1.699,2.232)--(1.700,2.232)--(1.700,2.233)--(1.700,2.234)%
  --(1.701,2.235)--(1.701,2.236)--(1.701,2.237)--(1.702,2.237)--(1.702,2.238)--(1.702,2.239)%
  --(1.703,2.239)--(1.703,2.240)--(1.703,2.241)--(1.703,2.242)--(1.704,2.242)--(1.704,2.243)%
  --(1.704,2.244)--(1.705,2.244)--(1.705,2.245)--(1.705,2.246)--(1.705,2.247)--(1.706,2.247)%
  --(1.706,2.248)--(1.706,2.249)--(1.707,2.249)--(1.707,2.250)--(1.707,2.251)--(1.708,2.252)%
  --(1.708,2.253)--(1.708,2.254)--(1.709,2.254)--(1.709,2.255)--(1.709,2.256)--(1.710,2.257)%
  --(1.710,2.258)--(1.710,2.259)--(1.711,2.259)--(1.711,2.260)--(1.711,2.261)--(1.712,2.261)%
  --(1.712,2.262)--(1.712,2.263)--(1.712,2.264)--(1.713,2.264)--(1.713,2.265)--(1.713,2.266)%
  --(1.714,2.266)--(1.714,2.267)--(1.714,2.268)--(1.714,2.269)--(1.715,2.269)--(1.715,2.270)%
  --(1.715,2.271)--(1.716,2.271)--(1.716,2.272)--(1.716,2.273)--(1.717,2.274)--(1.717,2.275)%
  --(1.717,2.276)--(1.718,2.276)--(1.718,2.277)--(1.718,2.278)--(1.719,2.278)--(1.719,2.279)%
  --(1.719,2.280)--(1.719,2.281)--(1.720,2.281)--(1.720,2.282)--(1.720,2.283)--(1.721,2.283)%
  --(1.721,2.284)--(1.721,2.285)--(1.721,2.286)--(1.722,2.286)--(1.722,2.287)--(1.722,2.288)%
  --(1.723,2.288)--(1.723,2.289)--(1.723,2.290)--(1.724,2.291)--(1.724,2.292)--(1.724,2.293)%
  --(1.725,2.293)--(1.725,2.294)--(1.725,2.295)--(1.726,2.295)--(1.726,2.296)--(1.726,2.297)%
  --(1.726,2.298)--(1.727,2.298)--(1.727,2.299)--(1.727,2.300)--(1.728,2.300)--(1.728,2.301)%
  --(1.728,2.302)--(1.728,2.303)--(1.729,2.303)--(1.729,2.304)--(1.729,2.305)--(1.730,2.305)%
  --(1.730,2.306)--(1.730,2.307)--(1.731,2.308)--(1.731,2.309)--(1.731,2.310)--(1.732,2.310)%
  --(1.732,2.311)--(1.732,2.312)--(1.733,2.312)--(1.733,2.313)--(1.733,2.314)--(1.733,2.315)%
  --(1.734,2.315)--(1.734,2.316)--(1.734,2.317)--(1.735,2.317)--(1.735,2.318)--(1.735,2.319)%
  --(1.735,2.320)--(1.736,2.320)--(1.736,2.321)--(1.736,2.322)--(1.737,2.322)--(1.737,2.323)%
  --(1.737,2.324)--(1.738,2.325)--(1.738,2.326)--(1.738,2.327)--(1.739,2.327)--(1.739,2.328)%
  --(1.739,2.329)--(1.740,2.329)--(1.740,2.330)--(1.740,2.331)--(1.740,2.332)--(1.741,2.332)%
  --(1.741,2.333)--(1.741,2.334)--(1.742,2.334)--(1.742,2.335)--(1.742,2.336)--(1.743,2.337)%
  --(1.743,2.338)--(1.743,2.339)--(1.744,2.339)--(1.744,2.340)--(1.744,2.341)--(1.745,2.342)%
  --(1.745,2.343)--(1.745,2.344)--(1.746,2.344)--(1.746,2.345)--(1.746,2.346)--(1.747,2.346)%
  --(1.747,2.347)--(1.747,2.348)--(1.747,2.349)--(1.748,2.349)--(1.748,2.350)--(1.748,2.351)%
  --(1.749,2.351)--(1.749,2.352)--(1.749,2.353)--(1.750,2.354)--(1.750,2.355)--(1.750,2.356)%
  --(1.751,2.356)--(1.751,2.357)--(1.751,2.358)--(1.752,2.358)--(1.752,2.359)--(1.752,2.360)%
  --(1.752,2.361)--(1.753,2.361)--(1.753,2.362)--(1.753,2.363)--(1.754,2.363)--(1.754,2.364)%
  --(1.754,2.365)--(1.755,2.366)--(1.755,2.367)--(1.755,2.368)--(1.756,2.368)--(1.756,2.369)%
  --(1.756,2.370)--(1.757,2.371)--(1.757,2.372)--(1.757,2.373)--(1.758,2.373)--(1.758,2.374)%
  --(1.758,2.375)--(1.759,2.375)--(1.759,2.376)--(1.759,2.377)--(1.759,2.378)--(1.760,2.378)%
  --(1.760,2.379)--(1.760,2.380)--(1.761,2.380)--(1.761,2.381)--(1.761,2.382)--(1.762,2.382)%
  --(1.762,2.383)--(1.762,2.384)--(1.762,2.385)--(1.763,2.385)--(1.763,2.386)--(1.763,2.387)%
  --(1.764,2.387)--(1.764,2.388)--(1.764,2.389)--(1.764,2.390)--(1.765,2.390)--(1.765,2.391)%
  --(1.765,2.392)--(1.766,2.392)--(1.766,2.393)--(1.766,2.394)--(1.767,2.395)--(1.767,2.396)%
  --(1.767,2.397)--(1.768,2.397)--(1.768,2.398)--(1.768,2.399)--(1.769,2.399)--(1.769,2.400)%
  --(1.769,2.401)--(1.769,2.402)--(1.770,2.402)--(1.770,2.403)--(1.770,2.404)--(1.771,2.404)%
  --(1.771,2.405)--(1.771,2.406)--(1.772,2.406)--(1.772,2.407)--(1.772,2.408)--(1.772,2.409)%
  --(1.773,2.409)--(1.773,2.410)--(1.773,2.411)--(1.774,2.411)--(1.774,2.412)--(1.774,2.413)%
  --(1.775,2.414)--(1.775,2.415)--(1.775,2.416)--(1.776,2.416)--(1.776,2.417)--(1.776,2.418)%
  --(1.777,2.418)--(1.777,2.419)--(1.777,2.420)--(1.777,2.421)--(1.778,2.421)--(1.778,2.422)%
  --(1.778,2.423)--(1.779,2.423)--(1.779,2.424)--(1.779,2.425)--(1.779,2.426)--(1.780,2.426)%
  --(1.780,2.427)--(1.780,2.428)--(1.781,2.428)--(1.781,2.429)--(1.781,2.430)--(1.782,2.431)%
  --(1.782,2.432)--(1.782,2.433)--(1.783,2.433)--(1.783,2.434)--(1.783,2.435)--(1.784,2.435)%
  --(1.784,2.436)--(1.784,2.437)--(1.784,2.438)--(1.785,2.438)--(1.785,2.439)--(1.785,2.440)%
  --(1.786,2.440)--(1.786,2.441)--(1.786,2.442)--(1.787,2.442)--(1.787,2.443)--(1.787,2.444)%
  --(1.787,2.445)--(1.788,2.445)--(1.788,2.446)--(1.788,2.447)--(1.789,2.447)--(1.789,2.448)%
  --(1.789,2.449)--(1.790,2.450)--(1.790,2.451)--(1.790,2.452)--(1.791,2.452)--(1.791,2.453)%
  --(1.791,2.454)--(1.792,2.454)--(1.792,2.455)--(1.792,2.456)--(1.792,2.457)--(1.793,2.457)%
  --(1.793,2.458)--(1.793,2.459)--(1.794,2.459)--(1.794,2.460)--(1.794,2.461)--(1.795,2.462)%
  --(1.795,2.463)--(1.795,2.464)--(1.796,2.464)--(1.796,2.465)--(1.796,2.466)--(1.797,2.466)%
  --(1.797,2.467)--(1.797,2.468)--(1.797,2.469)--(1.798,2.469)--(1.798,2.470)--(1.798,2.471)%
  --(1.799,2.471)--(1.799,2.472)--(1.799,2.473)--(1.800,2.473)--(1.800,2.474)--(1.800,2.475)%
  --(1.800,2.476)--(1.801,2.476)--(1.801,2.477)--(1.801,2.478)--(1.802,2.478)--(1.802,2.479)%
  --(1.802,2.480)--(1.803,2.480)--(1.803,2.481)--(1.803,2.482)--(1.803,2.483)--(1.804,2.483)%
  --(1.804,2.484)--(1.804,2.485)--(1.805,2.485)--(1.805,2.486)--(1.805,2.487)--(1.806,2.488)%
  --(1.806,2.489)--(1.806,2.490)--(1.807,2.490)--(1.807,2.491)--(1.807,2.492)--(1.808,2.492)%
  --(1.808,2.493)--(1.808,2.494)--(1.808,2.495)--(1.809,2.495)--(1.809,2.496)--(1.809,2.497)%
  --(1.810,2.497)--(1.810,2.498)--(1.810,2.499)--(1.811,2.500)--(1.811,2.501)--(1.811,2.502)%
  --(1.812,2.502)--(1.812,2.503)--(1.812,2.504)--(1.813,2.504)--(1.813,2.505)--(1.813,2.506)%
  --(1.814,2.507)--(1.814,2.508)--(1.814,2.509)--(1.815,2.509)--(1.815,2.510)--(1.815,2.511)%
  --(1.816,2.511)--(1.816,2.512)--(1.816,2.513)--(1.816,2.514)--(1.817,2.514)--(1.817,2.515)%
  --(1.817,2.516)--(1.818,2.516)--(1.818,2.517)--(1.818,2.518)--(1.819,2.519)--(1.819,2.520)%
  --(1.819,2.521)--(1.820,2.521)--(1.820,2.522)--(1.820,2.523)--(1.821,2.523)--(1.821,2.524)%
  --(1.821,2.525)--(1.822,2.526)--(1.822,2.527)--(1.822,2.528)--(1.823,2.528)--(1.823,2.529)%
  --(1.823,2.530)--(1.824,2.530)--(1.824,2.531)--(1.824,2.532)--(1.825,2.533)--(1.825,2.534)%
  --(1.825,2.535)--(1.826,2.535)--(1.826,2.536)--(1.826,2.537)--(1.827,2.537)--(1.827,2.538)%
  --(1.827,2.539)--(1.828,2.540)--(1.828,2.541)--(1.828,2.542)--(1.829,2.542)--(1.829,2.543)%
  --(1.829,2.544)--(1.830,2.544)--(1.830,2.545)--(1.830,2.546)--(1.830,2.547)--(1.831,2.547)%
  --(1.831,2.548)--(1.831,2.549)--(1.832,2.549)--(1.832,2.550)--(1.832,2.551)--(1.833,2.551)%
  --(1.833,2.552)--(1.833,2.553)--(1.833,2.554)--(1.834,2.554)--(1.834,2.555)--(1.834,2.556)%
  --(1.835,2.556)--(1.835,2.557)--(1.835,2.558)--(1.836,2.559)--(1.836,2.560)--(1.836,2.561)%
  --(1.837,2.561)--(1.837,2.562)--(1.837,2.563)--(1.838,2.563)--(1.838,2.564)--(1.838,2.565)%
  --(1.839,2.566)--(1.839,2.567)--(1.839,2.568)--(1.840,2.568)--(1.840,2.569)--(1.840,2.570)%
  --(1.841,2.570)--(1.841,2.571)--(1.841,2.572)--(1.842,2.573)--(1.842,2.574)--(1.842,2.575)%
  --(1.843,2.575)--(1.843,2.576)--(1.843,2.577)--(1.844,2.577)--(1.844,2.578)--(1.844,2.579)%
  --(1.844,2.580)--(1.845,2.580)--(1.845,2.581)--(1.845,2.582)--(1.846,2.582)--(1.846,2.583)%
  --(1.846,2.584)--(1.847,2.584)--(1.847,2.585)--(1.847,2.586)--(1.847,2.587)--(1.848,2.587)%
  --(1.848,2.588)--(1.848,2.589)--(1.849,2.589)--(1.849,2.590)--(1.849,2.591)--(1.850,2.591)%
  --(1.850,2.592)--(1.850,2.593)--(1.850,2.594)--(1.851,2.594)--(1.851,2.595)--(1.851,2.596)%
  --(1.852,2.596)--(1.852,2.597)--(1.852,2.598)--(1.853,2.598)--(1.853,2.599)--(1.853,2.600)%
  --(1.853,2.601)--(1.854,2.601)--(1.854,2.602)--(1.854,2.603)--(1.855,2.603)--(1.855,2.604)%
  --(1.855,2.605)--(1.856,2.605)--(1.856,2.606)--(1.856,2.607)--(1.856,2.608)--(1.857,2.608)%
  --(1.857,2.609)--(1.857,2.610)--(1.858,2.610)--(1.858,2.611)--(1.858,2.612)--(1.859,2.612)%
  --(1.859,2.613)--(1.859,2.614)--(1.859,2.615)--(1.860,2.615)--(1.860,2.616)--(1.860,2.617)%
  --(1.861,2.617)--(1.861,2.618)--(1.861,2.619)--(1.862,2.619)--(1.862,2.620)--(1.862,2.621)%
  --(1.862,2.622)--(1.863,2.622)--(1.863,2.623)--(1.863,2.624)--(1.864,2.624)--(1.864,2.625)%
  --(1.864,2.626)--(1.865,2.626)--(1.865,2.627)--(1.865,2.628)--(1.865,2.629)--(1.866,2.629)%
  --(1.866,2.630)--(1.866,2.631)--(1.867,2.631)--(1.867,2.632)--(1.867,2.633)--(1.868,2.633)%
  --(1.868,2.634)--(1.868,2.635)--(1.868,2.636)--(1.869,2.636)--(1.869,2.637)--(1.869,2.638)%
  --(1.870,2.638)--(1.870,2.639)--(1.870,2.640)--(1.871,2.640)--(1.871,2.641)--(1.871,2.642)%
  --(1.871,2.643)--(1.872,2.643)--(1.872,2.644)--(1.872,2.645)--(1.873,2.645)--(1.873,2.646)%
  --(1.873,2.647)--(1.874,2.647)--(1.874,2.648)--(1.874,2.649)--(1.874,2.650)--(1.875,2.650)%
  --(1.875,2.651)--(1.875,2.652)--(1.876,2.652)--(1.876,2.653)--(1.876,2.654)--(1.877,2.654)%
  --(1.877,2.655)--(1.877,2.656)--(1.878,2.657)--(1.878,2.658)--(1.878,2.659)--(1.879,2.659)%
  --(1.879,2.660)--(1.879,2.661)--(1.880,2.661)--(1.880,2.662)--(1.880,2.663)--(1.881,2.663)%
  --(1.881,2.664)--(1.881,2.665)--(1.881,2.666)--(1.882,2.666)--(1.882,2.667)--(1.882,2.668)%
  --(1.883,2.668)--(1.883,2.669)--(1.883,2.670)--(1.884,2.670)--(1.884,2.671)--(1.884,2.672)%
  --(1.884,2.673)--(1.885,2.673)--(1.885,2.674)--(1.885,2.675)--(1.886,2.675)--(1.886,2.676)%
  --(1.886,2.677)--(1.887,2.677)--(1.887,2.678)--(1.887,2.679)--(1.887,2.680)--(1.888,2.680)%
  --(1.888,2.681)--(1.888,2.682)--(1.889,2.682)--(1.889,2.683)--(1.889,2.684)--(1.890,2.684)%
  --(1.890,2.685)--(1.890,2.686)--(1.890,2.687)--(1.891,2.687)--(1.891,2.688)--(1.891,2.689)%
  --(1.892,2.689)--(1.892,2.690)--(1.892,2.691)--(1.893,2.691)--(1.893,2.692)--(1.893,2.693)%
  --(1.894,2.694)--(1.894,2.695)--(1.894,2.696)--(1.895,2.696)--(1.895,2.697)--(1.895,2.698)%
  --(1.896,2.698)--(1.896,2.699)--(1.896,2.700)--(1.897,2.701)--(1.897,2.702)--(1.897,2.703)%
  --(1.898,2.703)--(1.898,2.704)--(1.898,2.705)--(1.899,2.705)--(1.899,2.706)--(1.899,2.707)%
  --(1.900,2.707)--(1.900,2.708)--(1.900,2.709)--(1.900,2.710)--(1.901,2.710)--(1.901,2.711)%
  --(1.901,2.712)--(1.902,2.712)--(1.902,2.713)--(1.902,2.714)--(1.903,2.714)--(1.903,2.715)%
  --(1.903,2.716)--(1.904,2.717)--(1.904,2.718)--(1.904,2.719)--(1.905,2.719)--(1.905,2.720)%
  --(1.905,2.721)--(1.906,2.721)--(1.906,2.722)--(1.906,2.723)--(1.907,2.723)--(1.907,2.724)%
  --(1.907,2.725)--(1.907,2.726)--(1.908,2.726)--(1.908,2.727)--(1.908,2.728)--(1.909,2.728)%
  --(1.909,2.729)--(1.909,2.730)--(1.910,2.730)--(1.910,2.731)--(1.910,2.732)--(1.911,2.733)%
  --(1.911,2.734)--(1.911,2.735)--(1.912,2.735)--(1.912,2.736)--(1.912,2.737)--(1.913,2.737)%
  --(1.913,2.738)--(1.913,2.739)--(1.914,2.739)--(1.914,2.740)--(1.914,2.741)--(1.914,2.742)%
  --(1.915,2.742)--(1.915,2.743)--(1.915,2.744)--(1.916,2.744)--(1.916,2.745)--(1.916,2.746)%
  --(1.917,2.746)--(1.917,2.747)--(1.917,2.748)--(1.918,2.749)--(1.918,2.750)--(1.918,2.751)%
  --(1.919,2.751)--(1.919,2.752)--(1.919,2.753)--(1.920,2.753)--(1.920,2.754)--(1.920,2.755)%
  --(1.921,2.755)--(1.921,2.756)--(1.921,2.757)--(1.921,2.758)--(1.922,2.758)--(1.922,2.759)%
  --(1.922,2.760)--(1.923,2.760)--(1.923,2.761)--(1.923,2.762)--(1.924,2.762)--(1.924,2.763)%
  --(1.924,2.764)--(1.924,2.765)--(1.925,2.765)--(1.925,2.766)--(1.925,2.767)--(1.926,2.767)%
  --(1.926,2.768)--(1.926,2.769)--(1.927,2.769)--(1.927,2.770)--(1.927,2.771)--(1.928,2.771)%
  --(1.928,2.772)--(1.928,2.773)--(1.928,2.774)--(1.929,2.774)--(1.929,2.775)--(1.929,2.776)%
  --(1.930,2.776)--(1.930,2.777)--(1.930,2.778)--(1.931,2.778)--(1.931,2.779)--(1.931,2.780)%
  --(1.932,2.781)--(1.932,2.782)--(1.932,2.783)--(1.933,2.783)--(1.933,2.784)--(1.933,2.785)%
  --(1.934,2.785)--(1.934,2.786)--(1.934,2.787)--(1.935,2.787)--(1.935,2.788)--(1.935,2.789)%
  --(1.936,2.790)--(1.936,2.791)--(1.936,2.792)--(1.937,2.792)--(1.937,2.793)--(1.937,2.794)%
  --(1.938,2.794)--(1.938,2.795)--(1.938,2.796)--(1.939,2.796)--(1.939,2.797)--(1.939,2.798)%
  --(1.940,2.799)--(1.940,2.800)--(1.940,2.801)--(1.941,2.801)--(1.941,2.802)--(1.941,2.803)%
  --(1.942,2.803)--(1.942,2.804)--(1.942,2.805)--(1.943,2.805)--(1.943,2.806)--(1.943,2.807)%
  --(1.943,2.808)--(1.944,2.808)--(1.944,2.809)--(1.944,2.810)--(1.945,2.810)--(1.945,2.811)%
  --(1.945,2.812)--(1.946,2.812)--(1.946,2.813)--(1.946,2.814)--(1.947,2.815)--(1.947,2.816)%
  --(1.947,2.817)--(1.948,2.817)--(1.948,2.818)--(1.948,2.819)--(1.949,2.819)--(1.949,2.820)%
  --(1.949,2.821)--(1.950,2.821)--(1.950,2.822)--(1.950,2.823)--(1.951,2.824)--(1.951,2.825)%
  --(1.951,2.826)--(1.952,2.826)--(1.952,2.827)--(1.952,2.828)--(1.953,2.828)--(1.953,2.829)%
  --(1.953,2.830)--(1.954,2.830)--(1.954,2.831)--(1.954,2.832)--(1.955,2.833)--(1.955,2.834)%
  --(1.955,2.835)--(1.956,2.835)--(1.956,2.836)--(1.956,2.837)--(1.957,2.837)--(1.957,2.838)%
  --(1.957,2.839)--(1.958,2.839)--(1.958,2.840)--(1.958,2.841)--(1.959,2.841)--(1.959,2.842)%
  --(1.959,2.843)--(1.959,2.844)--(1.960,2.844)--(1.960,2.845)--(1.960,2.846)--(1.961,2.846)%
  --(1.961,2.847)--(1.961,2.848)--(1.962,2.848)--(1.962,2.849)--(1.962,2.850)--(1.963,2.851)%
  --(1.963,2.852)--(1.963,2.853)--(1.964,2.853)--(1.964,2.854)--(1.964,2.855)--(1.965,2.855)%
  --(1.965,2.856)--(1.965,2.857)--(1.966,2.857)--(1.966,2.858)--(1.966,2.859)--(1.967,2.859)%
  --(1.967,2.860)--(1.967,2.861)--(1.967,2.862)--(1.968,2.862)--(1.968,2.863)--(1.968,2.864)%
  --(1.969,2.864)--(1.969,2.865)--(1.969,2.866)--(1.970,2.866)--(1.970,2.867)--(1.970,2.868)%
  --(1.971,2.868)--(1.971,2.869)--(1.971,2.870)--(1.971,2.871)--(1.972,2.871)--(1.972,2.872)%
  --(1.972,2.873)--(1.973,2.873)--(1.973,2.874)--(1.973,2.875)--(1.974,2.875)--(1.974,2.876)%
  --(1.974,2.877)--(1.975,2.877)--(1.975,2.878)--(1.975,2.879)--(1.976,2.880)--(1.976,2.881)%
  --(1.976,2.882)--(1.977,2.882)--(1.977,2.883)--(1.977,2.884)--(1.978,2.884)--(1.978,2.885)%
  --(1.978,2.886)--(1.979,2.886)--(1.979,2.887)--(1.979,2.888)--(1.980,2.889)--(1.980,2.890)%
  --(1.980,2.891)--(1.981,2.891)--(1.981,2.892)--(1.981,2.893)--(1.982,2.893)--(1.982,2.894)%
  --(1.982,2.895)--(1.983,2.895)--(1.983,2.896)--(1.983,2.897)--(1.984,2.898)--(1.984,2.899)%
  --(1.984,2.900)--(1.985,2.900)--(1.985,2.901)--(1.985,2.902)--(1.986,2.902)--(1.986,2.903)%
  --(1.986,2.904)--(1.987,2.904)--(1.987,2.905)--(1.987,2.906)--(1.988,2.906)--(1.988,2.907)%
  --(1.988,2.908)--(1.989,2.909)--(1.989,2.910)--(1.989,2.911)--(1.990,2.911)--(1.990,2.912)%
  --(1.990,2.913)--(1.991,2.913)--(1.991,2.914)--(1.991,2.915)--(1.992,2.915)--(1.992,2.916)%
  --(1.992,2.917)--(1.993,2.917)--(1.993,2.918)--(1.993,2.919)--(1.993,2.920)--(1.994,2.920)%
  --(1.994,2.921)--(1.994,2.922)--(1.995,2.922)--(1.995,2.923)--(1.995,2.924)--(1.996,2.924)%
  --(1.996,2.925)--(1.996,2.926)--(1.997,2.926)--(1.997,2.927)--(1.997,2.928)--(1.998,2.928)%
  --(1.998,2.929)--(1.998,2.930)--(1.998,2.931)--(1.999,2.931)--(1.999,2.932)--(1.999,2.933)%
  --(2.000,2.933)--(2.000,2.934)--(2.000,2.935)--(2.001,2.935)--(2.001,2.936)--(2.001,2.937)%
  --(2.002,2.937)--(2.002,2.938)--(2.002,2.939)--(2.003,2.940)--(2.003,2.941)--(2.003,2.942)%
  --(2.004,2.942)--(2.004,2.943)--(2.004,2.944)--(2.005,2.944)--(2.005,2.945)--(2.005,2.946)%
  --(2.006,2.946)--(2.006,2.947)--(2.006,2.948)--(2.007,2.948)--(2.007,2.949)--(2.007,2.950)%
  --(2.008,2.951)--(2.008,2.952)--(2.008,2.953)--(2.009,2.953)--(2.009,2.954)--(2.009,2.955)%
  --(2.010,2.955)--(2.010,2.956)--(2.010,2.957)--(2.011,2.957)--(2.011,2.958)--(2.011,2.959)%
  --(2.012,2.959)--(2.012,2.960)--(2.012,2.961)--(2.013,2.962)--(2.013,2.963)--(2.013,2.964)%
  --(2.014,2.964)--(2.014,2.965)--(2.014,2.966)--(2.015,2.966)--(2.015,2.967)--(2.015,2.968)%
  --(2.016,2.968)--(2.016,2.969)--(2.016,2.970)--(2.017,2.970)--(2.017,2.971)--(2.017,2.972)%
  --(2.018,2.972)--(2.018,2.973)--(2.018,2.974)--(2.018,2.975)--(2.019,2.975)--(2.019,2.976)%
  --(2.019,2.977)--(2.020,2.977)--(2.020,2.978)--(2.020,2.979)--(2.021,2.979)--(2.021,2.980)%
  --(2.021,2.981)--(2.022,2.981)--(2.022,2.982)--(2.022,2.983)--(2.023,2.983)--(2.023,2.984)%
  --(2.023,2.985)--(2.023,2.986)--(2.024,2.986)--(2.024,2.987)--(2.024,2.988)--(2.025,2.988)%
  --(2.025,2.989)--(2.025,2.990)--(2.026,2.990)--(2.026,2.991)--(2.026,2.992)--(2.027,2.992)%
  --(2.027,2.993)--(2.027,2.994)--(2.028,2.994)--(2.028,2.995)--(2.028,2.996)--(2.029,2.997)%
  --(2.029,2.998)--(2.029,2.999)--(2.030,2.999)--(2.030,3.000)--(2.030,3.001)--(2.031,3.001)%
  --(2.031,3.002)--(2.031,3.003)--(2.032,3.003)--(2.032,3.004)--(2.032,3.005)--(2.033,3.005)%
  --(2.033,3.006)--(2.033,3.007)--(2.034,3.008)--(2.034,3.009)--(2.034,3.010)--(2.035,3.010)%
  --(2.035,3.011)--(2.035,3.012)--(2.036,3.012)--(2.036,3.013)--(2.036,3.014)--(2.037,3.014)%
  --(2.037,3.015)--(2.037,3.016)--(2.038,3.016)--(2.038,3.017)--(2.038,3.018)--(2.039,3.018)%
  --(2.039,3.019)--(2.039,3.020)--(2.040,3.021)--(2.040,3.022)--(2.040,3.023)--(2.041,3.023)%
  --(2.041,3.024)--(2.041,3.025)--(2.042,3.025)--(2.042,3.026)--(2.042,3.027)--(2.043,3.027)%
  --(2.043,3.028)--(2.043,3.029)--(2.044,3.029)--(2.044,3.030)--(2.044,3.031)--(2.045,3.031)%
  --(2.045,3.032)--(2.045,3.033)--(2.046,3.034)--(2.046,3.035)--(2.046,3.036)--(2.047,3.036)%
  --(2.047,3.037)--(2.047,3.038)--(2.048,3.038)--(2.048,3.039)--(2.048,3.040)--(2.049,3.040)%
  --(2.049,3.041)--(2.049,3.042)--(2.050,3.042)--(2.050,3.043)--(2.050,3.044)--(2.051,3.044)%
  --(2.051,3.045)--(2.051,3.046)--(2.052,3.047)--(2.052,3.048)--(2.052,3.049)--(2.053,3.049)%
  --(2.053,3.050)--(2.053,3.051)--(2.054,3.051)--(2.054,3.052)--(2.054,3.053)--(2.055,3.053)%
  --(2.055,3.054)--(2.055,3.055)--(2.056,3.055)--(2.056,3.056)--(2.056,3.057)--(2.057,3.057)%
  --(2.057,3.058)--(2.057,3.059)--(2.058,3.059)--(2.058,3.060)--(2.058,3.061)--(2.058,3.062)%
  --(2.059,3.062)--(2.059,3.063)--(2.059,3.064)--(2.060,3.064)--(2.060,3.065)--(2.060,3.066)%
  --(2.061,3.066)--(2.061,3.067)--(2.061,3.068)--(2.062,3.068)--(2.062,3.069)--(2.062,3.070)%
  --(2.063,3.070)--(2.063,3.071)--(2.063,3.072)--(2.064,3.072)--(2.064,3.073)--(2.064,3.074)%
  --(2.065,3.075)--(2.065,3.076)--(2.065,3.077)--(2.066,3.077)--(2.066,3.078)--(2.066,3.079)%
  --(2.067,3.079)--(2.067,3.080)--(2.067,3.081)--(2.068,3.081)--(2.068,3.082)--(2.068,3.083)%
  --(2.069,3.083)--(2.069,3.084)--(2.069,3.085)--(2.070,3.085)--(2.070,3.086)--(2.070,3.087)%
  --(2.071,3.087)--(2.071,3.088)--(2.071,3.089)--(2.072,3.090)--(2.072,3.091)--(2.072,3.092)%
  --(2.073,3.092)--(2.073,3.093)--(2.073,3.094)--(2.074,3.094)--(2.074,3.095)--(2.074,3.096)%
  --(2.075,3.096)--(2.075,3.097)--(2.075,3.098)--(2.076,3.098)--(2.076,3.099)--(2.076,3.100)%
  --(2.077,3.100)--(2.077,3.101)--(2.077,3.102)--(2.078,3.102)--(2.078,3.103)--(2.078,3.104)%
  --(2.079,3.105)--(2.079,3.106)--(2.080,3.107)--(2.080,3.108)--(2.080,3.109)--(2.081,3.109)%
  --(2.081,3.110)--(2.081,3.111)--(2.082,3.111)--(2.082,3.112)--(2.082,3.113)--(2.083,3.113)%
  --(2.083,3.114)--(2.083,3.115)--(2.084,3.115)--(2.084,3.116)--(2.084,3.117)--(2.085,3.117)%
  --(2.085,3.118)--(2.085,3.119)--(2.086,3.119)--(2.086,3.120)--(2.086,3.121)--(2.087,3.122)%
  --(2.087,3.123)--(2.087,3.124)--(2.088,3.124)--(2.088,3.125)--(2.088,3.126)--(2.089,3.126)%
  --(2.089,3.127)--(2.089,3.128)--(2.090,3.128)--(2.090,3.129)--(2.090,3.130)--(2.091,3.130)%
  --(2.091,3.131)--(2.091,3.132)--(2.092,3.132)--(2.092,3.133)--(2.092,3.134)--(2.093,3.134)%
  --(2.093,3.135)--(2.093,3.136)--(2.094,3.136)--(2.094,3.137)--(2.094,3.138)--(2.095,3.139)%
  --(2.095,3.140)--(2.096,3.141)--(2.096,3.142)--(2.096,3.143)--(2.097,3.143)--(2.097,3.144)%
  --(2.097,3.145)--(2.098,3.145)--(2.098,3.146)--(2.098,3.147)--(2.099,3.147)--(2.099,3.148)%
  --(2.099,3.149)--(2.100,3.149)--(2.100,3.150)--(2.100,3.151)--(2.101,3.151)--(2.101,3.152)%
  --(2.101,3.153)--(2.102,3.153)--(2.102,3.154)--(2.102,3.155)--(2.103,3.155)--(2.103,3.156)%
  --(2.103,3.157)--(2.104,3.158)--(2.104,3.159)--(2.104,3.160)--(2.105,3.160)--(2.105,3.161)%
  --(2.105,3.162)--(2.106,3.162)--(2.106,3.163)--(2.106,3.164)--(2.107,3.164)--(2.107,3.165)%
  --(2.107,3.166)--(2.108,3.166)--(2.108,3.167)--(2.108,3.168)--(2.109,3.168)--(2.109,3.169)%
  --(2.109,3.170)--(2.110,3.170)--(2.110,3.171)--(2.110,3.172)--(2.111,3.172)--(2.111,3.173)%
  --(2.111,3.174)--(2.112,3.174)--(2.112,3.175)--(2.112,3.176)--(2.113,3.177)--(2.113,3.178)%
  --(2.114,3.179)--(2.114,3.180)--(2.114,3.181)--(2.115,3.181)--(2.115,3.182)--(2.115,3.183)%
  --(2.116,3.183)--(2.116,3.184)--(2.116,3.185)--(2.117,3.185)--(2.117,3.186)--(2.117,3.187)%
  --(2.118,3.187)--(2.118,3.188)--(2.118,3.189)--(2.119,3.189)--(2.119,3.190)--(2.119,3.191)%
  --(2.120,3.191)--(2.120,3.192)--(2.120,3.193)--(2.121,3.193)--(2.121,3.194)--(2.121,3.195)%
  --(2.122,3.195)--(2.122,3.196)--(2.122,3.197)--(2.123,3.197)--(2.123,3.198)--(2.123,3.199)%
  --(2.124,3.199)--(2.124,3.200)--(2.124,3.201)--(2.124,3.202)--(2.125,3.202)--(2.125,3.203)%
  --(2.125,3.204)--(2.126,3.204)--(2.126,3.205)--(2.126,3.206)--(2.127,3.206)--(2.127,3.207)%
  --(2.127,3.208)--(2.128,3.208)--(2.128,3.209)--(2.128,3.210)--(2.129,3.210)--(2.129,3.211)%
  --(2.129,3.212)--(2.130,3.212)--(2.130,3.213)--(2.130,3.214)--(2.131,3.214)--(2.131,3.215)%
  --(2.131,3.216)--(2.132,3.216)--(2.132,3.217)--(2.132,3.218)--(2.133,3.218)--(2.133,3.219)%
  --(2.133,3.220)--(2.134,3.220)--(2.134,3.221)--(2.134,3.222)--(2.135,3.222)--(2.135,3.223)%
  --(2.135,3.224)--(2.136,3.224)--(2.136,3.225)--(2.136,3.226)--(2.136,3.227)--(2.137,3.227)%
  --(2.137,3.228)--(2.137,3.229)--(2.138,3.229)--(2.138,3.230)--(2.138,3.231)--(2.139,3.231)%
  --(2.139,3.232)--(2.139,3.233)--(2.140,3.233)--(2.140,3.234)--(2.140,3.235)--(2.141,3.235)%
  --(2.141,3.236)--(2.141,3.237)--(2.142,3.237)--(2.142,3.238)--(2.142,3.239)--(2.143,3.239)%
  --(2.143,3.240)--(2.143,3.241)--(2.144,3.241)--(2.144,3.242)--(2.144,3.243)--(2.145,3.243)%
  --(2.145,3.244)--(2.145,3.245)--(2.146,3.245)--(2.146,3.246)--(2.146,3.247)--(2.147,3.247)%
  --(2.147,3.248)--(2.147,3.249)--(2.148,3.249)--(2.148,3.250)--(2.148,3.251)--(2.149,3.251)%
  --(2.149,3.252)--(2.149,3.253)--(2.150,3.254)--(2.150,3.255)--(2.151,3.256)--(2.151,3.257)%
  --(2.152,3.258)--(2.152,3.259)--(2.152,3.260)--(2.153,3.260)--(2.153,3.261)--(2.153,3.262)%
  --(2.154,3.262)--(2.154,3.263)--(2.154,3.264)--(2.155,3.264)--(2.155,3.265)--(2.155,3.266)%
  --(2.156,3.266)--(2.156,3.267)--(2.156,3.268)--(2.157,3.268)--(2.157,3.269)--(2.157,3.270)%
  --(2.158,3.270)--(2.158,3.271)--(2.158,3.272)--(2.159,3.272)--(2.159,3.273)--(2.159,3.274)%
  --(2.160,3.274)--(2.160,3.275)--(2.160,3.276)--(2.161,3.276)--(2.161,3.277)--(2.161,3.278)%
  --(2.162,3.278)--(2.162,3.279)--(2.162,3.280)--(2.163,3.280)--(2.163,3.281)--(2.163,3.282)%
  --(2.164,3.282)--(2.164,3.283)--(2.164,3.284)--(2.165,3.284)--(2.165,3.285)--(2.165,3.286)%
  --(2.166,3.286)--(2.166,3.287)--(2.166,3.288)--(2.167,3.288)--(2.167,3.289)--(2.167,3.290)%
  --(2.168,3.290)--(2.168,3.291)--(2.168,3.292)--(2.169,3.292)--(2.169,3.293)--(2.169,3.294)%
  --(2.170,3.294)--(2.170,3.295)--(2.170,3.296)--(2.171,3.297)--(2.171,3.298)--(2.172,3.299)%
  --(2.172,3.300)--(2.173,3.301)--(2.173,3.302)--(2.173,3.303)--(2.174,3.303)--(2.174,3.304)%
  --(2.174,3.305)--(2.175,3.305)--(2.175,3.306)--(2.175,3.307)--(2.176,3.307)--(2.176,3.308)%
  --(2.176,3.309)--(2.177,3.309)--(2.177,3.310)--(2.177,3.311)--(2.178,3.311)--(2.178,3.312)%
  --(2.178,3.313)--(2.179,3.313)--(2.179,3.314)--(2.179,3.315)--(2.180,3.315)--(2.180,3.316)%
  --(2.180,3.317)--(2.181,3.317)--(2.181,3.318)--(2.181,3.319)--(2.182,3.319)--(2.182,3.320)%
  --(2.182,3.321)--(2.183,3.321)--(2.183,3.322)--(2.183,3.323)--(2.184,3.323)--(2.184,3.324)%
  --(2.184,3.325)--(2.185,3.325)--(2.185,3.326)--(2.185,3.327)--(2.186,3.327)--(2.186,3.328)%
  --(2.186,3.329)--(2.187,3.329)--(2.187,3.330)--(2.187,3.331)--(2.188,3.331)--(2.188,3.332)%
  --(2.188,3.333)--(2.189,3.333)--(2.189,3.334)--(2.189,3.335)--(2.190,3.335)--(2.190,3.336)%
  --(2.190,3.337)--(2.191,3.337)--(2.191,3.338)--(2.191,3.339)--(2.192,3.339)--(2.192,3.340)%
  --(2.192,3.341)--(2.193,3.341)--(2.193,3.342)--(2.193,3.343)--(2.194,3.343)--(2.194,3.344)%
  --(2.194,3.345)--(2.195,3.345)--(2.195,3.346)--(2.195,3.347)--(2.196,3.347)--(2.196,3.348)%
  --(2.196,3.349)--(2.197,3.349)--(2.197,3.350)--(2.197,3.351)--(2.198,3.351)--(2.198,3.352)%
  --(2.198,3.353)--(2.199,3.353)--(2.199,3.354)--(2.199,3.355)--(2.200,3.355)--(2.200,3.356)%
  --(2.200,3.357)--(2.201,3.357)--(2.201,3.358)--(2.201,3.359)--(2.202,3.359)--(2.202,3.360)%
  --(2.202,3.361)--(2.203,3.361)--(2.203,3.362)--(2.203,3.363)--(2.204,3.363)--(2.204,3.364)%
  --(2.204,3.365)--(2.205,3.365)--(2.205,3.366)--(2.205,3.367)--(2.206,3.367)--(2.206,3.368)%
  --(2.206,3.369)--(2.207,3.369)--(2.207,3.370)--(2.207,3.371)--(2.208,3.371)--(2.208,3.372)%
  --(2.208,3.373)--(2.209,3.373)--(2.209,3.374)--(2.209,3.375)--(2.210,3.375)--(2.210,3.376)%
  --(2.210,3.377)--(2.211,3.377)--(2.211,3.378)--(2.211,3.379)--(2.212,3.379)--(2.212,3.380)%
  --(2.212,3.381)--(2.213,3.381)--(2.213,3.382)--(2.213,3.383)--(2.214,3.383)--(2.214,3.384)%
  --(2.214,3.385)--(2.215,3.385)--(2.215,3.386)--(2.215,3.387)--(2.216,3.387)--(2.216,3.388)%
  --(2.216,3.389)--(2.217,3.389)--(2.217,3.390)--(2.217,3.391)--(2.218,3.391)--(2.218,3.392)%
  --(2.218,3.393)--(2.219,3.393)--(2.219,3.394)--(2.219,3.395)--(2.220,3.395)--(2.220,3.396)%
  --(2.220,3.397)--(2.221,3.397)--(2.221,3.398)--(2.221,3.399)--(2.222,3.399)--(2.222,3.400)%
  --(2.222,3.401)--(2.223,3.401)--(2.223,3.402)--(2.223,3.403)--(2.224,3.403)--(2.224,3.404)%
  --(2.224,3.405)--(2.225,3.405)--(2.225,3.406)--(2.225,3.407)--(2.226,3.407)--(2.226,3.408)%
  --(2.226,3.409)--(2.227,3.409)--(2.227,3.410)--(2.227,3.411)--(2.228,3.411)--(2.228,3.412)%
  --(2.228,3.413)--(2.229,3.413)--(2.229,3.414)--(2.229,3.415)--(2.230,3.415)--(2.230,3.416)%
  --(2.230,3.417)--(2.231,3.417)--(2.231,3.418)--(2.231,3.419)--(2.232,3.419)--(2.232,3.420)%
  --(2.232,3.421)--(2.233,3.421)--(2.233,3.422)--(2.233,3.423)--(2.234,3.423)--(2.234,3.424)%
  --(2.234,3.425)--(2.235,3.425)--(2.235,3.426)--(2.235,3.427)--(2.236,3.427)--(2.236,3.428)%
  --(2.236,3.429)--(2.237,3.429)--(2.237,3.430)--(2.237,3.431)--(2.238,3.431)--(2.238,3.432)%
  --(2.238,3.433)--(2.239,3.433)--(2.239,3.434)--(2.239,3.435)--(2.240,3.435)--(2.240,3.436)%
  --(2.240,3.437)--(2.241,3.437)--(2.241,3.438)--(2.241,3.439)--(2.242,3.439)--(2.242,3.440)%
  --(2.242,3.441)--(2.243,3.441)--(2.243,3.442)--(2.243,3.443)--(2.244,3.443)--(2.244,3.444)%
  --(2.245,3.445)--(2.245,3.446)--(2.245,3.447)--(2.246,3.447)--(2.246,3.448)--(2.247,3.449)%
  --(2.247,3.450)--(2.248,3.450)--(2.248,3.451)--(2.248,3.452)--(2.249,3.453)--(2.249,3.454)%
  --(2.250,3.454)--(2.250,3.455)--(2.250,3.456)--(2.251,3.456)--(2.251,3.457)--(2.251,3.458)%
  --(2.252,3.458)--(2.252,3.459)--(2.252,3.460)--(2.253,3.460)--(2.253,3.461)--(2.253,3.462)%
  --(2.254,3.462)--(2.254,3.463)--(2.254,3.464)--(2.255,3.464)--(2.255,3.465)--(2.255,3.466)%
  --(2.256,3.466)--(2.256,3.467)--(2.256,3.468)--(2.257,3.468)--(2.257,3.469)--(2.257,3.470)%
  --(2.258,3.470)--(2.258,3.471)--(2.258,3.472)--(2.259,3.472)--(2.259,3.473)--(2.259,3.474)%
  --(2.260,3.474)--(2.260,3.475)--(2.260,3.476)--(2.261,3.476)--(2.261,3.477)--(2.261,3.478)%
  --(2.262,3.478)--(2.262,3.479)--(2.262,3.480)--(2.263,3.480)--(2.263,3.481)--(2.263,3.482)%
  --(2.264,3.482)--(2.264,3.483)--(2.264,3.484)--(2.265,3.484)--(2.265,3.485)--(2.266,3.486)%
  --(2.266,3.487)--(2.267,3.487)--(2.267,3.488)--(2.267,3.489)--(2.268,3.490)--(2.268,3.491)%
  --(2.269,3.491)--(2.269,3.492)--(2.269,3.493)--(2.270,3.493)--(2.270,3.494)--(2.270,3.495)%
  --(2.271,3.495)--(2.271,3.496)--(2.271,3.497)--(2.272,3.497)--(2.272,3.498)--(2.272,3.499)%
  --(2.273,3.499)--(2.273,3.500)--(2.273,3.501)--(2.274,3.501)--(2.274,3.502)--(2.274,3.503)%
  --(2.275,3.503)--(2.275,3.504)--(2.275,3.505)--(2.276,3.505)--(2.276,3.506)--(2.276,3.507)%
  --(2.277,3.507)--(2.277,3.508)--(2.277,3.509)--(2.278,3.509)--(2.278,3.510)--(2.278,3.511)%
  --(2.279,3.511)--(2.279,3.512)--(2.279,3.513)--(2.280,3.513)--(2.280,3.514)--(2.281,3.515)%
  --(2.281,3.516)--(2.282,3.516)--(2.282,3.517)--(2.282,3.518)--(2.283,3.518)--(2.283,3.519)%
  --(2.283,3.520)--(2.284,3.520)--(2.284,3.521)--(2.284,3.522)--(2.285,3.522)--(2.285,3.523)%
  --(2.285,3.524)--(2.286,3.524)--(2.286,3.525)--(2.286,3.526)--(2.287,3.526)--(2.287,3.527)%
  --(2.287,3.528)--(2.288,3.528)--(2.288,3.529)--(2.288,3.530)--(2.289,3.530)--(2.289,3.531)%
  --(2.289,3.532)--(2.290,3.532)--(2.290,3.533)--(2.290,3.534)--(2.291,3.534)--(2.291,3.535)%
  --(2.292,3.536)--(2.292,3.537)--(2.293,3.538)--(2.293,3.539)--(2.294,3.539)--(2.294,3.540)%
  --(2.294,3.541)--(2.295,3.541)--(2.295,3.542)--(2.295,3.543)--(2.296,3.543)--(2.296,3.544)%
  --(2.296,3.545)--(2.297,3.545)--(2.297,3.546)--(2.297,3.547)--(2.298,3.547)--(2.298,3.548)%
  --(2.298,3.549)--(2.299,3.549)--(2.299,3.550)--(2.299,3.551)--(2.300,3.551)--(2.300,3.552)%
  --(2.300,3.553)--(2.301,3.553)--(2.301,3.554)--(2.301,3.555)--(2.302,3.555)--(2.302,3.556)%
  --(2.303,3.557)--(2.303,3.558)--(2.304,3.558)--(2.304,3.559)--(2.304,3.560)--(2.305,3.560)%
  --(2.305,3.561)--(2.305,3.562)--(2.306,3.562)--(2.306,3.563)--(2.306,3.564)--(2.307,3.564)%
  --(2.307,3.565)--(2.307,3.566)--(2.308,3.566)--(2.308,3.567)--(2.308,3.568)--(2.309,3.568)%
  --(2.309,3.569)--(2.309,3.570)--(2.310,3.570)--(2.310,3.571)--(2.310,3.572)--(2.311,3.572)%
  --(2.311,3.573)--(2.311,3.574)--(2.312,3.574)--(2.312,3.575)--(2.313,3.576)--(2.313,3.577)%
  --(2.314,3.577)--(2.314,3.578)--(2.314,3.579)--(2.315,3.579)--(2.315,3.580)--(2.315,3.581)%
  --(2.316,3.581)--(2.316,3.582)--(2.316,3.583)--(2.317,3.583)--(2.317,3.584)--(2.317,3.585)%
  --(2.318,3.585)--(2.318,3.586)--(2.318,3.587)--(2.319,3.587)--(2.319,3.588)--(2.319,3.589)%
  --(2.320,3.589)--(2.320,3.590)--(2.320,3.591)--(2.321,3.591)--(2.321,3.592)--(2.322,3.592)%
  --(2.322,3.593)--(2.322,3.594)--(2.323,3.594)--(2.323,3.595)--(2.323,3.596)--(2.324,3.596)%
  --(2.324,3.597)--(2.324,3.598)--(2.325,3.598)--(2.325,3.599)--(2.325,3.600)--(2.326,3.600)%
  --(2.326,3.601)--(2.326,3.602)--(2.327,3.602)--(2.327,3.603)--(2.327,3.604)--(2.328,3.604)%
  --(2.328,3.605)--(2.329,3.606)--(2.329,3.607)--(2.330,3.608)--(2.330,3.609)--(2.331,3.609)%
  --(2.331,3.610)--(2.331,3.611)--(2.332,3.611)--(2.332,3.612)--(2.332,3.613)--(2.333,3.613)%
  --(2.333,3.614)--(2.333,3.615)--(2.334,3.615)--(2.334,3.616)--(2.334,3.617)--(2.335,3.617)%
  --(2.335,3.618)--(2.335,3.619)--(2.336,3.619)--(2.336,3.620)--(2.337,3.621)--(2.337,3.622)%
  --(2.338,3.622)--(2.338,3.623)--(2.338,3.624)--(2.339,3.624)--(2.339,3.625)--(2.339,3.626)%
  --(2.340,3.626)--(2.340,3.627)--(2.340,3.628)--(2.341,3.628)--(2.341,3.629)--(2.341,3.630)%
  --(2.342,3.630)--(2.342,3.631)--(2.342,3.632)--(2.343,3.632)--(2.343,3.633)--(2.344,3.634)%
  --(2.344,3.635)--(2.345,3.635)--(2.345,3.636)--(2.345,3.637)--(2.346,3.637)--(2.346,3.638)%
  --(2.346,3.639)--(2.347,3.639)--(2.347,3.640)--(2.347,3.641)--(2.348,3.641)--(2.348,3.642)%
  --(2.348,3.643)--(2.349,3.643)--(2.349,3.644)--(2.349,3.645)--(2.350,3.645)--(2.350,3.646)%
  --(2.351,3.646)--(2.351,3.647)--(2.351,3.648)--(2.352,3.648)--(2.352,3.649)--(2.352,3.650)%
  --(2.353,3.650)--(2.353,3.651)--(2.353,3.652)--(2.354,3.652)--(2.354,3.653)--(2.354,3.654)%
  --(2.355,3.654)--(2.355,3.655)--(2.355,3.656)--(2.356,3.656)--(2.356,3.657)--(2.357,3.658)%
  --(2.357,3.659)--(2.358,3.659)--(2.358,3.660)--(2.358,3.661)--(2.359,3.661)--(2.359,3.662)%
  --(2.359,3.663)--(2.360,3.663)--(2.360,3.664)--(2.360,3.665)--(2.361,3.665)--(2.361,3.666)%
  --(2.361,3.667)--(2.362,3.667)--(2.362,3.668)--(2.363,3.669)--(2.363,3.670)--(2.364,3.670)%
  --(2.364,3.671)--(2.364,3.672)--(2.365,3.672)--(2.365,3.673)--(2.365,3.674)--(2.366,3.674)%
  --(2.366,3.675)--(2.366,3.676)--(2.367,3.676)--(2.367,3.677)--(2.367,3.678)--(2.368,3.678)%
  --(2.368,3.679)--(2.369,3.680)--(2.369,3.681)--(2.370,3.681)--(2.370,3.682)--(2.370,3.683)%
  --(2.371,3.683)--(2.371,3.684)--(2.371,3.685)--(2.372,3.685)--(2.372,3.686)--(2.372,3.687)%
  --(2.373,3.687)--(2.373,3.688)--(2.373,3.689)--(2.374,3.689)--(2.374,3.690)--(2.375,3.690)%
  --(2.375,3.691)--(2.375,3.692)--(2.376,3.692)--(2.376,3.693)--(2.376,3.694)--(2.377,3.694)%
  --(2.377,3.695)--(2.377,3.696)--(2.378,3.696)--(2.378,3.697)--(2.378,3.698)--(2.379,3.698)%
  --(2.379,3.699)--(2.379,3.700)--(2.380,3.700)--(2.380,3.701)--(2.381,3.701)--(2.381,3.702)%
  --(2.381,3.703)--(2.382,3.703)--(2.382,3.704)--(2.382,3.705)--(2.383,3.705)--(2.383,3.706)%
  --(2.383,3.707)--(2.384,3.707)--(2.384,3.708)--(2.385,3.709)--(2.385,3.710)--(2.386,3.710)%
  --(2.386,3.711)--(2.386,3.712)--(2.387,3.712)--(2.387,3.713)--(2.387,3.714)--(2.388,3.714)%
  --(2.388,3.715)--(2.388,3.716)--(2.389,3.716)--(2.389,3.717)--(2.389,3.718)--(2.390,3.718)%
  --(2.390,3.719)--(2.391,3.719)--(2.391,3.720)--(2.391,3.721)--(2.392,3.721)--(2.392,3.722)%
  --(2.392,3.723)--(2.393,3.723)--(2.393,3.724)--(2.393,3.725)--(2.394,3.725)--(2.394,3.726)%
  --(2.394,3.727)--(2.395,3.727)--(2.395,3.728)--(2.396,3.728)--(2.396,3.729)--(2.396,3.730)%
  --(2.397,3.730)--(2.397,3.731)--(2.397,3.732)--(2.398,3.732)--(2.398,3.733)--(2.398,3.734)%
  --(2.399,3.734)--(2.399,3.735)--(2.399,3.736)--(2.400,3.736)--(2.400,3.737)--(2.401,3.737)%
  --(2.401,3.738)--(2.401,3.739)--(2.402,3.739)--(2.402,3.740)--(2.402,3.741)--(2.403,3.741)%
  --(2.403,3.742)--(2.403,3.743)--(2.404,3.743)--(2.404,3.744)--(2.404,3.745)--(2.405,3.745)%
  --(2.405,3.746)--(2.406,3.746)--(2.406,3.747)--(2.406,3.748)--(2.407,3.748)--(2.407,3.749)%
  --(2.407,3.750)--(2.408,3.750)--(2.408,3.751)--(2.408,3.752)--(2.409,3.752)--(2.409,3.753)%
  --(2.410,3.753)--(2.410,3.754)--(2.410,3.755)--(2.411,3.755)--(2.411,3.756)--(2.411,3.757)%
  --(2.412,3.757)--(2.412,3.758)--(2.412,3.759)--(2.413,3.759)--(2.413,3.760)--(2.413,3.761)%
  --(2.414,3.761)--(2.414,3.762)--(2.415,3.762)--(2.415,3.763)--(2.415,3.764)--(2.416,3.764)%
  --(2.416,3.765)--(2.416,3.766)--(2.417,3.766)--(2.417,3.767)--(2.417,3.768)--(2.418,3.768)%
  --(2.418,3.769)--(2.419,3.769)--(2.419,3.770)--(2.419,3.771)--(2.420,3.771)--(2.420,3.772)%
  --(2.420,3.773)--(2.421,3.773)--(2.421,3.774)--(2.421,3.775)--(2.422,3.775)--(2.422,3.776)%
  --(2.423,3.777)--(2.423,3.778)--(2.424,3.778)--(2.424,3.779)--(2.424,3.780)--(2.425,3.780)%
  --(2.425,3.781)--(2.425,3.782)--(2.426,3.782)--(2.426,3.783)--(2.426,3.784)--(2.427,3.784)%
  --(2.427,3.785)--(2.428,3.785)--(2.428,3.786)--(2.428,3.787)--(2.429,3.787)--(2.429,3.788)%
  --(2.429,3.789)--(2.430,3.789)--(2.430,3.790)--(2.430,3.791)--(2.431,3.791)--(2.431,3.792)%
  --(2.432,3.792)--(2.432,3.793)--(2.432,3.794)--(2.433,3.794)--(2.433,3.795)--(2.433,3.796)%
  --(2.434,3.796)--(2.434,3.797)--(2.434,3.798)--(2.435,3.798)--(2.435,3.799)--(2.436,3.800)%
  --(2.436,3.801)--(2.437,3.801)--(2.437,3.802)--(2.437,3.803)--(2.438,3.803)--(2.438,3.804)%
  --(2.438,3.805)--(2.439,3.805)--(2.439,3.806)--(2.440,3.806)--(2.440,3.807)--(2.440,3.808)%
  --(2.441,3.808)--(2.441,3.809)--(2.441,3.810)--(2.442,3.810)--(2.442,3.811)--(2.442,3.812)%
  --(2.443,3.812)--(2.443,3.813)--(2.444,3.813)--(2.444,3.814)--(2.444,3.815)--(2.445,3.815)%
  --(2.445,3.816)--(2.445,3.817)--(2.446,3.817)--(2.446,3.818)--(2.446,3.819)--(2.447,3.819)%
  --(2.447,3.820)--(2.448,3.820)--(2.448,3.821)--(2.448,3.822)--(2.449,3.822)--(2.449,3.823)%
  --(2.449,3.824)--(2.450,3.824)--(2.450,3.825)--(2.451,3.826)--(2.451,3.827)--(2.452,3.827)%
  --(2.452,3.828)--(2.452,3.829)--(2.453,3.829)--(2.453,3.830)--(2.453,3.831)--(2.454,3.831)%
  --(2.454,3.832)--(2.454,3.833)--(2.455,3.833)--(2.455,3.834)--(2.456,3.834)--(2.456,3.835)%
  --(2.456,3.836)--(2.457,3.836)--(2.457,3.837)--(2.457,3.838)--(2.458,3.838)--(2.458,3.839)%
  --(2.459,3.840)--(2.459,3.841)--(2.460,3.841)--(2.460,3.842)--(2.460,3.843)--(2.461,3.843)%
  --(2.461,3.844)--(2.461,3.845)--(2.462,3.845)--(2.462,3.846)--(2.463,3.846)--(2.463,3.847)%
  --(2.463,3.848)--(2.464,3.848)--(2.464,3.849)--(2.464,3.850)--(2.465,3.850)--(2.465,3.851)%
  --(2.466,3.852)--(2.466,3.853)--(2.467,3.853)--(2.467,3.854)--(2.467,3.855)--(2.468,3.855)%
  --(2.468,3.856)--(2.468,3.857)--(2.469,3.857)--(2.469,3.858)--(2.470,3.858)--(2.470,3.859)%
  --(2.470,3.860)--(2.471,3.860)--(2.471,3.861)--(2.471,3.862)--(2.472,3.862)--(2.472,3.863)%
  --(2.473,3.864)--(2.473,3.865)--(2.474,3.865)--(2.474,3.866)--(2.474,3.867)--(2.475,3.867)%
  --(2.475,3.868)--(2.475,3.869)--(2.476,3.869)--(2.476,3.870)--(2.477,3.870)--(2.477,3.871)%
  --(2.477,3.872)--(2.478,3.872)--(2.478,3.873)--(2.478,3.874)--(2.479,3.874)--(2.479,3.875)%
  --(2.480,3.876)--(2.480,3.877)--(2.481,3.877)--(2.481,3.878)--(2.481,3.879)--(2.482,3.879)%
  --(2.482,3.880)--(2.482,3.881)--(2.483,3.881)--(2.483,3.882)--(2.484,3.882)--(2.484,3.883)%
  --(2.484,3.884)--(2.485,3.884)--(2.485,3.885)--(2.485,3.886)--(2.486,3.886)--(2.486,3.887)%
  --(2.487,3.887)--(2.487,3.888)--(2.487,3.889)--(2.488,3.889)--(2.488,3.890)--(2.488,3.891)%
  --(2.489,3.891)--(2.489,3.892)--(2.490,3.893)--(2.490,3.894)--(2.491,3.894)--(2.491,3.895)%
  --(2.491,3.896)--(2.492,3.896)--(2.492,3.897)--(2.492,3.898)--(2.493,3.898)--(2.493,3.899)%
  --(2.494,3.899)--(2.494,3.900)--(2.494,3.901)--(2.495,3.901)--(2.495,3.902)--(2.495,3.903)%
  --(2.496,3.903)--(2.496,3.904)--(2.497,3.904)--(2.497,3.905)--(2.497,3.906)--(2.498,3.906)%
  --(2.498,3.907)--(2.498,3.908)--(2.499,3.908)--(2.499,3.909)--(2.500,3.909)--(2.500,3.910)%
  --(2.500,3.911)--(2.501,3.911)--(2.501,3.912)--(2.501,3.913)--(2.502,3.913)--(2.502,3.914)%
  --(2.503,3.914)--(2.503,3.915)--(2.503,3.916)--(2.504,3.916)--(2.504,3.917)--(2.504,3.918)%
  --(2.505,3.918)--(2.505,3.919)--(2.506,3.919)--(2.506,3.920)--(2.506,3.921)--(2.507,3.921)%
  --(2.507,3.922)--(2.507,3.923)--(2.508,3.923)--(2.508,3.924)--(2.509,3.925)--(2.509,3.926)%
  --(2.510,3.926)--(2.510,3.927)--(2.510,3.928)--(2.511,3.928)--(2.511,3.929)--(2.512,3.930)%
  --(2.512,3.931)--(2.513,3.931)--(2.513,3.932)--(2.513,3.933)--(2.514,3.933)--(2.514,3.934)%
  --(2.515,3.935)--(2.515,3.936)--(2.516,3.936)--(2.516,3.937)--(2.516,3.938)--(2.517,3.938)%
  --(2.517,3.939)--(2.518,3.940)--(2.518,3.941)--(2.519,3.941)--(2.519,3.942)--(2.519,3.943)%
  --(2.520,3.943)--(2.520,3.944)--(2.521,3.945)--(2.521,3.946)--(2.522,3.946)--(2.522,3.947)%
  --(2.522,3.948)--(2.523,3.948)--(2.523,3.949)--(2.524,3.950)--(2.524,3.951)--(2.525,3.951)%
  --(2.525,3.952)--(2.525,3.953)--(2.526,3.953)--(2.526,3.954)--(2.527,3.954)--(2.527,3.955)%
  --(2.527,3.956)--(2.528,3.956)--(2.528,3.957)--(2.528,3.958)--(2.529,3.958)--(2.529,3.959)%
  --(2.530,3.959)--(2.530,3.960)--(2.530,3.961)--(2.531,3.961)--(2.531,3.962)--(2.531,3.963)%
  --(2.532,3.963)--(2.532,3.964)--(2.533,3.964)--(2.533,3.965)--(2.533,3.966)--(2.534,3.966)%
  --(2.534,3.967)--(2.534,3.968)--(2.535,3.968)--(2.535,3.969)--(2.536,3.969)--(2.536,3.970)%
  --(2.536,3.971)--(2.537,3.971)--(2.537,3.972)--(2.538,3.973)--(2.538,3.974)--(2.539,3.974)%
  --(2.539,3.975)--(2.539,3.976)--(2.540,3.976)--(2.540,3.977)--(2.541,3.977)--(2.541,3.978)%
  --(2.541,3.979)--(2.542,3.979)--(2.542,3.980)--(2.542,3.981)--(2.543,3.981)--(2.543,3.982)%
  --(2.544,3.982)--(2.544,3.983)--(2.544,3.984)--(2.545,3.984)--(2.545,3.985)--(2.545,3.986)%
  --(2.546,3.986)--(2.546,3.987)--(2.547,3.987)--(2.547,3.988)--(2.547,3.989)--(2.548,3.989)%
  --(2.548,3.990)--(2.549,3.991)--(2.549,3.992)--(2.550,3.992)--(2.550,3.993)--(2.550,3.994)%
  --(2.551,3.994)--(2.551,3.995)--(2.552,3.995)--(2.552,3.996)--(2.552,3.997)--(2.553,3.997)%
  --(2.553,3.998)--(2.553,3.999)--(2.554,3.999)--(2.554,4.000)--(2.555,4.000)--(2.555,4.001)%
  --(2.555,4.002)--(2.556,4.002)--(2.556,4.003)--(2.557,4.004)--(2.557,4.005)--(2.558,4.005)%
  --(2.558,4.006)--(2.558,4.007)--(2.559,4.007)--(2.559,4.008)--(2.560,4.008)--(2.560,4.009)%
  --(2.560,4.010)--(2.561,4.010)--(2.561,4.011)--(2.561,4.012)--(2.562,4.012)--(2.562,4.013)%
  --(2.563,4.013)--(2.563,4.014)--(2.563,4.015)--(2.564,4.015)--(2.564,4.016)--(2.565,4.016)%
  --(2.565,4.017)--(2.565,4.018)--(2.566,4.018)--(2.566,4.019)--(2.566,4.020)--(2.567,4.020)%
  --(2.567,4.021)--(2.568,4.021)--(2.568,4.022)--(2.568,4.023)--(2.569,4.023)--(2.569,4.024)%
  --(2.570,4.025)--(2.570,4.026)--(2.571,4.026)--(2.571,4.027)--(2.571,4.028)--(2.572,4.028)%
  --(2.572,4.029)--(2.573,4.029)--(2.573,4.030)--(2.573,4.031)--(2.574,4.031)--(2.574,4.032)%
  --(2.575,4.033)--(2.575,4.034)--(2.576,4.034)--(2.576,4.035)--(2.576,4.036)--(2.577,4.036)%
  --(2.577,4.037)--(2.578,4.037)--(2.578,4.038)--(2.578,4.039)--(2.579,4.039)--(2.579,4.040)%
  --(2.580,4.041)--(2.580,4.042)--(2.581,4.042)--(2.581,4.043)--(2.581,4.044)--(2.582,4.044)%
  --(2.582,4.045)--(2.583,4.045)--(2.583,4.046)--(2.583,4.047)--(2.584,4.047)--(2.584,4.048)%
  --(2.585,4.048)--(2.585,4.049)--(2.585,4.050)--(2.586,4.050)--(2.586,4.051)--(2.586,4.052)%
  --(2.587,4.052)--(2.587,4.053)--(2.588,4.053)--(2.588,4.054)--(2.588,4.055)--(2.589,4.055)%
  --(2.589,4.056)--(2.590,4.056)--(2.590,4.057)--(2.590,4.058)--(2.591,4.058)--(2.591,4.059)%
  --(2.591,4.060)--(2.592,4.060)--(2.592,4.061)--(2.593,4.061)--(2.593,4.062)--(2.593,4.063)%
  --(2.594,4.063)--(2.594,4.064)--(2.595,4.064)--(2.595,4.065)--(2.595,4.066)--(2.596,4.066)%
  --(2.596,4.067)--(2.597,4.067)--(2.597,4.068)--(2.597,4.069)--(2.598,4.069)--(2.598,4.070)%
  --(2.598,4.071)--(2.599,4.071)--(2.599,4.072)--(2.600,4.072)--(2.600,4.073)--(2.600,4.074)%
  --(2.601,4.074)--(2.601,4.075)--(2.602,4.075)--(2.602,4.076)--(2.602,4.077)--(2.603,4.077)%
  --(2.603,4.078)--(2.604,4.079)--(2.604,4.080)--(2.605,4.080)--(2.605,4.081)--(2.605,4.082)%
  --(2.606,4.082)--(2.606,4.083)--(2.607,4.083)--(2.607,4.084)--(2.607,4.085)--(2.608,4.085)%
  --(2.608,4.086)--(2.609,4.086)--(2.609,4.087)--(2.609,4.088)--(2.610,4.088)--(2.610,4.089)%
  --(2.611,4.089)--(2.611,4.090)--(2.611,4.091)--(2.612,4.091)--(2.612,4.092)--(2.613,4.093)%
  --(2.613,4.094)--(2.614,4.094)--(2.614,4.095)--(2.614,4.096)--(2.615,4.096)--(2.615,4.097)%
  --(2.616,4.097)--(2.616,4.098)--(2.616,4.099)--(2.617,4.099)--(2.617,4.100)--(2.618,4.100)%
  --(2.618,4.101)--(2.618,4.102)--(2.619,4.102)--(2.619,4.103)--(2.620,4.104)--(2.620,4.105)%
  --(2.621,4.105)--(2.621,4.106)--(2.622,4.107)--(2.622,4.108)--(2.623,4.108)--(2.623,4.109)%
  --(2.623,4.110)--(2.624,4.110)--(2.624,4.111)--(2.625,4.111)--(2.625,4.112)--(2.625,4.113)%
  --(2.626,4.113)--(2.626,4.114)--(2.627,4.114)--(2.627,4.115)--(2.627,4.116)--(2.628,4.116)%
  --(2.628,4.117)--(2.629,4.117)--(2.629,4.118)--(2.629,4.119)--(2.630,4.119)--(2.630,4.120)%
  --(2.631,4.120)--(2.631,4.121)--(2.631,4.122)--(2.632,4.122)--(2.632,4.123)--(2.632,4.124)%
  --(2.633,4.124)--(2.633,4.125)--(2.634,4.125)--(2.634,4.126)--(2.634,4.127)--(2.635,4.127)%
  --(2.635,4.128)--(2.636,4.128)--(2.636,4.129)--(2.636,4.130)--(2.637,4.130)--(2.637,4.131)%
  --(2.638,4.131)--(2.638,4.132)--(2.638,4.133)--(2.639,4.133)--(2.639,4.134)--(2.640,4.134)%
  --(2.640,4.135)--(2.640,4.136)--(2.641,4.136)--(2.641,4.137)--(2.642,4.137)--(2.642,4.138)%
  --(2.642,4.139)--(2.643,4.139)--(2.643,4.140)--(2.644,4.140)--(2.644,4.141)--(2.644,4.142)%
  --(2.645,4.142)--(2.645,4.143)--(2.646,4.143)--(2.646,4.144)--(2.646,4.145)--(2.647,4.145)%
  --(2.647,4.146)--(2.648,4.147)--(2.648,4.148)--(2.649,4.148)--(2.649,4.149)--(2.650,4.150)%
  --(2.650,4.151)--(2.651,4.151)--(2.651,4.152)--(2.651,4.153)--(2.652,4.153)--(2.652,4.154)%
  --(2.653,4.154)--(2.653,4.155)--(2.653,4.156)--(2.654,4.156)--(2.654,4.157)--(2.655,4.157)%
  --(2.655,4.158)--(2.655,4.159)--(2.656,4.159)--(2.656,4.160)--(2.657,4.160)--(2.657,4.161)%
  --(2.657,4.162)--(2.658,4.162)--(2.658,4.163)--(2.659,4.163)--(2.659,4.164)--(2.659,4.165)%
  --(2.660,4.165)--(2.660,4.166)--(2.661,4.166)--(2.661,4.167)--(2.661,4.168)--(2.662,4.168)%
  --(2.662,4.169)--(2.663,4.169)--(2.663,4.170)--(2.663,4.171)--(2.664,4.171)--(2.664,4.172)%
  --(2.665,4.172)--(2.665,4.173)--(2.665,4.174)--(2.666,4.174)--(2.666,4.175)--(2.667,4.175)%
  --(2.667,4.176)--(2.667,4.177)--(2.668,4.177)--(2.668,4.178)--(2.669,4.178)--(2.669,4.179)%
  --(2.669,4.180)--(2.670,4.180)--(2.670,4.181)--(2.671,4.181)--(2.671,4.182)--(2.671,4.183)%
  --(2.672,4.183)--(2.672,4.184)--(2.673,4.184)--(2.673,4.185)--(2.673,4.186)--(2.674,4.186)%
  --(2.674,4.187)--(2.675,4.187)--(2.675,4.188)--(2.675,4.189)--(2.676,4.189)--(2.676,4.190)%
  --(2.677,4.190)--(2.677,4.191)--(2.677,4.192)--(2.678,4.192)--(2.678,4.193)--(2.679,4.193)%
  --(2.679,4.194)--(2.679,4.195)--(2.680,4.195)--(2.680,4.196)--(2.681,4.196)--(2.681,4.197)%
  --(2.681,4.198)--(2.682,4.198)--(2.682,4.199)--(2.683,4.199)--(2.683,4.200)--(2.683,4.201)%
  --(2.684,4.201)--(2.684,4.202)--(2.685,4.202)--(2.685,4.203)--(2.685,4.204)--(2.686,4.204)%
  --(2.686,4.205)--(2.687,4.205)--(2.687,4.206)--(2.687,4.207)--(2.688,4.207)--(2.688,4.208)%
  --(2.689,4.208)--(2.689,4.209)--(2.690,4.210)--(2.690,4.211)--(2.691,4.211)--(2.691,4.212)%
  --(2.692,4.213)--(2.692,4.214)--(2.693,4.214)--(2.693,4.215)--(2.694,4.215)--(2.694,4.216)%
  --(2.694,4.217)--(2.695,4.217)--(2.695,4.218)--(2.696,4.218)--(2.696,4.219)--(2.696,4.220)%
  --(2.697,4.220)--(2.697,4.221)--(2.698,4.221)--(2.698,4.222)--(2.698,4.223)--(2.699,4.223)%
  --(2.699,4.224)--(2.700,4.224)--(2.700,4.225)--(2.700,4.226)--(2.701,4.226)--(2.701,4.227)%
  --(2.702,4.227)--(2.702,4.228)--(2.702,4.229)--(2.703,4.229)--(2.703,4.230)--(2.704,4.230)%
  --(2.704,4.231)--(2.705,4.232)--(2.705,4.233)--(2.706,4.233)--(2.706,4.234)--(2.707,4.235)%
  --(2.707,4.236)--(2.708,4.236)--(2.708,4.237)--(2.709,4.237)--(2.709,4.238)--(2.709,4.239)%
  --(2.710,4.239)--(2.710,4.240)--(2.711,4.240)--(2.711,4.241)--(2.711,4.242)--(2.712,4.242)%
  --(2.712,4.243)--(2.713,4.243)--(2.713,4.244)--(2.713,4.245)--(2.714,4.245)--(2.714,4.246)%
  --(2.715,4.246)--(2.715,4.247)--(2.715,4.248)--(2.716,4.248)--(2.716,4.249)--(2.717,4.249)%
  --(2.717,4.250)--(2.718,4.250)--(2.718,4.251)--(2.718,4.252)--(2.719,4.252)--(2.719,4.253)%
  --(2.720,4.253)--(2.720,4.254)--(2.720,4.255)--(2.721,4.255)--(2.721,4.256)--(2.722,4.256)%
  --(2.722,4.257)--(2.722,4.258)--(2.723,4.258)--(2.723,4.259)--(2.724,4.259)--(2.724,4.260)%
  --(2.725,4.261)--(2.725,4.262)--(2.726,4.262)--(2.726,4.263)--(2.727,4.263)--(2.727,4.264)%
  --(2.727,4.265)--(2.728,4.265)--(2.728,4.266)--(2.729,4.266)--(2.729,4.267)--(2.729,4.268)%
  --(2.730,4.268)--(2.730,4.269)--(2.731,4.269)--(2.731,4.270)--(2.731,4.271)--(2.732,4.271)%
  --(2.732,4.272)--(2.733,4.272)--(2.733,4.273)--(2.734,4.274)--(2.734,4.275)--(2.735,4.275)%
  --(2.735,4.276)--(2.736,4.276)--(2.736,4.277)--(2.736,4.278)--(2.737,4.278)--(2.737,4.279)%
  --(2.738,4.279)--(2.738,4.280)--(2.738,4.281)--(2.739,4.281)--(2.739,4.282)--(2.740,4.282)%
  --(2.740,4.283)--(2.741,4.284)--(2.741,4.285)--(2.742,4.285)--(2.742,4.286)--(2.743,4.286)%
  --(2.743,4.287)--(2.743,4.288)--(2.744,4.288)--(2.744,4.289)--(2.745,4.289)--(2.745,4.290)%
  --(2.745,4.291)--(2.746,4.291)--(2.746,4.292)--(2.747,4.292)--(2.747,4.293)--(2.748,4.293)%
  --(2.748,4.294)--(2.748,4.295)--(2.749,4.295)--(2.749,4.296)--(2.750,4.296)--(2.750,4.297)%
  --(2.750,4.298)--(2.751,4.298)--(2.751,4.299)--(2.752,4.299)--(2.752,4.300)--(2.753,4.301)%
  --(2.753,4.302)--(2.754,4.302)--(2.754,4.303)--(2.755,4.303)--(2.755,4.304)--(2.755,4.305)%
  --(2.756,4.305)--(2.756,4.306)--(2.757,4.306)--(2.757,4.307)--(2.758,4.308)--(2.758,4.309)%
  --(2.759,4.309)--(2.759,4.310)--(2.760,4.310)--(2.760,4.311)--(2.760,4.312)--(2.761,4.312)%
  --(2.761,4.313)--(2.762,4.313)--(2.762,4.314)--(2.762,4.315)--(2.763,4.315)--(2.763,4.316)%
  --(2.764,4.316)--(2.764,4.317)--(2.765,4.317)--(2.765,4.318)--(2.765,4.319)--(2.766,4.319)%
  --(2.766,4.320)--(2.767,4.320)--(2.767,4.321)--(2.768,4.322)--(2.768,4.323)--(2.769,4.323)%
  --(2.769,4.324)--(2.770,4.324)--(2.770,4.325)--(2.770,4.326)--(2.771,4.326)--(2.771,4.327)%
  --(2.772,4.327)--(2.772,4.328)--(2.773,4.329)--(2.773,4.330)--(2.774,4.330)--(2.774,4.331)%
  --(2.775,4.331)--(2.775,4.332)--(2.775,4.333)--(2.776,4.333)--(2.776,4.334)--(2.777,4.334)%
  --(2.777,4.335)--(2.778,4.335)--(2.778,4.336)--(2.778,4.337)--(2.779,4.337)--(2.779,4.338)%
  --(2.780,4.338)--(2.780,4.339)--(2.780,4.340)--(2.781,4.340)--(2.781,4.341)--(2.782,4.341)%
  --(2.782,4.342)--(2.783,4.342)--(2.783,4.343)--(2.783,4.344)--(2.784,4.344)--(2.784,4.345)%
  --(2.785,4.345)--(2.785,4.346)--(2.786,4.347)--(2.786,4.348)--(2.787,4.348)--(2.787,4.349)%
  --(2.788,4.349)--(2.788,4.350)--(2.788,4.351)--(2.789,4.351)--(2.789,4.352)--(2.790,4.352)%
  --(2.790,4.353)--(2.791,4.353)--(2.791,4.354)--(2.791,4.355)--(2.792,4.355)--(2.792,4.356)%
  --(2.793,4.356)--(2.793,4.357)--(2.793,4.358)--(2.794,4.358)--(2.794,4.359)--(2.795,4.359)%
  --(2.795,4.360)--(2.796,4.360)--(2.796,4.361)--(2.796,4.362)--(2.797,4.362)--(2.797,4.363)%
  --(2.798,4.363)--(2.798,4.364)--(2.799,4.364)--(2.799,4.365)--(2.799,4.366)--(2.800,4.366)%
  --(2.800,4.367)--(2.801,4.367)--(2.801,4.368)--(2.802,4.368)--(2.802,4.369)--(2.802,4.370)%
  --(2.803,4.370)--(2.803,4.371)--(2.804,4.371)--(2.804,4.372)--(2.805,4.373)--(2.805,4.374)%
  --(2.806,4.374)--(2.806,4.375)--(2.807,4.375)--(2.807,4.376)--(2.807,4.377)--(2.808,4.377)%
  --(2.808,4.378)--(2.809,4.378)--(2.809,4.379)--(2.810,4.379)--(2.810,4.380)--(2.810,4.381)%
  --(2.811,4.381)--(2.811,4.382)--(2.812,4.382)--(2.812,4.383)--(2.813,4.383)--(2.813,4.384)%
  --(2.813,4.385)--(2.814,4.385)--(2.814,4.386)--(2.815,4.386)--(2.815,4.387)--(2.816,4.387)%
  --(2.816,4.388)--(2.816,4.389)--(2.817,4.389)--(2.817,4.390)--(2.818,4.390)--(2.818,4.391)%
  --(2.819,4.392)--(2.819,4.393)--(2.820,4.393)--(2.820,4.394)--(2.821,4.394)--(2.821,4.395)%
  --(2.822,4.396)--(2.822,4.397)--(2.823,4.397)--(2.823,4.398)--(2.824,4.398)--(2.824,4.399)%
  --(2.825,4.400)--(2.825,4.401)--(2.826,4.401)--(2.826,4.402)--(2.827,4.402)--(2.827,4.403)%
  --(2.828,4.403)--(2.828,4.404)--(2.828,4.405)--(2.829,4.405)--(2.829,4.406)--(2.830,4.406)%
  --(2.830,4.407)--(2.831,4.408)--(2.831,4.409)--(2.832,4.409)--(2.832,4.410)--(2.833,4.410)%
  --(2.833,4.411)--(2.834,4.411)--(2.834,4.412)--(2.834,4.413)--(2.835,4.413)--(2.835,4.414)%
  --(2.836,4.414)--(2.836,4.415)--(2.837,4.415)--(2.837,4.416)--(2.837,4.417)--(2.838,4.417)%
  --(2.838,4.418)--(2.839,4.418)--(2.839,4.419)--(2.840,4.419)--(2.840,4.420)--(2.840,4.421)%
  --(2.841,4.421)--(2.841,4.422)--(2.842,4.422)--(2.842,4.423)--(2.843,4.423)--(2.843,4.424)%
  --(2.843,4.425)--(2.844,4.425)--(2.844,4.426)--(2.845,4.426)--(2.845,4.427)--(2.846,4.427)%
  --(2.846,4.428)--(2.846,4.429)--(2.847,4.429)--(2.847,4.430)--(2.848,4.430)--(2.848,4.431)%
  --(2.849,4.431)--(2.849,4.432)--(2.849,4.433)--(2.850,4.433)--(2.850,4.434)--(2.851,4.434)%
  --(2.851,4.435)--(2.852,4.435)--(2.852,4.436)--(2.853,4.437)--(2.853,4.438)--(2.854,4.438)%
  --(2.854,4.439)--(2.855,4.439)--(2.855,4.440)--(2.856,4.440)--(2.856,4.441)--(2.856,4.442)%
  --(2.857,4.442)--(2.857,4.443)--(2.858,4.443)--(2.858,4.444)--(2.859,4.444)--(2.859,4.445)%
  --(2.859,4.446)--(2.860,4.446)--(2.860,4.447)--(2.861,4.447)--(2.861,4.448)--(2.862,4.448)%
  --(2.862,4.449)--(2.862,4.450)--(2.863,4.450)--(2.863,4.451)--(2.864,4.451)--(2.864,4.452)%
  --(2.865,4.452)--(2.865,4.453)--(2.865,4.454)--(2.866,4.454)--(2.866,4.455)--(2.867,4.455)%
  --(2.867,4.456)--(2.868,4.456)--(2.868,4.457)--(2.869,4.457)--(2.869,4.458)--(2.869,4.459)%
  --(2.870,4.459)--(2.870,4.460)--(2.871,4.460)--(2.871,4.461)--(2.872,4.461)--(2.872,4.462)%
  --(2.872,4.463)--(2.873,4.463)--(2.873,4.464)--(2.874,4.464)--(2.874,4.465)--(2.875,4.465)%
  --(2.875,4.466)--(2.876,4.466)--(2.876,4.467)--(2.876,4.468)--(2.877,4.468)--(2.877,4.469)%
  --(2.878,4.469)--(2.878,4.470)--(2.879,4.470)--(2.879,4.471)--(2.879,4.472)--(2.880,4.472)%
  --(2.880,4.473)--(2.881,4.473)--(2.881,4.474)--(2.882,4.474)--(2.882,4.475)--(2.883,4.475)%
  --(2.883,4.476)--(2.883,4.477)--(2.884,4.477)--(2.884,4.478)--(2.885,4.478)--(2.885,4.479)%
  --(2.886,4.479)--(2.886,4.480)--(2.887,4.481)--(2.887,4.482)--(2.888,4.482)--(2.888,4.483)%
  --(2.889,4.483)--(2.889,4.484)--(2.890,4.484)--(2.890,4.485)--(2.890,4.486)--(2.891,4.486)%
  --(2.891,4.487)--(2.892,4.487)--(2.892,4.488)--(2.893,4.488)--(2.893,4.489)--(2.894,4.490)%
  --(2.894,4.491)--(2.895,4.491)--(2.895,4.492)--(2.896,4.492)--(2.896,4.493)--(2.897,4.493)%
  --(2.897,4.494)--(2.898,4.495)--(2.898,4.496)--(2.899,4.496)--(2.899,4.497)--(2.900,4.497)%
  --(2.900,4.498)--(2.901,4.498)--(2.901,4.499)--(2.901,4.500)--(2.902,4.500)--(2.902,4.501)%
  --(2.903,4.501)--(2.903,4.502)--(2.904,4.502)--(2.904,4.503)--(2.905,4.503)--(2.905,4.504)%
  --(2.905,4.505)--(2.906,4.505)--(2.906,4.506)--(2.907,4.506)--(2.907,4.507)--(2.908,4.507)%
  --(2.908,4.508)--(2.909,4.508)--(2.909,4.509)--(2.909,4.510)--(2.910,4.510)--(2.910,4.511)%
  --(2.911,4.511)--(2.911,4.512)--(2.912,4.512)--(2.912,4.513)--(2.913,4.513)--(2.913,4.514)%
  --(2.913,4.515)--(2.914,4.515)--(2.914,4.516)--(2.915,4.516)--(2.915,4.517)--(2.916,4.517)%
  --(2.916,4.518)--(2.917,4.518)--(2.917,4.519)--(2.917,4.520)--(2.918,4.520)--(2.918,4.521)%
  --(2.919,4.521)--(2.919,4.522)--(2.920,4.522)--(2.920,4.523)--(2.921,4.523)--(2.921,4.524)%
  --(2.921,4.525)--(2.922,4.525)--(2.922,4.526)--(2.923,4.526)--(2.923,4.527)--(2.924,4.527)%
  --(2.924,4.528)--(2.925,4.528)--(2.925,4.529)--(2.925,4.530)--(2.926,4.530)--(2.926,4.531)%
  --(2.927,4.531)--(2.927,4.532)--(2.928,4.532)--(2.928,4.533)--(2.929,4.533)--(2.929,4.534)%
  --(2.930,4.535)--(2.930,4.536)--(2.931,4.536)--(2.931,4.537)--(2.932,4.537)--(2.932,4.538)%
  --(2.933,4.538)--(2.933,4.539)--(2.934,4.539)--(2.934,4.540)--(2.934,4.541)--(2.935,4.541)%
  --(2.935,4.542)--(2.936,4.542)--(2.936,4.543)--(2.937,4.543)--(2.937,4.544)--(2.938,4.544)%
  --(2.938,4.545)--(2.938,4.546)--(2.939,4.546)--(2.939,4.547)--(2.940,4.547)--(2.940,4.548)%
  --(2.941,4.548)--(2.941,4.549)--(2.942,4.549)--(2.942,4.550)--(2.943,4.551)--(2.943,4.552)%
  --(2.944,4.552)--(2.944,4.553)--(2.945,4.553)--(2.945,4.554)--(2.946,4.554)--(2.946,4.555)%
  --(2.947,4.555)--(2.947,4.556)--(2.947,4.557)--(2.948,4.557)--(2.948,4.558)--(2.949,4.558)%
  --(2.949,4.559)--(2.950,4.559)--(2.950,4.560)--(2.951,4.560)--(2.951,4.561)--(2.952,4.562)%
  --(2.952,4.563)--(2.953,4.563)--(2.953,4.564)--(2.954,4.564)--(2.954,4.565)--(2.955,4.565)%
  --(2.955,4.566)--(2.956,4.566)--(2.956,4.567)--(2.956,4.568)--(2.957,4.568)--(2.957,4.569)%
  --(2.958,4.569)--(2.958,4.570)--(2.959,4.570)--(2.959,4.571)--(2.960,4.571)--(2.960,4.572)%
  --(2.961,4.572)--(2.961,4.573)--(2.961,4.574)--(2.962,4.574)--(2.962,4.575)--(2.963,4.575)%
  --(2.963,4.576)--(2.964,4.576)--(2.964,4.577)--(2.965,4.577)--(2.965,4.578)--(2.966,4.578)%
  --(2.966,4.579)--(2.966,4.580)--(2.967,4.580)--(2.967,4.581)--(2.968,4.581)--(2.968,4.582)%
  --(2.969,4.582)--(2.969,4.583)--(2.970,4.583)--(2.970,4.584)--(2.971,4.584)--(2.971,4.585)%
  --(2.971,4.586)--(2.972,4.586)--(2.972,4.587)--(2.973,4.587)--(2.973,4.588)--(2.974,4.588)%
  --(2.974,4.589)--(2.975,4.589)--(2.975,4.590)--(2.976,4.590)--(2.976,4.591)--(2.977,4.592)%
  --(2.977,4.593)--(2.978,4.593)--(2.978,4.594)--(2.979,4.594)--(2.979,4.595)--(2.980,4.595)%
  --(2.980,4.596)--(2.981,4.596)--(2.981,4.597)--(2.982,4.598)--(2.982,4.599)--(2.983,4.599)%
  --(2.983,4.600)--(2.984,4.600)--(2.984,4.601)--(2.985,4.601)--(2.985,4.602)--(2.986,4.602)%
  --(2.986,4.603)--(2.987,4.603)--(2.987,4.604)--(2.987,4.605)--(2.988,4.605)--(2.988,4.606)%
  --(2.989,4.606)--(2.989,4.607)--(2.990,4.607)--(2.990,4.608)--(2.991,4.608)--(2.991,4.609)%
  --(2.992,4.609)--(2.992,4.610)--(2.993,4.610)--(2.993,4.611)--(2.993,4.612)--(2.994,4.612)%
  --(2.994,4.613)--(2.995,4.613)--(2.995,4.614)--(2.996,4.614)--(2.996,4.615)--(2.997,4.615)%
  --(2.997,4.616)--(2.998,4.616)--(2.998,4.617)--(2.999,4.618)--(2.999,4.619)--(3.000,4.619)%
  --(3.000,4.620)--(3.001,4.620)--(3.001,4.621)--(3.002,4.621)--(3.002,4.622)--(3.003,4.622)%
  --(3.003,4.623)--(3.004,4.623)--(3.004,4.624)--(3.005,4.625)--(3.005,4.626)--(3.006,4.626)%
  --(3.006,4.627)--(3.007,4.627)--(3.007,4.628)--(3.008,4.628)--(3.008,4.629)--(3.009,4.629)%
  --(3.009,4.630)--(3.010,4.630)--(3.010,4.631)--(3.011,4.632)--(3.011,4.633)--(3.012,4.633)%
  --(3.012,4.634)--(3.013,4.634)--(3.013,4.635)--(3.014,4.635)--(3.014,4.636)--(3.015,4.636)%
  --(3.015,4.637)--(3.016,4.637)--(3.016,4.638)--(3.017,4.638)--(3.017,4.639)--(3.017,4.640)%
  --(3.018,4.640)--(3.018,4.641)--(3.019,4.641)--(3.019,4.642)--(3.020,4.642)--(3.020,4.643)%
  --(3.021,4.643)--(3.021,4.644)--(3.022,4.644)--(3.022,4.645)--(3.023,4.645)--(3.023,4.646)%
  --(3.024,4.646)--(3.024,4.647)--(3.024,4.648)--(3.025,4.648)--(3.025,4.649)--(3.026,4.649)%
  --(3.026,4.650)--(3.027,4.650)--(3.027,4.651)--(3.028,4.651)--(3.028,4.652)--(3.029,4.652)%
  --(3.029,4.653)--(3.030,4.653)--(3.030,4.654)--(3.031,4.655)--(3.031,4.656)--(3.032,4.656)%
  --(3.032,4.657)--(3.033,4.657)--(3.033,4.658)--(3.034,4.658)--(3.034,4.659)--(3.035,4.659)%
  --(3.035,4.660)--(3.036,4.660)--(3.036,4.661)--(3.037,4.661)--(3.037,4.662)--(3.038,4.662)%
  --(3.038,4.663)--(3.038,4.664)--(3.039,4.664)--(3.039,4.665)--(3.040,4.665)--(3.040,4.666)%
  --(3.041,4.666)--(3.041,4.667)--(3.042,4.667)--(3.042,4.668)--(3.043,4.668)--(3.043,4.669)%
  --(3.044,4.669)--(3.044,4.670)--(3.045,4.670)--(3.045,4.671)--(3.046,4.671)--(3.046,4.672)%
  --(3.046,4.673)--(3.047,4.673)--(3.047,4.674)--(3.048,4.674)--(3.048,4.675)--(3.049,4.675)%
  --(3.049,4.676)--(3.050,4.676)--(3.050,4.677)--(3.051,4.677)--(3.051,4.678)--(3.052,4.678)%
  --(3.052,4.679)--(3.053,4.679)--(3.053,4.680)--(3.054,4.681)--(3.054,4.682)--(3.055,4.682)%
  --(3.055,4.683)--(3.056,4.683)--(3.056,4.684)--(3.057,4.684)--(3.057,4.685)--(3.058,4.685)%
  --(3.058,4.686)--(3.059,4.686)--(3.059,4.687)--(3.060,4.687)--(3.060,4.688)--(3.061,4.688)%
  --(3.061,4.689)--(3.062,4.689)--(3.062,4.690)--(3.063,4.691)--(3.063,4.692)--(3.064,4.692)%
  --(3.064,4.693)--(3.065,4.693)--(3.065,4.694)--(3.066,4.694)--(3.066,4.695)--(3.067,4.695)%
  --(3.067,4.696)--(3.068,4.696)--(3.068,4.697)--(3.069,4.697)--(3.069,4.698)--(3.070,4.698)%
  --(3.070,4.699)--(3.071,4.699)--(3.071,4.700)--(3.072,4.701)--(3.072,4.702)--(3.073,4.702)%
  --(3.073,4.703)--(3.074,4.703)--(3.074,4.704)--(3.075,4.704)--(3.075,4.705)--(3.076,4.705)%
  --(3.076,4.706)--(3.077,4.706)--(3.077,4.707)--(3.078,4.707)--(3.078,4.708)--(3.079,4.708)%
  --(3.079,4.709)--(3.080,4.709)--(3.080,4.710)--(3.081,4.710)--(3.081,4.711)--(3.082,4.712)%
  --(3.082,4.713)--(3.083,4.713)--(3.083,4.714)--(3.084,4.714)--(3.084,4.715)--(3.085,4.715)%
  --(3.085,4.716)--(3.086,4.716)--(3.086,4.717)--(3.087,4.717)--(3.087,4.718)--(3.088,4.718)%
  --(3.088,4.719)--(3.089,4.719)--(3.089,4.720)--(3.090,4.720)--(3.090,4.721)--(3.091,4.721)%
  --(3.091,4.722)--(3.092,4.722)--(3.092,4.723)--(3.093,4.724)--(3.093,4.725)--(3.094,4.725)%
  --(3.094,4.726)--(3.095,4.726)--(3.095,4.727)--(3.096,4.727)--(3.096,4.728)--(3.097,4.728)%
  --(3.097,4.729)--(3.098,4.729)--(3.098,4.730)--(3.099,4.730)--(3.099,4.731)--(3.100,4.731)%
  --(3.100,4.732)--(3.101,4.732)--(3.101,4.733)--(3.102,4.733)--(3.102,4.734)--(3.103,4.734)%
  --(3.103,4.735)--(3.104,4.735)--(3.104,4.736)--(3.105,4.736)--(3.105,4.737)--(3.106,4.738)%
  --(3.106,4.739)--(3.107,4.739)--(3.107,4.740)--(3.108,4.740)--(3.108,4.741)--(3.109,4.741)%
  --(3.109,4.742)--(3.110,4.742)--(3.110,4.743)--(3.111,4.743)--(3.111,4.744)--(3.112,4.744)%
  --(3.112,4.745)--(3.113,4.745)--(3.113,4.746)--(3.114,4.746)--(3.114,4.747)--(3.115,4.747)%
  --(3.115,4.748)--(3.116,4.748)--(3.116,4.749)--(3.117,4.749)--(3.117,4.750)--(3.118,4.750)%
  --(3.118,4.751)--(3.119,4.751)--(3.119,4.752)--(3.120,4.752)--(3.120,4.753)--(3.120,4.754)%
  --(3.121,4.754)--(3.121,4.755)--(3.122,4.755)--(3.122,4.756)--(3.123,4.756)--(3.123,4.757)%
  --(3.124,4.757)--(3.124,4.758)--(3.125,4.758)--(3.125,4.759)--(3.126,4.759)--(3.126,4.760)%
  --(3.127,4.760)--(3.127,4.761)--(3.128,4.761)--(3.128,4.762)--(3.129,4.762)--(3.129,4.763)%
  --(3.130,4.763)--(3.130,4.764)--(3.131,4.764)--(3.131,4.765)--(3.132,4.765)--(3.132,4.766)%
  --(3.133,4.766)--(3.133,4.767)--(3.134,4.767)--(3.134,4.768)--(3.135,4.768)--(3.135,4.769)%
  --(3.136,4.769)--(3.136,4.770)--(3.137,4.770)--(3.137,4.771)--(3.138,4.771)--(3.138,4.772)%
  --(3.139,4.772)--(3.139,4.773)--(3.139,4.774)--(3.140,4.774)--(3.140,4.775)--(3.141,4.775)%
  --(3.141,4.776)--(3.142,4.776)--(3.142,4.777)--(3.143,4.777)--(3.143,4.778)--(3.144,4.778)%
  --(3.144,4.779)--(3.145,4.779)--(3.145,4.780)--(3.146,4.780)--(3.146,4.781)--(3.147,4.781)%
  --(3.147,4.782)--(3.148,4.782)--(3.148,4.783)--(3.149,4.783)--(3.149,4.784)--(3.150,4.784)%
  --(3.150,4.785)--(3.151,4.785)--(3.151,4.786)--(3.152,4.786)--(3.152,4.787)--(3.153,4.787)%
  --(3.153,4.788)--(3.154,4.788)--(3.154,4.789)--(3.155,4.789)--(3.155,4.790)--(3.156,4.790)%
  --(3.156,4.791)--(3.157,4.791)--(3.157,4.792)--(3.158,4.792)--(3.158,4.793)--(3.159,4.793)%
  --(3.159,4.794)--(3.160,4.794)--(3.160,4.795)--(3.161,4.795)--(3.161,4.796)--(3.162,4.796)%
  --(3.162,4.797)--(3.163,4.797)--(3.163,4.798)--(3.164,4.798)--(3.164,4.799)--(3.165,4.799)%
  --(3.165,4.800)--(3.166,4.800)--(3.166,4.801)--(3.167,4.801)--(3.167,4.802)--(3.168,4.803)%
  --(3.169,4.804)--(3.170,4.805)--(3.170,4.806)--(3.171,4.806)--(3.172,4.807)--(3.173,4.808)%
  --(3.173,4.809)--(3.174,4.809)--(3.174,4.810)--(3.175,4.810)--(3.175,4.811)--(3.176,4.811)%
  --(3.176,4.812)--(3.177,4.812)--(3.177,4.813)--(3.178,4.813)--(3.178,4.814)--(3.179,4.814)%
  --(3.179,4.815)--(3.180,4.815)--(3.180,4.816)--(3.181,4.816)--(3.181,4.817)--(3.182,4.817)%
  --(3.182,4.818)--(3.183,4.818)--(3.183,4.819)--(3.184,4.819)--(3.184,4.820)--(3.185,4.820)%
  --(3.185,4.821)--(3.186,4.821)--(3.186,4.822)--(3.187,4.822)--(3.187,4.823)--(3.188,4.823)%
  --(3.188,4.824)--(3.189,4.824)--(3.189,4.825)--(3.190,4.825)--(3.190,4.826)--(3.191,4.826)%
  --(3.191,4.827)--(3.192,4.827)--(3.192,4.828)--(3.193,4.828)--(3.193,4.829)--(3.194,4.829)%
  --(3.194,4.830)--(3.195,4.830)--(3.195,4.831)--(3.196,4.831)--(3.196,4.832)--(3.197,4.832)%
  --(3.197,4.833)--(3.198,4.833)--(3.198,4.834)--(3.199,4.834)--(3.199,4.835)--(3.200,4.835)%
  --(3.200,4.836)--(3.201,4.836)--(3.201,4.837)--(3.202,4.837)--(3.202,4.838)--(3.203,4.838)%
  --(3.203,4.839)--(3.204,4.839)--(3.204,4.840)--(3.205,4.840)--(3.205,4.841)--(3.206,4.841)%
  --(3.206,4.842)--(3.207,4.842)--(3.207,4.843)--(3.208,4.843)--(3.208,4.844)--(3.209,4.844)%
  --(3.209,4.845)--(3.210,4.845)--(3.210,4.846)--(3.211,4.846)--(3.211,4.847)--(3.212,4.847)%
  --(3.212,4.848)--(3.213,4.848)--(3.214,4.849)--(3.214,4.850)--(3.215,4.850)--(3.216,4.851)%
  --(3.217,4.851)--(3.217,4.852)--(3.218,4.853)--(3.219,4.853)--(3.219,4.854)--(3.220,4.854)%
  --(3.220,4.855)--(3.221,4.855)--(3.221,4.856)--(3.222,4.856)--(3.222,4.857)--(3.223,4.857)%
  --(3.223,4.858)--(3.224,4.858)--(3.224,4.859)--(3.225,4.859)--(3.225,4.860)--(3.226,4.860)%
  --(3.226,4.861)--(3.227,4.861)--(3.227,4.862)--(3.228,4.862)--(3.228,4.863)--(3.229,4.863)%
  --(3.229,4.864)--(3.230,4.864)--(3.230,4.865)--(3.231,4.865)--(3.231,4.866)--(3.232,4.866)%
  --(3.232,4.867)--(3.233,4.867)--(3.233,4.868)--(3.234,4.868)--(3.234,4.869)--(3.235,4.869)%
  --(3.235,4.870)--(3.236,4.870)--(3.236,4.871)--(3.237,4.871)--(3.237,4.872)--(3.238,4.872)%
  --(3.238,4.873)--(3.239,4.873)--(3.239,4.874)--(3.240,4.874)--(3.240,4.875)--(3.241,4.875)%
  --(3.241,4.876)--(3.242,4.876)--(3.242,4.877)--(3.243,4.877)--(3.243,4.878)--(3.244,4.878)%
  --(3.244,4.879)--(3.245,4.879)--(3.245,4.880)--(3.246,4.880)--(3.246,4.881)--(3.247,4.881)%
  --(3.248,4.882)--(3.249,4.882)--(3.249,4.883)--(3.250,4.883)--(3.250,4.884)--(3.251,4.884)%
  --(3.251,4.885)--(3.252,4.885)--(3.252,4.886)--(3.253,4.886)--(3.253,4.887)--(3.254,4.887)%
  --(3.254,4.888)--(3.255,4.888)--(3.255,4.889)--(3.256,4.889)--(3.256,4.890)--(3.257,4.890)%
  --(3.257,4.891)--(3.258,4.891)--(3.258,4.892)--(3.259,4.892)--(3.259,4.893)--(3.260,4.893)%
  --(3.260,4.894)--(3.261,4.894)--(3.261,4.895)--(3.262,4.895)--(3.262,4.896)--(3.263,4.896)%
  --(3.263,4.897)--(3.264,4.897)--(3.264,4.898)--(3.265,4.898)--(3.265,4.899)--(3.266,4.899)%
  --(3.267,4.900)--(3.268,4.900)--(3.268,4.901)--(3.269,4.901)--(3.269,4.902)--(3.270,4.902)%
  --(3.270,4.903)--(3.271,4.903)--(3.271,4.904)--(3.272,4.904)--(3.272,4.905)--(3.273,4.905)%
  --(3.273,4.906)--(3.274,4.906)--(3.274,4.907)--(3.275,4.907)--(3.275,4.908)--(3.276,4.908)%
  --(3.276,4.909)--(3.277,4.909)--(3.277,4.910)--(3.278,4.910)--(3.278,4.911)--(3.279,4.911)%
  --(3.279,4.912)--(3.280,4.912)--(3.280,4.913)--(3.281,4.913)--(3.282,4.914)--(3.283,4.915)%
  --(3.284,4.915)--(3.284,4.916)--(3.285,4.916)--(3.285,4.917)--(3.286,4.917)--(3.286,4.918)%
  --(3.287,4.918)--(3.287,4.919)--(3.288,4.919)--(3.288,4.920)--(3.289,4.920)--(3.289,4.921)%
  --(3.290,4.921)--(3.290,4.922)--(3.291,4.922)--(3.291,4.923)--(3.292,4.923)--(3.292,4.924)%
  --(3.293,4.924)--(3.293,4.925)--(3.294,4.925)--(3.295,4.926)--(3.296,4.927)--(3.297,4.927)%
  --(3.297,4.928)--(3.298,4.928)--(3.298,4.929)--(3.299,4.929)--(3.299,4.930)--(3.300,4.930)%
  --(3.300,4.931)--(3.301,4.931)--(3.301,4.932)--(3.302,4.932)--(3.302,4.933)--(3.303,4.933)%
  --(3.303,4.934)--(3.304,4.934)--(3.304,4.935)--(3.305,4.935)--(3.305,4.936)--(3.306,4.936)%
  --(3.307,4.937)--(3.308,4.937)--(3.308,4.938)--(3.309,4.938)--(3.309,4.939)--(3.310,4.939)%
  --(3.310,4.940)--(3.311,4.940)--(3.311,4.941)--(3.312,4.941)--(3.312,4.942)--(3.313,4.942)%
  --(3.313,4.943)--(3.314,4.943)--(3.314,4.944)--(3.315,4.944)--(3.315,4.945)--(3.316,4.945)%
  --(3.316,4.946)--(3.317,4.946)--(3.318,4.947)--(3.319,4.947)--(3.319,4.948)--(3.320,4.948)%
  --(3.320,4.949)--(3.321,4.949)--(3.321,4.950)--(3.322,4.950)--(3.322,4.951)--(3.323,4.951)%
  --(3.323,4.952)--(3.324,4.952)--(3.324,4.953)--(3.325,4.953)--(3.325,4.954)--(3.326,4.954)%
  --(3.326,4.955)--(3.327,4.955)--(3.328,4.955)--(3.328,4.956)--(3.329,4.956)--(3.329,4.957)%
  --(3.330,4.957)--(3.330,4.958)--(3.331,4.958)--(3.331,4.959)--(3.332,4.959)--(3.332,4.960)%
  --(3.333,4.960)--(3.333,4.961)--(3.334,4.961)--(3.334,4.962)--(3.335,4.962)--(3.335,4.963)%
  --(3.336,4.963)--(3.337,4.963)--(3.337,4.964)--(3.338,4.964)--(3.338,4.965)--(3.339,4.965)%
  --(3.339,4.966)--(3.340,4.966)--(3.340,4.967)--(3.341,4.967)--(3.341,4.968)--(3.342,4.968)%
  --(3.342,4.969)--(3.343,4.969)--(3.343,4.970)--(3.344,4.970)--(3.345,4.971)--(3.346,4.971)%
  --(3.346,4.972)--(3.347,4.972)--(3.347,4.973)--(3.348,4.973)--(3.348,4.974)--(3.349,4.974)%
  --(3.349,4.975)--(3.350,4.975)--(3.350,4.976)--(3.351,4.976)--(3.351,4.977)--(3.352,4.977)%
  --(3.352,4.978)--(3.353,4.978)--(3.354,4.978)--(3.354,4.979)--(3.355,4.979)--(3.355,4.980)%
  --(3.356,4.980)--(3.356,4.981)--(3.357,4.981)--(3.357,4.982)--(3.358,4.982)--(3.358,4.983)%
  --(3.359,4.983)--(3.359,4.984)--(3.360,4.984)--(3.360,4.985)--(3.361,4.985)--(3.362,4.985)%
  --(3.362,4.986)--(3.363,4.986)--(3.363,4.987)--(3.364,4.987)--(3.364,4.988)--(3.365,4.988)%
  --(3.365,4.989)--(3.366,4.989)--(3.366,4.990)--(3.367,4.990)--(3.367,4.991)--(3.368,4.991)%
  --(3.369,4.991)--(3.369,4.992)--(3.370,4.992)--(3.370,4.993)--(3.371,4.993)--(3.371,4.994)%
  --(3.372,4.994)--(3.372,4.995)--(3.373,4.995)--(3.373,4.996)--(3.374,4.996)--(3.374,4.997)%
  --(3.375,4.997)--(3.376,4.997)--(3.376,4.998)--(3.377,4.998)--(3.377,4.999)--(3.378,4.999)%
  --(3.378,5.000)--(3.379,5.000)--(3.379,5.001)--(3.380,5.001)--(3.380,5.002)--(3.381,5.002)%
  --(3.381,5.003)--(3.382,5.003)--(3.383,5.004)--(3.384,5.004)--(3.384,5.005)--(3.385,5.005)%
  --(3.385,5.006)--(3.386,5.006)--(3.386,5.007)--(3.387,5.007)--(3.387,5.008)--(3.388,5.008)%
  --(3.388,5.009)--(3.389,5.009)--(3.390,5.009)--(3.390,5.010)--(3.391,5.010)--(3.391,5.011)%
  --(3.392,5.011)--(3.392,5.012)--(3.393,5.012)--(3.393,5.013)--(3.394,5.013)--(3.394,5.014)%
  --(3.395,5.014)--(3.396,5.015)--(3.397,5.015)--(3.397,5.016)--(3.398,5.016)--(3.398,5.017)%
  --(3.399,5.017)--(3.399,5.018)--(3.400,5.018)--(3.400,5.019)--(3.401,5.019)--(3.401,5.020)%
  --(3.402,5.020)--(3.403,5.020)--(3.403,5.021)--(3.404,5.021)--(3.404,5.022)--(3.405,5.022)%
  --(3.405,5.023)--(3.406,5.023)--(3.406,5.024)--(3.407,5.024)--(3.407,5.025)--(3.408,5.025)%
  --(3.409,5.025)--(3.409,5.026)--(3.410,5.026)--(3.410,5.027)--(3.411,5.027)--(3.411,5.028)%
  --(3.412,5.028)--(3.412,5.029)--(3.413,5.029)--(3.413,5.030)--(3.414,5.030)--(3.415,5.030)%
  --(3.415,5.031)--(3.416,5.031)--(3.416,5.032)--(3.417,5.032)--(3.417,5.033)--(3.418,5.033)%
  --(3.418,5.034)--(3.419,5.034)--(3.419,5.035)--(3.420,5.035)--(3.421,5.035)--(3.421,5.036)%
  --(3.422,5.036)--(3.422,5.037)--(3.423,5.037)--(3.423,5.038)--(3.424,5.038)--(3.424,5.039)%
  --(3.425,5.039)--(3.426,5.040)--(3.427,5.040)--(3.427,5.041)--(3.428,5.041)--(3.428,5.042)%
  --(3.429,5.042)--(3.429,5.043)--(3.430,5.043)--(3.430,5.044)--(3.431,5.044)--(3.432,5.044)%
  --(3.432,5.045)--(3.433,5.045)--(3.433,5.046)--(3.434,5.046)--(3.434,5.047)--(3.435,5.047)%
  --(3.435,5.048)--(3.436,5.048)--(3.437,5.049)--(3.438,5.049)--(3.438,5.050)--(3.439,5.050)%
  --(3.439,5.051)--(3.440,5.051)--(3.440,5.052)--(3.441,5.052)--(3.441,5.053)--(3.442,5.053)%
  --(3.443,5.053)--(3.443,5.054)--(3.444,5.054)--(3.444,5.055)--(3.445,5.055)--(3.445,5.056)%
  --(3.446,5.056)--(3.446,5.057)--(3.447,5.057)--(3.448,5.057)--(3.448,5.058)--(3.449,5.058)%
  --(3.449,5.059)--(3.450,5.059)--(3.450,5.060)--(3.451,5.060)--(3.451,5.061)--(3.452,5.061)%
  --(3.453,5.061)--(3.453,5.062)--(3.454,5.062)--(3.454,5.063)--(3.455,5.063)--(3.455,5.064)%
  --(3.456,5.064)--(3.456,5.065)--(3.457,5.065)--(3.458,5.066)--(3.459,5.066)--(3.459,5.067)%
  --(3.460,5.067)--(3.460,5.068)--(3.461,5.068)--(3.461,5.069)--(3.462,5.069)--(3.463,5.069)%
  --(3.463,5.070)--(3.464,5.070)--(3.464,5.071)--(3.465,5.071)--(3.465,5.072)--(3.466,5.072)%
  --(3.466,5.073)--(3.467,5.073)--(3.468,5.073)--(3.468,5.074)--(3.469,5.074)--(3.469,5.075)%
  --(3.470,5.075)--(3.470,5.076)--(3.471,5.076)--(3.471,5.077)--(3.472,5.077)--(3.473,5.077)%
  --(3.473,5.078)--(3.474,5.078)--(3.474,5.079)--(3.475,5.079)--(3.475,5.080)--(3.476,5.080)%
  --(3.476,5.081)--(3.477,5.081)--(3.478,5.081)--(3.478,5.082)--(3.479,5.082)--(3.479,5.083)%
  --(3.480,5.083)--(3.480,5.084)--(3.481,5.084)--(3.482,5.084)--(3.482,5.085)--(3.483,5.085)%
  --(3.483,5.086)--(3.484,5.086)--(3.484,5.087)--(3.485,5.087)--(3.485,5.088)--(3.486,5.088)%
  --(3.487,5.088)--(3.487,5.089)--(3.488,5.089)--(3.488,5.090)--(3.489,5.090)--(3.489,5.091)%
  --(3.490,5.091)--(3.491,5.092)--(3.492,5.092)--(3.492,5.093)--(3.493,5.093)--(3.493,5.094)%
  --(3.494,5.094)--(3.494,5.095)--(3.495,5.095)--(3.496,5.095)--(3.496,5.096)--(3.497,5.096)%
  --(3.497,5.097)--(3.498,5.097)--(3.498,5.098)--(3.499,5.098)--(3.500,5.099)--(3.501,5.099)%
  --(3.501,5.100)--(3.502,5.100)--(3.502,5.101)--(3.503,5.101)--(3.503,5.102)--(3.504,5.102)%
  --(3.505,5.102)--(3.505,5.103)--(3.506,5.103)--(3.506,5.104)--(3.507,5.104)--(3.507,5.105)%
  --(3.508,5.105)--(3.509,5.105)--(3.509,5.106)--(3.510,5.106)--(3.510,5.107)--(3.511,5.107)%
  --(3.511,5.108)--(3.512,5.108)--(3.513,5.108)--(3.513,5.109)--(3.514,5.109)--(3.514,5.110)%
  --(3.515,5.110)--(3.515,5.111)--(3.516,5.111)--(3.516,5.112)--(3.517,5.112)--(3.518,5.112)%
  --(3.518,5.113)--(3.519,5.113)--(3.519,5.114)--(3.520,5.114)--(3.520,5.115)--(3.521,5.115)%
  --(3.522,5.115)--(3.522,5.116)--(3.523,5.116)--(3.523,5.117)--(3.524,5.117)--(3.524,5.118)%
  --(3.525,5.118)--(3.526,5.118)--(3.526,5.119)--(3.527,5.119)--(3.527,5.120)--(3.528,5.120)%
  --(3.528,5.121)--(3.529,5.121)--(3.530,5.121)--(3.530,5.122)--(3.531,5.122)--(3.531,5.123)%
  --(3.532,5.123)--(3.532,5.124)--(3.533,5.124)--(3.534,5.124)--(3.534,5.125)--(3.535,5.125)%
  --(3.535,5.126)--(3.536,5.126)--(3.536,5.127)--(3.537,5.127)--(3.538,5.127)--(3.538,5.128)%
  --(3.539,5.128)--(3.539,5.129)--(3.540,5.129)--(3.540,5.130)--(3.541,5.130)--(3.542,5.130)%
  --(3.542,5.131)--(3.543,5.131)--(3.543,5.132)--(3.544,5.132)--(3.544,5.133)--(3.545,5.133)%
  --(3.546,5.133)--(3.546,5.134)--(3.547,5.134)--(3.547,5.135)--(3.548,5.135)--(3.548,5.136)%
  --(3.549,5.136)--(3.550,5.136)--(3.550,5.137)--(3.551,5.137)--(3.551,5.138)--(3.552,5.138)%
  --(3.553,5.139)--(3.554,5.139)--(3.554,5.140)--(3.555,5.140)--(3.555,5.141)--(3.556,5.141)%
  --(3.557,5.141)--(3.557,5.142)--(3.558,5.142)--(3.558,5.143)--(3.559,5.143)--(3.559,5.144)%
  --(3.560,5.144)--(3.561,5.144)--(3.561,5.145)--(3.562,5.145)--(3.562,5.146)--(3.563,5.146)%
  --(3.563,5.147)--(3.564,5.147)--(3.565,5.147)--(3.565,5.148)--(3.566,5.148)--(3.566,5.149)%
  --(3.567,5.149)--(3.568,5.150)--(3.569,5.150)--(3.569,5.151)--(3.570,5.151)--(3.570,5.152)%
  --(3.571,5.152)--(3.572,5.152)--(3.572,5.153)--(3.573,5.153)--(3.573,5.154)--(3.574,5.154)%
  --(3.574,5.155)--(3.575,5.155)--(3.576,5.155)--(3.576,5.156)--(3.577,5.156)--(3.577,5.157)%
  --(3.578,5.157)--(3.579,5.157)--(3.579,5.158)--(3.580,5.158)--(3.580,5.159)--(3.581,5.159)%
  --(3.581,5.160)--(3.582,5.160)--(3.583,5.160)--(3.583,5.161)--(3.584,5.161)--(3.584,5.162)%
  --(3.585,5.162)--(3.586,5.163)--(3.587,5.163)--(3.587,5.164)--(3.588,5.164)--(3.588,5.165)%
  --(3.589,5.165)--(3.590,5.165)--(3.590,5.166)--(3.591,5.166)--(3.591,5.167)--(3.592,5.167)%
  --(3.593,5.168)--(3.594,5.168)--(3.594,5.169)--(3.595,5.169)--(3.595,5.170)--(3.596,5.170)%
  --(3.597,5.170)--(3.597,5.171)--(3.598,5.171)--(3.598,5.172)--(3.599,5.172)--(3.600,5.172)%
  --(3.600,5.173)--(3.601,5.173)--(3.601,5.174)--(3.602,5.174)--(3.602,5.175)--(3.603,5.175)%
  --(3.604,5.175)--(3.604,5.176)--(3.605,5.176)--(3.605,5.177)--(3.606,5.177)--(3.607,5.177)%
  --(3.607,5.178)--(3.608,5.178)--(3.608,5.179)--(3.609,5.179)--(3.609,5.180)--(3.610,5.180)%
  --(3.611,5.180)--(3.611,5.181)--(3.612,5.181)--(3.612,5.182)--(3.613,5.182)--(3.614,5.182)%
  --(3.614,5.183)--(3.615,5.183)--(3.615,5.184)--(3.616,5.184)--(3.617,5.184)--(3.617,5.185)%
  --(3.618,5.185)--(3.618,5.186)--(3.619,5.186)--(3.620,5.187)--(3.621,5.187)--(3.621,5.188)%
  --(3.622,5.188)--(3.622,5.189)--(3.623,5.189)--(3.624,5.189)--(3.624,5.190)--(3.625,5.190)%
  --(3.625,5.191)--(3.626,5.191)--(3.627,5.191)--(3.627,5.192)--(3.628,5.192)--(3.628,5.193)%
  --(3.629,5.193)--(3.630,5.193)--(3.630,5.194)--(3.631,5.194)--(3.631,5.195)--(3.632,5.195)%
  --(3.633,5.196)--(3.634,5.196)--(3.634,5.197)--(3.635,5.197)--(3.635,5.198)--(3.636,5.198)%
  --(3.637,5.198)--(3.637,5.199)--(3.638,5.199)--(3.638,5.200)--(3.639,5.200)--(3.640,5.200)%
  --(3.640,5.201)--(3.641,5.201)--(3.641,5.202)--(3.642,5.202)--(3.643,5.202)--(3.643,5.203)%
  --(3.644,5.203)--(3.644,5.204)--(3.645,5.204)--(3.646,5.204)--(3.646,5.205)--(3.647,5.205)%
  --(3.647,5.206)--(3.648,5.206)--(3.649,5.206)--(3.649,5.207)--(3.650,5.207)--(3.650,5.208)%
  --(3.651,5.208)--(3.652,5.208)--(3.652,5.209)--(3.653,5.209)--(3.653,5.210)--(3.654,5.210)%
  --(3.655,5.211)--(3.656,5.211)--(3.656,5.212)--(3.657,5.212)--(3.657,5.213)--(3.658,5.213)%
  --(3.659,5.213)--(3.659,5.214)--(3.660,5.214)--(3.660,5.215)--(3.661,5.215)--(3.662,5.215)%
  --(3.662,5.216)--(3.663,5.216)--(3.663,5.217)--(3.664,5.217)--(3.665,5.217)--(3.665,5.218)%
  --(3.666,5.218)--(3.666,5.219)--(3.667,5.219)--(3.668,5.219)--(3.668,5.220)--(3.669,5.220)%
  --(3.669,5.221)--(3.670,5.221)--(3.671,5.221)--(3.671,5.222)--(3.672,5.222)--(3.672,5.223)%
  --(3.673,5.223)--(3.674,5.223)--(3.674,5.224)--(3.675,5.224)--(3.675,5.225)--(3.676,5.225)%
  --(3.677,5.225)--(3.677,5.226)--(3.678,5.226)--(3.678,5.227)--(3.679,5.227)--(3.680,5.227)%
  --(3.680,5.228)--(3.681,5.228)--(3.681,5.229)--(3.682,5.229)--(3.683,5.229)--(3.683,5.230)%
  --(3.684,5.230)--(3.685,5.231)--(3.686,5.231)--(3.686,5.232)--(3.687,5.232)--(3.688,5.233)%
  --(3.689,5.233)--(3.689,5.234)--(3.690,5.234)--(3.691,5.234)--(3.691,5.235)--(3.692,5.235)%
  --(3.692,5.236)--(3.693,5.236)--(3.694,5.236)--(3.694,5.237)--(3.695,5.237)--(3.695,5.238)%
  --(3.696,5.238)--(3.697,5.238)--(3.697,5.239)--(3.698,5.239)--(3.698,5.240)--(3.699,5.240)%
  --(3.700,5.240)--(3.700,5.241)--(3.701,5.241)--(3.701,5.242)--(3.702,5.242)--(3.703,5.242)%
  --(3.703,5.243)--(3.704,5.243)--(3.704,5.244)--(3.705,5.244)--(3.706,5.244)--(3.706,5.245)%
  --(3.707,5.245)--(3.708,5.246)--(3.709,5.246)--(3.709,5.247)--(3.710,5.247)--(3.711,5.247)%
  --(3.711,5.248)--(3.712,5.248)--(3.712,5.249)--(3.713,5.249)--(3.714,5.249)--(3.714,5.250)%
  --(3.715,5.250)--(3.715,5.251)--(3.716,5.251)--(3.717,5.251)--(3.717,5.252)--(3.718,5.252)%
  --(3.718,5.253)--(3.719,5.253)--(3.720,5.253)--(3.720,5.254)--(3.721,5.254)--(3.722,5.254)%
  --(3.722,5.255)--(3.723,5.255)--(3.723,5.256)--(3.724,5.256)--(3.725,5.256)--(3.725,5.257)%
  --(3.726,5.257)--(3.726,5.258)--(3.727,5.258)--(3.728,5.258)--(3.728,5.259)--(3.729,5.259)%
  --(3.729,5.260)--(3.730,5.260)--(3.731,5.260)--(3.731,5.261)--(3.732,5.261)--(3.733,5.261)%
  --(3.733,5.262)--(3.734,5.262)--(3.734,5.263)--(3.735,5.263)--(3.736,5.263)--(3.736,5.264)%
  --(3.737,5.264)--(3.737,5.265)--(3.738,5.265)--(3.739,5.265)--(3.739,5.266)--(3.740,5.266)%
  --(3.741,5.266)--(3.741,5.267)--(3.742,5.267)--(3.742,5.268)--(3.743,5.268)--(3.744,5.268)%
  --(3.744,5.269)--(3.745,5.269)--(3.745,5.270)--(3.746,5.270)--(3.747,5.270)--(3.747,5.271)%
  --(3.748,5.271)--(3.749,5.271)--(3.749,5.272)--(3.750,5.272)--(3.750,5.273)--(3.751,5.273)%
  --(3.752,5.273)--(3.752,5.274)--(3.753,5.274)--(3.753,5.275)--(3.754,5.275)--(3.755,5.275)%
  --(3.755,5.276)--(3.756,5.276)--(3.757,5.276)--(3.757,5.277)--(3.758,5.277)--(3.758,5.278)%
  --(3.759,5.278)--(3.760,5.278)--(3.760,5.279)--(3.761,5.279)--(3.762,5.279)--(3.762,5.280)%
  --(3.763,5.280)--(3.763,5.281)--(3.764,5.281)--(3.765,5.281)--(3.765,5.282)--(3.766,5.282)%
  --(3.766,5.283)--(3.767,5.283)--(3.768,5.283)--(3.768,5.284)--(3.769,5.284)--(3.770,5.284)%
  --(3.770,5.285)--(3.771,5.285)--(3.771,5.286)--(3.772,5.286)--(3.773,5.286)--(3.773,5.287)%
  --(3.774,5.287)--(3.775,5.287)--(3.775,5.288)--(3.776,5.288)--(3.776,5.289)--(3.777,5.289)%
  --(3.778,5.289)--(3.778,5.290)--(3.779,5.290)--(3.780,5.290)--(3.780,5.291)--(3.781,5.291)%
  --(3.781,5.292)--(3.782,5.292)--(3.783,5.292)--(3.783,5.293)--(3.784,5.293)--(3.785,5.294)%
  --(3.786,5.294)--(3.786,5.295)--(3.787,5.295)--(3.788,5.295)--(3.788,5.296)--(3.789,5.296)%
  --(3.790,5.297)--(3.791,5.297)--(3.791,5.298)--(3.792,5.298)--(3.793,5.298)--(3.793,5.299)%
  --(3.794,5.299)--(3.795,5.300)--(3.796,5.300)--(3.796,5.301)--(3.797,5.301)--(3.798,5.301)%
  --(3.798,5.302)--(3.799,5.302)--(3.800,5.303)--(3.801,5.303)--(3.801,5.304)--(3.802,5.304)%
  --(3.803,5.304)--(3.803,5.305)--(3.804,5.305)--(3.805,5.305)--(3.805,5.306)--(3.806,5.306)%
  --(3.806,5.307)--(3.807,5.307)--(3.808,5.307)--(3.808,5.308)--(3.809,5.308)--(3.810,5.308)%
  --(3.810,5.309)--(3.811,5.309)--(3.811,5.310)--(3.812,5.310)--(3.813,5.310)--(3.813,5.311)%
  --(3.814,5.311)--(3.815,5.311)--(3.815,5.312)--(3.816,5.312)--(3.817,5.313)--(3.818,5.313)%
  --(3.818,5.314)--(3.819,5.314)--(3.820,5.314)--(3.820,5.315)--(3.821,5.315)--(3.822,5.315)%
  --(3.822,5.316)--(3.823,5.316)--(3.823,5.317)--(3.824,5.317)--(3.825,5.317)--(3.825,5.318)%
  --(3.826,5.318)--(3.827,5.318)--(3.827,5.319)--(3.828,5.319)--(3.828,5.320)--(3.829,5.320)%
  --(3.830,5.320)--(3.830,5.321)--(3.831,5.321)--(3.832,5.321)--(3.832,5.322)--(3.833,5.322)%
  --(3.834,5.322)--(3.834,5.323)--(3.835,5.323)--(3.835,5.324)--(3.836,5.324)--(3.837,5.324)%
  --(3.837,5.325)--(3.838,5.325)--(3.839,5.325)--(3.839,5.326)--(3.840,5.326)--(3.841,5.326)%
  --(3.841,5.327)--(3.842,5.327)--(3.842,5.328)--(3.843,5.328)--(3.844,5.328)--(3.844,5.329)%
  --(3.845,5.329)--(3.846,5.329)--(3.846,5.330)--(3.847,5.330)--(3.848,5.330)--(3.848,5.331)%
  --(3.849,5.331)--(3.849,5.332)--(3.850,5.332)--(3.851,5.332)--(3.851,5.333)--(3.852,5.333)%
  --(3.853,5.333)--(3.853,5.334)--(3.854,5.334)--(3.855,5.334)--(3.855,5.335)--(3.856,5.335)%
  --(3.856,5.336)--(3.857,5.336)--(3.858,5.336)--(3.858,5.337)--(3.859,5.337)--(3.860,5.337)%
  --(3.860,5.338)--(3.861,5.338)--(3.862,5.338)--(3.862,5.339)--(3.863,5.339)--(3.864,5.340)%
  --(3.865,5.340)--(3.865,5.341)--(3.866,5.341)--(3.867,5.341)--(3.867,5.342)--(3.868,5.342)%
  --(3.869,5.342)--(3.869,5.343)--(3.870,5.343)--(3.871,5.343)--(3.871,5.344)--(3.872,5.344)%
  --(3.872,5.345)--(3.873,5.345)--(3.874,5.345)--(3.874,5.346)--(3.875,5.346)--(3.876,5.346)%
  --(3.876,5.347)--(3.877,5.347)--(3.878,5.347)--(3.878,5.348)--(3.879,5.348)--(3.880,5.348)%
  --(3.880,5.349)--(3.881,5.349)--(3.881,5.350)--(3.882,5.350)--(3.883,5.350)--(3.883,5.351)%
  --(3.884,5.351)--(3.885,5.351)--(3.885,5.352)--(3.886,5.352)--(3.887,5.352)--(3.887,5.353)%
  --(3.888,5.353)--(3.889,5.353)--(3.889,5.354)--(3.890,5.354)--(3.890,5.355)--(3.891,5.355)%
  --(3.892,5.355)--(3.892,5.356)--(3.893,5.356)--(3.894,5.356)--(3.894,5.357)--(3.895,5.357)%
  --(3.896,5.357)--(3.896,5.358)--(3.897,5.358)--(3.898,5.358)--(3.898,5.359)--(3.899,5.359)%
  --(3.900,5.359)--(3.900,5.360)--(3.901,5.360)--(3.901,5.361)--(3.902,5.361)--(3.903,5.361)%
  --(3.903,5.362)--(3.904,5.362)--(3.905,5.362)--(3.905,5.363)--(3.906,5.363)--(3.907,5.363)%
  --(3.907,5.364)--(3.908,5.364)--(3.909,5.364)--(3.909,5.365)--(3.910,5.365)--(3.911,5.365)%
  --(3.911,5.366)--(3.912,5.366)--(3.913,5.367)--(3.914,5.367)--(3.914,5.368)--(3.915,5.368)%
  --(3.916,5.368)--(3.916,5.369)--(3.917,5.369)--(3.918,5.369)--(3.918,5.370)--(3.919,5.370)%
  --(3.920,5.370)--(3.920,5.371)--(3.921,5.371)--(3.922,5.371)--(3.922,5.372)--(3.923,5.372)%
  --(3.924,5.372)--(3.924,5.373)--(3.925,5.373)--(3.926,5.373)--(3.926,5.374)--(3.927,5.374)%
  --(3.927,5.375)--(3.928,5.375)--(3.929,5.375)--(3.929,5.376)--(3.930,5.376)--(3.931,5.376)%
  --(3.931,5.377)--(3.932,5.377)--(3.933,5.377)--(3.933,5.378)--(3.934,5.378)--(3.935,5.378)%
  --(3.935,5.379)--(3.936,5.379)--(3.937,5.379)--(3.937,5.380)--(3.938,5.380)--(3.939,5.380)%
  --(3.939,5.381)--(3.940,5.381)--(3.941,5.381)--(3.941,5.382)--(3.942,5.382)--(3.943,5.382)%
  --(3.943,5.383)--(3.944,5.383)--(3.945,5.383)--(3.945,5.384)--(3.946,5.384)--(3.946,5.385)%
  --(3.947,5.385)--(3.948,5.385)--(3.948,5.386)--(3.949,5.386)--(3.950,5.386)--(3.950,5.387)%
  --(3.951,5.387)--(3.952,5.387)--(3.952,5.388)--(3.953,5.388)--(3.954,5.388)--(3.954,5.389)%
  --(3.955,5.389)--(3.956,5.389)--(3.956,5.390)--(3.957,5.390)--(3.958,5.390)--(3.958,5.391)%
  --(3.959,5.391)--(3.960,5.391)--(3.960,5.392)--(3.961,5.392)--(3.962,5.392)--(3.962,5.393)%
  --(3.963,5.393)--(3.964,5.393)--(3.964,5.394)--(3.965,5.394)--(3.966,5.394)--(3.966,5.395)%
  --(3.967,5.395)--(3.968,5.395)--(3.968,5.396)--(3.969,5.396)--(3.970,5.396)--(3.970,5.397)%
  --(3.971,5.397)--(3.972,5.397)--(3.972,5.398)--(3.973,5.398)--(3.974,5.398)--(3.974,5.399)%
  --(3.975,5.399)--(3.976,5.400)--(3.977,5.400)--(3.977,5.401)--(3.978,5.401)--(3.979,5.401)%
  --(3.979,5.402)--(3.980,5.402)--(3.981,5.402)--(3.981,5.403)--(3.982,5.403)--(3.983,5.403)%
  --(3.983,5.404)--(3.984,5.404)--(3.985,5.404)--(3.985,5.405)--(3.986,5.405)--(3.987,5.405)%
  --(3.987,5.406)--(3.988,5.406)--(3.989,5.406)--(3.989,5.407)--(3.990,5.407)--(3.991,5.407)%
  --(3.991,5.408)--(3.992,5.408)--(3.993,5.408)--(3.993,5.409)--(3.994,5.409)--(3.995,5.409)%
  --(3.995,5.410)--(3.996,5.410)--(3.997,5.410)--(3.997,5.411)--(3.998,5.411)--(3.999,5.411)%
  --(3.999,5.412)--(4.000,5.412)--(4.001,5.412)--(4.001,5.413)--(4.002,5.413)--(4.003,5.413)%
  --(4.003,5.414)--(4.004,5.414)--(4.005,5.414)--(4.005,5.415)--(4.006,5.415)--(4.007,5.415)%
  --(4.007,5.416)--(4.008,5.416)--(4.009,5.416)--(4.009,5.417)--(4.010,5.417)--(4.011,5.417)%
  --(4.011,5.418)--(4.012,5.418)--(4.013,5.418)--(4.013,5.419)--(4.014,5.419)--(4.015,5.419)%
  --(4.015,5.420)--(4.016,5.420)--(4.017,5.420)--(4.017,5.421)--(4.018,5.421)--(4.019,5.421)%
  --(4.020,5.422)--(4.021,5.422)--(4.022,5.422)--(4.022,5.423)--(4.023,5.423)--(4.024,5.423)%
  --(4.024,5.424)--(4.025,5.424)--(4.026,5.424)--(4.026,5.425)--(4.027,5.425)--(4.028,5.425)%
  --(4.028,5.426)--(4.029,5.426)--(4.030,5.426)--(4.030,5.427)--(4.031,5.427)--(4.032,5.427)%
  --(4.032,5.428)--(4.033,5.428)--(4.034,5.428)--(4.034,5.429)--(4.035,5.429)--(4.036,5.429)%
  --(4.036,5.430)--(4.037,5.430)--(4.038,5.430)--(4.038,5.431)--(4.039,5.431)--(4.040,5.431)%
  --(4.040,5.432)--(4.041,5.432)--(4.042,5.432)--(4.042,5.433)--(4.043,5.433)--(4.044,5.433)%
  --(4.044,5.434)--(4.045,5.434)--(4.046,5.434)--(4.046,5.435)--(4.047,5.435)--(4.048,5.435)%
  --(4.048,5.436)--(4.049,5.436)--(4.050,5.436)--(4.051,5.436)--(4.051,5.437)--(4.052,5.437)%
  --(4.053,5.437)--(4.053,5.438)--(4.054,5.438)--(4.055,5.438)--(4.055,5.439)--(4.056,5.439)%
  --(4.057,5.439)--(4.057,5.440)--(4.058,5.440)--(4.059,5.440)--(4.059,5.441)--(4.060,5.441)%
  --(4.061,5.441)--(4.061,5.442)--(4.062,5.442)--(4.063,5.442)--(4.063,5.443)--(4.064,5.443)%
  --(4.065,5.443)--(4.065,5.444)--(4.066,5.444)--(4.067,5.444)--(4.067,5.445)--(4.068,5.445)%
  --(4.069,5.445)--(4.070,5.445)--(4.070,5.446)--(4.071,5.446)--(4.072,5.446)--(4.072,5.447)%
  --(4.073,5.447)--(4.074,5.447)--(4.074,5.448)--(4.075,5.448)--(4.076,5.448)--(4.076,5.449)%
  --(4.077,5.449)--(4.078,5.449)--(4.078,5.450)--(4.079,5.450)--(4.080,5.450)--(4.080,5.451)%
  --(4.081,5.451)--(4.082,5.451)--(4.082,5.452)--(4.083,5.452)--(4.084,5.452)--(4.085,5.452)%
  --(4.085,5.453)--(4.086,5.453)--(4.087,5.453)--(4.087,5.454)--(4.088,5.454)--(4.089,5.454)%
  --(4.089,5.455)--(4.090,5.455)--(4.091,5.455)--(4.091,5.456)--(4.092,5.456)--(4.093,5.456)%
  --(4.093,5.457)--(4.094,5.457)--(4.095,5.457)--(4.095,5.458)--(4.096,5.458)--(4.097,5.458)%
  --(4.098,5.458)--(4.098,5.459)--(4.099,5.459)--(4.100,5.459)--(4.100,5.460)--(4.101,5.460)%
  --(4.102,5.460)--(4.102,5.461)--(4.103,5.461)--(4.104,5.461)--(4.104,5.462)--(4.105,5.462)%
  --(4.106,5.462)--(4.106,5.463)--(4.107,5.463)--(4.108,5.463)--(4.109,5.463)--(4.109,5.464)%
  --(4.110,5.464)--(4.111,5.464)--(4.111,5.465)--(4.112,5.465)--(4.113,5.465)--(4.113,5.466)%
  --(4.114,5.466)--(4.115,5.466)--(4.115,5.467)--(4.116,5.467)--(4.117,5.467)--(4.117,5.468)%
  --(4.118,5.468)--(4.119,5.468)--(4.120,5.468)--(4.120,5.469)--(4.121,5.469)--(4.122,5.469)%
  --(4.122,5.470)--(4.123,5.470)--(4.124,5.470)--(4.124,5.471)--(4.125,5.471)--(4.126,5.471)%
  --(4.126,5.472)--(4.127,5.472)--(4.128,5.472)--(4.129,5.472)--(4.129,5.473)--(4.130,5.473)%
  --(4.131,5.473)--(4.131,5.474)--(4.132,5.474)--(4.133,5.474)--(4.133,5.475)--(4.134,5.475)%
  --(4.135,5.475)--(4.135,5.476)--(4.136,5.476)--(4.137,5.476)--(4.138,5.476)--(4.138,5.477)%
  --(4.139,5.477)--(4.140,5.477)--(4.140,5.478)--(4.141,5.478)--(4.142,5.478)--(4.142,5.479)%
  --(4.143,5.479)--(4.144,5.479)--(4.144,5.480)--(4.145,5.480)--(4.146,5.480)--(4.147,5.480)%
  --(4.147,5.481)--(4.148,5.481)--(4.149,5.481)--(4.149,5.482)--(4.150,5.482)--(4.151,5.482)%
  --(4.151,5.483)--(4.152,5.483)--(4.153,5.483)--(4.154,5.483)--(4.154,5.484)--(4.155,5.484)%
  --(4.156,5.484)--(4.156,5.485)--(4.157,5.485)--(4.158,5.485)--(4.158,5.486)--(4.159,5.486)%
  --(4.160,5.486)--(4.161,5.487)--(4.162,5.487)--(4.163,5.487)--(4.163,5.488)--(4.164,5.488)%
  --(4.165,5.488)--(4.165,5.489)--(4.166,5.489)--(4.167,5.489)--(4.167,5.490)--(4.168,5.490)%
  --(4.169,5.490)--(4.170,5.490)--(4.170,5.491)--(4.171,5.491)--(4.172,5.491)--(4.172,5.492)%
  --(4.173,5.492)--(4.174,5.492)--(4.174,5.493)--(4.175,5.493)--(4.176,5.493)--(4.177,5.493)%
  --(4.177,5.494)--(4.178,5.494)--(4.179,5.494)--(4.179,5.495)--(4.180,5.495)--(4.181,5.495)%
  --(4.182,5.496)--(4.183,5.496)--(4.184,5.496)--(4.184,5.497)--(4.185,5.497)--(4.186,5.497)%
  --(4.186,5.498)--(4.187,5.498)--(4.188,5.498)--(4.189,5.498)--(4.189,5.499)--(4.190,5.499)%
  --(4.191,5.499)--(4.191,5.500)--(4.192,5.500)--(4.193,5.500)--(4.193,5.501)--(4.194,5.501)%
  --(4.195,5.501)--(4.196,5.501)--(4.196,5.502)--(4.197,5.502)--(4.198,5.502)--(4.198,5.503)%
  --(4.199,5.503)--(4.200,5.503)--(4.201,5.504)--(4.202,5.504)--(4.203,5.504)--(4.203,5.505)%
  --(4.204,5.505)--(4.205,5.505)--(4.205,5.506)--(4.206,5.506)--(4.207,5.506)--(4.208,5.506)%
  --(4.208,5.507)--(4.209,5.507)--(4.210,5.507)--(4.210,5.508)--(4.211,5.508)--(4.212,5.508)%
  --(4.213,5.509)--(4.214,5.509)--(4.215,5.509)--(4.215,5.510)--(4.216,5.510)--(4.217,5.510)%
  --(4.217,5.511)--(4.218,5.511)--(4.219,5.511)--(4.220,5.511)--(4.220,5.512)--(4.221,5.512)%
  --(4.222,5.512)--(4.222,5.513)--(4.223,5.513)--(4.224,5.513)--(4.225,5.513)--(4.225,5.514)%
  --(4.226,5.514)--(4.227,5.514)--(4.227,5.515)--(4.228,5.515)--(4.229,5.515)--(4.230,5.516)%
  --(4.231,5.516)--(4.232,5.516)--(4.232,5.517)--(4.233,5.517)--(4.234,5.517)--(4.235,5.518)%
  --(4.236,5.518)--(4.237,5.518)--(4.237,5.519)--(4.238,5.519)--(4.239,5.519)--(4.239,5.520)%
  --(4.240,5.520)--(4.241,5.520)--(4.242,5.520)--(4.242,5.521)--(4.243,5.521)--(4.244,5.521)%
  --(4.244,5.522)--(4.245,5.522)--(4.246,5.522)--(4.247,5.522)--(4.247,5.523)--(4.248,5.523)%
  --(4.249,5.523)--(4.249,5.524)--(4.250,5.524)--(4.251,5.524)--(4.252,5.524)--(4.252,5.525)%
  --(4.253,5.525)--(4.254,5.525)--(4.254,5.526)--(4.255,5.526)--(4.256,5.526)--(4.257,5.526)%
  --(4.257,5.527)--(4.258,5.527)--(4.259,5.527)--(4.259,5.528)--(4.260,5.528)--(4.261,5.528)%
  --(4.262,5.528)--(4.262,5.529)--(4.263,5.529)--(4.264,5.529)--(4.264,5.530)--(4.265,5.530)%
  --(4.266,5.530)--(4.267,5.530)--(4.267,5.531)--(4.268,5.531)--(4.269,5.531)--(4.270,5.531)%
  --(4.270,5.532)--(4.271,5.532)--(4.272,5.532)--(4.272,5.533)--(4.273,5.533)--(4.274,5.533)%
  --(4.275,5.533)--(4.275,5.534)--(4.276,5.534)--(4.277,5.534)--(4.277,5.535)--(4.278,5.535)%
  --(4.279,5.535)--(4.280,5.535)--(4.280,5.536)--(4.281,5.536)--(4.282,5.536)--(4.282,5.537)%
  --(4.283,5.537)--(4.284,5.537)--(4.285,5.537)--(4.285,5.538)--(4.286,5.538)--(4.287,5.538)%
  --(4.288,5.539)--(4.289,5.539)--(4.290,5.539)--(4.290,5.540)--(4.291,5.540)--(4.292,5.540)%
  --(4.293,5.540)--(4.293,5.541)--(4.294,5.541)--(4.295,5.541)--(4.295,5.542)--(4.296,5.542)%
  --(4.297,5.542)--(4.298,5.542)--(4.298,5.543)--(4.299,5.543)--(4.300,5.543)--(4.301,5.544)%
  --(4.302,5.544)--(4.303,5.544)--(4.303,5.545)--(4.304,5.545)--(4.305,5.545)--(4.306,5.545)%
  --(4.306,5.546)--(4.307,5.546)--(4.308,5.546)--(4.308,5.547)--(4.309,5.547)--(4.310,5.547)%
  --(4.311,5.547)--(4.311,5.548)--(4.312,5.548)--(4.313,5.548)--(4.314,5.548)--(4.314,5.549)%
  --(4.315,5.549)--(4.316,5.549)--(4.316,5.550)--(4.317,5.550)--(4.318,5.550)--(4.319,5.550)%
  --(4.319,5.551)--(4.320,5.551)--(4.321,5.551)--(4.322,5.551)--(4.322,5.552)--(4.323,5.552)%
  --(4.324,5.552)--(4.324,5.553)--(4.325,5.553)--(4.326,5.553)--(4.327,5.553)--(4.327,5.554)%
  --(4.328,5.554)--(4.329,5.554)--(4.330,5.554)--(4.330,5.555)--(4.331,5.555)--(4.332,5.555)%
  --(4.332,5.556)--(4.333,5.556)--(4.334,5.556)--(4.335,5.556)--(4.335,5.557)--(4.336,5.557)%
  --(4.337,5.557)--(4.338,5.557)--(4.338,5.558)--(4.339,5.558)--(4.340,5.558)--(4.340,5.559)%
  --(4.341,5.559)--(4.342,5.559)--(4.343,5.559)--(4.343,5.560)--(4.344,5.560)--(4.345,5.560)%
  --(4.346,5.560)--(4.346,5.561)--(4.347,5.561)--(4.348,5.561)--(4.349,5.562)--(4.350,5.562)%
  --(4.351,5.562)--(4.351,5.563)--(4.352,5.563)--(4.353,5.563)--(4.354,5.563)--(4.354,5.564)%
  --(4.355,5.564)--(4.356,5.564)--(4.357,5.564)--(4.357,5.565)--(4.358,5.565)--(4.359,5.565)%
  --(4.360,5.566)--(4.361,5.566)--(4.362,5.566)--(4.362,5.567)--(4.363,5.567)--(4.364,5.567)%
  --(4.365,5.567)--(4.365,5.568)--(4.366,5.568)--(4.367,5.568)--(4.368,5.568)--(4.368,5.569)%
  --(4.369,5.569)--(4.370,5.569)--(4.371,5.569)--(4.371,5.570)--(4.372,5.570)--(4.373,5.570)%
  --(4.373,5.571)--(4.374,5.571)--(4.375,5.571)--(4.376,5.571)--(4.376,5.572)--(4.377,5.572)%
  --(4.378,5.572)--(4.379,5.572)--(4.379,5.573)--(4.380,5.573)--(4.381,5.573)--(4.382,5.573)%
  --(4.382,5.574)--(4.383,5.574)--(4.384,5.574)--(4.385,5.574)--(4.385,5.575)--(4.386,5.575)%
  --(4.387,5.575)--(4.387,5.576)--(4.388,5.576)--(4.389,5.576)--(4.390,5.576)--(4.390,5.577)%
  --(4.391,5.577)--(4.392,5.577)--(4.393,5.577)--(4.393,5.578)--(4.394,5.578)--(4.395,5.578)%
  --(4.396,5.578)--(4.396,5.579)--(4.397,5.579)--(4.398,5.579)--(4.399,5.579)--(4.399,5.580)%
  --(4.400,5.580)--(4.401,5.580)--(4.402,5.580)--(4.402,5.581)--(4.403,5.581)--(4.404,5.581)%
  --(4.404,5.582)--(4.405,5.582)--(4.406,5.582)--(4.407,5.582)--(4.407,5.583)--(4.408,5.583)%
  --(4.409,5.583)--(4.410,5.583)--(4.410,5.584)--(4.411,5.584)--(4.412,5.584)--(4.413,5.584)%
  --(4.413,5.585)--(4.414,5.585)--(4.415,5.585)--(4.416,5.585)--(4.416,5.586)--(4.417,5.586)%
  --(4.418,5.586)--(4.419,5.586)--(4.419,5.587)--(4.420,5.587)--(4.421,5.587)--(4.422,5.587)%
  --(4.422,5.588)--(4.423,5.588)--(4.424,5.588)--(4.425,5.588)--(4.425,5.589)--(4.426,5.589)%
  --(4.427,5.589)--(4.428,5.589)--(4.428,5.590)--(4.429,5.590)--(4.430,5.590)--(4.431,5.591)%
  --(4.432,5.591)--(4.433,5.591)--(4.433,5.592)--(4.434,5.592)--(4.435,5.592)--(4.436,5.592)%
  --(4.436,5.593)--(4.437,5.593)--(4.438,5.593)--(4.439,5.593)--(4.439,5.594)--(4.440,5.594)%
  --(4.441,5.594)--(4.442,5.594)--(4.442,5.595)--(4.443,5.595)--(4.444,5.595)--(4.445,5.595)%
  --(4.445,5.596)--(4.446,5.596)--(4.447,5.596)--(4.448,5.596)--(4.448,5.597)--(4.449,5.597)%
  --(4.450,5.597)--(4.451,5.597)--(4.451,5.598)--(4.452,5.598)--(4.453,5.598)--(4.454,5.598)%
  --(4.454,5.599)--(4.455,5.599)--(4.456,5.599)--(4.457,5.599)--(4.457,5.600)--(4.458,5.600)%
  --(4.459,5.600)--(4.460,5.600)--(4.460,5.601)--(4.461,5.601)--(4.462,5.601)--(4.463,5.601)%
  --(4.463,5.602)--(4.464,5.602)--(4.465,5.602)--(4.466,5.602)--(4.466,5.603)--(4.467,5.603)%
  --(4.468,5.603)--(4.469,5.603)--(4.469,5.604)--(4.470,5.604)--(4.471,5.604)--(4.472,5.604)%
  --(4.472,5.605)--(4.473,5.605)--(4.474,5.605)--(4.475,5.605)--(4.475,5.606)--(4.476,5.606)%
  --(4.477,5.606)--(4.478,5.606)--(4.478,5.607)--(4.479,5.607)--(4.480,5.607)--(4.481,5.607)%
  --(4.481,5.608)--(4.482,5.608)--(4.483,5.608)--(4.484,5.608)--(4.485,5.608)--(4.485,5.609)%
  --(4.486,5.609)--(4.487,5.609)--(4.488,5.609)--(4.488,5.610)--(4.489,5.610)--(4.490,5.610)%
  --(4.491,5.610)--(4.491,5.611)--(4.492,5.611)--(4.493,5.611)--(4.494,5.611)--(4.494,5.612)%
  --(4.495,5.612)--(4.496,5.612)--(4.497,5.612)--(4.497,5.613)--(4.498,5.613)--(4.499,5.613)%
  --(4.500,5.613)--(4.500,5.614)--(4.501,5.614)--(4.502,5.614)--(4.503,5.614)--(4.503,5.615)%
  --(4.504,5.615)--(4.505,5.615)--(4.506,5.615)--(4.506,5.616)--(4.507,5.616)--(4.508,5.616)%
  --(4.509,5.616)--(4.509,5.617)--(4.510,5.617)--(4.511,5.617)--(4.512,5.617)--(4.513,5.617)%
  --(4.513,5.618)--(4.514,5.618)--(4.515,5.618)--(4.516,5.618)--(4.516,5.619)--(4.517,5.619)%
  --(4.518,5.619)--(4.519,5.619)--(4.519,5.620)--(4.520,5.620)--(4.521,5.620)--(4.522,5.620)%
  --(4.522,5.621)--(4.523,5.621)--(4.524,5.621)--(4.525,5.621)--(4.525,5.622)--(4.526,5.622)%
  --(4.527,5.622)--(4.528,5.622)--(4.528,5.623)--(4.529,5.623)--(4.530,5.623)--(4.531,5.623)%
  --(4.532,5.623)--(4.532,5.624)--(4.533,5.624)--(4.534,5.624)--(4.535,5.624)--(4.535,5.625)%
  --(4.536,5.625)--(4.537,5.625)--(4.538,5.625)--(4.538,5.626)--(4.539,5.626)--(4.540,5.626)%
  --(4.541,5.626)--(4.541,5.627)--(4.542,5.627)--(4.543,5.627)--(4.544,5.627)--(4.544,5.628)%
  --(4.545,5.628)--(4.546,5.628)--(4.547,5.628)--(4.548,5.628)--(4.548,5.629)--(4.549,5.629)%
  --(4.550,5.629)--(4.551,5.629)--(4.551,5.630)--(4.552,5.630)--(4.553,5.630)--(4.554,5.630)%
  --(4.554,5.631)--(4.555,5.631)--(4.556,5.631)--(4.557,5.631)--(4.558,5.632)--(4.559,5.632)%
  --(4.560,5.632)--(4.561,5.632)--(4.561,5.633)--(4.562,5.633)--(4.563,5.633)--(4.564,5.633)%
  --(4.564,5.634)--(4.565,5.634)--(4.566,5.634)--(4.567,5.634)--(4.567,5.635)--(4.568,5.635)%
  --(4.569,5.635)--(4.570,5.635)--(4.571,5.635)--(4.571,5.636)--(4.572,5.636)--(4.573,5.636)%
  --(4.574,5.636)--(4.574,5.637)--(4.575,5.637)--(4.576,5.637)--(4.577,5.637)--(4.577,5.638)%
  --(4.578,5.638)--(4.579,5.638)--(4.580,5.638)--(4.581,5.638)--(4.581,5.639)--(4.582,5.639)%
  --(4.583,5.639)--(4.584,5.639)--(4.584,5.640)--(4.585,5.640)--(4.586,5.640)--(4.587,5.640)%
  --(4.587,5.641)--(4.588,5.641)--(4.589,5.641)--(4.590,5.641)--(4.591,5.641)--(4.591,5.642)%
  --(4.592,5.642)--(4.593,5.642)--(4.594,5.642)--(4.594,5.643)--(4.595,5.643)--(4.596,5.643)%
  --(4.597,5.643)--(4.598,5.644)--(4.599,5.644)--(4.600,5.644)--(4.601,5.644)--(4.601,5.645)%
  --(4.602,5.645)--(4.603,5.645)--(4.604,5.645)--(4.604,5.646)--(4.605,5.646)--(4.606,5.646)%
  --(4.607,5.646)--(4.608,5.646)--(4.608,5.647)--(4.609,5.647)--(4.610,5.647)--(4.611,5.647)%
  --(4.611,5.648)--(4.612,5.648)--(4.613,5.648)--(4.614,5.648)--(4.615,5.648)--(4.615,5.649)%
  --(4.616,5.649)--(4.617,5.649)--(4.618,5.649)--(4.618,5.650)--(4.619,5.650)--(4.620,5.650)%
  --(4.621,5.650)--(4.622,5.650)--(4.622,5.651)--(4.623,5.651)--(4.624,5.651)--(4.625,5.651)%
  --(4.625,5.652)--(4.626,5.652)--(4.627,5.652)--(4.628,5.652)--(4.629,5.653)--(4.630,5.653)%
  --(4.631,5.653)--(4.632,5.653)--(4.632,5.654)--(4.633,5.654)--(4.634,5.654)--(4.635,5.654)%
  --(4.636,5.655)--(4.637,5.655)--(4.638,5.655)--(4.639,5.655)--(4.639,5.656)--(4.640,5.656)%
  --(4.641,5.656)--(4.642,5.656)--(4.643,5.656)--(4.643,5.657)--(4.644,5.657)--(4.645,5.657)%
  --(4.646,5.657)--(4.646,5.658)--(4.647,5.658)--(4.648,5.658)--(4.649,5.658)--(4.650,5.658)%
  --(4.650,5.659)--(4.651,5.659)--(4.652,5.659)--(4.653,5.659)--(4.653,5.660)--(4.654,5.660)%
  --(4.655,5.660)--(4.656,5.660)--(4.657,5.660)--(4.657,5.661)--(4.658,5.661)--(4.659,5.661)%
  --(4.660,5.661)--(4.660,5.662)--(4.661,5.662)--(4.662,5.662)--(4.663,5.662)--(4.664,5.662)%
  --(4.664,5.663)--(4.665,5.663)--(4.666,5.663)--(4.667,5.663)--(4.668,5.663)--(4.668,5.664)%
  --(4.669,5.664)--(4.670,5.664)--(4.671,5.664)--(4.671,5.665)--(4.672,5.665)--(4.673,5.665)%
  --(4.674,5.665)--(4.675,5.665)--(4.675,5.666)--(4.676,5.666)--(4.677,5.666)--(4.678,5.666)%
  --(4.678,5.667)--(4.679,5.667)--(4.680,5.667)--(4.681,5.667)--(4.682,5.667)--(4.682,5.668)%
  --(4.683,5.668)--(4.684,5.668)--(4.685,5.668)--(4.686,5.668)--(4.686,5.669)--(4.687,5.669)%
  --(4.688,5.669)--(4.689,5.669)--(4.689,5.670)--(4.690,5.670)--(4.691,5.670)--(4.692,5.670)%
  --(4.693,5.670)--(4.693,5.671)--(4.694,5.671)--(4.695,5.671)--(4.696,5.671)--(4.697,5.671)%
  --(4.697,5.672)--(4.698,5.672)--(4.699,5.672)--(4.700,5.672)--(4.701,5.673)--(4.702,5.673)%
  --(4.703,5.673)--(4.704,5.673)--(4.704,5.674)--(4.705,5.674)--(4.706,5.674)--(4.707,5.674)%
  --(4.708,5.674)--(4.708,5.675)--(4.709,5.675)--(4.710,5.675)--(4.711,5.675)--(4.712,5.675)%
  --(4.712,5.676)--(4.713,5.676)--(4.714,5.676)--(4.715,5.676)--(4.715,5.677)--(4.716,5.677)%
  --(4.717,5.677)--(4.718,5.677)--(4.719,5.677)--(4.719,5.678)--(4.720,5.678)--(4.721,5.678)%
  --(4.722,5.678)--(4.723,5.678)--(4.723,5.679)--(4.724,5.679)--(4.725,5.679)--(4.726,5.679)%
  --(4.727,5.679)--(4.727,5.680)--(4.728,5.680)--(4.729,5.680)--(4.730,5.680)--(4.731,5.680)%
  --(4.731,5.681)--(4.732,5.681)--(4.733,5.681)--(4.734,5.681)--(4.734,5.682)--(4.735,5.682)%
  --(4.736,5.682)--(4.737,5.682)--(4.738,5.682)--(4.738,5.683)--(4.739,5.683)--(4.740,5.683)%
  --(4.741,5.683)--(4.742,5.683)--(4.742,5.684)--(4.743,5.684)--(4.744,5.684)--(4.745,5.684)%
  --(4.746,5.684)--(4.746,5.685)--(4.747,5.685)--(4.748,5.685)--(4.749,5.685)--(4.750,5.685)%
  --(4.750,5.686)--(4.751,5.686)--(4.752,5.686)--(4.753,5.686)--(4.754,5.686)--(4.754,5.687)%
  --(4.755,5.687)--(4.756,5.687)--(4.757,5.687)--(4.758,5.687)--(4.758,5.688)--(4.759,5.688)%
  --(4.760,5.688)--(4.761,5.688)--(4.762,5.688)--(4.762,5.689)--(4.763,5.689)--(4.764,5.689)%
  --(4.765,5.689)--(4.766,5.690)--(4.767,5.690)--(4.768,5.690)--(4.769,5.690)--(4.769,5.691)%
  --(4.770,5.691)--(4.771,5.691)--(4.772,5.691)--(4.773,5.691)--(4.773,5.692)--(4.774,5.692)%
  --(4.775,5.692)--(4.776,5.692)--(4.777,5.692)--(4.777,5.693)--(4.778,5.693)--(4.779,5.693)%
  --(4.780,5.693)--(4.781,5.693)--(4.781,5.694)--(4.782,5.694)--(4.783,5.694)--(4.784,5.694)%
  --(4.785,5.694)--(4.785,5.695)--(4.786,5.695)--(4.787,5.695)--(4.788,5.695)--(4.789,5.695)%
  --(4.789,5.696)--(4.790,5.696)--(4.791,5.696)--(4.792,5.696)--(4.793,5.696)--(4.793,5.697)%
  --(4.794,5.697)--(4.795,5.697)--(4.796,5.697)--(4.797,5.697)--(4.797,5.698)--(4.798,5.698)%
  --(4.799,5.698)--(4.800,5.698)--(4.801,5.698)--(4.802,5.699)--(4.803,5.699)--(4.804,5.699)%
  --(4.805,5.699)--(4.806,5.699)--(4.806,5.700)--(4.807,5.700)--(4.808,5.700)--(4.809,5.700)%
  --(4.810,5.700)--(4.810,5.701)--(4.811,5.701)--(4.812,5.701)--(4.813,5.701)--(4.814,5.701)%
  --(4.814,5.702)--(4.815,5.702)--(4.816,5.702)--(4.817,5.702)--(4.818,5.702)--(4.818,5.703)%
  --(4.819,5.703)--(4.820,5.703)--(4.821,5.703)--(4.822,5.703)--(4.822,5.704)--(4.823,5.704)%
  --(4.824,5.704)--(4.825,5.704)--(4.826,5.704)--(4.826,5.705)--(4.827,5.705)--(4.828,5.705)%
  --(4.829,5.705)--(4.830,5.705)--(4.830,5.706)--(4.831,5.706)--(4.832,5.706)--(4.833,5.706)%
  --(4.834,5.706)--(4.834,5.707)--(4.835,5.707)--(4.836,5.707)--(4.837,5.707)--(4.838,5.707)%
  --(4.839,5.707)--(4.839,5.708)--(4.840,5.708)--(4.841,5.708)--(4.842,5.708)--(4.843,5.708)%
  --(4.843,5.709)--(4.844,5.709)--(4.845,5.709)--(4.846,5.709)--(4.847,5.709)--(4.847,5.710)%
  --(4.848,5.710)--(4.849,5.710)--(4.850,5.710)--(4.851,5.710)--(4.851,5.711)--(4.852,5.711)%
  --(4.853,5.711)--(4.854,5.711)--(4.855,5.711)--(4.856,5.712)--(4.857,5.712)--(4.858,5.712)%
  --(4.859,5.712)--(4.860,5.712)--(4.860,5.713)--(4.861,5.713)--(4.862,5.713)--(4.863,5.713)%
  --(4.864,5.713)--(4.864,5.714)--(4.865,5.714)--(4.866,5.714)--(4.867,5.714)--(4.868,5.714)%
  --(4.868,5.715)--(4.869,5.715)--(4.870,5.715)--(4.871,5.715)--(4.872,5.715)--(4.873,5.715)%
  --(4.873,5.716)--(4.874,5.716)--(4.875,5.716)--(4.876,5.716)--(4.877,5.716)--(4.877,5.717)%
  --(4.878,5.717)--(4.879,5.717)--(4.880,5.717)--(4.881,5.717)--(4.881,5.718)--(4.882,5.718)%
  --(4.883,5.718)--(4.884,5.718)--(4.885,5.718)--(4.886,5.718)--(4.886,5.719)--(4.887,5.719)%
  --(4.888,5.719)--(4.889,5.719)--(4.890,5.719)--(4.890,5.720)--(4.891,5.720)--(4.892,5.720)%
  --(4.893,5.720)--(4.894,5.720)--(4.895,5.721)--(4.896,5.721)--(4.897,5.721)--(4.898,5.721)%
  --(4.899,5.721)--(4.899,5.722)--(4.900,5.722)--(4.901,5.722)--(4.902,5.722)--(4.903,5.722)%
  --(4.903,5.723)--(4.904,5.723)--(4.905,5.723)--(4.906,5.723)--(4.907,5.723)--(4.908,5.723)%
  --(4.908,5.724)--(4.909,5.724)--(4.910,5.724)--(4.911,5.724)--(4.912,5.724)--(4.912,5.725)%
  --(4.913,5.725)--(4.914,5.725)--(4.915,5.725)--(4.916,5.725)--(4.917,5.725)--(4.917,5.726)%
  --(4.918,5.726)--(4.919,5.726)--(4.920,5.726)--(4.921,5.726)--(4.921,5.727)--(4.922,5.727)%
  --(4.923,5.727)--(4.924,5.727)--(4.925,5.727)--(4.926,5.727)--(4.926,5.728)--(4.927,5.728)%
  --(4.928,5.728)--(4.929,5.728)--(4.930,5.728)--(4.930,5.729)--(4.931,5.729)--(4.932,5.729)%
  --(4.933,5.729)--(4.934,5.729)--(4.935,5.729)--(4.935,5.730)--(4.936,5.730)--(4.937,5.730)%
  --(4.938,5.730)--(4.939,5.730)--(4.939,5.731)--(4.940,5.731)--(4.941,5.731)--(4.942,5.731)%
  --(4.943,5.731)--(4.944,5.731)--(4.944,5.732)--(4.945,5.732)--(4.946,5.732)--(4.947,5.732)%
  --(4.948,5.732)--(4.949,5.732)--(4.949,5.733)--(4.950,5.733)--(4.951,5.733)--(4.952,5.733)%
  --(4.953,5.733)--(4.953,5.734)--(4.954,5.734)--(4.955,5.734)--(4.956,5.734)--(4.957,5.734)%
  --(4.958,5.734)--(4.958,5.735)--(4.959,5.735)--(4.960,5.735)--(4.961,5.735)--(4.962,5.735)%
  --(4.963,5.736)--(4.964,5.736)--(4.965,5.736)--(4.966,5.736)--(4.967,5.736)--(4.967,5.737)%
  --(4.968,5.737)--(4.969,5.737)--(4.970,5.737)--(4.971,5.737)--(4.972,5.737)--(4.972,5.738)%
  --(4.973,5.738)--(4.974,5.738)--(4.975,5.738)--(4.976,5.738)--(4.977,5.738)--(4.977,5.739)%
  --(4.978,5.739)--(4.979,5.739)--(4.980,5.739)--(4.981,5.739)--(4.981,5.740)--(4.982,5.740)%
  --(4.983,5.740)--(4.984,5.740)--(4.985,5.740)--(4.986,5.740)--(4.986,5.741)--(4.987,5.741)%
  --(4.988,5.741)--(4.989,5.741)--(4.990,5.741)--(4.991,5.741)--(4.991,5.742)--(4.992,5.742)%
  --(4.993,5.742)--(4.994,5.742)--(4.995,5.742)--(4.996,5.742)--(4.996,5.743)--(4.997,5.743)%
  --(4.998,5.743)--(4.999,5.743)--(5.000,5.743)--(5.001,5.744)--(5.002,5.744)--(5.003,5.744)%
  --(5.004,5.744)--(5.005,5.744)--(5.005,5.745)--(5.006,5.745)--(5.007,5.745)--(5.008,5.745)%
  --(5.009,5.745)--(5.010,5.745)--(5.010,5.746)--(5.011,5.746)--(5.012,5.746)--(5.013,5.746)%
  --(5.014,5.746)--(5.015,5.746)--(5.015,5.747)--(5.016,5.747)--(5.017,5.747)--(5.018,5.747)%
  --(5.019,5.747)--(5.020,5.747)--(5.020,5.748)--(5.021,5.748)--(5.022,5.748)--(5.023,5.748)%
  --(5.024,5.748)--(5.025,5.748)--(5.025,5.749)--(5.026,5.749)--(5.027,5.749)--(5.028,5.749)%
  --(5.029,5.749)--(5.030,5.749)--(5.030,5.750)--(5.031,5.750)--(5.032,5.750)--(5.033,5.750)%
  --(5.034,5.750)--(5.035,5.750)--(5.035,5.751)--(5.036,5.751)--(5.037,5.751)--(5.038,5.751)%
  --(5.039,5.751)--(5.040,5.751)--(5.040,5.752)--(5.041,5.752)--(5.042,5.752)--(5.043,5.752)%
  --(5.044,5.752)--(5.045,5.752)--(5.045,5.753)--(5.046,5.753)--(5.047,5.753)--(5.048,5.753)%
  --(5.049,5.753)--(5.050,5.753)--(5.050,5.754)--(5.051,5.754)--(5.052,5.754)--(5.053,5.754)%
  --(5.054,5.754)--(5.055,5.754)--(5.055,5.755)--(5.056,5.755)--(5.057,5.755)--(5.058,5.755)%
  --(5.059,5.755)--(5.060,5.755)--(5.060,5.756)--(5.061,5.756)--(5.062,5.756)--(5.063,5.756)%
  --(5.064,5.756)--(5.065,5.756)--(5.065,5.757)--(5.066,5.757)--(5.067,5.757)--(5.068,5.757)%
  --(5.069,5.757)--(5.070,5.757)--(5.070,5.758)--(5.071,5.758)--(5.072,5.758)--(5.073,5.758)%
  --(5.074,5.758)--(5.075,5.758)--(5.075,5.759)--(5.076,5.759)--(5.077,5.759)--(5.078,5.759)%
  --(5.079,5.759)--(5.080,5.759)--(5.081,5.760)--(5.082,5.760)--(5.083,5.760)--(5.084,5.760)%
  --(5.085,5.760)--(5.086,5.760)--(5.086,5.761)--(5.087,5.761)--(5.088,5.761)--(5.089,5.761)%
  --(5.090,5.761)--(5.091,5.761)--(5.091,5.762)--(5.092,5.762)--(5.093,5.762)--(5.094,5.762)%
  --(5.095,5.762)--(5.096,5.762)--(5.096,5.763)--(5.097,5.763)--(5.098,5.763)--(5.099,5.763)%
  --(5.100,5.763)--(5.101,5.763)--(5.101,5.764)--(5.102,5.764)--(5.103,5.764)--(5.104,5.764)%
  --(5.105,5.764)--(5.106,5.764)--(5.107,5.764)--(5.107,5.765)--(5.108,5.765)--(5.109,5.765)%
  --(5.110,5.765)--(5.111,5.765)--(5.112,5.765)--(5.112,5.766)--(5.113,5.766)--(5.114,5.766)%
  --(5.115,5.766)--(5.116,5.766)--(5.117,5.766)--(5.117,5.767)--(5.118,5.767)--(5.119,5.767)%
  --(5.120,5.767)--(5.121,5.767)--(5.122,5.767)--(5.123,5.767)--(5.123,5.768)--(5.124,5.768)%
  --(5.125,5.768)--(5.126,5.768)--(5.127,5.768)--(5.128,5.768)--(5.128,5.769)--(5.129,5.769)%
  --(5.130,5.769)--(5.131,5.769)--(5.132,5.769)--(5.133,5.769)--(5.133,5.770)--(5.134,5.770)%
  --(5.135,5.770)--(5.136,5.770)--(5.137,5.770)--(5.138,5.770)--(5.139,5.770)--(5.139,5.771)%
  --(5.140,5.771)--(5.141,5.771)--(5.142,5.771)--(5.143,5.771)--(5.144,5.771)--(5.144,5.772)%
  --(5.145,5.772)--(5.146,5.772)--(5.147,5.772)--(5.148,5.772)--(5.149,5.772)--(5.150,5.772)%
  --(5.150,5.773)--(5.151,5.773)--(5.152,5.773)--(5.153,5.773)--(5.154,5.773)--(5.155,5.773)%
  --(5.155,5.774)--(5.156,5.774)--(5.157,5.774)--(5.158,5.774)--(5.159,5.774)--(5.160,5.774)%
  --(5.161,5.774)--(5.161,5.775)--(5.162,5.775)--(5.163,5.775)--(5.164,5.775)--(5.165,5.775)%
  --(5.166,5.775)--(5.167,5.776)--(5.168,5.776)--(5.169,5.776)--(5.170,5.776)--(5.171,5.776)%
  --(5.172,5.776)--(5.172,5.777)--(5.173,5.777)--(5.174,5.777)--(5.175,5.777)--(5.176,5.777)%
  --(5.177,5.777)--(5.178,5.777)--(5.178,5.778)--(5.179,5.778)--(5.180,5.778)--(5.181,5.778)%
  --(5.182,5.778)--(5.183,5.778)--(5.183,5.779)--(5.184,5.779)--(5.185,5.779)--(5.186,5.779)%
  --(5.187,5.779)--(5.188,5.779)--(5.189,5.779)--(5.189,5.780)--(5.190,5.780)--(5.191,5.780)%
  --(5.192,5.780)--(5.193,5.780)--(5.194,5.780)--(5.195,5.780)--(5.195,5.781)--(5.196,5.781)%
  --(5.197,5.781)--(5.198,5.781)--(5.199,5.781)--(5.200,5.781)--(5.200,5.782)--(5.201,5.782)%
  --(5.202,5.782)--(5.203,5.782)--(5.204,5.782)--(5.205,5.782)--(5.206,5.782)--(5.206,5.783)%
  --(5.207,5.783)--(5.208,5.783)--(5.209,5.783)--(5.210,5.783)--(5.211,5.783)--(5.212,5.783)%
  --(5.212,5.784)--(5.213,5.784)--(5.214,5.784)--(5.215,5.784)--(5.216,5.784)--(5.217,5.784)%
  --(5.218,5.784)--(5.218,5.785)--(5.219,5.785)--(5.220,5.785)--(5.221,5.785)--(5.222,5.785)%
  --(5.223,5.785)--(5.224,5.785)--(5.224,5.786)--(5.225,5.786)--(5.226,5.786)--(5.227,5.786)%
  --(5.228,5.786)--(5.229,5.786)--(5.229,5.787)--(5.230,5.787)--(5.231,5.787)--(5.232,5.787)%
  --(5.233,5.787)--(5.234,5.787)--(5.235,5.787)--(5.235,5.788)--(5.236,5.788)--(5.237,5.788)%
  --(5.238,5.788)--(5.239,5.788)--(5.240,5.788)--(5.241,5.788)--(5.241,5.789)--(5.242,5.789)%
  --(5.243,5.789)--(5.244,5.789)--(5.245,5.789)--(5.246,5.789)--(5.247,5.789)--(5.247,5.790)%
  --(5.248,5.790)--(5.249,5.790)--(5.250,5.790)--(5.251,5.790)--(5.252,5.790)--(5.253,5.790)%
  --(5.253,5.791)--(5.254,5.791)--(5.255,5.791)--(5.256,5.791)--(5.257,5.791)--(5.258,5.791)%
  --(5.259,5.791)--(5.259,5.792)--(5.260,5.792)--(5.261,5.792)--(5.262,5.792)--(5.263,5.792)%
  --(5.264,5.792)--(5.265,5.792)--(5.265,5.793)--(5.266,5.793)--(5.267,5.793)--(5.268,5.793)%
  --(5.269,5.793)--(5.270,5.793)--(5.271,5.793)--(5.272,5.794)--(5.273,5.794)--(5.274,5.794)%
  --(5.275,5.794)--(5.276,5.794)--(5.277,5.794)--(5.278,5.794)--(5.278,5.795)--(5.279,5.795)%
  --(5.280,5.795)--(5.281,5.795)--(5.282,5.795)--(5.283,5.795)--(5.284,5.795)--(5.284,5.796)%
  --(5.285,5.796)--(5.286,5.796)--(5.287,5.796)--(5.288,5.796)--(5.289,5.796)--(5.290,5.796)%
  --(5.290,5.797)--(5.291,5.797)--(5.292,5.797)--(5.293,5.797)--(5.294,5.797)--(5.295,5.797)%
  --(5.296,5.797)--(5.296,5.798)--(5.297,5.798)--(5.298,5.798)--(5.299,5.798)--(5.300,5.798)%
  --(5.301,5.798)--(5.302,5.798)--(5.302,5.799)--(5.303,5.799)--(5.304,5.799)--(5.305,5.799)%
  --(5.306,5.799)--(5.307,5.799)--(5.308,5.799)--(5.309,5.799)--(5.309,5.800)--(5.310,5.800)%
  --(5.311,5.800)--(5.312,5.800)--(5.313,5.800)--(5.314,5.800)--(5.315,5.800)--(5.315,5.801)%
  --(5.316,5.801)--(5.317,5.801)--(5.318,5.801)--(5.319,5.801)--(5.320,5.801)--(5.321,5.801)%
  --(5.321,5.802)--(5.322,5.802)--(5.323,5.802)--(5.324,5.802)--(5.325,5.802)--(5.326,5.802)%
  --(5.327,5.802)--(5.328,5.802)--(5.328,5.803)--(5.329,5.803)--(5.330,5.803)--(5.331,5.803)%
  --(5.332,5.803)--(5.333,5.803)--(5.334,5.803)--(5.334,5.804)--(5.335,5.804)--(5.336,5.804)%
  --(5.337,5.804)--(5.338,5.804)--(5.339,5.804)--(5.340,5.804)--(5.341,5.804)--(5.341,5.805)%
  --(5.342,5.805)--(5.343,5.805)--(5.344,5.805)--(5.345,5.805)--(5.346,5.805)--(5.347,5.805)%
  --(5.347,5.806)--(5.348,5.806)--(5.349,5.806)--(5.350,5.806)--(5.351,5.806)--(5.352,5.806)%
  --(5.353,5.806)--(5.354,5.806)--(5.354,5.807)--(5.355,5.807)--(5.356,5.807)--(5.357,5.807)%
  --(5.358,5.807)--(5.359,5.807)--(5.360,5.807)--(5.360,5.808)--(5.361,5.808)--(5.362,5.808)%
  --(5.363,5.808)--(5.364,5.808)--(5.365,5.808)--(5.366,5.808)--(5.367,5.808)--(5.367,5.809)%
  --(5.368,5.809)--(5.369,5.809)--(5.370,5.809)--(5.371,5.809)--(5.372,5.809)--(5.373,5.809)%
  --(5.374,5.809)--(5.374,5.810)--(5.375,5.810)--(5.376,5.810)--(5.377,5.810)--(5.378,5.810)%
  --(5.379,5.810)--(5.380,5.810)--(5.380,5.811)--(5.381,5.811)--(5.382,5.811)--(5.383,5.811)%
  --(5.384,5.811)--(5.385,5.811)--(5.386,5.811)--(5.387,5.811)--(5.387,5.812)--(5.388,5.812)%
  --(5.389,5.812)--(5.390,5.812)--(5.391,5.812)--(5.392,5.812)--(5.393,5.812)--(5.394,5.812)%
  --(5.394,5.813)--(5.395,5.813)--(5.396,5.813)--(5.397,5.813)--(5.398,5.813)--(5.399,5.813)%
  --(5.400,5.813)--(5.401,5.813)--(5.401,5.814)--(5.402,5.814)--(5.403,5.814)--(5.404,5.814)%
  --(5.405,5.814)--(5.406,5.814)--(5.407,5.814)--(5.407,5.815)--(5.408,5.815)--(5.409,5.815)%
  --(5.410,5.815)--(5.411,5.815)--(5.412,5.815)--(5.413,5.815)--(5.414,5.815)--(5.414,5.816)%
  --(5.415,5.816)--(5.416,5.816)--(5.417,5.816)--(5.418,5.816)--(5.419,5.816)--(5.420,5.816)%
  --(5.421,5.816)--(5.421,5.817)--(5.422,5.817)--(5.423,5.817)--(5.424,5.817)--(5.425,5.817)%
  --(5.426,5.817)--(5.427,5.817)--(5.428,5.817)--(5.428,5.818)--(5.429,5.818)--(5.430,5.818)%
  --(5.431,5.818)--(5.432,5.818)--(5.433,5.818)--(5.434,5.818)--(5.435,5.818)--(5.435,5.819)%
  --(5.436,5.819)--(5.437,5.819)--(5.438,5.819)--(5.439,5.819)--(5.440,5.819)--(5.441,5.819)%
  --(5.442,5.819)--(5.442,5.820)--(5.443,5.820)--(5.444,5.820)--(5.445,5.820)--(5.446,5.820)%
  --(5.447,5.820)--(5.448,5.820)--(5.449,5.820)--(5.449,5.821)--(5.450,5.821)--(5.451,5.821)%
  --(5.452,5.821)--(5.453,5.821)--(5.454,5.821)--(5.455,5.821)--(5.456,5.821)--(5.457,5.822)%
  --(5.458,5.822)--(5.459,5.822)--(5.460,5.822)--(5.461,5.822)--(5.462,5.822)--(5.463,5.822)%
  --(5.464,5.822)--(5.464,5.823)--(5.465,5.823)--(5.466,5.823)--(5.467,5.823)--(5.468,5.823)%
  --(5.469,5.823)--(5.470,5.823)--(5.471,5.823)--(5.471,5.824)--(5.472,5.824)--(5.473,5.824)%
  --(5.474,5.824)--(5.475,5.824)--(5.476,5.824)--(5.477,5.824)--(5.478,5.824)--(5.478,5.825)%
  --(5.479,5.825)--(5.480,5.825)--(5.481,5.825)--(5.482,5.825)--(5.483,5.825)--(5.484,5.825)%
  --(5.485,5.825)--(5.485,5.826)--(5.486,5.826)--(5.487,5.826)--(5.488,5.826)--(5.489,5.826)%
  --(5.490,5.826)--(5.491,5.826)--(5.492,5.826)--(5.493,5.826)--(5.493,5.827)--(5.494,5.827)%
  --(5.495,5.827)--(5.496,5.827)--(5.497,5.827)--(5.498,5.827)--(5.499,5.827)--(5.500,5.827)%
  --(5.500,5.828)--(5.501,5.828)--(5.502,5.828)--(5.503,5.828)--(5.504,5.828)--(5.505,5.828)%
  --(5.506,5.828)--(5.507,5.828)--(5.508,5.828)--(5.508,5.829)--(5.509,5.829)--(5.510,5.829)%
  --(5.511,5.829)--(5.512,5.829)--(5.513,5.829)--(5.514,5.829)--(5.515,5.829)--(5.515,5.830)%
  --(5.516,5.830)--(5.517,5.830)--(5.518,5.830)--(5.519,5.830)--(5.520,5.830)--(5.521,5.830)%
  --(5.522,5.830)--(5.523,5.830)--(5.523,5.831)--(5.524,5.831)--(5.525,5.831)--(5.526,5.831)%
  --(5.527,5.831)--(5.528,5.831)--(5.529,5.831)--(5.530,5.831)--(5.530,5.832)--(5.531,5.832)%
  --(5.532,5.832)--(5.533,5.832)--(5.534,5.832)--(5.535,5.832)--(5.536,5.832)--(5.537,5.832)%
  --(5.538,5.832)--(5.538,5.833)--(5.539,5.833)--(5.540,5.833)--(5.541,5.833)--(5.542,5.833)%
  --(5.543,5.833)--(5.544,5.833)--(5.545,5.833)--(5.546,5.833)--(5.546,5.834)--(5.547,5.834)%
  --(5.548,5.834)--(5.549,5.834)--(5.550,5.834)--(5.551,5.834)--(5.552,5.834)--(5.553,5.834)%
  --(5.553,5.835)--(5.554,5.835)--(5.555,5.835)--(5.556,5.835)--(5.557,5.835)--(5.558,5.835)%
  --(5.559,5.835)--(5.560,5.835)--(5.561,5.835)--(5.561,5.836)--(5.562,5.836)--(5.563,5.836)%
  --(5.564,5.836)--(5.565,5.836)--(5.566,5.836)--(5.567,5.836)--(5.568,5.836)--(5.569,5.836)%
  --(5.569,5.837)--(5.570,5.837)--(5.571,5.837)--(5.572,5.837)--(5.573,5.837)--(5.574,5.837)%
  --(5.575,5.837)--(5.576,5.837)--(5.577,5.837)--(5.577,5.838)--(5.578,5.838)--(5.579,5.838)%
  --(5.580,5.838)--(5.581,5.838)--(5.582,5.838)--(5.583,5.838)--(5.584,5.838)--(5.585,5.838)%
  --(5.585,5.839)--(5.586,5.839)--(5.587,5.839)--(5.588,5.839)--(5.589,5.839)--(5.590,5.839)%
  --(5.591,5.839)--(5.592,5.839)--(5.593,5.839)--(5.593,5.840)--(5.594,5.840)--(5.595,5.840)%
  --(5.596,5.840)--(5.597,5.840)--(5.598,5.840)--(5.599,5.840)--(5.600,5.840)--(5.601,5.840)%
  --(5.601,5.841)--(5.602,5.841)--(5.603,5.841)--(5.604,5.841)--(5.605,5.841)--(5.606,5.841)%
  --(5.607,5.841)--(5.608,5.841)--(5.609,5.841)--(5.609,5.842)--(5.610,5.842)--(5.611,5.842)%
  --(5.612,5.842)--(5.613,5.842)--(5.614,5.842)--(5.615,5.842)--(5.616,5.842)--(5.617,5.842)%
  --(5.617,5.843)--(5.618,5.843)--(5.619,5.843)--(5.620,5.843)--(5.621,5.843)--(5.622,5.843)%
  --(5.623,5.843)--(5.624,5.843)--(5.625,5.843)--(5.626,5.843)--(5.626,5.844)--(5.627,5.844)%
  --(5.628,5.844)--(5.629,5.844)--(5.630,5.844)--(5.631,5.844)--(5.632,5.844)--(5.633,5.844)%
  --(5.634,5.844)--(5.634,5.845)--(5.635,5.845)--(5.636,5.845)--(5.637,5.845)--(5.638,5.845)%
  --(5.639,5.845)--(5.640,5.845)--(5.641,5.845)--(5.642,5.845)--(5.642,5.846)--(5.643,5.846)%
  --(5.644,5.846)--(5.645,5.846)--(5.646,5.846)--(5.647,5.846)--(5.648,5.846)--(5.649,5.846)%
  --(5.650,5.846)--(5.651,5.846)--(5.651,5.847)--(5.652,5.847)--(5.653,5.847)--(5.654,5.847)%
  --(5.655,5.847)--(5.656,5.847)--(5.657,5.847)--(5.658,5.847)--(5.659,5.847)--(5.659,5.848)%
  --(5.660,5.848)--(5.661,5.848)--(5.662,5.848)--(5.663,5.848)--(5.664,5.848)--(5.665,5.848)%
  --(5.666,5.848)--(5.667,5.848)--(5.668,5.848)--(5.668,5.849)--(5.669,5.849)--(5.670,5.849)%
  --(5.671,5.849)--(5.672,5.849)--(5.673,5.849)--(5.674,5.849)--(5.675,5.849)--(5.676,5.849)%
  --(5.676,5.850)--(5.677,5.850)--(5.678,5.850)--(5.679,5.850)--(5.680,5.850)--(5.681,5.850)%
  --(5.682,5.850)--(5.683,5.850)--(5.684,5.850)--(5.685,5.850)--(5.685,5.851)--(5.686,5.851)%
  --(5.687,5.851)--(5.688,5.851)--(5.689,5.851)--(5.690,5.851)--(5.691,5.851)--(5.692,5.851)%
  --(5.693,5.851)--(5.694,5.851)--(5.694,5.852)--(5.695,5.852)--(5.696,5.852)--(5.697,5.852)%
  --(5.698,5.852)--(5.699,5.852)--(5.700,5.852)--(5.701,5.852)--(5.702,5.852)--(5.703,5.852)%
  --(5.703,5.853)--(5.704,5.853)--(5.705,5.853)--(5.706,5.853)--(5.707,5.853)--(5.708,5.853)%
  --(5.709,5.853)--(5.710,5.853)--(5.711,5.853)--(5.712,5.853)--(5.712,5.854)--(5.713,5.854)%
  --(5.714,5.854)--(5.715,5.854)--(5.716,5.854)--(5.717,5.854)--(5.718,5.854)--(5.719,5.854)%
  --(5.720,5.854)--(5.721,5.855)--(5.722,5.855)--(5.723,5.855)--(5.724,5.855)--(5.725,5.855)%
  --(5.726,5.855)--(5.727,5.855)--(5.728,5.855)--(5.729,5.855)--(5.730,5.856)--(5.731,5.856)%
  --(5.732,5.856)--(5.733,5.856)--(5.734,5.856)--(5.735,5.856)--(5.736,5.856)--(5.737,5.856)%
  --(5.738,5.856)--(5.739,5.856)--(5.739,5.857)--(5.740,5.857)--(5.741,5.857)--(5.742,5.857)%
  --(5.743,5.857)--(5.744,5.857)--(5.745,5.857)--(5.746,5.857)--(5.747,5.857)--(5.748,5.857)%
  --(5.748,5.858)--(5.749,5.858)--(5.750,5.858)--(5.751,5.858)--(5.752,5.858)--(5.753,5.858)%
  --(5.754,5.858)--(5.755,5.858)--(5.756,5.858)--(5.757,5.858)--(5.757,5.859)--(5.758,5.859)%
  --(5.759,5.859)--(5.760,5.859)--(5.761,5.859)--(5.762,5.859)--(5.763,5.859)--(5.764,5.859)%
  --(5.765,5.859)--(5.766,5.859)--(5.766,5.860)--(5.767,5.860)--(5.768,5.860)--(5.769,5.860)%
  --(5.770,5.860)--(5.771,5.860)--(5.772,5.860)--(5.773,5.860)--(5.774,5.860)--(5.775,5.860)%
  --(5.776,5.860)--(5.776,5.861)--(5.777,5.861)--(5.778,5.861)--(5.779,5.861)--(5.780,5.861)%
  --(5.781,5.861)--(5.782,5.861)--(5.783,5.861)--(5.784,5.861)--(5.785,5.861)--(5.785,5.862)%
  --(5.786,5.862)--(5.787,5.862)--(5.788,5.862)--(5.789,5.862)--(5.790,5.862)--(5.791,5.862)%
  --(5.792,5.862)--(5.793,5.862)--(5.794,5.862)--(5.795,5.862)--(5.795,5.863)--(5.796,5.863)%
  --(5.797,5.863)--(5.798,5.863)--(5.799,5.863)--(5.800,5.863)--(5.801,5.863)--(5.802,5.863)%
  --(5.803,5.863)--(5.804,5.863)--(5.804,5.864)--(5.805,5.864)--(5.806,5.864)--(5.807,5.864)%
  --(5.808,5.864)--(5.809,5.864)--(5.810,5.864)--(5.811,5.864)--(5.812,5.864)--(5.813,5.864)%
  --(5.814,5.864)--(5.814,5.865)--(5.815,5.865)--(5.816,5.865)--(5.817,5.865)--(5.818,5.865)%
  --(5.819,5.865)--(5.820,5.865)--(5.821,5.865)--(5.822,5.865)--(5.823,5.865)--(5.824,5.865)%
  --(5.824,5.866)--(5.825,5.866)--(5.826,5.866)--(5.827,5.866)--(5.828,5.866)--(5.829,5.866)%
  --(5.830,5.866)--(5.831,5.866)--(5.832,5.866)--(5.833,5.866)--(5.833,5.867)--(5.834,5.867)%
  --(5.835,5.867)--(5.836,5.867)--(5.837,5.867)--(5.838,5.867)--(5.839,5.867)--(5.840,5.867)%
  --(5.841,5.867)--(5.842,5.867)--(5.843,5.867)--(5.843,5.868)--(5.844,5.868)--(5.845,5.868)%
  --(5.846,5.868)--(5.847,5.868)--(5.848,5.868)--(5.849,5.868)--(5.850,5.868)--(5.851,5.868)%
  --(5.852,5.868)--(5.853,5.868)--(5.853,5.869)--(5.854,5.869)--(5.855,5.869)--(5.856,5.869)%
  --(5.857,5.869)--(5.858,5.869)--(5.859,5.869)--(5.860,5.869)--(5.861,5.869)--(5.862,5.869)%
  --(5.863,5.869)--(5.863,5.870)--(5.864,5.870)--(5.865,5.870)--(5.866,5.870)--(5.867,5.870)%
  --(5.868,5.870)--(5.869,5.870)--(5.870,5.870)--(5.871,5.870)--(5.872,5.870)--(5.873,5.870)%
  --(5.874,5.870)--(5.874,5.871)--(5.875,5.871)--(5.876,5.871)--(5.877,5.871)--(5.878,5.871)%
  --(5.879,5.871)--(5.880,5.871)--(5.881,5.871)--(5.882,5.871)--(5.883,5.871)--(5.884,5.871)%
  --(5.884,5.872)--(5.885,5.872)--(5.886,5.872)--(5.887,5.872)--(5.888,5.872)--(5.889,5.872)%
  --(5.890,5.872)--(5.891,5.872)--(5.892,5.872)--(5.893,5.872)--(5.894,5.872)--(5.894,5.873)%
  --(5.895,5.873)--(5.896,5.873)--(5.897,5.873)--(5.898,5.873)--(5.899,5.873)--(5.900,5.873)%
  --(5.901,5.873)--(5.902,5.873)--(5.903,5.873)--(5.904,5.873)--(5.905,5.873)--(5.905,5.874)%
  --(5.906,5.874)--(5.907,5.874)--(5.908,5.874)--(5.909,5.874)--(5.910,5.874)--(5.911,5.874)%
  --(5.912,5.874)--(5.913,5.874)--(5.914,5.874)--(5.915,5.874)--(5.915,5.875)--(5.916,5.875)%
  --(5.917,5.875)--(5.918,5.875)--(5.919,5.875)--(5.920,5.875)--(5.921,5.875)--(5.922,5.875)%
  --(5.923,5.875)--(5.924,5.875)--(5.925,5.875)--(5.926,5.875)--(5.926,5.876)--(5.927,5.876)%
  --(5.928,5.876)--(5.929,5.876)--(5.930,5.876)--(5.931,5.876)--(5.932,5.876)--(5.933,5.876)%
  --(5.934,5.876)--(5.935,5.876)--(5.936,5.876)--(5.937,5.876)--(5.937,5.877)--(5.938,5.877)%
  --(5.939,5.877)--(5.940,5.877)--(5.941,5.877)--(5.942,5.877)--(5.943,5.877)--(5.944,5.877)%
  --(5.945,5.877)--(5.946,5.877)--(5.947,5.877)--(5.947,5.878)--(5.948,5.878)--(5.949,5.878)%
  --(5.950,5.878)--(5.951,5.878)--(5.952,5.878)--(5.953,5.878)--(5.954,5.878)--(5.955,5.878)%
  --(5.956,5.878)--(5.957,5.878)--(5.958,5.878)--(5.958,5.879)--(5.959,5.879)--(5.960,5.879)%
  --(5.961,5.879)--(5.962,5.879)--(5.963,5.879)--(5.964,5.879)--(5.965,5.879)--(5.966,5.879)%
  --(5.967,5.879)--(5.968,5.879)--(5.969,5.879)--(5.969,5.880)--(5.970,5.880)--(5.971,5.880)%
  --(5.972,5.880)--(5.973,5.880)--(5.974,5.880)--(5.975,5.880)--(5.976,5.880)--(5.977,5.880)%
  --(5.978,5.880)--(5.979,5.880)--(5.980,5.880)--(5.981,5.880)--(5.981,5.881)--(5.982,5.881)%
  --(5.983,5.881)--(5.984,5.881)--(5.985,5.881)--(5.986,5.881)--(5.987,5.881)--(5.988,5.881)%
  --(5.989,5.881)--(5.990,5.881)--(5.991,5.881)--(5.992,5.881)--(5.992,5.882)--(5.993,5.882)%
  --(5.994,5.882)--(5.995,5.882)--(5.996,5.882)--(5.997,5.882)--(5.998,5.882)--(5.999,5.882)%
  --(6.000,5.882)--(6.001,5.882)--(6.002,5.882)--(6.003,5.882)--(6.003,5.883)--(6.004,5.883)%
  --(6.005,5.883)--(6.006,5.883)--(6.007,5.883)--(6.008,5.883)--(6.009,5.883)--(6.010,5.883)%
  --(6.011,5.883)--(6.012,5.883)--(6.013,5.883)--(6.014,5.883)--(6.015,5.883)--(6.015,5.884)%
  --(6.016,5.884)--(6.017,5.884)--(6.018,5.884)--(6.019,5.884)--(6.020,5.884)--(6.021,5.884)%
  --(6.022,5.884)--(6.023,5.884)--(6.024,5.884)--(6.025,5.884)--(6.026,5.884)--(6.026,5.885)%
  --(6.027,5.885)--(6.028,5.885)--(6.029,5.885)--(6.030,5.885)--(6.031,5.885)--(6.032,5.885)%
  --(6.033,5.885)--(6.034,5.885)--(6.035,5.885)--(6.036,5.885)--(6.037,5.885)--(6.038,5.885)%
  --(6.038,5.886)--(6.039,5.886)--(6.040,5.886)--(6.041,5.886)--(6.042,5.886)--(6.043,5.886)%
  --(6.044,5.886)--(6.045,5.886)--(6.046,5.886)--(6.047,5.886)--(6.048,5.886)--(6.049,5.886)%
  --(6.050,5.886)--(6.050,5.887)--(6.051,5.887)--(6.052,5.887)--(6.053,5.887)--(6.054,5.887)%
  --(6.055,5.887)--(6.056,5.887)--(6.057,5.887)--(6.058,5.887)--(6.059,5.887)--(6.060,5.887)%
  --(6.061,5.887)--(6.062,5.887)--(6.062,5.888)--(6.063,5.888)--(6.064,5.888)--(6.065,5.888)%
  --(6.066,5.888)--(6.067,5.888)--(6.068,5.888)--(6.069,5.888)--(6.070,5.888)--(6.071,5.888)%
  --(6.072,5.888)--(6.073,5.888)--(6.074,5.888)--(6.074,5.889)--(6.075,5.889)--(6.076,5.889)%
  --(6.077,5.889)--(6.078,5.889)--(6.079,5.889)--(6.080,5.889)--(6.081,5.889)--(6.082,5.889)%
  --(6.083,5.889)--(6.084,5.889)--(6.085,5.889)--(6.086,5.889)--(6.086,5.890)--(6.087,5.890)%
  --(6.088,5.890)--(6.089,5.890)--(6.090,5.890)--(6.091,5.890)--(6.092,5.890)--(6.093,5.890)%
  --(6.094,5.890)--(6.095,5.890)--(6.096,5.890)--(6.097,5.890)--(6.098,5.890)--(6.098,5.891)%
  --(6.099,5.891)--(6.100,5.891)--(6.101,5.891)--(6.102,5.891)--(6.103,5.891)--(6.104,5.891)%
  --(6.105,5.891)--(6.106,5.891)--(6.107,5.891)--(6.108,5.891)--(6.109,5.891)--(6.110,5.891)%
  --(6.111,5.891)--(6.111,5.892)--(6.112,5.892)--(6.113,5.892)--(6.114,5.892)--(6.115,5.892)%
  --(6.116,5.892)--(6.117,5.892)--(6.118,5.892)--(6.119,5.892)--(6.120,5.892)--(6.121,5.892)%
  --(6.122,5.892)--(6.123,5.892)--(6.123,5.893)--(6.124,5.893)--(6.125,5.893)--(6.126,5.893)%
  --(6.127,5.893)--(6.128,5.893)--(6.129,5.893)--(6.130,5.893)--(6.131,5.893)--(6.132,5.893)%
  --(6.133,5.893)--(6.134,5.893)--(6.135,5.893)--(6.136,5.893)--(6.136,5.894)--(6.137,5.894)%
  --(6.138,5.894)--(6.139,5.894)--(6.140,5.894)--(6.141,5.894)--(6.142,5.894)--(6.143,5.894)%
  --(6.144,5.894)--(6.145,5.894)--(6.146,5.894)--(6.147,5.894)--(6.148,5.894)--(6.149,5.894)%
  --(6.149,5.895)--(6.150,5.895)--(6.151,5.895)--(6.152,5.895)--(6.153,5.895)--(6.154,5.895)%
  --(6.155,5.895)--(6.156,5.895)--(6.157,5.895)--(6.158,5.895)--(6.159,5.895)--(6.160,5.895)%
  --(6.161,5.895)--(6.162,5.895)--(6.162,5.896)--(6.163,5.896)--(6.164,5.896)--(6.165,5.896)%
  --(6.166,5.896)--(6.167,5.896)--(6.168,5.896)--(6.169,5.896)--(6.170,5.896)--(6.171,5.896)%
  --(6.172,5.896)--(6.173,5.896)--(6.174,5.896)--(6.175,5.896)--(6.175,5.897)--(6.176,5.897)%
  --(6.177,5.897)--(6.178,5.897)--(6.179,5.897)--(6.180,5.897)--(6.181,5.897)--(6.182,5.897)%
  --(6.183,5.897)--(6.184,5.897)--(6.185,5.897)--(6.186,5.897)--(6.187,5.897)--(6.188,5.897)%
  --(6.188,5.898)--(6.189,5.898)--(6.190,5.898)--(6.191,5.898)--(6.192,5.898)--(6.193,5.898)%
  --(6.194,5.898)--(6.195,5.898)--(6.196,5.898)--(6.197,5.898)--(6.198,5.898)--(6.199,5.898)%
  --(6.200,5.898)--(6.201,5.898)--(6.201,5.899)--(6.202,5.899)--(6.203,5.899)--(6.204,5.899)%
  --(6.205,5.899)--(6.206,5.899)--(6.207,5.899)--(6.208,5.899)--(6.209,5.899)--(6.210,5.899)%
  --(6.211,5.899)--(6.212,5.899)--(6.213,5.899)--(6.214,5.899)--(6.215,5.899)--(6.215,5.900)%
  --(6.216,5.900)--(6.217,5.900)--(6.218,5.900)--(6.219,5.900)--(6.220,5.900)--(6.221,5.900)%
  --(6.222,5.900)--(6.223,5.900)--(6.224,5.900)--(6.225,5.900)--(6.226,5.900)--(6.227,5.900)%
  --(6.228,5.900)--(6.229,5.900)--(6.229,5.901)--(6.230,5.901)--(6.231,5.901)--(6.232,5.901)%
  --(6.233,5.901)--(6.234,5.901)--(6.235,5.901)--(6.236,5.901)--(6.237,5.901)--(6.238,5.901)%
  --(6.239,5.901)--(6.240,5.901)--(6.241,5.901)--(6.242,5.901)--(6.243,5.901)--(6.243,5.902)%
  --(6.244,5.902)--(6.245,5.902)--(6.246,5.902)--(6.247,5.902)--(6.248,5.902)--(6.249,5.902)%
  --(6.250,5.902)--(6.251,5.902)--(6.252,5.902)--(6.253,5.902)--(6.254,5.902)--(6.255,5.902)%
  --(6.256,5.902)--(6.257,5.902)--(6.257,5.903)--(6.258,5.903)--(6.259,5.903)--(6.260,5.903)%
  --(6.261,5.903)--(6.262,5.903)--(6.263,5.903)--(6.264,5.903)--(6.265,5.903)--(6.266,5.903)%
  --(6.267,5.903)--(6.268,5.903)--(6.269,5.903)--(6.270,5.903)--(6.271,5.903)--(6.271,5.904)%
  --(6.272,5.904)--(6.273,5.904)--(6.274,5.904)--(6.275,5.904)--(6.276,5.904)--(6.277,5.904)%
  --(6.278,5.904)--(6.279,5.904)--(6.280,5.904)--(6.281,5.904)--(6.282,5.904)--(6.283,5.904)%
  --(6.284,5.904)--(6.285,5.904)--(6.285,5.905)--(6.286,5.905)--(6.287,5.905)--(6.288,5.905)%
  --(6.289,5.905)--(6.290,5.905)--(6.291,5.905)--(6.292,5.905)--(6.293,5.905)--(6.294,5.905)%
  --(6.295,5.905)--(6.296,5.905)--(6.297,5.905)--(6.298,5.905)--(6.299,5.905)--(6.300,5.905)%
  --(6.300,5.906)--(6.301,5.906)--(6.302,5.906)--(6.303,5.906)--(6.304,5.906)--(6.305,5.906)%
  --(6.306,5.906)--(6.307,5.906)--(6.308,5.906)--(6.309,5.906)--(6.310,5.906)--(6.311,5.906)%
  --(6.312,5.906)--(6.313,5.906)--(6.314,5.906)--(6.315,5.907)--(6.316,5.907)--(6.317,5.907)%
  --(6.318,5.907)--(6.319,5.907)--(6.320,5.907)--(6.321,5.907)--(6.322,5.907)--(6.323,5.907)%
  --(6.324,5.907)--(6.325,5.907)--(6.326,5.907)--(6.327,5.907)--(6.328,5.907)--(6.329,5.907)%
  --(6.329,5.908)--(6.330,5.908)--(6.331,5.908)--(6.332,5.908)--(6.333,5.908)--(6.334,5.908)%
  --(6.335,5.908)--(6.336,5.908)--(6.337,5.908)--(6.338,5.908)--(6.339,5.908)--(6.340,5.908)%
  --(6.341,5.908)--(6.342,5.908)--(6.343,5.908)--(6.344,5.908)--(6.345,5.909)--(6.346,5.909)%
  --(6.347,5.909)--(6.348,5.909)--(6.349,5.909)--(6.350,5.909)--(6.351,5.909)--(6.352,5.909)%
  --(6.353,5.909)--(6.354,5.909)--(6.355,5.909)--(6.356,5.909)--(6.357,5.909)--(6.358,5.909)%
  --(6.359,5.909)--(6.360,5.909)--(6.360,5.910)--(6.361,5.910)--(6.362,5.910)--(6.363,5.910)%
  --(6.364,5.910)--(6.365,5.910)--(6.366,5.910)--(6.367,5.910)--(6.368,5.910)--(6.369,5.910)%
  --(6.370,5.910)--(6.371,5.910)--(6.372,5.910)--(6.373,5.910)--(6.374,5.910)--(6.375,5.910)%
  --(6.375,5.911)--(6.376,5.911)--(6.377,5.911)--(6.378,5.911)--(6.379,5.911)--(6.380,5.911)%
  --(6.381,5.911)--(6.382,5.911)--(6.383,5.911)--(6.384,5.911)--(6.385,5.911)--(6.386,5.911)%
  --(6.387,5.911)--(6.388,5.911)--(6.389,5.911)--(6.390,5.911)--(6.391,5.911)--(6.391,5.912)%
  --(6.392,5.912)--(6.393,5.912)--(6.394,5.912)--(6.395,5.912)--(6.396,5.912)--(6.397,5.912)%
  --(6.398,5.912)--(6.399,5.912)--(6.400,5.912)--(6.401,5.912)--(6.402,5.912)--(6.403,5.912)%
  --(6.404,5.912)--(6.405,5.912)--(6.406,5.912)--(6.407,5.912)--(6.407,5.913)--(6.408,5.913)%
  --(6.409,5.913)--(6.410,5.913)--(6.411,5.913)--(6.412,5.913)--(6.413,5.913)--(6.414,5.913)%
  --(6.415,5.913)--(6.416,5.913)--(6.417,5.913)--(6.418,5.913)--(6.419,5.913)--(6.420,5.913)%
  --(6.421,5.913)--(6.422,5.913)--(6.423,5.913)--(6.423,5.914)--(6.424,5.914)--(6.425,5.914)%
  --(6.426,5.914)--(6.427,5.914)--(6.428,5.914)--(6.429,5.914)--(6.430,5.914)--(6.431,5.914)%
  --(6.432,5.914)--(6.433,5.914)--(6.434,5.914)--(6.435,5.914)--(6.436,5.914)--(6.437,5.914)%
  --(6.438,5.914)--(6.439,5.914)--(6.440,5.915)--(6.441,5.915)--(6.442,5.915)--(6.443,5.915)%
  --(6.444,5.915)--(6.445,5.915)--(6.446,5.915)--(6.447,5.915)--(6.448,5.915)--(6.449,5.915)%
  --(6.450,5.915)--(6.451,5.915)--(6.452,5.915)--(6.453,5.915)--(6.454,5.915)--(6.455,5.915)%
  --(6.456,5.915)--(6.456,5.916)--(6.457,5.916)--(6.458,5.916)--(6.459,5.916)--(6.460,5.916)%
  --(6.461,5.916)--(6.462,5.916)--(6.463,5.916)--(6.464,5.916)--(6.465,5.916)--(6.466,5.916)%
  --(6.467,5.916)--(6.468,5.916)--(6.469,5.916)--(6.470,5.916)--(6.471,5.916)--(6.472,5.916)%
  --(6.473,5.916)--(6.473,5.917)--(6.474,5.917)--(6.475,5.917)--(6.476,5.917)--(6.477,5.917)%
  --(6.478,5.917)--(6.479,5.917)--(6.480,5.917)--(6.481,5.917)--(6.482,5.917)--(6.483,5.917)%
  --(6.484,5.917)--(6.485,5.917)--(6.486,5.917)--(6.487,5.917)--(6.488,5.917)--(6.489,5.917)%
  --(6.490,5.917)--(6.490,5.918)--(6.491,5.918)--(6.492,5.918)--(6.493,5.918)--(6.494,5.918)%
  --(6.495,5.918)--(6.496,5.918)--(6.497,5.918)--(6.498,5.918)--(6.499,5.918)--(6.500,5.918)%
  --(6.501,5.918)--(6.502,5.918)--(6.503,5.918)--(6.504,5.918)--(6.505,5.918)--(6.506,5.918)%
  --(6.507,5.918)--(6.507,5.919)--(6.508,5.919)--(6.509,5.919)--(6.510,5.919)--(6.511,5.919)%
  --(6.512,5.919)--(6.513,5.919)--(6.514,5.919)--(6.515,5.919)--(6.516,5.919)--(6.517,5.919)%
  --(6.518,5.919)--(6.519,5.919)--(6.520,5.919)--(6.521,5.919)--(6.522,5.919)--(6.523,5.919)%
  --(6.524,5.919)--(6.525,5.919)--(6.525,5.920)--(6.526,5.920)--(6.527,5.920)--(6.528,5.920)%
  --(6.529,5.920)--(6.530,5.920)--(6.531,5.920)--(6.532,5.920)--(6.533,5.920)--(6.534,5.920)%
  --(6.535,5.920)--(6.536,5.920)--(6.537,5.920)--(6.538,5.920)--(6.539,5.920)--(6.540,5.920)%
  --(6.541,5.920)--(6.542,5.920)--(6.543,5.920)--(6.543,5.921)--(6.544,5.921)--(6.545,5.921)%
  --(6.546,5.921)--(6.547,5.921)--(6.548,5.921)--(6.549,5.921)--(6.550,5.921)--(6.551,5.921)%
  --(6.552,5.921)--(6.553,5.921)--(6.554,5.921)--(6.555,5.921)--(6.556,5.921)--(6.557,5.921)%
  --(6.558,5.921)--(6.559,5.921)--(6.560,5.921)--(6.561,5.921)--(6.561,5.922)--(6.562,5.922)%
  --(6.563,5.922)--(6.564,5.922)--(6.565,5.922)--(6.566,5.922)--(6.567,5.922)--(6.568,5.922)%
  --(6.569,5.922)--(6.570,5.922)--(6.571,5.922)--(6.572,5.922)--(6.573,5.922)--(6.574,5.922)%
  --(6.575,5.922)--(6.576,5.922)--(6.577,5.922)--(6.578,5.922)--(6.579,5.922)--(6.579,5.923)%
  --(6.580,5.923)--(6.581,5.923)--(6.582,5.923)--(6.583,5.923)--(6.584,5.923)--(6.585,5.923)%
  --(6.586,5.923)--(6.587,5.923)--(6.588,5.923)--(6.589,5.923)--(6.590,5.923)--(6.591,5.923)%
  --(6.592,5.923)--(6.593,5.923)--(6.594,5.923)--(6.595,5.923)--(6.596,5.923)--(6.597,5.923)%
  --(6.598,5.923)--(6.598,5.924)--(6.599,5.924)--(6.600,5.924)--(6.601,5.924)--(6.602,5.924)%
  --(6.603,5.924)--(6.604,5.924)--(6.605,5.924)--(6.606,5.924)--(6.607,5.924)--(6.608,5.924)%
  --(6.609,5.924)--(6.610,5.924)--(6.611,5.924)--(6.612,5.924)--(6.613,5.924)--(6.614,5.924)%
  --(6.615,5.924)--(6.616,5.924)--(6.617,5.924)--(6.617,5.925)--(6.618,5.925)--(6.619,5.925)%
  --(6.620,5.925)--(6.621,5.925)--(6.622,5.925)--(6.623,5.925)--(6.624,5.925)--(6.625,5.925)%
  --(6.626,5.925)--(6.627,5.925)--(6.628,5.925)--(6.629,5.925)--(6.630,5.925)--(6.631,5.925)%
  --(6.632,5.925)--(6.633,5.925)--(6.634,5.925)--(6.635,5.925)--(6.636,5.925)--(6.636,5.926)%
  --(6.637,5.926)--(6.638,5.926)--(6.639,5.926)--(6.640,5.926)--(6.641,5.926)--(6.642,5.926)%
  --(6.643,5.926)--(6.644,5.926)--(6.645,5.926)--(6.646,5.926)--(6.647,5.926)--(6.648,5.926)%
  --(6.649,5.926)--(6.650,5.926)--(6.651,5.926)--(6.652,5.926)--(6.653,5.926)--(6.654,5.926)%
  --(6.655,5.926)--(6.656,5.926)--(6.656,5.927)--(6.657,5.927)--(6.658,5.927)--(6.659,5.927)%
  --(6.660,5.927)--(6.661,5.927)--(6.662,5.927)--(6.663,5.927)--(6.664,5.927)--(6.665,5.927)%
  --(6.666,5.927)--(6.667,5.927)--(6.668,5.927)--(6.669,5.927)--(6.670,5.927)--(6.671,5.927)%
  --(6.672,5.927)--(6.673,5.927)--(6.674,5.927)--(6.675,5.927)--(6.676,5.928)--(6.677,5.928)%
  --(6.678,5.928)--(6.679,5.928)--(6.680,5.928)--(6.681,5.928)--(6.682,5.928)--(6.683,5.928)%
  --(6.684,5.928)--(6.685,5.928)--(6.686,5.928)--(6.687,5.928)--(6.688,5.928)--(6.689,5.928)%
  --(6.690,5.928)--(6.691,5.928)--(6.692,5.928)--(6.693,5.928)--(6.694,5.928)--(6.695,5.928)%
  --(6.696,5.928)--(6.696,5.929)--(6.697,5.929)--(6.698,5.929)--(6.699,5.929)--(6.700,5.929)%
  --(6.701,5.929)--(6.702,5.929)--(6.703,5.929)--(6.704,5.929)--(6.705,5.929)--(6.706,5.929)%
  --(6.707,5.929)--(6.708,5.929)--(6.709,5.929)--(6.710,5.929)--(6.711,5.929)--(6.712,5.929)%
  --(6.713,5.929)--(6.714,5.929)--(6.715,5.929)--(6.716,5.929)--(6.716,5.930)--(6.717,5.930)%
  --(6.718,5.930)--(6.719,5.930)--(6.720,5.930)--(6.721,5.930)--(6.722,5.930)--(6.723,5.930)%
  --(6.724,5.930)--(6.725,5.930)--(6.726,5.930)--(6.727,5.930)--(6.728,5.930)--(6.729,5.930)%
  --(6.730,5.930)--(6.731,5.930)--(6.732,5.930)--(6.733,5.930)--(6.734,5.930)--(6.735,5.930)%
  --(6.736,5.930)--(6.737,5.930)--(6.737,5.931)--(6.738,5.931)--(6.739,5.931)--(6.740,5.931)%
  --(6.741,5.931)--(6.742,5.931)--(6.743,5.931)--(6.744,5.931)--(6.745,5.931)--(6.746,5.931)%
  --(6.747,5.931)--(6.748,5.931)--(6.749,5.931)--(6.750,5.931)--(6.751,5.931)--(6.752,5.931)%
  --(6.753,5.931)--(6.754,5.931)--(6.755,5.931)--(6.756,5.931)--(6.757,5.931)--(6.758,5.931)%
  --(6.759,5.931)--(6.759,5.932)--(6.760,5.932)--(6.761,5.932)--(6.762,5.932)--(6.763,5.932)%
  --(6.764,5.932)--(6.765,5.932)--(6.766,5.932)--(6.767,5.932)--(6.768,5.932)--(6.769,5.932)%
  --(6.770,5.932)--(6.771,5.932)--(6.772,5.932)--(6.773,5.932)--(6.774,5.932)--(6.775,5.932)%
  --(6.776,5.932)--(6.777,5.932)--(6.778,5.932)--(6.779,5.932)--(6.780,5.932)--(6.780,5.933)%
  --(6.781,5.933)--(6.782,5.933)--(6.783,5.933)--(6.784,5.933)--(6.785,5.933)--(6.786,5.933)%
  --(6.787,5.933)--(6.788,5.933)--(6.789,5.933)--(6.790,5.933)--(6.791,5.933)--(6.792,5.933)%
  --(6.793,5.933)--(6.794,5.933)--(6.795,5.933)--(6.796,5.933)--(6.797,5.933)--(6.798,5.933)%
  --(6.799,5.933)--(6.800,5.933)--(6.801,5.933)--(6.802,5.933)--(6.802,5.934)--(6.803,5.934)%
  --(6.804,5.934)--(6.805,5.934)--(6.806,5.934)--(6.807,5.934)--(6.808,5.934)--(6.809,5.934)%
  --(6.810,5.934)--(6.811,5.934)--(6.812,5.934)--(6.813,5.934)--(6.814,5.934)--(6.815,5.934)%
  --(6.816,5.934)--(6.817,5.934)--(6.818,5.934)--(6.819,5.934)--(6.820,5.934)--(6.821,5.934)%
  --(6.822,5.934)--(6.823,5.934)--(6.824,5.934)--(6.825,5.934)--(6.825,5.935)--(6.826,5.935)%
  --(6.827,5.935)--(6.828,5.935)--(6.829,5.935)--(6.830,5.935)--(6.831,5.935)--(6.832,5.935)%
  --(6.833,5.935)--(6.834,5.935)--(6.835,5.935)--(6.836,5.935)--(6.837,5.935)--(6.838,5.935)%
  --(6.839,5.935)--(6.840,5.935)--(6.841,5.935)--(6.842,5.935)--(6.843,5.935)--(6.844,5.935)%
  --(6.845,5.935)--(6.846,5.935)--(6.847,5.935)--(6.848,5.935)--(6.848,5.936)--(6.849,5.936)%
  --(6.850,5.936)--(6.851,5.936)--(6.852,5.936)--(6.853,5.936)--(6.854,5.936)--(6.855,5.936)%
  --(6.856,5.936)--(6.857,5.936)--(6.858,5.936)--(6.859,5.936)--(6.860,5.936)--(6.861,5.936)%
  --(6.862,5.936)--(6.863,5.936)--(6.864,5.936)--(6.865,5.936)--(6.866,5.936)--(6.867,5.936)%
  --(6.868,5.936)--(6.869,5.936)--(6.870,5.936)--(6.871,5.936)--(6.871,5.937)--(6.872,5.937)%
  --(6.873,5.937)--(6.874,5.937)--(6.875,5.937)--(6.876,5.937)--(6.877,5.937)--(6.878,5.937)%
  --(6.879,5.937)--(6.880,5.937)--(6.881,5.937)--(6.882,5.937)--(6.883,5.937)--(6.884,5.937)%
  --(6.885,5.937)--(6.886,5.937)--(6.887,5.937)--(6.888,5.937)--(6.889,5.937)--(6.890,5.937)%
  --(6.891,5.937)--(6.892,5.937)--(6.893,5.937)--(6.894,5.937)--(6.895,5.937)--(6.895,5.938)%
  --(6.896,5.938)--(6.897,5.938)--(6.898,5.938)--(6.899,5.938)--(6.900,5.938)--(6.901,5.938)%
  --(6.902,5.938)--(6.903,5.938)--(6.904,5.938)--(6.905,5.938)--(6.906,5.938)--(6.907,5.938)%
  --(6.908,5.938)--(6.909,5.938)--(6.910,5.938)--(6.911,5.938)--(6.912,5.938)--(6.913,5.938)%
  --(6.914,5.938)--(6.915,5.938)--(6.916,5.938)--(6.917,5.938)--(6.918,5.938)--(6.919,5.938)%
  --(6.920,5.939)--(6.921,5.939)--(6.922,5.939)--(6.923,5.939)--(6.924,5.939)--(6.925,5.939)%
  --(6.926,5.939)--(6.927,5.939)--(6.928,5.939)--(6.929,5.939)--(6.930,5.939)--(6.931,5.939)%
  --(6.932,5.939)--(6.933,5.939)--(6.934,5.939)--(6.935,5.939)--(6.936,5.939)--(6.937,5.939)%
  --(6.938,5.939)--(6.939,5.939)--(6.940,5.939)--(6.941,5.939)--(6.942,5.939)--(6.943,5.939)%
  --(6.944,5.939)--(6.944,5.940)--(6.945,5.940)--(6.946,5.940)--(6.947,5.940)--(6.948,5.940)%
  --(6.949,5.940)--(6.950,5.940)--(6.951,5.940)--(6.952,5.940)--(6.953,5.940)--(6.954,5.940)%
  --(6.955,5.940)--(6.956,5.940)--(6.957,5.940)--(6.958,5.940)--(6.959,5.940)--(6.960,5.940)%
  --(6.961,5.940)--(6.962,5.940)--(6.963,5.940)--(6.964,5.940)--(6.965,5.940)--(6.966,5.940)%
  --(6.967,5.940)--(6.968,5.940)--(6.969,5.940)--(6.970,5.940)--(6.970,5.941)--(6.971,5.941)%
  --(6.972,5.941)--(6.973,5.941)--(6.974,5.941)--(6.975,5.941)--(6.976,5.941)--(6.977,5.941)%
  --(6.978,5.941)--(6.979,5.941)--(6.980,5.941)--(6.981,5.941)--(6.982,5.941)--(6.983,5.941)%
  --(6.984,5.941)--(6.985,5.941)--(6.986,5.941)--(6.987,5.941)--(6.988,5.941)--(6.989,5.941)%
  --(6.990,5.941)--(6.991,5.941)--(6.992,5.941)--(6.993,5.941)--(6.994,5.941)--(6.995,5.941)%
  --(6.996,5.941)--(6.996,5.942)--(6.997,5.942)--(6.998,5.942)--(6.999,5.942)--(7.000,5.942)%
  --(7.001,5.942)--(7.002,5.942)--(7.003,5.942)--(7.004,5.942)--(7.005,5.942)--(7.006,5.942)%
  --(7.007,5.942)--(7.008,5.942)--(7.009,5.942)--(7.010,5.942)--(7.011,5.942)--(7.012,5.942)%
  --(7.013,5.942)--(7.014,5.942)--(7.015,5.942)--(7.016,5.942)--(7.017,5.942)--(7.018,5.942)%
  --(7.019,5.942)--(7.020,5.942)--(7.021,5.942)--(7.022,5.942)--(7.022,5.943)--(7.023,5.943)%
  --(7.024,5.943)--(7.025,5.943)--(7.026,5.943)--(7.027,5.943)--(7.028,5.943)--(7.029,5.943)%
  --(7.030,5.943)--(7.031,5.943)--(7.032,5.943)--(7.033,5.943)--(7.034,5.943)--(7.035,5.943)%
  --(7.036,5.943)--(7.037,5.943)--(7.038,5.943)--(7.039,5.943)--(7.040,5.943)--(7.041,5.943)%
  --(7.042,5.943)--(7.043,5.943)--(7.044,5.943)--(7.045,5.943)--(7.046,5.943)--(7.047,5.943)%
  --(7.048,5.943)--(7.049,5.943)--(7.049,5.944)--(7.050,5.944)--(7.051,5.944)--(7.052,5.944)%
  --(7.053,5.944)--(7.054,5.944)--(7.055,5.944)--(7.056,5.944)--(7.057,5.944)--(7.058,5.944)%
  --(7.059,5.944)--(7.060,5.944)--(7.061,5.944)--(7.062,5.944)--(7.063,5.944)--(7.064,5.944)%
  --(7.065,5.944)--(7.066,5.944)--(7.067,5.944)--(7.068,5.944)--(7.069,5.944)--(7.070,5.944)%
  --(7.071,5.944)--(7.072,5.944)--(7.073,5.944)--(7.074,5.944)--(7.075,5.944)--(7.076,5.944)%
  --(7.077,5.944)--(7.077,5.945)--(7.078,5.945)--(7.079,5.945)--(7.080,5.945)--(7.081,5.945)%
  --(7.082,5.945)--(7.083,5.945)--(7.084,5.945)--(7.085,5.945)--(7.086,5.945)--(7.087,5.945)%
  --(7.088,5.945)--(7.089,5.945)--(7.090,5.945)--(7.091,5.945)--(7.092,5.945)--(7.093,5.945)%
  --(7.094,5.945)--(7.095,5.945)--(7.096,5.945)--(7.097,5.945)--(7.098,5.945)--(7.099,5.945)%
  --(7.100,5.945)--(7.101,5.945)--(7.102,5.945)--(7.103,5.945)--(7.104,5.945)--(7.105,5.945)%
  --(7.105,5.946)--(7.106,5.946)--(7.107,5.946)--(7.108,5.946)--(7.109,5.946)--(7.110,5.946)%
  --(7.111,5.946)--(7.112,5.946)--(7.113,5.946)--(7.114,5.946)--(7.115,5.946)--(7.116,5.946)%
  --(7.117,5.946)--(7.118,5.946)--(7.119,5.946)--(7.120,5.946)--(7.121,5.946)--(7.122,5.946)%
  --(7.123,5.946)--(7.124,5.946)--(7.125,5.946)--(7.126,5.946)--(7.127,5.946)--(7.128,5.946)%
  --(7.129,5.946)--(7.130,5.946)--(7.131,5.946)--(7.132,5.946)--(7.133,5.946)--(7.134,5.946)%
  --(7.134,5.947)--(7.135,5.947)--(7.136,5.947)--(7.137,5.947)--(7.138,5.947)--(7.139,5.947)%
  --(7.140,5.947)--(7.141,5.947)--(7.142,5.947)--(7.143,5.947)--(7.144,5.947)--(7.145,5.947)%
  --(7.146,5.947)--(7.147,5.947)--(7.148,5.947)--(7.149,5.947)--(7.150,5.947)--(7.151,5.947)%
  --(7.152,5.947)--(7.153,5.947)--(7.154,5.947)--(7.155,5.947)--(7.156,5.947)--(7.157,5.947)%
  --(7.158,5.947)--(7.159,5.947)--(7.160,5.947)--(7.161,5.947)--(7.162,5.947)--(7.163,5.947)%
  --(7.164,5.947)--(7.164,5.948)--(7.165,5.948)--(7.166,5.948)--(7.167,5.948)--(7.168,5.948)%
  --(7.169,5.948)--(7.170,5.948)--(7.171,5.948)--(7.172,5.948)--(7.173,5.948)--(7.174,5.948)%
  --(7.175,5.948)--(7.176,5.948)--(7.177,5.948)--(7.178,5.948)--(7.179,5.948)--(7.180,5.948)%
  --(7.181,5.948)--(7.182,5.948)--(7.183,5.948)--(7.184,5.948)--(7.185,5.948)--(7.186,5.948)%
  --(7.187,5.948)--(7.188,5.948)--(7.189,5.948)--(7.190,5.948)--(7.191,5.948)--(7.192,5.948)%
  --(7.193,5.948)--(7.194,5.948)--(7.195,5.948)--(7.195,5.949)--(7.196,5.949)--(7.197,5.949)%
  --(7.198,5.949)--(7.199,5.949)--(7.200,5.949)--(7.201,5.949)--(7.202,5.949)--(7.203,5.949)%
  --(7.204,5.949)--(7.205,5.949)--(7.206,5.949)--(7.207,5.949)--(7.208,5.949)--(7.209,5.949)%
  --(7.210,5.949)--(7.211,5.949)--(7.212,5.949)--(7.213,5.949)--(7.214,5.949)--(7.215,5.949)%
  --(7.216,5.949)--(7.217,5.949)--(7.218,5.949)--(7.219,5.949)--(7.220,5.949)--(7.221,5.949)%
  --(7.222,5.949)--(7.223,5.949)--(7.224,5.949)--(7.225,5.949)--(7.226,5.949)--(7.226,5.950)%
  --(7.227,5.950)--(7.228,5.950)--(7.229,5.950)--(7.230,5.950)--(7.231,5.950)--(7.232,5.950)%
  --(7.233,5.950)--(7.234,5.950)--(7.235,5.950)--(7.236,5.950)--(7.237,5.950)--(7.238,5.950)%
  --(7.239,5.950)--(7.240,5.950)--(7.241,5.950)--(7.242,5.950)--(7.243,5.950)--(7.244,5.950)%
  --(7.245,5.950)--(7.246,5.950)--(7.247,5.950)--(7.248,5.950)--(7.249,5.950)--(7.250,5.950)%
  --(7.251,5.950)--(7.252,5.950)--(7.253,5.950)--(7.254,5.950)--(7.255,5.950)--(7.256,5.950)%
  --(7.257,5.950)--(7.258,5.950)--(7.258,5.951)--(7.259,5.951)--(7.260,5.951)--(7.261,5.951)%
  --(7.262,5.951)--(7.263,5.951)--(7.264,5.951)--(7.265,5.951)--(7.266,5.951)--(7.267,5.951)%
  --(7.268,5.951)--(7.269,5.951)--(7.270,5.951)--(7.271,5.951)--(7.272,5.951)--(7.273,5.951)%
  --(7.274,5.951)--(7.275,5.951)--(7.276,5.951)--(7.277,5.951)--(7.278,5.951)--(7.279,5.951)%
  --(7.280,5.951)--(7.281,5.951)--(7.282,5.951)--(7.283,5.951)--(7.284,5.951)--(7.285,5.951)%
  --(7.286,5.951)--(7.287,5.951)--(7.288,5.951)--(7.289,5.951)--(7.290,5.951)--(7.291,5.951)%
  --(7.291,5.952)--(7.292,5.952)--(7.293,5.952)--(7.294,5.952)--(7.295,5.952)--(7.296,5.952)%
  --(7.297,5.952)--(7.298,5.952)--(7.299,5.952)--(7.300,5.952)--(7.301,5.952)--(7.302,5.952)%
  --(7.303,5.952)--(7.304,5.952)--(7.305,5.952)--(7.306,5.952)--(7.307,5.952)--(7.308,5.952)%
  --(7.309,5.952)--(7.310,5.952)--(7.311,5.952)--(7.312,5.952)--(7.313,5.952)--(7.314,5.952)%
  --(7.315,5.952)--(7.316,5.952)--(7.317,5.952)--(7.318,5.952)--(7.319,5.952)--(7.320,5.952)%
  --(7.321,5.952)--(7.322,5.952)--(7.323,5.952)--(7.324,5.952)--(7.325,5.952)--(7.325,5.953)%
  --(7.326,5.953)--(7.327,5.953)--(7.328,5.953)--(7.329,5.953)--(7.330,5.953)--(7.331,5.953)%
  --(7.332,5.953)--(7.333,5.953)--(7.334,5.953)--(7.335,5.953)--(7.336,5.953)--(7.337,5.953)%
  --(7.338,5.953)--(7.339,5.953)--(7.340,5.953)--(7.341,5.953)--(7.342,5.953)--(7.343,5.953)%
  --(7.344,5.953)--(7.345,5.953)--(7.346,5.953)--(7.347,5.953)--(7.348,5.953)--(7.349,5.953)%
  --(7.350,5.953)--(7.351,5.953)--(7.352,5.953)--(7.353,5.953)--(7.354,5.953)--(7.355,5.953)%
  --(7.356,5.953)--(7.357,5.953)--(7.358,5.953)--(7.359,5.953)--(7.360,5.953)--(7.360,5.954)%
  --(7.361,5.954)--(7.362,5.954)--(7.363,5.954)--(7.364,5.954)--(7.365,5.954)--(7.366,5.954)%
  --(7.367,5.954)--(7.368,5.954)--(7.369,5.954)--(7.370,5.954)--(7.371,5.954)--(7.372,5.954)%
  --(7.373,5.954)--(7.374,5.954)--(7.375,5.954)--(7.376,5.954)--(7.377,5.954)--(7.378,5.954)%
  --(7.379,5.954)--(7.380,5.954)--(7.381,5.954)--(7.382,5.954)--(7.383,5.954)--(7.384,5.954)%
  --(7.385,5.954)--(7.386,5.954)--(7.387,5.954)--(7.388,5.954)--(7.389,5.954)--(7.390,5.954)%
  --(7.391,5.954)--(7.392,5.954)--(7.393,5.954)--(7.394,5.954)--(7.395,5.954)--(7.396,5.954)%
  --(7.396,5.955)--(7.397,5.955)--(7.398,5.955)--(7.399,5.955)--(7.400,5.955)--(7.401,5.955)%
  --(7.402,5.955)--(7.403,5.955)--(7.404,5.955)--(7.405,5.955)--(7.406,5.955)--(7.407,5.955)%
  --(7.408,5.955)--(7.409,5.955)--(7.410,5.955)--(7.411,5.955)--(7.412,5.955)--(7.413,5.955)%
  --(7.414,5.955)--(7.415,5.955)--(7.416,5.955)--(7.417,5.955)--(7.418,5.955)--(7.419,5.955)%
  --(7.420,5.955)--(7.421,5.955)--(7.422,5.955)--(7.423,5.955)--(7.424,5.955)--(7.425,5.955)%
  --(7.426,5.955)--(7.427,5.955)--(7.428,5.955)--(7.429,5.955)--(7.430,5.955)--(7.431,5.955)%
  --(7.432,5.955)--(7.433,5.955)--(7.433,5.956)--(7.434,5.956)--(7.435,5.956)--(7.436,5.956)%
  --(7.437,5.956)--(7.438,5.956)--(7.439,5.956)--(7.440,5.956)--(7.441,5.956)--(7.442,5.956)%
  --(7.443,5.956)--(7.444,5.956)--(7.445,5.956)--(7.446,5.956)--(7.447,5.956);
\gpsetdashtype{gp dt 2}
\gpsetlinewidth{1.00}
\draw[gp path] (1.320,5.988)--(1.321,5.988)--(1.322,5.988)--(1.323,5.988)--(1.324,5.988)%
  --(1.325,5.988)--(1.326,5.988)--(1.327,5.988)--(1.328,5.988)--(1.329,5.988)--(1.330,5.988)%
  --(1.331,5.988)--(1.332,5.988)--(1.333,5.988)--(1.334,5.988)--(1.335,5.988)--(1.336,5.988)%
  --(1.337,5.988)--(1.338,5.988)--(1.339,5.988)--(1.340,5.988)--(1.341,5.988)--(1.342,5.988)%
  --(1.343,5.988)--(1.344,5.988)--(1.345,5.988)--(1.346,5.988)--(1.347,5.988)--(1.348,5.988)%
  --(1.349,5.988)--(1.350,5.988)--(1.351,5.988)--(1.352,5.988)--(1.353,5.988)--(1.354,5.988)%
  --(1.355,5.988)--(1.356,5.988)--(1.357,5.988)--(1.358,5.988)--(1.359,5.988)--(1.360,5.988)%
  --(1.361,5.988)--(1.362,5.988)--(1.363,5.988)--(1.364,5.988)--(1.365,5.988)--(1.366,5.988)%
  --(1.367,5.988)--(1.368,5.988)--(1.369,5.988)--(1.370,5.988)--(1.371,5.988)--(1.372,5.988)%
  --(1.373,5.988)--(1.374,5.988)--(1.375,5.988)--(1.376,5.988)--(1.377,5.988)--(1.378,5.988)%
  --(1.379,5.988)--(1.380,5.988)--(1.381,5.988)--(1.382,5.988)--(1.383,5.988)--(1.384,5.988)%
  --(1.385,5.988)--(1.386,5.988)--(1.387,5.988)--(1.388,5.988)--(1.389,5.988)--(1.390,5.988)%
  --(1.391,5.988)--(1.392,5.988)--(1.393,5.988)--(1.394,5.988)--(1.395,5.988)--(1.396,5.988)%
  --(1.397,5.988)--(1.398,5.988)--(1.399,5.988)--(1.400,5.988)--(1.401,5.988)--(1.402,5.988)%
  --(1.403,5.988)--(1.404,5.988)--(1.405,5.988)--(1.406,5.988)--(1.407,5.988)--(1.408,5.988)%
  --(1.409,5.988)--(1.410,5.988)--(1.411,5.988)--(1.412,5.988)--(1.413,5.988)--(1.414,5.988)%
  --(1.415,5.988)--(1.416,5.988)--(1.417,5.988)--(1.418,5.988)--(1.419,5.988)--(1.420,5.988)%
  --(1.421,5.988)--(1.422,5.988)--(1.423,5.988)--(1.424,5.988)--(1.425,5.988)--(1.426,5.988)%
  --(1.427,5.988)--(1.428,5.988)--(1.429,5.988)--(1.430,5.988)--(1.431,5.988)--(1.432,5.988)%
  --(1.433,5.988)--(1.434,5.988)--(1.435,5.988)--(1.436,5.988)--(1.437,5.988)--(1.438,5.988)%
  --(1.439,5.988)--(1.440,5.988)--(1.441,5.988)--(1.442,5.988)--(1.443,5.988)--(1.444,5.988)%
  --(1.445,5.988)--(1.446,5.988)--(1.447,5.988)--(1.448,5.988)--(1.449,5.988)--(1.450,5.988)%
  --(1.451,5.988)--(1.452,5.988)--(1.453,5.988)--(1.454,5.988)--(1.455,5.988)--(1.456,5.988)%
  --(1.457,5.988)--(1.458,5.988)--(1.459,5.988)--(1.460,5.988)--(1.461,5.988)--(1.462,5.988)%
  --(1.463,5.988)--(1.464,5.988)--(1.465,5.988)--(1.466,5.988)--(1.467,5.988)--(1.468,5.988)%
  --(1.469,5.988)--(1.470,5.988)--(1.471,5.988)--(1.472,5.988)--(1.473,5.988)--(1.474,5.988)%
  --(1.475,5.988)--(1.476,5.988)--(1.477,5.988)--(1.478,5.988)--(1.479,5.988)--(1.480,5.988)%
  --(1.481,5.988)--(1.482,5.988)--(1.483,5.988)--(1.484,5.988)--(1.485,5.988)--(1.486,5.988)%
  --(1.487,5.988)--(1.488,5.988)--(1.489,5.988)--(1.490,5.988)--(1.491,5.988)--(1.492,5.988)%
  --(1.493,5.988)--(1.494,5.988)--(1.495,5.988)--(1.496,5.988)--(1.497,5.988)--(1.498,5.988)%
  --(1.499,5.988)--(1.500,5.988)--(1.501,5.988)--(1.502,5.988)--(1.503,5.988)--(1.504,5.988)%
  --(1.505,5.988)--(1.506,5.988)--(1.507,5.988)--(1.508,5.988)--(1.509,5.988)--(1.510,5.988)%
  --(1.511,5.988)--(1.512,5.988)--(1.513,5.988)--(1.514,5.988)--(1.515,5.988)--(1.516,5.988)%
  --(1.517,5.988)--(1.518,5.988)--(1.519,5.988)--(1.520,5.988)--(1.521,5.988)--(1.522,5.988)%
  --(1.523,5.988)--(1.524,5.988)--(1.525,5.988)--(1.526,5.988)--(1.527,5.988)--(1.528,5.988)%
  --(1.529,5.988)--(1.530,5.988)--(1.531,5.988)--(1.532,5.988)--(1.533,5.988)--(1.534,5.988)%
  --(1.535,5.988)--(1.536,5.988)--(1.537,5.988)--(1.538,5.988)--(1.539,5.988)--(1.540,5.988)%
  --(1.541,5.988)--(1.542,5.988)--(1.543,5.988)--(1.544,5.988)--(1.545,5.988)--(1.546,5.988)%
  --(1.547,5.988)--(1.548,5.988)--(1.549,5.988)--(1.550,5.988)--(1.551,5.988)--(1.552,5.988)%
  --(1.553,5.988)--(1.554,5.988)--(1.555,5.988)--(1.556,5.988)--(1.557,5.988)--(1.558,5.988)%
  --(1.559,5.988)--(1.560,5.988)--(1.561,5.988)--(1.562,5.988)--(1.563,5.988)--(1.564,5.988)%
  --(1.565,5.988)--(1.566,5.988)--(1.567,5.988)--(1.568,5.988)--(1.569,5.988)--(1.570,5.988)%
  --(1.571,5.988)--(1.572,5.988)--(1.573,5.988)--(1.574,5.988)--(1.575,5.988)--(1.576,5.988)%
  --(1.577,5.988)--(1.578,5.988)--(1.579,5.988)--(1.580,5.988)--(1.581,5.988)--(1.582,5.988)%
  --(1.583,5.988)--(1.584,5.988)--(1.585,5.988)--(1.586,5.988)--(1.587,5.988)--(1.588,5.988)%
  --(1.589,5.988)--(1.590,5.988)--(1.591,5.988)--(1.592,5.988)--(1.593,5.988)--(1.594,5.988)%
  --(1.595,5.988)--(1.596,5.988)--(1.597,5.988)--(1.598,5.988)--(1.599,5.988)--(1.600,5.988)%
  --(1.601,5.988)--(1.602,5.988)--(1.603,5.988)--(1.604,5.988)--(1.605,5.988)--(1.606,5.988)%
  --(1.607,5.988)--(1.608,5.988)--(1.609,5.988)--(1.610,5.988)--(1.611,5.988)--(1.612,5.988)%
  --(1.613,5.988)--(1.614,5.988)--(1.615,5.988)--(1.616,5.988)--(1.617,5.988)--(1.618,5.988)%
  --(1.619,5.988)--(1.620,5.988)--(1.621,5.988)--(1.622,5.988)--(1.623,5.988)--(1.624,5.988)%
  --(1.625,5.988)--(1.626,5.988)--(1.627,5.988)--(1.628,5.988)--(1.629,5.988)--(1.630,5.988)%
  --(1.631,5.988)--(1.632,5.988)--(1.633,5.988)--(1.634,5.988)--(1.635,5.988)--(1.636,5.988)%
  --(1.637,5.988)--(1.638,5.988)--(1.639,5.988)--(1.640,5.988)--(1.641,5.988)--(1.642,5.988)%
  --(1.643,5.988)--(1.644,5.988)--(1.645,5.988)--(1.646,5.988)--(1.647,5.988)--(1.648,5.988)%
  --(1.649,5.988)--(1.650,5.988)--(1.651,5.988)--(1.652,5.988)--(1.653,5.988)--(1.654,5.988)%
  --(1.655,5.988)--(1.656,5.988)--(1.657,5.988)--(1.658,5.988)--(1.659,5.988)--(1.660,5.988)%
  --(1.661,5.988)--(1.662,5.988)--(1.663,5.988)--(1.664,5.988)--(1.665,5.988)--(1.666,5.988)%
  --(1.667,5.988)--(1.668,5.988)--(1.669,5.988)--(1.670,5.988)--(1.671,5.988)--(1.672,5.988)%
  --(1.673,5.988)--(1.674,5.988)--(1.675,5.988)--(1.676,5.988)--(1.677,5.988)--(1.678,5.988)%
  --(1.679,5.988)--(1.680,5.988)--(1.681,5.988)--(1.682,5.988)--(1.683,5.988)--(1.684,5.988)%
  --(1.685,5.988)--(1.686,5.988)--(1.687,5.988)--(1.688,5.988)--(1.689,5.988)--(1.690,5.988)%
  --(1.691,5.988)--(1.692,5.988)--(1.693,5.988)--(1.694,5.988)--(1.695,5.988)--(1.696,5.988)%
  --(1.697,5.988)--(1.698,5.988)--(1.699,5.988)--(1.700,5.988)--(1.701,5.988)--(1.702,5.988)%
  --(1.703,5.988)--(1.704,5.988)--(1.705,5.988)--(1.706,5.988)--(1.707,5.988)--(1.708,5.988)%
  --(1.709,5.988)--(1.710,5.988)--(1.711,5.988)--(1.712,5.988)--(1.713,5.988)--(1.714,5.988)%
  --(1.715,5.988)--(1.716,5.988)--(1.717,5.988)--(1.718,5.988)--(1.719,5.988)--(1.720,5.988)%
  --(1.721,5.988)--(1.722,5.988)--(1.723,5.988)--(1.724,5.988)--(1.725,5.988)--(1.726,5.988)%
  --(1.727,5.988)--(1.728,5.988)--(1.729,5.988)--(1.730,5.988)--(1.731,5.988)--(1.732,5.988)%
  --(1.733,5.988)--(1.734,5.988)--(1.735,5.988)--(1.736,5.988)--(1.737,5.988)--(1.738,5.988)%
  --(1.739,5.988)--(1.740,5.988)--(1.741,5.988)--(1.742,5.988)--(1.743,5.988)--(1.744,5.988)%
  --(1.745,5.988)--(1.746,5.988)--(1.747,5.988)--(1.748,5.988)--(1.749,5.988)--(1.750,5.988)%
  --(1.751,5.988)--(1.752,5.988)--(1.753,5.988)--(1.754,5.988)--(1.755,5.988)--(1.756,5.988)%
  --(1.757,5.988)--(1.758,5.988)--(1.759,5.988)--(1.760,5.988)--(1.761,5.988)--(1.762,5.988)%
  --(1.763,5.988)--(1.764,5.988)--(1.765,5.988)--(1.766,5.988)--(1.767,5.988)--(1.768,5.988)%
  --(1.769,5.988)--(1.770,5.988)--(1.771,5.988)--(1.772,5.988)--(1.773,5.988)--(1.774,5.988)%
  --(1.775,5.988)--(1.776,5.988)--(1.777,5.988)--(1.778,5.988)--(1.779,5.988)--(1.780,5.988)%
  --(1.781,5.988)--(1.782,5.988)--(1.783,5.988)--(1.784,5.988)--(1.785,5.988)--(1.786,5.988)%
  --(1.787,5.988)--(1.788,5.988)--(1.789,5.988)--(1.790,5.988)--(1.791,5.988)--(1.792,5.988)%
  --(1.793,5.988)--(1.794,5.988)--(1.795,5.988)--(1.796,5.988)--(1.797,5.988)--(1.798,5.988)%
  --(1.799,5.988)--(1.800,5.988)--(1.801,5.988)--(1.802,5.988)--(1.803,5.988)--(1.804,5.988)%
  --(1.805,5.988)--(1.806,5.988)--(1.807,5.988)--(1.808,5.988)--(1.809,5.988)--(1.810,5.988)%
  --(1.811,5.988)--(1.812,5.988)--(1.813,5.988)--(1.814,5.988)--(1.815,5.988)--(1.816,5.988)%
  --(1.817,5.988)--(1.818,5.988)--(1.819,5.988)--(1.820,5.988)--(1.821,5.988)--(1.822,5.988)%
  --(1.823,5.988)--(1.824,5.988)--(1.825,5.988)--(1.826,5.988)--(1.827,5.988)--(1.828,5.988)%
  --(1.829,5.988)--(1.830,5.988)--(1.831,5.988)--(1.832,5.988)--(1.833,5.988)--(1.834,5.988)%
  --(1.835,5.988)--(1.836,5.988)--(1.837,5.988)--(1.838,5.988)--(1.839,5.988)--(1.840,5.988)%
  --(1.841,5.988)--(1.842,5.988)--(1.843,5.988)--(1.844,5.988)--(1.845,5.988)--(1.846,5.988)%
  --(1.847,5.988)--(1.848,5.988)--(1.849,5.988)--(1.850,5.988)--(1.851,5.988)--(1.852,5.988)%
  --(1.853,5.988)--(1.854,5.988)--(1.855,5.988)--(1.856,5.988)--(1.857,5.988)--(1.858,5.988)%
  --(1.859,5.988)--(1.860,5.988)--(1.861,5.988)--(1.862,5.988)--(1.863,5.988)--(1.864,5.988)%
  --(1.865,5.988)--(1.866,5.988)--(1.867,5.988)--(1.868,5.988)--(1.869,5.988)--(1.870,5.988)%
  --(1.871,5.988)--(1.872,5.988)--(1.873,5.988)--(1.874,5.988)--(1.875,5.988)--(1.876,5.988)%
  --(1.877,5.988)--(1.878,5.988)--(1.879,5.988)--(1.880,5.988)--(1.881,5.988)--(1.882,5.988)%
  --(1.883,5.988)--(1.884,5.988)--(1.885,5.988)--(1.886,5.988)--(1.887,5.988)--(1.888,5.988)%
  --(1.889,5.988)--(1.890,5.988)--(1.891,5.988)--(1.892,5.988)--(1.893,5.988)--(1.894,5.988)%
  --(1.895,5.988)--(1.896,5.988)--(1.897,5.988)--(1.898,5.988)--(1.899,5.988)--(1.900,5.988)%
  --(1.901,5.988)--(1.902,5.988)--(1.903,5.988)--(1.904,5.988)--(1.905,5.988)--(1.906,5.988)%
  --(1.907,5.988)--(1.908,5.988)--(1.909,5.988)--(1.910,5.988)--(1.911,5.988)--(1.912,5.988)%
  --(1.913,5.988)--(1.914,5.988)--(1.915,5.988)--(1.916,5.988)--(1.917,5.988)--(1.918,5.988)%
  --(1.919,5.988)--(1.920,5.988)--(1.921,5.988)--(1.922,5.988)--(1.923,5.988)--(1.924,5.988)%
  --(1.925,5.988)--(1.926,5.988)--(1.927,5.988)--(1.928,5.988)--(1.929,5.988)--(1.930,5.988)%
  --(1.931,5.988)--(1.932,5.988)--(1.933,5.988)--(1.934,5.988)--(1.935,5.988)--(1.936,5.988)%
  --(1.937,5.988)--(1.938,5.988)--(1.939,5.988)--(1.940,5.988)--(1.941,5.988)--(1.942,5.988)%
  --(1.943,5.988)--(1.944,5.988)--(1.945,5.988)--(1.946,5.988)--(1.947,5.988)--(1.948,5.988)%
  --(1.949,5.988)--(1.950,5.988)--(1.951,5.988)--(1.952,5.988)--(1.953,5.988)--(1.954,5.988)%
  --(1.955,5.988)--(1.956,5.988)--(1.957,5.988)--(1.958,5.988)--(1.959,5.988)--(1.960,5.988)%
  --(1.961,5.988)--(1.962,5.988)--(1.963,5.988)--(1.964,5.988)--(1.965,5.988)--(1.966,5.988)%
  --(1.967,5.988)--(1.968,5.988)--(1.969,5.988)--(1.970,5.988)--(1.971,5.988)--(1.972,5.988)%
  --(1.973,5.988)--(1.974,5.988)--(1.975,5.988)--(1.976,5.988)--(1.977,5.988)--(1.978,5.988)%
  --(1.979,5.988)--(1.980,5.988)--(1.981,5.988)--(1.982,5.988)--(1.983,5.988)--(1.984,5.988)%
  --(1.985,5.988)--(1.986,5.988)--(1.987,5.988)--(1.988,5.988)--(1.989,5.988)--(1.990,5.988)%
  --(1.991,5.988)--(1.992,5.988)--(1.993,5.988)--(1.994,5.988)--(1.995,5.988)--(1.996,5.988)%
  --(1.997,5.988)--(1.998,5.988)--(1.999,5.988)--(2.000,5.988)--(2.001,5.988)--(2.002,5.988)%
  --(2.003,5.988)--(2.004,5.988)--(2.005,5.988)--(2.006,5.988)--(2.007,5.988)--(2.008,5.988)%
  --(2.009,5.988)--(2.010,5.988)--(2.011,5.988)--(2.012,5.988)--(2.013,5.988)--(2.014,5.988)%
  --(2.015,5.988)--(2.016,5.988)--(2.017,5.988)--(2.018,5.988)--(2.019,5.988)--(2.020,5.988)%
  --(2.021,5.988)--(2.022,5.988)--(2.023,5.988)--(2.024,5.988)--(2.025,5.988)--(2.026,5.988)%
  --(2.027,5.988)--(2.028,5.988)--(2.029,5.988)--(2.030,5.988)--(2.031,5.988)--(2.032,5.988)%
  --(2.033,5.988)--(2.034,5.988)--(2.035,5.988)--(2.036,5.988)--(2.037,5.988)--(2.038,5.988)%
  --(2.039,5.988)--(2.040,5.988)--(2.041,5.988)--(2.042,5.988)--(2.043,5.988)--(2.044,5.988)%
  --(2.045,5.988)--(2.046,5.988)--(2.047,5.988)--(2.048,5.988)--(2.049,5.988)--(2.050,5.988)%
  --(2.051,5.988)--(2.052,5.988)--(2.053,5.988)--(2.054,5.988)--(2.055,5.988)--(2.056,5.988)%
  --(2.057,5.988)--(2.058,5.988)--(2.059,5.988)--(2.060,5.988)--(2.061,5.988)--(2.062,5.988)%
  --(2.063,5.988)--(2.064,5.988)--(2.065,5.988)--(2.066,5.988)--(2.067,5.988)--(2.068,5.988)%
  --(2.069,5.988)--(2.070,5.988)--(2.071,5.988)--(2.072,5.988)--(2.073,5.988)--(2.074,5.988)%
  --(2.075,5.988)--(2.076,5.988)--(2.077,5.988)--(2.078,5.988)--(2.079,5.988)--(2.080,5.988)%
  --(2.081,5.988)--(2.082,5.988)--(2.083,5.988)--(2.084,5.988)--(2.085,5.988)--(2.086,5.988)%
  --(2.087,5.988)--(2.088,5.988)--(2.089,5.988)--(2.090,5.988)--(2.091,5.988)--(2.092,5.988)%
  --(2.093,5.988)--(2.094,5.988)--(2.095,5.988)--(2.096,5.988)--(2.097,5.988)--(2.098,5.988)%
  --(2.099,5.988)--(2.100,5.988)--(2.101,5.988)--(2.102,5.988)--(2.103,5.988)--(2.104,5.988)%
  --(2.105,5.988)--(2.106,5.988)--(2.107,5.988)--(2.108,5.988)--(2.109,5.988)--(2.110,5.988)%
  --(2.111,5.988)--(2.112,5.988)--(2.113,5.988)--(2.114,5.988)--(2.115,5.988)--(2.116,5.988)%
  --(2.117,5.988)--(2.118,5.988)--(2.119,5.988)--(2.120,5.988)--(2.121,5.988)--(2.122,5.988)%
  --(2.123,5.988)--(2.124,5.988)--(2.125,5.988)--(2.126,5.988)--(2.127,5.988)--(2.128,5.988)%
  --(2.129,5.988)--(2.130,5.988)--(2.131,5.988)--(2.132,5.988)--(2.133,5.988)--(2.134,5.988)%
  --(2.135,5.988)--(2.136,5.988)--(2.137,5.988)--(2.138,5.988)--(2.139,5.988)--(2.140,5.988)%
  --(2.141,5.988)--(2.142,5.988)--(2.143,5.988)--(2.144,5.988)--(2.145,5.988)--(2.146,5.988)%
  --(2.147,5.988)--(2.148,5.988)--(2.149,5.988)--(2.150,5.988)--(2.151,5.988)--(2.152,5.988)%
  --(2.153,5.988)--(2.154,5.988)--(2.155,5.988)--(2.156,5.988)--(2.157,5.988)--(2.158,5.988)%
  --(2.159,5.988)--(2.160,5.988)--(2.161,5.988)--(2.162,5.988)--(2.163,5.988)--(2.164,5.988)%
  --(2.165,5.988)--(2.166,5.988)--(2.167,5.988)--(2.168,5.988)--(2.169,5.988)--(2.170,5.988)%
  --(2.171,5.988)--(2.172,5.988)--(2.173,5.988)--(2.174,5.988)--(2.175,5.988)--(2.176,5.988)%
  --(2.177,5.988)--(2.178,5.988)--(2.179,5.988)--(2.180,5.988)--(2.181,5.988)--(2.182,5.988)%
  --(2.183,5.988)--(2.184,5.988)--(2.185,5.988)--(2.186,5.988)--(2.187,5.988)--(2.188,5.988)%
  --(2.189,5.988)--(2.190,5.988)--(2.191,5.988)--(2.192,5.988)--(2.193,5.988)--(2.194,5.988)%
  --(2.195,5.988)--(2.196,5.988)--(2.197,5.988)--(2.198,5.988)--(2.199,5.988)--(2.200,5.988)%
  --(2.201,5.988)--(2.202,5.988)--(2.203,5.988)--(2.204,5.988)--(2.205,5.988)--(2.206,5.988)%
  --(2.207,5.988)--(2.208,5.988)--(2.209,5.988)--(2.210,5.988)--(2.211,5.988)--(2.212,5.988)%
  --(2.213,5.988)--(2.214,5.988)--(2.215,5.988)--(2.216,5.988)--(2.217,5.988)--(2.218,5.988)%
  --(2.219,5.988)--(2.220,5.988)--(2.221,5.988)--(2.222,5.988)--(2.223,5.988)--(2.224,5.988)%
  --(2.225,5.988)--(2.226,5.988)--(2.227,5.988)--(2.228,5.988)--(2.229,5.988)--(2.230,5.988)%
  --(2.231,5.988)--(2.232,5.988)--(2.233,5.988)--(2.234,5.988)--(2.235,5.988)--(2.236,5.988)%
  --(2.237,5.988)--(2.238,5.988)--(2.239,5.988)--(2.240,5.988)--(2.241,5.988)--(2.242,5.988)%
  --(2.243,5.988)--(2.244,5.988)--(2.245,5.988)--(2.246,5.988)--(2.247,5.988)--(2.248,5.988)%
  --(2.249,5.988)--(2.250,5.988)--(2.251,5.988)--(2.252,5.988)--(2.253,5.988)--(2.254,5.988)%
  --(2.255,5.988)--(2.256,5.988)--(2.257,5.988)--(2.258,5.988)--(2.259,5.988)--(2.260,5.988)%
  --(2.261,5.988)--(2.262,5.988)--(2.263,5.988)--(2.264,5.988)--(2.265,5.988)--(2.266,5.988)%
  --(2.267,5.988)--(2.268,5.988)--(2.269,5.988)--(2.270,5.988)--(2.271,5.988)--(2.272,5.988)%
  --(2.273,5.988)--(2.274,5.988)--(2.275,5.988)--(2.276,5.988)--(2.277,5.988)--(2.278,5.988)%
  --(2.279,5.988)--(2.280,5.988)--(2.281,5.988)--(2.282,5.988)--(2.283,5.988)--(2.284,5.988)%
  --(2.285,5.988)--(2.286,5.988)--(2.287,5.988)--(2.288,5.988)--(2.289,5.988)--(2.290,5.988)%
  --(2.291,5.988)--(2.292,5.988)--(2.293,5.988)--(2.294,5.988)--(2.295,5.988)--(2.296,5.988)%
  --(2.297,5.988)--(2.298,5.988)--(2.299,5.988)--(2.300,5.988)--(2.301,5.988)--(2.302,5.988)%
  --(2.303,5.988)--(2.304,5.988)--(2.305,5.988)--(2.306,5.988)--(2.307,5.988)--(2.308,5.988)%
  --(2.309,5.988)--(2.310,5.988)--(2.311,5.988)--(2.312,5.988)--(2.313,5.988)--(2.314,5.988)%
  --(2.315,5.988)--(2.316,5.988)--(2.317,5.988)--(2.318,5.988)--(2.319,5.988)--(2.320,5.988)%
  --(2.321,5.988)--(2.322,5.988)--(2.323,5.988)--(2.324,5.988)--(2.325,5.988)--(2.326,5.988)%
  --(2.327,5.988)--(2.328,5.988)--(2.329,5.988)--(2.330,5.988)--(2.331,5.988)--(2.332,5.988)%
  --(2.333,5.988)--(2.334,5.988)--(2.335,5.988)--(2.336,5.988)--(2.337,5.988)--(2.338,5.988)%
  --(2.339,5.988)--(2.340,5.988)--(2.341,5.988)--(2.342,5.988)--(2.343,5.988)--(2.344,5.988)%
  --(2.345,5.988)--(2.346,5.988)--(2.347,5.988)--(2.348,5.988)--(2.349,5.988)--(2.350,5.988)%
  --(2.351,5.988)--(2.352,5.988)--(2.353,5.988)--(2.354,5.988)--(2.355,5.988)--(2.356,5.988)%
  --(2.357,5.988)--(2.358,5.988)--(2.359,5.988)--(2.360,5.988)--(2.361,5.988)--(2.362,5.988)%
  --(2.363,5.988)--(2.364,5.988)--(2.365,5.988)--(2.366,5.988)--(2.367,5.988)--(2.368,5.988)%
  --(2.369,5.988)--(2.370,5.988)--(2.371,5.988)--(2.372,5.988)--(2.373,5.988)--(2.374,5.988)%
  --(2.375,5.988)--(2.376,5.988)--(2.377,5.988)--(2.378,5.988)--(2.379,5.988)--(2.380,5.988)%
  --(2.381,5.988)--(2.382,5.988)--(2.383,5.988)--(2.384,5.988)--(2.385,5.988)--(2.386,5.988)%
  --(2.387,5.988)--(2.388,5.988)--(2.389,5.988)--(2.390,5.988)--(2.391,5.988)--(2.392,5.988)%
  --(2.393,5.988)--(2.394,5.988)--(2.395,5.988)--(2.396,5.988)--(2.397,5.988)--(2.398,5.988)%
  --(2.399,5.988)--(2.400,5.988)--(2.401,5.988)--(2.402,5.988)--(2.403,5.988)--(2.404,5.988)%
  --(2.405,5.988)--(2.406,5.988)--(2.407,5.988)--(2.408,5.988)--(2.409,5.988)--(2.410,5.988)%
  --(2.411,5.988)--(2.412,5.988)--(2.413,5.988)--(2.414,5.988)--(2.415,5.988)--(2.416,5.988)%
  --(2.417,5.988)--(2.418,5.988)--(2.419,5.988)--(2.420,5.988)--(2.421,5.988)--(2.422,5.988)%
  --(2.423,5.988)--(2.424,5.988)--(2.425,5.988)--(2.426,5.988)--(2.427,5.988)--(2.428,5.988)%
  --(2.429,5.988)--(2.430,5.988)--(2.431,5.988)--(2.432,5.988)--(2.433,5.988)--(2.434,5.988)%
  --(2.435,5.988)--(2.436,5.988)--(2.437,5.988)--(2.438,5.988)--(2.439,5.988)--(2.440,5.988)%
  --(2.441,5.988)--(2.442,5.988)--(2.443,5.988)--(2.444,5.988)--(2.445,5.988)--(2.446,5.988)%
  --(2.447,5.988)--(2.448,5.988)--(2.449,5.988)--(2.450,5.988)--(2.451,5.988)--(2.452,5.988)%
  --(2.453,5.988)--(2.454,5.988)--(2.455,5.988)--(2.456,5.988)--(2.457,5.988)--(2.458,5.988)%
  --(2.459,5.988)--(2.460,5.988)--(2.461,5.988)--(2.462,5.988)--(2.463,5.988)--(2.464,5.988)%
  --(2.465,5.988)--(2.466,5.988)--(2.467,5.988)--(2.468,5.988)--(2.469,5.988)--(2.470,5.988)%
  --(2.471,5.988)--(2.472,5.988)--(2.473,5.988)--(2.474,5.988)--(2.475,5.988)--(2.476,5.988)%
  --(2.477,5.988)--(2.478,5.988)--(2.479,5.988)--(2.480,5.988)--(2.481,5.988)--(2.482,5.988)%
  --(2.483,5.988)--(2.484,5.988)--(2.485,5.988)--(2.486,5.988)--(2.487,5.988)--(2.488,5.988)%
  --(2.489,5.988)--(2.490,5.988)--(2.491,5.988)--(2.492,5.988)--(2.493,5.988)--(2.494,5.988)%
  --(2.495,5.988)--(2.496,5.988)--(2.497,5.988)--(2.498,5.988)--(2.499,5.988)--(2.500,5.988)%
  --(2.501,5.988)--(2.502,5.988)--(2.503,5.988)--(2.504,5.988)--(2.505,5.988)--(2.506,5.988)%
  --(2.507,5.988)--(2.508,5.988)--(2.509,5.988)--(2.510,5.988)--(2.511,5.988)--(2.512,5.988)%
  --(2.513,5.988)--(2.514,5.988)--(2.515,5.988)--(2.516,5.988)--(2.517,5.988)--(2.518,5.988)%
  --(2.519,5.988)--(2.520,5.988)--(2.521,5.988)--(2.522,5.988)--(2.523,5.988)--(2.524,5.988)%
  --(2.525,5.988)--(2.526,5.988)--(2.527,5.988)--(2.528,5.988)--(2.529,5.988)--(2.530,5.988)%
  --(2.531,5.988)--(2.532,5.988)--(2.533,5.988)--(2.534,5.988)--(2.535,5.988)--(2.536,5.988)%
  --(2.537,5.988)--(2.538,5.988)--(2.539,5.988)--(2.540,5.988)--(2.541,5.988)--(2.542,5.988)%
  --(2.543,5.988)--(2.544,5.988)--(2.545,5.988)--(2.546,5.988)--(2.547,5.988)--(2.548,5.988)%
  --(2.549,5.988)--(2.550,5.988)--(2.551,5.988)--(2.552,5.988)--(2.553,5.988)--(2.554,5.988)%
  --(2.555,5.988)--(2.556,5.988)--(2.557,5.988)--(2.558,5.988)--(2.559,5.988)--(2.560,5.988)%
  --(2.561,5.988)--(2.562,5.988)--(2.563,5.988)--(2.564,5.988)--(2.565,5.988)--(2.566,5.988)%
  --(2.567,5.988)--(2.568,5.988)--(2.569,5.988)--(2.570,5.988)--(2.571,5.988)--(2.572,5.988)%
  --(2.573,5.988)--(2.574,5.988)--(2.575,5.988)--(2.576,5.988)--(2.577,5.988)--(2.578,5.988)%
  --(2.579,5.988)--(2.580,5.988)--(2.581,5.988)--(2.582,5.988)--(2.583,5.988)--(2.584,5.988)%
  --(2.585,5.988)--(2.586,5.988)--(2.587,5.988)--(2.588,5.988)--(2.589,5.988)--(2.590,5.988)%
  --(2.591,5.988)--(2.592,5.988)--(2.593,5.988)--(2.594,5.988)--(2.595,5.988)--(2.596,5.988)%
  --(2.597,5.988)--(2.598,5.988)--(2.599,5.988)--(2.600,5.988)--(2.601,5.988)--(2.602,5.988)%
  --(2.603,5.988)--(2.604,5.988)--(2.605,5.988)--(2.606,5.988)--(2.607,5.988)--(2.608,5.988)%
  --(2.609,5.988)--(2.610,5.988)--(2.611,5.988)--(2.612,5.988)--(2.613,5.988)--(2.614,5.988)%
  --(2.615,5.988)--(2.616,5.988)--(2.617,5.988)--(2.618,5.988)--(2.619,5.988)--(2.620,5.988)%
  --(2.621,5.988)--(2.622,5.988)--(2.623,5.988)--(2.624,5.988)--(2.625,5.988)--(2.626,5.988)%
  --(2.627,5.988)--(2.628,5.988)--(2.629,5.988)--(2.630,5.988)--(2.631,5.988)--(2.632,5.988)%
  --(2.633,5.988)--(2.634,5.988)--(2.635,5.988)--(2.636,5.988)--(2.637,5.988)--(2.638,5.988)%
  --(2.639,5.988)--(2.640,5.988)--(2.641,5.988)--(2.642,5.988)--(2.643,5.988)--(2.644,5.988)%
  --(2.645,5.988)--(2.646,5.988)--(2.647,5.988)--(2.648,5.988)--(2.649,5.988)--(2.650,5.988)%
  --(2.651,5.988)--(2.652,5.988)--(2.653,5.988)--(2.654,5.988)--(2.655,5.988)--(2.656,5.988)%
  --(2.657,5.988)--(2.658,5.988)--(2.659,5.988)--(2.660,5.988)--(2.661,5.988)--(2.662,5.988)%
  --(2.663,5.988)--(2.664,5.988)--(2.665,5.988)--(2.666,5.988)--(2.667,5.988)--(2.668,5.988)%
  --(2.669,5.988)--(2.670,5.988)--(2.671,5.988)--(2.672,5.988)--(2.673,5.988)--(2.674,5.988)%
  --(2.675,5.988)--(2.676,5.988)--(2.677,5.988)--(2.678,5.988)--(2.679,5.988)--(2.680,5.988)%
  --(2.681,5.988)--(2.682,5.988)--(2.683,5.988)--(2.684,5.988)--(2.685,5.988)--(2.686,5.988)%
  --(2.687,5.988)--(2.688,5.988)--(2.689,5.988)--(2.690,5.988)--(2.691,5.988)--(2.692,5.988)%
  --(2.693,5.988)--(2.694,5.988)--(2.695,5.988)--(2.696,5.988)--(2.697,5.988)--(2.698,5.988)%
  --(2.699,5.988)--(2.700,5.988)--(2.701,5.988)--(2.702,5.988)--(2.703,5.988)--(2.704,5.988)%
  --(2.705,5.988)--(2.706,5.988)--(2.707,5.988)--(2.708,5.988)--(2.709,5.988)--(2.710,5.988)%
  --(2.711,5.988)--(2.712,5.988)--(2.713,5.988)--(2.714,5.988)--(2.715,5.988)--(2.716,5.988)%
  --(2.717,5.988)--(2.718,5.988)--(2.719,5.988)--(2.720,5.988)--(2.721,5.988)--(2.722,5.988)%
  --(2.723,5.988)--(2.724,5.988)--(2.725,5.988)--(2.726,5.988)--(2.727,5.988)--(2.728,5.988)%
  --(2.729,5.988)--(2.730,5.988)--(2.731,5.988)--(2.732,5.988)--(2.733,5.988)--(2.734,5.988)%
  --(2.735,5.988)--(2.736,5.988)--(2.737,5.988)--(2.738,5.988)--(2.739,5.988)--(2.740,5.988)%
  --(2.741,5.988)--(2.742,5.988)--(2.743,5.988)--(2.744,5.988)--(2.745,5.988)--(2.746,5.988)%
  --(2.747,5.988)--(2.748,5.988)--(2.749,5.988)--(2.750,5.988)--(2.751,5.988)--(2.752,5.988)%
  --(2.753,5.988)--(2.754,5.988)--(2.755,5.988)--(2.756,5.988)--(2.757,5.988)--(2.758,5.988)%
  --(2.759,5.988)--(2.760,5.988)--(2.761,5.988)--(2.762,5.988)--(2.763,5.988)--(2.764,5.988)%
  --(2.765,5.988)--(2.766,5.988)--(2.767,5.988)--(2.768,5.988)--(2.769,5.988)--(2.770,5.988)%
  --(2.771,5.988)--(2.772,5.988)--(2.773,5.988)--(2.774,5.988)--(2.775,5.988)--(2.776,5.988)%
  --(2.777,5.988)--(2.778,5.988)--(2.779,5.988)--(2.780,5.988)--(2.781,5.988)--(2.782,5.988)%
  --(2.783,5.988)--(2.784,5.988)--(2.785,5.988)--(2.786,5.988)--(2.787,5.988)--(2.788,5.988)%
  --(2.789,5.988)--(2.790,5.988)--(2.791,5.988)--(2.792,5.988)--(2.793,5.988)--(2.794,5.988)%
  --(2.795,5.988)--(2.796,5.988)--(2.797,5.988)--(2.798,5.988)--(2.799,5.988)--(2.800,5.988)%
  --(2.801,5.988)--(2.802,5.988)--(2.803,5.988)--(2.804,5.988)--(2.805,5.988)--(2.806,5.988)%
  --(2.807,5.988)--(2.808,5.988)--(2.809,5.988)--(2.810,5.988)--(2.811,5.988)--(2.812,5.988)%
  --(2.813,5.988)--(2.814,5.988)--(2.815,5.988)--(2.816,5.988)--(2.817,5.988)--(2.818,5.988)%
  --(2.819,5.988)--(2.820,5.988)--(2.821,5.988)--(2.822,5.988)--(2.823,5.988)--(2.824,5.988)%
  --(2.825,5.988)--(2.826,5.988)--(2.827,5.988)--(2.828,5.988)--(2.829,5.988)--(2.830,5.988)%
  --(2.831,5.988)--(2.832,5.988)--(2.833,5.988)--(2.834,5.988)--(2.835,5.988)--(2.836,5.988)%
  --(2.837,5.988)--(2.838,5.988)--(2.839,5.988)--(2.840,5.988)--(2.841,5.988)--(2.842,5.988)%
  --(2.843,5.988)--(2.844,5.988)--(2.845,5.988)--(2.846,5.988)--(2.847,5.988)--(2.848,5.988)%
  --(2.849,5.988)--(2.850,5.988)--(2.851,5.988)--(2.852,5.988)--(2.853,5.988)--(2.854,5.988)%
  --(2.855,5.988)--(2.856,5.988)--(2.857,5.988)--(2.858,5.988)--(2.859,5.988)--(2.860,5.988)%
  --(2.861,5.988)--(2.862,5.988)--(2.863,5.988)--(2.864,5.988)--(2.865,5.988)--(2.866,5.988)%
  --(2.867,5.988)--(2.868,5.988)--(2.869,5.988)--(2.870,5.988)--(2.871,5.988)--(2.872,5.988)%
  --(2.873,5.988)--(2.874,5.988)--(2.875,5.988)--(2.876,5.988)--(2.877,5.988)--(2.878,5.988)%
  --(2.879,5.988)--(2.880,5.988)--(2.881,5.988)--(2.882,5.988)--(2.883,5.988)--(2.884,5.988)%
  --(2.885,5.988)--(2.886,5.988)--(2.887,5.988)--(2.888,5.988)--(2.889,5.988)--(2.890,5.988)%
  --(2.891,5.988)--(2.892,5.988)--(2.893,5.988)--(2.894,5.988)--(2.895,5.988)--(2.896,5.988)%
  --(2.897,5.988)--(2.898,5.988)--(2.899,5.988)--(2.900,5.988)--(2.901,5.988)--(2.902,5.988)%
  --(2.903,5.988)--(2.904,5.988)--(2.905,5.988)--(2.906,5.988)--(2.907,5.988)--(2.908,5.988)%
  --(2.909,5.988)--(2.910,5.988)--(2.911,5.988)--(2.912,5.988)--(2.913,5.988)--(2.914,5.988)%
  --(2.915,5.988)--(2.916,5.988)--(2.917,5.988)--(2.918,5.988)--(2.919,5.988)--(2.920,5.988)%
  --(2.921,5.988)--(2.922,5.988)--(2.923,5.988)--(2.924,5.988)--(2.925,5.988)--(2.926,5.988)%
  --(2.927,5.988)--(2.928,5.988)--(2.929,5.988)--(2.930,5.988)--(2.931,5.988)--(2.932,5.988)%
  --(2.933,5.988)--(2.934,5.988)--(2.935,5.988)--(2.936,5.988)--(2.937,5.988)--(2.938,5.988)%
  --(2.939,5.988)--(2.940,5.988)--(2.941,5.988)--(2.942,5.988)--(2.943,5.988)--(2.944,5.988)%
  --(2.945,5.988)--(2.946,5.988)--(2.947,5.988)--(2.948,5.988)--(2.949,5.988)--(2.950,5.988)%
  --(2.951,5.988)--(2.952,5.988)--(2.953,5.988)--(2.954,5.988)--(2.955,5.988)--(2.956,5.988)%
  --(2.957,5.988)--(2.958,5.988)--(2.959,5.988)--(2.960,5.988)--(2.961,5.988)--(2.962,5.988)%
  --(2.963,5.988)--(2.964,5.988)--(2.965,5.988)--(2.966,5.988)--(2.967,5.988)--(2.968,5.988)%
  --(2.969,5.988)--(2.970,5.988)--(2.971,5.988)--(2.972,5.988)--(2.973,5.988)--(2.974,5.988)%
  --(2.975,5.988)--(2.976,5.988)--(2.977,5.988)--(2.978,5.988)--(2.979,5.988)--(2.980,5.988)%
  --(2.981,5.988)--(2.982,5.988)--(2.983,5.988)--(2.984,5.988)--(2.985,5.988)--(2.986,5.988)%
  --(2.987,5.988)--(2.988,5.988)--(2.989,5.988)--(2.990,5.988)--(2.991,5.988)--(2.992,5.988)%
  --(2.993,5.988)--(2.994,5.988)--(2.995,5.988)--(2.996,5.988)--(2.997,5.988)--(2.998,5.988)%
  --(2.999,5.988)--(3.000,5.988)--(3.001,5.988)--(3.002,5.988)--(3.003,5.988)--(3.004,5.988)%
  --(3.005,5.988)--(3.006,5.988)--(3.007,5.988)--(3.008,5.988)--(3.009,5.988)--(3.010,5.988)%
  --(3.011,5.988)--(3.012,5.988)--(3.013,5.988)--(3.014,5.988)--(3.015,5.988)--(3.016,5.988)%
  --(3.017,5.988)--(3.018,5.988)--(3.019,5.988)--(3.020,5.988)--(3.021,5.988)--(3.022,5.988)%
  --(3.023,5.988)--(3.024,5.988)--(3.025,5.988)--(3.026,5.988)--(3.027,5.988)--(3.028,5.988)%
  --(3.029,5.988)--(3.030,5.988)--(3.031,5.988)--(3.032,5.988)--(3.033,5.988)--(3.034,5.988)%
  --(3.035,5.988)--(3.036,5.988)--(3.037,5.988)--(3.038,5.988)--(3.039,5.988)--(3.040,5.988)%
  --(3.041,5.988)--(3.042,5.988)--(3.043,5.988)--(3.044,5.988)--(3.045,5.988)--(3.046,5.988)%
  --(3.047,5.988)--(3.048,5.988)--(3.049,5.988)--(3.050,5.988)--(3.051,5.988)--(3.052,5.988)%
  --(3.053,5.988)--(3.054,5.988)--(3.055,5.988)--(3.056,5.988)--(3.057,5.988)--(3.058,5.988)%
  --(3.059,5.988)--(3.060,5.988)--(3.061,5.988)--(3.062,5.988)--(3.063,5.988)--(3.064,5.988)%
  --(3.065,5.988)--(3.066,5.988)--(3.067,5.988)--(3.068,5.988)--(3.069,5.988)--(3.070,5.988)%
  --(3.071,5.988)--(3.072,5.988)--(3.073,5.988)--(3.074,5.988)--(3.075,5.988)--(3.076,5.988)%
  --(3.077,5.988)--(3.078,5.988)--(3.079,5.988)--(3.080,5.988)--(3.081,5.988)--(3.082,5.988)%
  --(3.083,5.988)--(3.084,5.988)--(3.085,5.988)--(3.086,5.988)--(3.087,5.988)--(3.088,5.988)%
  --(3.089,5.988)--(3.090,5.988)--(3.091,5.988)--(3.092,5.988)--(3.093,5.988)--(3.094,5.988)%
  --(3.095,5.988)--(3.096,5.988)--(3.097,5.988)--(3.098,5.988)--(3.099,5.988)--(3.100,5.988)%
  --(3.101,5.988)--(3.102,5.988)--(3.103,5.988)--(3.104,5.988)--(3.105,5.988)--(3.106,5.988)%
  --(3.107,5.988)--(3.108,5.988)--(3.109,5.988)--(3.110,5.988)--(3.111,5.988)--(3.112,5.988)%
  --(3.113,5.988)--(3.114,5.988)--(3.115,5.988)--(3.116,5.988)--(3.117,5.988)--(3.118,5.988)%
  --(3.119,5.988)--(3.120,5.988)--(3.121,5.988)--(3.122,5.988)--(3.123,5.988)--(3.124,5.988)%
  --(3.125,5.988)--(3.126,5.988)--(3.127,5.988)--(3.128,5.988)--(3.129,5.988)--(3.130,5.988)%
  --(3.131,5.988)--(3.132,5.988)--(3.133,5.988)--(3.134,5.988)--(3.135,5.988)--(3.136,5.988)%
  --(3.137,5.988)--(3.138,5.988)--(3.139,5.988)--(3.140,5.988)--(3.141,5.988)--(3.142,5.988)%
  --(3.143,5.988)--(3.144,5.988)--(3.145,5.988)--(3.146,5.988)--(3.147,5.988)--(3.148,5.988)%
  --(3.149,5.988)--(3.150,5.988)--(3.151,5.988)--(3.152,5.988)--(3.153,5.988)--(3.154,5.988)%
  --(3.155,5.988)--(3.156,5.988)--(3.157,5.988)--(3.158,5.988)--(3.159,5.988)--(3.160,5.988)%
  --(3.161,5.988)--(3.162,5.988)--(3.163,5.988)--(3.164,5.988)--(3.165,5.988)--(3.166,5.988)%
  --(3.167,5.988)--(3.168,5.988)--(3.169,5.988)--(3.170,5.988)--(3.171,5.988)--(3.172,5.988)%
  --(3.173,5.988)--(3.174,5.988)--(3.175,5.988)--(3.176,5.988)--(3.177,5.988)--(3.178,5.988)%
  --(3.179,5.988)--(3.180,5.988)--(3.181,5.988)--(3.182,5.988)--(3.183,5.988)--(3.184,5.988)%
  --(3.185,5.988)--(3.186,5.988)--(3.187,5.988)--(3.188,5.988)--(3.189,5.988)--(3.190,5.988)%
  --(3.191,5.988)--(3.192,5.988)--(3.193,5.988)--(3.194,5.988)--(3.195,5.988)--(3.196,5.988)%
  --(3.197,5.988)--(3.198,5.988)--(3.199,5.988)--(3.200,5.988)--(3.201,5.988)--(3.202,5.988)%
  --(3.203,5.988)--(3.204,5.988)--(3.205,5.988)--(3.206,5.988)--(3.207,5.988)--(3.208,5.988)%
  --(3.209,5.988)--(3.210,5.988)--(3.211,5.988)--(3.212,5.988)--(3.213,5.988)--(3.214,5.988)%
  --(3.215,5.988)--(3.216,5.988)--(3.217,5.988)--(3.218,5.988)--(3.219,5.988)--(3.220,5.988)%
  --(3.221,5.988)--(3.222,5.988)--(3.223,5.988)--(3.224,5.988)--(3.225,5.988)--(3.226,5.988)%
  --(3.227,5.988)--(3.228,5.988)--(3.229,5.988)--(3.230,5.988)--(3.231,5.988)--(3.232,5.988)%
  --(3.233,5.988)--(3.234,5.988)--(3.235,5.988)--(3.236,5.988)--(3.237,5.988)--(3.238,5.988)%
  --(3.239,5.988)--(3.240,5.988)--(3.241,5.988)--(3.242,5.988)--(3.243,5.988)--(3.244,5.988)%
  --(3.245,5.988)--(3.246,5.988)--(3.247,5.988)--(3.248,5.988)--(3.249,5.988)--(3.250,5.988)%
  --(3.251,5.988)--(3.252,5.988)--(3.253,5.988)--(3.254,5.988)--(3.255,5.988)--(3.256,5.988)%
  --(3.257,5.988)--(3.258,5.988)--(3.259,5.988)--(3.260,5.988)--(3.261,5.988)--(3.262,5.988)%
  --(3.263,5.988)--(3.264,5.988)--(3.265,5.988)--(3.266,5.988)--(3.267,5.988)--(3.268,5.988)%
  --(3.269,5.988)--(3.270,5.988)--(3.271,5.988)--(3.272,5.988)--(3.273,5.988)--(3.274,5.988)%
  --(3.275,5.988)--(3.276,5.988)--(3.277,5.988)--(3.278,5.988)--(3.279,5.988)--(3.280,5.988)%
  --(3.281,5.988)--(3.282,5.988)--(3.283,5.988)--(3.284,5.988)--(3.285,5.988)--(3.286,5.988)%
  --(3.287,5.988)--(3.288,5.988)--(3.289,5.988)--(3.290,5.988)--(3.291,5.988)--(3.292,5.988)%
  --(3.293,5.988)--(3.294,5.988)--(3.295,5.988)--(3.296,5.988)--(3.297,5.988)--(3.298,5.988)%
  --(3.299,5.988)--(3.300,5.988)--(3.301,5.988)--(3.302,5.988)--(3.303,5.988)--(3.304,5.988)%
  --(3.305,5.988)--(3.306,5.988)--(3.307,5.988)--(3.308,5.988)--(3.309,5.988)--(3.310,5.988)%
  --(3.311,5.988)--(3.312,5.988)--(3.313,5.988)--(3.314,5.988)--(3.315,5.988)--(3.316,5.988)%
  --(3.317,5.988)--(3.318,5.988)--(3.319,5.988)--(3.320,5.988)--(3.321,5.988)--(3.322,5.988)%
  --(3.323,5.988)--(3.324,5.988)--(3.325,5.988)--(3.326,5.988)--(3.327,5.988)--(3.328,5.988)%
  --(3.329,5.988)--(3.330,5.988)--(3.331,5.988)--(3.332,5.988)--(3.333,5.988)--(3.334,5.988)%
  --(3.335,5.988)--(3.336,5.988)--(3.337,5.988)--(3.338,5.988)--(3.339,5.988)--(3.340,5.988)%
  --(3.341,5.988)--(3.342,5.988)--(3.343,5.988)--(3.344,5.988)--(3.345,5.988)--(3.346,5.988)%
  --(3.347,5.988)--(3.348,5.988)--(3.349,5.988)--(3.350,5.988)--(3.351,5.988)--(3.352,5.988)%
  --(3.353,5.988)--(3.354,5.988)--(3.355,5.988)--(3.356,5.988)--(3.357,5.988)--(3.358,5.988)%
  --(3.359,5.988)--(3.360,5.988)--(3.361,5.988)--(3.362,5.988)--(3.363,5.988)--(3.364,5.988)%
  --(3.365,5.988)--(3.366,5.988)--(3.367,5.988)--(3.368,5.988)--(3.369,5.988)--(3.370,5.988)%
  --(3.371,5.988)--(3.372,5.988)--(3.373,5.988)--(3.374,5.988)--(3.375,5.988)--(3.376,5.988)%
  --(3.377,5.988)--(3.378,5.988)--(3.379,5.988)--(3.380,5.988)--(3.381,5.988)--(3.382,5.988)%
  --(3.383,5.988)--(3.384,5.988)--(3.385,5.988)--(3.386,5.988)--(3.387,5.988)--(3.388,5.988)%
  --(3.389,5.988)--(3.390,5.988)--(3.391,5.988)--(3.392,5.988)--(3.393,5.988)--(3.394,5.988)%
  --(3.395,5.988)--(3.396,5.988)--(3.397,5.988)--(3.398,5.988)--(3.399,5.988)--(3.400,5.988)%
  --(3.401,5.988)--(3.402,5.988)--(3.403,5.988)--(3.404,5.988)--(3.405,5.988)--(3.406,5.988)%
  --(3.407,5.988)--(3.408,5.988)--(3.409,5.988)--(3.410,5.988)--(3.411,5.988)--(3.412,5.988)%
  --(3.413,5.988)--(3.414,5.988)--(3.415,5.988)--(3.416,5.988)--(3.417,5.988)--(3.418,5.988)%
  --(3.419,5.988)--(3.420,5.988)--(3.421,5.988)--(3.422,5.988)--(3.423,5.988)--(3.424,5.988)%
  --(3.425,5.988)--(3.426,5.988)--(3.427,5.988)--(3.428,5.988)--(3.429,5.988)--(3.430,5.988)%
  --(3.431,5.988)--(3.432,5.988)--(3.433,5.988)--(3.434,5.988)--(3.435,5.988)--(3.436,5.988)%
  --(3.437,5.988)--(3.438,5.988)--(3.439,5.988)--(3.440,5.988)--(3.441,5.988)--(3.442,5.988)%
  --(3.443,5.988)--(3.444,5.988)--(3.445,5.988)--(3.446,5.988)--(3.447,5.988)--(3.448,5.988)%
  --(3.449,5.988)--(3.450,5.988)--(3.451,5.988)--(3.452,5.988)--(3.453,5.988)--(3.454,5.988)%
  --(3.455,5.988)--(3.456,5.988)--(3.457,5.988)--(3.458,5.988)--(3.459,5.988)--(3.460,5.988)%
  --(3.461,5.988)--(3.462,5.988)--(3.463,5.988)--(3.464,5.988)--(3.465,5.988)--(3.466,5.988)%
  --(3.467,5.988)--(3.468,5.988)--(3.469,5.988)--(3.470,5.988)--(3.471,5.988)--(3.472,5.988)%
  --(3.473,5.988)--(3.474,5.988)--(3.475,5.988)--(3.476,5.988)--(3.477,5.988)--(3.478,5.988)%
  --(3.479,5.988)--(3.480,5.988)--(3.481,5.988)--(3.482,5.988)--(3.483,5.988)--(3.484,5.988)%
  --(3.485,5.988)--(3.486,5.988)--(3.487,5.988)--(3.488,5.988)--(3.489,5.988)--(3.490,5.988)%
  --(3.491,5.988)--(3.492,5.988)--(3.493,5.988)--(3.494,5.988)--(3.495,5.988)--(3.496,5.988)%
  --(3.497,5.988)--(3.498,5.988)--(3.499,5.988)--(3.500,5.988)--(3.501,5.988)--(3.502,5.988)%
  --(3.503,5.988)--(3.504,5.988)--(3.505,5.988)--(3.506,5.988)--(3.507,5.988)--(3.508,5.988)%
  --(3.509,5.988)--(3.510,5.988)--(3.511,5.988)--(3.512,5.988)--(3.513,5.988)--(3.514,5.988)%
  --(3.515,5.988)--(3.516,5.988)--(3.517,5.988)--(3.518,5.988)--(3.519,5.988)--(3.520,5.988)%
  --(3.521,5.988)--(3.522,5.988)--(3.523,5.988)--(3.524,5.988)--(3.525,5.988)--(3.526,5.988)%
  --(3.527,5.988)--(3.528,5.988)--(3.529,5.988)--(3.530,5.988)--(3.531,5.988)--(3.532,5.988)%
  --(3.533,5.988)--(3.534,5.988)--(3.535,5.988)--(3.536,5.988)--(3.537,5.988)--(3.538,5.988)%
  --(3.539,5.988)--(3.540,5.988)--(3.541,5.988)--(3.542,5.988)--(3.543,5.988)--(3.544,5.988)%
  --(3.545,5.988)--(3.546,5.988)--(3.547,5.988)--(3.548,5.988)--(3.549,5.988)--(3.550,5.988)%
  --(3.551,5.988)--(3.552,5.988)--(3.553,5.988)--(3.554,5.988)--(3.555,5.988)--(3.556,5.988)%
  --(3.557,5.988)--(3.558,5.988)--(3.559,5.988)--(3.560,5.988)--(3.561,5.988)--(3.562,5.988)%
  --(3.563,5.988)--(3.564,5.988)--(3.565,5.988)--(3.566,5.988)--(3.567,5.988)--(3.568,5.988)%
  --(3.569,5.988)--(3.570,5.988)--(3.571,5.988)--(3.572,5.988)--(3.573,5.988)--(3.574,5.988)%
  --(3.575,5.988)--(3.576,5.988)--(3.577,5.988)--(3.578,5.988)--(3.579,5.988)--(3.580,5.988)%
  --(3.581,5.988)--(3.582,5.988)--(3.583,5.988)--(3.584,5.988)--(3.585,5.988)--(3.586,5.988)%
  --(3.587,5.988)--(3.588,5.988)--(3.589,5.988)--(3.590,5.988)--(3.591,5.988)--(3.592,5.988)%
  --(3.593,5.988)--(3.594,5.988)--(3.595,5.988)--(3.596,5.988)--(3.597,5.988)--(3.598,5.988)%
  --(3.599,5.988)--(3.600,5.988)--(3.601,5.988)--(3.602,5.988)--(3.603,5.988)--(3.604,5.988)%
  --(3.605,5.988)--(3.606,5.988)--(3.607,5.988)--(3.608,5.988)--(3.609,5.988)--(3.610,5.988)%
  --(3.611,5.988)--(3.612,5.988)--(3.613,5.988)--(3.614,5.988)--(3.615,5.988)--(3.616,5.988)%
  --(3.617,5.988)--(3.618,5.988)--(3.619,5.988)--(3.620,5.988)--(3.621,5.988)--(3.622,5.988)%
  --(3.623,5.988)--(3.624,5.988)--(3.625,5.988)--(3.626,5.988)--(3.627,5.988)--(3.628,5.988)%
  --(3.629,5.988)--(3.630,5.988)--(3.631,5.988)--(3.632,5.988)--(3.633,5.988)--(3.634,5.988)%
  --(3.635,5.988)--(3.636,5.988)--(3.637,5.988)--(3.638,5.988)--(3.639,5.988)--(3.640,5.988)%
  --(3.641,5.988)--(3.642,5.988)--(3.643,5.988)--(3.644,5.988)--(3.645,5.988)--(3.646,5.988)%
  --(3.647,5.988)--(3.648,5.988)--(3.649,5.988)--(3.650,5.988)--(3.651,5.988)--(3.652,5.988)%
  --(3.653,5.988)--(3.654,5.988)--(3.655,5.988)--(3.656,5.988)--(3.657,5.988)--(3.658,5.988)%
  --(3.659,5.988)--(3.660,5.988)--(3.661,5.988)--(3.662,5.988)--(3.663,5.988)--(3.664,5.988)%
  --(3.665,5.988)--(3.666,5.988)--(3.667,5.988)--(3.668,5.988)--(3.669,5.988)--(3.670,5.988)%
  --(3.671,5.988)--(3.672,5.988)--(3.673,5.988)--(3.674,5.988)--(3.675,5.988)--(3.676,5.988)%
  --(3.677,5.988)--(3.678,5.988)--(3.679,5.988)--(3.680,5.988)--(3.681,5.988)--(3.682,5.988)%
  --(3.683,5.988)--(3.684,5.988)--(3.685,5.988)--(3.686,5.988)--(3.687,5.988)--(3.688,5.988)%
  --(3.689,5.988)--(3.690,5.988)--(3.691,5.988)--(3.692,5.988)--(3.693,5.988)--(3.694,5.988)%
  --(3.695,5.988)--(3.696,5.988)--(3.697,5.988)--(3.698,5.988)--(3.699,5.988)--(3.700,5.988)%
  --(3.701,5.988)--(3.702,5.988)--(3.703,5.988)--(3.704,5.988)--(3.705,5.988)--(3.706,5.988)%
  --(3.707,5.988)--(3.708,5.988)--(3.709,5.988)--(3.710,5.988)--(3.711,5.988)--(3.712,5.988)%
  --(3.713,5.988)--(3.714,5.988)--(3.715,5.988)--(3.716,5.988)--(3.717,5.988)--(3.718,5.988)%
  --(3.719,5.988)--(3.720,5.988)--(3.721,5.988)--(3.722,5.988)--(3.723,5.988)--(3.724,5.988)%
  --(3.725,5.988)--(3.726,5.988)--(3.727,5.988)--(3.728,5.988)--(3.729,5.988)--(3.730,5.988)%
  --(3.731,5.988)--(3.732,5.988)--(3.733,5.988)--(3.734,5.988)--(3.735,5.988)--(3.736,5.988)%
  --(3.737,5.988)--(3.738,5.988)--(3.739,5.988)--(3.740,5.988)--(3.741,5.988)--(3.742,5.988)%
  --(3.743,5.988)--(3.744,5.988)--(3.745,5.988)--(3.746,5.988)--(3.747,5.988)--(3.748,5.988)%
  --(3.749,5.988)--(3.750,5.988)--(3.751,5.988)--(3.752,5.988)--(3.753,5.988)--(3.754,5.988)%
  --(3.755,5.988)--(3.756,5.988)--(3.757,5.988)--(3.758,5.988)--(3.759,5.988)--(3.760,5.988)%
  --(3.761,5.988)--(3.762,5.988)--(3.763,5.988)--(3.764,5.988)--(3.765,5.988)--(3.766,5.988)%
  --(3.767,5.988)--(3.768,5.988)--(3.769,5.988)--(3.770,5.988)--(3.771,5.988)--(3.772,5.988)%
  --(3.773,5.988)--(3.774,5.988)--(3.775,5.988)--(3.776,5.988)--(3.777,5.988)--(3.778,5.988)%
  --(3.779,5.988)--(3.780,5.988)--(3.781,5.988)--(3.782,5.988)--(3.783,5.988)--(3.784,5.988)%
  --(3.785,5.988)--(3.786,5.988)--(3.787,5.988)--(3.788,5.988)--(3.789,5.988)--(3.790,5.988)%
  --(3.791,5.988)--(3.792,5.988)--(3.793,5.988)--(3.794,5.988)--(3.795,5.988)--(3.796,5.988)%
  --(3.797,5.988)--(3.798,5.988)--(3.799,5.988)--(3.800,5.988)--(3.801,5.988)--(3.802,5.988)%
  --(3.803,5.988)--(3.804,5.988)--(3.805,5.988)--(3.806,5.988)--(3.807,5.988)--(3.808,5.988)%
  --(3.809,5.988)--(3.810,5.988)--(3.811,5.988)--(3.812,5.988)--(3.813,5.988)--(3.814,5.988)%
  --(3.815,5.988)--(3.816,5.988)--(3.817,5.988)--(3.818,5.988)--(3.819,5.988)--(3.820,5.988)%
  --(3.821,5.988)--(3.822,5.988)--(3.823,5.988)--(3.824,5.988)--(3.825,5.988)--(3.826,5.988)%
  --(3.827,5.988)--(3.828,5.988)--(3.829,5.988)--(3.830,5.988)--(3.831,5.988)--(3.832,5.988)%
  --(3.833,5.988)--(3.834,5.988)--(3.835,5.988)--(3.836,5.988)--(3.837,5.988)--(3.838,5.988)%
  --(3.839,5.988)--(3.840,5.988)--(3.841,5.988)--(3.842,5.988)--(3.843,5.988)--(3.844,5.988)%
  --(3.845,5.988)--(3.846,5.988)--(3.847,5.988)--(3.848,5.988)--(3.849,5.988)--(3.850,5.988)%
  --(3.851,5.988)--(3.852,5.988)--(3.853,5.988)--(3.854,5.988)--(3.855,5.988)--(3.856,5.988)%
  --(3.857,5.988)--(3.858,5.988)--(3.859,5.988)--(3.860,5.988)--(3.861,5.988)--(3.862,5.988)%
  --(3.863,5.988)--(3.864,5.988)--(3.865,5.988)--(3.866,5.988)--(3.867,5.988)--(3.868,5.988)%
  --(3.869,5.988)--(3.870,5.988)--(3.871,5.988)--(3.872,5.988)--(3.873,5.988)--(3.874,5.988)%
  --(3.875,5.988)--(3.876,5.988)--(3.877,5.988)--(3.878,5.988)--(3.879,5.988)--(3.880,5.988)%
  --(3.881,5.988)--(3.882,5.988)--(3.883,5.988)--(3.884,5.988)--(3.885,5.988)--(3.886,5.988)%
  --(3.887,5.988)--(3.888,5.988)--(3.889,5.988)--(3.890,5.988)--(3.891,5.988)--(3.892,5.988)%
  --(3.893,5.988)--(3.894,5.988)--(3.895,5.988)--(3.896,5.988)--(3.897,5.988)--(3.898,5.988)%
  --(3.899,5.988)--(3.900,5.988)--(3.901,5.988)--(3.902,5.988)--(3.903,5.988)--(3.904,5.988)%
  --(3.905,5.988)--(3.906,5.988)--(3.907,5.988)--(3.908,5.988)--(3.909,5.988)--(3.910,5.988)%
  --(3.911,5.988)--(3.912,5.988)--(3.913,5.988)--(3.914,5.988)--(3.915,5.988)--(3.916,5.988)%
  --(3.917,5.988)--(3.918,5.988)--(3.919,5.988)--(3.920,5.988)--(3.921,5.988)--(3.922,5.988)%
  --(3.923,5.988)--(3.924,5.988)--(3.925,5.988)--(3.926,5.988)--(3.927,5.988)--(3.928,5.988)%
  --(3.929,5.988)--(3.930,5.988)--(3.931,5.988)--(3.932,5.988)--(3.933,5.988)--(3.934,5.988)%
  --(3.935,5.988)--(3.936,5.988)--(3.937,5.988)--(3.938,5.988)--(3.939,5.988)--(3.940,5.988)%
  --(3.941,5.988)--(3.942,5.988)--(3.943,5.988)--(3.944,5.988)--(3.945,5.988)--(3.946,5.988)%
  --(3.947,5.988)--(3.948,5.988)--(3.949,5.988)--(3.950,5.988)--(3.951,5.988)--(3.952,5.988)%
  --(3.953,5.988)--(3.954,5.988)--(3.955,5.988)--(3.956,5.988)--(3.957,5.988)--(3.958,5.988)%
  --(3.959,5.988)--(3.960,5.988)--(3.961,5.988)--(3.962,5.988)--(3.963,5.988)--(3.964,5.988)%
  --(3.965,5.988)--(3.966,5.988)--(3.967,5.988)--(3.968,5.988)--(3.969,5.988)--(3.970,5.988)%
  --(3.971,5.988)--(3.972,5.988)--(3.973,5.988)--(3.974,5.988)--(3.975,5.988)--(3.976,5.988)%
  --(3.977,5.988)--(3.978,5.988)--(3.979,5.988)--(3.980,5.988)--(3.981,5.988)--(3.982,5.988)%
  --(3.983,5.988)--(3.984,5.988)--(3.985,5.988)--(3.986,5.988)--(3.987,5.988)--(3.988,5.988)%
  --(3.989,5.988)--(3.990,5.988)--(3.991,5.988)--(3.992,5.988)--(3.993,5.988)--(3.994,5.988)%
  --(3.995,5.988)--(3.996,5.988)--(3.997,5.988)--(3.998,5.988)--(3.999,5.988)--(4.000,5.988)%
  --(4.001,5.988)--(4.002,5.988)--(4.003,5.988)--(4.004,5.988)--(4.005,5.988)--(4.006,5.988)%
  --(4.007,5.988)--(4.008,5.988)--(4.009,5.988)--(4.010,5.988)--(4.011,5.988)--(4.012,5.988)%
  --(4.013,5.988)--(4.014,5.988)--(4.015,5.988)--(4.016,5.988)--(4.017,5.988)--(4.018,5.988)%
  --(4.019,5.988)--(4.020,5.988)--(4.021,5.988)--(4.022,5.988)--(4.023,5.988)--(4.024,5.988)%
  --(4.025,5.988)--(4.026,5.988)--(4.027,5.988)--(4.028,5.988)--(4.029,5.988)--(4.030,5.988)%
  --(4.031,5.988)--(4.032,5.988)--(4.033,5.988)--(4.034,5.988)--(4.035,5.988)--(4.036,5.988)%
  --(4.037,5.988)--(4.038,5.988)--(4.039,5.988)--(4.040,5.988)--(4.041,5.988)--(4.042,5.988)%
  --(4.043,5.988)--(4.044,5.988)--(4.045,5.988)--(4.046,5.988)--(4.047,5.988)--(4.048,5.988)%
  --(4.049,5.988)--(4.050,5.988)--(4.051,5.988)--(4.052,5.988)--(4.053,5.988)--(4.054,5.988)%
  --(4.055,5.988)--(4.056,5.988)--(4.057,5.988)--(4.058,5.988)--(4.059,5.988)--(4.060,5.988)%
  --(4.061,5.988)--(4.062,5.988)--(4.063,5.988)--(4.064,5.988)--(4.065,5.988)--(4.066,5.988)%
  --(4.067,5.988)--(4.068,5.988)--(4.069,5.988)--(4.070,5.988)--(4.071,5.988)--(4.072,5.988)%
  --(4.073,5.988)--(4.074,5.988)--(4.075,5.988)--(4.076,5.988)--(4.077,5.988)--(4.078,5.988)%
  --(4.079,5.988)--(4.080,5.988)--(4.081,5.988)--(4.082,5.988)--(4.083,5.988)--(4.084,5.988)%
  --(4.085,5.988)--(4.086,5.988)--(4.087,5.988)--(4.088,5.988)--(4.089,5.988)--(4.090,5.988)%
  --(4.091,5.988)--(4.092,5.988)--(4.093,5.988)--(4.094,5.988)--(4.095,5.988)--(4.096,5.988)%
  --(4.097,5.988)--(4.098,5.988)--(4.099,5.988)--(4.100,5.988)--(4.101,5.988)--(4.102,5.988)%
  --(4.103,5.988)--(4.104,5.988)--(4.105,5.988)--(4.106,5.988)--(4.107,5.988)--(4.108,5.988)%
  --(4.109,5.988)--(4.110,5.988)--(4.111,5.988)--(4.112,5.988)--(4.113,5.988)--(4.114,5.988)%
  --(4.115,5.988)--(4.116,5.988)--(4.117,5.988)--(4.118,5.988)--(4.119,5.988)--(4.120,5.988)%
  --(4.121,5.988)--(4.122,5.988)--(4.123,5.988)--(4.124,5.988)--(4.125,5.988)--(4.126,5.988)%
  --(4.127,5.988)--(4.128,5.988)--(4.129,5.988)--(4.130,5.988)--(4.131,5.988)--(4.132,5.988)%
  --(4.133,5.988)--(4.134,5.988)--(4.135,5.988)--(4.136,5.988)--(4.137,5.988)--(4.138,5.988)%
  --(4.139,5.988)--(4.140,5.988)--(4.141,5.988)--(4.142,5.988)--(4.143,5.988)--(4.144,5.988)%
  --(4.145,5.988)--(4.146,5.988)--(4.147,5.988)--(4.148,5.988)--(4.149,5.988)--(4.150,5.988)%
  --(4.151,5.988)--(4.152,5.988)--(4.153,5.988)--(4.154,5.988)--(4.155,5.988)--(4.156,5.988)%
  --(4.157,5.988)--(4.158,5.988)--(4.159,5.988)--(4.160,5.988)--(4.161,5.988)--(4.162,5.988)%
  --(4.163,5.988)--(4.164,5.988)--(4.165,5.988)--(4.166,5.988)--(4.167,5.988)--(4.168,5.988)%
  --(4.169,5.988)--(4.170,5.988)--(4.171,5.988)--(4.172,5.988)--(4.173,5.988)--(4.174,5.988)%
  --(4.175,5.988)--(4.176,5.988)--(4.177,5.988)--(4.178,5.988)--(4.179,5.988)--(4.180,5.988)%
  --(4.181,5.988)--(4.182,5.988)--(4.183,5.988)--(4.184,5.988)--(4.185,5.988)--(4.186,5.988)%
  --(4.187,5.988)--(4.188,5.988)--(4.189,5.988)--(4.190,5.988)--(4.191,5.988)--(4.192,5.988)%
  --(4.193,5.988)--(4.194,5.988)--(4.195,5.988)--(4.196,5.988)--(4.197,5.988)--(4.198,5.988)%
  --(4.199,5.988)--(4.200,5.988)--(4.201,5.988)--(4.202,5.988)--(4.203,5.988)--(4.204,5.988)%
  --(4.205,5.988)--(4.206,5.988)--(4.207,5.988)--(4.208,5.988)--(4.209,5.988)--(4.210,5.988)%
  --(4.211,5.988)--(4.212,5.988)--(4.213,5.988)--(4.214,5.988)--(4.215,5.988)--(4.216,5.988)%
  --(4.217,5.988)--(4.218,5.988)--(4.219,5.988)--(4.220,5.988)--(4.221,5.988)--(4.222,5.988)%
  --(4.223,5.988)--(4.224,5.988)--(4.225,5.988)--(4.226,5.988)--(4.227,5.988)--(4.228,5.988)%
  --(4.229,5.988)--(4.230,5.988)--(4.231,5.988)--(4.232,5.988)--(4.233,5.988)--(4.234,5.988)%
  --(4.235,5.988)--(4.236,5.988)--(4.237,5.988)--(4.238,5.988)--(4.239,5.988)--(4.240,5.988)%
  --(4.241,5.988)--(4.242,5.988)--(4.243,5.988)--(4.244,5.988)--(4.245,5.988)--(4.246,5.988)%
  --(4.247,5.988)--(4.248,5.988)--(4.249,5.988)--(4.250,5.988)--(4.251,5.988)--(4.252,5.988)%
  --(4.253,5.988)--(4.254,5.988)--(4.255,5.988)--(4.256,5.988)--(4.257,5.988)--(4.258,5.988)%
  --(4.259,5.988)--(4.260,5.988)--(4.261,5.988)--(4.262,5.988)--(4.263,5.988)--(4.264,5.988)%
  --(4.265,5.988)--(4.266,5.988)--(4.267,5.988)--(4.268,5.988)--(4.269,5.988)--(4.270,5.988)%
  --(4.271,5.988)--(4.272,5.988)--(4.273,5.988)--(4.274,5.988)--(4.275,5.988)--(4.276,5.988)%
  --(4.277,5.988)--(4.278,5.988)--(4.279,5.988)--(4.280,5.988)--(4.281,5.988)--(4.282,5.988)%
  --(4.283,5.988)--(4.284,5.988)--(4.285,5.988)--(4.286,5.988)--(4.287,5.988)--(4.288,5.988)%
  --(4.289,5.988)--(4.290,5.988)--(4.291,5.988)--(4.292,5.988)--(4.293,5.988)--(4.294,5.988)%
  --(4.295,5.988)--(4.296,5.988)--(4.297,5.988)--(4.298,5.988)--(4.299,5.988)--(4.300,5.988)%
  --(4.301,5.988)--(4.302,5.988)--(4.303,5.988)--(4.304,5.988)--(4.305,5.988)--(4.306,5.988)%
  --(4.307,5.988)--(4.308,5.988)--(4.309,5.988)--(4.310,5.988)--(4.311,5.988)--(4.312,5.988)%
  --(4.313,5.988)--(4.314,5.988)--(4.315,5.988)--(4.316,5.988)--(4.317,5.988)--(4.318,5.988)%
  --(4.319,5.988)--(4.320,5.988)--(4.321,5.988)--(4.322,5.988)--(4.323,5.988)--(4.324,5.988)%
  --(4.325,5.988)--(4.326,5.988)--(4.327,5.988)--(4.328,5.988)--(4.329,5.988)--(4.330,5.988)%
  --(4.331,5.988)--(4.332,5.988)--(4.333,5.988)--(4.334,5.988)--(4.335,5.988)--(4.336,5.988)%
  --(4.337,5.988)--(4.338,5.988)--(4.339,5.988)--(4.340,5.988)--(4.341,5.988)--(4.342,5.988)%
  --(4.343,5.988)--(4.344,5.988)--(4.345,5.988)--(4.346,5.988)--(4.347,5.988)--(4.348,5.988)%
  --(4.349,5.988)--(4.350,5.988)--(4.351,5.988)--(4.352,5.988)--(4.353,5.988)--(4.354,5.988)%
  --(4.355,5.988)--(4.356,5.988)--(4.357,5.988)--(4.358,5.988)--(4.359,5.988)--(4.360,5.988)%
  --(4.361,5.988)--(4.362,5.988)--(4.363,5.988)--(4.364,5.988)--(4.365,5.988)--(4.366,5.988)%
  --(4.367,5.988)--(4.368,5.988)--(4.369,5.988)--(4.370,5.988)--(4.371,5.988)--(4.372,5.988)%
  --(4.373,5.988)--(4.374,5.988)--(4.375,5.988)--(4.376,5.988)--(4.377,5.988)--(4.378,5.988)%
  --(4.379,5.988)--(4.380,5.988)--(4.381,5.988)--(4.382,5.988)--(4.383,5.988)--(4.384,5.988)%
  --(4.385,5.988)--(4.386,5.988)--(4.387,5.988)--(4.388,5.988)--(4.389,5.988)--(4.390,5.988)%
  --(4.391,5.988)--(4.392,5.988)--(4.393,5.988)--(4.394,5.988)--(4.395,5.988)--(4.396,5.988)%
  --(4.397,5.988)--(4.398,5.988)--(4.399,5.988)--(4.400,5.988)--(4.401,5.988)--(4.402,5.988)%
  --(4.403,5.988)--(4.404,5.988)--(4.405,5.988)--(4.406,5.988)--(4.407,5.988)--(4.408,5.988)%
  --(4.409,5.988)--(4.410,5.988)--(4.411,5.988)--(4.412,5.988)--(4.413,5.988)--(4.414,5.988)%
  --(4.415,5.988)--(4.416,5.988)--(4.417,5.988)--(4.418,5.988)--(4.419,5.988)--(4.420,5.988)%
  --(4.421,5.988)--(4.422,5.988)--(4.423,5.988)--(4.424,5.988)--(4.425,5.988)--(4.426,5.988)%
  --(4.427,5.988)--(4.428,5.988)--(4.429,5.988)--(4.430,5.988)--(4.431,5.988)--(4.432,5.988)%
  --(4.433,5.988)--(4.434,5.988)--(4.435,5.988)--(4.436,5.988)--(4.437,5.988)--(4.438,5.988)%
  --(4.439,5.988)--(4.440,5.988)--(4.441,5.988)--(4.442,5.988)--(4.443,5.988)--(4.444,5.988)%
  --(4.445,5.988)--(4.446,5.988)--(4.447,5.988)--(4.448,5.988)--(4.449,5.988)--(4.450,5.988)%
  --(4.451,5.988)--(4.452,5.988)--(4.453,5.988)--(4.454,5.988)--(4.455,5.988)--(4.456,5.988)%
  --(4.457,5.988)--(4.458,5.988)--(4.459,5.988)--(4.460,5.988)--(4.461,5.988)--(4.462,5.988)%
  --(4.463,5.988)--(4.464,5.988)--(4.465,5.988)--(4.466,5.988)--(4.467,5.988)--(4.468,5.988)%
  --(4.469,5.988)--(4.470,5.988)--(4.471,5.988)--(4.472,5.988)--(4.473,5.988)--(4.474,5.988)%
  --(4.475,5.988)--(4.476,5.988)--(4.477,5.988)--(4.478,5.988)--(4.479,5.988)--(4.480,5.988)%
  --(4.481,5.988)--(4.482,5.988)--(4.483,5.988)--(4.484,5.988)--(4.485,5.988)--(4.486,5.988)%
  --(4.487,5.988)--(4.488,5.988)--(4.489,5.988)--(4.490,5.988)--(4.491,5.988)--(4.492,5.988)%
  --(4.493,5.988)--(4.494,5.988)--(4.495,5.988)--(4.496,5.988)--(4.497,5.988)--(4.498,5.988)%
  --(4.499,5.988)--(4.500,5.988)--(4.501,5.988)--(4.502,5.988)--(4.503,5.988)--(4.504,5.988)%
  --(4.505,5.988)--(4.506,5.988)--(4.507,5.988)--(4.508,5.988)--(4.509,5.988)--(4.510,5.988)%
  --(4.511,5.988)--(4.512,5.988)--(4.513,5.988)--(4.514,5.988)--(4.515,5.988)--(4.516,5.988)%
  --(4.517,5.988)--(4.518,5.988)--(4.519,5.988)--(4.520,5.988)--(4.521,5.988)--(4.522,5.988)%
  --(4.523,5.988)--(4.524,5.988)--(4.525,5.988)--(4.526,5.988)--(4.527,5.988)--(4.528,5.988)%
  --(4.529,5.988)--(4.530,5.988)--(4.531,5.988)--(4.532,5.988)--(4.533,5.988)--(4.534,5.988)%
  --(4.535,5.988)--(4.536,5.988)--(4.537,5.988)--(4.538,5.988)--(4.539,5.988)--(4.540,5.988)%
  --(4.541,5.988)--(4.542,5.988)--(4.543,5.988)--(4.544,5.988)--(4.545,5.988)--(4.546,5.988)%
  --(4.547,5.988)--(4.548,5.988)--(4.549,5.988)--(4.550,5.988)--(4.551,5.988)--(4.552,5.988)%
  --(4.553,5.988)--(4.554,5.988)--(4.555,5.988)--(4.556,5.988)--(4.557,5.988)--(4.558,5.988)%
  --(4.559,5.988)--(4.560,5.988)--(4.561,5.988)--(4.562,5.988)--(4.563,5.988)--(4.564,5.988)%
  --(4.565,5.988)--(4.566,5.988)--(4.567,5.988)--(4.568,5.988)--(4.569,5.988)--(4.570,5.988)%
  --(4.571,5.988)--(4.572,5.988)--(4.573,5.988)--(4.574,5.988)--(4.575,5.988)--(4.576,5.988)%
  --(4.577,5.988)--(4.578,5.988)--(4.579,5.988)--(4.580,5.988)--(4.581,5.988)--(4.582,5.988)%
  --(4.583,5.988)--(4.584,5.988)--(4.585,5.988)--(4.586,5.988)--(4.587,5.988)--(4.588,5.988)%
  --(4.589,5.988)--(4.590,5.988)--(4.591,5.988)--(4.592,5.988)--(4.593,5.988)--(4.594,5.988)%
  --(4.595,5.988)--(4.596,5.988)--(4.597,5.988)--(4.598,5.988)--(4.599,5.988)--(4.600,5.988)%
  --(4.601,5.988)--(4.602,5.988)--(4.603,5.988)--(4.604,5.988)--(4.605,5.988)--(4.606,5.988)%
  --(4.607,5.988)--(4.608,5.988)--(4.609,5.988)--(4.610,5.988)--(4.611,5.988)--(4.612,5.988)%
  --(4.613,5.988)--(4.614,5.988)--(4.615,5.988)--(4.616,5.988)--(4.617,5.988)--(4.618,5.988)%
  --(4.619,5.988)--(4.620,5.988)--(4.621,5.988)--(4.622,5.988)--(4.623,5.988)--(4.624,5.988)%
  --(4.625,5.988)--(4.626,5.988)--(4.627,5.988)--(4.628,5.988)--(4.629,5.988)--(4.630,5.988)%
  --(4.631,5.988)--(4.632,5.988)--(4.633,5.988)--(4.634,5.988)--(4.635,5.988)--(4.636,5.988)%
  --(4.637,5.988)--(4.638,5.988)--(4.639,5.988)--(4.640,5.988)--(4.641,5.988)--(4.642,5.988)%
  --(4.643,5.988)--(4.644,5.988)--(4.645,5.988)--(4.646,5.988)--(4.647,5.988)--(4.648,5.988)%
  --(4.649,5.988)--(4.650,5.988)--(4.651,5.988)--(4.652,5.988)--(4.653,5.988)--(4.654,5.988)%
  --(4.655,5.988)--(4.656,5.988)--(4.657,5.988)--(4.658,5.988)--(4.659,5.988)--(4.660,5.988)%
  --(4.661,5.988)--(4.662,5.988)--(4.663,5.988)--(4.664,5.988)--(4.665,5.988)--(4.666,5.988)%
  --(4.667,5.988)--(4.668,5.988)--(4.669,5.988)--(4.670,5.988)--(4.671,5.988)--(4.672,5.988)%
  --(4.673,5.988)--(4.674,5.988)--(4.675,5.988)--(4.676,5.988)--(4.677,5.988)--(4.678,5.988)%
  --(4.679,5.988)--(4.680,5.988)--(4.681,5.988)--(4.682,5.988)--(4.683,5.988)--(4.684,5.988)%
  --(4.685,5.988)--(4.686,5.988)--(4.687,5.988)--(4.688,5.988)--(4.689,5.988)--(4.690,5.988)%
  --(4.691,5.988)--(4.692,5.988)--(4.693,5.988)--(4.694,5.988)--(4.695,5.988)--(4.696,5.988)%
  --(4.697,5.988)--(4.698,5.988)--(4.699,5.988)--(4.700,5.988)--(4.701,5.988)--(4.702,5.988)%
  --(4.703,5.988)--(4.704,5.988)--(4.705,5.988)--(4.706,5.988)--(4.707,5.988)--(4.708,5.988)%
  --(4.709,5.988)--(4.710,5.988)--(4.711,5.988)--(4.712,5.988)--(4.713,5.988)--(4.714,5.988)%
  --(4.715,5.988)--(4.716,5.988)--(4.717,5.988)--(4.718,5.988)--(4.719,5.988)--(4.720,5.988)%
  --(4.721,5.988)--(4.722,5.988)--(4.723,5.988)--(4.724,5.988)--(4.725,5.988)--(4.726,5.988)%
  --(4.727,5.988)--(4.728,5.988)--(4.729,5.988)--(4.730,5.988)--(4.731,5.988)--(4.732,5.988)%
  --(4.733,5.988)--(4.734,5.988)--(4.735,5.988)--(4.736,5.988)--(4.737,5.988)--(4.738,5.988)%
  --(4.739,5.988)--(4.740,5.988)--(4.741,5.988)--(4.742,5.988)--(4.743,5.988)--(4.744,5.988)%
  --(4.745,5.988)--(4.746,5.988)--(4.747,5.988)--(4.748,5.988)--(4.749,5.988)--(4.750,5.988)%
  --(4.751,5.988)--(4.752,5.988)--(4.753,5.988)--(4.754,5.988)--(4.755,5.988)--(4.756,5.988)%
  --(4.757,5.988)--(4.758,5.988)--(4.759,5.988)--(4.760,5.988)--(4.761,5.988)--(4.762,5.988)%
  --(4.763,5.988)--(4.764,5.988)--(4.765,5.988)--(4.766,5.988)--(4.767,5.988)--(4.768,5.988)%
  --(4.769,5.988)--(4.770,5.988)--(4.771,5.988)--(4.772,5.988)--(4.773,5.988)--(4.774,5.988)%
  --(4.775,5.988)--(4.776,5.988)--(4.777,5.988)--(4.778,5.988)--(4.779,5.988)--(4.780,5.988)%
  --(4.781,5.988)--(4.782,5.988)--(4.783,5.988)--(4.784,5.988)--(4.785,5.988)--(4.786,5.988)%
  --(4.787,5.988)--(4.788,5.988)--(4.789,5.988)--(4.790,5.988)--(4.791,5.988)--(4.792,5.988)%
  --(4.793,5.988)--(4.794,5.988)--(4.795,5.988)--(4.796,5.988)--(4.797,5.988)--(4.798,5.988)%
  --(4.799,5.988)--(4.800,5.988)--(4.801,5.988)--(4.802,5.988)--(4.803,5.988)--(4.804,5.988)%
  --(4.805,5.988)--(4.806,5.988)--(4.807,5.988)--(4.808,5.988)--(4.809,5.988)--(4.810,5.988)%
  --(4.811,5.988)--(4.812,5.988)--(4.813,5.988)--(4.814,5.988)--(4.815,5.988)--(4.816,5.988)%
  --(4.817,5.988)--(4.818,5.988)--(4.819,5.988)--(4.820,5.988)--(4.821,5.988)--(4.822,5.988)%
  --(4.823,5.988)--(4.824,5.988)--(4.825,5.988)--(4.826,5.988)--(4.827,5.988)--(4.828,5.988)%
  --(4.829,5.988)--(4.830,5.988)--(4.831,5.988)--(4.832,5.988)--(4.833,5.988)--(4.834,5.988)%
  --(4.835,5.988)--(4.836,5.988)--(4.837,5.988)--(4.838,5.988)--(4.839,5.988)--(4.840,5.988)%
  --(4.841,5.988)--(4.842,5.988)--(4.843,5.988)--(4.844,5.988)--(4.845,5.988)--(4.846,5.988)%
  --(4.847,5.988)--(4.848,5.988)--(4.849,5.988)--(4.850,5.988)--(4.851,5.988)--(4.852,5.988)%
  --(4.853,5.988)--(4.854,5.988)--(4.855,5.988)--(4.856,5.988)--(4.857,5.988)--(4.858,5.988)%
  --(4.859,5.988)--(4.860,5.988)--(4.861,5.988)--(4.862,5.988)--(4.863,5.988)--(4.864,5.988)%
  --(4.865,5.988)--(4.866,5.988)--(4.867,5.988)--(4.868,5.988)--(4.869,5.988)--(4.870,5.988)%
  --(4.871,5.988)--(4.872,5.988)--(4.873,5.988)--(4.874,5.988)--(4.875,5.988)--(4.876,5.988)%
  --(4.877,5.988)--(4.878,5.988)--(4.879,5.988)--(4.880,5.988)--(4.881,5.988)--(4.882,5.988)%
  --(4.883,5.988)--(4.884,5.988)--(4.885,5.988)--(4.886,5.988)--(4.887,5.988)--(4.888,5.988)%
  --(4.889,5.988)--(4.890,5.988)--(4.891,5.988)--(4.892,5.988)--(4.893,5.988)--(4.894,5.988)%
  --(4.895,5.988)--(4.896,5.988)--(4.897,5.988)--(4.898,5.988)--(4.899,5.988)--(4.900,5.988)%
  --(4.901,5.988)--(4.902,5.988)--(4.903,5.988)--(4.904,5.988)--(4.905,5.988)--(4.906,5.988)%
  --(4.907,5.988)--(4.908,5.988)--(4.909,5.988)--(4.910,5.988)--(4.911,5.988)--(4.912,5.988)%
  --(4.913,5.988)--(4.914,5.988)--(4.915,5.988)--(4.916,5.988)--(4.917,5.988)--(4.918,5.988)%
  --(4.919,5.988)--(4.920,5.988)--(4.921,5.988)--(4.922,5.988)--(4.923,5.988)--(4.924,5.988)%
  --(4.925,5.988)--(4.926,5.988)--(4.927,5.988)--(4.928,5.988)--(4.929,5.988)--(4.930,5.988)%
  --(4.931,5.988)--(4.932,5.988)--(4.933,5.988)--(4.934,5.988)--(4.935,5.988)--(4.936,5.988)%
  --(4.937,5.988)--(4.938,5.988)--(4.939,5.988)--(4.940,5.988)--(4.941,5.988)--(4.942,5.988)%
  --(4.943,5.988)--(4.944,5.988)--(4.945,5.988)--(4.946,5.988)--(4.947,5.988)--(4.948,5.988)%
  --(4.949,5.988)--(4.950,5.988)--(4.951,5.988)--(4.952,5.988)--(4.953,5.988)--(4.954,5.988)%
  --(4.955,5.988)--(4.956,5.988)--(4.957,5.988)--(4.958,5.988)--(4.959,5.988)--(4.960,5.988)%
  --(4.961,5.988)--(4.962,5.988)--(4.963,5.988)--(4.964,5.988)--(4.965,5.988)--(4.966,5.988)%
  --(4.967,5.988)--(4.968,5.988)--(4.969,5.988)--(4.970,5.988)--(4.971,5.988)--(4.972,5.988)%
  --(4.973,5.988)--(4.974,5.988)--(4.975,5.988)--(4.976,5.988)--(4.977,5.988)--(4.978,5.988)%
  --(4.979,5.988)--(4.980,5.988)--(4.981,5.988)--(4.982,5.988)--(4.983,5.988)--(4.984,5.988)%
  --(4.985,5.988)--(4.986,5.988)--(4.987,5.988)--(4.988,5.988)--(4.989,5.988)--(4.990,5.988)%
  --(4.991,5.988)--(4.992,5.988)--(4.993,5.988)--(4.994,5.988)--(4.995,5.988)--(4.996,5.988)%
  --(4.997,5.988)--(4.998,5.988)--(4.999,5.988)--(5.000,5.988)--(5.001,5.988)--(5.002,5.988)%
  --(5.003,5.988)--(5.004,5.988)--(5.005,5.988)--(5.006,5.988)--(5.007,5.988)--(5.008,5.988)%
  --(5.009,5.988)--(5.010,5.988)--(5.011,5.988)--(5.012,5.988)--(5.013,5.988)--(5.014,5.988)%
  --(5.015,5.988)--(5.016,5.988)--(5.017,5.988)--(5.018,5.988)--(5.019,5.988)--(5.020,5.988)%
  --(5.021,5.988)--(5.022,5.988)--(5.023,5.988)--(5.024,5.988)--(5.025,5.988)--(5.026,5.988)%
  --(5.027,5.988)--(5.028,5.988)--(5.029,5.988)--(5.030,5.988)--(5.031,5.988)--(5.032,5.988)%
  --(5.033,5.988)--(5.034,5.988)--(5.035,5.988)--(5.036,5.988)--(5.037,5.988)--(5.038,5.988)%
  --(5.039,5.988)--(5.040,5.988)--(5.041,5.988)--(5.042,5.988)--(5.043,5.988)--(5.044,5.988)%
  --(5.045,5.988)--(5.046,5.988)--(5.047,5.988)--(5.048,5.988)--(5.049,5.988)--(5.050,5.988)%
  --(5.051,5.988)--(5.052,5.988)--(5.053,5.988)--(5.054,5.988)--(5.055,5.988)--(5.056,5.988)%
  --(5.057,5.988)--(5.058,5.988)--(5.059,5.988)--(5.060,5.988)--(5.061,5.988)--(5.062,5.988)%
  --(5.063,5.988)--(5.064,5.988)--(5.065,5.988)--(5.066,5.988)--(5.067,5.988)--(5.068,5.988)%
  --(5.069,5.988)--(5.070,5.988)--(5.071,5.988)--(5.072,5.988)--(5.073,5.988)--(5.074,5.988)%
  --(5.075,5.988)--(5.076,5.988)--(5.077,5.988)--(5.078,5.988)--(5.079,5.988)--(5.080,5.988)%
  --(5.081,5.988)--(5.082,5.988)--(5.083,5.988)--(5.084,5.988)--(5.085,5.988)--(5.086,5.988)%
  --(5.087,5.988)--(5.088,5.988)--(5.089,5.988)--(5.090,5.988)--(5.091,5.988)--(5.092,5.988)%
  --(5.093,5.988)--(5.094,5.988)--(5.095,5.988)--(5.096,5.988)--(5.097,5.988)--(5.098,5.988)%
  --(5.099,5.988)--(5.100,5.988)--(5.101,5.988)--(5.102,5.988)--(5.103,5.988)--(5.104,5.988)%
  --(5.105,5.988)--(5.106,5.988)--(5.107,5.988)--(5.108,5.988)--(5.109,5.988)--(5.110,5.988)%
  --(5.111,5.988)--(5.112,5.988)--(5.113,5.988)--(5.114,5.988)--(5.115,5.988)--(5.116,5.988)%
  --(5.117,5.988)--(5.118,5.988)--(5.119,5.988)--(5.120,5.988)--(5.121,5.988)--(5.122,5.988)%
  --(5.123,5.988)--(5.124,5.988)--(5.125,5.988)--(5.126,5.988)--(5.127,5.988)--(5.128,5.988)%
  --(5.129,5.988)--(5.130,5.988)--(5.131,5.988)--(5.132,5.988)--(5.133,5.988)--(5.134,5.988)%
  --(5.135,5.988)--(5.136,5.988)--(5.137,5.988)--(5.138,5.988)--(5.139,5.988)--(5.140,5.988)%
  --(5.141,5.988)--(5.142,5.988)--(5.143,5.988)--(5.144,5.988)--(5.145,5.988)--(5.146,5.988)%
  --(5.147,5.988)--(5.148,5.988)--(5.149,5.988)--(5.150,5.988)--(5.151,5.988)--(5.152,5.988)%
  --(5.153,5.988)--(5.154,5.988)--(5.155,5.988)--(5.156,5.988)--(5.157,5.988)--(5.158,5.988)%
  --(5.159,5.988)--(5.160,5.988)--(5.161,5.988)--(5.162,5.988)--(5.163,5.988)--(5.164,5.988)%
  --(5.165,5.988)--(5.166,5.988)--(5.167,5.988)--(5.168,5.988)--(5.169,5.988)--(5.170,5.988)%
  --(5.171,5.988)--(5.172,5.988)--(5.173,5.988)--(5.174,5.988)--(5.175,5.988)--(5.176,5.988)%
  --(5.177,5.988)--(5.178,5.988)--(5.179,5.988)--(5.180,5.988)--(5.181,5.988)--(5.182,5.988)%
  --(5.183,5.988)--(5.184,5.988)--(5.185,5.988)--(5.186,5.988)--(5.187,5.988)--(5.188,5.988)%
  --(5.189,5.988)--(5.190,5.988)--(5.191,5.988)--(5.192,5.988)--(5.193,5.988)--(5.194,5.988)%
  --(5.195,5.988)--(5.196,5.988)--(5.197,5.988)--(5.198,5.988)--(5.199,5.988)--(5.200,5.988)%
  --(5.201,5.988)--(5.202,5.988)--(5.203,5.988)--(5.204,5.988)--(5.205,5.988)--(5.206,5.988)%
  --(5.207,5.988)--(5.208,5.988)--(5.209,5.988)--(5.210,5.988)--(5.211,5.988)--(5.212,5.988)%
  --(5.213,5.988)--(5.214,5.988)--(5.215,5.988)--(5.216,5.988)--(5.217,5.988)--(5.218,5.988)%
  --(5.219,5.988)--(5.220,5.988)--(5.221,5.988)--(5.222,5.988)--(5.223,5.988)--(5.224,5.988)%
  --(5.225,5.988)--(5.226,5.988)--(5.227,5.988)--(5.228,5.988)--(5.229,5.988)--(5.230,5.988)%
  --(5.231,5.988)--(5.232,5.988)--(5.233,5.988)--(5.234,5.988)--(5.235,5.988)--(5.236,5.988)%
  --(5.237,5.988)--(5.238,5.988)--(5.239,5.988)--(5.240,5.988)--(5.241,5.988)--(5.242,5.988)%
  --(5.243,5.988)--(5.244,5.988)--(5.245,5.988)--(5.246,5.988)--(5.247,5.988)--(5.248,5.988)%
  --(5.249,5.988)--(5.250,5.988)--(5.251,5.988)--(5.252,5.988)--(5.253,5.988)--(5.254,5.988)%
  --(5.255,5.988)--(5.256,5.988)--(5.257,5.988)--(5.258,5.988)--(5.259,5.988)--(5.260,5.988)%
  --(5.261,5.988)--(5.262,5.988)--(5.263,5.988)--(5.264,5.988)--(5.265,5.988)--(5.266,5.988)%
  --(5.267,5.988)--(5.268,5.988)--(5.269,5.988)--(5.270,5.988)--(5.271,5.988)--(5.272,5.988)%
  --(5.273,5.988)--(5.274,5.988)--(5.275,5.988)--(5.276,5.988)--(5.277,5.988)--(5.278,5.988)%
  --(5.279,5.988)--(5.280,5.988)--(5.281,5.988)--(5.282,5.988)--(5.283,5.988)--(5.284,5.988)%
  --(5.285,5.988)--(5.286,5.988)--(5.287,5.988)--(5.288,5.988)--(5.289,5.988)--(5.290,5.988)%
  --(5.291,5.988)--(5.292,5.988)--(5.293,5.988)--(5.294,5.988)--(5.295,5.988)--(5.296,5.988)%
  --(5.297,5.988)--(5.298,5.988)--(5.299,5.988)--(5.300,5.988)--(5.301,5.988)--(5.302,5.988)%
  --(5.303,5.988)--(5.304,5.988)--(5.305,5.988)--(5.306,5.988)--(5.307,5.988)--(5.308,5.988)%
  --(5.309,5.988)--(5.310,5.988)--(5.311,5.988)--(5.312,5.988)--(5.313,5.988)--(5.314,5.988)%
  --(5.315,5.988)--(5.316,5.988)--(5.317,5.988)--(5.318,5.988)--(5.319,5.988)--(5.320,5.988)%
  --(5.321,5.988)--(5.322,5.988)--(5.323,5.988)--(5.324,5.988)--(5.325,5.988)--(5.326,5.988)%
  --(5.327,5.988)--(5.328,5.988)--(5.329,5.988)--(5.330,5.988)--(5.331,5.988)--(5.332,5.988)%
  --(5.333,5.988)--(5.334,5.988)--(5.335,5.988)--(5.336,5.988)--(5.337,5.988)--(5.338,5.988)%
  --(5.339,5.988)--(5.340,5.988)--(5.341,5.988)--(5.342,5.988)--(5.343,5.988)--(5.344,5.988)%
  --(5.345,5.988)--(5.346,5.988)--(5.347,5.988)--(5.348,5.988)--(5.349,5.988)--(5.350,5.988)%
  --(5.351,5.988)--(5.352,5.988)--(5.353,5.988)--(5.354,5.988)--(5.355,5.988)--(5.356,5.988)%
  --(5.357,5.988)--(5.358,5.988)--(5.359,5.988)--(5.360,5.988)--(5.361,5.988)--(5.362,5.988)%
  --(5.363,5.988)--(5.364,5.988)--(5.365,5.988)--(5.366,5.988)--(5.367,5.988)--(5.368,5.988)%
  --(5.369,5.988)--(5.370,5.988)--(5.371,5.988)--(5.372,5.988)--(5.373,5.988)--(5.374,5.988)%
  --(5.375,5.988)--(5.376,5.988)--(5.377,5.988)--(5.378,5.988)--(5.379,5.988)--(5.380,5.988)%
  --(5.381,5.988)--(5.382,5.988)--(5.383,5.988)--(5.384,5.988)--(5.385,5.988)--(5.386,5.988)%
  --(5.387,5.988)--(5.388,5.988)--(5.389,5.988)--(5.390,5.988)--(5.391,5.988)--(5.392,5.988)%
  --(5.393,5.988)--(5.394,5.988)--(5.395,5.988)--(5.396,5.988)--(5.397,5.988)--(5.398,5.988)%
  --(5.399,5.988)--(5.400,5.988)--(5.401,5.988)--(5.402,5.988)--(5.403,5.988)--(5.404,5.988)%
  --(5.405,5.988)--(5.406,5.988)--(5.407,5.988)--(5.408,5.988)--(5.409,5.988)--(5.410,5.988)%
  --(5.411,5.988)--(5.412,5.988)--(5.413,5.988)--(5.414,5.988)--(5.415,5.988)--(5.416,5.988)%
  --(5.417,5.988)--(5.418,5.988)--(5.419,5.988)--(5.420,5.988)--(5.421,5.988)--(5.422,5.988)%
  --(5.423,5.988)--(5.424,5.988)--(5.425,5.988)--(5.426,5.988)--(5.427,5.988)--(5.428,5.988)%
  --(5.429,5.988)--(5.430,5.988)--(5.431,5.988)--(5.432,5.988)--(5.433,5.988)--(5.434,5.988)%
  --(5.435,5.988)--(5.436,5.988)--(5.437,5.988)--(5.438,5.988)--(5.439,5.988)--(5.440,5.988)%
  --(5.441,5.988)--(5.442,5.988)--(5.443,5.988)--(5.444,5.988)--(5.445,5.988)--(5.446,5.988)%
  --(5.447,5.988)--(5.448,5.988)--(5.449,5.988)--(5.450,5.988)--(5.451,5.988)--(5.452,5.988)%
  --(5.453,5.988)--(5.454,5.988)--(5.455,5.988)--(5.456,5.988)--(5.457,5.988)--(5.458,5.988)%
  --(5.459,5.988)--(5.460,5.988)--(5.461,5.988)--(5.462,5.988)--(5.463,5.988)--(5.464,5.988)%
  --(5.465,5.988)--(5.466,5.988)--(5.467,5.988)--(5.468,5.988)--(5.469,5.988)--(5.470,5.988)%
  --(5.471,5.988)--(5.472,5.988)--(5.473,5.988)--(5.474,5.988)--(5.475,5.988)--(5.476,5.988)%
  --(5.477,5.988)--(5.478,5.988)--(5.479,5.988)--(5.480,5.988)--(5.481,5.988)--(5.482,5.988)%
  --(5.483,5.988)--(5.484,5.988)--(5.485,5.988)--(5.486,5.988)--(5.487,5.988)--(5.488,5.988)%
  --(5.489,5.988)--(5.490,5.988)--(5.491,5.988)--(5.492,5.988)--(5.493,5.988)--(5.494,5.988)%
  --(5.495,5.988)--(5.496,5.988)--(5.497,5.988)--(5.498,5.988)--(5.499,5.988)--(5.500,5.988)%
  --(5.501,5.988)--(5.502,5.988)--(5.503,5.988)--(5.504,5.988)--(5.505,5.988)--(5.506,5.988)%
  --(5.507,5.988)--(5.508,5.988)--(5.509,5.988)--(5.510,5.988)--(5.511,5.988)--(5.512,5.988)%
  --(5.513,5.988)--(5.514,5.988)--(5.515,5.988)--(5.516,5.988)--(5.517,5.988)--(5.518,5.988)%
  --(5.519,5.988)--(5.520,5.988)--(5.521,5.988)--(5.522,5.988)--(5.523,5.988)--(5.524,5.988)%
  --(5.525,5.988)--(5.526,5.988)--(5.527,5.988)--(5.528,5.988)--(5.529,5.988)--(5.530,5.988)%
  --(5.531,5.988)--(5.532,5.988)--(5.533,5.988)--(5.534,5.988)--(5.535,5.988)--(5.536,5.988)%
  --(5.537,5.988)--(5.538,5.988)--(5.539,5.988)--(5.540,5.988)--(5.541,5.988)--(5.542,5.988)%
  --(5.543,5.988)--(5.544,5.988)--(5.545,5.988)--(5.546,5.988)--(5.547,5.988)--(5.548,5.988)%
  --(5.549,5.988)--(5.550,5.988)--(5.551,5.988)--(5.552,5.988)--(5.553,5.988)--(5.554,5.988)%
  --(5.555,5.988)--(5.556,5.988)--(5.557,5.988)--(5.558,5.988)--(5.559,5.988)--(5.560,5.988)%
  --(5.561,5.988)--(5.562,5.988)--(5.563,5.988)--(5.564,5.988)--(5.565,5.988)--(5.566,5.988)%
  --(5.567,5.988)--(5.568,5.988)--(5.569,5.988)--(5.570,5.988)--(5.571,5.988)--(5.572,5.988)%
  --(5.573,5.988)--(5.574,5.988)--(5.575,5.988)--(5.576,5.988)--(5.577,5.988)--(5.578,5.988)%
  --(5.579,5.988)--(5.580,5.988)--(5.581,5.988)--(5.582,5.988)--(5.583,5.988)--(5.584,5.988)%
  --(5.585,5.988)--(5.586,5.988)--(5.587,5.988)--(5.588,5.988)--(5.589,5.988)--(5.590,5.988)%
  --(5.591,5.988)--(5.592,5.988)--(5.593,5.988)--(5.594,5.988)--(5.595,5.988)--(5.596,5.988)%
  --(5.597,5.988)--(5.598,5.988)--(5.599,5.988)--(5.600,5.988)--(5.601,5.988)--(5.602,5.988)%
  --(5.603,5.988)--(5.604,5.988)--(5.605,5.988)--(5.606,5.988)--(5.607,5.988)--(5.608,5.988)%
  --(5.609,5.988)--(5.610,5.988)--(5.611,5.988)--(5.612,5.988)--(5.613,5.988)--(5.614,5.988)%
  --(5.615,5.988)--(5.616,5.988)--(5.617,5.988)--(5.618,5.988)--(5.619,5.988)--(5.620,5.988)%
  --(5.621,5.988)--(5.622,5.988)--(5.623,5.988)--(5.624,5.988)--(5.625,5.988)--(5.626,5.988)%
  --(5.627,5.988)--(5.628,5.988)--(5.629,5.988)--(5.630,5.988)--(5.631,5.988)--(5.632,5.988)%
  --(5.633,5.988)--(5.634,5.988)--(5.635,5.988)--(5.636,5.988)--(5.637,5.988)--(5.638,5.988)%
  --(5.639,5.988)--(5.640,5.988)--(5.641,5.988)--(5.642,5.988)--(5.643,5.988)--(5.644,5.988)%
  --(5.645,5.988)--(5.646,5.988)--(5.647,5.988)--(5.648,5.988)--(5.649,5.988)--(5.650,5.988)%
  --(5.651,5.988)--(5.652,5.988)--(5.653,5.988)--(5.654,5.988)--(5.655,5.988)--(5.656,5.988)%
  --(5.657,5.988)--(5.658,5.988)--(5.659,5.988)--(5.660,5.988)--(5.661,5.988)--(5.662,5.988)%
  --(5.663,5.988)--(5.664,5.988)--(5.665,5.988)--(5.666,5.988)--(5.667,5.988)--(5.668,5.988)%
  --(5.669,5.988)--(5.670,5.988)--(5.671,5.988)--(5.672,5.988)--(5.673,5.988)--(5.674,5.988)%
  --(5.675,5.988)--(5.676,5.988)--(5.677,5.988)--(5.678,5.988)--(5.679,5.988)--(5.680,5.988)%
  --(5.681,5.988)--(5.682,5.988)--(5.683,5.988)--(5.684,5.988)--(5.685,5.988)--(5.686,5.988)%
  --(5.687,5.988)--(5.688,5.988)--(5.689,5.988)--(5.690,5.988)--(5.691,5.988)--(5.692,5.988)%
  --(5.693,5.988)--(5.694,5.988)--(5.695,5.988)--(5.696,5.988)--(5.697,5.988)--(5.698,5.988)%
  --(5.699,5.988)--(5.700,5.988)--(5.701,5.988)--(5.702,5.988)--(5.703,5.988)--(5.704,5.988)%
  --(5.705,5.988)--(5.706,5.988)--(5.707,5.988)--(5.708,5.988)--(5.709,5.988)--(5.710,5.988)%
  --(5.711,5.988)--(5.712,5.988)--(5.713,5.988)--(5.714,5.988)--(5.715,5.988)--(5.716,5.988)%
  --(5.717,5.988)--(5.718,5.988)--(5.719,5.988)--(5.720,5.988)--(5.721,5.988)--(5.722,5.988)%
  --(5.723,5.988)--(5.724,5.988)--(5.725,5.988)--(5.726,5.988)--(5.727,5.988)--(5.728,5.988)%
  --(5.729,5.988)--(5.730,5.988)--(5.731,5.988)--(5.732,5.988)--(5.733,5.988)--(5.734,5.988)%
  --(5.735,5.988)--(5.736,5.988)--(5.737,5.988)--(5.738,5.988)--(5.739,5.988)--(5.740,5.988)%
  --(5.741,5.988)--(5.742,5.988)--(5.743,5.988)--(5.744,5.988)--(5.745,5.988)--(5.746,5.988)%
  --(5.747,5.988)--(5.748,5.988)--(5.749,5.988)--(5.750,5.988)--(5.751,5.988)--(5.752,5.988)%
  --(5.753,5.988)--(5.754,5.988)--(5.755,5.988)--(5.756,5.988)--(5.757,5.988)--(5.758,5.988)%
  --(5.759,5.988)--(5.760,5.988)--(5.761,5.988)--(5.762,5.988)--(5.763,5.988)--(5.764,5.988)%
  --(5.765,5.988)--(5.766,5.988)--(5.767,5.988)--(5.768,5.988)--(5.769,5.988)--(5.770,5.988)%
  --(5.771,5.988)--(5.772,5.988)--(5.773,5.988)--(5.774,5.988)--(5.775,5.988)--(5.776,5.988)%
  --(5.777,5.988)--(5.778,5.988)--(5.779,5.988)--(5.780,5.988)--(5.781,5.988)--(5.782,5.988)%
  --(5.783,5.988)--(5.784,5.988)--(5.785,5.988)--(5.786,5.988)--(5.787,5.988)--(5.788,5.988)%
  --(5.789,5.988)--(5.790,5.988)--(5.791,5.988)--(5.792,5.988)--(5.793,5.988)--(5.794,5.988)%
  --(5.795,5.988)--(5.796,5.988)--(5.797,5.988)--(5.798,5.988)--(5.799,5.988)--(5.800,5.988)%
  --(5.801,5.988)--(5.802,5.988)--(5.803,5.988)--(5.804,5.988)--(5.805,5.988)--(5.806,5.988)%
  --(5.807,5.988)--(5.808,5.988)--(5.809,5.988)--(5.810,5.988)--(5.811,5.988)--(5.812,5.988)%
  --(5.813,5.988)--(5.814,5.988)--(5.815,5.988)--(5.816,5.988)--(5.817,5.988)--(5.818,5.988)%
  --(5.819,5.988)--(5.820,5.988)--(5.821,5.988)--(5.822,5.988)--(5.823,5.988)--(5.824,5.988)%
  --(5.825,5.988)--(5.826,5.988)--(5.827,5.988)--(5.828,5.988)--(5.829,5.988)--(5.830,5.988)%
  --(5.831,5.988)--(5.832,5.988)--(5.833,5.988)--(5.834,5.988)--(5.835,5.988)--(5.836,5.988)%
  --(5.837,5.988)--(5.838,5.988)--(5.839,5.988)--(5.840,5.988)--(5.841,5.988)--(5.842,5.988)%
  --(5.843,5.988)--(5.844,5.988)--(5.845,5.988)--(5.846,5.988)--(5.847,5.988)--(5.848,5.988)%
  --(5.849,5.988)--(5.850,5.988)--(5.851,5.988)--(5.852,5.988)--(5.853,5.988)--(5.854,5.988)%
  --(5.855,5.988)--(5.856,5.988)--(5.857,5.988)--(5.858,5.988)--(5.859,5.988)--(5.860,5.988)%
  --(5.861,5.988)--(5.862,5.988)--(5.863,5.988)--(5.864,5.988)--(5.865,5.988)--(5.866,5.988)%
  --(5.867,5.988)--(5.868,5.988)--(5.869,5.988)--(5.870,5.988)--(5.871,5.988)--(5.872,5.988)%
  --(5.873,5.988)--(5.874,5.988)--(5.875,5.988)--(5.876,5.988)--(5.877,5.988)--(5.878,5.988)%
  --(5.879,5.988)--(5.880,5.988)--(5.881,5.988)--(5.882,5.988)--(5.883,5.988)--(5.884,5.988)%
  --(5.885,5.988)--(5.886,5.988)--(5.887,5.988)--(5.888,5.988)--(5.889,5.988)--(5.890,5.988)%
  --(5.891,5.988)--(5.892,5.988)--(5.893,5.988)--(5.894,5.988)--(5.895,5.988)--(5.896,5.988)%
  --(5.897,5.988)--(5.898,5.988)--(5.899,5.988)--(5.900,5.988)--(5.901,5.988)--(5.902,5.988)%
  --(5.903,5.988)--(5.904,5.988)--(5.905,5.988)--(5.906,5.988)--(5.907,5.988)--(5.908,5.988)%
  --(5.909,5.988)--(5.910,5.988)--(5.911,5.988)--(5.912,5.988)--(5.913,5.988)--(5.914,5.988)%
  --(5.915,5.988)--(5.916,5.988)--(5.917,5.988)--(5.918,5.988)--(5.919,5.988)--(5.920,5.988)%
  --(5.921,5.988)--(5.922,5.988)--(5.923,5.988)--(5.924,5.988)--(5.925,5.988)--(5.926,5.988)%
  --(5.927,5.988)--(5.928,5.988)--(5.929,5.988)--(5.930,5.988)--(5.931,5.988)--(5.932,5.988)%
  --(5.933,5.988)--(5.934,5.988)--(5.935,5.988)--(5.936,5.988)--(5.937,5.988)--(5.938,5.988)%
  --(5.939,5.988)--(5.940,5.988)--(5.941,5.988)--(5.942,5.988)--(5.943,5.988)--(5.944,5.988)%
  --(5.945,5.988)--(5.946,5.988)--(5.947,5.988)--(5.948,5.988)--(5.949,5.988)--(5.950,5.988)%
  --(5.951,5.988)--(5.952,5.988)--(5.953,5.988)--(5.954,5.988)--(5.955,5.988)--(5.956,5.988)%
  --(5.957,5.988)--(5.958,5.988)--(5.959,5.988)--(5.960,5.988)--(5.961,5.988)--(5.962,5.988)%
  --(5.963,5.988)--(5.964,5.988)--(5.965,5.988)--(5.966,5.988)--(5.967,5.988)--(5.968,5.988)%
  --(5.969,5.988)--(5.970,5.988)--(5.971,5.988)--(5.972,5.988)--(5.973,5.988)--(5.974,5.988)%
  --(5.975,5.988)--(5.976,5.988)--(5.977,5.988)--(5.978,5.988)--(5.979,5.988)--(5.980,5.988)%
  --(5.981,5.988)--(5.982,5.988)--(5.983,5.988)--(5.984,5.988)--(5.985,5.988)--(5.986,5.988)%
  --(5.987,5.988)--(5.988,5.988)--(5.989,5.988)--(5.990,5.988)--(5.991,5.988)--(5.992,5.988)%
  --(5.993,5.988)--(5.994,5.988)--(5.995,5.988)--(5.996,5.988)--(5.997,5.988)--(5.998,5.988)%
  --(5.999,5.988)--(6.000,5.988)--(6.001,5.988)--(6.002,5.988)--(6.003,5.988)--(6.004,5.988)%
  --(6.005,5.988)--(6.006,5.988)--(6.007,5.988)--(6.008,5.988)--(6.009,5.988)--(6.010,5.988)%
  --(6.011,5.988)--(6.012,5.988)--(6.013,5.988)--(6.014,5.988)--(6.015,5.988)--(6.016,5.988)%
  --(6.017,5.988)--(6.018,5.988)--(6.019,5.988)--(6.020,5.988)--(6.021,5.988)--(6.022,5.988)%
  --(6.023,5.988)--(6.024,5.988)--(6.025,5.988)--(6.026,5.988)--(6.027,5.988)--(6.028,5.988)%
  --(6.029,5.988)--(6.030,5.988)--(6.031,5.988)--(6.032,5.988)--(6.033,5.988)--(6.034,5.988)%
  --(6.035,5.988)--(6.036,5.988)--(6.037,5.988)--(6.038,5.988)--(6.039,5.988)--(6.040,5.988)%
  --(6.041,5.988)--(6.042,5.988)--(6.043,5.988)--(6.044,5.988)--(6.045,5.988)--(6.046,5.988)%
  --(6.047,5.988)--(6.048,5.988)--(6.049,5.988)--(6.050,5.988)--(6.051,5.988)--(6.052,5.988)%
  --(6.053,5.988)--(6.054,5.988)--(6.055,5.988)--(6.056,5.988)--(6.057,5.988)--(6.058,5.988)%
  --(6.059,5.988)--(6.060,5.988)--(6.061,5.988)--(6.062,5.988)--(6.063,5.988)--(6.064,5.988)%
  --(6.065,5.988)--(6.066,5.988)--(6.067,5.988)--(6.068,5.988)--(6.069,5.988)--(6.070,5.988)%
  --(6.071,5.988)--(6.072,5.988)--(6.073,5.988)--(6.074,5.988)--(6.075,5.988)--(6.076,5.988)%
  --(6.077,5.988)--(6.078,5.988)--(6.079,5.988)--(6.080,5.988)--(6.081,5.988)--(6.082,5.988)%
  --(6.083,5.988)--(6.084,5.988)--(6.085,5.988)--(6.086,5.988)--(6.087,5.988)--(6.088,5.988)%
  --(6.089,5.988)--(6.090,5.988)--(6.091,5.988)--(6.092,5.988)--(6.093,5.988)--(6.094,5.988)%
  --(6.095,5.988)--(6.096,5.988)--(6.097,5.988)--(6.098,5.988)--(6.099,5.988)--(6.100,5.988)%
  --(6.101,5.988)--(6.102,5.988)--(6.103,5.988)--(6.104,5.988)--(6.105,5.988)--(6.106,5.988)%
  --(6.107,5.988)--(6.108,5.988)--(6.109,5.988)--(6.110,5.988)--(6.111,5.988)--(6.112,5.988)%
  --(6.113,5.988)--(6.114,5.988)--(6.115,5.988)--(6.116,5.988)--(6.117,5.988)--(6.118,5.988)%
  --(6.119,5.988)--(6.120,5.988)--(6.121,5.988)--(6.122,5.988)--(6.123,5.988)--(6.124,5.988)%
  --(6.125,5.988)--(6.126,5.988)--(6.127,5.988)--(6.128,5.988)--(6.129,5.988)--(6.130,5.988)%
  --(6.131,5.988)--(6.132,5.988)--(6.133,5.988)--(6.134,5.988)--(6.135,5.988)--(6.136,5.988)%
  --(6.137,5.988)--(6.138,5.988)--(6.139,5.988)--(6.140,5.988)--(6.141,5.988)--(6.142,5.988)%
  --(6.143,5.988)--(6.144,5.988)--(6.145,5.988)--(6.146,5.988)--(6.147,5.988)--(6.148,5.988)%
  --(6.149,5.988)--(6.150,5.988)--(6.151,5.988)--(6.152,5.988)--(6.153,5.988)--(6.154,5.988)%
  --(6.155,5.988)--(6.156,5.988)--(6.157,5.988)--(6.158,5.988)--(6.159,5.988)--(6.160,5.988)%
  --(6.161,5.988)--(6.162,5.988)--(6.163,5.988)--(6.164,5.988)--(6.165,5.988)--(6.166,5.988)%
  --(6.167,5.988)--(6.168,5.988)--(6.169,5.988)--(6.170,5.988)--(6.171,5.988)--(6.172,5.988)%
  --(6.173,5.988)--(6.174,5.988)--(6.175,5.988)--(6.176,5.988)--(6.177,5.988)--(6.178,5.988)%
  --(6.179,5.988)--(6.180,5.988)--(6.181,5.988)--(6.182,5.988)--(6.183,5.988)--(6.184,5.988)%
  --(6.185,5.988)--(6.186,5.988)--(6.187,5.988)--(6.188,5.988)--(6.189,5.988)--(6.190,5.988)%
  --(6.191,5.988)--(6.192,5.988)--(6.193,5.988)--(6.194,5.988)--(6.195,5.988)--(6.196,5.988)%
  --(6.197,5.988)--(6.198,5.988)--(6.199,5.988)--(6.200,5.988)--(6.201,5.988)--(6.202,5.988)%
  --(6.203,5.988)--(6.204,5.988)--(6.205,5.988)--(6.206,5.988)--(6.207,5.988)--(6.208,5.988)%
  --(6.209,5.988)--(6.210,5.988)--(6.211,5.988)--(6.212,5.988)--(6.213,5.988)--(6.214,5.988)%
  --(6.215,5.988)--(6.216,5.988)--(6.217,5.988)--(6.218,5.988)--(6.219,5.988)--(6.220,5.988)%
  --(6.221,5.988)--(6.222,5.988)--(6.223,5.988)--(6.224,5.988)--(6.225,5.988)--(6.226,5.988)%
  --(6.227,5.988)--(6.228,5.988)--(6.229,5.988)--(6.230,5.988)--(6.231,5.988)--(6.232,5.988)%
  --(6.233,5.988)--(6.234,5.988)--(6.235,5.988)--(6.236,5.988)--(6.237,5.988)--(6.238,5.988)%
  --(6.239,5.988)--(6.240,5.988)--(6.241,5.988)--(6.242,5.988)--(6.243,5.988)--(6.244,5.988)%
  --(6.245,5.988)--(6.246,5.988)--(6.247,5.988)--(6.248,5.988)--(6.249,5.988)--(6.250,5.988)%
  --(6.251,5.988)--(6.252,5.988)--(6.253,5.988)--(6.254,5.988)--(6.255,5.988)--(6.256,5.988)%
  --(6.257,5.988)--(6.258,5.988)--(6.259,5.988)--(6.260,5.988)--(6.261,5.988)--(6.262,5.988)%
  --(6.263,5.988)--(6.264,5.988)--(6.265,5.988)--(6.266,5.988)--(6.267,5.988)--(6.268,5.988)%
  --(6.269,5.988)--(6.270,5.988)--(6.271,5.988)--(6.272,5.988)--(6.273,5.988)--(6.274,5.988)%
  --(6.275,5.988)--(6.276,5.988)--(6.277,5.988)--(6.278,5.988)--(6.279,5.988)--(6.280,5.988)%
  --(6.281,5.988)--(6.282,5.988)--(6.283,5.988)--(6.284,5.988)--(6.285,5.988)--(6.286,5.988)%
  --(6.287,5.988)--(6.288,5.988)--(6.289,5.988)--(6.290,5.988)--(6.291,5.988)--(6.292,5.988)%
  --(6.293,5.988)--(6.294,5.988)--(6.295,5.988)--(6.296,5.988)--(6.297,5.988)--(6.298,5.988)%
  --(6.299,5.988)--(6.300,5.988)--(6.301,5.988)--(6.302,5.988)--(6.303,5.988)--(6.304,5.988)%
  --(6.305,5.988)--(6.306,5.988)--(6.307,5.988)--(6.308,5.988)--(6.309,5.988)--(6.310,5.988)%
  --(6.311,5.988)--(6.312,5.988)--(6.313,5.988)--(6.314,5.988)--(6.315,5.988)--(6.316,5.988)%
  --(6.317,5.988)--(6.318,5.988)--(6.319,5.988)--(6.320,5.988)--(6.321,5.988)--(6.322,5.988)%
  --(6.323,5.988)--(6.324,5.988)--(6.325,5.988)--(6.326,5.988)--(6.327,5.988)--(6.328,5.988)%
  --(6.329,5.988)--(6.330,5.988)--(6.331,5.988)--(6.332,5.988)--(6.333,5.988)--(6.334,5.988)%
  --(6.335,5.988)--(6.336,5.988)--(6.337,5.988)--(6.338,5.988)--(6.339,5.988)--(6.340,5.988)%
  --(6.341,5.988)--(6.342,5.988)--(6.343,5.988)--(6.344,5.988)--(6.345,5.988)--(6.346,5.988)%
  --(6.347,5.988)--(6.348,5.988)--(6.349,5.988)--(6.350,5.988)--(6.351,5.988)--(6.352,5.988)%
  --(6.353,5.988)--(6.354,5.988)--(6.355,5.988)--(6.356,5.988)--(6.357,5.988)--(6.358,5.988)%
  --(6.359,5.988)--(6.360,5.988)--(6.361,5.988)--(6.362,5.988)--(6.363,5.988)--(6.364,5.988)%
  --(6.365,5.988)--(6.366,5.988)--(6.367,5.988)--(6.368,5.988)--(6.369,5.988)--(6.370,5.988)%
  --(6.371,5.988)--(6.372,5.988)--(6.373,5.988)--(6.374,5.988)--(6.375,5.988)--(6.376,5.988)%
  --(6.377,5.988)--(6.378,5.988)--(6.379,5.988)--(6.380,5.988)--(6.381,5.988)--(6.382,5.988)%
  --(6.383,5.988)--(6.384,5.988)--(6.385,5.988)--(6.386,5.988)--(6.387,5.988)--(6.388,5.988)%
  --(6.389,5.988)--(6.390,5.988)--(6.391,5.988)--(6.392,5.988)--(6.393,5.988)--(6.394,5.988)%
  --(6.395,5.988)--(6.396,5.988)--(6.397,5.988)--(6.398,5.988)--(6.399,5.988)--(6.400,5.988)%
  --(6.401,5.988)--(6.402,5.988)--(6.403,5.988)--(6.404,5.988)--(6.405,5.988)--(6.406,5.988)%
  --(6.407,5.988)--(6.408,5.988)--(6.409,5.988)--(6.410,5.988)--(6.411,5.988)--(6.412,5.988)%
  --(6.413,5.988)--(6.414,5.988)--(6.415,5.988)--(6.416,5.988)--(6.417,5.988)--(6.418,5.988)%
  --(6.419,5.988)--(6.420,5.988)--(6.421,5.988)--(6.422,5.988)--(6.423,5.988)--(6.424,5.988)%
  --(6.425,5.988)--(6.426,5.988)--(6.427,5.988)--(6.428,5.988)--(6.429,5.988)--(6.430,5.988)%
  --(6.431,5.988)--(6.432,5.988)--(6.433,5.988)--(6.434,5.988)--(6.435,5.988)--(6.436,5.988)%
  --(6.437,5.988)--(6.438,5.988)--(6.439,5.988)--(6.440,5.988)--(6.441,5.988)--(6.442,5.988)%
  --(6.443,5.988)--(6.444,5.988)--(6.445,5.988)--(6.446,5.988)--(6.447,5.988)--(6.448,5.988)%
  --(6.449,5.988)--(6.450,5.988)--(6.451,5.988)--(6.452,5.988)--(6.453,5.988)--(6.454,5.988)%
  --(6.455,5.988)--(6.456,5.988)--(6.457,5.988)--(6.458,5.988)--(6.459,5.988)--(6.460,5.988)%
  --(6.461,5.988)--(6.462,5.988)--(6.463,5.988)--(6.464,5.988)--(6.465,5.988)--(6.466,5.988)%
  --(6.467,5.988)--(6.468,5.988)--(6.469,5.988)--(6.470,5.988)--(6.471,5.988)--(6.472,5.988)%
  --(6.473,5.988)--(6.474,5.988)--(6.475,5.988)--(6.476,5.988)--(6.477,5.988)--(6.478,5.988)%
  --(6.479,5.988)--(6.480,5.988)--(6.481,5.988)--(6.482,5.988)--(6.483,5.988)--(6.484,5.988)%
  --(6.485,5.988)--(6.486,5.988)--(6.487,5.988)--(6.488,5.988)--(6.489,5.988)--(6.490,5.988)%
  --(6.491,5.988)--(6.492,5.988)--(6.493,5.988)--(6.494,5.988)--(6.495,5.988)--(6.496,5.988)%
  --(6.497,5.988)--(6.498,5.988)--(6.499,5.988)--(6.500,5.988)--(6.501,5.988)--(6.502,5.988)%
  --(6.503,5.988)--(6.504,5.988)--(6.505,5.988)--(6.506,5.988)--(6.507,5.988)--(6.508,5.988)%
  --(6.509,5.988)--(6.510,5.988)--(6.511,5.988)--(6.512,5.988)--(6.513,5.988)--(6.514,5.988)%
  --(6.515,5.988)--(6.516,5.988)--(6.517,5.988)--(6.518,5.988)--(6.519,5.988)--(6.520,5.988)%
  --(6.521,5.988)--(6.522,5.988)--(6.523,5.988)--(6.524,5.988)--(6.525,5.988)--(6.526,5.988)%
  --(6.527,5.988)--(6.528,5.988)--(6.529,5.988)--(6.530,5.988)--(6.531,5.988)--(6.532,5.988)%
  --(6.533,5.988)--(6.534,5.988)--(6.535,5.988)--(6.536,5.988)--(6.537,5.988)--(6.538,5.988)%
  --(6.539,5.988)--(6.540,5.988)--(6.541,5.988)--(6.542,5.988)--(6.543,5.988)--(6.544,5.988)%
  --(6.545,5.988)--(6.546,5.988)--(6.547,5.988)--(6.548,5.988)--(6.549,5.988)--(6.550,5.988)%
  --(6.551,5.988)--(6.552,5.988)--(6.553,5.988)--(6.554,5.988)--(6.555,5.988)--(6.556,5.988)%
  --(6.557,5.988)--(6.558,5.988)--(6.559,5.988)--(6.560,5.988)--(6.561,5.988)--(6.562,5.988)%
  --(6.563,5.988)--(6.564,5.988)--(6.565,5.988)--(6.566,5.988)--(6.567,5.988)--(6.568,5.988)%
  --(6.569,5.988)--(6.570,5.988)--(6.571,5.988)--(6.572,5.988)--(6.573,5.988)--(6.574,5.988)%
  --(6.575,5.988)--(6.576,5.988)--(6.577,5.988)--(6.578,5.988)--(6.579,5.988)--(6.580,5.988)%
  --(6.581,5.988)--(6.582,5.988)--(6.583,5.988)--(6.584,5.988)--(6.585,5.988)--(6.586,5.988)%
  --(6.587,5.988)--(6.588,5.988)--(6.589,5.988)--(6.590,5.988)--(6.591,5.988)--(6.592,5.988)%
  --(6.593,5.988)--(6.594,5.988)--(6.595,5.988)--(6.596,5.988)--(6.597,5.988)--(6.598,5.988)%
  --(6.599,5.988)--(6.600,5.988)--(6.601,5.988)--(6.602,5.988)--(6.603,5.988)--(6.604,5.988)%
  --(6.605,5.988)--(6.606,5.988)--(6.607,5.988)--(6.608,5.988)--(6.609,5.988)--(6.610,5.988)%
  --(6.611,5.988)--(6.612,5.988)--(6.613,5.988)--(6.614,5.988)--(6.615,5.988)--(6.616,5.988)%
  --(6.617,5.988)--(6.618,5.988)--(6.619,5.988)--(6.620,5.988)--(6.621,5.988)--(6.622,5.988)%
  --(6.623,5.988)--(6.624,5.988)--(6.625,5.988)--(6.626,5.988)--(6.627,5.988)--(6.628,5.988)%
  --(6.629,5.988)--(6.630,5.988)--(6.631,5.988)--(6.632,5.988)--(6.633,5.988)--(6.634,5.988)%
  --(6.635,5.988)--(6.636,5.988)--(6.637,5.988)--(6.638,5.988)--(6.639,5.988)--(6.640,5.988)%
  --(6.641,5.988)--(6.642,5.988)--(6.643,5.988)--(6.644,5.988)--(6.645,5.988)--(6.646,5.988)%
  --(6.647,5.988)--(6.648,5.988)--(6.649,5.988)--(6.650,5.988)--(6.651,5.988)--(6.652,5.988)%
  --(6.653,5.988)--(6.654,5.988)--(6.655,5.988)--(6.656,5.988)--(6.657,5.988)--(6.658,5.988)%
  --(6.659,5.988)--(6.660,5.988)--(6.661,5.988)--(6.662,5.988)--(6.663,5.988)--(6.664,5.988)%
  --(6.665,5.988)--(6.666,5.988)--(6.667,5.988)--(6.668,5.988)--(6.669,5.988)--(6.670,5.988)%
  --(6.671,5.988)--(6.672,5.988)--(6.673,5.988)--(6.674,5.988)--(6.675,5.988)--(6.676,5.988)%
  --(6.677,5.988)--(6.678,5.988)--(6.679,5.988)--(6.680,5.988)--(6.681,5.988)--(6.682,5.988)%
  --(6.683,5.988)--(6.684,5.988)--(6.685,5.988)--(6.686,5.988)--(6.687,5.988)--(6.688,5.988)%
  --(6.689,5.988)--(6.690,5.988)--(6.691,5.988)--(6.692,5.988)--(6.693,5.988)--(6.694,5.988)%
  --(6.695,5.988)--(6.696,5.988)--(6.697,5.988)--(6.698,5.988)--(6.699,5.988)--(6.700,5.988)%
  --(6.701,5.988)--(6.702,5.988)--(6.703,5.988)--(6.704,5.988)--(6.705,5.988)--(6.706,5.988)%
  --(6.707,5.988)--(6.708,5.988)--(6.709,5.988)--(6.710,5.988)--(6.711,5.988)--(6.712,5.988)%
  --(6.713,5.988)--(6.714,5.988)--(6.715,5.988)--(6.716,5.988)--(6.717,5.988)--(6.718,5.988)%
  --(6.719,5.988)--(6.720,5.988)--(6.721,5.988)--(6.722,5.988)--(6.723,5.988)--(6.724,5.988)%
  --(6.725,5.988)--(6.726,5.988)--(6.727,5.988)--(6.728,5.988)--(6.729,5.988)--(6.730,5.988)%
  --(6.731,5.988)--(6.732,5.988)--(6.733,5.988)--(6.734,5.988)--(6.735,5.988)--(6.736,5.988)%
  --(6.737,5.988)--(6.738,5.988)--(6.739,5.988)--(6.740,5.988)--(6.741,5.988)--(6.742,5.988)%
  --(6.743,5.988)--(6.744,5.988)--(6.745,5.988)--(6.746,5.988)--(6.747,5.988)--(6.748,5.988)%
  --(6.749,5.988)--(6.750,5.988)--(6.751,5.988)--(6.752,5.988)--(6.753,5.988)--(6.754,5.988)%
  --(6.755,5.988)--(6.756,5.988)--(6.757,5.988)--(6.758,5.988)--(6.759,5.988)--(6.760,5.988)%
  --(6.761,5.988)--(6.762,5.988)--(6.763,5.988)--(6.764,5.988)--(6.765,5.988)--(6.766,5.988)%
  --(6.767,5.988)--(6.768,5.988)--(6.769,5.988)--(6.770,5.988)--(6.771,5.988)--(6.772,5.988)%
  --(6.773,5.988)--(6.774,5.988)--(6.775,5.988)--(6.776,5.988)--(6.777,5.988)--(6.778,5.988)%
  --(6.779,5.988)--(6.780,5.988)--(6.781,5.988)--(6.782,5.988)--(6.783,5.988)--(6.784,5.988)%
  --(6.785,5.988)--(6.786,5.988)--(6.787,5.988)--(6.788,5.988)--(6.789,5.988)--(6.790,5.988)%
  --(6.791,5.988)--(6.792,5.988)--(6.793,5.988)--(6.794,5.988)--(6.795,5.988)--(6.796,5.988)%
  --(6.797,5.988)--(6.798,5.988)--(6.799,5.988)--(6.800,5.988)--(6.801,5.988)--(6.802,5.988)%
  --(6.803,5.988)--(6.804,5.988)--(6.805,5.988)--(6.806,5.988)--(6.807,5.988)--(6.808,5.988)%
  --(6.809,5.988)--(6.810,5.988)--(6.811,5.988)--(6.812,5.988)--(6.813,5.988)--(6.814,5.988)%
  --(6.815,5.988)--(6.816,5.988)--(6.817,5.988)--(6.818,5.988)--(6.819,5.988)--(6.820,5.988)%
  --(6.821,5.988)--(6.822,5.988)--(6.823,5.988)--(6.824,5.988)--(6.825,5.988)--(6.826,5.988)%
  --(6.827,5.988)--(6.828,5.988)--(6.829,5.988)--(6.830,5.988)--(6.831,5.988)--(6.832,5.988)%
  --(6.833,5.988)--(6.834,5.988)--(6.835,5.988)--(6.836,5.988)--(6.837,5.988)--(6.838,5.988)%
  --(6.839,5.988)--(6.840,5.988)--(6.841,5.988)--(6.842,5.988)--(6.843,5.988)--(6.844,5.988)%
  --(6.845,5.988)--(6.846,5.988)--(6.847,5.988)--(6.848,5.988)--(6.849,5.988)--(6.850,5.988)%
  --(6.851,5.988)--(6.852,5.988)--(6.853,5.988)--(6.854,5.988)--(6.855,5.988)--(6.856,5.988)%
  --(6.857,5.988)--(6.858,5.988)--(6.859,5.988)--(6.860,5.988)--(6.861,5.988)--(6.862,5.988)%
  --(6.863,5.988)--(6.864,5.988)--(6.865,5.988)--(6.866,5.988)--(6.867,5.988)--(6.868,5.988)%
  --(6.869,5.988)--(6.870,5.988)--(6.871,5.988)--(6.872,5.988)--(6.873,5.988)--(6.874,5.988)%
  --(6.875,5.988)--(6.876,5.988)--(6.877,5.988)--(6.878,5.988)--(6.879,5.988)--(6.880,5.988)%
  --(6.881,5.988)--(6.882,5.988)--(6.883,5.988)--(6.884,5.988)--(6.885,5.988)--(6.886,5.988)%
  --(6.887,5.988)--(6.888,5.988)--(6.889,5.988)--(6.890,5.988)--(6.891,5.988)--(6.892,5.988)%
  --(6.893,5.988)--(6.894,5.988)--(6.895,5.988)--(6.896,5.988)--(6.897,5.988)--(6.898,5.988)%
  --(6.899,5.988)--(6.900,5.988)--(6.901,5.988)--(6.902,5.988)--(6.903,5.988)--(6.904,5.988)%
  --(6.905,5.988)--(6.906,5.988)--(6.907,5.988)--(6.908,5.988)--(6.909,5.988)--(6.910,5.988)%
  --(6.911,5.988)--(6.912,5.988)--(6.913,5.988)--(6.914,5.988)--(6.915,5.988)--(6.916,5.988)%
  --(6.917,5.988)--(6.918,5.988)--(6.919,5.988)--(6.920,5.988)--(6.921,5.988)--(6.922,5.988)%
  --(6.923,5.988)--(6.924,5.988)--(6.925,5.988)--(6.926,5.988)--(6.927,5.988)--(6.928,5.988)%
  --(6.929,5.988)--(6.930,5.988)--(6.931,5.988)--(6.932,5.988)--(6.933,5.988)--(6.934,5.988)%
  --(6.935,5.988)--(6.936,5.988)--(6.937,5.988)--(6.938,5.988)--(6.939,5.988)--(6.940,5.988)%
  --(6.941,5.988)--(6.942,5.988)--(6.943,5.988)--(6.944,5.988)--(6.945,5.988)--(6.946,5.988)%
  --(6.947,5.988)--(6.948,5.988)--(6.949,5.988)--(6.950,5.988)--(6.951,5.988)--(6.952,5.988)%
  --(6.953,5.988)--(6.954,5.988)--(6.955,5.988)--(6.956,5.988)--(6.957,5.988)--(6.958,5.988)%
  --(6.959,5.988)--(6.960,5.988)--(6.961,5.988)--(6.962,5.988)--(6.963,5.988)--(6.964,5.988)%
  --(6.965,5.988)--(6.966,5.988)--(6.967,5.988)--(6.968,5.988)--(6.969,5.988)--(6.970,5.988)%
  --(6.971,5.988)--(6.972,5.988)--(6.973,5.988)--(6.974,5.988)--(6.975,5.988)--(6.976,5.988)%
  --(6.977,5.988)--(6.978,5.988)--(6.979,5.988)--(6.980,5.988)--(6.981,5.988)--(6.982,5.988)%
  --(6.983,5.988)--(6.984,5.988)--(6.985,5.988)--(6.986,5.988)--(6.987,5.988)--(6.988,5.988)%
  --(6.989,5.988)--(6.990,5.988)--(6.991,5.988)--(6.992,5.988)--(6.993,5.988)--(6.994,5.988)%
  --(6.995,5.988)--(6.996,5.988)--(6.997,5.988)--(6.998,5.988)--(6.999,5.988)--(7.000,5.988)%
  --(7.001,5.988)--(7.002,5.988)--(7.003,5.988)--(7.004,5.988)--(7.005,5.988)--(7.006,5.988)%
  --(7.007,5.988)--(7.008,5.988)--(7.009,5.988)--(7.010,5.988)--(7.011,5.988)--(7.012,5.988)%
  --(7.013,5.988)--(7.014,5.988)--(7.015,5.988)--(7.016,5.988)--(7.017,5.988)--(7.018,5.988)%
  --(7.019,5.988)--(7.020,5.988)--(7.021,5.988)--(7.022,5.988)--(7.023,5.988)--(7.024,5.988)%
  --(7.025,5.988)--(7.026,5.988)--(7.027,5.988)--(7.028,5.988)--(7.029,5.988)--(7.030,5.988)%
  --(7.031,5.988)--(7.032,5.988)--(7.033,5.988)--(7.034,5.988)--(7.035,5.988)--(7.036,5.988)%
  --(7.037,5.988)--(7.038,5.988)--(7.039,5.988)--(7.040,5.988)--(7.041,5.988)--(7.042,5.988)%
  --(7.043,5.988)--(7.044,5.988)--(7.045,5.988)--(7.046,5.988)--(7.047,5.988)--(7.048,5.988)%
  --(7.049,5.988)--(7.050,5.988)--(7.051,5.988)--(7.052,5.988)--(7.053,5.988)--(7.054,5.988)%
  --(7.055,5.988)--(7.056,5.988)--(7.057,5.988)--(7.058,5.988)--(7.059,5.988)--(7.060,5.988)%
  --(7.061,5.988)--(7.062,5.988)--(7.063,5.988)--(7.064,5.988)--(7.065,5.988)--(7.066,5.988)%
  --(7.067,5.988)--(7.068,5.988)--(7.069,5.988)--(7.070,5.988)--(7.071,5.988)--(7.072,5.988)%
  --(7.073,5.988)--(7.074,5.988)--(7.075,5.988)--(7.076,5.988)--(7.077,5.988)--(7.078,5.988)%
  --(7.079,5.988)--(7.080,5.988)--(7.081,5.988)--(7.082,5.988)--(7.083,5.988)--(7.084,5.988)%
  --(7.085,5.988)--(7.086,5.988)--(7.087,5.988)--(7.088,5.988)--(7.089,5.988)--(7.090,5.988)%
  --(7.091,5.988)--(7.092,5.988)--(7.093,5.988)--(7.094,5.988)--(7.095,5.988)--(7.096,5.988)%
  --(7.097,5.988)--(7.098,5.988)--(7.099,5.988)--(7.100,5.988)--(7.101,5.988)--(7.102,5.988)%
  --(7.103,5.988)--(7.104,5.988)--(7.105,5.988)--(7.106,5.988)--(7.107,5.988)--(7.108,5.988)%
  --(7.109,5.988)--(7.110,5.988)--(7.111,5.988)--(7.112,5.988)--(7.113,5.988)--(7.114,5.988)%
  --(7.115,5.988)--(7.116,5.988)--(7.117,5.988)--(7.118,5.988)--(7.119,5.988)--(7.120,5.988)%
  --(7.121,5.988)--(7.122,5.988)--(7.123,5.988)--(7.124,5.988)--(7.125,5.988)--(7.126,5.988)%
  --(7.127,5.988)--(7.128,5.988)--(7.129,5.988)--(7.130,5.988)--(7.131,5.988)--(7.132,5.988)%
  --(7.133,5.988)--(7.134,5.988)--(7.135,5.988)--(7.136,5.988)--(7.137,5.988)--(7.138,5.988)%
  --(7.139,5.988)--(7.140,5.988)--(7.141,5.988)--(7.142,5.988)--(7.143,5.988)--(7.144,5.988)%
  --(7.145,5.988)--(7.146,5.988)--(7.147,5.988)--(7.148,5.988)--(7.149,5.988)--(7.150,5.988)%
  --(7.151,5.988)--(7.152,5.988)--(7.153,5.988)--(7.154,5.988)--(7.155,5.988)--(7.156,5.988)%
  --(7.157,5.988)--(7.158,5.988)--(7.159,5.988)--(7.160,5.988)--(7.161,5.988)--(7.162,5.988)%
  --(7.163,5.988)--(7.164,5.988)--(7.165,5.988)--(7.166,5.988)--(7.167,5.988)--(7.168,5.988)%
  --(7.169,5.988)--(7.170,5.988)--(7.171,5.988)--(7.172,5.988)--(7.173,5.988)--(7.174,5.988)%
  --(7.175,5.988)--(7.176,5.988)--(7.177,5.988)--(7.178,5.988)--(7.179,5.988)--(7.180,5.988)%
  --(7.181,5.988)--(7.182,5.988)--(7.183,5.988)--(7.184,5.988)--(7.185,5.988)--(7.186,5.988)%
  --(7.187,5.988)--(7.188,5.988)--(7.189,5.988)--(7.190,5.988)--(7.191,5.988)--(7.192,5.988)%
  --(7.193,5.988)--(7.194,5.988)--(7.195,5.988)--(7.196,5.988)--(7.197,5.988)--(7.198,5.988)%
  --(7.199,5.988)--(7.200,5.988)--(7.201,5.988)--(7.202,5.988)--(7.203,5.988)--(7.204,5.988)%
  --(7.205,5.988)--(7.206,5.988)--(7.207,5.988)--(7.208,5.988)--(7.209,5.988)--(7.210,5.988)%
  --(7.211,5.988)--(7.212,5.988)--(7.213,5.988)--(7.214,5.988)--(7.215,5.988)--(7.216,5.988)%
  --(7.217,5.988)--(7.218,5.988)--(7.219,5.988)--(7.220,5.988)--(7.221,5.988)--(7.222,5.988)%
  --(7.223,5.988)--(7.224,5.988)--(7.225,5.988)--(7.226,5.988)--(7.227,5.988)--(7.228,5.988)%
  --(7.229,5.988)--(7.230,5.988)--(7.231,5.988)--(7.232,5.988)--(7.233,5.988)--(7.234,5.988)%
  --(7.235,5.988)--(7.236,5.988)--(7.237,5.988)--(7.238,5.988)--(7.239,5.988)--(7.240,5.988)%
  --(7.241,5.988)--(7.242,5.988)--(7.243,5.988)--(7.244,5.988)--(7.245,5.988)--(7.246,5.988)%
  --(7.247,5.988)--(7.248,5.988)--(7.249,5.988)--(7.250,5.988)--(7.251,5.988)--(7.252,5.988)%
  --(7.253,5.988)--(7.254,5.988)--(7.255,5.988)--(7.256,5.988)--(7.257,5.988)--(7.258,5.988)%
  --(7.259,5.988)--(7.260,5.988)--(7.261,5.988)--(7.262,5.988)--(7.263,5.988)--(7.264,5.988)%
  --(7.265,5.988)--(7.266,5.988)--(7.267,5.988)--(7.268,5.988)--(7.269,5.988)--(7.270,5.988)%
  --(7.271,5.988)--(7.272,5.988)--(7.273,5.988)--(7.274,5.988)--(7.275,5.988)--(7.276,5.988)%
  --(7.277,5.988)--(7.278,5.988)--(7.279,5.988)--(7.280,5.988)--(7.281,5.988)--(7.282,5.988)%
  --(7.283,5.988)--(7.284,5.988)--(7.285,5.988)--(7.286,5.988)--(7.287,5.988)--(7.288,5.988)%
  --(7.289,5.988)--(7.290,5.988)--(7.291,5.988)--(7.292,5.988)--(7.293,5.988)--(7.294,5.988)%
  --(7.295,5.988)--(7.296,5.988)--(7.297,5.988)--(7.298,5.988)--(7.299,5.988)--(7.300,5.988)%
  --(7.301,5.988)--(7.302,5.988)--(7.303,5.988)--(7.304,5.988)--(7.305,5.988)--(7.306,5.988)%
  --(7.307,5.988)--(7.308,5.988)--(7.309,5.988)--(7.310,5.988)--(7.311,5.988)--(7.312,5.988)%
  --(7.313,5.988)--(7.314,5.988)--(7.315,5.988)--(7.316,5.988)--(7.317,5.988)--(7.318,5.988)%
  --(7.319,5.988)--(7.320,5.988)--(7.321,5.988)--(7.322,5.988)--(7.323,5.988)--(7.324,5.988)%
  --(7.325,5.988)--(7.326,5.988)--(7.327,5.988)--(7.328,5.988)--(7.329,5.988)--(7.330,5.988)%
  --(7.331,5.988)--(7.332,5.988)--(7.333,5.988)--(7.334,5.988)--(7.335,5.988)--(7.336,5.988)%
  --(7.337,5.988)--(7.338,5.988)--(7.339,5.988)--(7.340,5.988)--(7.341,5.988)--(7.342,5.988)%
  --(7.343,5.988)--(7.344,5.988)--(7.345,5.988)--(7.346,5.988)--(7.347,5.988)--(7.348,5.988)%
  --(7.349,5.988)--(7.350,5.988)--(7.351,5.988)--(7.352,5.988)--(7.353,5.988)--(7.354,5.988)%
  --(7.355,5.988)--(7.356,5.988)--(7.357,5.988)--(7.358,5.988)--(7.359,5.988)--(7.360,5.988)%
  --(7.361,5.988)--(7.362,5.988)--(7.363,5.988)--(7.364,5.988)--(7.365,5.988)--(7.366,5.988)%
  --(7.367,5.988)--(7.368,5.988)--(7.369,5.988)--(7.370,5.988)--(7.371,5.988)--(7.372,5.988)%
  --(7.373,5.988)--(7.374,5.988)--(7.375,5.988)--(7.376,5.988)--(7.377,5.988)--(7.378,5.988)%
  --(7.379,5.988)--(7.380,5.988)--(7.381,5.988)--(7.382,5.988)--(7.383,5.988)--(7.384,5.988)%
  --(7.385,5.988)--(7.386,5.988)--(7.387,5.988)--(7.388,5.988)--(7.389,5.988)--(7.390,5.988)%
  --(7.391,5.988)--(7.392,5.988)--(7.393,5.988)--(7.394,5.988)--(7.395,5.988)--(7.396,5.988)%
  --(7.397,5.988)--(7.398,5.988)--(7.399,5.988)--(7.400,5.988)--(7.401,5.988)--(7.402,5.988)%
  --(7.403,5.988)--(7.404,5.988)--(7.405,5.988)--(7.406,5.988)--(7.407,5.988)--(7.408,5.988)%
  --(7.409,5.988)--(7.410,5.988)--(7.411,5.988)--(7.412,5.988)--(7.413,5.988)--(7.414,5.988)%
  --(7.415,5.988)--(7.416,5.988)--(7.417,5.988)--(7.418,5.988)--(7.419,5.988)--(7.420,5.988)%
  --(7.421,5.988)--(7.422,5.988)--(7.423,5.988)--(7.424,5.988)--(7.425,5.988)--(7.426,5.988)%
  --(7.427,5.988)--(7.428,5.988)--(7.429,5.988)--(7.430,5.988)--(7.431,5.988)--(7.432,5.988)%
  --(7.433,5.988)--(7.434,5.988)--(7.435,5.988)--(7.436,5.988)--(7.437,5.988)--(7.438,5.988)%
  --(7.439,5.988)--(7.440,5.988)--(7.441,5.988)--(7.442,5.988)--(7.443,5.988)--(7.444,5.988)%
  --(7.445,5.988)--(7.446,5.988)--(7.447,5.988);
\gpcolor{color=gp lt color border}
\node[gp node right] at (5.979,1.916) {$y={k\Theta x}/{Ng^2\mu_B^2J^2\omega}$};
\gpcolor{rgb color={0.000,0.000,0.000}}
\gpsetdashtype{gp dt 5}
\draw[gp path] (6.163,1.916)--(7.079,1.916);
\draw[gp path] (1.320,1.274)--(1.320,1.275)--(1.321,1.275)--(1.321,1.276)--(1.321,1.277)%
  --(1.321,1.278)--(1.322,1.278)--(1.322,1.279)--(1.322,1.280)--(1.323,1.280)--(1.323,1.281)%
  --(1.323,1.282)--(1.323,1.283)--(1.324,1.283)--(1.324,1.284)--(1.324,1.285)--(1.325,1.286)%
  --(1.325,1.287)--(1.325,1.288)--(1.326,1.288)--(1.326,1.289)--(1.326,1.290)--(1.326,1.291)%
  --(1.327,1.291)--(1.327,1.292)--(1.327,1.293)--(1.328,1.293)--(1.328,1.294)--(1.328,1.295)%
  --(1.328,1.296)--(1.329,1.296)--(1.329,1.297)--(1.329,1.298)--(1.330,1.299)--(1.330,1.300)%
  --(1.330,1.301)--(1.331,1.301)--(1.331,1.302)--(1.331,1.303)--(1.332,1.304)--(1.332,1.305)%
  --(1.332,1.306)--(1.333,1.306)--(1.333,1.307)--(1.333,1.308)--(1.333,1.309)--(1.334,1.309)%
  --(1.334,1.310)--(1.334,1.311)--(1.335,1.311)--(1.335,1.312)--(1.335,1.313)--(1.335,1.314)%
  --(1.336,1.314)--(1.336,1.315)--(1.336,1.316)--(1.337,1.316)--(1.337,1.317)--(1.337,1.318)%
  --(1.337,1.319)--(1.338,1.319)--(1.338,1.320)--(1.338,1.321)--(1.339,1.321)--(1.339,1.322)%
  --(1.339,1.323)--(1.339,1.324)--(1.340,1.324)--(1.340,1.325)--(1.340,1.326)--(1.341,1.327)%
  --(1.341,1.328)--(1.341,1.329)--(1.342,1.329)--(1.342,1.330)--(1.342,1.331)--(1.342,1.332)%
  --(1.343,1.332)--(1.343,1.333)--(1.343,1.334)--(1.344,1.334)--(1.344,1.335)--(1.344,1.336)%
  --(1.344,1.337)--(1.345,1.337)--(1.345,1.338)--(1.345,1.339)--(1.346,1.340)--(1.346,1.341)%
  --(1.346,1.342)--(1.347,1.342)--(1.347,1.343)--(1.347,1.344)--(1.348,1.345)--(1.348,1.346)%
  --(1.348,1.347)--(1.349,1.347)--(1.349,1.348)--(1.349,1.349)--(1.349,1.350)--(1.350,1.350)%
  --(1.350,1.351)--(1.350,1.352)--(1.351,1.352)--(1.351,1.353)--(1.351,1.354)--(1.351,1.355)%
  --(1.352,1.355)--(1.352,1.356)--(1.352,1.357)--(1.353,1.357)--(1.353,1.358)--(1.353,1.359)%
  --(1.353,1.360)--(1.354,1.360)--(1.354,1.361)--(1.354,1.362)--(1.355,1.363)--(1.355,1.364)%
  --(1.355,1.365)--(1.356,1.365)--(1.356,1.366)--(1.356,1.367)--(1.357,1.368)--(1.357,1.369)%
  --(1.357,1.370)--(1.358,1.370)--(1.358,1.371)--(1.358,1.372)--(1.358,1.373)--(1.359,1.373)%
  --(1.359,1.374)--(1.359,1.375)--(1.360,1.375)--(1.360,1.376)--(1.360,1.377)--(1.360,1.378)%
  --(1.361,1.378)--(1.361,1.379)--(1.361,1.380)--(1.362,1.381)--(1.362,1.382)--(1.362,1.383)%
  --(1.363,1.383)--(1.363,1.384)--(1.363,1.385)--(1.364,1.386)--(1.364,1.387)--(1.364,1.388)%
  --(1.365,1.388)--(1.365,1.389)--(1.365,1.390)--(1.365,1.391)--(1.366,1.391)--(1.366,1.392)%
  --(1.366,1.393)--(1.367,1.393)--(1.367,1.394)--(1.367,1.395)--(1.367,1.396)--(1.368,1.396)%
  --(1.368,1.397)--(1.368,1.398)--(1.369,1.398)--(1.369,1.399)--(1.369,1.400)--(1.369,1.401)%
  --(1.370,1.401)--(1.370,1.402)--(1.370,1.403)--(1.371,1.404)--(1.371,1.405)--(1.371,1.406)%
  --(1.372,1.406)--(1.372,1.407)--(1.372,1.408)--(1.372,1.409)--(1.373,1.409)--(1.373,1.410)%
  --(1.373,1.411)--(1.374,1.411)--(1.374,1.412)--(1.374,1.413)--(1.374,1.414)--(1.375,1.414)%
  --(1.375,1.415)--(1.375,1.416)--(1.376,1.416)--(1.376,1.417)--(1.376,1.418)--(1.376,1.419)%
  --(1.377,1.419)--(1.377,1.420)--(1.377,1.421)--(1.378,1.422)--(1.378,1.423)--(1.378,1.424)%
  --(1.379,1.424)--(1.379,1.425)--(1.379,1.426)--(1.379,1.427)--(1.380,1.427)--(1.380,1.428)%
  --(1.380,1.429)--(1.381,1.429)--(1.381,1.430)--(1.381,1.431)--(1.381,1.432)--(1.382,1.432)%
  --(1.382,1.433)--(1.382,1.434)--(1.383,1.434)--(1.383,1.435)--(1.383,1.436)--(1.383,1.437)%
  --(1.384,1.437)--(1.384,1.438)--(1.384,1.439)--(1.385,1.439)--(1.385,1.440)--(1.385,1.441)%
  --(1.385,1.442)--(1.386,1.442)--(1.386,1.443)--(1.386,1.444)--(1.387,1.445)--(1.387,1.446)%
  --(1.387,1.447)--(1.388,1.447)--(1.388,1.448)--(1.388,1.449)--(1.388,1.450)--(1.389,1.450)%
  --(1.389,1.451)--(1.389,1.452)--(1.390,1.452)--(1.390,1.453)--(1.390,1.454)--(1.390,1.455)%
  --(1.391,1.455)--(1.391,1.456)--(1.391,1.457)--(1.392,1.457)--(1.392,1.458)--(1.392,1.459)%
  --(1.392,1.460)--(1.393,1.460)--(1.393,1.461)--(1.393,1.462)--(1.394,1.463)--(1.394,1.464)%
  --(1.394,1.465)--(1.395,1.465)--(1.395,1.466)--(1.395,1.467)--(1.395,1.468)--(1.396,1.468)%
  --(1.396,1.469)--(1.396,1.470)--(1.397,1.470)--(1.397,1.471)--(1.397,1.472)--(1.397,1.473)%
  --(1.398,1.473)--(1.398,1.474)--(1.398,1.475)--(1.399,1.475)--(1.399,1.476)--(1.399,1.477)%
  --(1.399,1.478)--(1.400,1.478)--(1.400,1.479)--(1.400,1.480)--(1.401,1.480)--(1.401,1.481)%
  --(1.401,1.482)--(1.401,1.483)--(1.402,1.483)--(1.402,1.484)--(1.402,1.485)--(1.403,1.486)%
  --(1.403,1.487)--(1.403,1.488)--(1.404,1.488)--(1.404,1.489)--(1.404,1.490)--(1.404,1.491)%
  --(1.405,1.491)--(1.405,1.492)--(1.405,1.493)--(1.406,1.493)--(1.406,1.494)--(1.406,1.495)%
  --(1.406,1.496)--(1.407,1.496)--(1.407,1.497)--(1.407,1.498)--(1.408,1.499)--(1.408,1.500)%
  --(1.408,1.501)--(1.409,1.501)--(1.409,1.502)--(1.409,1.503)--(1.410,1.504)--(1.410,1.505)%
  --(1.410,1.506)--(1.411,1.506)--(1.411,1.507)--(1.411,1.508)--(1.411,1.509)--(1.412,1.509)%
  --(1.412,1.510)--(1.412,1.511)--(1.413,1.511)--(1.413,1.512)--(1.413,1.513)--(1.413,1.514)%
  --(1.414,1.514)--(1.414,1.515)--(1.414,1.516)--(1.415,1.516)--(1.415,1.517)--(1.415,1.518)%
  --(1.415,1.519)--(1.416,1.519)--(1.416,1.520)--(1.416,1.521)--(1.417,1.521)--(1.417,1.522)%
  --(1.417,1.523)--(1.417,1.524)--(1.418,1.524)--(1.418,1.525)--(1.418,1.526)--(1.418,1.527)%
  --(1.419,1.527)--(1.419,1.528)--(1.419,1.529)--(1.420,1.529)--(1.420,1.530)--(1.420,1.531)%
  --(1.420,1.532)--(1.421,1.532)--(1.421,1.533)--(1.421,1.534)--(1.422,1.534)--(1.422,1.535)%
  --(1.422,1.536)--(1.422,1.537)--(1.423,1.537)--(1.423,1.538)--(1.423,1.539)--(1.424,1.540)%
  --(1.424,1.541)--(1.424,1.542)--(1.425,1.542)--(1.425,1.543)--(1.425,1.544)--(1.426,1.545)%
  --(1.426,1.546)--(1.426,1.547)--(1.427,1.547)--(1.427,1.548)--(1.427,1.549)--(1.427,1.550)%
  --(1.428,1.550)--(1.428,1.551)--(1.428,1.552)--(1.429,1.552)--(1.429,1.553)--(1.429,1.554)%
  --(1.429,1.555)--(1.430,1.555)--(1.430,1.556)--(1.430,1.557)--(1.431,1.557)--(1.431,1.558)%
  --(1.431,1.559)--(1.431,1.560)--(1.432,1.560)--(1.432,1.561)--(1.432,1.562)--(1.433,1.563)%
  --(1.433,1.564)--(1.433,1.565)--(1.434,1.565)--(1.434,1.566)--(1.434,1.567)--(1.434,1.568)%
  --(1.435,1.568)--(1.435,1.569)--(1.435,1.570)--(1.436,1.570)--(1.436,1.571)--(1.436,1.572)%
  --(1.436,1.573)--(1.437,1.573)--(1.437,1.574)--(1.437,1.575)--(1.438,1.575)--(1.438,1.576)%
  --(1.438,1.577)--(1.438,1.578)--(1.439,1.578)--(1.439,1.579)--(1.439,1.580)--(1.440,1.581)%
  --(1.440,1.582)--(1.440,1.583)--(1.441,1.583)--(1.441,1.584)--(1.441,1.585)--(1.441,1.586)%
  --(1.442,1.586)--(1.442,1.587)--(1.442,1.588)--(1.443,1.588)--(1.443,1.589)--(1.443,1.590)%
  --(1.443,1.591)--(1.444,1.591)--(1.444,1.592)--(1.444,1.593)--(1.445,1.593)--(1.445,1.594)%
  --(1.445,1.595)--(1.445,1.596)--(1.446,1.596)--(1.446,1.597)--(1.446,1.598)--(1.447,1.598)%
  --(1.447,1.599)--(1.447,1.600)--(1.447,1.601)--(1.448,1.601)--(1.448,1.602)--(1.448,1.603)%
  --(1.448,1.604)--(1.449,1.604)--(1.449,1.605)--(1.449,1.606)--(1.450,1.606)--(1.450,1.607)%
  --(1.450,1.608)--(1.450,1.609)--(1.451,1.609)--(1.451,1.610)--(1.451,1.611)--(1.452,1.611)%
  --(1.452,1.612)--(1.452,1.613)--(1.452,1.614)--(1.453,1.614)--(1.453,1.615)--(1.453,1.616)%
  --(1.454,1.616)--(1.454,1.617)--(1.454,1.618)--(1.454,1.619)--(1.455,1.619)--(1.455,1.620)%
  --(1.455,1.621)--(1.456,1.622)--(1.456,1.623)--(1.456,1.624)--(1.457,1.624)--(1.457,1.625)%
  --(1.457,1.626)--(1.457,1.627)--(1.458,1.627)--(1.458,1.628)--(1.458,1.629)--(1.459,1.629)%
  --(1.459,1.630)--(1.459,1.631)--(1.459,1.632)--(1.460,1.632)--(1.460,1.633)--(1.460,1.634)%
  --(1.461,1.634)--(1.461,1.635)--(1.461,1.636)--(1.461,1.637)--(1.462,1.637)--(1.462,1.638)%
  --(1.462,1.639)--(1.463,1.639)--(1.463,1.640)--(1.463,1.641)--(1.463,1.642)--(1.464,1.642)%
  --(1.464,1.643)--(1.464,1.644)--(1.464,1.645)--(1.465,1.645)--(1.465,1.646)--(1.465,1.647)%
  --(1.466,1.647)--(1.466,1.648)--(1.466,1.649)--(1.466,1.650)--(1.467,1.650)--(1.467,1.651)%
  --(1.467,1.652)--(1.468,1.652)--(1.468,1.653)--(1.468,1.654)--(1.468,1.655)--(1.469,1.655)%
  --(1.469,1.656)--(1.469,1.657)--(1.470,1.657)--(1.470,1.658)--(1.470,1.659)--(1.470,1.660)%
  --(1.471,1.660)--(1.471,1.661)--(1.471,1.662)--(1.472,1.663)--(1.472,1.664)--(1.472,1.665)%
  --(1.473,1.665)--(1.473,1.666)--(1.473,1.667)--(1.473,1.668)--(1.474,1.668)--(1.474,1.669)%
  --(1.474,1.670)--(1.475,1.670)--(1.475,1.671)--(1.475,1.672)--(1.475,1.673)--(1.476,1.673)%
  --(1.476,1.674)--(1.476,1.675)--(1.477,1.675)--(1.477,1.676)--(1.477,1.677)--(1.477,1.678)%
  --(1.478,1.678)--(1.478,1.679)--(1.478,1.680)--(1.479,1.680)--(1.479,1.681)--(1.479,1.682)%
  --(1.479,1.683)--(1.480,1.683)--(1.480,1.684)--(1.480,1.685)--(1.480,1.686)--(1.481,1.686)%
  --(1.481,1.687)--(1.481,1.688)--(1.482,1.688)--(1.482,1.689)--(1.482,1.690)--(1.482,1.691)%
  --(1.483,1.691)--(1.483,1.692)--(1.483,1.693)--(1.484,1.693)--(1.484,1.694)--(1.484,1.695)%
  --(1.484,1.696)--(1.485,1.696)--(1.485,1.697)--(1.485,1.698)--(1.486,1.699)--(1.486,1.700)%
  --(1.486,1.701)--(1.487,1.701)--(1.487,1.702)--(1.487,1.703)--(1.488,1.704)--(1.488,1.705)%
  --(1.488,1.706)--(1.489,1.706)--(1.489,1.707)--(1.489,1.708)--(1.489,1.709)--(1.490,1.709)%
  --(1.490,1.710)--(1.490,1.711)--(1.491,1.711)--(1.491,1.712)--(1.491,1.713)--(1.491,1.714)%
  --(1.492,1.714)--(1.492,1.715)--(1.492,1.716)--(1.493,1.716)--(1.493,1.717)--(1.493,1.718)%
  --(1.493,1.719)--(1.494,1.719)--(1.494,1.720)--(1.494,1.721)--(1.494,1.722)--(1.495,1.722)%
  --(1.495,1.723)--(1.495,1.724)--(1.496,1.724)--(1.496,1.725)--(1.496,1.726)--(1.496,1.727)%
  --(1.497,1.727)--(1.497,1.728)--(1.497,1.729)--(1.498,1.729)--(1.498,1.730)--(1.498,1.731)%
  --(1.498,1.732)--(1.499,1.732)--(1.499,1.733)--(1.499,1.734)--(1.500,1.734)--(1.500,1.735)%
  --(1.500,1.736)--(1.500,1.737)--(1.501,1.737)--(1.501,1.738)--(1.501,1.739)--(1.502,1.740)%
  --(1.502,1.741)--(1.502,1.742)--(1.503,1.742)--(1.503,1.743)--(1.503,1.744)--(1.504,1.745)%
  --(1.504,1.746)--(1.504,1.747)--(1.505,1.747)--(1.505,1.748)--(1.505,1.749)--(1.505,1.750)%
  --(1.506,1.750)--(1.506,1.751)--(1.506,1.752)--(1.507,1.752)--(1.507,1.753)--(1.507,1.754)%
  --(1.507,1.755)--(1.508,1.755)--(1.508,1.756)--(1.508,1.757)--(1.509,1.757)--(1.509,1.758)%
  --(1.509,1.759)--(1.509,1.760)--(1.510,1.760)--(1.510,1.761)--(1.510,1.762)--(1.510,1.763)%
  --(1.511,1.763)--(1.511,1.764)--(1.511,1.765)--(1.512,1.765)--(1.512,1.766)--(1.512,1.767)%
  --(1.512,1.768)--(1.513,1.768)--(1.513,1.769)--(1.513,1.770)--(1.514,1.770)--(1.514,1.771)%
  --(1.514,1.772)--(1.514,1.773)--(1.515,1.773)--(1.515,1.774)--(1.515,1.775)--(1.516,1.775)%
  --(1.516,1.776)--(1.516,1.777)--(1.516,1.778)--(1.517,1.778)--(1.517,1.779)--(1.517,1.780)%
  --(1.518,1.781)--(1.518,1.782)--(1.518,1.783)--(1.519,1.783)--(1.519,1.784)--(1.519,1.785)%
  --(1.519,1.786)--(1.520,1.786)--(1.520,1.787)--(1.520,1.788)--(1.521,1.788)--(1.521,1.789)%
  --(1.521,1.790)--(1.521,1.791)--(1.522,1.791)--(1.522,1.792)--(1.522,1.793)--(1.523,1.793)%
  --(1.523,1.794)--(1.523,1.795)--(1.523,1.796)--(1.524,1.796)--(1.524,1.797)--(1.524,1.798)%
  --(1.525,1.798)--(1.525,1.799)--(1.525,1.800)--(1.525,1.801)--(1.526,1.801)--(1.526,1.802)%
  --(1.526,1.803)--(1.526,1.804)--(1.527,1.804)--(1.527,1.805)--(1.527,1.806)--(1.528,1.806)%
  --(1.528,1.807)--(1.528,1.808)--(1.528,1.809)--(1.529,1.809)--(1.529,1.810)--(1.529,1.811)%
  --(1.530,1.811)--(1.530,1.812)--(1.530,1.813)--(1.530,1.814)--(1.531,1.814)--(1.531,1.815)%
  --(1.531,1.816)--(1.532,1.816)--(1.532,1.817)--(1.532,1.818)--(1.532,1.819)--(1.533,1.819)%
  --(1.533,1.820)--(1.533,1.821)--(1.534,1.822)--(1.534,1.823)--(1.534,1.824)--(1.535,1.824)%
  --(1.535,1.825)--(1.535,1.826)--(1.535,1.827)--(1.536,1.827)--(1.536,1.828)--(1.536,1.829)%
  --(1.537,1.829)--(1.537,1.830)--(1.537,1.831)--(1.537,1.832)--(1.538,1.832)--(1.538,1.833)%
  --(1.538,1.834)--(1.539,1.834)--(1.539,1.835)--(1.539,1.836)--(1.539,1.837)--(1.540,1.837)%
  --(1.540,1.838)--(1.540,1.839)--(1.541,1.840)--(1.541,1.841)--(1.541,1.842)--(1.542,1.842)%
  --(1.542,1.843)--(1.542,1.844)--(1.542,1.845)--(1.543,1.845)--(1.543,1.846)--(1.543,1.847)%
  --(1.544,1.847)--(1.544,1.848)--(1.544,1.849)--(1.544,1.850)--(1.545,1.850)--(1.545,1.851)%
  --(1.545,1.852)--(1.546,1.852)--(1.546,1.853)--(1.546,1.854)--(1.546,1.855)--(1.547,1.855)%
  --(1.547,1.856)--(1.547,1.857)--(1.548,1.858)--(1.548,1.859)--(1.548,1.860)--(1.549,1.860)%
  --(1.549,1.861)--(1.549,1.862)--(1.550,1.863)--(1.550,1.864)--(1.550,1.865)--(1.551,1.865)%
  --(1.551,1.866)--(1.551,1.867)--(1.551,1.868)--(1.552,1.868)--(1.552,1.869)--(1.552,1.870)%
  --(1.553,1.870)--(1.553,1.871)--(1.553,1.872)--(1.553,1.873)--(1.554,1.873)--(1.554,1.874)%
  --(1.554,1.875)--(1.555,1.875)--(1.555,1.876)--(1.555,1.877)--(1.555,1.878)--(1.556,1.878)%
  --(1.556,1.879)--(1.556,1.880)--(1.557,1.881)--(1.557,1.882)--(1.557,1.883)--(1.558,1.883)%
  --(1.558,1.884)--(1.558,1.885)--(1.558,1.886)--(1.559,1.886)--(1.559,1.887)--(1.559,1.888)%
  --(1.560,1.888)--(1.560,1.889)--(1.560,1.890)--(1.560,1.891)--(1.561,1.891)--(1.561,1.892)%
  --(1.561,1.893)--(1.562,1.893)--(1.562,1.894)--(1.562,1.895)--(1.562,1.896)--(1.563,1.896)%
  --(1.563,1.897)--(1.563,1.898)--(1.564,1.899)--(1.564,1.900)--(1.564,1.901)--(1.565,1.901)%
  --(1.565,1.902)--(1.565,1.903)--(1.566,1.904)--(1.566,1.905)--(1.566,1.906)--(1.567,1.906)%
  --(1.567,1.907)--(1.567,1.908)--(1.567,1.909)--(1.568,1.909)--(1.568,1.910)--(1.568,1.911)%
  --(1.569,1.911)--(1.569,1.912)--(1.569,1.913)--(1.569,1.914)--(1.570,1.914)--(1.570,1.915)%
  --(1.570,1.916)--(1.571,1.917)--(1.571,1.918)--(1.571,1.919)--(1.572,1.919)--(1.572,1.920)%
  --(1.572,1.921)--(1.572,1.922)--(1.573,1.922)--(1.573,1.923)--(1.573,1.924)--(1.574,1.924)%
  --(1.574,1.925)--(1.574,1.926)--(1.574,1.927)--(1.575,1.927)--(1.575,1.928)--(1.575,1.929)%
  --(1.576,1.929)--(1.576,1.930)--(1.576,1.931)--(1.576,1.932)--(1.577,1.932)--(1.577,1.933)%
  --(1.577,1.934)--(1.578,1.934)--(1.578,1.935)--(1.578,1.936)--(1.578,1.937)--(1.579,1.937)%
  --(1.579,1.938)--(1.579,1.939)--(1.580,1.940)--(1.580,1.941)--(1.580,1.942)--(1.581,1.942)%
  --(1.581,1.943)--(1.581,1.944)--(1.582,1.945)--(1.582,1.946)--(1.582,1.947)--(1.583,1.947)%
  --(1.583,1.948)--(1.583,1.949)--(1.583,1.950)--(1.584,1.950)--(1.584,1.951)--(1.584,1.952)%
  --(1.585,1.952)--(1.585,1.953)--(1.585,1.954)--(1.585,1.955)--(1.586,1.955)--(1.586,1.956)%
  --(1.586,1.957)--(1.587,1.958)--(1.587,1.959)--(1.587,1.960)--(1.588,1.960)--(1.588,1.961)%
  --(1.588,1.962)--(1.588,1.963)--(1.589,1.963)--(1.589,1.964)--(1.589,1.965)--(1.590,1.965)%
  --(1.590,1.966)--(1.590,1.967)--(1.590,1.968)--(1.591,1.968)--(1.591,1.969)--(1.591,1.970)%
  --(1.592,1.970)--(1.592,1.971)--(1.592,1.972)--(1.592,1.973)--(1.593,1.973)--(1.593,1.974)%
  --(1.593,1.975)--(1.594,1.975)--(1.594,1.976)--(1.594,1.977)--(1.594,1.978)--(1.595,1.978)%
  --(1.595,1.979)--(1.595,1.980)--(1.596,1.981)--(1.596,1.982)--(1.596,1.983)--(1.597,1.983)%
  --(1.597,1.984)--(1.597,1.985)--(1.597,1.986)--(1.598,1.986)--(1.598,1.987)--(1.598,1.988)%
  --(1.599,1.988)--(1.599,1.989)--(1.599,1.990)--(1.599,1.991)--(1.600,1.991)--(1.600,1.992)%
  --(1.600,1.993)--(1.601,1.993)--(1.601,1.994)--(1.601,1.995)--(1.601,1.996)--(1.602,1.996)%
  --(1.602,1.997)--(1.602,1.998)--(1.603,1.999)--(1.603,2.000)--(1.603,2.001)--(1.604,2.001)%
  --(1.604,2.002)--(1.604,2.003)--(1.604,2.004)--(1.605,2.004)--(1.605,2.005)--(1.605,2.006)%
  --(1.606,2.006)--(1.606,2.007)--(1.606,2.008)--(1.606,2.009)--(1.607,2.009)--(1.607,2.010)%
  --(1.607,2.011)--(1.608,2.011)--(1.608,2.012)--(1.608,2.013)--(1.608,2.014)--(1.609,2.014)%
  --(1.609,2.015)--(1.609,2.016)--(1.610,2.016)--(1.610,2.017)--(1.610,2.018)--(1.610,2.019)%
  --(1.611,2.019)--(1.611,2.020)--(1.611,2.021)--(1.612,2.022)--(1.612,2.023)--(1.612,2.024)%
  --(1.613,2.024)--(1.613,2.025)--(1.613,2.026)--(1.613,2.027)--(1.614,2.027)--(1.614,2.028)%
  --(1.614,2.029)--(1.615,2.029)--(1.615,2.030)--(1.615,2.031)--(1.615,2.032)--(1.616,2.032)%
  --(1.616,2.033)--(1.616,2.034)--(1.617,2.035)--(1.617,2.036)--(1.617,2.037)--(1.618,2.037)%
  --(1.618,2.038)--(1.618,2.039)--(1.619,2.040)--(1.619,2.041)--(1.619,2.042)--(1.620,2.042)%
  --(1.620,2.043)--(1.620,2.044)--(1.620,2.045)--(1.621,2.045)--(1.621,2.046)--(1.621,2.047)%
  --(1.622,2.047)--(1.622,2.048)--(1.622,2.049)--(1.622,2.050)--(1.623,2.050)--(1.623,2.051)%
  --(1.623,2.052)--(1.624,2.052)--(1.624,2.053)--(1.624,2.054)--(1.624,2.055)--(1.625,2.055)%
  --(1.625,2.056)--(1.625,2.057)--(1.626,2.058)--(1.626,2.059)--(1.626,2.060)--(1.627,2.060)%
  --(1.627,2.061)--(1.627,2.062)--(1.628,2.063)--(1.628,2.064)--(1.628,2.065)--(1.629,2.065)%
  --(1.629,2.066)--(1.629,2.067)--(1.629,2.068)--(1.630,2.068)--(1.630,2.069)--(1.630,2.070)%
  --(1.631,2.070)--(1.631,2.071)--(1.631,2.072)--(1.631,2.073)--(1.632,2.073)--(1.632,2.074)%
  --(1.632,2.075)--(1.633,2.076)--(1.633,2.077)--(1.633,2.078)--(1.634,2.078)--(1.634,2.079)%
  --(1.634,2.080)--(1.635,2.081)--(1.635,2.082)--(1.635,2.083)--(1.636,2.083)--(1.636,2.084)%
  --(1.636,2.085)--(1.636,2.086)--(1.637,2.086)--(1.637,2.087)--(1.637,2.088)--(1.638,2.088)%
  --(1.638,2.089)--(1.638,2.090)--(1.638,2.091)--(1.639,2.091)--(1.639,2.092)--(1.639,2.093)%
  --(1.640,2.093)--(1.640,2.094)--(1.640,2.095)--(1.640,2.096)--(1.641,2.096)--(1.641,2.097)%
  --(1.641,2.098)--(1.642,2.099)--(1.642,2.100)--(1.642,2.101)--(1.643,2.101)--(1.643,2.102)%
  --(1.643,2.103)--(1.644,2.104)--(1.644,2.105)--(1.644,2.106)--(1.645,2.106)--(1.645,2.107)%
  --(1.645,2.108)--(1.645,2.109)--(1.646,2.109)--(1.646,2.110)--(1.646,2.111)--(1.647,2.111)%
  --(1.647,2.112)--(1.647,2.113)--(1.647,2.114)--(1.648,2.114)--(1.648,2.115)--(1.648,2.116)%
  --(1.649,2.117)--(1.649,2.118)--(1.649,2.119)--(1.650,2.119)--(1.650,2.120)--(1.650,2.121)%
  --(1.650,2.122)--(1.651,2.122)--(1.651,2.123)--(1.651,2.124)--(1.652,2.124)--(1.652,2.125)%
  --(1.652,2.126)--(1.652,2.127)--(1.653,2.127)--(1.653,2.128)--(1.653,2.129)--(1.654,2.129)%
  --(1.654,2.130)--(1.654,2.131)--(1.654,2.132)--(1.655,2.132)--(1.655,2.133)--(1.655,2.134)%
  --(1.656,2.134)--(1.656,2.135)--(1.656,2.136)--(1.656,2.137)--(1.657,2.137)--(1.657,2.138)%
  --(1.657,2.139)--(1.658,2.140)--(1.658,2.141)--(1.658,2.142)--(1.659,2.142)--(1.659,2.143)%
  --(1.659,2.144)--(1.660,2.145)--(1.660,2.146)--(1.660,2.147)--(1.661,2.147)--(1.661,2.148)%
  --(1.661,2.149)--(1.661,2.150)--(1.662,2.150)--(1.662,2.151)--(1.662,2.152)--(1.663,2.152)%
  --(1.663,2.153)--(1.663,2.154)--(1.663,2.155)--(1.664,2.155)--(1.664,2.156)--(1.664,2.157)%
  --(1.665,2.158)--(1.665,2.159)--(1.665,2.160)--(1.666,2.160)--(1.666,2.161)--(1.666,2.162)%
  --(1.666,2.163)--(1.667,2.163)--(1.667,2.164)--(1.667,2.165)--(1.668,2.165)--(1.668,2.166)%
  --(1.668,2.167)--(1.668,2.168)--(1.669,2.168)--(1.669,2.169)--(1.669,2.170)--(1.670,2.170)%
  --(1.670,2.171)--(1.670,2.172)--(1.670,2.173)--(1.671,2.173)--(1.671,2.174)--(1.671,2.175)%
  --(1.672,2.175)--(1.672,2.176)--(1.672,2.177)--(1.672,2.178)--(1.673,2.178)--(1.673,2.179)%
  --(1.673,2.180)--(1.674,2.181)--(1.674,2.182)--(1.674,2.183)--(1.675,2.183)--(1.675,2.184)%
  --(1.675,2.185)--(1.675,2.186)--(1.676,2.186)--(1.676,2.187)--(1.676,2.188)--(1.677,2.188)%
  --(1.677,2.189)--(1.677,2.190)--(1.677,2.191)--(1.678,2.191)--(1.678,2.192)--(1.678,2.193)%
  --(1.679,2.194)--(1.679,2.195)--(1.679,2.196)--(1.680,2.196)--(1.680,2.197)--(1.680,2.198)%
  --(1.681,2.199)--(1.681,2.200)--(1.681,2.201)--(1.682,2.201)--(1.682,2.202)--(1.682,2.203)%
  --(1.682,2.204)--(1.683,2.204)--(1.683,2.205)--(1.683,2.206)--(1.684,2.206)--(1.684,2.207)%
  --(1.684,2.208)--(1.684,2.209)--(1.685,2.209)--(1.685,2.210)--(1.685,2.211)--(1.686,2.211)%
  --(1.686,2.212)--(1.686,2.213)--(1.686,2.214)--(1.687,2.214)--(1.687,2.215)--(1.687,2.216)%
  --(1.688,2.216)--(1.688,2.217)--(1.688,2.218)--(1.688,2.219)--(1.689,2.219)--(1.689,2.220)%
  --(1.689,2.221)--(1.690,2.222)--(1.690,2.223)--(1.690,2.224)--(1.691,2.224)--(1.691,2.225)%
  --(1.691,2.226)--(1.691,2.227)--(1.692,2.227)--(1.692,2.228)--(1.692,2.229)--(1.693,2.229)%
  --(1.693,2.230)--(1.693,2.231)--(1.693,2.232)--(1.694,2.232)--(1.694,2.233)--(1.694,2.234)%
  --(1.695,2.235)--(1.695,2.236)--(1.695,2.237)--(1.696,2.237)--(1.696,2.238)--(1.696,2.239)%
  --(1.697,2.240)--(1.697,2.241)--(1.697,2.242)--(1.698,2.242)--(1.698,2.243)--(1.698,2.244)%
  --(1.698,2.245)--(1.699,2.245)--(1.699,2.246)--(1.699,2.247)--(1.700,2.247)--(1.700,2.248)%
  --(1.700,2.249)--(1.700,2.250)--(1.701,2.250)--(1.701,2.251)--(1.701,2.252)--(1.702,2.252)%
  --(1.702,2.253)--(1.702,2.254)--(1.702,2.255)--(1.703,2.255)--(1.703,2.256)--(1.703,2.257)%
  --(1.704,2.258)--(1.704,2.259)--(1.704,2.260)--(1.705,2.260)--(1.705,2.261)--(1.705,2.262)%
  --(1.705,2.263)--(1.706,2.263)--(1.706,2.264)--(1.706,2.265)--(1.707,2.265)--(1.707,2.266)%
  --(1.707,2.267)--(1.707,2.268)--(1.708,2.268)--(1.708,2.269)--(1.708,2.270)--(1.709,2.270)%
  --(1.709,2.271)--(1.709,2.272)--(1.709,2.273)--(1.710,2.273)--(1.710,2.274)--(1.710,2.275)%
  --(1.711,2.276)--(1.711,2.277)--(1.711,2.278)--(1.712,2.278)--(1.712,2.279)--(1.712,2.280)%
  --(1.712,2.281)--(1.713,2.281)--(1.713,2.282)--(1.713,2.283)--(1.714,2.283)--(1.714,2.284)%
  --(1.714,2.285)--(1.714,2.286)--(1.715,2.286)--(1.715,2.287)--(1.715,2.288)--(1.716,2.288)%
  --(1.716,2.289)--(1.716,2.290)--(1.716,2.291)--(1.717,2.291)--(1.717,2.292)--(1.717,2.293)%
  --(1.718,2.293)--(1.718,2.294)--(1.718,2.295)--(1.718,2.296)--(1.719,2.296)--(1.719,2.297)%
  --(1.719,2.298)--(1.720,2.299)--(1.720,2.300)--(1.720,2.301)--(1.721,2.301)--(1.721,2.302)%
  --(1.721,2.303)--(1.721,2.304)--(1.722,2.304)--(1.722,2.305)--(1.722,2.306)--(1.723,2.306)%
  --(1.723,2.307)--(1.723,2.308)--(1.723,2.309)--(1.724,2.309)--(1.724,2.310)--(1.724,2.311)%
  --(1.725,2.311)--(1.725,2.312)--(1.725,2.313)--(1.725,2.314)--(1.726,2.314)--(1.726,2.315)%
  --(1.726,2.316)--(1.727,2.317)--(1.727,2.318)--(1.727,2.319)--(1.728,2.319)--(1.728,2.320)%
  --(1.728,2.321)--(1.728,2.322)--(1.729,2.322)--(1.729,2.323)--(1.729,2.324)--(1.730,2.324)%
  --(1.730,2.325)--(1.730,2.326)--(1.730,2.327)--(1.731,2.327)--(1.731,2.328)--(1.731,2.329)%
  --(1.732,2.329)--(1.732,2.330)--(1.732,2.331)--(1.732,2.332)--(1.733,2.332)--(1.733,2.333)%
  --(1.733,2.334)--(1.734,2.334)--(1.734,2.335)--(1.734,2.336)--(1.734,2.337)--(1.735,2.337)%
  --(1.735,2.338)--(1.735,2.339)--(1.736,2.340)--(1.736,2.341)--(1.736,2.342)--(1.737,2.342)%
  --(1.737,2.343)--(1.737,2.344)--(1.737,2.345)--(1.738,2.345)--(1.738,2.346)--(1.738,2.347)%
  --(1.739,2.347)--(1.739,2.348)--(1.739,2.349)--(1.739,2.350)--(1.740,2.350)--(1.740,2.351)%
  --(1.740,2.352)--(1.741,2.352)--(1.741,2.353)--(1.741,2.354)--(1.741,2.355)--(1.742,2.355)%
  --(1.742,2.356)--(1.742,2.357)--(1.743,2.358)--(1.743,2.359)--(1.743,2.360)--(1.744,2.360)%
  --(1.744,2.361)--(1.744,2.362)--(1.744,2.363)--(1.745,2.363)--(1.745,2.364)--(1.745,2.365)%
  --(1.746,2.365)--(1.746,2.366)--(1.746,2.367)--(1.746,2.368)--(1.747,2.368)--(1.747,2.369)%
  --(1.747,2.370)--(1.748,2.370)--(1.748,2.371)--(1.748,2.372)--(1.748,2.373)--(1.749,2.373)%
  --(1.749,2.374)--(1.749,2.375)--(1.750,2.375)--(1.750,2.376)--(1.750,2.377)--(1.750,2.378)%
  --(1.751,2.378)--(1.751,2.379)--(1.751,2.380)--(1.751,2.381)--(1.752,2.381)--(1.752,2.382)%
  --(1.752,2.383)--(1.753,2.383)--(1.753,2.384)--(1.753,2.385)--(1.753,2.386)--(1.754,2.386)%
  --(1.754,2.387)--(1.754,2.388)--(1.755,2.388)--(1.755,2.389)--(1.755,2.390)--(1.755,2.391)%
  --(1.756,2.391)--(1.756,2.392)--(1.756,2.393)--(1.757,2.394)--(1.757,2.395)--(1.757,2.396)%
  --(1.758,2.396)--(1.758,2.397)--(1.758,2.398)--(1.759,2.399)--(1.759,2.400)--(1.759,2.401)%
  --(1.760,2.401)--(1.760,2.402)--(1.760,2.403)--(1.760,2.404)--(1.761,2.404)--(1.761,2.405)%
  --(1.761,2.406)--(1.762,2.406)--(1.762,2.407)--(1.762,2.408)--(1.762,2.409)--(1.763,2.409)%
  --(1.763,2.410)--(1.763,2.411)--(1.764,2.411)--(1.764,2.412)--(1.764,2.413)--(1.764,2.414)%
  --(1.765,2.414)--(1.765,2.415)--(1.765,2.416)--(1.766,2.417)--(1.766,2.418)--(1.766,2.419)%
  --(1.767,2.419)--(1.767,2.420)--(1.767,2.421)--(1.767,2.422)--(1.768,2.422)--(1.768,2.423)%
  --(1.768,2.424)--(1.769,2.424)--(1.769,2.425)--(1.769,2.426)--(1.769,2.427)--(1.770,2.427)%
  --(1.770,2.428)--(1.770,2.429)--(1.771,2.429)--(1.771,2.430)--(1.771,2.431)--(1.771,2.432)%
  --(1.772,2.432)--(1.772,2.433)--(1.772,2.434)--(1.773,2.435)--(1.773,2.436)--(1.773,2.437)%
  --(1.774,2.437)--(1.774,2.438)--(1.774,2.439)--(1.775,2.440)--(1.775,2.441)--(1.775,2.442)%
  --(1.776,2.442)--(1.776,2.443)--(1.776,2.444)--(1.776,2.445)--(1.777,2.445)--(1.777,2.446)%
  --(1.777,2.447)--(1.778,2.447)--(1.778,2.448)--(1.778,2.449)--(1.778,2.450)--(1.779,2.450)%
  --(1.779,2.451)--(1.779,2.452)--(1.780,2.452)--(1.780,2.453)--(1.780,2.454)--(1.780,2.455)%
  --(1.781,2.455)--(1.781,2.456)--(1.781,2.457)--(1.781,2.458)--(1.782,2.458)--(1.782,2.459)%
  --(1.782,2.460)--(1.783,2.460)--(1.783,2.461)--(1.783,2.462)--(1.783,2.463)--(1.784,2.463)%
  --(1.784,2.464)--(1.784,2.465)--(1.785,2.465)--(1.785,2.466)--(1.785,2.467)--(1.785,2.468)%
  --(1.786,2.468)--(1.786,2.469)--(1.786,2.470)--(1.787,2.470)--(1.787,2.471)--(1.787,2.472)%
  --(1.787,2.473)--(1.788,2.473)--(1.788,2.474)--(1.788,2.475)--(1.789,2.476)--(1.789,2.477)%
  --(1.789,2.478)--(1.790,2.478)--(1.790,2.479)--(1.790,2.480)--(1.790,2.481)--(1.791,2.481)%
  --(1.791,2.482)--(1.791,2.483)--(1.792,2.483)--(1.792,2.484)--(1.792,2.485)--(1.792,2.486)%
  --(1.793,2.486)--(1.793,2.487)--(1.793,2.488)--(1.794,2.488)--(1.794,2.489)--(1.794,2.490)%
  --(1.794,2.491)--(1.795,2.491)--(1.795,2.492)--(1.795,2.493)--(1.796,2.493)--(1.796,2.494)%
  --(1.796,2.495)--(1.796,2.496)--(1.797,2.496)--(1.797,2.497)--(1.797,2.498)--(1.797,2.499)%
  --(1.798,2.499)--(1.798,2.500)--(1.798,2.501)--(1.799,2.501)--(1.799,2.502)--(1.799,2.503)%
  --(1.799,2.504)--(1.800,2.504)--(1.800,2.505)--(1.800,2.506)--(1.801,2.506)--(1.801,2.507)%
  --(1.801,2.508)--(1.801,2.509)--(1.802,2.509)--(1.802,2.510)--(1.802,2.511)--(1.803,2.511)%
  --(1.803,2.512)--(1.803,2.513)--(1.803,2.514)--(1.804,2.514)--(1.804,2.515)--(1.804,2.516)%
  --(1.805,2.517)--(1.805,2.518)--(1.805,2.519)--(1.806,2.519)--(1.806,2.520)--(1.806,2.521)%
  --(1.806,2.522)--(1.807,2.522)--(1.807,2.523)--(1.807,2.524)--(1.808,2.524)--(1.808,2.525)%
  --(1.808,2.526)--(1.808,2.527)--(1.809,2.527)--(1.809,2.528)--(1.809,2.529)--(1.810,2.529)%
  --(1.810,2.530)--(1.810,2.531)--(1.810,2.532)--(1.811,2.532)--(1.811,2.533)--(1.811,2.534)%
  --(1.812,2.534)--(1.812,2.535)--(1.812,2.536)--(1.812,2.537)--(1.813,2.537)--(1.813,2.538)%
  --(1.813,2.539)--(1.813,2.540)--(1.814,2.540)--(1.814,2.541)--(1.814,2.542)--(1.815,2.542)%
  --(1.815,2.543)--(1.815,2.544)--(1.815,2.545)--(1.816,2.545)--(1.816,2.546)--(1.816,2.547)%
  --(1.817,2.547)--(1.817,2.548)--(1.817,2.549)--(1.817,2.550)--(1.818,2.550)--(1.818,2.551)%
  --(1.818,2.552)--(1.819,2.553)--(1.819,2.554)--(1.819,2.555)--(1.820,2.555)--(1.820,2.556)%
  --(1.820,2.557)--(1.821,2.558)--(1.821,2.559)--(1.821,2.560)--(1.822,2.560)--(1.822,2.561)%
  --(1.822,2.562)--(1.822,2.563)--(1.823,2.563)--(1.823,2.564)--(1.823,2.565)--(1.824,2.565)%
  --(1.824,2.566)--(1.824,2.567)--(1.824,2.568)--(1.825,2.568)--(1.825,2.569)--(1.825,2.570)%
  --(1.826,2.570)--(1.826,2.571)--(1.826,2.572)--(1.826,2.573)--(1.827,2.573)--(1.827,2.574)%
  --(1.827,2.575)--(1.828,2.576)--(1.828,2.577)--(1.828,2.578)--(1.829,2.578)--(1.829,2.579)%
  --(1.829,2.580)--(1.829,2.581)--(1.830,2.581)--(1.830,2.582)--(1.830,2.583)--(1.831,2.583)%
  --(1.831,2.584)--(1.831,2.585)--(1.831,2.586)--(1.832,2.586)--(1.832,2.587)--(1.832,2.588)%
  --(1.833,2.588)--(1.833,2.589)--(1.833,2.590)--(1.833,2.591)--(1.834,2.591)--(1.834,2.592)%
  --(1.834,2.593)--(1.835,2.594)--(1.835,2.595)--(1.835,2.596)--(1.836,2.596)--(1.836,2.597)%
  --(1.836,2.598)--(1.837,2.599)--(1.837,2.600)--(1.837,2.601)--(1.838,2.601)--(1.838,2.602)%
  --(1.838,2.603)--(1.838,2.604)--(1.839,2.604)--(1.839,2.605)--(1.839,2.606)--(1.840,2.606)%
  --(1.840,2.607)--(1.840,2.608)--(1.840,2.609)--(1.841,2.609)--(1.841,2.610)--(1.841,2.611)%
  --(1.842,2.611)--(1.842,2.612)--(1.842,2.613)--(1.842,2.614)--(1.843,2.614)--(1.843,2.615)%
  --(1.843,2.616)--(1.843,2.617)--(1.844,2.617)--(1.844,2.618)--(1.844,2.619)--(1.845,2.619)%
  --(1.845,2.620)--(1.845,2.621)--(1.845,2.622)--(1.846,2.622)--(1.846,2.623)--(1.846,2.624)%
  --(1.847,2.624)--(1.847,2.625)--(1.847,2.626)--(1.847,2.627)--(1.848,2.627)--(1.848,2.628)%
  --(1.848,2.629)--(1.849,2.629)--(1.849,2.630)--(1.849,2.631)--(1.849,2.632)--(1.850,2.632)%
  --(1.850,2.633)--(1.850,2.634)--(1.851,2.635)--(1.851,2.636)--(1.851,2.637)--(1.852,2.637)%
  --(1.852,2.638)--(1.852,2.639)--(1.853,2.640)--(1.853,2.641)--(1.853,2.642)--(1.854,2.642)%
  --(1.854,2.643)--(1.854,2.644)--(1.854,2.645)--(1.855,2.645)--(1.855,2.646)--(1.855,2.647)%
  --(1.856,2.647)--(1.856,2.648)--(1.856,2.649)--(1.856,2.650)--(1.857,2.650)--(1.857,2.651)%
  --(1.857,2.652)--(1.858,2.652)--(1.858,2.653)--(1.858,2.654)--(1.858,2.655)--(1.859,2.655)%
  --(1.859,2.656)--(1.859,2.657)--(1.859,2.658)--(1.860,2.658)--(1.860,2.659)--(1.860,2.660)%
  --(1.861,2.660)--(1.861,2.661)--(1.861,2.662)--(1.861,2.663)--(1.862,2.663)--(1.862,2.664)%
  --(1.862,2.665)--(1.863,2.665)--(1.863,2.666)--(1.863,2.667)--(1.863,2.668)--(1.864,2.668)%
  --(1.864,2.669)--(1.864,2.670)--(1.865,2.670)--(1.865,2.671)--(1.865,2.672)--(1.865,2.673)%
  --(1.866,2.673)--(1.866,2.674)--(1.866,2.675)--(1.867,2.676)--(1.867,2.677)--(1.867,2.678)%
  --(1.868,2.678)--(1.868,2.679)--(1.868,2.680)--(1.868,2.681)--(1.869,2.681)--(1.869,2.682)%
  --(1.869,2.683)--(1.870,2.683)--(1.870,2.684)--(1.870,2.685)--(1.870,2.686)--(1.871,2.686)%
  --(1.871,2.687)--(1.871,2.688)--(1.872,2.688)--(1.872,2.689)--(1.872,2.690)--(1.872,2.691)%
  --(1.873,2.691)--(1.873,2.692)--(1.873,2.693)--(1.874,2.694)--(1.874,2.695)--(1.874,2.696)%
  --(1.875,2.696)--(1.875,2.697)--(1.875,2.698)--(1.875,2.699)--(1.876,2.699)--(1.876,2.700)%
  --(1.876,2.701)--(1.877,2.701)--(1.877,2.702)--(1.877,2.703)--(1.877,2.704)--(1.878,2.704)%
  --(1.878,2.705)--(1.878,2.706)--(1.879,2.706)--(1.879,2.707)--(1.879,2.708)--(1.879,2.709)%
  --(1.880,2.709)--(1.880,2.710)--(1.880,2.711)--(1.881,2.711)--(1.881,2.712)--(1.881,2.713)%
  --(1.881,2.714)--(1.882,2.714)--(1.882,2.715)--(1.882,2.716)--(1.883,2.717)--(1.883,2.718)%
  --(1.883,2.719)--(1.884,2.719)--(1.884,2.720)--(1.884,2.721)--(1.884,2.722)--(1.885,2.722)%
  --(1.885,2.723)--(1.885,2.724)--(1.886,2.724)--(1.886,2.725)--(1.886,2.726)--(1.886,2.727)%
  --(1.887,2.727)--(1.887,2.728)--(1.887,2.729)--(1.888,2.729)--(1.888,2.730)--(1.888,2.731)%
  --(1.888,2.732)--(1.889,2.732)--(1.889,2.733)--(1.889,2.734)--(1.890,2.735)--(1.890,2.736)%
  --(1.890,2.737)--(1.891,2.737)--(1.891,2.738)--(1.891,2.739)--(1.891,2.740)--(1.892,2.740)%
  --(1.892,2.741)--(1.892,2.742)--(1.893,2.742)--(1.893,2.743)--(1.893,2.744)--(1.893,2.745)%
  --(1.894,2.745)--(1.894,2.746)--(1.894,2.747)--(1.895,2.747)--(1.895,2.748)--(1.895,2.749)%
  --(1.895,2.750)--(1.896,2.750)--(1.896,2.751)--(1.896,2.752)--(1.897,2.753)--(1.897,2.754)%
  --(1.897,2.755)--(1.898,2.755)--(1.898,2.756)--(1.898,2.757)--(1.899,2.758)--(1.899,2.759)%
  --(1.899,2.760)--(1.900,2.760)--(1.900,2.761)--(1.900,2.762)--(1.900,2.763)--(1.901,2.763)%
  --(1.901,2.764)--(1.901,2.765)--(1.902,2.765)--(1.902,2.766)--(1.902,2.767)--(1.902,2.768)%
  --(1.903,2.768)--(1.903,2.769)--(1.903,2.770)--(1.904,2.771)--(1.904,2.772)--(1.904,2.773)%
  --(1.905,2.773)--(1.905,2.774)--(1.905,2.775)--(1.906,2.776)--(1.906,2.777)--(1.906,2.778)%
  --(1.907,2.778)--(1.907,2.779)--(1.907,2.780)--(1.907,2.781)--(1.908,2.781)--(1.908,2.782)%
  --(1.908,2.783)--(1.909,2.783)--(1.909,2.784)--(1.909,2.785)--(1.909,2.786)--(1.910,2.786)%
  --(1.910,2.787)--(1.910,2.788)--(1.911,2.788)--(1.911,2.789)--(1.911,2.790)--(1.911,2.791)%
  --(1.912,2.791)--(1.912,2.792)--(1.912,2.793)--(1.913,2.794)--(1.913,2.795)--(1.913,2.796)%
  --(1.914,2.796)--(1.914,2.797)--(1.914,2.798)--(1.915,2.799)--(1.915,2.800)--(1.915,2.801)%
  --(1.916,2.801)--(1.916,2.802)--(1.916,2.803)--(1.916,2.804)--(1.917,2.804)--(1.917,2.805)%
  --(1.917,2.806)--(1.918,2.806)--(1.918,2.807)--(1.918,2.808)--(1.918,2.809)--(1.919,2.809)%
  --(1.919,2.810)--(1.919,2.811)--(1.920,2.812)--(1.920,2.813)--(1.920,2.814)--(1.921,2.814)%
  --(1.921,2.815)--(1.921,2.816)--(1.921,2.817)--(1.922,2.817)--(1.922,2.818)--(1.922,2.819)%
  --(1.923,2.819)--(1.923,2.820)--(1.923,2.821)--(1.923,2.822)--(1.924,2.822)--(1.924,2.823)%
  --(1.924,2.824)--(1.925,2.824)--(1.925,2.825)--(1.925,2.826)--(1.925,2.827)--(1.926,2.827)%
  --(1.926,2.828)--(1.926,2.829)--(1.927,2.829)--(1.927,2.830)--(1.927,2.831)--(1.927,2.832)%
  --(1.928,2.832)--(1.928,2.833)--(1.928,2.834)--(1.929,2.835)--(1.929,2.836)--(1.929,2.837)%
  --(1.930,2.837)--(1.930,2.838)--(1.930,2.839)--(1.931,2.840)--(1.931,2.841)--(1.931,2.842)%
  --(1.932,2.842)--(1.932,2.843)--(1.932,2.844)--(1.932,2.845)--(1.933,2.845)--(1.933,2.846)%
  --(1.933,2.847)--(1.934,2.847)--(1.934,2.848)--(1.934,2.849)--(1.934,2.850)--(1.935,2.850)%
  --(1.935,2.851)--(1.935,2.852)--(1.936,2.853)--(1.936,2.854)--(1.936,2.855)--(1.937,2.855)%
  --(1.937,2.856)--(1.937,2.857)--(1.937,2.858)--(1.938,2.858)--(1.938,2.859)--(1.938,2.860)%
  --(1.939,2.860)--(1.939,2.861)--(1.939,2.862)--(1.939,2.863)--(1.940,2.863)--(1.940,2.864)%
  --(1.940,2.865)--(1.941,2.865)--(1.941,2.866)--(1.941,2.867)--(1.941,2.868)--(1.942,2.868)%
  --(1.942,2.869)--(1.942,2.870)--(1.943,2.870)--(1.943,2.871)--(1.943,2.872)--(1.943,2.873)%
  --(1.944,2.873)--(1.944,2.874)--(1.944,2.875)--(1.945,2.876)--(1.945,2.877)--(1.945,2.878)%
  --(1.946,2.878)--(1.946,2.879)--(1.946,2.880)--(1.946,2.881)--(1.947,2.881)--(1.947,2.882)%
  --(1.947,2.883)--(1.948,2.883)--(1.948,2.884)--(1.948,2.885)--(1.948,2.886)--(1.949,2.886)%
  --(1.949,2.887)--(1.949,2.888)--(1.950,2.889)--(1.950,2.890)--(1.950,2.891)--(1.951,2.891)%
  --(1.951,2.892)--(1.951,2.893)--(1.952,2.894)--(1.952,2.895)--(1.952,2.896)--(1.953,2.896)%
  --(1.953,2.897)--(1.953,2.898)--(1.953,2.899)--(1.954,2.899)--(1.954,2.900)--(1.954,2.901)%
  --(1.955,2.901)--(1.955,2.902)--(1.955,2.903)--(1.955,2.904)--(1.956,2.904)--(1.956,2.905)%
  --(1.956,2.906)--(1.957,2.906)--(1.957,2.907)--(1.957,2.908)--(1.957,2.909)--(1.958,2.909)%
  --(1.958,2.910)--(1.958,2.911)--(1.959,2.911)--(1.959,2.912)--(1.959,2.913)--(1.959,2.914)%
  --(1.960,2.914)--(1.960,2.915)--(1.960,2.916)--(1.961,2.917)--(1.961,2.918)--(1.961,2.919)%
  --(1.962,2.919)--(1.962,2.920)--(1.962,2.921)--(1.962,2.922)--(1.963,2.922)--(1.963,2.923)%
  --(1.963,2.924)--(1.964,2.924)--(1.964,2.925)--(1.964,2.926)--(1.964,2.927)--(1.965,2.927)%
  --(1.965,2.928)--(1.965,2.929)--(1.966,2.930)--(1.966,2.931)--(1.966,2.932)--(1.967,2.932)%
  --(1.967,2.933)--(1.967,2.934)--(1.968,2.935)--(1.968,2.936)--(1.968,2.937)--(1.969,2.937)%
  --(1.969,2.938)--(1.969,2.939)--(1.969,2.940)--(1.970,2.940)--(1.970,2.941)--(1.970,2.942)%
  --(1.971,2.942)--(1.971,2.943)--(1.971,2.944)--(1.971,2.945)--(1.972,2.945)--(1.972,2.946)%
  --(1.972,2.947)--(1.973,2.947)--(1.973,2.948)--(1.973,2.949)--(1.973,2.950)--(1.974,2.950)%
  --(1.974,2.951)--(1.974,2.952)--(1.975,2.953)--(1.975,2.954)--(1.975,2.955)--(1.976,2.955)%
  --(1.976,2.956)--(1.976,2.957)--(1.977,2.958)--(1.977,2.959)--(1.977,2.960)--(1.978,2.960)%
  --(1.978,2.961)--(1.978,2.962)--(1.978,2.963)--(1.979,2.963)--(1.979,2.964)--(1.979,2.965)%
  --(1.980,2.965)--(1.980,2.966)--(1.980,2.967)--(1.980,2.968)--(1.981,2.968)--(1.981,2.969)%
  --(1.981,2.970)--(1.982,2.971)--(1.982,2.972)--(1.982,2.973)--(1.983,2.973)--(1.983,2.974)%
  --(1.983,2.975)--(1.983,2.976)--(1.984,2.976)--(1.984,2.977)--(1.984,2.978)--(1.985,2.978)%
  --(1.985,2.979)--(1.985,2.980)--(1.985,2.981)--(1.986,2.981)--(1.986,2.982)--(1.986,2.983)%
  --(1.987,2.983)--(1.987,2.984)--(1.987,2.985)--(1.987,2.986)--(1.988,2.986)--(1.988,2.987)%
  --(1.988,2.988)--(1.989,2.988)--(1.989,2.989)--(1.989,2.990)--(1.989,2.991)--(1.990,2.991)%
  --(1.990,2.992)--(1.990,2.993)--(1.991,2.994)--(1.991,2.995)--(1.991,2.996)--(1.992,2.996)%
  --(1.992,2.997)--(1.992,2.998)--(1.993,2.999)--(1.993,3.000)--(1.993,3.001)--(1.994,3.001)%
  --(1.994,3.002)--(1.994,3.003)--(1.994,3.004)--(1.995,3.004)--(1.995,3.005)--(1.995,3.006)%
  --(1.996,3.006)--(1.996,3.007)--(1.996,3.008)--(1.996,3.009)--(1.997,3.009)--(1.997,3.010)%
  --(1.997,3.011)--(1.998,3.012)--(1.998,3.013)--(1.998,3.014)--(1.999,3.014)--(1.999,3.015)%
  --(1.999,3.016)--(1.999,3.017)--(2.000,3.017)--(2.000,3.018)--(2.000,3.019)--(2.001,3.019)%
  --(2.001,3.020)--(2.001,3.021)--(2.001,3.022)--(2.002,3.022)--(2.002,3.023)--(2.002,3.024)%
  --(2.003,3.024)--(2.003,3.025)--(2.003,3.026)--(2.003,3.027)--(2.004,3.027)--(2.004,3.028)%
  --(2.004,3.029)--(2.005,3.029)--(2.005,3.030)--(2.005,3.031)--(2.005,3.032)--(2.006,3.032)%
  --(2.006,3.033)--(2.006,3.034)--(2.007,3.035)--(2.007,3.036)--(2.007,3.037)--(2.008,3.037)%
  --(2.008,3.038)--(2.008,3.039)--(2.008,3.040)--(2.009,3.040)--(2.009,3.041)--(2.009,3.042)%
  --(2.010,3.042)--(2.010,3.043)--(2.010,3.044)--(2.010,3.045)--(2.011,3.045)--(2.011,3.046)%
  --(2.011,3.047)--(2.012,3.047)--(2.012,3.048)--(2.012,3.049)--(2.012,3.050)--(2.013,3.050)%
  --(2.013,3.051)--(2.013,3.052)--(2.014,3.053)--(2.014,3.054)--(2.014,3.055)--(2.015,3.055)%
  --(2.015,3.056)--(2.015,3.057)--(2.015,3.058)--(2.016,3.058)--(2.016,3.059)--(2.016,3.060)%
  --(2.017,3.060)--(2.017,3.061)--(2.017,3.062)--(2.017,3.063)--(2.018,3.063)--(2.018,3.064)%
  --(2.018,3.065)--(2.019,3.065)--(2.019,3.066)--(2.019,3.067)--(2.019,3.068)--(2.020,3.068)%
  --(2.020,3.069)--(2.020,3.070)--(2.021,3.070)--(2.021,3.071)--(2.021,3.072)--(2.021,3.073)%
  --(2.022,3.073)--(2.022,3.074)--(2.022,3.075)--(2.023,3.076)--(2.023,3.077)--(2.023,3.078)%
  --(2.024,3.078)--(2.024,3.079)--(2.024,3.080)--(2.024,3.081)--(2.025,3.081)--(2.025,3.082)%
  --(2.025,3.083)--(2.026,3.083)--(2.026,3.084)--(2.026,3.085)--(2.026,3.086)--(2.027,3.086)%
  --(2.027,3.087)--(2.027,3.088)--(2.028,3.089)--(2.028,3.090)--(2.028,3.091)--(2.029,3.091)%
  --(2.029,3.092)--(2.029,3.093)--(2.030,3.094)--(2.030,3.095)--(2.030,3.096)--(2.031,3.096)%
  --(2.031,3.097)--(2.031,3.098)--(2.031,3.099)--(2.032,3.099)--(2.032,3.100)--(2.032,3.101)%
  --(2.033,3.101)--(2.033,3.102)--(2.033,3.103)--(2.033,3.104)--(2.034,3.104)--(2.034,3.105)%
  --(2.034,3.106)--(2.035,3.106)--(2.035,3.107)--(2.035,3.108)--(2.035,3.109)--(2.036,3.109)%
  --(2.036,3.110)--(2.036,3.111)--(2.037,3.112)--(2.037,3.113)--(2.037,3.114)--(2.038,3.114)%
  --(2.038,3.115)--(2.038,3.116)--(2.038,3.117)--(2.039,3.117)--(2.039,3.118)--(2.039,3.119)%
  --(2.040,3.119)--(2.040,3.120)--(2.040,3.121)--(2.040,3.122)--(2.041,3.122)--(2.041,3.123)%
  --(2.041,3.124)--(2.042,3.124)--(2.042,3.125)--(2.042,3.126)--(2.042,3.127)--(2.043,3.127)%
  --(2.043,3.128)--(2.043,3.129)--(2.044,3.130)--(2.044,3.131)--(2.044,3.132)--(2.045,3.132)%
  --(2.045,3.133)--(2.045,3.134)--(2.046,3.135)--(2.046,3.136)--(2.046,3.137)--(2.047,3.137)%
  --(2.047,3.138)--(2.047,3.139)--(2.047,3.140)--(2.048,3.140)--(2.048,3.141)--(2.048,3.142)%
  --(2.049,3.142)--(2.049,3.143)--(2.049,3.144)--(2.049,3.145)--(2.050,3.145)--(2.050,3.146)%
  --(2.050,3.147)--(2.051,3.147)--(2.051,3.148)--(2.051,3.149)--(2.051,3.150)--(2.052,3.150)%
  --(2.052,3.151)--(2.052,3.152)--(2.053,3.153)--(2.053,3.154)--(2.053,3.155)--(2.054,3.155)%
  --(2.054,3.156)--(2.054,3.157)--(2.054,3.158)--(2.055,3.158)--(2.055,3.159)--(2.055,3.160)%
  --(2.056,3.160)--(2.056,3.161)--(2.056,3.162)--(2.056,3.163)--(2.057,3.163)--(2.057,3.164)%
  --(2.057,3.165)--(2.058,3.165)--(2.058,3.166)--(2.058,3.167)--(2.058,3.168)--(2.059,3.168)%
  --(2.059,3.169)--(2.059,3.170)--(2.060,3.171)--(2.060,3.172)--(2.060,3.173)--(2.061,3.173)%
  --(2.061,3.174)--(2.061,3.175)--(2.061,3.176)--(2.062,3.176)--(2.062,3.177)--(2.062,3.178)%
  --(2.063,3.178)--(2.063,3.179)--(2.063,3.180)--(2.063,3.181)--(2.064,3.181)--(2.064,3.182)%
  --(2.064,3.183)--(2.065,3.183)--(2.065,3.184)--(2.065,3.185)--(2.065,3.186)--(2.066,3.186)%
  --(2.066,3.187)--(2.066,3.188)--(2.067,3.188)--(2.067,3.189)--(2.067,3.190)--(2.067,3.191)%
  --(2.068,3.191)--(2.068,3.192)--(2.068,3.193)--(2.069,3.194)--(2.069,3.195)--(2.069,3.196)%
  --(2.070,3.196)--(2.070,3.197)--(2.070,3.198)--(2.070,3.199)--(2.071,3.199)--(2.071,3.200)%
  --(2.071,3.201)--(2.072,3.201)--(2.072,3.202)--(2.072,3.203)--(2.072,3.204)--(2.073,3.204)%
  --(2.073,3.205)--(2.073,3.206)--(2.074,3.206)--(2.074,3.207)--(2.074,3.208)--(2.074,3.209)%
  --(2.075,3.209)--(2.075,3.210)--(2.075,3.211)--(2.076,3.212)--(2.076,3.213)--(2.076,3.214)%
  --(2.077,3.214)--(2.077,3.215)--(2.077,3.216)--(2.077,3.217)--(2.078,3.217)--(2.078,3.218)%
  --(2.078,3.219)--(2.079,3.219)--(2.079,3.220)--(2.079,3.221)--(2.079,3.222)--(2.080,3.222)%
  --(2.080,3.223)--(2.080,3.224)--(2.081,3.224)--(2.081,3.225)--(2.081,3.226)--(2.081,3.227)%
  --(2.082,3.227)--(2.082,3.228)--(2.082,3.229)--(2.083,3.229)--(2.083,3.230)--(2.083,3.231)%
  --(2.083,3.232)--(2.084,3.232)--(2.084,3.233)--(2.084,3.234)--(2.084,3.235)--(2.085,3.235)%
  --(2.085,3.236)--(2.085,3.237)--(2.086,3.237)--(2.086,3.238)--(2.086,3.239)--(2.086,3.240)%
  --(2.087,3.240)--(2.087,3.241)--(2.087,3.242)--(2.088,3.242)--(2.088,3.243)--(2.088,3.244)%
  --(2.088,3.245)--(2.089,3.245)--(2.089,3.246)--(2.089,3.247)--(2.090,3.248)--(2.090,3.249)%
  --(2.090,3.250)--(2.091,3.250)--(2.091,3.251)--(2.091,3.252)--(2.092,3.253)--(2.092,3.254)%
  --(2.092,3.255)--(2.093,3.255)--(2.093,3.256)--(2.093,3.257)--(2.093,3.258)--(2.094,3.258)%
  --(2.094,3.259)--(2.094,3.260)--(2.095,3.260)--(2.095,3.261)--(2.095,3.262)--(2.095,3.263)%
  --(2.096,3.263)--(2.096,3.264)--(2.096,3.265)--(2.097,3.265)--(2.097,3.266)--(2.097,3.267)%
  --(2.097,3.268)--(2.098,3.268)--(2.098,3.269)--(2.098,3.270)--(2.099,3.270)--(2.099,3.271)%
  --(2.099,3.272)--(2.099,3.273)--(2.100,3.273)--(2.100,3.274)--(2.100,3.275)--(2.100,3.276)%
  --(2.101,3.276)--(2.101,3.277)--(2.101,3.278)--(2.102,3.278)--(2.102,3.279)--(2.102,3.280)%
  --(2.102,3.281)--(2.103,3.281)--(2.103,3.282)--(2.103,3.283)--(2.104,3.283)--(2.104,3.284)%
  --(2.104,3.285)--(2.104,3.286)--(2.105,3.286)--(2.105,3.287)--(2.105,3.288)--(2.106,3.289)%
  --(2.106,3.290)--(2.106,3.291)--(2.107,3.291)--(2.107,3.292)--(2.107,3.293)--(2.108,3.294)%
  --(2.108,3.295)--(2.108,3.296)--(2.109,3.296)--(2.109,3.297)--(2.109,3.298)--(2.109,3.299)%
  --(2.110,3.299)--(2.110,3.300)--(2.110,3.301)--(2.111,3.301)--(2.111,3.302)--(2.111,3.303)%
  --(2.111,3.304)--(2.112,3.304)--(2.112,3.305)--(2.112,3.306)--(2.113,3.306)--(2.113,3.307)%
  --(2.113,3.308)--(2.113,3.309)--(2.114,3.309)--(2.114,3.310)--(2.114,3.311)--(2.114,3.312)%
  --(2.115,3.312)--(2.115,3.313)--(2.115,3.314)--(2.116,3.314)--(2.116,3.315)--(2.116,3.316)%
  --(2.116,3.317)--(2.117,3.317)--(2.117,3.318)--(2.117,3.319)--(2.118,3.319)--(2.118,3.320)%
  --(2.118,3.321)--(2.118,3.322)--(2.119,3.322)--(2.119,3.323)--(2.119,3.324)--(2.120,3.324)%
  --(2.120,3.325)--(2.120,3.326)--(2.120,3.327)--(2.121,3.327)--(2.121,3.328)--(2.121,3.329)%
  --(2.122,3.330)--(2.122,3.331)--(2.122,3.332)--(2.123,3.332)--(2.123,3.333)--(2.123,3.334)%
  --(2.124,3.335)--(2.124,3.336)--(2.124,3.337)--(2.125,3.337)--(2.125,3.338)--(2.125,3.339)%
  --(2.125,3.340)--(2.126,3.340)--(2.126,3.341)--(2.126,3.342)--(2.127,3.342)--(2.127,3.343)%
  --(2.127,3.344)--(2.127,3.345)--(2.128,3.345)--(2.128,3.346)--(2.128,3.347)--(2.129,3.347)%
  --(2.129,3.348)--(2.129,3.349)--(2.129,3.350)--(2.130,3.350)--(2.130,3.351)--(2.130,3.352)%
  --(2.130,3.353)--(2.131,3.353)--(2.131,3.354)--(2.131,3.355)--(2.132,3.355)--(2.132,3.356)%
  --(2.132,3.357)--(2.132,3.358)--(2.133,3.358)--(2.133,3.359)--(2.133,3.360)--(2.134,3.360)%
  --(2.134,3.361)--(2.134,3.362)--(2.134,3.363)--(2.135,3.363)--(2.135,3.364)--(2.135,3.365)%
  --(2.136,3.365)--(2.136,3.366)--(2.136,3.367)--(2.136,3.368)--(2.137,3.368)--(2.137,3.369)%
  --(2.137,3.370)--(2.138,3.371)--(2.138,3.372)--(2.138,3.373)--(2.139,3.373)--(2.139,3.374)%
  --(2.139,3.375)--(2.139,3.376)--(2.140,3.376)--(2.140,3.377)--(2.140,3.378)--(2.141,3.378)%
  --(2.141,3.379)--(2.141,3.380)--(2.141,3.381)--(2.142,3.381)--(2.142,3.382)--(2.142,3.383)%
  --(2.143,3.383)--(2.143,3.384)--(2.143,3.385)--(2.143,3.386)--(2.144,3.386)--(2.144,3.387)%
  --(2.144,3.388)--(2.145,3.388)--(2.145,3.389)--(2.145,3.390)--(2.145,3.391)--(2.146,3.391)%
  --(2.146,3.392)--(2.146,3.393)--(2.146,3.394)--(2.147,3.394)--(2.147,3.395)--(2.147,3.396)%
  --(2.148,3.396)--(2.148,3.397)--(2.148,3.398)--(2.148,3.399)--(2.149,3.399)--(2.149,3.400)%
  --(2.149,3.401)--(2.150,3.401)--(2.150,3.402)--(2.150,3.403)--(2.150,3.404)--(2.151,3.404)%
  --(2.151,3.405)--(2.151,3.406)--(2.152,3.406)--(2.152,3.407)--(2.152,3.408)--(2.152,3.409)%
  --(2.153,3.409)--(2.153,3.410)--(2.153,3.411)--(2.154,3.412)--(2.154,3.413)--(2.154,3.414)%
  --(2.155,3.414)--(2.155,3.415)--(2.155,3.416)--(2.155,3.417)--(2.156,3.417)--(2.156,3.418)%
  --(2.156,3.419)--(2.157,3.419)--(2.157,3.420)--(2.157,3.421)--(2.157,3.422)--(2.158,3.422)%
  --(2.158,3.423)--(2.158,3.424)--(2.159,3.424)--(2.159,3.425)--(2.159,3.426)--(2.159,3.427)%
  --(2.160,3.427)--(2.160,3.428)--(2.160,3.429)--(2.161,3.430)--(2.161,3.431)--(2.161,3.432)%
  --(2.162,3.432)--(2.162,3.433)--(2.162,3.434)--(2.162,3.435)--(2.163,3.435)--(2.163,3.436)%
  --(2.163,3.437)--(2.164,3.437)--(2.164,3.438)--(2.164,3.439)--(2.164,3.440)--(2.165,3.440)%
  --(2.165,3.441)--(2.165,3.442)--(2.166,3.442)--(2.166,3.443)--(2.166,3.444)--(2.166,3.445)%
  --(2.167,3.445)--(2.167,3.446)--(2.167,3.447)--(2.168,3.448)--(2.168,3.449)--(2.168,3.450)%
  --(2.169,3.450)--(2.169,3.451)--(2.169,3.452)--(2.170,3.453)--(2.170,3.454)--(2.170,3.455)%
  --(2.171,3.455)--(2.171,3.456)--(2.171,3.457)--(2.171,3.458)--(2.172,3.458)--(2.172,3.459)%
  --(2.172,3.460)--(2.173,3.460)--(2.173,3.461)--(2.173,3.462)--(2.173,3.463)--(2.174,3.463)%
  --(2.174,3.464)--(2.174,3.465)--(2.175,3.465)--(2.175,3.466)--(2.175,3.467)--(2.175,3.468)%
  --(2.176,3.468)--(2.176,3.469)--(2.176,3.470)--(2.177,3.471)--(2.177,3.472)--(2.177,3.473)%
  --(2.178,3.473)--(2.178,3.474)--(2.178,3.475)--(2.178,3.476)--(2.179,3.476)--(2.179,3.477)%
  --(2.179,3.478)--(2.180,3.478)--(2.180,3.479)--(2.180,3.480)--(2.180,3.481)--(2.181,3.481)%
  --(2.181,3.482)--(2.181,3.483)--(2.182,3.483)--(2.182,3.484)--(2.182,3.485)--(2.182,3.486)%
  --(2.183,3.486)--(2.183,3.487)--(2.183,3.488)--(2.184,3.489)--(2.184,3.490)--(2.184,3.491)%
  --(2.185,3.491)--(2.185,3.492)--(2.185,3.493)--(2.186,3.494)--(2.186,3.495)--(2.186,3.496)%
  --(2.187,3.496)--(2.187,3.497)--(2.187,3.498)--(2.187,3.499)--(2.188,3.499)--(2.188,3.500)%
  --(2.188,3.501)--(2.189,3.501)--(2.189,3.502)--(2.189,3.503)--(2.189,3.504)--(2.190,3.504)%
  --(2.190,3.505)--(2.190,3.506)--(2.191,3.506)--(2.191,3.507)--(2.191,3.508)--(2.191,3.509)%
  --(2.192,3.509)--(2.192,3.510)--(2.192,3.511)--(2.192,3.512)--(2.193,3.512)--(2.193,3.513)%
  --(2.193,3.514)--(2.194,3.514)--(2.194,3.515)--(2.194,3.516)--(2.194,3.517)--(2.195,3.517)%
  --(2.195,3.518)--(2.195,3.519)--(2.196,3.519)--(2.196,3.520)--(2.196,3.521)--(2.196,3.522)%
  --(2.197,3.522)--(2.197,3.523)--(2.197,3.524)--(2.198,3.524)--(2.198,3.525)--(2.198,3.526)%
  --(2.198,3.527)--(2.199,3.527)--(2.199,3.528)--(2.199,3.529)--(2.200,3.530)--(2.200,3.531)%
  --(2.200,3.532)--(2.201,3.532)--(2.201,3.533)--(2.201,3.534)--(2.202,3.535)--(2.202,3.536)%
  --(2.202,3.537)--(2.203,3.537)--(2.203,3.538)--(2.203,3.539)--(2.203,3.540)--(2.204,3.540)%
  --(2.204,3.541)--(2.204,3.542)--(2.205,3.542)--(2.205,3.543)--(2.205,3.544)--(2.205,3.545)%
  --(2.206,3.545)--(2.206,3.546)--(2.206,3.547)--(2.207,3.548)--(2.207,3.549)--(2.207,3.550)%
  --(2.208,3.550)--(2.208,3.551)--(2.208,3.552)--(2.208,3.553)--(2.209,3.553)--(2.209,3.554)%
  --(2.209,3.555)--(2.210,3.555)--(2.210,3.556)--(2.210,3.557)--(2.210,3.558)--(2.211,3.558)%
  --(2.211,3.559)--(2.211,3.560)--(2.212,3.560)--(2.212,3.561)--(2.212,3.562)--(2.212,3.563)%
  --(2.213,3.563)--(2.213,3.564)--(2.213,3.565)--(2.214,3.565)--(2.214,3.566)--(2.214,3.567)%
  --(2.214,3.568)--(2.215,3.568)--(2.215,3.569)--(2.215,3.570)--(2.216,3.571)--(2.216,3.572)%
  --(2.216,3.573)--(2.217,3.573)--(2.217,3.574)--(2.217,3.575)--(2.217,3.576)--(2.218,3.576)%
  --(2.218,3.577)--(2.218,3.578)--(2.219,3.578)--(2.219,3.579)--(2.219,3.580)--(2.219,3.581)%
  --(2.220,3.581)--(2.220,3.582)--(2.220,3.583)--(2.221,3.583)--(2.221,3.584)--(2.221,3.585)%
  --(2.221,3.586)--(2.222,3.586)--(2.222,3.587)--(2.222,3.588)--(2.223,3.589)--(2.223,3.590)%
  --(2.223,3.591)--(2.224,3.591)--(2.224,3.592)--(2.224,3.593)--(2.224,3.594)--(2.225,3.594)%
  --(2.225,3.595)--(2.225,3.596)--(2.226,3.596)--(2.226,3.597)--(2.226,3.598)--(2.226,3.599)%
  --(2.227,3.599)--(2.227,3.600)--(2.227,3.601)--(2.228,3.601)--(2.228,3.602)--(2.228,3.603)%
  --(2.228,3.604)--(2.229,3.604)--(2.229,3.605)--(2.229,3.606)--(2.230,3.606)--(2.230,3.607)%
  --(2.230,3.608)--(2.230,3.609)--(2.231,3.609)--(2.231,3.610)--(2.231,3.611)--(2.232,3.612)%
  --(2.232,3.613)--(2.232,3.614)--(2.233,3.614)--(2.233,3.615)--(2.233,3.616)--(2.233,3.617)%
  --(2.234,3.617)--(2.234,3.618)--(2.234,3.619)--(2.235,3.619)--(2.235,3.620)--(2.235,3.621)%
  --(2.235,3.622)--(2.236,3.622)--(2.236,3.623)--(2.236,3.624)--(2.237,3.625)--(2.237,3.626)%
  --(2.237,3.627)--(2.238,3.627)--(2.238,3.628)--(2.238,3.629)--(2.239,3.630)--(2.239,3.631)%
  --(2.239,3.632)--(2.240,3.632)--(2.240,3.633)--(2.240,3.634)--(2.240,3.635)--(2.241,3.635)%
  --(2.241,3.636)--(2.241,3.637)--(2.242,3.637)--(2.242,3.638)--(2.242,3.639)--(2.242,3.640)%
  --(2.243,3.640)--(2.243,3.641)--(2.243,3.642)--(2.244,3.642)--(2.244,3.643)--(2.244,3.644)%
  --(2.244,3.645)--(2.245,3.645)--(2.245,3.646)--(2.245,3.647)--(2.246,3.648)--(2.246,3.649)%
  --(2.246,3.650)--(2.247,3.650)--(2.247,3.651)--(2.247,3.652)--(2.248,3.653)--(2.248,3.654)%
  --(2.248,3.655)--(2.249,3.655)--(2.249,3.656)--(2.249,3.657)--(2.249,3.658)--(2.250,3.658)%
  --(2.250,3.659)--(2.250,3.660)--(2.251,3.660)--(2.251,3.661)--(2.251,3.662)--(2.251,3.663)%
  --(2.252,3.663)--(2.252,3.664)--(2.252,3.665)--(2.253,3.666)--(2.253,3.667)--(2.253,3.668)%
  --(2.254,3.668)--(2.254,3.669)--(2.254,3.670)--(2.254,3.671)--(2.255,3.671)--(2.255,3.672)%
  --(2.255,3.673)--(2.256,3.673)--(2.256,3.674)--(2.256,3.675)--(2.256,3.676)--(2.257,3.676)%
  --(2.257,3.677)--(2.257,3.678)--(2.258,3.678)--(2.258,3.679)--(2.258,3.680)--(2.258,3.681)%
  --(2.259,3.681)--(2.259,3.682)--(2.259,3.683)--(2.260,3.683)--(2.260,3.684)--(2.260,3.685)%
  --(2.260,3.686)--(2.261,3.686)--(2.261,3.687)--(2.261,3.688)--(2.262,3.689)--(2.262,3.690)%
  --(2.262,3.691)--(2.263,3.691)--(2.263,3.692)--(2.263,3.693)--(2.264,3.694)--(2.264,3.695)%
  --(2.264,3.696)--(2.265,3.696)--(2.265,3.697)--(2.265,3.698)--(2.265,3.699)--(2.266,3.699)%
  --(2.266,3.700)--(2.266,3.701)--(2.267,3.701)--(2.267,3.702)--(2.267,3.703)--(2.267,3.704)%
  --(2.268,3.704)--(2.268,3.705)--(2.268,3.706)--(2.269,3.707)--(2.269,3.708)--(2.269,3.709)%
  --(2.270,3.709)--(2.270,3.710)--(2.270,3.711)--(2.270,3.712)--(2.271,3.712)--(2.271,3.713)%
  --(2.271,3.714)--(2.272,3.714)--(2.272,3.715)--(2.272,3.716)--(2.272,3.717)--(2.273,3.717)%
  --(2.273,3.718)--(2.273,3.719)--(2.274,3.719)--(2.274,3.720)--(2.274,3.721)--(2.274,3.722)%
  --(2.275,3.722)--(2.275,3.723)--(2.275,3.724)--(2.276,3.724)--(2.276,3.725)--(2.276,3.726)%
  --(2.276,3.727)--(2.277,3.727)--(2.277,3.728)--(2.277,3.729)--(2.278,3.730)--(2.278,3.731)%
  --(2.278,3.732)--(2.279,3.732)--(2.279,3.733)--(2.279,3.734)--(2.279,3.735)--(2.280,3.735)%
  --(2.280,3.736)--(2.280,3.737)--(2.281,3.737)--(2.281,3.738)--(2.281,3.739)--(2.281,3.740)%
  --(2.282,3.740)--(2.282,3.741)--(2.282,3.742)--(2.283,3.742)--(2.283,3.743)--(2.283,3.744)%
  --(2.283,3.745)--(2.284,3.745)--(2.284,3.746)--(2.284,3.747)--(2.285,3.748)--(2.285,3.749)%
  --(2.285,3.750)--(2.286,3.750)--(2.286,3.751)--(2.286,3.752)--(2.286,3.753)--(2.287,3.753)%
  --(2.287,3.754)--(2.287,3.755)--(2.288,3.755)--(2.288,3.756)--(2.288,3.757)--(2.288,3.758)%
  --(2.289,3.758)--(2.289,3.759)--(2.289,3.760)--(2.290,3.760)--(2.290,3.761)--(2.290,3.762)%
  --(2.290,3.763)--(2.291,3.763)--(2.291,3.764)--(2.291,3.765)--(2.292,3.765)--(2.292,3.766)%
  --(2.292,3.767)--(2.292,3.768)--(2.293,3.768)--(2.293,3.769)--(2.293,3.770)--(2.294,3.771)%
  --(2.294,3.772)--(2.294,3.773)--(2.295,3.773)--(2.295,3.774)--(2.295,3.775)--(2.295,3.776)%
  --(2.296,3.776)--(2.296,3.777)--(2.296,3.778)--(2.297,3.778)--(2.297,3.779)--(2.297,3.780)%
  --(2.297,3.781)--(2.298,3.781)--(2.298,3.782)--(2.298,3.783)--(2.299,3.784)--(2.299,3.785)%
  --(2.299,3.786)--(2.300,3.786)--(2.300,3.787)--(2.300,3.788)--(2.301,3.789)--(2.301,3.790)%
  --(2.301,3.791)--(2.302,3.791)--(2.302,3.792)--(2.302,3.793)--(2.302,3.794)--(2.303,3.794)%
  --(2.303,3.795)--(2.303,3.796)--(2.304,3.796)--(2.304,3.797)--(2.304,3.798)--(2.304,3.799)%
  --(2.305,3.799)--(2.305,3.800)--(2.305,3.801)--(2.306,3.801)--(2.306,3.802)--(2.306,3.803)%
  --(2.306,3.804)--(2.307,3.804)--(2.307,3.805)--(2.307,3.806)--(2.308,3.807)--(2.308,3.808)%
  --(2.308,3.809)--(2.309,3.809)--(2.309,3.810)--(2.309,3.811)--(2.310,3.812)--(2.310,3.813)%
  --(2.310,3.814)--(2.311,3.814)--(2.311,3.815)--(2.311,3.816)--(2.311,3.817)--(2.312,3.817)%
  --(2.312,3.818)--(2.312,3.819)--(2.313,3.819)--(2.313,3.820)--(2.313,3.821)--(2.313,3.822)%
  --(2.314,3.822)--(2.314,3.823)--(2.314,3.824)--(2.315,3.825)--(2.315,3.826)--(2.315,3.827)%
  --(2.316,3.827)--(2.316,3.828)--(2.316,3.829)--(2.317,3.830)--(2.317,3.831)--(2.317,3.832)%
  --(2.318,3.832)--(2.318,3.833)--(2.318,3.834)--(2.318,3.835)--(2.319,3.835)--(2.319,3.836)%
  --(2.319,3.837)--(2.320,3.837)--(2.320,3.838)--(2.320,3.839)--(2.320,3.840)--(2.321,3.840)%
  --(2.321,3.841)--(2.321,3.842)--(2.322,3.842)--(2.322,3.843)--(2.322,3.844)--(2.322,3.845)%
  --(2.323,3.845)--(2.323,3.846)--(2.323,3.847)--(2.324,3.848)--(2.324,3.849)--(2.324,3.850)%
  --(2.325,3.850)--(2.325,3.851)--(2.325,3.852)--(2.326,3.853)--(2.326,3.854)--(2.326,3.855)%
  --(2.327,3.855)--(2.327,3.856)--(2.327,3.857)--(2.327,3.858)--(2.328,3.858)--(2.328,3.859)%
  --(2.328,3.860)--(2.329,3.860)--(2.329,3.861)--(2.329,3.862)--(2.329,3.863)--(2.330,3.863)%
  --(2.330,3.864)--(2.330,3.865)--(2.331,3.866)--(2.331,3.867)--(2.331,3.868)--(2.332,3.868)%
  --(2.332,3.869)--(2.332,3.870)--(2.332,3.871)--(2.333,3.871)--(2.333,3.872)--(2.333,3.873)%
  --(2.334,3.873)--(2.334,3.874)--(2.334,3.875)--(2.334,3.876)--(2.335,3.876)--(2.335,3.877)%
  --(2.335,3.878)--(2.336,3.878)--(2.336,3.879)--(2.336,3.880)--(2.336,3.881)--(2.337,3.881)%
  --(2.337,3.882)--(2.337,3.883)--(2.338,3.883)--(2.338,3.884)--(2.338,3.885)--(2.338,3.886)%
  --(2.339,3.886)--(2.339,3.887)--(2.339,3.888)--(2.340,3.889)--(2.340,3.890)--(2.340,3.891)%
  --(2.341,3.891)--(2.341,3.892)--(2.341,3.893)--(2.341,3.894)--(2.342,3.894)--(2.342,3.895)%
  --(2.342,3.896)--(2.343,3.896)--(2.343,3.897)--(2.343,3.898)--(2.343,3.899)--(2.344,3.899)%
  --(2.344,3.900)--(2.344,3.901)--(2.345,3.901)--(2.345,3.902)--(2.345,3.903)--(2.345,3.904)%
  --(2.346,3.904)--(2.346,3.905)--(2.346,3.906)--(2.347,3.907)--(2.347,3.908)--(2.347,3.909)%
  --(2.348,3.909)--(2.348,3.910)--(2.348,3.911)--(2.348,3.912)--(2.349,3.912)--(2.349,3.913)%
  --(2.349,3.914)--(2.350,3.914)--(2.350,3.915)--(2.350,3.916)--(2.350,3.917)--(2.351,3.917)%
  --(2.351,3.918)--(2.351,3.919)--(2.352,3.919)--(2.352,3.920)--(2.352,3.921)--(2.352,3.922)%
  --(2.353,3.922)--(2.353,3.923)--(2.353,3.924)--(2.354,3.924)--(2.354,3.925)--(2.354,3.926)%
  --(2.354,3.927)--(2.355,3.927)--(2.355,3.928)--(2.355,3.929)--(2.356,3.930)--(2.356,3.931)%
  --(2.356,3.932)--(2.357,3.932)--(2.357,3.933)--(2.357,3.934)--(2.357,3.935)--(2.358,3.935)%
  --(2.358,3.936)--(2.358,3.937)--(2.359,3.937)--(2.359,3.938)--(2.359,3.939)--(2.359,3.940)%
  --(2.360,3.940)--(2.360,3.941)--(2.360,3.942)--(2.361,3.943)--(2.361,3.944)--(2.361,3.945)%
  --(2.362,3.945)--(2.362,3.946)--(2.362,3.947)--(2.363,3.948)--(2.363,3.949)--(2.363,3.950)%
  --(2.364,3.950)--(2.364,3.951)--(2.364,3.952)--(2.364,3.953)--(2.365,3.953)--(2.365,3.954)%
  --(2.365,3.955)--(2.366,3.955)--(2.366,3.956)--(2.366,3.957)--(2.366,3.958)--(2.367,3.958)%
  --(2.367,3.959)--(2.367,3.960)--(2.368,3.960)--(2.368,3.961)--(2.368,3.962)--(2.368,3.963)%
  --(2.369,3.963)--(2.369,3.964)--(2.369,3.965)--(2.370,3.965)--(2.370,3.966)--(2.370,3.967)%
  --(2.370,3.968)--(2.371,3.968)--(2.371,3.969)--(2.371,3.970)--(2.371,3.971)--(2.372,3.971)%
  --(2.372,3.972)--(2.372,3.973)--(2.373,3.973)--(2.373,3.974)--(2.373,3.975)--(2.373,3.976)%
  --(2.374,3.976)--(2.374,3.977)--(2.374,3.978)--(2.375,3.978)--(2.375,3.979)--(2.375,3.980)%
  --(2.375,3.981)--(2.376,3.981)--(2.376,3.982)--(2.376,3.983)--(2.377,3.984)--(2.377,3.985)%
  --(2.377,3.986)--(2.378,3.986)--(2.378,3.987)--(2.378,3.988)--(2.379,3.989)--(2.379,3.990)%
  --(2.379,3.991)--(2.380,3.991)--(2.380,3.992)--(2.380,3.993)--(2.380,3.994)--(2.381,3.994)%
  --(2.381,3.995)--(2.381,3.996)--(2.382,3.996)--(2.382,3.997)--(2.382,3.998)--(2.382,3.999)%
  --(2.383,3.999)--(2.383,4.000)--(2.383,4.001)--(2.384,4.001)--(2.384,4.002)--(2.384,4.003)%
  --(2.384,4.004)--(2.385,4.004)--(2.385,4.005)--(2.385,4.006)--(2.386,4.007)--(2.386,4.008)%
  --(2.386,4.009)--(2.387,4.009)--(2.387,4.010)--(2.387,4.011)--(2.387,4.012)--(2.388,4.012)%
  --(2.388,4.013)--(2.388,4.014)--(2.389,4.014)--(2.389,4.015)--(2.389,4.016)--(2.389,4.017)%
  --(2.390,4.017)--(2.390,4.018)--(2.390,4.019)--(2.391,4.019)--(2.391,4.020)--(2.391,4.021)%
  --(2.391,4.022)--(2.392,4.022)--(2.392,4.023)--(2.392,4.024)--(2.393,4.025)--(2.393,4.026)%
  --(2.393,4.027)--(2.394,4.027)--(2.394,4.028)--(2.394,4.029)--(2.395,4.030)--(2.395,4.031)%
  --(2.395,4.032)--(2.396,4.032)--(2.396,4.033)--(2.396,4.034)--(2.396,4.035)--(2.397,4.035)%
  --(2.397,4.036)--(2.397,4.037)--(2.398,4.037)--(2.398,4.038)--(2.398,4.039)--(2.398,4.040)%
  --(2.399,4.040)--(2.399,4.041)--(2.399,4.042)--(2.400,4.042)--(2.400,4.043)--(2.400,4.044)%
  --(2.400,4.045)--(2.401,4.045)--(2.401,4.046)--(2.401,4.047)--(2.402,4.048)--(2.402,4.049)%
  --(2.402,4.050)--(2.403,4.050)--(2.403,4.051)--(2.403,4.052)--(2.403,4.053)--(2.404,4.053)%
  --(2.404,4.054)--(2.404,4.055)--(2.405,4.055)--(2.405,4.056)--(2.405,4.057)--(2.405,4.058)%
  --(2.406,4.058)--(2.406,4.059)--(2.406,4.060)--(2.407,4.060)--(2.407,4.061)--(2.407,4.062)%
  --(2.407,4.063)--(2.408,4.063)--(2.408,4.064)--(2.408,4.065)--(2.409,4.066)--(2.409,4.067)%
  --(2.409,4.068)--(2.410,4.068)--(2.410,4.069)--(2.410,4.070)--(2.410,4.071)--(2.411,4.071)%
  --(2.411,4.072)--(2.411,4.073)--(2.412,4.073)--(2.412,4.074)--(2.412,4.075)--(2.412,4.076)%
  --(2.413,4.076)--(2.413,4.077)--(2.413,4.078)--(2.414,4.078)--(2.414,4.079)--(2.414,4.080)%
  --(2.414,4.081)--(2.415,4.081)--(2.415,4.082)--(2.415,4.083)--(2.416,4.083)--(2.416,4.084)%
  --(2.416,4.085)--(2.416,4.086)--(2.417,4.086)--(2.417,4.087)--(2.417,4.088)--(2.417,4.089)%
  --(2.418,4.089)--(2.418,4.090)--(2.418,4.091)--(2.419,4.091)--(2.419,4.092)--(2.419,4.093)%
  --(2.419,4.094)--(2.420,4.094)--(2.420,4.095)--(2.420,4.096)--(2.421,4.096)--(2.421,4.097)%
  --(2.421,4.098)--(2.421,4.099)--(2.422,4.099)--(2.422,4.100)--(2.422,4.101)--(2.423,4.101)%
  --(2.423,4.102)--(2.423,4.103)--(2.423,4.104)--(2.424,4.104)--(2.424,4.105)--(2.424,4.106)%
  --(2.425,4.107)--(2.425,4.108)--(2.425,4.109)--(2.426,4.109)--(2.426,4.110)--(2.426,4.111)%
  --(2.426,4.112)--(2.427,4.112)--(2.427,4.113)--(2.427,4.114)--(2.428,4.114)--(2.428,4.115)%
  --(2.428,4.116)--(2.428,4.117)--(2.429,4.117)--(2.429,4.118)--(2.429,4.119)--(2.430,4.119)%
  --(2.430,4.120)--(2.430,4.121)--(2.430,4.122)--(2.431,4.122)--(2.431,4.123)--(2.431,4.124)%
  --(2.432,4.124)--(2.432,4.125)--(2.432,4.126)--(2.432,4.127)--(2.433,4.127)--(2.433,4.128)%
  --(2.433,4.129)--(2.433,4.130)--(2.434,4.130)--(2.434,4.131)--(2.434,4.132)--(2.435,4.132)%
  --(2.435,4.133)--(2.435,4.134)--(2.435,4.135)--(2.436,4.135)--(2.436,4.136)--(2.436,4.137)%
  --(2.437,4.137)--(2.437,4.138)--(2.437,4.139)--(2.437,4.140)--(2.438,4.140)--(2.438,4.141)%
  --(2.438,4.142)--(2.439,4.143)--(2.439,4.144)--(2.439,4.145)--(2.440,4.145)--(2.440,4.146)%
  --(2.440,4.147)--(2.441,4.148)--(2.441,4.149)--(2.441,4.150)--(2.442,4.150)--(2.442,4.151)%
  --(2.442,4.152)--(2.442,4.153)--(2.443,4.153)--(2.443,4.154)--(2.443,4.155)--(2.444,4.155)%
  --(2.444,4.156)--(2.444,4.157)--(2.444,4.158)--(2.445,4.158)--(2.445,4.159)--(2.445,4.160)%
  --(2.446,4.160)--(2.446,4.161)--(2.446,4.162)--(2.446,4.163)--(2.447,4.163)--(2.447,4.164)%
  --(2.447,4.165)--(2.448,4.166)--(2.448,4.167)--(2.448,4.168)--(2.449,4.168)--(2.449,4.169)%
  --(2.449,4.170)--(2.449,4.171)--(2.450,4.171)--(2.450,4.172)--(2.450,4.173)--(2.451,4.173)%
  --(2.451,4.174)--(2.451,4.175)--(2.451,4.176)--(2.452,4.176)--(2.452,4.177)--(2.452,4.178)%
  --(2.453,4.178)--(2.453,4.179)--(2.453,4.180)--(2.453,4.181)--(2.454,4.181)--(2.454,4.182)%
  --(2.454,4.183)--(2.455,4.184)--(2.455,4.185)--(2.455,4.186)--(2.456,4.186)--(2.456,4.187)%
  --(2.456,4.188)--(2.457,4.189)--(2.457,4.190)--(2.457,4.191)--(2.458,4.191)--(2.458,4.192)%
  --(2.458,4.193)--(2.458,4.194)--(2.459,4.194)--(2.459,4.195)--(2.459,4.196)--(2.460,4.196)%
  --(2.460,4.197)--(2.460,4.198)--(2.460,4.199)--(2.461,4.199)--(2.461,4.200)--(2.461,4.201)%
  --(2.462,4.201)--(2.462,4.202)--(2.462,4.203)--(2.462,4.204)--(2.463,4.204)--(2.463,4.205)%
  --(2.463,4.206)--(2.463,4.207)--(2.464,4.207)--(2.464,4.208)--(2.464,4.209)--(2.465,4.209)%
  --(2.465,4.210)--(2.465,4.211)--(2.465,4.212)--(2.466,4.212)--(2.466,4.213)--(2.466,4.214)%
  --(2.467,4.214)--(2.467,4.215)--(2.467,4.216)--(2.467,4.217)--(2.468,4.217)--(2.468,4.218)%
  --(2.468,4.219)--(2.469,4.219)--(2.469,4.220)--(2.469,4.221)--(2.469,4.222)--(2.470,4.222)%
  --(2.470,4.223)--(2.470,4.224)--(2.471,4.225)--(2.471,4.226)--(2.471,4.227)--(2.472,4.227)%
  --(2.472,4.228)--(2.472,4.229)--(2.473,4.230)--(2.473,4.231)--(2.473,4.232)--(2.474,4.232)%
  --(2.474,4.233)--(2.474,4.234)--(2.474,4.235)--(2.475,4.235)--(2.475,4.236)--(2.475,4.237)%
  --(2.476,4.237)--(2.476,4.238)--(2.476,4.239)--(2.476,4.240)--(2.477,4.240)--(2.477,4.241)%
  --(2.477,4.242)--(2.478,4.242)--(2.478,4.243)--(2.478,4.244)--(2.478,4.245)--(2.479,4.245)%
  --(2.479,4.246)--(2.479,4.247)--(2.479,4.248)--(2.480,4.248)--(2.480,4.249)--(2.480,4.250)%
  --(2.481,4.250)--(2.481,4.251)--(2.481,4.252)--(2.481,4.253)--(2.482,4.253)--(2.482,4.254)%
  --(2.482,4.255)--(2.483,4.255)--(2.483,4.256)--(2.483,4.257)--(2.483,4.258)--(2.484,4.258)%
  --(2.484,4.259)--(2.484,4.260)--(2.485,4.260)--(2.485,4.261)--(2.485,4.262)--(2.485,4.263)%
  --(2.486,4.263)--(2.486,4.264)--(2.486,4.265)--(2.487,4.266)--(2.487,4.267)--(2.487,4.268)%
  --(2.488,4.268)--(2.488,4.269)--(2.488,4.270)--(2.488,4.271)--(2.489,4.271)--(2.489,4.272)%
  --(2.489,4.273)--(2.490,4.273)--(2.490,4.274)--(2.490,4.275)--(2.490,4.276)--(2.491,4.276)%
  --(2.491,4.277)--(2.491,4.278)--(2.492,4.278)--(2.492,4.279)--(2.492,4.280)--(2.492,4.281)%
  --(2.493,4.281)--(2.493,4.282)--(2.493,4.283)--(2.494,4.284)--(2.494,4.285)--(2.494,4.286)%
  --(2.495,4.286)--(2.495,4.287)--(2.495,4.288)--(2.495,4.289)--(2.496,4.289)--(2.496,4.290)%
  --(2.496,4.291)--(2.497,4.291)--(2.497,4.292)--(2.497,4.293)--(2.497,4.294)--(2.498,4.294)%
  --(2.498,4.295)--(2.498,4.296)--(2.499,4.296)--(2.499,4.297)--(2.499,4.298)--(2.499,4.299)%
  --(2.500,4.299)--(2.500,4.300)--(2.500,4.301)--(2.501,4.301)--(2.501,4.302)--(2.501,4.303)%
  --(2.501,4.304)--(2.502,4.304)--(2.502,4.305)--(2.502,4.306)--(2.503,4.307)--(2.503,4.308)%
  --(2.503,4.309)--(2.504,4.309)--(2.504,4.310)--(2.504,4.311)--(2.504,4.312)--(2.505,4.312)%
  --(2.505,4.313)--(2.505,4.314)--(2.506,4.314)--(2.506,4.315)--(2.506,4.316)--(2.506,4.317)%
  --(2.507,4.317)--(2.507,4.318)--(2.507,4.319)--(2.508,4.319)--(2.508,4.320)--(2.508,4.321)%
  --(2.508,4.322)--(2.509,4.322)--(2.509,4.323)--(2.509,4.324)--(2.510,4.325)--(2.510,4.326)%
  --(2.510,4.327)--(2.511,4.327)--(2.511,4.328)--(2.511,4.329)--(2.511,4.330)--(2.512,4.330)%
  --(2.512,4.331)--(2.512,4.332)--(2.513,4.332)--(2.513,4.333)--(2.513,4.334)--(2.513,4.335)%
  --(2.514,4.335)--(2.514,4.336)--(2.514,4.337)--(2.515,4.337)--(2.515,4.338)--(2.515,4.339)%
  --(2.515,4.340)--(2.516,4.340)--(2.516,4.341)--(2.516,4.342)--(2.517,4.343)--(2.517,4.344)%
  --(2.517,4.345)--(2.518,4.345)--(2.518,4.346)--(2.518,4.347)--(2.519,4.348)--(2.519,4.349)%
  --(2.519,4.350)--(2.520,4.350)--(2.520,4.351)--(2.520,4.352)--(2.520,4.353)--(2.521,4.353)%
  --(2.521,4.354)--(2.521,4.355)--(2.522,4.355)--(2.522,4.356)--(2.522,4.357)--(2.522,4.358)%
  --(2.523,4.358)--(2.523,4.359)--(2.523,4.360)--(2.524,4.361)--(2.524,4.362)--(2.524,4.363)%
  --(2.525,4.363)--(2.525,4.364)--(2.525,4.365)--(2.525,4.366)--(2.526,4.366)--(2.526,4.367)%
  --(2.526,4.368)--(2.527,4.368)--(2.527,4.369)--(2.527,4.370)--(2.527,4.371)--(2.528,4.371)%
  --(2.528,4.372)--(2.528,4.373)--(2.529,4.373)--(2.529,4.374)--(2.529,4.375)--(2.529,4.376)%
  --(2.530,4.376)--(2.530,4.377)--(2.530,4.378)--(2.531,4.378)--(2.531,4.379)--(2.531,4.380)%
  --(2.531,4.381)--(2.532,4.381)--(2.532,4.382)--(2.532,4.383)--(2.533,4.384)--(2.533,4.385)%
  --(2.533,4.386)--(2.534,4.386)--(2.534,4.387)--(2.534,4.388)--(2.535,4.389)--(2.535,4.390)%
  --(2.535,4.391)--(2.536,4.391)--(2.536,4.392)--(2.536,4.393)--(2.536,4.394)--(2.537,4.394)%
  --(2.537,4.395)--(2.537,4.396)--(2.538,4.396)--(2.538,4.397)--(2.538,4.398)--(2.538,4.399)%
  --(2.539,4.399)--(2.539,4.400)--(2.539,4.401)--(2.540,4.402)--(2.540,4.403)--(2.540,4.404)%
  --(2.541,4.404)--(2.541,4.405)--(2.541,4.406)--(2.541,4.407)--(2.542,4.407)--(2.542,4.408)%
  --(2.542,4.409)--(2.543,4.409)--(2.543,4.410)--(2.543,4.411)--(2.543,4.412)--(2.544,4.412)%
  --(2.544,4.413)--(2.544,4.414)--(2.545,4.414)--(2.545,4.415)--(2.545,4.416)--(2.545,4.417)%
  --(2.546,4.417)--(2.546,4.418)--(2.546,4.419)--(2.547,4.419)--(2.547,4.420)--(2.547,4.421)%
  --(2.547,4.422)--(2.548,4.422)--(2.548,4.423)--(2.548,4.424)--(2.549,4.425)--(2.549,4.426)%
  --(2.549,4.427)--(2.550,4.427)--(2.550,4.428)--(2.550,4.429)--(2.550,4.430)--(2.551,4.430)%
  --(2.551,4.431)--(2.551,4.432)--(2.552,4.432)--(2.552,4.433)--(2.552,4.434)--(2.552,4.435)%
  --(2.553,4.435)--(2.553,4.436)--(2.553,4.437)--(2.554,4.437)--(2.554,4.438)--(2.554,4.439)%
  --(2.554,4.440)--(2.555,4.440)--(2.555,4.441)--(2.555,4.442)--(2.556,4.443)--(2.556,4.444)%
  --(2.556,4.445)--(2.557,4.445)--(2.557,4.446)--(2.557,4.447)--(2.557,4.448)--(2.558,4.448)%
  --(2.558,4.449)--(2.558,4.450)--(2.559,4.450)--(2.559,4.451)--(2.559,4.452)--(2.559,4.453)%
  --(2.560,4.453)--(2.560,4.454)--(2.560,4.455)--(2.561,4.455)--(2.561,4.456)--(2.561,4.457)%
  --(2.561,4.458)--(2.562,4.458)--(2.562,4.459)--(2.562,4.460)--(2.563,4.460)--(2.563,4.461)%
  --(2.563,4.462)--(2.563,4.463)--(2.564,4.463)--(2.564,4.464)--(2.564,4.465)--(2.565,4.466)%
  --(2.565,4.467)--(2.565,4.468)--(2.566,4.468)--(2.566,4.469)--(2.566,4.470)--(2.566,4.471)%
  --(2.567,4.471)--(2.567,4.472)--(2.567,4.473)--(2.568,4.473)--(2.568,4.474)--(2.568,4.475)%
  --(2.568,4.476)--(2.569,4.476)--(2.569,4.477)--(2.569,4.478)--(2.570,4.479)--(2.570,4.480)%
  --(2.570,4.481)--(2.571,4.481)--(2.571,4.482)--(2.571,4.483)--(2.572,4.484)--(2.572,4.485)%
  --(2.572,4.486)--(2.573,4.486)--(2.573,4.487)--(2.573,4.488)--(2.573,4.489)--(2.574,4.489)%
  --(2.574,4.490)--(2.574,4.491)--(2.575,4.491)--(2.575,4.492)--(2.575,4.493)--(2.575,4.494)%
  --(2.576,4.494)--(2.576,4.495)--(2.576,4.496)--(2.577,4.496)--(2.577,4.497)--(2.577,4.498)%
  --(2.577,4.499)--(2.578,4.499)--(2.578,4.500)--(2.578,4.501)--(2.579,4.502)--(2.579,4.503)%
  --(2.579,4.504)--(2.580,4.504)--(2.580,4.505)--(2.580,4.506)--(2.581,4.507)--(2.581,4.508)%
  --(2.581,4.509)--(2.582,4.509)--(2.582,4.510)--(2.582,4.511)--(2.582,4.512)--(2.583,4.512)%
  --(2.583,4.513)--(2.583,4.514)--(2.584,4.514)--(2.584,4.515)--(2.584,4.516)--(2.584,4.517)%
  --(2.585,4.517)--(2.585,4.518)--(2.585,4.519)--(2.586,4.520)--(2.586,4.521)--(2.586,4.522)%
  --(2.587,4.522)--(2.587,4.523)--(2.587,4.524)--(2.588,4.525)--(2.588,4.526)--(2.588,4.527)%
  --(2.589,4.527)--(2.589,4.528)--(2.589,4.529)--(2.589,4.530)--(2.590,4.530)--(2.590,4.531)%
  --(2.590,4.532)--(2.591,4.532)--(2.591,4.533)--(2.591,4.534)--(2.591,4.535)--(2.592,4.535)%
  --(2.592,4.536)--(2.592,4.537)--(2.593,4.537)--(2.593,4.538)--(2.593,4.539)--(2.593,4.540)%
  --(2.594,4.540)--(2.594,4.541)--(2.594,4.542)--(2.595,4.543)--(2.595,4.544)--(2.595,4.545)%
  --(2.596,4.545)--(2.596,4.546)--(2.596,4.547)--(2.597,4.548)--(2.597,4.549)--(2.597,4.550)%
  --(2.598,4.550)--(2.598,4.551)--(2.598,4.552)--(2.598,4.553)--(2.599,4.553)--(2.599,4.554)%
  --(2.599,4.555)--(2.600,4.555)--(2.600,4.556)--(2.600,4.557)--(2.600,4.558)--(2.601,4.558)%
  --(2.601,4.559)--(2.601,4.560)--(2.602,4.561)--(2.602,4.562)--(2.602,4.563)--(2.603,4.563)%
  --(2.603,4.564)--(2.603,4.565)--(2.603,4.566)--(2.604,4.566)--(2.604,4.567)--(2.604,4.568)%
  --(2.605,4.568)--(2.605,4.569)--(2.605,4.570)--(2.605,4.571)--(2.606,4.571)--(2.606,4.572)%
  --(2.606,4.573)--(2.607,4.573)--(2.607,4.574)--(2.607,4.575)--(2.607,4.576)--(2.608,4.576)%
  --(2.608,4.577)--(2.608,4.578)--(2.609,4.578)--(2.609,4.579)--(2.609,4.580)--(2.609,4.581)%
  --(2.610,4.581)--(2.610,4.582)--(2.610,4.583)--(2.611,4.584)--(2.611,4.585)--(2.611,4.586)%
  --(2.612,4.586)--(2.612,4.587)--(2.612,4.588)--(2.613,4.589)--(2.613,4.590)--(2.613,4.591)%
  --(2.614,4.591)--(2.614,4.592)--(2.614,4.593)--(2.614,4.594)--(2.615,4.594)--(2.615,4.595)%
  --(2.615,4.596)--(2.616,4.596)--(2.616,4.597)--(2.616,4.598)--(2.616,4.599)--(2.617,4.599)%
  --(2.617,4.600)--(2.617,4.601)--(2.618,4.602)--(2.618,4.603)--(2.618,4.604)--(2.619,4.604)%
  --(2.619,4.605)--(2.619,4.606)--(2.619,4.607)--(2.620,4.607)--(2.620,4.608)--(2.620,4.609)%
  --(2.621,4.609)--(2.621,4.610)--(2.621,4.611)--(2.621,4.612)--(2.622,4.612)--(2.622,4.613)%
  --(2.622,4.614)--(2.623,4.614)--(2.623,4.615)--(2.623,4.616)--(2.623,4.617)--(2.624,4.617)%
  --(2.624,4.618)--(2.624,4.619)--(2.625,4.619)--(2.625,4.620)--(2.625,4.621)--(2.625,4.622)%
  --(2.626,4.622)--(2.626,4.623)--(2.626,4.624)--(2.627,4.625)--(2.627,4.626)--(2.627,4.627)%
  --(2.628,4.627)--(2.628,4.628)--(2.628,4.629)--(2.628,4.630)--(2.629,4.630)--(2.629,4.631)%
  --(2.629,4.632)--(2.630,4.632)--(2.630,4.633)--(2.630,4.634)--(2.630,4.635)--(2.631,4.635)%
  --(2.631,4.636)--(2.631,4.637)--(2.632,4.638)--(2.632,4.639)--(2.632,4.640)--(2.633,4.640)%
  --(2.633,4.641)--(2.633,4.642)--(2.634,4.643)--(2.634,4.644)--(2.634,4.645)--(2.635,4.645)%
  --(2.635,4.646)--(2.635,4.647)--(2.635,4.648)--(2.636,4.648)--(2.636,4.649)--(2.636,4.650)%
  --(2.637,4.650)--(2.637,4.651)--(2.637,4.652)--(2.637,4.653)--(2.638,4.653)--(2.638,4.654)%
  --(2.638,4.655)--(2.639,4.655)--(2.639,4.656)--(2.639,4.657)--(2.639,4.658)--(2.640,4.658)%
  --(2.640,4.659)--(2.640,4.660)--(2.641,4.660)--(2.641,4.661)--(2.641,4.662)--(2.641,4.663)%
  --(2.642,4.663)--(2.642,4.664)--(2.642,4.665)--(2.643,4.666)--(2.643,4.667)--(2.643,4.668)%
  --(2.644,4.668)--(2.644,4.669)--(2.644,4.670)--(2.644,4.671)--(2.645,4.671)--(2.645,4.672)%
  --(2.645,4.673)--(2.646,4.673)--(2.646,4.674)--(2.646,4.675)--(2.646,4.676)--(2.647,4.676)%
  --(2.647,4.677)--(2.647,4.678)--(2.648,4.679)--(2.648,4.680)--(2.648,4.681)--(2.649,4.681)%
  --(2.649,4.682)--(2.649,4.683)--(2.650,4.684)--(2.650,4.685)--(2.650,4.686)--(2.651,4.686)%
  --(2.651,4.687)--(2.651,4.688)--(2.651,4.689)--(2.652,4.689)--(2.652,4.690)--(2.652,4.691)%
  --(2.653,4.691)--(2.653,4.692)--(2.653,4.693)--(2.653,4.694)--(2.654,4.694)--(2.654,4.695)%
  --(2.654,4.696)--(2.655,4.696)--(2.655,4.697)--(2.655,4.698)--(2.655,4.699)--(2.656,4.699)%
  --(2.656,4.700)--(2.656,4.701)--(2.657,4.702)--(2.657,4.703)--(2.657,4.704)--(2.658,4.704)%
  --(2.658,4.705)--(2.658,4.706)--(2.659,4.707)--(2.659,4.708)--(2.659,4.709)--(2.660,4.709)%
  --(2.660,4.710)--(2.660,4.711)--(2.660,4.712)--(2.661,4.712)--(2.661,4.713)--(2.661,4.714)%
  --(2.662,4.714)--(2.662,4.715)--(2.662,4.716)--(2.662,4.717)--(2.663,4.717)--(2.663,4.718)%
  --(2.663,4.719)--(2.664,4.720)--(2.664,4.721)--(2.664,4.722)--(2.665,4.722)--(2.665,4.723)%
  --(2.665,4.724)--(2.666,4.725)--(2.666,4.726)--(2.666,4.727)--(2.667,4.727)--(2.667,4.728)%
  --(2.667,4.729)--(2.667,4.730)--(2.668,4.730)--(2.668,4.731)--(2.668,4.732)--(2.669,4.732)%
  --(2.669,4.733)--(2.669,4.734)--(2.669,4.735)--(2.670,4.735)--(2.670,4.736)--(2.670,4.737)%
  --(2.671,4.737)--(2.671,4.738)--(2.671,4.739)--(2.671,4.740)--(2.672,4.740)--(2.672,4.741)%
  --(2.672,4.742)--(2.673,4.743)--(2.673,4.744)--(2.673,4.745)--(2.674,4.745)--(2.674,4.746)%
  --(2.674,4.747)--(2.674,4.748)--(2.675,4.748)--(2.675,4.749)--(2.675,4.750)--(2.676,4.750)%
  --(2.676,4.751)--(2.676,4.752)--(2.676,4.753)--(2.677,4.753)--(2.677,4.754)--(2.677,4.755)%
  --(2.678,4.755)--(2.678,4.756)--(2.678,4.757)--(2.678,4.758)--(2.679,4.758)--(2.679,4.759)%
  --(2.679,4.760)--(2.680,4.761)--(2.680,4.762)--(2.680,4.763)--(2.681,4.763)--(2.681,4.764)%
  --(2.681,4.765)--(2.681,4.766)--(2.682,4.766)--(2.682,4.767)--(2.682,4.768)--(2.683,4.768)%
  --(2.683,4.769)--(2.683,4.770)--(2.683,4.771)--(2.684,4.771)--(2.684,4.772)--(2.684,4.773)%
  --(2.685,4.773)--(2.685,4.774)--(2.685,4.775)--(2.685,4.776)--(2.686,4.776)--(2.686,4.777)%
  --(2.686,4.778)--(2.687,4.778)--(2.687,4.779)--(2.687,4.780)--(2.687,4.781)--(2.688,4.781)%
  --(2.688,4.782)--(2.688,4.783)--(2.689,4.784)--(2.689,4.785)--(2.689,4.786)--(2.690,4.786)%
  --(2.690,4.787)--(2.690,4.788)--(2.690,4.789)--(2.691,4.789)--(2.691,4.790)--(2.691,4.791)%
  --(2.692,4.791)--(2.692,4.792)--(2.692,4.793)--(2.692,4.794)--(2.693,4.794)--(2.693,4.795)%
  --(2.693,4.796)--(2.694,4.796)--(2.694,4.797)--(2.694,4.798)--(2.694,4.799)--(2.695,4.799)%
  --(2.695,4.800)--(2.695,4.801)--(2.696,4.802)--(2.696,4.803)--(2.696,4.804)--(2.697,4.804)%
  --(2.697,4.805)--(2.697,4.806)--(2.697,4.807)--(2.698,4.807)--(2.698,4.808)--(2.698,4.809)%
  --(2.699,4.809)--(2.699,4.810)--(2.699,4.811)--(2.699,4.812)--(2.700,4.812)--(2.700,4.813)%
  --(2.700,4.814)--(2.701,4.814)--(2.701,4.815)--(2.701,4.816)--(2.701,4.817)--(2.702,4.817)%
  --(2.702,4.818)--(2.702,4.819)--(2.703,4.819)--(2.703,4.820)--(2.703,4.821)--(2.703,4.822)%
  --(2.704,4.822)--(2.704,4.823)--(2.704,4.824)--(2.704,4.825)--(2.705,4.825)--(2.705,4.826)%
  --(2.705,4.827)--(2.706,4.827)--(2.706,4.828)--(2.706,4.829)--(2.706,4.830)--(2.707,4.830)%
  --(2.707,4.831)--(2.707,4.832)--(2.708,4.832)--(2.708,4.833)--(2.708,4.834)--(2.708,4.835)%
  --(2.709,4.835)--(2.709,4.836)--(2.709,4.837)--(2.710,4.838)--(2.710,4.839)--(2.710,4.840)%
  --(2.711,4.840)--(2.711,4.841)--(2.711,4.842)--(2.712,4.843)--(2.712,4.844)--(2.712,4.845)%
  --(2.713,4.845)--(2.713,4.846)--(2.713,4.847)--(2.713,4.848)--(2.714,4.848)--(2.714,4.849)%
  --(2.714,4.850)--(2.715,4.850)--(2.715,4.851)--(2.715,4.852)--(2.715,4.853)--(2.716,4.853)%
  --(2.716,4.854)--(2.716,4.855)--(2.717,4.855)--(2.717,4.856)--(2.717,4.857)--(2.717,4.858)%
  --(2.718,4.858)--(2.718,4.859)--(2.718,4.860)--(2.719,4.860)--(2.719,4.861)--(2.719,4.862)%
  --(2.719,4.863)--(2.720,4.863)--(2.720,4.864)--(2.720,4.865)--(2.720,4.866)--(2.721,4.866)%
  --(2.721,4.867)--(2.721,4.868)--(2.722,4.868)--(2.722,4.869)--(2.722,4.870)--(2.722,4.871)%
  --(2.723,4.871)--(2.723,4.872)--(2.723,4.873)--(2.724,4.873)--(2.724,4.874)--(2.724,4.875)%
  --(2.724,4.876)--(2.725,4.876)--(2.725,4.877)--(2.725,4.878)--(2.726,4.879)--(2.726,4.880)%
  --(2.726,4.881)--(2.727,4.881)--(2.727,4.882)--(2.727,4.883)--(2.728,4.884)--(2.728,4.885)%
  --(2.728,4.886)--(2.729,4.886)--(2.729,4.887)--(2.729,4.888)--(2.729,4.889)--(2.730,4.889)%
  --(2.730,4.890)--(2.730,4.891)--(2.731,4.891)--(2.731,4.892)--(2.731,4.893)--(2.731,4.894)%
  --(2.732,4.894)--(2.732,4.895)--(2.732,4.896)--(2.733,4.896)--(2.733,4.897)--(2.733,4.898)%
  --(2.733,4.899)--(2.734,4.899)--(2.734,4.900)--(2.734,4.901)--(2.735,4.902)--(2.735,4.903)%
  --(2.735,4.904)--(2.736,4.904)--(2.736,4.905)--(2.736,4.906)--(2.736,4.907)--(2.737,4.907)%
  --(2.737,4.908)--(2.737,4.909)--(2.738,4.909)--(2.738,4.910)--(2.738,4.911)--(2.738,4.912)%
  --(2.739,4.912)--(2.739,4.913)--(2.739,4.914)--(2.740,4.914)--(2.740,4.915)--(2.740,4.916)%
  --(2.740,4.917)--(2.741,4.917)--(2.741,4.918)--(2.741,4.919)--(2.742,4.920)--(2.742,4.921)%
  --(2.742,4.922)--(2.743,4.922)--(2.743,4.923)--(2.743,4.924)--(2.744,4.925)--(2.744,4.926)%
  --(2.744,4.927)--(2.745,4.927)--(2.745,4.928)--(2.745,4.929)--(2.745,4.930)--(2.746,4.930)%
  --(2.746,4.931)--(2.746,4.932)--(2.747,4.932)--(2.747,4.933)--(2.747,4.934)--(2.747,4.935)%
  --(2.748,4.935)--(2.748,4.936)--(2.748,4.937)--(2.749,4.937)--(2.749,4.938)--(2.749,4.939)%
  --(2.749,4.940)--(2.750,4.940)--(2.750,4.941)--(2.750,4.942)--(2.750,4.943)--(2.751,4.943)%
  --(2.751,4.944)--(2.751,4.945)--(2.752,4.945)--(2.752,4.946)--(2.752,4.947)--(2.752,4.948)%
  --(2.753,4.948)--(2.753,4.949)--(2.753,4.950)--(2.754,4.950)--(2.754,4.951)--(2.754,4.952)%
  --(2.754,4.953)--(2.755,4.953)--(2.755,4.954)--(2.755,4.955)--(2.756,4.955)--(2.756,4.956)%
  --(2.756,4.957)--(2.756,4.958)--(2.757,4.958)--(2.757,4.959)--(2.757,4.960)--(2.758,4.961)%
  --(2.758,4.962)--(2.758,4.963)--(2.759,4.963)--(2.759,4.964)--(2.759,4.965)--(2.759,4.966)%
  --(2.760,4.966)--(2.760,4.967)--(2.760,4.968)--(2.761,4.968)--(2.761,4.969)--(2.761,4.970)%
  --(2.761,4.971)--(2.762,4.971)--(2.762,4.972)--(2.762,4.973)--(2.763,4.973)--(2.763,4.974)%
  --(2.763,4.975)--(2.763,4.976)--(2.764,4.976)--(2.764,4.977)--(2.764,4.978)--(2.765,4.978)%
  --(2.765,4.979)--(2.765,4.980)--(2.765,4.981)--(2.766,4.981)--(2.766,4.982)--(2.766,4.983)%
  --(2.766,4.984)--(2.767,4.984)--(2.767,4.985)--(2.767,4.986)--(2.768,4.986)--(2.768,4.987)%
  --(2.768,4.988)--(2.768,4.989)--(2.769,4.989)--(2.769,4.990)--(2.769,4.991)--(2.770,4.991)%
  --(2.770,4.992)--(2.770,4.993)--(2.770,4.994)--(2.771,4.994)--(2.771,4.995)--(2.771,4.996)%
  --(2.772,4.996)--(2.772,4.997)--(2.772,4.998)--(2.772,4.999)--(2.773,4.999)--(2.773,5.000)%
  --(2.773,5.001)--(2.774,5.002)--(2.774,5.003)--(2.774,5.004)--(2.775,5.004)--(2.775,5.005)%
  --(2.775,5.006)--(2.775,5.007)--(2.776,5.007)--(2.776,5.008)--(2.776,5.009)--(2.777,5.009)%
  --(2.777,5.010)--(2.777,5.011)--(2.777,5.012)--(2.778,5.012)--(2.778,5.013)--(2.778,5.014)%
  --(2.779,5.014)--(2.779,5.015)--(2.779,5.016)--(2.779,5.017)--(2.780,5.017)--(2.780,5.018)%
  --(2.780,5.019)--(2.781,5.020)--(2.781,5.021)--(2.781,5.022)--(2.782,5.022)--(2.782,5.023)%
  --(2.782,5.024)--(2.782,5.025)--(2.783,5.025)--(2.783,5.026)--(2.783,5.027)--(2.784,5.027)%
  --(2.784,5.028)--(2.784,5.029)--(2.784,5.030)--(2.785,5.030)--(2.785,5.031)--(2.785,5.032)%
  --(2.786,5.032)--(2.786,5.033)--(2.786,5.034)--(2.786,5.035)--(2.787,5.035)--(2.787,5.036)%
  --(2.787,5.037)--(2.788,5.038)--(2.788,5.039)--(2.788,5.040)--(2.789,5.040)--(2.789,5.041)%
  --(2.789,5.042)--(2.790,5.043)--(2.790,5.044)--(2.790,5.045)--(2.791,5.045)--(2.791,5.046)%
  --(2.791,5.047)--(2.791,5.048)--(2.792,5.048)--(2.792,5.049)--(2.792,5.050)--(2.793,5.050)%
  --(2.793,5.051)--(2.793,5.052)--(2.793,5.053)--(2.794,5.053)--(2.794,5.054)--(2.794,5.055)%
  --(2.795,5.055)--(2.795,5.056)--(2.795,5.057)--(2.795,5.058)--(2.796,5.058)--(2.796,5.059)%
  --(2.796,5.060)--(2.796,5.061)--(2.797,5.061)--(2.797,5.062)--(2.797,5.063)--(2.798,5.063)%
  --(2.798,5.064)--(2.798,5.065)--(2.798,5.066)--(2.799,5.066)--(2.799,5.067)--(2.799,5.068)%
  --(2.800,5.068)--(2.800,5.069)--(2.800,5.070)--(2.800,5.071)--(2.801,5.071)--(2.801,5.072)%
  --(2.801,5.073)--(2.802,5.073)--(2.802,5.074)--(2.802,5.075)--(2.802,5.076)--(2.803,5.076)%
  --(2.803,5.077)--(2.803,5.078)--(2.804,5.079)--(2.804,5.080)--(2.804,5.081)--(2.805,5.081)%
  --(2.805,5.082)--(2.805,5.083)--(2.806,5.084)--(2.806,5.085)--(2.806,5.086)--(2.807,5.086)%
  --(2.807,5.087)--(2.807,5.088)--(2.807,5.089)--(2.808,5.089)--(2.808,5.090)--(2.808,5.091)%
  --(2.809,5.091)--(2.809,5.092)--(2.809,5.093)--(2.809,5.094)--(2.810,5.094)--(2.810,5.095)%
  --(2.810,5.096)--(2.811,5.096)--(2.811,5.097)--(2.811,5.098)--(2.811,5.099)--(2.812,5.099)%
  --(2.812,5.100)--(2.812,5.101)--(2.812,5.102)--(2.813,5.102)--(2.813,5.103)--(2.813,5.104)%
  --(2.814,5.104)--(2.814,5.105)--(2.814,5.106)--(2.814,5.107)--(2.815,5.107)--(2.815,5.108)%
  --(2.815,5.109)--(2.816,5.109)--(2.816,5.110)--(2.816,5.111)--(2.816,5.112)--(2.817,5.112)%
  --(2.817,5.113)--(2.817,5.114)--(2.818,5.114)--(2.818,5.115)--(2.818,5.116)--(2.818,5.117)%
  --(2.819,5.117)--(2.819,5.118)--(2.819,5.119)--(2.820,5.120)--(2.820,5.121)--(2.820,5.122)%
  --(2.821,5.122)--(2.821,5.123)--(2.821,5.124)--(2.821,5.125)--(2.822,5.125)--(2.822,5.126)%
  --(2.822,5.127)--(2.823,5.127)--(2.823,5.128)--(2.823,5.129)--(2.823,5.130)--(2.824,5.130)%
  --(2.824,5.131)--(2.824,5.132)--(2.825,5.132)--(2.825,5.133)--(2.825,5.134)--(2.825,5.135)%
  --(2.826,5.135)--(2.826,5.136)--(2.826,5.137)--(2.827,5.138)--(2.827,5.139)--(2.827,5.140)%
  --(2.828,5.140)--(2.828,5.141)--(2.828,5.142)--(2.828,5.143)--(2.829,5.143)--(2.829,5.144)%
  --(2.829,5.145)--(2.830,5.145)--(2.830,5.146)--(2.830,5.147)--(2.830,5.148)--(2.831,5.148)%
  --(2.831,5.149)--(2.831,5.150)--(2.832,5.150)--(2.832,5.151)--(2.832,5.152)--(2.832,5.153)%
  --(2.833,5.153)--(2.833,5.154)--(2.833,5.155)--(2.834,5.155)--(2.834,5.156)--(2.834,5.157)%
  --(2.834,5.158)--(2.835,5.158)--(2.835,5.159)--(2.835,5.160)--(2.836,5.161)--(2.836,5.162)%
  --(2.836,5.163)--(2.837,5.163)--(2.837,5.164)--(2.837,5.165)--(2.837,5.166)--(2.838,5.166)%
  --(2.838,5.167)--(2.838,5.168)--(2.839,5.168)--(2.839,5.169)--(2.839,5.170)--(2.839,5.171)%
  --(2.840,5.171)--(2.840,5.172)--(2.840,5.173)--(2.841,5.173)--(2.841,5.174)--(2.841,5.175)%
  --(2.841,5.176)--(2.842,5.176)--(2.842,5.177)--(2.842,5.178)--(2.843,5.179)--(2.843,5.180)%
  --(2.843,5.181)--(2.844,5.181)--(2.844,5.182)--(2.844,5.183)--(2.844,5.184)--(2.845,5.184)%
  --(2.845,5.185)--(2.845,5.186)--(2.846,5.186)--(2.846,5.187)--(2.846,5.188)--(2.846,5.189)%
  --(2.847,5.189)--(2.847,5.190)--(2.847,5.191)--(2.848,5.191)--(2.848,5.192)--(2.848,5.193)%
  --(2.848,5.194)--(2.849,5.194)--(2.849,5.195)--(2.849,5.196)--(2.850,5.197)--(2.850,5.198)%
  --(2.850,5.199)--(2.851,5.199)--(2.851,5.200)--(2.851,5.201)--(2.852,5.202)--(2.852,5.203)%
  --(2.852,5.204)--(2.853,5.204)--(2.853,5.205)--(2.853,5.206)--(2.853,5.207)--(2.854,5.207)%
  --(2.854,5.208)--(2.854,5.209)--(2.855,5.209)--(2.855,5.210)--(2.855,5.211)--(2.855,5.212)%
  --(2.856,5.212)--(2.856,5.213)--(2.856,5.214)--(2.857,5.215)--(2.857,5.216)--(2.857,5.217)%
  --(2.858,5.217)--(2.858,5.218)--(2.858,5.219)--(2.859,5.220)--(2.859,5.221)--(2.859,5.222)%
  --(2.860,5.222)--(2.860,5.223)--(2.860,5.224)--(2.860,5.225)--(2.861,5.225)--(2.861,5.226)%
  --(2.861,5.227)--(2.862,5.227)--(2.862,5.228)--(2.862,5.229)--(2.862,5.230)--(2.863,5.230)%
  --(2.863,5.231)--(2.863,5.232)--(2.864,5.232)--(2.864,5.233)--(2.864,5.234)--(2.864,5.235)%
  --(2.865,5.235)--(2.865,5.236)--(2.865,5.237)--(2.866,5.238)--(2.866,5.239)--(2.866,5.240)%
  --(2.867,5.240)--(2.867,5.241)--(2.867,5.242)--(2.868,5.243)--(2.868,5.244)--(2.868,5.245)%
  --(2.869,5.245)--(2.869,5.246)--(2.869,5.247)--(2.869,5.248)--(2.870,5.248)--(2.870,5.249)%
  --(2.870,5.250)--(2.871,5.250)--(2.871,5.251)--(2.871,5.252)--(2.871,5.253)--(2.872,5.253)%
  --(2.872,5.254)--(2.872,5.255)--(2.873,5.256)--(2.873,5.257)--(2.873,5.258)--(2.874,5.258)%
  --(2.874,5.259)--(2.874,5.260)--(2.874,5.261)--(2.875,5.261)--(2.875,5.262)--(2.875,5.263)%
  --(2.876,5.263)--(2.876,5.264)--(2.876,5.265)--(2.876,5.266)--(2.877,5.266)--(2.877,5.267)%
  --(2.877,5.268)--(2.878,5.268)--(2.878,5.269)--(2.878,5.270)--(2.878,5.271)--(2.879,5.271)%
  --(2.879,5.272)--(2.879,5.273)--(2.880,5.273)--(2.880,5.274)--(2.880,5.275)--(2.880,5.276)%
  --(2.881,5.276)--(2.881,5.277)--(2.881,5.278)--(2.882,5.279)--(2.882,5.280)--(2.882,5.281)%
  --(2.883,5.281)--(2.883,5.282)--(2.883,5.283)--(2.884,5.284)--(2.884,5.285)--(2.884,5.286)%
  --(2.885,5.286)--(2.885,5.287)--(2.885,5.288)--(2.885,5.289)--(2.886,5.289)--(2.886,5.290)%
  --(2.886,5.291)--(2.887,5.291)--(2.887,5.292)--(2.887,5.293)--(2.887,5.294)--(2.888,5.294)%
  --(2.888,5.295)--(2.888,5.296)--(2.889,5.297)--(2.889,5.298)--(2.889,5.299)--(2.890,5.299)%
  --(2.890,5.300)--(2.890,5.301)--(2.890,5.302)--(2.891,5.302)--(2.891,5.303)--(2.891,5.304)%
  --(2.892,5.304)--(2.892,5.305)--(2.892,5.306)--(2.892,5.307)--(2.893,5.307)--(2.893,5.308)%
  --(2.893,5.309)--(2.894,5.309)--(2.894,5.310)--(2.894,5.311)--(2.894,5.312)--(2.895,5.312)%
  --(2.895,5.313)--(2.895,5.314)--(2.896,5.314)--(2.896,5.315)--(2.896,5.316)--(2.896,5.317)%
  --(2.897,5.317)--(2.897,5.318)--(2.897,5.319)--(2.898,5.320)--(2.898,5.321)--(2.898,5.322)%
  --(2.899,5.322)--(2.899,5.323)--(2.899,5.324)--(2.899,5.325)--(2.900,5.325)--(2.900,5.326)%
  --(2.900,5.327)--(2.901,5.327)--(2.901,5.328)--(2.901,5.329)--(2.901,5.330)--(2.902,5.330)%
  --(2.902,5.331)--(2.902,5.332)--(2.903,5.333)--(2.903,5.334)--(2.903,5.335)--(2.904,5.335)%
  --(2.904,5.336)--(2.904,5.337)--(2.905,5.338)--(2.905,5.339)--(2.905,5.340)--(2.906,5.340)%
  --(2.906,5.341)--(2.906,5.342)--(2.906,5.343)--(2.907,5.343)--(2.907,5.344)--(2.907,5.345)%
  --(2.908,5.345)--(2.908,5.346)--(2.908,5.347)--(2.908,5.348)--(2.909,5.348)--(2.909,5.349)%
  --(2.909,5.350)--(2.910,5.350)--(2.910,5.351)--(2.910,5.352)--(2.910,5.353)--(2.911,5.353)%
  --(2.911,5.354)--(2.911,5.355)--(2.912,5.355)--(2.912,5.356)--(2.912,5.357)--(2.912,5.358)%
  --(2.913,5.358)--(2.913,5.359)--(2.913,5.360)--(2.914,5.361)--(2.914,5.362)--(2.914,5.363)%
  --(2.915,5.363)--(2.915,5.364)--(2.915,5.365)--(2.915,5.366)--(2.916,5.366)--(2.916,5.367)%
  --(2.916,5.368)--(2.917,5.368)--(2.917,5.369)--(2.917,5.370)--(2.917,5.371)--(2.918,5.371)%
  --(2.918,5.372)--(2.918,5.373)--(2.919,5.374)--(2.919,5.375)--(2.919,5.376)--(2.920,5.376)%
  --(2.920,5.377)--(2.920,5.378)--(2.921,5.379)--(2.921,5.380)--(2.921,5.381)--(2.922,5.381)%
  --(2.922,5.382)--(2.922,5.383)--(2.922,5.384)--(2.923,5.384)--(2.923,5.385)--(2.923,5.386)%
  --(2.924,5.386)--(2.924,5.387)--(2.924,5.388)--(2.924,5.389)--(2.925,5.389)--(2.925,5.390)%
  --(2.925,5.391)--(2.926,5.391)--(2.926,5.392)--(2.926,5.393)--(2.926,5.394)--(2.927,5.394)%
  --(2.927,5.395)--(2.927,5.396)--(2.928,5.397)--(2.928,5.398)--(2.928,5.399)--(2.929,5.399)%
  --(2.929,5.400)--(2.929,5.401)--(2.930,5.402)--(2.930,5.403)--(2.930,5.404)--(2.931,5.404)%
  --(2.931,5.405)--(2.931,5.406)--(2.931,5.407)--(2.932,5.407)--(2.932,5.408)--(2.932,5.409)%
  --(2.933,5.409)--(2.933,5.410)--(2.933,5.411)--(2.933,5.412)--(2.934,5.412)--(2.934,5.413)%
  --(2.934,5.414)--(2.935,5.415)--(2.935,5.416)--(2.935,5.417)--(2.936,5.417)--(2.936,5.418)%
  --(2.936,5.419)--(2.937,5.420)--(2.937,5.421)--(2.937,5.422)--(2.938,5.422)--(2.938,5.423)%
  --(2.938,5.424)--(2.938,5.425)--(2.939,5.425)--(2.939,5.426)--(2.939,5.427)--(2.940,5.427)%
  --(2.940,5.428)--(2.940,5.429)--(2.940,5.430)--(2.941,5.430)--(2.941,5.431)--(2.941,5.432)%
  --(2.942,5.432)--(2.942,5.433)--(2.942,5.434)--(2.942,5.435)--(2.943,5.435)--(2.943,5.436)%
  --(2.943,5.437)--(2.944,5.438)--(2.944,5.439)--(2.944,5.440)--(2.945,5.440)--(2.945,5.441)%
  --(2.945,5.442)--(2.946,5.443)--(2.946,5.444)--(2.946,5.445)--(2.947,5.445)--(2.947,5.446)%
  --(2.947,5.447)--(2.947,5.448)--(2.948,5.448)--(2.948,5.449)--(2.948,5.450)--(2.949,5.450)%
  --(2.949,5.451)--(2.949,5.452)--(2.949,5.453)--(2.950,5.453)--(2.950,5.454)--(2.950,5.455)%
  --(2.951,5.456)--(2.951,5.457)--(2.951,5.458)--(2.952,5.458)--(2.952,5.459)--(2.952,5.460)%
  --(2.952,5.461)--(2.953,5.461)--(2.953,5.462)--(2.953,5.463)--(2.954,5.463)--(2.954,5.464)%
  --(2.954,5.465)--(2.954,5.466)--(2.955,5.466)--(2.955,5.467)--(2.955,5.468)--(2.956,5.468)%
  --(2.956,5.469)--(2.956,5.470)--(2.956,5.471)--(2.957,5.471)--(2.957,5.472)--(2.957,5.473)%
  --(2.958,5.473)--(2.958,5.474)--(2.958,5.475)--(2.958,5.476)--(2.959,5.476)--(2.959,5.477)%
  --(2.959,5.478)--(2.960,5.479)--(2.960,5.480)--(2.960,5.481)--(2.961,5.481)--(2.961,5.482)%
  --(2.961,5.483)--(2.961,5.484)--(2.962,5.484)--(2.962,5.485)--(2.962,5.486)--(2.963,5.486)%
  --(2.963,5.487)--(2.963,5.488)--(2.963,5.489)--(2.964,5.489)--(2.964,5.490)--(2.964,5.491)%
  --(2.965,5.491)--(2.965,5.492)--(2.965,5.493)--(2.965,5.494)--(2.966,5.494)--(2.966,5.495)%
  --(2.966,5.496)--(2.967,5.497)--(2.967,5.498)--(2.967,5.499)--(2.968,5.499)--(2.968,5.500)%
  --(2.968,5.501)--(2.968,5.502)--(2.969,5.502)--(2.969,5.503)--(2.969,5.504)--(2.970,5.504)%
  --(2.970,5.505)--(2.970,5.506)--(2.970,5.507)--(2.971,5.507)--(2.971,5.508)--(2.971,5.509)%
  --(2.972,5.509)--(2.972,5.510)--(2.972,5.511)--(2.972,5.512)--(2.973,5.512)--(2.973,5.513)%
  --(2.973,5.514)--(2.974,5.514)--(2.974,5.515)--(2.974,5.516)--(2.974,5.517)--(2.975,5.517)%
  --(2.975,5.518)--(2.975,5.519)--(2.976,5.520)--(2.976,5.521)--(2.976,5.522)--(2.977,5.522)%
  --(2.977,5.523)--(2.977,5.524)--(2.977,5.525)--(2.978,5.525)--(2.978,5.526)--(2.978,5.527)%
  --(2.979,5.527)--(2.979,5.528)--(2.979,5.529)--(2.979,5.530)--(2.980,5.530)--(2.980,5.531)%
  --(2.980,5.532)--(2.981,5.533)--(2.981,5.534)--(2.981,5.535)--(2.982,5.535)--(2.982,5.536)%
  --(2.982,5.537)--(2.983,5.538)--(2.983,5.539)--(2.983,5.540)--(2.984,5.540)--(2.984,5.541)%
  --(2.984,5.542)--(2.984,5.543)--(2.985,5.543)--(2.985,5.544)--(2.985,5.545)--(2.986,5.545)%
  --(2.986,5.546)--(2.986,5.547)--(2.986,5.548)--(2.987,5.548)--(2.987,5.549)--(2.987,5.550)%
  --(2.988,5.550)--(2.988,5.551)--(2.988,5.552)--(2.988,5.553)--(2.989,5.553)--(2.989,5.554)%
  --(2.989,5.555)--(2.990,5.555)--(2.990,5.556)--(2.990,5.557)--(2.990,5.558)--(2.991,5.558)%
  --(2.991,5.559)--(2.991,5.560)--(2.991,5.561)--(2.992,5.561)--(2.992,5.562)--(2.992,5.563)%
  --(2.993,5.563)--(2.993,5.564)--(2.993,5.565)--(2.993,5.566)--(2.994,5.566)--(2.994,5.567)%
  --(2.994,5.568)--(2.995,5.568)--(2.995,5.569)--(2.995,5.570)--(2.995,5.571)--(2.996,5.571)%
  --(2.996,5.572)--(2.996,5.573)--(2.997,5.574)--(2.997,5.575)--(2.997,5.576)--(2.998,5.576)%
  --(2.998,5.577)--(2.998,5.578)--(2.999,5.579)--(2.999,5.580)--(2.999,5.581)--(3.000,5.581)%
  --(3.000,5.582)--(3.000,5.583)--(3.000,5.584)--(3.001,5.584)--(3.001,5.585)--(3.001,5.586)%
  --(3.002,5.586)--(3.002,5.587)--(3.002,5.588)--(3.002,5.589)--(3.003,5.589)--(3.003,5.590)%
  --(3.003,5.591)--(3.004,5.591)--(3.004,5.592)--(3.004,5.593)--(3.004,5.594)--(3.005,5.594)%
  --(3.005,5.595)--(3.005,5.596)--(3.006,5.597)--(3.006,5.598)--(3.006,5.599)--(3.007,5.599)%
  --(3.007,5.600)--(3.007,5.601)--(3.007,5.602)--(3.008,5.602)--(3.008,5.603)--(3.008,5.604)%
  --(3.009,5.604)--(3.009,5.605)--(3.009,5.606)--(3.009,5.607)--(3.010,5.607)--(3.010,5.608)%
  --(3.010,5.609)--(3.011,5.609)--(3.011,5.610)--(3.011,5.611)--(3.011,5.612)--(3.012,5.612)%
  --(3.012,5.613)--(3.012,5.614)--(3.013,5.615)--(3.013,5.616)--(3.013,5.617)--(3.014,5.617)%
  --(3.014,5.618)--(3.014,5.619)--(3.015,5.620)--(3.015,5.621)--(3.015,5.622)--(3.016,5.622)%
  --(3.016,5.623)--(3.016,5.624)--(3.016,5.625)--(3.017,5.625)--(3.017,5.626)--(3.017,5.627)%
  --(3.018,5.627)--(3.018,5.628)--(3.018,5.629)--(3.018,5.630)--(3.019,5.630)--(3.019,5.631)%
  --(3.019,5.632)--(3.020,5.632)--(3.020,5.633)--(3.020,5.634)--(3.020,5.635)--(3.021,5.635)%
  --(3.021,5.636)--(3.021,5.637)--(3.022,5.638)--(3.022,5.639)--(3.022,5.640)--(3.023,5.640)%
  --(3.023,5.641)--(3.023,5.642)--(3.023,5.643)--(3.024,5.643)--(3.024,5.644)--(3.024,5.645)%
  --(3.025,5.645)--(3.025,5.646)--(3.025,5.647)--(3.025,5.648)--(3.026,5.648)--(3.026,5.649)%
  --(3.026,5.650)--(3.027,5.650)--(3.027,5.651)--(3.027,5.652)--(3.027,5.653)--(3.028,5.653)%
  --(3.028,5.654)--(3.028,5.655)--(3.029,5.656)--(3.029,5.657)--(3.029,5.658)--(3.030,5.658)%
  --(3.030,5.659)--(3.030,5.660)--(3.030,5.661)--(3.031,5.661)--(3.031,5.662)--(3.031,5.663)%
  --(3.032,5.663)--(3.032,5.664)--(3.032,5.665)--(3.032,5.666)--(3.033,5.666)--(3.033,5.667)%
  --(3.033,5.668)--(3.034,5.668)--(3.034,5.669)--(3.034,5.670)--(3.034,5.671)--(3.035,5.671)%
  --(3.035,5.672)--(3.035,5.673)--(3.036,5.673)--(3.036,5.674)--(3.036,5.675)--(3.036,5.676)%
  --(3.037,5.676)--(3.037,5.677)--(3.037,5.678)--(3.037,5.679)--(3.038,5.679)--(3.038,5.680)%
  --(3.038,5.681)--(3.039,5.681)--(3.039,5.682)--(3.039,5.683)--(3.039,5.684)--(3.040,5.684)%
  --(3.040,5.685)--(3.040,5.686)--(3.041,5.686)--(3.041,5.687)--(3.041,5.688)--(3.041,5.689)%
  --(3.042,5.689)--(3.042,5.690)--(3.042,5.691)--(3.043,5.691)--(3.043,5.692)--(3.043,5.693)%
  --(3.043,5.694)--(3.044,5.694)--(3.044,5.695)--(3.044,5.696)--(3.045,5.697)--(3.045,5.698)%
  --(3.045,5.699)--(3.046,5.699)--(3.046,5.700)--(3.046,5.701)--(3.046,5.702)--(3.047,5.702)%
  --(3.047,5.703)--(3.047,5.704)--(3.048,5.704)--(3.048,5.705)--(3.048,5.706)--(3.048,5.707)%
  --(3.049,5.707)--(3.049,5.708)--(3.049,5.709)--(3.050,5.709)--(3.050,5.710)--(3.050,5.711)%
  --(3.050,5.712)--(3.051,5.712)--(3.051,5.713)--(3.051,5.714)--(3.052,5.714)--(3.052,5.715)%
  --(3.052,5.716)--(3.052,5.717)--(3.053,5.717)--(3.053,5.718)--(3.053,5.719)--(3.053,5.720)%
  --(3.054,5.720)--(3.054,5.721)--(3.054,5.722)--(3.055,5.722)--(3.055,5.723)--(3.055,5.724)%
  --(3.055,5.725)--(3.056,5.725)--(3.056,5.726)--(3.056,5.727)--(3.057,5.727)--(3.057,5.728)%
  --(3.057,5.729)--(3.057,5.730)--(3.058,5.730)--(3.058,5.731)--(3.058,5.732)--(3.059,5.733)%
  --(3.059,5.734)--(3.059,5.735)--(3.060,5.735)--(3.060,5.736)--(3.060,5.737)--(3.061,5.738)%
  --(3.061,5.739)--(3.061,5.740)--(3.062,5.740)--(3.062,5.741)--(3.062,5.742)--(3.062,5.743)%
  --(3.063,5.743)--(3.063,5.744)--(3.063,5.745)--(3.064,5.745)--(3.064,5.746)--(3.064,5.747)%
  --(3.064,5.748)--(3.065,5.748)--(3.065,5.749)--(3.065,5.750)--(3.066,5.750)--(3.066,5.751)%
  --(3.066,5.752)--(3.066,5.753)--(3.067,5.753)--(3.067,5.754)--(3.067,5.755)--(3.068,5.756)%
  --(3.068,5.757)--(3.068,5.758)--(3.069,5.758)--(3.069,5.759)--(3.069,5.760)--(3.069,5.761)%
  --(3.070,5.761)--(3.070,5.762)--(3.070,5.763)--(3.071,5.763)--(3.071,5.764)--(3.071,5.765)%
  --(3.071,5.766)--(3.072,5.766)--(3.072,5.767)--(3.072,5.768)--(3.073,5.768)--(3.073,5.769)%
  --(3.073,5.770)--(3.073,5.771)--(3.074,5.771)--(3.074,5.772)--(3.074,5.773)--(3.075,5.774)%
  --(3.075,5.775)--(3.075,5.776)--(3.076,5.776)--(3.076,5.777)--(3.076,5.778)--(3.077,5.779)%
  --(3.077,5.780)--(3.077,5.781)--(3.078,5.781)--(3.078,5.782)--(3.078,5.783)--(3.078,5.784)%
  --(3.079,5.784)--(3.079,5.785)--(3.079,5.786)--(3.080,5.786)--(3.080,5.787)--(3.080,5.788)%
  --(3.080,5.789)--(3.081,5.789)--(3.081,5.790)--(3.081,5.791)--(3.082,5.791)--(3.082,5.792)%
  --(3.082,5.793)--(3.082,5.794)--(3.083,5.794)--(3.083,5.795)--(3.083,5.796)--(3.083,5.797)%
  --(3.084,5.797)--(3.084,5.798)--(3.084,5.799)--(3.085,5.799)--(3.085,5.800)--(3.085,5.801)%
  --(3.085,5.802)--(3.086,5.802)--(3.086,5.803)--(3.086,5.804)--(3.087,5.804)--(3.087,5.805)%
  --(3.087,5.806)--(3.087,5.807)--(3.088,5.807)--(3.088,5.808)--(3.088,5.809)--(3.089,5.809)%
  --(3.089,5.810)--(3.089,5.811)--(3.089,5.812)--(3.090,5.812)--(3.090,5.813)--(3.090,5.814)%
  --(3.091,5.815)--(3.091,5.816)--(3.091,5.817)--(3.092,5.817)--(3.092,5.818)--(3.092,5.819)%
  --(3.092,5.820)--(3.093,5.820)--(3.093,5.821)--(3.093,5.822)--(3.094,5.822)--(3.094,5.823)%
  --(3.094,5.824)--(3.094,5.825)--(3.095,5.825)--(3.095,5.826)--(3.095,5.827)--(3.096,5.827)%
  --(3.096,5.828)--(3.096,5.829)--(3.096,5.830)--(3.097,5.830)--(3.097,5.831)--(3.097,5.832)%
  --(3.098,5.832)--(3.098,5.833)--(3.098,5.834)--(3.098,5.835)--(3.099,5.835)--(3.099,5.836)%
  --(3.099,5.837)--(3.099,5.838)--(3.100,5.838)--(3.100,5.839)--(3.100,5.840)--(3.101,5.840)%
  --(3.101,5.841)--(3.101,5.842)--(3.101,5.843)--(3.102,5.843)--(3.102,5.844)--(3.102,5.845)%
  --(3.103,5.845)--(3.103,5.846)--(3.103,5.847)--(3.103,5.848)--(3.104,5.848)--(3.104,5.849)%
  --(3.104,5.850)--(3.105,5.850)--(3.105,5.851)--(3.105,5.852)--(3.105,5.853)--(3.106,5.853)%
  --(3.106,5.854)--(3.106,5.855)--(3.107,5.856)--(3.107,5.857)--(3.107,5.858)--(3.108,5.858)%
  --(3.108,5.859)--(3.108,5.860)--(3.108,5.861)--(3.109,5.861)--(3.109,5.862)--(3.109,5.863)%
  --(3.110,5.863)--(3.110,5.864)--(3.110,5.865)--(3.110,5.866)--(3.111,5.866)--(3.111,5.867)%
  --(3.111,5.868)--(3.112,5.868)--(3.112,5.869)--(3.112,5.870)--(3.112,5.871)--(3.113,5.871)%
  --(3.113,5.872)--(3.113,5.873)--(3.114,5.874)--(3.114,5.875)--(3.114,5.876)--(3.115,5.876)%
  --(3.115,5.877)--(3.115,5.878)--(3.115,5.879)--(3.116,5.879)--(3.116,5.880)--(3.116,5.881)%
  --(3.117,5.881)--(3.117,5.882)--(3.117,5.883)--(3.117,5.884)--(3.118,5.884)--(3.118,5.885)%
  --(3.118,5.886)--(3.119,5.886)--(3.119,5.887)--(3.119,5.888)--(3.119,5.889)--(3.120,5.889)%
  --(3.120,5.890)--(3.120,5.891)--(3.121,5.892)--(3.121,5.893)--(3.121,5.894)--(3.122,5.894)%
  --(3.122,5.895)--(3.122,5.896)--(3.123,5.897)--(3.123,5.898)--(3.123,5.899)--(3.124,5.899)%
  --(3.124,5.900)--(3.124,5.901)--(3.124,5.902)--(3.125,5.902)--(3.125,5.903)--(3.125,5.904)%
  --(3.126,5.904)--(3.126,5.905)--(3.126,5.906)--(3.126,5.907)--(3.127,5.907)--(3.127,5.908)%
  --(3.127,5.909)--(3.128,5.909)--(3.128,5.910)--(3.128,5.911)--(3.128,5.912)--(3.129,5.912)%
  --(3.129,5.913)--(3.129,5.914)--(3.130,5.915)--(3.130,5.916)--(3.130,5.917)--(3.131,5.917)%
  --(3.131,5.918)--(3.131,5.919)--(3.131,5.920)--(3.132,5.920)--(3.132,5.921)--(3.132,5.922)%
  --(3.133,5.922)--(3.133,5.923)--(3.133,5.924)--(3.133,5.925)--(3.134,5.925)--(3.134,5.926)%
  --(3.134,5.927)--(3.135,5.927)--(3.135,5.928)--(3.135,5.929)--(3.135,5.930)--(3.136,5.930)%
  --(3.136,5.931)--(3.136,5.932)--(3.137,5.933)--(3.137,5.934)--(3.137,5.935)--(3.138,5.935)%
  --(3.138,5.936)--(3.138,5.937)--(3.139,5.938)--(3.139,5.939)--(3.139,5.940)--(3.140,5.940)%
  --(3.140,5.941)--(3.140,5.942)--(3.140,5.943)--(3.141,5.943)--(3.141,5.944)--(3.141,5.945)%
  --(3.142,5.945)--(3.142,5.946)--(3.142,5.947)--(3.142,5.948)--(3.143,5.948)--(3.143,5.949)%
  --(3.143,5.950)--(3.144,5.950)--(3.144,5.951)--(3.144,5.952)--(3.144,5.953)--(3.145,5.953)%
  --(3.145,5.954)--(3.145,5.955)--(3.145,5.956)--(3.146,5.956)--(3.146,5.957)--(3.146,5.958)%
  --(3.147,5.958)--(3.147,5.959)--(3.147,5.960)--(3.147,5.961)--(3.148,5.961)--(3.148,5.962)%
  --(3.148,5.963)--(3.149,5.963)--(3.149,5.964)--(3.149,5.965)--(3.149,5.966)--(3.150,5.966)%
  --(3.150,5.967)--(3.150,5.968)--(3.151,5.968)--(3.151,5.969)--(3.151,5.970)--(3.151,5.971)%
  --(3.152,5.971)--(3.152,5.972)--(3.152,5.973)--(3.153,5.974)--(3.153,5.975)--(3.153,5.976)%
  --(3.154,5.976)--(3.154,5.977)--(3.154,5.978)--(3.155,5.979)--(3.155,5.980)--(3.155,5.981)%
  --(3.156,5.981)--(3.156,5.982)--(3.156,5.983)--(3.156,5.984)--(3.157,5.984)--(3.157,5.985)%
  --(3.157,5.986)--(3.158,5.986)--(3.158,5.987)--(3.158,5.988)--(3.158,5.989)--(3.159,5.989)%
  --(3.159,5.990)--(3.159,5.991)--(3.160,5.992)--(3.160,5.993)--(3.160,5.994)--(3.161,5.994)%
  --(3.161,5.995)--(3.161,5.996)--(3.161,5.997)--(3.162,5.997)--(3.162,5.998)--(3.162,5.999)%
  --(3.163,5.999)--(3.163,6.000)--(3.163,6.001)--(3.163,6.002)--(3.164,6.002)--(3.164,6.003)%
  --(3.164,6.004)--(3.165,6.004)--(3.165,6.005)--(3.165,6.006)--(3.165,6.007)--(3.166,6.007)%
  --(3.166,6.008)--(3.166,6.009)--(3.167,6.009)--(3.167,6.010)--(3.167,6.011)--(3.167,6.012)%
  --(3.168,6.012)--(3.168,6.013)--(3.168,6.014)--(3.169,6.015)--(3.169,6.016)--(3.169,6.017)%
  --(3.170,6.017)--(3.170,6.018)--(3.170,6.019)--(3.170,6.020)--(3.171,6.020)--(3.171,6.021)%
  --(3.171,6.022)--(3.172,6.022)--(3.172,6.023)--(3.172,6.024)--(3.172,6.025)--(3.173,6.025)%
  --(3.173,6.026)--(3.173,6.027)--(3.174,6.027)--(3.174,6.028)--(3.174,6.029)--(3.174,6.030)%
  --(3.175,6.030)--(3.175,6.031)--(3.175,6.032)--(3.176,6.033)--(3.176,6.034)--(3.176,6.035)%
  --(3.177,6.035)--(3.177,6.036)--(3.177,6.037)--(3.177,6.038)--(3.178,6.038)--(3.178,6.039)%
  --(3.178,6.040)--(3.179,6.040)--(3.179,6.041)--(3.179,6.042)--(3.179,6.043)--(3.180,6.043)%
  --(3.180,6.044)--(3.180,6.045)--(3.181,6.045)--(3.181,6.046)--(3.181,6.047)--(3.181,6.048)%
  --(3.182,6.048)--(3.182,6.049)--(3.182,6.050)--(3.183,6.050)--(3.183,6.051)--(3.183,6.052)%
  --(3.183,6.053)--(3.184,6.053)--(3.184,6.054)--(3.184,6.055)--(3.185,6.056)--(3.185,6.057)%
  --(3.185,6.058)--(3.186,6.058)--(3.186,6.059)--(3.186,6.060)--(3.186,6.061)--(3.187,6.061)%
  --(3.187,6.062)--(3.187,6.063)--(3.188,6.063)--(3.188,6.064)--(3.188,6.065)--(3.188,6.066)%
  --(3.189,6.066)--(3.189,6.067)--(3.189,6.068)--(3.190,6.069)--(3.190,6.070)--(3.190,6.071)%
  --(3.191,6.071)--(3.191,6.072)--(3.191,6.073)--(3.192,6.074)--(3.192,6.075)--(3.192,6.076)%
  --(3.193,6.076)--(3.193,6.077)--(3.193,6.078)--(3.193,6.079)--(3.194,6.079)--(3.194,6.080)%
  --(3.194,6.081)--(3.195,6.081)--(3.195,6.082)--(3.195,6.083)--(3.195,6.084)--(3.196,6.084)%
  --(3.196,6.085)--(3.196,6.086)--(3.197,6.086)--(3.197,6.087)--(3.197,6.088)--(3.197,6.089)%
  --(3.198,6.089)--(3.198,6.090)--(3.198,6.091)--(3.199,6.092)--(3.199,6.093)--(3.199,6.094)%
  --(3.200,6.094)--(3.200,6.095)--(3.200,6.096)--(3.201,6.097)--(3.201,6.098)--(3.201,6.099)%
  --(3.202,6.099)--(3.202,6.100)--(3.202,6.101)--(3.202,6.102)--(3.203,6.102)--(3.203,6.103)%
  --(3.203,6.104)--(3.204,6.104)--(3.204,6.105)--(3.204,6.106)--(3.204,6.107)--(3.205,6.107)%
  --(3.205,6.108)--(3.205,6.109)--(3.206,6.110)--(3.206,6.111)--(3.206,6.112)--(3.207,6.112)%
  --(3.207,6.113)--(3.207,6.114)--(3.208,6.115)--(3.208,6.116)--(3.208,6.117)--(3.209,6.117)%
  --(3.209,6.118)--(3.209,6.119)--(3.209,6.120)--(3.210,6.120)--(3.210,6.121)--(3.210,6.122)%
  --(3.211,6.122)--(3.211,6.123)--(3.211,6.124)--(3.211,6.125)--(3.212,6.125)--(3.212,6.126)%
  --(3.212,6.127)--(3.213,6.127)--(3.213,6.128)--(3.213,6.129)--(3.213,6.130)--(3.214,6.130)%
  --(3.214,6.131)--(3.214,6.132)--(3.215,6.133)--(3.215,6.134)--(3.215,6.135)--(3.216,6.135)%
  --(3.216,6.136)--(3.216,6.137)--(3.217,6.138)--(3.217,6.139)--(3.217,6.140)--(3.218,6.140)%
  --(3.218,6.141)--(3.218,6.142)--(3.218,6.143)--(3.219,6.143)--(3.219,6.144)--(3.219,6.145)%
  --(3.220,6.145)--(3.220,6.146)--(3.220,6.147)--(3.220,6.148)--(3.221,6.148)--(3.221,6.149)%
  --(3.221,6.150)--(3.222,6.151)--(3.222,6.152)--(3.222,6.153)--(3.223,6.153)--(3.223,6.154)%
  --(3.223,6.155)--(3.223,6.156)--(3.224,6.156)--(3.224,6.157)--(3.224,6.158)--(3.225,6.158)%
  --(3.225,6.159)--(3.225,6.160)--(3.225,6.161)--(3.226,6.161)--(3.226,6.162)--(3.226,6.163)%
  --(3.227,6.163)--(3.227,6.164)--(3.227,6.165)--(3.227,6.166)--(3.228,6.166)--(3.228,6.167)%
  --(3.228,6.168)--(3.229,6.168)--(3.229,6.169)--(3.229,6.170)--(3.229,6.171)--(3.230,6.171)%
  --(3.230,6.172)--(3.230,6.173)--(3.231,6.174)--(3.231,6.175)--(3.231,6.176)--(3.232,6.176)%
  --(3.232,6.177)--(3.232,6.178)--(3.233,6.179)--(3.233,6.180)--(3.233,6.181)--(3.234,6.181)%
  --(3.234,6.182)--(3.234,6.183)--(3.234,6.184)--(3.235,6.184)--(3.235,6.185)--(3.235,6.186)%
  --(3.236,6.186)--(3.236,6.187)--(3.236,6.188)--(3.236,6.189)--(3.237,6.189)--(3.237,6.190)%
  --(3.237,6.191)--(3.238,6.192)--(3.238,6.193)--(3.238,6.194)--(3.239,6.194)--(3.239,6.195)%
  --(3.239,6.196)--(3.239,6.197)--(3.240,6.197)--(3.240,6.198)--(3.240,6.199)--(3.241,6.199)%
  --(3.241,6.200)--(3.241,6.201)--(3.241,6.202)--(3.242,6.202)--(3.242,6.203)--(3.242,6.204)%
  --(3.243,6.204)--(3.243,6.205)--(3.243,6.206)--(3.243,6.207)--(3.244,6.207)--(3.244,6.208)%
  --(3.244,6.209)--(3.245,6.209)--(3.245,6.210)--(3.245,6.211)--(3.245,6.212)--(3.246,6.212)%
  --(3.246,6.213)--(3.246,6.214)--(3.247,6.215)--(3.247,6.216)--(3.247,6.217)--(3.248,6.217)%
  --(3.248,6.218)--(3.248,6.219)--(3.248,6.220)--(3.249,6.220)--(3.249,6.221)--(3.249,6.222)%
  --(3.250,6.222)--(3.250,6.223)--(3.250,6.224)--(3.250,6.225)--(3.251,6.225)--(3.251,6.226)%
  --(3.251,6.227)--(3.252,6.228)--(3.252,6.229)--(3.252,6.230)--(3.253,6.230)--(3.253,6.231)%
  --(3.253,6.232)--(3.254,6.233)--(3.254,6.234)--(3.254,6.235)--(3.255,6.235)--(3.255,6.236)%
  --(3.255,6.237)--(3.255,6.238)--(3.256,6.238)--(3.256,6.239)--(3.256,6.240)--(3.257,6.240)%
  --(3.257,6.241)--(3.257,6.242)--(3.257,6.243)--(3.258,6.243)--(3.258,6.244)--(3.258,6.245)%
  --(3.259,6.245)--(3.259,6.246)--(3.259,6.247)--(3.259,6.248)--(3.260,6.248)--(3.260,6.249)%
  --(3.260,6.250)--(3.261,6.250)--(3.261,6.251)--(3.261,6.252)--(3.261,6.253)--(3.262,6.253)%
  --(3.262,6.254)--(3.262,6.255)--(3.263,6.256)--(3.263,6.257)--(3.263,6.258)--(3.264,6.258)%
  --(3.264,6.259)--(3.264,6.260)--(3.264,6.261)--(3.265,6.261)--(3.265,6.262)--(3.265,6.263)%
  --(3.266,6.263)--(3.266,6.264)--(3.266,6.265)--(3.266,6.266)--(3.267,6.266)--(3.267,6.267)%
  --(3.267,6.268)--(3.268,6.269)--(3.268,6.270)--(3.268,6.271)--(3.269,6.271)--(3.269,6.272)%
  --(3.269,6.273)--(3.270,6.274)--(3.270,6.275)--(3.270,6.276)--(3.271,6.276)--(3.271,6.277)%
  --(3.271,6.278)--(3.271,6.279)--(3.272,6.279)--(3.272,6.280)--(3.272,6.281)--(3.273,6.281)%
  --(3.273,6.282)--(3.273,6.283)--(3.273,6.284)--(3.274,6.284)--(3.274,6.285)--(3.274,6.286)%
  --(3.275,6.286)--(3.275,6.287)--(3.275,6.288)--(3.275,6.289)--(3.276,6.289)--(3.276,6.290)%
  --(3.276,6.291)--(3.277,6.292)--(3.277,6.293)--(3.277,6.294)--(3.278,6.294)--(3.278,6.295)%
  --(3.278,6.296)--(3.279,6.297)--(3.279,6.298)--(3.279,6.299)--(3.280,6.299)--(3.280,6.300)%
  --(3.280,6.301)--(3.280,6.302)--(3.281,6.302)--(3.281,6.303)--(3.281,6.304)--(3.282,6.304)%
  --(3.282,6.305)--(3.282,6.306)--(3.282,6.307)--(3.283,6.307)--(3.283,6.308)--(3.283,6.309)%
  --(3.284,6.310)--(3.284,6.311)--(3.284,6.312)--(3.285,6.312)--(3.285,6.313)--(3.285,6.314)%
  --(3.285,6.315)--(3.286,6.315)--(3.286,6.316)--(3.286,6.317)--(3.287,6.317)--(3.287,6.318)%
  --(3.287,6.319)--(3.287,6.320)--(3.288,6.320)--(3.288,6.321)--(3.288,6.322)--(3.289,6.322)%
  --(3.289,6.323)--(3.289,6.324)--(3.289,6.325)--(3.290,6.325)--(3.290,6.326)--(3.290,6.327)%
  --(3.291,6.327)--(3.291,6.328)--(3.291,6.329)--(3.291,6.330)--(3.292,6.330)--(3.292,6.331)%
  --(3.292,6.332)--(3.293,6.333)--(3.293,6.334)--(3.293,6.335)--(3.294,6.335)--(3.294,6.336)%
  --(3.294,6.337)--(3.294,6.338)--(3.295,6.338)--(3.295,6.339)--(3.295,6.340)--(3.296,6.340)%
  --(3.296,6.341)--(3.296,6.342)--(3.296,6.343)--(3.297,6.343)--(3.297,6.344)--(3.297,6.345)%
  --(3.298,6.345)--(3.298,6.346)--(3.298,6.347)--(3.298,6.348)--(3.299,6.348)--(3.299,6.349)%
  --(3.299,6.350)--(3.300,6.351)--(3.300,6.352)--(3.300,6.353)--(3.301,6.353)--(3.301,6.354)%
  --(3.301,6.355)--(3.301,6.356)--(3.302,6.356)--(3.302,6.357)--(3.302,6.358)--(3.303,6.358)%
  --(3.303,6.359)--(3.303,6.360)--(3.303,6.361)--(3.304,6.361)--(3.304,6.362)--(3.304,6.363)%
  --(3.305,6.363)--(3.305,6.364)--(3.305,6.365)--(3.305,6.366)--(3.306,6.366)--(3.306,6.367)%
  --(3.306,6.368)--(3.307,6.368)--(3.307,6.369)--(3.307,6.370)--(3.307,6.371)--(3.308,6.371)%
  --(3.308,6.372)--(3.308,6.373)--(3.309,6.374)--(3.309,6.375)--(3.309,6.376)--(3.310,6.376)%
  --(3.310,6.377)--(3.310,6.378)--(3.310,6.379)--(3.311,6.379)--(3.311,6.380)--(3.311,6.381)%
  --(3.312,6.381)--(3.312,6.382)--(3.312,6.383)--(3.312,6.384)--(3.313,6.384)--(3.313,6.385)%
  --(3.313,6.386)--(3.314,6.386)--(3.314,6.387)--(3.314,6.388)--(3.314,6.389)--(3.315,6.389)%
  --(3.315,6.390)--(3.315,6.391)--(3.316,6.392)--(3.316,6.393)--(3.316,6.394)--(3.317,6.394)%
  --(3.317,6.395)--(3.317,6.396)--(3.317,6.397)--(3.318,6.397)--(3.318,6.398)--(3.318,6.399)%
  --(3.319,6.399)--(3.319,6.400)--(3.319,6.401)--(3.319,6.402)--(3.320,6.402)--(3.320,6.403)%
  --(3.320,6.404)--(3.321,6.404)--(3.321,6.405)--(3.321,6.406)--(3.321,6.407)--(3.322,6.407)%
  --(3.322,6.408)--(3.322,6.409)--(3.323,6.409)--(3.323,6.410)--(3.323,6.411)--(3.323,6.412)%
  --(3.324,6.412)--(3.324,6.413)--(3.324,6.414)--(3.324,6.415)--(3.325,6.415)--(3.325,6.416)%
  --(3.325,6.417)--(3.326,6.417)--(3.326,6.418)--(3.326,6.419)--(3.326,6.420)--(3.327,6.420)%
  --(3.327,6.421)--(3.327,6.422)--(3.328,6.422)--(3.328,6.423)--(3.328,6.424)--(3.328,6.425)%
  --(3.329,6.425)--(3.329,6.426)--(3.329,6.427)--(3.330,6.428)--(3.330,6.429)--(3.330,6.430)%
  --(3.331,6.430)--(3.331,6.431)--(3.331,6.432)--(3.332,6.433)--(3.332,6.434)--(3.332,6.435)%
  --(3.333,6.435)--(3.333,6.436)--(3.333,6.437)--(3.333,6.438)--(3.334,6.438)--(3.334,6.439)%
  --(3.334,6.440)--(3.335,6.440)--(3.335,6.441)--(3.335,6.442)--(3.335,6.443)--(3.336,6.443)%
  --(3.336,6.444)--(3.336,6.445)--(3.337,6.445)--(3.337,6.446)--(3.337,6.447)--(3.337,6.448)%
  --(3.338,6.448)--(3.338,6.449)--(3.338,6.450)--(3.339,6.451)--(3.339,6.452)--(3.339,6.453)%
  --(3.340,6.453)--(3.340,6.454)--(3.340,6.455)--(3.340,6.456)--(3.341,6.456)--(3.341,6.457)%
  --(3.341,6.458)--(3.342,6.458)--(3.342,6.459)--(3.342,6.460)--(3.342,6.461)--(3.343,6.461)%
  --(3.343,6.462)--(3.343,6.463)--(3.344,6.463)--(3.344,6.464)--(3.344,6.465)--(3.344,6.466)%
  --(3.345,6.466)--(3.345,6.467)--(3.345,6.468)--(3.346,6.469)--(3.346,6.470)--(3.346,6.471)%
  --(3.347,6.471)--(3.347,6.472)--(3.347,6.473)--(3.348,6.474)--(3.348,6.475)--(3.348,6.476)%
  --(3.349,6.476)--(3.349,6.477)--(3.349,6.478)--(3.349,6.479)--(3.350,6.479)--(3.350,6.480)%
  --(3.350,6.481)--(3.351,6.481)--(3.351,6.482)--(3.351,6.483)--(3.351,6.484)--(3.352,6.484)%
  --(3.352,6.485)--(3.352,6.486)--(3.353,6.486)--(3.353,6.487)--(3.353,6.488)--(3.353,6.489)%
  --(3.354,6.489)--(3.354,6.490)--(3.354,6.491)--(3.355,6.492)--(3.355,6.493)--(3.355,6.494)%
  --(3.356,6.494)--(3.356,6.495)--(3.356,6.496)--(3.356,6.497)--(3.357,6.497)--(3.357,6.498)%
  --(3.357,6.499)--(3.358,6.499)--(3.358,6.500)--(3.358,6.501)--(3.358,6.502)--(3.359,6.502)%
  --(3.359,6.503)--(3.359,6.504)--(3.360,6.504)--(3.360,6.505)--(3.360,6.506)--(3.360,6.507)%
  --(3.361,6.507)--(3.361,6.508)--(3.361,6.509)--(3.362,6.510)--(3.362,6.511)--(3.362,6.512)%
  --(3.363,6.512)--(3.363,6.513)--(3.363,6.514)--(3.363,6.515)--(3.364,6.515)--(3.364,6.516)%
  --(3.364,6.517)--(3.365,6.517)--(3.365,6.518)--(3.365,6.519)--(3.365,6.520)--(3.366,6.520)%
  --(3.366,6.521)--(3.366,6.522)--(3.367,6.522)--(3.367,6.523)--(3.367,6.524)--(3.367,6.525)%
  --(3.368,6.525)--(3.368,6.526)--(3.368,6.527)--(3.369,6.527)--(3.369,6.528)--(3.369,6.529)%
  --(3.369,6.530)--(3.370,6.530)--(3.370,6.531)--(3.370,6.532)--(3.370,6.533)--(3.371,6.533)%
  --(3.371,6.534)--(3.371,6.535)--(3.372,6.535)--(3.372,6.536)--(3.372,6.537)--(3.372,6.538)%
  --(3.373,6.538)--(3.373,6.539)--(3.373,6.540)--(3.374,6.540)--(3.374,6.541)--(3.374,6.542)%
  --(3.374,6.543)--(3.375,6.543)--(3.375,6.544)--(3.375,6.545)--(3.376,6.545)--(3.376,6.546)%
  --(3.376,6.547)--(3.376,6.548)--(3.377,6.548)--(3.377,6.549)--(3.377,6.550)--(3.378,6.551)%
  --(3.378,6.552)--(3.378,6.553)--(3.379,6.553)--(3.379,6.554)--(3.379,6.555)--(3.379,6.556)%
  --(3.380,6.556)--(3.380,6.557)--(3.380,6.558)--(3.381,6.558)--(3.381,6.559)--(3.381,6.560)%
  --(3.381,6.561)--(3.382,6.561)--(3.382,6.562)--(3.382,6.563)--(3.383,6.563)--(3.383,6.564)%
  --(3.383,6.565)--(3.383,6.566)--(3.384,6.566)--(3.384,6.567)--(3.384,6.568)--(3.385,6.568)%
  --(3.385,6.569)--(3.385,6.570)--(3.385,6.571)--(3.386,6.571)--(3.386,6.572)--(3.386,6.573)%
  --(3.386,6.574)--(3.387,6.574)--(3.387,6.575)--(3.387,6.576)--(3.388,6.576)--(3.388,6.577)%
  --(3.388,6.578)--(3.388,6.579)--(3.389,6.579)--(3.389,6.580)--(3.389,6.581)--(3.390,6.581)%
  --(3.390,6.582)--(3.390,6.583)--(3.390,6.584)--(3.391,6.584)--(3.391,6.585)--(3.391,6.586)%
  --(3.392,6.587)--(3.392,6.588)--(3.392,6.589)--(3.393,6.589)--(3.393,6.590)--(3.393,6.591)%
  --(3.394,6.592)--(3.394,6.593)--(3.394,6.594)--(3.395,6.594)--(3.395,6.595)--(3.395,6.596)%
  --(3.395,6.597)--(3.396,6.597)--(3.396,6.598)--(3.396,6.599)--(3.397,6.599)--(3.397,6.600)%
  --(3.397,6.601)--(3.397,6.602)--(3.398,6.602)--(3.398,6.603)--(3.398,6.604)--(3.399,6.604)%
  --(3.399,6.605)--(3.399,6.606)--(3.399,6.607)--(3.400,6.607)--(3.400,6.608)--(3.400,6.609)%
  --(3.401,6.609)--(3.401,6.610)--(3.401,6.611)--(3.401,6.612)--(3.402,6.612)--(3.402,6.613)%
  --(3.402,6.614)--(3.402,6.615)--(3.403,6.615)--(3.403,6.616)--(3.403,6.617)--(3.404,6.617)%
  --(3.404,6.618)--(3.404,6.619)--(3.404,6.620)--(3.405,6.620)--(3.405,6.621)--(3.405,6.622)%
  --(3.406,6.622)--(3.406,6.623)--(3.406,6.624)--(3.406,6.625)--(3.407,6.625)--(3.407,6.626)%
  --(3.407,6.627)--(3.408,6.628)--(3.408,6.629)--(3.408,6.630)--(3.409,6.630)--(3.409,6.631)%
  --(3.409,6.632)--(3.410,6.633)--(3.410,6.634)--(3.410,6.635)--(3.411,6.635)--(3.411,6.636)%
  --(3.411,6.637)--(3.411,6.638)--(3.412,6.638)--(3.412,6.639)--(3.412,6.640)--(3.413,6.640)%
  --(3.413,6.641)--(3.413,6.642)--(3.413,6.643)--(3.414,6.643)--(3.414,6.644)--(3.414,6.645)%
  --(3.415,6.645)--(3.415,6.646)--(3.415,6.647)--(3.415,6.648)--(3.416,6.648)--(3.416,6.649)%
  --(3.416,6.650)--(3.416,6.651)--(3.417,6.651)--(3.417,6.652)--(3.417,6.653)--(3.418,6.653)%
  --(3.418,6.654)--(3.418,6.655)--(3.418,6.656)--(3.419,6.656)--(3.419,6.657)--(3.419,6.658)%
  --(3.420,6.658)--(3.420,6.659)--(3.420,6.660)--(3.420,6.661)--(3.421,6.661)--(3.421,6.662)%
  --(3.421,6.663)--(3.422,6.663)--(3.422,6.664)--(3.422,6.665)--(3.422,6.666)--(3.423,6.666)%
  --(3.423,6.667)--(3.423,6.668)--(3.424,6.669)--(3.424,6.670)--(3.424,6.671)--(3.425,6.671)%
  --(3.425,6.672)--(3.425,6.673)--(3.426,6.674)--(3.426,6.675)--(3.426,6.676)--(3.427,6.676)%
  --(3.427,6.677)--(3.427,6.678)--(3.427,6.679)--(3.428,6.679)--(3.428,6.680)--(3.428,6.681)%
  --(3.429,6.681)--(3.429,6.682)--(3.429,6.683)--(3.429,6.684)--(3.430,6.684)--(3.430,6.685)%
  --(3.430,6.686)--(3.431,6.686)--(3.431,6.687)--(3.431,6.688)--(3.431,6.689)--(3.432,6.689)%
  --(3.432,6.690)--(3.432,6.691)--(3.432,6.692)--(3.433,6.692)--(3.433,6.693)--(3.433,6.694)%
  --(3.434,6.694)--(3.434,6.695)--(3.434,6.696)--(3.434,6.697)--(3.435,6.697)--(3.435,6.698)%
  --(3.435,6.699)--(3.436,6.699)--(3.436,6.700)--(3.436,6.701)--(3.436,6.702)--(3.437,6.702)%
  --(3.437,6.703)--(3.437,6.704)--(3.438,6.704)--(3.438,6.705)--(3.438,6.706)--(3.438,6.707)%
  --(3.439,6.707)--(3.439,6.708)--(3.439,6.709)--(3.440,6.710)--(3.440,6.711)--(3.440,6.712)%
  --(3.441,6.712)--(3.441,6.713)--(3.441,6.714)--(3.441,6.715)--(3.442,6.715)--(3.442,6.716)%
  --(3.442,6.717)--(3.443,6.717)--(3.443,6.718)--(3.443,6.719)--(3.443,6.720)--(3.444,6.720)%
  --(3.444,6.721)--(3.444,6.722)--(3.445,6.722)--(3.445,6.723)--(3.445,6.724)--(3.445,6.725)%
  --(3.446,6.725)--(3.446,6.726)--(3.446,6.727)--(3.447,6.728)--(3.447,6.729)--(3.447,6.730)%
  --(3.448,6.730)--(3.448,6.731)--(3.448,6.732)--(3.448,6.733)--(3.449,6.733)--(3.449,6.734)%
  --(3.449,6.735)--(3.450,6.735)--(3.450,6.736)--(3.450,6.737)--(3.450,6.738)--(3.451,6.738)%
  --(3.451,6.739)--(3.451,6.740)--(3.452,6.740)--(3.452,6.741)--(3.452,6.742)--(3.452,6.743)%
  --(3.453,6.743)--(3.453,6.744)--(3.453,6.745)--(3.454,6.745)--(3.454,6.746)--(3.454,6.747)%
  --(3.454,6.748)--(3.455,6.748)--(3.455,6.749)--(3.455,6.750)--(3.456,6.751)--(3.456,6.752)%
  --(3.456,6.753)--(3.457,6.753)--(3.457,6.754)--(3.457,6.755)--(3.457,6.756)--(3.458,6.756)%
  --(3.458,6.757)--(3.458,6.758)--(3.459,6.758)--(3.459,6.759)--(3.459,6.760)--(3.459,6.761)%
  --(3.460,6.761)--(3.460,6.762)--(3.460,6.763)--(3.461,6.763)--(3.461,6.764)--(3.461,6.765)%
  --(3.461,6.766)--(3.462,6.766)--(3.462,6.767)--(3.462,6.768)--(3.463,6.769)--(3.463,6.770)%
  --(3.463,6.771)--(3.464,6.771)--(3.464,6.772)--(3.464,6.773)--(3.464,6.774)--(3.465,6.774)%
  --(3.465,6.775)--(3.465,6.776)--(3.466,6.776)--(3.466,6.777)--(3.466,6.778)--(3.466,6.779)%
  --(3.467,6.779)--(3.467,6.780)--(3.467,6.781)--(3.468,6.781)--(3.468,6.782)--(3.468,6.783)%
  --(3.468,6.784)--(3.469,6.784)--(3.469,6.785)--(3.469,6.786)--(3.470,6.787)--(3.470,6.788)%
  --(3.470,6.789)--(3.471,6.789)--(3.471,6.790)--(3.471,6.791)--(3.472,6.792)--(3.472,6.793)%
  --(3.472,6.794)--(3.473,6.794)--(3.473,6.795)--(3.473,6.796)--(3.473,6.797)--(3.474,6.797)%
  --(3.474,6.798)--(3.474,6.799)--(3.475,6.799)--(3.475,6.800)--(3.475,6.801)--(3.475,6.802)%
  --(3.476,6.802)--(3.476,6.803)--(3.476,6.804)--(3.477,6.804)--(3.477,6.805)--(3.477,6.806)%
  --(3.477,6.807)--(3.478,6.807)--(3.478,6.808)--(3.478,6.809)--(3.479,6.810)--(3.479,6.811)%
  --(3.479,6.812)--(3.480,6.812)--(3.480,6.813)--(3.480,6.814)--(3.480,6.815)--(3.481,6.815)%
  --(3.481,6.816)--(3.481,6.817)--(3.482,6.817)--(3.482,6.818)--(3.482,6.819)--(3.482,6.820)%
  --(3.483,6.820)--(3.483,6.821)--(3.483,6.822)--(3.484,6.822)--(3.484,6.823)--(3.484,6.824)%
  --(3.484,6.825)--(3.485,6.825)--(3.485,6.826)--(3.485,6.827)--(3.486,6.828)--(3.486,6.829)%
  --(3.486,6.830)--(3.487,6.830)--(3.487,6.831)--(3.487,6.832)--(3.488,6.833)--(3.488,6.834)%
  --(3.488,6.835)--(3.489,6.835)--(3.489,6.836)--(3.489,6.837)--(3.489,6.838)--(3.490,6.838)%
  --(3.490,6.839)--(3.490,6.840)--(3.491,6.840)--(3.491,6.841)--(3.491,6.842)--(3.491,6.843)%
  --(3.492,6.843)--(3.492,6.844)--(3.492,6.845)--(3.493,6.846)--(3.493,6.847)--(3.493,6.848)%
  --(3.494,6.848)--(3.494,6.849)--(3.494,6.850)--(3.494,6.851)--(3.495,6.851)--(3.495,6.852)%
  --(3.495,6.853)--(3.496,6.853)--(3.496,6.854)--(3.496,6.855)--(3.496,6.856)--(3.497,6.856)%
  --(3.497,6.857)--(3.497,6.858)--(3.498,6.858)--(3.498,6.859)--(3.498,6.860)--(3.498,6.861)%
  --(3.499,6.861)--(3.499,6.862)--(3.499,6.863)--(3.500,6.863)--(3.500,6.864)--(3.500,6.865)%
  --(3.500,6.866)--(3.501,6.866)--(3.501,6.867)--(3.501,6.868)--(3.502,6.869)--(3.502,6.870)%
  --(3.502,6.871)--(3.503,6.871)--(3.503,6.872)--(3.503,6.873)--(3.504,6.874)--(3.504,6.875)%
  --(3.504,6.876)--(3.505,6.876)--(3.505,6.877)--(3.505,6.878)--(3.505,6.879)--(3.506,6.879)%
  --(3.506,6.880)--(3.506,6.881)--(3.507,6.881)--(3.507,6.882)--(3.507,6.883)--(3.507,6.884)%
  --(3.508,6.884)--(3.508,6.885)--(3.508,6.886)--(3.509,6.887)--(3.509,6.888)--(3.509,6.889)%
  --(3.510,6.889)--(3.510,6.890)--(3.510,6.891)--(3.510,6.892)--(3.511,6.892)--(3.511,6.893)%
  --(3.511,6.894)--(3.512,6.894)--(3.512,6.895)--(3.512,6.896)--(3.512,6.897)--(3.513,6.897)%
  --(3.513,6.898)--(3.513,6.899)--(3.514,6.899)--(3.514,6.900)--(3.514,6.901)--(3.514,6.902)%
  --(3.515,6.902)--(3.515,6.903)--(3.515,6.904)--(3.516,6.904)--(3.516,6.905)--(3.516,6.906)%
  --(3.516,6.907)--(3.517,6.907)--(3.517,6.908)--(3.517,6.909)--(3.518,6.910)--(3.518,6.911)%
  --(3.518,6.912)--(3.519,6.912)--(3.519,6.913)--(3.519,6.914)--(3.519,6.915)--(3.520,6.915)%
  --(3.520,6.916)--(3.520,6.917)--(3.521,6.917)--(3.521,6.918)--(3.521,6.919)--(3.521,6.920)%
  --(3.522,6.920)--(3.522,6.921)--(3.522,6.922)--(3.523,6.923)--(3.523,6.924)--(3.523,6.925)%
  --(3.524,6.925)--(3.524,6.926)--(3.524,6.927)--(3.525,6.928)--(3.525,6.929)--(3.525,6.930)%
  --(3.526,6.930)--(3.526,6.931)--(3.526,6.932)--(3.526,6.933)--(3.527,6.933)--(3.527,6.934)%
  --(3.527,6.935)--(3.528,6.935)--(3.528,6.936)--(3.528,6.937)--(3.528,6.938)--(3.529,6.938)%
  --(3.529,6.939)--(3.529,6.940)--(3.530,6.940)--(3.530,6.941)--(3.530,6.942)--(3.530,6.943)%
  --(3.531,6.943)--(3.531,6.944)--(3.531,6.945)--(3.532,6.945)--(3.532,6.946)--(3.532,6.947)%
  --(3.532,6.948)--(3.533,6.948)--(3.533,6.949)--(3.533,6.950)--(3.534,6.951)--(3.534,6.952)%
  --(3.534,6.953)--(3.535,6.953)--(3.535,6.954)--(3.535,6.955)--(3.535,6.956)--(3.536,6.956)%
  --(3.536,6.957)--(3.536,6.958)--(3.537,6.958)--(3.537,6.959)--(3.537,6.960)--(3.537,6.961)%
  --(3.538,6.961)--(3.538,6.962)--(3.538,6.963)--(3.539,6.964)--(3.539,6.965)--(3.539,6.966)%
  --(3.540,6.966)--(3.540,6.967)--(3.540,6.968)--(3.541,6.969)--(3.541,6.970)--(3.541,6.971)%
  --(3.542,6.971)--(3.542,6.972)--(3.542,6.973)--(3.542,6.974)--(3.543,6.974)--(3.543,6.975)%
  --(3.543,6.976)--(3.544,6.976)--(3.544,6.977)--(3.544,6.978)--(3.544,6.979)--(3.545,6.979)%
  --(3.545,6.980)--(3.545,6.981)--(3.546,6.981)--(3.546,6.982)--(3.546,6.983)--(3.546,6.984)%
  --(3.547,6.984)--(3.547,6.985)--(3.547,6.986)--(3.548,6.987)--(3.548,6.988)--(3.548,6.989)%
  --(3.549,6.989)--(3.549,6.990)--(3.549,6.991)--(3.550,6.992)--(3.550,6.993)--(3.550,6.994)%
  --(3.551,6.994)--(3.551,6.995)--(3.551,6.996)--(3.551,6.997)--(3.552,6.997)--(3.552,6.998)%
  --(3.552,6.999)--(3.553,6.999)--(3.553,7.000)--(3.553,7.001)--(3.553,7.002)--(3.554,7.002)%
  --(3.554,7.003)--(3.554,7.004)--(3.555,7.005)--(3.555,7.006)--(3.555,7.007)--(3.556,7.007)%
  --(3.556,7.008)--(3.556,7.009)--(3.556,7.010)--(3.557,7.010)--(3.557,7.011)--(3.557,7.012)%
  --(3.558,7.012)--(3.558,7.013)--(3.558,7.014)--(3.558,7.015)--(3.559,7.015)--(3.559,7.016)%
  --(3.559,7.017)--(3.560,7.017)--(3.560,7.018)--(3.560,7.019)--(3.560,7.020)--(3.561,7.020)%
  --(3.561,7.021)--(3.561,7.022)--(3.562,7.022)--(3.562,7.023)--(3.562,7.024)--(3.562,7.025)%
  --(3.563,7.025)--(3.563,7.026)--(3.563,7.027)--(3.564,7.028)--(3.564,7.029)--(3.564,7.030)%
  --(3.565,7.030)--(3.565,7.031)--(3.565,7.032)--(3.566,7.033)--(3.566,7.034)--(3.566,7.035)%
  --(3.567,7.035)--(3.567,7.036)--(3.567,7.037)--(3.567,7.038)--(3.568,7.038)--(3.568,7.039)%
  --(3.568,7.040)--(3.569,7.040)--(3.569,7.041)--(3.569,7.042)--(3.569,7.043)--(3.570,7.043)%
  --(3.570,7.044)--(3.570,7.045)--(3.571,7.046)--(3.571,7.047)--(3.571,7.048)--(3.572,7.048)%
  --(3.572,7.049)--(3.572,7.050)--(3.572,7.051)--(3.573,7.051)--(3.573,7.052)--(3.573,7.053)%
  --(3.574,7.053)--(3.574,7.054)--(3.574,7.055)--(3.574,7.056)--(3.575,7.056)--(3.575,7.057)%
  --(3.575,7.058)--(3.576,7.058)--(3.576,7.059)--(3.576,7.060)--(3.576,7.061)--(3.577,7.061)%
  --(3.577,7.062)--(3.577,7.063)--(3.578,7.063)--(3.578,7.064)--(3.578,7.065)--(3.578,7.066)%
  --(3.579,7.066)--(3.579,7.067)--(3.579,7.068)--(3.580,7.069)--(3.580,7.070)--(3.580,7.071)%
  --(3.581,7.071)--(3.581,7.072)--(3.581,7.073)--(3.581,7.074)--(3.582,7.074)--(3.582,7.075)%
  --(3.582,7.076)--(3.583,7.076)--(3.583,7.077)--(3.583,7.078)--(3.583,7.079)--(3.584,7.079)%
  --(3.584,7.080)--(3.584,7.081)--(3.585,7.081)--(3.585,7.082)--(3.585,7.083)--(3.585,7.084)%
  --(3.586,7.084)--(3.586,7.085)--(3.586,7.086)--(3.587,7.087)--(3.587,7.088)--(3.587,7.089)%
  --(3.588,7.089)--(3.588,7.090)--(3.588,7.091)--(3.588,7.092)--(3.589,7.092)--(3.589,7.093)%
  --(3.589,7.094)--(3.590,7.094)--(3.590,7.095)--(3.590,7.096)--(3.590,7.097)--(3.591,7.097)%
  --(3.591,7.098)--(3.591,7.099)--(3.592,7.099)--(3.592,7.100)--(3.592,7.101)--(3.592,7.102)%
  --(3.593,7.102)--(3.593,7.103)--(3.593,7.104)--(3.594,7.104)--(3.594,7.105)--(3.594,7.106)%
  --(3.594,7.107)--(3.595,7.107)--(3.595,7.108)--(3.595,7.109)--(3.596,7.110)--(3.596,7.111)%
  --(3.596,7.112)--(3.597,7.112)--(3.597,7.113)--(3.597,7.114)--(3.597,7.115)--(3.598,7.115)%
  --(3.598,7.116)--(3.598,7.117)--(3.599,7.117)--(3.599,7.118)--(3.599,7.119)--(3.599,7.120)%
  --(3.600,7.120)--(3.600,7.121)--(3.600,7.122)--(3.601,7.123)--(3.601,7.124)--(3.601,7.125)%
  --(3.602,7.125)--(3.602,7.126)--(3.602,7.127)--(3.603,7.128)--(3.603,7.129)--(3.603,7.130)%
  --(3.604,7.130)--(3.604,7.131)--(3.604,7.132)--(3.604,7.133)--(3.605,7.133)--(3.605,7.134)%
  --(3.605,7.135)--(3.606,7.135)--(3.606,7.136)--(3.606,7.137)--(3.606,7.138)--(3.607,7.138)%
  --(3.607,7.139)--(3.607,7.140)--(3.608,7.140)--(3.608,7.141)--(3.608,7.142)--(3.608,7.143)%
  --(3.609,7.143)--(3.609,7.144)--(3.609,7.145)--(3.610,7.146)--(3.610,7.147)--(3.610,7.148)%
  --(3.611,7.148)--(3.611,7.149)--(3.611,7.150)--(3.612,7.151)--(3.612,7.152)--(3.612,7.153)%
  --(3.613,7.153)--(3.613,7.154)--(3.613,7.155)--(3.613,7.156)--(3.614,7.156)--(3.614,7.157)%
  --(3.614,7.158)--(3.615,7.158)--(3.615,7.159)--(3.615,7.160)--(3.615,7.161)--(3.616,7.161)%
  --(3.616,7.162)--(3.616,7.163)--(3.617,7.164)--(3.617,7.165)--(3.617,7.166)--(3.618,7.166)%
  --(3.618,7.167)--(3.618,7.168)--(3.619,7.169)--(3.619,7.170)--(3.619,7.171)--(3.620,7.171)%
  --(3.620,7.172)--(3.620,7.173)--(3.620,7.174)--(3.621,7.174)--(3.621,7.175)--(3.621,7.176)%
  --(3.622,7.176)--(3.622,7.177)--(3.622,7.178)--(3.622,7.179)--(3.623,7.179)--(3.623,7.180)%
  --(3.623,7.181)--(3.624,7.181)--(3.624,7.182)--(3.624,7.183)--(3.624,7.184)--(3.625,7.184)%
  --(3.625,7.185)--(3.625,7.186)--(3.626,7.187)--(3.626,7.188)--(3.626,7.189)--(3.627,7.189)%
  --(3.627,7.190)--(3.627,7.191)--(3.627,7.192)--(3.628,7.192)--(3.628,7.193)--(3.628,7.194)%
  --(3.629,7.194)--(3.629,7.195)--(3.629,7.196)--(3.629,7.197)--(3.630,7.197)--(3.630,7.198)%
  --(3.630,7.199)--(3.631,7.199)--(3.631,7.200)--(3.631,7.201)--(3.631,7.202)--(3.632,7.202)%
  --(3.632,7.203)--(3.632,7.204)--(3.633,7.205)--(3.633,7.206)--(3.633,7.207)--(3.634,7.207)%
  --(3.634,7.208)--(3.634,7.209)--(3.634,7.210)--(3.635,7.210)--(3.635,7.211)--(3.635,7.212)%
  --(3.636,7.212)--(3.636,7.213)--(3.636,7.214)--(3.636,7.215)--(3.637,7.215)--(3.637,7.216)%
  --(3.637,7.217)--(3.638,7.217)--(3.638,7.218)--(3.638,7.219)--(3.638,7.220)--(3.639,7.220)%
  --(3.639,7.221)--(3.639,7.222)--(3.640,7.222)--(3.640,7.223)--(3.640,7.224)--(3.640,7.225)%
  --(3.641,7.225)--(3.641,7.226)--(3.641,7.227)--(3.642,7.228)--(3.642,7.229)--(3.642,7.230)%
  --(3.643,7.230)--(3.643,7.231)--(3.643,7.232)--(3.643,7.233)--(3.644,7.233)--(3.644,7.234)%
  --(3.644,7.235)--(3.645,7.235)--(3.645,7.236)--(3.645,7.237)--(3.645,7.238)--(3.646,7.238)%
  --(3.646,7.239)--(3.646,7.240)--(3.647,7.240)--(3.647,7.241)--(3.647,7.242)--(3.647,7.243)%
  --(3.648,7.243)--(3.648,7.244)--(3.648,7.245)--(3.649,7.246)--(3.649,7.247)--(3.649,7.248)%
  --(3.650,7.248)--(3.650,7.249)--(3.650,7.250)--(3.650,7.251)--(3.651,7.251)--(3.651,7.252)%
  --(3.651,7.253)--(3.652,7.253)--(3.652,7.254)--(3.652,7.255)--(3.652,7.256)--(3.653,7.256)%
  --(3.653,7.257)--(3.653,7.258)--(3.654,7.258)--(3.654,7.259)--(3.654,7.260)--(3.654,7.261)%
  --(3.655,7.261)--(3.655,7.262)--(3.655,7.263)--(3.656,7.263)--(3.656,7.264)--(3.656,7.265)%
  --(3.656,7.266)--(3.657,7.266)--(3.657,7.267)--(3.657,7.268)--(3.657,7.269)--(3.658,7.269)%
  --(3.658,7.270)--(3.658,7.271)--(3.659,7.271)--(3.659,7.272)--(3.659,7.273)--(3.659,7.274)%
  --(3.660,7.274)--(3.660,7.275)--(3.660,7.276)--(3.661,7.276)--(3.661,7.277)--(3.661,7.278)%
  --(3.661,7.279)--(3.662,7.279)--(3.662,7.280)--(3.662,7.281)--(3.663,7.282)--(3.663,7.283)%
  --(3.663,7.284)--(3.664,7.284)--(3.664,7.285)--(3.664,7.286)--(3.665,7.287)--(3.665,7.288)%
  --(3.665,7.289)--(3.666,7.289)--(3.666,7.290)--(3.666,7.291)--(3.666,7.292)--(3.667,7.292)%
  --(3.667,7.293)--(3.667,7.294)--(3.668,7.294)--(3.668,7.295)--(3.668,7.296)--(3.668,7.297)%
  --(3.669,7.297)--(3.669,7.298)--(3.669,7.299)--(3.670,7.299)--(3.670,7.300)--(3.670,7.301)%
  --(3.670,7.302)--(3.671,7.302)--(3.671,7.303)--(3.671,7.304)--(3.672,7.304)--(3.672,7.305)%
  --(3.672,7.306)--(3.672,7.307)--(3.673,7.307)--(3.673,7.308)--(3.673,7.309)--(3.673,7.310)%
  --(3.674,7.310)--(3.674,7.311)--(3.674,7.312)--(3.675,7.312)--(3.675,7.313)--(3.675,7.314)%
  --(3.675,7.315)--(3.676,7.315)--(3.676,7.316)--(3.676,7.317)--(3.677,7.317)--(3.677,7.318)%
  --(3.677,7.319)--(3.677,7.320)--(3.678,7.320)--(3.678,7.321)--(3.678,7.322)--(3.679,7.323)%
  --(3.679,7.324)--(3.679,7.325)--(3.680,7.325)--(3.680,7.326)--(3.680,7.327)--(3.681,7.328)%
  --(3.681,7.329)--(3.681,7.330)--(3.682,7.330)--(3.682,7.331)--(3.682,7.332)--(3.682,7.333)%
  --(3.683,7.333)--(3.683,7.334)--(3.683,7.335)--(3.684,7.335)--(3.684,7.336)--(3.684,7.337)%
  --(3.684,7.338)--(3.685,7.338)--(3.685,7.339)--(3.685,7.340)--(3.686,7.340)--(3.686,7.341)%
  --(3.686,7.342)--(3.686,7.343)--(3.687,7.343)--(3.687,7.344)--(3.687,7.345)--(3.688,7.346)%
  --(3.688,7.347)--(3.688,7.348)--(3.689,7.348)--(3.689,7.349)--(3.689,7.350)--(3.689,7.351)%
  --(3.690,7.351)--(3.690,7.352)--(3.690,7.353)--(3.691,7.353)--(3.691,7.354)--(3.691,7.355)%
  --(3.691,7.356)--(3.692,7.356)--(3.692,7.357)--(3.692,7.358)--(3.693,7.358)--(3.693,7.359)%
  --(3.693,7.360)--(3.693,7.361)--(3.694,7.361)--(3.694,7.362)--(3.694,7.363)--(3.695,7.364)%
  --(3.695,7.365)--(3.695,7.366)--(3.696,7.366)--(3.696,7.367)--(3.696,7.368)--(3.697,7.369)%
  --(3.697,7.370)--(3.697,7.371)--(3.698,7.371)--(3.698,7.372)--(3.698,7.373)--(3.698,7.374)%
  --(3.699,7.374)--(3.699,7.375)--(3.699,7.376)--(3.700,7.376)--(3.700,7.377)--(3.700,7.378)%
  --(3.700,7.379)--(3.701,7.379)--(3.701,7.380)--(3.701,7.381)--(3.702,7.381)--(3.702,7.382)%
  --(3.702,7.383)--(3.702,7.384)--(3.703,7.384)--(3.703,7.385)--(3.703,7.386)--(3.703,7.387)%
  --(3.704,7.387)--(3.704,7.388)--(3.704,7.389)--(3.705,7.389)--(3.705,7.390)--(3.705,7.391)%
  --(3.705,7.392)--(3.706,7.392)--(3.706,7.393)--(3.706,7.394)--(3.707,7.394)--(3.707,7.395)%
  --(3.707,7.396)--(3.707,7.397)--(3.708,7.397)--(3.708,7.398)--(3.708,7.399)--(3.709,7.399)%
  --(3.709,7.400)--(3.709,7.401)--(3.709,7.402)--(3.710,7.402);
\gpcolor{color=gp lt color border}
\gpsetdashtype{gp dt solid}
\draw[gp path] (1.320,7.402)--(1.320,1.274)--(7.447,1.274)--(7.447,7.402)--cycle;
%% coordinates of the plot area
\gpdefrectangularnode{gp plot 1}{\pgfpoint{1.320cm}{1.274cm}}{\pgfpoint{7.447cm}{7.402cm}}
\end{tikzpicture}
%% gnuplot variables

    \caption{ブリュアン関数$B(x)$}
    \label{fig:Brillouin}
  \end{center}
\end{figure}