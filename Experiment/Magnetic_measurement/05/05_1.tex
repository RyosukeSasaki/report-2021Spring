\subsection{Ni円盤の反磁場補正}
\subsubsection{円盤型楕円体近似による反磁場補正}
\label{subsubsec:cons_hanjiba_daen}
Ni円盤試料を円盤型楕円体とみなすと反磁場係数$N_{\parallel}$, $N_{\perp}$は$m=5.05/1.85$を用いて以下で与えられる.
\begin{align}
  N_{\parallel}&=\frac{1}{2(m^2-1)}\left(\frac{m^2}{\sqrt{m^2-1}}\cos^{-1}\left(\frac{1}{m}\right)-1\right)\simeq0.194\\
  N_{\perp}&=1-2N_{\parallel}\simeq0.611
\end{align}
これと(\ref{equ:cons_Heff})を用いて外部磁場$H_{ext}$を$H_{eff}$に変換することで反磁場補正した磁気ヒステリシス曲線(図\ref{fig:cons_hanjiba_shape})を得る.
\begin{align}
  \label{equ:cons_Heff}
  H_{eff}=H_{ext}-\frac{NI}{\mu_0}
\end{align}
\begin{figure}[hptb]
  \begin{center}
    \begin{tikzpicture}[gnuplot]
%% generated with GNUPLOT 5.2p8 (Lua 5.3; terminal rev. Nov 2018, script rev. 108)
%% 2021年04月23日 17時07分28秒
\path (0.000,0.000) rectangle (12.500,8.750);
\gpcolor{color=gp lt color border}
\gpsetlinetype{gp lt border}
\gpsetdashtype{gp dt solid}
\gpsetlinewidth{1.00}
\draw[gp path] (2.997,0.985)--(3.177,0.985);
\draw[gp path] (10.454,0.985)--(10.274,0.985);
\node[gp node right] at (2.813,0.985) {$-0.6$};
\draw[gp path] (2.997,1.606)--(3.087,1.606);
\draw[gp path] (10.454,1.606)--(10.364,1.606);
\draw[gp path] (2.997,2.228)--(3.177,2.228);
\draw[gp path] (10.454,2.228)--(10.274,2.228);
\node[gp node right] at (2.813,2.228) {$-0.4$};
\draw[gp path] (2.997,2.849)--(3.087,2.849);
\draw[gp path] (10.454,2.849)--(10.364,2.849);
\draw[gp path] (2.997,3.470)--(3.177,3.470);
\draw[gp path] (10.454,3.470)--(10.274,3.470);
\node[gp node right] at (2.813,3.470) {$-0.2$};
\draw[gp path] (2.997,4.092)--(3.087,4.092);
\draw[gp path] (10.454,4.092)--(10.364,4.092);
\draw[gp path] (2.997,4.713)--(3.177,4.713);
\draw[gp path] (10.454,4.713)--(10.274,4.713);
\node[gp node right] at (2.813,4.713) {$0$};
\draw[gp path] (2.997,5.334)--(3.087,5.334);
\draw[gp path] (10.454,5.334)--(10.364,5.334);
\draw[gp path] (2.997,5.956)--(3.177,5.956);
\draw[gp path] (10.454,5.956)--(10.274,5.956);
\node[gp node right] at (2.813,5.956) {$0.2$};
\draw[gp path] (2.997,6.577)--(3.087,6.577);
\draw[gp path] (10.454,6.577)--(10.364,6.577);
\draw[gp path] (2.997,7.198)--(3.177,7.198);
\draw[gp path] (10.454,7.198)--(10.274,7.198);
\node[gp node right] at (2.813,7.198) {$0.4$};
\draw[gp path] (2.997,7.820)--(3.087,7.820);
\draw[gp path] (10.454,7.820)--(10.364,7.820);
\draw[gp path] (2.997,8.441)--(3.177,8.441);
\draw[gp path] (10.454,8.441)--(10.274,8.441);
\node[gp node right] at (2.813,8.441) {$0.6$};
\draw[gp path] (4.240,0.985)--(4.240,1.165);
\draw[gp path] (4.240,8.441)--(4.240,8.261);
\node[gp node center] at (4.240,0.677) {$-2.0$};
\draw[gp path] (5.483,0.985)--(5.483,1.075);
\draw[gp path] (5.483,8.441)--(5.483,8.351);
\draw[gp path] (6.726,0.985)--(6.726,1.165);
\draw[gp path] (6.726,8.441)--(6.726,8.261);
\node[gp node center] at (6.726,0.677) {$0.0$};
\draw[gp path] (7.968,0.985)--(7.968,1.075);
\draw[gp path] (7.968,8.441)--(7.968,8.351);
\draw[gp path] (9.211,0.985)--(9.211,1.165);
\draw[gp path] (9.211,8.441)--(9.211,8.261);
\node[gp node center] at (9.211,0.677) {$2.0$};
\draw[gp path] (2.997,8.441)--(2.997,0.985)--(10.454,0.985)--(10.454,8.441)--cycle;
\node[gp node center,rotate=-270] at (1.785,4.713) {磁化$J$ / $\si{Wb.m^{-2}}$};
\node[gp node center] at (6.725,0.215) {磁場$H$ / $\times 10^{5}\ \si{A.m^{-1}}$};
\draw[gp path] (3.181,6.875)--(3.181,8.261)--(6.489,8.261)--(6.489,6.875)--cycle;
\node[gp node right] at (5.205,7.799) {面直磁場};
\gpcolor{rgb color={0.000,0.000,0.000}}
\gpsetpointsize{4.00}
\gppoint{gp mark 6}{(6.731,4.661)}
\gppoint{gp mark 6}{(6.799,4.966)}
\gppoint{gp mark 6}{(6.865,5.260)}
\gppoint{gp mark 6}{(6.933,5.550)}
\gppoint{gp mark 6}{(7.005,5.836)}
\gppoint{gp mark 6}{(7.075,6.125)}
\gppoint{gp mark 6}{(7.159,6.393)}
\gppoint{gp mark 6}{(7.239,6.660)}
\gppoint{gp mark 6}{(7.339,6.918)}
\gppoint{gp mark 6}{(7.444,7.155)}
\gppoint{gp mark 6}{(7.564,7.382)}
\gppoint{gp mark 6}{(7.448,7.196)}
\gppoint{gp mark 6}{(7.332,6.980)}
\gppoint{gp mark 6}{(7.242,6.743)}
\gppoint{gp mark 6}{(7.152,6.485)}
\gppoint{gp mark 6}{(7.062,6.217)}
\gppoint{gp mark 6}{(6.992,5.939)}
\gppoint{gp mark 6}{(6.914,5.655)}
\gppoint{gp mark 6}{(6.842,5.368)}
\gppoint{gp mark 6}{(6.775,5.077)}
\gppoint{gp mark 6}{(6.704,4.758)}
\gppoint{gp mark 6}{(6.638,4.457)}
\gppoint{gp mark 6}{(6.575,4.172)}
\gppoint{gp mark 6}{(6.504,3.880)}
\gppoint{gp mark 6}{(6.437,3.580)}
\gppoint{gp mark 6}{(6.362,3.312)}
\gppoint{gp mark 6}{(6.287,3.023)}
\gppoint{gp mark 6}{(6.197,2.766)}
\gppoint{gp mark 6}{(6.102,2.508)}
\gppoint{gp mark 6}{(5.997,2.271)}
\gppoint{gp mark 6}{(5.882,2.044)}
\gppoint{gp mark 6}{(5.988,2.230)}
\gppoint{gp mark 6}{(6.104,2.446)}
\gppoint{gp mark 6}{(6.199,2.693)}
\gppoint{gp mark 6}{(6.289,2.941)}
\gppoint{gp mark 6}{(6.374,3.209)}
\gppoint{gp mark 6}{(6.454,3.487)}
\gppoint{gp mark 6}{(6.529,3.774)}
\gppoint{gp mark 6}{(6.596,4.056)}
\gppoint{gp mark 6}{(6.665,4.349)}
\gppoint{gp mark 6}{(6.735,4.668)}
\gppoint{gp mark 6}{(6.800,4.962)}
\gppoint{gp mark 6}{(6.864,5.256)}
\gppoint{gp mark 6}{(6.932,5.551)}
\gppoint{gp mark 6}{(7.005,5.836)}
\gppoint{gp mark 6}{(7.075,6.125)}
\gppoint{gp mark 6}{(7.154,6.403)}
\gppoint{gp mark 6}{(7.239,6.671)}
\gppoint{gp mark 6}{(7.334,6.918)}
\gppoint{gp mark 6}{(7.444,7.155)}
\gppoint{gp mark 6}{(7.570,7.371)}
\gppoint{gp mark 6}{(5.847,7.799)}
\gpcolor{color=gp lt color border}
\node[gp node right] at (5.205,7.337) {面内磁場};
\gpcolor{rgb color={0.000,0.000,0.000}}
\gppoint{gp mark 7}{(6.726,4.713)}
\gppoint{gp mark 7}{(6.577,6.300)}
\gppoint{gp mark 7}{(6.588,7.361)}
\gppoint{gp mark 7}{(6.779,7.876)}
\gppoint{gp mark 7}{(7.067,8.062)}
\gppoint{gp mark 7}{(7.395,8.134)}
\gppoint{gp mark 7}{(7.736,8.165)}
\gppoint{gp mark 7}{(8.070,8.185)}
\gppoint{gp mark 7}{(8.409,8.206)}
\gppoint{gp mark 7}{(8.740,8.206)}
\gppoint{gp mark 7}{(9.071,8.206)}
\gppoint{gp mark 7}{(8.779,8.206)}
\gppoint{gp mark 7}{(8.471,8.196)}
\gppoint{gp mark 7}{(8.138,8.185)}
\gppoint{gp mark 7}{(7.804,8.165)}
\gppoint{gp mark 7}{(7.468,8.134)}
\gppoint{gp mark 7}{(7.137,8.072)}
\gppoint{gp mark 7}{(6.836,7.928)}
\gppoint{gp mark 7}{(6.607,7.536)}
\gppoint{gp mark 7}{(6.535,6.630)}
\gppoint{gp mark 7}{(6.671,4.970)}
\gppoint{gp mark 7}{(6.833,3.312)}
\gppoint{gp mark 7}{(6.852,2.116)}
\gppoint{gp mark 7}{(6.669,1.560)}
\gppoint{gp mark 7}{(6.381,1.374)}
\gppoint{gp mark 7}{(6.053,1.302)}
\gppoint{gp mark 7}{(5.712,1.271)}
\gppoint{gp mark 7}{(5.378,1.251)}
\gppoint{gp mark 7}{(5.038,1.230)}
\gppoint{gp mark 7}{(4.698,1.230)}
\gppoint{gp mark 7}{(4.362,1.230)}
\gppoint{gp mark 7}{(4.651,1.241)}
\gppoint{gp mark 7}{(4.967,1.241)}
\gppoint{gp mark 7}{(5.300,1.251)}
\gppoint{gp mark 7}{(5.629,1.271)}
\gppoint{gp mark 7}{(5.965,1.302)}
\gppoint{gp mark 7}{(6.291,1.364)}
\gppoint{gp mark 7}{(6.597,1.508)}
\gppoint{gp mark 7}{(6.831,1.900)}
\gppoint{gp mark 7}{(6.901,2.827)}
\gppoint{gp mark 7}{(6.765,4.467)}
\gppoint{gp mark 7}{(6.605,6.125)}
\gppoint{gp mark 7}{(6.589,7.310)}
\gppoint{gp mark 7}{(6.771,7.856)}
\gppoint{gp mark 7}{(7.064,8.041)}
\gppoint{gp mark 7}{(7.397,8.113)}
\gppoint{gp mark 7}{(7.737,8.144)}
\gppoint{gp mark 7}{(8.067,8.165)}
\gppoint{gp mark 7}{(8.419,8.175)}
\gppoint{gp mark 7}{(8.747,8.185)}
\gppoint{gp mark 7}{(9.079,8.196)}
\gppoint{gp mark 7}{(5.847,7.337)}
\gpcolor{color=gp lt color border}
\draw[gp path] (2.997,8.441)--(2.997,0.985)--(10.454,0.985)--(10.454,8.441)--cycle;
%% coordinates of the plot area
\gpdefrectangularnode{gp plot 1}{\pgfpoint{2.997cm}{0.985cm}}{\pgfpoint{10.454cm}{8.441cm}}
\end{tikzpicture}
%% gnuplot variables

    \caption{楕円体近似から算出した反磁場係数による反磁場補正}
    \label{fig:cons_hanjiba_shape}
  \end{center}
\end{figure}
\subsubsection{磁気トルクによる反磁場補正}
(\ref{equ:calc_alpha})から磁気トルクの最大値$L_{max}$は
\begin{align}
  L_{max}=\frac{1}{2\mu_0}\left(N_{\perp}-N_{\parallel}\right)I_s^2V
\end{align}
であり,さらに対称性から$N_{\perp}=1-2N_{\parallel}$の関係があるので
\begin{align}
  L_{max}=\frac{1}{2\mu_0}\left(1-3N_{\parallel}\right)I_s^2V\nonumber\\
  \therefore N_{\parallel}=\frac{1}{3}\left(1-2\frac{L_{max}\mu_0}{I_s^2V}\right)
\end{align}
であえう.ここで$L_{max}/V$は単位体積あたりのトルクの最大値であり,これは$9.14\times10^4\ \si{\newton.\metre^{-2}}$であった.
したがって反磁場係数$N_{\parallel}$, $N_{\perp}$は
\begin{align}
  N_{\parallel}\simeq0.132\\
  N_{\perp}=1-2N_{\parallel}\simeq0.737
\end{align}
となる.よって\ref{subsubsec:cons_hanjiba_daen}と同様に反磁場補正をすることで図\ref{fig:cons_hanjiba_torque}を得る.
\begin{figure}[hptb]
  \begin{center}
    \begin{tikzpicture}[gnuplot]
%% generated with GNUPLOT 5.2p8 (Lua 5.3; terminal rev. Nov 2018, script rev. 108)
%% 2021年04月23日 17時08分14秒
\path (0.000,0.000) rectangle (12.500,8.750);
\gpcolor{color=gp lt color border}
\gpsetlinetype{gp lt border}
\gpsetdashtype{gp dt solid}
\gpsetlinewidth{1.00}
\draw[gp path] (2.997,0.985)--(3.177,0.985);
\draw[gp path] (10.454,0.985)--(10.274,0.985);
\node[gp node right] at (2.813,0.985) {$-0.6$};
\draw[gp path] (2.997,1.606)--(3.087,1.606);
\draw[gp path] (10.454,1.606)--(10.364,1.606);
\draw[gp path] (2.997,2.228)--(3.177,2.228);
\draw[gp path] (10.454,2.228)--(10.274,2.228);
\node[gp node right] at (2.813,2.228) {$-0.4$};
\draw[gp path] (2.997,2.849)--(3.087,2.849);
\draw[gp path] (10.454,2.849)--(10.364,2.849);
\draw[gp path] (2.997,3.470)--(3.177,3.470);
\draw[gp path] (10.454,3.470)--(10.274,3.470);
\node[gp node right] at (2.813,3.470) {$-0.2$};
\draw[gp path] (2.997,4.092)--(3.087,4.092);
\draw[gp path] (10.454,4.092)--(10.364,4.092);
\draw[gp path] (2.997,4.713)--(3.177,4.713);
\draw[gp path] (10.454,4.713)--(10.274,4.713);
\node[gp node right] at (2.813,4.713) {$0$};
\draw[gp path] (2.997,5.334)--(3.087,5.334);
\draw[gp path] (10.454,5.334)--(10.364,5.334);
\draw[gp path] (2.997,5.956)--(3.177,5.956);
\draw[gp path] (10.454,5.956)--(10.274,5.956);
\node[gp node right] at (2.813,5.956) {$0.2$};
\draw[gp path] (2.997,6.577)--(3.087,6.577);
\draw[gp path] (10.454,6.577)--(10.364,6.577);
\draw[gp path] (2.997,7.198)--(3.177,7.198);
\draw[gp path] (10.454,7.198)--(10.274,7.198);
\node[gp node right] at (2.813,7.198) {$0.4$};
\draw[gp path] (2.997,7.820)--(3.087,7.820);
\draw[gp path] (10.454,7.820)--(10.364,7.820);
\draw[gp path] (2.997,8.441)--(3.177,8.441);
\draw[gp path] (10.454,8.441)--(10.274,8.441);
\node[gp node right] at (2.813,8.441) {$0.6$};
\draw[gp path] (4.240,0.985)--(4.240,1.165);
\draw[gp path] (4.240,8.441)--(4.240,8.261);
\node[gp node center] at (4.240,0.677) {$-2.0$};
\draw[gp path] (5.483,0.985)--(5.483,1.075);
\draw[gp path] (5.483,8.441)--(5.483,8.351);
\draw[gp path] (6.726,0.985)--(6.726,1.165);
\draw[gp path] (6.726,8.441)--(6.726,8.261);
\node[gp node center] at (6.726,0.677) {$0.0$};
\draw[gp path] (7.968,0.985)--(7.968,1.075);
\draw[gp path] (7.968,8.441)--(7.968,8.351);
\draw[gp path] (9.211,0.985)--(9.211,1.165);
\draw[gp path] (9.211,8.441)--(9.211,8.261);
\node[gp node center] at (9.211,0.677) {$2.0$};
\draw[gp path] (2.997,8.441)--(2.997,0.985)--(10.454,0.985)--(10.454,8.441)--cycle;
\node[gp node center,rotate=-270] at (1.785,4.713) {磁化$J$ / $\si{Wb.m^{-2}}$};
\node[gp node center] at (6.725,0.215) {磁場$H$ / $\times 10^{5}\ \si{A.m^{-1}}$};
\draw[gp path] (3.181,6.875)--(3.181,8.261)--(6.489,8.261)--(6.489,6.875)--cycle;
\node[gp node right] at (5.205,7.799) {面直磁場};
\gpcolor{rgb color={0.000,0.000,0.000}}
\gpsetpointsize{4.00}
\gppoint{gp mark 6}{(6.741,4.661)}
\gppoint{gp mark 6}{(6.748,4.966)}
\gppoint{gp mark 6}{(6.755,5.260)}
\gppoint{gp mark 6}{(6.766,5.550)}
\gppoint{gp mark 6}{(6.781,5.836)}
\gppoint{gp mark 6}{(6.793,6.125)}
\gppoint{gp mark 6}{(6.824,6.393)}
\gppoint{gp mark 6}{(6.851,6.660)}
\gppoint{gp mark 6}{(6.899,6.918)}
\gppoint{gp mark 6}{(6.957,7.155)}
\gppoint{gp mark 6}{(7.032,7.382)}
\gppoint{gp mark 6}{(6.952,7.196)}
\gppoint{gp mark 6}{(6.880,6.980)}
\gppoint{gp mark 6}{(6.837,6.743)}
\gppoint{gp mark 6}{(6.798,6.485)}
\gppoint{gp mark 6}{(6.762,6.217)}
\gppoint{gp mark 6}{(6.748,5.939)}
\gppoint{gp mark 6}{(6.726,5.655)}
\gppoint{gp mark 6}{(6.711,5.368)}
\gppoint{gp mark 6}{(6.702,5.077)}
\gppoint{gp mark 6}{(6.695,4.758)}
\gppoint{gp mark 6}{(6.689,4.457)}
\gppoint{gp mark 6}{(6.683,4.172)}
\gppoint{gp mark 6}{(6.670,3.880)}
\gppoint{gp mark 6}{(6.663,3.580)}
\gppoint{gp mark 6}{(6.641,3.312)}
\gppoint{gp mark 6}{(6.624,3.023)}
\gppoint{gp mark 6}{(6.586,2.766)}
\gppoint{gp mark 6}{(6.542,2.508)}
\gppoint{gp mark 6}{(6.484,2.271)}
\gppoint{gp mark 6}{(6.414,2.044)}
\gppoint{gp mark 6}{(6.484,2.230)}
\gppoint{gp mark 6}{(6.556,2.446)}
\gppoint{gp mark 6}{(6.602,2.693)}
\gppoint{gp mark 6}{(6.643,2.941)}
\gppoint{gp mark 6}{(6.674,3.209)}
\gppoint{gp mark 6}{(6.699,3.487)}
\gppoint{gp mark 6}{(6.717,3.774)}
\gppoint{gp mark 6}{(6.727,4.056)}
\gppoint{gp mark 6}{(6.737,4.349)}
\gppoint{gp mark 6}{(6.744,4.668)}
\gppoint{gp mark 6}{(6.750,4.962)}
\gppoint{gp mark 6}{(6.755,5.256)}
\gppoint{gp mark 6}{(6.765,5.551)}
\gppoint{gp mark 6}{(6.781,5.836)}
\gppoint{gp mark 6}{(6.793,6.125)}
\gppoint{gp mark 6}{(6.817,6.403)}
\gppoint{gp mark 6}{(6.848,6.671)}
\gppoint{gp mark 6}{(6.894,6.918)}
\gppoint{gp mark 6}{(6.957,7.155)}
\gppoint{gp mark 6}{(7.039,7.371)}
\gppoint{gp mark 6}{(5.847,7.799)}
\gpcolor{color=gp lt color border}
\node[gp node right] at (5.205,7.337) {面内磁場};
\gpcolor{rgb color={0.000,0.000,0.000}}
\gppoint{gp mark 7}{(6.726,4.713)}
\gppoint{gp mark 7}{(6.736,6.300)}
\gppoint{gp mark 7}{(6.852,7.361)}
\gppoint{gp mark 7}{(7.095,7.876)}
\gppoint{gp mark 7}{(7.401,8.062)}
\gppoint{gp mark 7}{(7.737,8.134)}
\gppoint{gp mark 7}{(8.080,8.165)}
\gppoint{gp mark 7}{(8.417,8.185)}
\gppoint{gp mark 7}{(8.758,8.206)}
\gppoint{gp mark 7}{(9.089,8.206)}
\gppoint{gp mark 7}{(9.420,8.206)}
\gppoint{gp mark 7}{(9.128,8.206)}
\gppoint{gp mark 7}{(8.818,8.196)}
\gppoint{gp mark 7}{(8.485,8.185)}
\gppoint{gp mark 7}{(8.149,8.165)}
\gppoint{gp mark 7}{(7.809,8.134)}
\gppoint{gp mark 7}{(7.472,8.072)}
\gppoint{gp mark 7}{(7.157,7.928)}
\gppoint{gp mark 7}{(6.888,7.536)}
\gppoint{gp mark 7}{(6.727,6.630)}
\gppoint{gp mark 7}{(6.697,4.970)}
\gppoint{gp mark 7}{(6.693,3.312)}
\gppoint{gp mark 7}{(6.593,2.116)}
\gppoint{gp mark 7}{(6.354,1.560)}
\gppoint{gp mark 7}{(6.048,1.374)}
\gppoint{gp mark 7}{(5.712,1.302)}
\gppoint{gp mark 7}{(5.368,1.271)}
\gppoint{gp mark 7}{(5.032,1.251)}
\gppoint{gp mark 7}{(4.691,1.230)}
\gppoint{gp mark 7}{(4.350,1.230)}
\gppoint{gp mark 7}{(4.015,1.230)}
\gppoint{gp mark 7}{(4.304,1.241)}
\gppoint{gp mark 7}{(4.621,1.241)}
\gppoint{gp mark 7}{(4.954,1.251)}
\gppoint{gp mark 7}{(5.286,1.271)}
\gppoint{gp mark 7}{(5.625,1.302)}
\gppoint{gp mark 7}{(5.957,1.364)}
\gppoint{gp mark 7}{(6.277,1.508)}
\gppoint{gp mark 7}{(6.551,1.900)}
\gppoint{gp mark 7}{(6.713,2.827)}
\gppoint{gp mark 7}{(6.741,4.467)}
\gppoint{gp mark 7}{(6.746,6.125)}
\gppoint{gp mark 7}{(6.848,7.310)}
\gppoint{gp mark 7}{(7.084,7.856)}
\gppoint{gp mark 7}{(7.396,8.041)}
\gppoint{gp mark 7}{(7.736,8.113)}
\gppoint{gp mark 7}{(8.080,8.144)}
\gppoint{gp mark 7}{(8.411,8.165)}
\gppoint{gp mark 7}{(8.764,8.175)}
\gppoint{gp mark 7}{(9.093,8.185)}
\gppoint{gp mark 7}{(9.427,8.196)}
\gppoint{gp mark 7}{(5.847,7.337)}
\gpcolor{color=gp lt color border}
\draw[gp path] (2.997,8.441)--(2.997,0.985)--(10.454,0.985)--(10.454,8.441)--cycle;
%% coordinates of the plot area
\gpdefrectangularnode{gp plot 1}{\pgfpoint{2.997cm}{0.985cm}}{\pgfpoint{10.454cm}{8.441cm}}
\end{tikzpicture}
%% gnuplot variables

    \caption{磁気トルクから算出した反磁場係数による反磁場補正}
    \label{fig:cons_hanjiba_torque}
  \end{center}
\end{figure}
\subsubsection{磁気ヒステリシス曲線の傾きによる反磁場補正}
(\ref{equ:hysteresis_hanjiba})から
\begin{align}
  N=\mu_0\frac{H_{ext}}{I}
\end{align}
である.ここで$I/H_{ext}$は磁気ヒステリシス曲線の傾きなので,これから反磁場係数を求めることができる.
ここでは電磁石電流を$0\rightarrow0.3$にしたときの磁場の増加を${\rm d}x$,磁化の増加を${\rm d}y$とし${\rm d}y/{\rm d}x$を原点近傍での磁気ヒステリシス曲線の傾きとした.
このとき反磁場係数$N_{\parallel}$, $N_{\perp}$は
\begin{align}
  N_{\parallel}=\mu_0\frac{{\rm d}x}{{\rm d}y}=\mu_0\frac{2.55\times10^{-1}}{2.76\times10^{4}}\simeq0.136\\
  N_{\perp}=\mu_0\frac{{\rm d}x}{{\rm d}y}=\mu_0\frac{4.92\times10^{-2}}{2.95\times10^{4}}\simeq0.752
\end{align}
となる.よって\ref{subsubsec:cons_hanjiba_daen}と同様に反磁場補正をすることで図\ref{fig:cons_hanjiba_slope}を得る.
また$N_{\parallel}\times2+N_{\perp}=1.02$である.これは相対誤差$2\ \%$のズレであり$N_x+N_y+N_z=1$の条件を満たしていると言える.
\begin{figure}[hptb]
  \begin{center}
    \begin{tikzpicture}[gnuplot]
%% generated with GNUPLOT 5.2p8 (Lua 5.3; terminal rev. Nov 2018, script rev. 108)
%% 2021年04月23日 17時07分48秒
\path (0.000,0.000) rectangle (12.500,8.750);
\gpcolor{color=gp lt color border}
\gpsetlinetype{gp lt border}
\gpsetdashtype{gp dt solid}
\gpsetlinewidth{1.00}
\draw[gp path] (2.997,0.985)--(3.177,0.985);
\draw[gp path] (10.454,0.985)--(10.274,0.985);
\node[gp node right] at (2.813,0.985) {$-0.6$};
\draw[gp path] (2.997,1.606)--(3.087,1.606);
\draw[gp path] (10.454,1.606)--(10.364,1.606);
\draw[gp path] (2.997,2.228)--(3.177,2.228);
\draw[gp path] (10.454,2.228)--(10.274,2.228);
\node[gp node right] at (2.813,2.228) {$-0.4$};
\draw[gp path] (2.997,2.849)--(3.087,2.849);
\draw[gp path] (10.454,2.849)--(10.364,2.849);
\draw[gp path] (2.997,3.470)--(3.177,3.470);
\draw[gp path] (10.454,3.470)--(10.274,3.470);
\node[gp node right] at (2.813,3.470) {$-0.2$};
\draw[gp path] (2.997,4.092)--(3.087,4.092);
\draw[gp path] (10.454,4.092)--(10.364,4.092);
\draw[gp path] (2.997,4.713)--(3.177,4.713);
\draw[gp path] (10.454,4.713)--(10.274,4.713);
\node[gp node right] at (2.813,4.713) {$0$};
\draw[gp path] (2.997,5.334)--(3.087,5.334);
\draw[gp path] (10.454,5.334)--(10.364,5.334);
\draw[gp path] (2.997,5.956)--(3.177,5.956);
\draw[gp path] (10.454,5.956)--(10.274,5.956);
\node[gp node right] at (2.813,5.956) {$0.2$};
\draw[gp path] (2.997,6.577)--(3.087,6.577);
\draw[gp path] (10.454,6.577)--(10.364,6.577);
\draw[gp path] (2.997,7.198)--(3.177,7.198);
\draw[gp path] (10.454,7.198)--(10.274,7.198);
\node[gp node right] at (2.813,7.198) {$0.4$};
\draw[gp path] (2.997,7.820)--(3.087,7.820);
\draw[gp path] (10.454,7.820)--(10.364,7.820);
\draw[gp path] (2.997,8.441)--(3.177,8.441);
\draw[gp path] (10.454,8.441)--(10.274,8.441);
\node[gp node right] at (2.813,8.441) {$0.6$};
\draw[gp path] (4.240,0.985)--(4.240,1.165);
\draw[gp path] (4.240,8.441)--(4.240,8.261);
\node[gp node center] at (4.240,0.677) {$-2.0$};
\draw[gp path] (5.483,0.985)--(5.483,1.075);
\draw[gp path] (5.483,8.441)--(5.483,8.351);
\draw[gp path] (6.726,0.985)--(6.726,1.165);
\draw[gp path] (6.726,8.441)--(6.726,8.261);
\node[gp node center] at (6.726,0.677) {$0.0$};
\draw[gp path] (7.968,0.985)--(7.968,1.075);
\draw[gp path] (7.968,8.441)--(7.968,8.351);
\draw[gp path] (9.211,0.985)--(9.211,1.165);
\draw[gp path] (9.211,8.441)--(9.211,8.261);
\node[gp node center] at (9.211,0.677) {$2.0$};
\draw[gp path] (2.997,8.441)--(2.997,0.985)--(10.454,0.985)--(10.454,8.441)--cycle;
\node[gp node center,rotate=-270] at (1.785,4.713) {磁化$J$ / $\si{Wb.m^{-2}}$};
\node[gp node center] at (6.725,0.215) {磁場$H$ / $\times 10^{5}\ \si{A.m^{-1}}$};
\draw[gp path] (3.181,6.875)--(3.181,8.261)--(6.489,8.261)--(6.489,6.875)--cycle;
\node[gp node right] at (5.205,7.799) {面直磁場};
\gpcolor{rgb color={0.000,0.000,0.000}}
\gpsetpointsize{4.00}
\gppoint{gp mark 6}{(6.742,4.661)}
\gppoint{gp mark 6}{(6.742,4.966)}
\gppoint{gp mark 6}{(6.743,5.260)}
\gppoint{gp mark 6}{(6.747,5.550)}
\gppoint{gp mark 6}{(6.754,5.836)}
\gppoint{gp mark 6}{(6.759,6.125)}
\gppoint{gp mark 6}{(6.784,6.393)}
\gppoint{gp mark 6}{(6.805,6.660)}
\gppoint{gp mark 6}{(6.847,6.918)}
\gppoint{gp mark 6}{(6.899,7.155)}
\gppoint{gp mark 6}{(6.969,7.382)}
\gppoint{gp mark 6}{(6.894,7.196)}
\gppoint{gp mark 6}{(6.826,6.980)}
\gppoint{gp mark 6}{(6.789,6.743)}
\gppoint{gp mark 6}{(6.756,6.485)}
\gppoint{gp mark 6}{(6.726,6.217)}
\gppoint{gp mark 6}{(6.719,5.939)}
\gppoint{gp mark 6}{(6.704,5.655)}
\gppoint{gp mark 6}{(6.696,5.368)}
\gppoint{gp mark 6}{(6.693,5.077)}
\gppoint{gp mark 6}{(6.694,4.758)}
\gppoint{gp mark 6}{(6.695,4.457)}
\gppoint{gp mark 6}{(6.696,4.172)}
\gppoint{gp mark 6}{(6.690,3.880)}
\gppoint{gp mark 6}{(6.690,3.580)}
\gppoint{gp mark 6}{(6.674,3.312)}
\gppoint{gp mark 6}{(6.664,3.023)}
\gppoint{gp mark 6}{(6.632,2.766)}
\gppoint{gp mark 6}{(6.594,2.508)}
\gppoint{gp mark 6}{(6.542,2.271)}
\gppoint{gp mark 6}{(6.478,2.044)}
\gppoint{gp mark 6}{(6.543,2.230)}
\gppoint{gp mark 6}{(6.610,2.446)}
\gppoint{gp mark 6}{(6.650,2.693)}
\gppoint{gp mark 6}{(6.685,2.941)}
\gppoint{gp mark 6}{(6.710,3.209)}
\gppoint{gp mark 6}{(6.728,3.487)}
\gppoint{gp mark 6}{(6.739,3.774)}
\gppoint{gp mark 6}{(6.743,4.056)}
\gppoint{gp mark 6}{(6.746,4.349)}
\gppoint{gp mark 6}{(6.745,4.668)}
\gppoint{gp mark 6}{(6.744,4.962)}
\gppoint{gp mark 6}{(6.743,5.256)}
\gppoint{gp mark 6}{(6.745,5.551)}
\gppoint{gp mark 6}{(6.754,5.836)}
\gppoint{gp mark 6}{(6.759,6.125)}
\gppoint{gp mark 6}{(6.777,6.403)}
\gppoint{gp mark 6}{(6.802,6.671)}
\gppoint{gp mark 6}{(6.842,6.918)}
\gppoint{gp mark 6}{(6.899,7.155)}
\gppoint{gp mark 6}{(6.976,7.371)}
\gppoint{gp mark 6}{(5.847,7.799)}
\gpcolor{color=gp lt color border}
\node[gp node right] at (5.205,7.337) {面内磁場};
\gpcolor{rgb color={0.000,0.000,0.000}}
\gppoint{gp mark 7}{(6.726,4.713)}
\gppoint{gp mark 7}{(6.726,6.300)}
\gppoint{gp mark 7}{(6.835,7.361)}
\gppoint{gp mark 7}{(7.074,7.876)}
\gppoint{gp mark 7}{(7.380,8.062)}
\gppoint{gp mark 7}{(7.715,8.134)}
\gppoint{gp mark 7}{(8.058,8.165)}
\gppoint{gp mark 7}{(8.395,8.185)}
\gppoint{gp mark 7}{(8.736,8.206)}
\gppoint{gp mark 7}{(9.067,8.206)}
\gppoint{gp mark 7}{(9.397,8.206)}
\gppoint{gp mark 7}{(9.105,8.206)}
\gppoint{gp mark 7}{(8.796,8.196)}
\gppoint{gp mark 7}{(8.463,8.185)}
\gppoint{gp mark 7}{(8.127,8.165)}
\gppoint{gp mark 7}{(7.788,8.134)}
\gppoint{gp mark 7}{(7.451,8.072)}
\gppoint{gp mark 7}{(7.136,7.928)}
\gppoint{gp mark 7}{(6.871,7.536)}
\gppoint{gp mark 7}{(6.714,6.630)}
\gppoint{gp mark 7}{(6.695,4.970)}
\gppoint{gp mark 7}{(6.702,3.312)}
\gppoint{gp mark 7}{(6.609,2.116)}
\gppoint{gp mark 7}{(6.374,1.560)}
\gppoint{gp mark 7}{(6.069,1.374)}
\gppoint{gp mark 7}{(5.734,1.302)}
\gppoint{gp mark 7}{(5.390,1.271)}
\gppoint{gp mark 7}{(5.054,1.251)}
\gppoint{gp mark 7}{(4.713,1.230)}
\gppoint{gp mark 7}{(4.373,1.230)}
\gppoint{gp mark 7}{(4.037,1.230)}
\gppoint{gp mark 7}{(4.327,1.241)}
\gppoint{gp mark 7}{(4.643,1.241)}
\gppoint{gp mark 7}{(4.976,1.251)}
\gppoint{gp mark 7}{(5.308,1.271)}
\gppoint{gp mark 7}{(5.646,1.302)}
\gppoint{gp mark 7}{(5.979,1.364)}
\gppoint{gp mark 7}{(6.298,1.508)}
\gppoint{gp mark 7}{(6.569,1.900)}
\gppoint{gp mark 7}{(6.725,2.827)}
\gppoint{gp mark 7}{(6.742,4.467)}
\gppoint{gp mark 7}{(6.737,6.125)}
\gppoint{gp mark 7}{(6.832,7.310)}
\gppoint{gp mark 7}{(7.064,7.856)}
\gppoint{gp mark 7}{(7.375,8.041)}
\gppoint{gp mark 7}{(7.714,8.113)}
\gppoint{gp mark 7}{(8.058,8.144)}
\gppoint{gp mark 7}{(8.389,8.165)}
\gppoint{gp mark 7}{(8.742,8.175)}
\gppoint{gp mark 7}{(9.071,8.185)}
\gppoint{gp mark 7}{(9.404,8.196)}
\gppoint{gp mark 7}{(5.847,7.337)}
\gpcolor{color=gp lt color border}
\draw[gp path] (2.997,8.441)--(2.997,0.985)--(10.454,0.985)--(10.454,8.441)--cycle;
%% coordinates of the plot area
\gpdefrectangularnode{gp plot 1}{\pgfpoint{2.997cm}{0.985cm}}{\pgfpoint{10.454cm}{8.441cm}}
\end{tikzpicture}
%% gnuplot variables

    \caption{磁気ヒステリシス曲線の傾きから算出した反磁場係数による反磁場補正}
    \label{fig:cons_hanjiba_slope}
  \end{center}
\end{figure}
\subsubsection{考察}
Niのような強磁性体は外部磁場0でも磁化を持つため,反磁場補正を施すと磁気ヒステリシス曲線は直角に立ち上がった形状になることが考えられる.
ここで3つの方法で行った反磁場補正を比較すると図\ref{fig:cons_hanjiba_torque},図\ref{fig:cons_hanjiba_slope}が予想通り直角に立ち上がった形状をしているのに対し,
図\ref{fig:cons_hanjiba_shape}は面直方向に対して過剰に,面内方向に対して弱く反磁場補正がかかっていることがわかる.
これは他の2つが実験値に基づく反磁場係数であるのに対して,図\ref{fig:cons_hanjiba_shape}は反磁場係数の計算に楕円体近似を用いており,実際の試料の反磁場係数と乖離が生じたためと考えられる.