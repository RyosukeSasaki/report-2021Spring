\subsection{Gd原子の磁気モーメントの算出}
(\ref{equ:Brillouin})において$B_J(a)$を$a$の1次まで取ると
\begin{align}
  \label{equ:cons_C-m}
  I=Ng\mu_BJ\frac{J+1}{3J}a=\frac{Ng^2\mu_B^2J(J+1)}{3kT}H\nonumber\\
  \therefore\chi=\frac{I}{H}=\frac{Ng^2\mu_B^2J(J+1)}{3kT}=\frac{Nm^2}{3kT}=\frac{C}{T}
\end{align}
となる.ここで$m$は原子1つあたりの磁気モーメント, $C$はキュリー定数である.
キュリーワイスの法則から
\begin{align}
  \chi=\frac{C}{T-\Theta}\Leftrightarrow\frac{1}{\chi}=\frac{T-\Theta}{C}
\end{align}
これは$T-(1/\chi)$グラフにおいて傾き$1/C$, $x$切片$\Theta$の直線になる.
図\ref{fig:cons_LaGd_T-chi}に$T-(1/\chi)$グラフを示す.
更に$260\si{\kelvin}$以降の直線部を最小二乗法(\ref{equ:saisyojijoho})を用いて直線でフィットすると図\ref{fig:cons_LaGd_T-chi_fit}を得る.
ただし測定は$1\si{\kelvin}$ごとに行っているが,グラフには$5\si{\kelvin}$ごとにプロットしている.
最小二乗法によりキュリー定数$C$とキュリー温度$\Theta$は
\begin{align}
  C&=8.91\times10^{-6}\ \si{\weber.\kelvin.\ampere^{-1}.\metre^{-1}}\\
  \Theta&=251\si{\kelvin}
\end{align}
である.ここで試料にGdのみが含まれていたとするとGdの密度$7.90\si{\gram.\centi\metre^{-3}}$,原子量$157$から単位体積あたりの原子数$N$は
\begin{align}
  N=\frac{7.90\ \si{\gram.\centi\metre^{-3}}}{157\ \si{\gram.\mole^{-1}}}N_A\ \si{\mole^{-1}}\times10^6=3.02\times10^{28}\ \si{\metre^{-3}}
\end{align}
となる\cite{alma990008542450204034}.したがって原子1つあたりの磁気モーメント$m$は
\begin{align}
  m&=\sqrt{\frac{3kC}{N}}\nonumber\\
  &=\sqrt{\frac{3\times1.38\times10^{-23}\si{\joule.\kelvin^{-1}} \times8.91\times10^{-6}\si{\weber.\kelvin.\ampere^{-1}.\metre^{-1}}}{3.02\times10^{28}\si{\metre^{-3}}}}\nonumber\\
  &=1.11\times10^{-28}\ \si{\weber.\metre}
\end{align}
となる.さらに(\ref{equ:Theta})から分子場係数$\omega$は
\begin{align}
  \omega&=\frac{3k}{Nm^2}\Theta\nonumber\\
  &=\frac{3\times1.38\times10^{-23}\si{\joule.\kelvin^{-1}}}{3.02\times10^{28}\si{\metre^{-3}}\times\left(1.11\times10^{-28}\ \si{\weber.\metre}\right)^2}251\si{\kelvin}\nonumber\\
  &=2.82\times10^7\si{\ampere.\metre.\weber^{-1}}
\end{align}
さらに$m$をボーア磁子$\mu_B$で表すと
\begin{align}
  m=9.48\mu_B
\end{align}
となる\cite{alma990008542450204034}.
Gdはフント則によればおよそ$7\mu_B$の磁気モーメントを持つが,以上からLaを添加したことでより大きな磁気モーメントが得られていることがわかる\cite{alma990008542450204034}.
\begin{figure}[hptb]
  \begin{center}
    \begin{tikzpicture}[gnuplot]
%% generated with GNUPLOT 5.2p8 (Lua 5.3; terminal rev. Nov 2018, script rev. 108)
%% 2021年04月26日 14時11分13秒
\path (0.000,0.000) rectangle (12.500,8.750);
\gpcolor{color=gp lt color border}
\gpsetlinetype{gp lt border}
\gpsetdashtype{gp dt solid}
\gpsetlinewidth{1.00}
\draw[gp path] (2.905,0.985)--(3.085,0.985);
\draw[gp path] (10.362,0.985)--(10.182,0.985);
\node[gp node right] at (2.721,0.985) {$0.0$};
\draw[gp path] (2.905,1.778)--(2.995,1.778);
\draw[gp path] (10.362,1.778)--(10.272,1.778);
\draw[gp path] (2.905,2.571)--(3.085,2.571);
\draw[gp path] (10.362,2.571)--(10.182,2.571);
\node[gp node right] at (2.721,2.571) {$1.0$};
\draw[gp path] (2.905,3.365)--(2.995,3.365);
\draw[gp path] (10.362,3.365)--(10.272,3.365);
\draw[gp path] (2.905,4.158)--(3.085,4.158);
\draw[gp path] (10.362,4.158)--(10.182,4.158);
\node[gp node right] at (2.721,4.158) {$2.0$};
\draw[gp path] (2.905,4.951)--(2.995,4.951);
\draw[gp path] (10.362,4.951)--(10.272,4.951);
\draw[gp path] (2.905,5.744)--(3.085,5.744);
\draw[gp path] (10.362,5.744)--(10.182,5.744);
\node[gp node right] at (2.721,5.744) {$3.0$};
\draw[gp path] (2.905,6.537)--(2.995,6.537);
\draw[gp path] (10.362,6.537)--(10.272,6.537);
\draw[gp path] (2.905,7.331)--(3.085,7.331);
\draw[gp path] (10.362,7.331)--(10.182,7.331);
\node[gp node right] at (2.721,7.331) {$4.0$};
\draw[gp path] (2.905,8.124)--(2.995,8.124);
\draw[gp path] (10.362,8.124)--(10.272,8.124);
\draw[gp path] (3.069,0.985)--(3.069,1.075);
\draw[gp path] (3.069,8.441)--(3.069,8.351);
\draw[gp path] (3.886,0.985)--(3.886,1.165);
\draw[gp path] (3.886,8.441)--(3.886,8.261);
\node[gp node center] at (3.886,0.677) {$100$};
\draw[gp path] (4.704,0.985)--(4.704,1.075);
\draw[gp path] (4.704,8.441)--(4.704,8.351);
\draw[gp path] (5.521,0.985)--(5.521,1.165);
\draw[gp path] (5.521,8.441)--(5.521,8.261);
\node[gp node center] at (5.521,0.677) {$150$};
\draw[gp path] (6.339,0.985)--(6.339,1.075);
\draw[gp path] (6.339,8.441)--(6.339,8.351);
\draw[gp path] (7.157,0.985)--(7.157,1.165);
\draw[gp path] (7.157,8.441)--(7.157,8.261);
\node[gp node center] at (7.157,0.677) {$200$};
\draw[gp path] (7.974,0.985)--(7.974,1.075);
\draw[gp path] (7.974,8.441)--(7.974,8.351);
\draw[gp path] (8.792,0.985)--(8.792,1.165);
\draw[gp path] (8.792,8.441)--(8.792,8.261);
\node[gp node center] at (8.792,0.677) {$250$};
\draw[gp path] (9.610,0.985)--(9.610,1.075);
\draw[gp path] (9.610,8.441)--(9.610,8.351);
\draw[gp path] (2.905,8.441)--(2.905,0.985)--(10.362,0.985)--(10.362,8.441)--cycle;
\node[gp node center,rotate=-270] at (1.877,4.713) {$\chi^{-1}$ / $\times 10^{6}$};
\node[gp node center] at (6.633,0.215) {温度$T$ / $\si{K}$};
\gpcolor{rgb color={0.000,0.000,0.000}}
\gpsetpointsize{4.00}
\gppoint{gp mark 6}{(3.069,1.304)}
\gppoint{gp mark 6}{(3.396,1.306)}
\gppoint{gp mark 6}{(3.559,1.307)}
\gppoint{gp mark 6}{(3.723,1.308)}
\gppoint{gp mark 6}{(3.886,1.309)}
\gppoint{gp mark 6}{(4.050,1.310)}
\gppoint{gp mark 6}{(4.213,1.312)}
\gppoint{gp mark 6}{(4.377,1.313)}
\gppoint{gp mark 6}{(4.540,1.314)}
\gppoint{gp mark 6}{(4.704,1.316)}
\gppoint{gp mark 6}{(4.867,1.317)}
\gppoint{gp mark 6}{(5.031,1.319)}
\gppoint{gp mark 6}{(5.194,1.320)}
\gppoint{gp mark 6}{(5.358,1.322)}
\gppoint{gp mark 6}{(5.521,1.324)}
\gppoint{gp mark 6}{(5.685,1.326)}
\gppoint{gp mark 6}{(5.881,1.328)}
\gppoint{gp mark 6}{(6.045,1.329)}
\gppoint{gp mark 6}{(6.208,1.331)}
\gppoint{gp mark 6}{(6.372,1.331)}
\gppoint{gp mark 6}{(6.535,1.333)}
\gppoint{gp mark 6}{(6.699,1.335)}
\gppoint{gp mark 6}{(6.862,1.337)}
\gppoint{gp mark 6}{(7.026,1.338)}
\gppoint{gp mark 6}{(7.190,1.341)}
\gppoint{gp mark 6}{(7.353,1.343)}
\gppoint{gp mark 6}{(7.517,1.346)}
\gppoint{gp mark 6}{(7.680,1.349)}
\gppoint{gp mark 6}{(7.844,1.353)}
\gppoint{gp mark 6}{(8.007,1.358)}
\gppoint{gp mark 6}{(8.171,1.367)}
\gppoint{gp mark 6}{(8.334,1.381)}
\gppoint{gp mark 6}{(8.498,1.404)}
\gppoint{gp mark 6}{(8.661,1.443)}
\gppoint{gp mark 6}{(8.825,1.544)}
\gppoint{gp mark 6}{(8.988,1.936)}
\gppoint{gp mark 6}{(9.152,2.748)}
\gppoint{gp mark 6}{(9.315,3.619)}
\gppoint{gp mark 6}{(9.479,4.547)}
\gppoint{gp mark 6}{(9.642,5.340)}
\gppoint{gp mark 6}{(9.806,6.215)}
\gppoint{gp mark 6}{(9.970,7.125)}
\gppoint{gp mark 6}{(10.133,8.071)}
\gpcolor{color=gp lt color border}
\draw[gp path] (2.905,8.441)--(2.905,0.985)--(10.362,0.985)--(10.362,8.441)--cycle;
%% coordinates of the plot area
\gpdefrectangularnode{gp plot 1}{\pgfpoint{2.905cm}{0.985cm}}{\pgfpoint{10.362cm}{8.441cm}}
\end{tikzpicture}
%% gnuplot variables

    \caption{La-Gd合金の磁化率の逆数の温度依存性}
    \label{fig:cons_LaGd_T-chi}
  \end{center}
\end{figure}
\begin{figure}[hptb]
  \begin{center}
    \begin{tikzpicture}[gnuplot]
%% generated with GNUPLOT 5.2p8 (Lua 5.3; terminal rev. Nov 2018, script rev. 108)
%% 2021年04月26日 14時11分23秒
\path (0.000,0.000) rectangle (12.500,8.750);
\gpcolor{color=gp lt color border}
\gpsetlinetype{gp lt border}
\gpsetdashtype{gp dt solid}
\gpsetlinewidth{1.00}
\draw[gp path] (2.905,0.985)--(3.085,0.985);
\draw[gp path] (10.362,0.985)--(10.182,0.985);
\node[gp node right] at (2.721,0.985) {$0.0$};
\draw[gp path] (2.905,1.778)--(2.995,1.778);
\draw[gp path] (10.362,1.778)--(10.272,1.778);
\draw[gp path] (2.905,2.571)--(3.085,2.571);
\draw[gp path] (10.362,2.571)--(10.182,2.571);
\node[gp node right] at (2.721,2.571) {$1.0$};
\draw[gp path] (2.905,3.365)--(2.995,3.365);
\draw[gp path] (10.362,3.365)--(10.272,3.365);
\draw[gp path] (2.905,4.158)--(3.085,4.158);
\draw[gp path] (10.362,4.158)--(10.182,4.158);
\node[gp node right] at (2.721,4.158) {$2.0$};
\draw[gp path] (2.905,4.951)--(2.995,4.951);
\draw[gp path] (10.362,4.951)--(10.272,4.951);
\draw[gp path] (2.905,5.744)--(3.085,5.744);
\draw[gp path] (10.362,5.744)--(10.182,5.744);
\node[gp node right] at (2.721,5.744) {$3.0$};
\draw[gp path] (2.905,6.537)--(2.995,6.537);
\draw[gp path] (10.362,6.537)--(10.272,6.537);
\draw[gp path] (2.905,7.331)--(3.085,7.331);
\draw[gp path] (10.362,7.331)--(10.182,7.331);
\node[gp node right] at (2.721,7.331) {$4.0$};
\draw[gp path] (2.905,8.124)--(2.995,8.124);
\draw[gp path] (10.362,8.124)--(10.272,8.124);
\draw[gp path] (3.069,0.985)--(3.069,1.075);
\draw[gp path] (3.069,8.441)--(3.069,8.351);
\draw[gp path] (3.886,0.985)--(3.886,1.165);
\draw[gp path] (3.886,8.441)--(3.886,8.261);
\node[gp node center] at (3.886,0.677) {$100$};
\draw[gp path] (4.704,0.985)--(4.704,1.075);
\draw[gp path] (4.704,8.441)--(4.704,8.351);
\draw[gp path] (5.521,0.985)--(5.521,1.165);
\draw[gp path] (5.521,8.441)--(5.521,8.261);
\node[gp node center] at (5.521,0.677) {$150$};
\draw[gp path] (6.339,0.985)--(6.339,1.075);
\draw[gp path] (6.339,8.441)--(6.339,8.351);
\draw[gp path] (7.157,0.985)--(7.157,1.165);
\draw[gp path] (7.157,8.441)--(7.157,8.261);
\node[gp node center] at (7.157,0.677) {$200$};
\draw[gp path] (7.974,0.985)--(7.974,1.075);
\draw[gp path] (7.974,8.441)--(7.974,8.351);
\draw[gp path] (8.792,0.985)--(8.792,1.165);
\draw[gp path] (8.792,8.441)--(8.792,8.261);
\node[gp node center] at (8.792,0.677) {$250$};
\draw[gp path] (9.610,0.985)--(9.610,1.075);
\draw[gp path] (9.610,8.441)--(9.610,8.351);
\draw[gp path] (2.905,8.441)--(2.905,0.985)--(10.362,0.985)--(10.362,8.441)--cycle;
\node[gp node center,rotate=-270] at (1.877,4.713) {$\chi^{-1}$ / $\times 10^{6}$};
\node[gp node center] at (6.633,0.215) {温度$T$ / $\si{K}$};
\draw[gp path] (3.089,6.875)--(3.089,8.261)--(5.845,8.261)--(5.845,6.875)--cycle;
\node[gp node right] at (4.561,7.799) {測定値};
\gpcolor{rgb color={0.000,0.000,0.000}}
\gpsetpointsize{4.00}
\gppoint{gp mark 6}{(3.069,1.304)}
\gppoint{gp mark 6}{(3.396,1.306)}
\gppoint{gp mark 6}{(3.559,1.307)}
\gppoint{gp mark 6}{(3.723,1.308)}
\gppoint{gp mark 6}{(3.886,1.309)}
\gppoint{gp mark 6}{(4.050,1.310)}
\gppoint{gp mark 6}{(4.213,1.312)}
\gppoint{gp mark 6}{(4.377,1.313)}
\gppoint{gp mark 6}{(4.540,1.314)}
\gppoint{gp mark 6}{(4.704,1.316)}
\gppoint{gp mark 6}{(4.867,1.317)}
\gppoint{gp mark 6}{(5.031,1.319)}
\gppoint{gp mark 6}{(5.194,1.320)}
\gppoint{gp mark 6}{(5.358,1.322)}
\gppoint{gp mark 6}{(5.521,1.324)}
\gppoint{gp mark 6}{(5.685,1.326)}
\gppoint{gp mark 6}{(5.881,1.328)}
\gppoint{gp mark 6}{(6.045,1.329)}
\gppoint{gp mark 6}{(6.208,1.331)}
\gppoint{gp mark 6}{(6.372,1.331)}
\gppoint{gp mark 6}{(6.535,1.333)}
\gppoint{gp mark 6}{(6.699,1.335)}
\gppoint{gp mark 6}{(6.862,1.337)}
\gppoint{gp mark 6}{(7.026,1.338)}
\gppoint{gp mark 6}{(7.190,1.341)}
\gppoint{gp mark 6}{(7.353,1.343)}
\gppoint{gp mark 6}{(7.517,1.346)}
\gppoint{gp mark 6}{(7.680,1.349)}
\gppoint{gp mark 6}{(7.844,1.353)}
\gppoint{gp mark 6}{(8.007,1.358)}
\gppoint{gp mark 6}{(8.171,1.367)}
\gppoint{gp mark 6}{(8.334,1.381)}
\gppoint{gp mark 6}{(8.498,1.404)}
\gppoint{gp mark 6}{(8.661,1.443)}
\gppoint{gp mark 6}{(8.825,1.544)}
\gppoint{gp mark 6}{(8.988,1.936)}
\gppoint{gp mark 6}{(9.152,2.748)}
\gppoint{gp mark 6}{(9.315,3.619)}
\gppoint{gp mark 6}{(9.479,4.547)}
\gppoint{gp mark 6}{(9.642,5.340)}
\gppoint{gp mark 6}{(9.806,6.215)}
\gppoint{gp mark 6}{(9.970,7.125)}
\gppoint{gp mark 6}{(10.133,8.071)}
\gppoint{gp mark 6}{(5.203,7.799)}
\gpcolor{color=gp lt color border}
\node[gp node right] at (4.561,7.337) {$f(x)$};
\gpcolor{rgb color={0.000,0.000,0.000}}
\gpsetdashtype{gp dt 3}
\draw[gp path] (4.745,7.337)--(5.661,7.337);
\draw[gp path] (8.833,0.985)--(8.856,1.109)--(8.931,1.519)--(9.006,1.929)--(9.082,2.339)%
  --(9.157,2.749)--(9.232,3.159)--(9.307,3.569)--(9.383,3.979)--(9.458,4.389)--(9.533,4.799)%
  --(9.609,5.209)--(9.684,5.619)--(9.759,6.029)--(9.835,6.438)--(9.910,6.848)--(9.985,7.258)%
  --(10.061,7.668)--(10.136,8.078)--(10.202,8.441);
\gpcolor{color=gp lt color border}
\gpsetdashtype{gp dt solid}
\draw[gp path] (2.905,8.441)--(2.905,0.985)--(10.362,0.985)--(10.362,8.441)--cycle;
%% coordinates of the plot area
\gpdefrectangularnode{gp plot 1}{\pgfpoint{2.905cm}{0.985cm}}{\pgfpoint{10.362cm}{8.441cm}}
\end{tikzpicture}
%% gnuplot variables

    \caption{La-Gd合金の磁化率の逆数の温度依存性}
    \label{fig:cons_LaGd_T-chi_fit}
  \end{center}
\end{figure}