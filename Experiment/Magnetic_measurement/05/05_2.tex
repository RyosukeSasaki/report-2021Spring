\subsection{Fe-Si結晶の結晶磁気異方性定数$K_1$, $K_2$の決定}
図\ref{fig:Fe-Si_torque}のデータに対して離散フーリエ変換を施した後,
最小二乗法を用いて(\ref{equ:K_theory})の定数$K_1$, $K_2$を決定する.
まず(\ref{equ:K_theory})を展開すると
\begin{align}
  \label{equ:cons_expand_L}
  L(\theta)&=-(\frac{K_1}{4}+\frac{K_2}{64})\sin2(\theta-\theta_0)-(\frac{3K_1}{8}+\frac{K_2}{16})\sin4(\theta-\theta_0)+\frac{3K_2}{64}\sin6(\theta-\theta_0)\nonumber\\
  &=A_1\sin2(\theta-\theta_0)+A_2\sin4(\theta-\theta_0)+A_3\sin6(\theta-\theta_0)\nonumber\\
  &=a_1\sin2\theta+b_1\cos2\theta+a_2\sin4\theta+b_2\cos4\theta+a_3\sin6\theta+b_3\cos6\theta
\end{align}
となる. $a_i$, $b_i$は離散フーリエ変換のフーリエ係数であり以下で求まる.
\begin{align}
  a_i&=\frac{1}{60}\sum^{120}_j=L(\theta_j)\sin(k_i\theta_j)\\
  b_i&=\frac{1}{60}\sum^{120}_j=L(\theta_j)\cos(k_i\theta_j)\\
  k_1&=2,\ k_2=4,\ k_3=6
\end{align}
さらに$A_i$, $\theta_0$は
\begin{align}
  A_i=\pm\sqrt{a_i^2+b_i^2},\ \theta_{0,i}=\frac{1}{k_i}\tan^{-1}\left(\frac{b_i}{a_i}\right)+\frac{\pi}{k_i}n_i
\end{align}
である.ここまでを測定値から計算すると
\begin{align}
  A_1&=\pm9.14\times10^3,\ \theta_{0,1}=0.705+\frac{\pi}{2}n_1\\
  A_2&=\pm1.23\times10^4,\ \theta_{0,2}=-0.0934+\frac{\pi}{4}n_2\\
  A_3&=\pm5.65\times10^2,\ \theta_{0,3}=0.194+\frac{\pi}{6}n_3
\end{align}
さらに(\ref{equ:cons_expand_L})から
\begin{align}
  \begin{cases}
    -64A_1=16K_1+K_2\\
    -16A_2=6K_1+K_2\\
    \cfrac{64}{3}A_3=K_2
  \end{cases}
\end{align}
であり$y=K_1x+K_2$でフィットできるので,最小二乗法
\begin{align}
  \label{equ:saisyojijoho}
  a&=\frac{n\sum_k x_ky_k-\sum_k x_k\sum_k y_k}{n\sum_k x_k^2-\left(\sum_k x_k\right)^2}\\
  b&=\frac{\sum_k x_k^2\sum y_k-\sum_k x_ky_k\sum_k x_k}{n\sum_k x_k^2-\left(\sum_k x_k\right)^2}
\end{align}
により$K_1$, $K_2$を求める.実際に計算すると
\begin{align}
  K_1=\pm3.61\times10^4,\ K_2=\pm4.15\times10^2\ (復号同順)
\end{align}
となる.図示すると図\ref{fig:cons_FeSi_fit}のようになる.
\begin{figure}[hptb]
  \begin{center}
    \begin{tikzpicture}[gnuplot]
%% generated with GNUPLOT 5.2p8 (Lua 5.3; terminal rev. Nov 2018, script rev. 108)
%% 2021年05月02日 15時35分08秒
\path (0.000,0.000) rectangle (8.000,8.000);
\gpcolor{color=gp lt color border}
\gpsetlinetype{gp lt border}
\gpsetdashtype{gp dt solid}
\gpsetlinewidth{1.00}
\draw[gp path] (1.504,1.791)--(1.684,1.791);
\draw[gp path] (7.447,1.791)--(7.267,1.791);
\node[gp node right] at (1.320,1.791) {$-6.0$};
\draw[gp path] (1.504,2.640)--(1.684,2.640);
\draw[gp path] (7.447,2.640)--(7.267,2.640);
\node[gp node right] at (1.320,2.640) {$-4.0$};
\draw[gp path] (1.504,3.489)--(1.684,3.489);
\draw[gp path] (7.447,3.489)--(7.267,3.489);
\node[gp node right] at (1.320,3.489) {$-2.0$};
\draw[gp path] (1.504,4.338)--(1.684,4.338);
\draw[gp path] (7.447,4.338)--(7.267,4.338);
\node[gp node right] at (1.320,4.338) {$0.0$};
\draw[gp path] (1.504,5.187)--(1.684,5.187);
\draw[gp path] (7.447,5.187)--(7.267,5.187);
\node[gp node right] at (1.320,5.187) {$2.0$};
\draw[gp path] (1.504,6.036)--(1.684,6.036);
\draw[gp path] (7.447,6.036)--(7.267,6.036);
\node[gp node right] at (1.320,6.036) {$4.0$};
\draw[gp path] (1.504,6.885)--(1.684,6.885);
\draw[gp path] (7.447,6.885)--(7.267,6.885);
\node[gp node right] at (1.320,6.885) {$6.0$};
\draw[gp path] (1.504,1.366)--(1.504,1.546);
\draw[gp path] (1.504,7.310)--(1.504,7.130);
\node[gp node center] at (1.504,1.058) {$0$};
\draw[gp path] (2.203,1.366)--(2.203,1.546);
\draw[gp path] (2.203,7.310)--(2.203,7.130);
\node[gp node center] at (2.203,1.058) {$2$};
\draw[gp path] (2.902,1.366)--(2.902,1.546);
\draw[gp path] (2.902,7.310)--(2.902,7.130);
\node[gp node center] at (2.902,1.058) {$4$};
\draw[gp path] (3.602,1.366)--(3.602,1.546);
\draw[gp path] (3.602,7.310)--(3.602,7.130);
\node[gp node center] at (3.602,1.058) {$6$};
\draw[gp path] (4.301,1.366)--(4.301,1.546);
\draw[gp path] (4.301,7.310)--(4.301,7.130);
\node[gp node center] at (4.301,1.058) {$8$};
\draw[gp path] (5.000,1.366)--(5.000,1.546);
\draw[gp path] (5.000,7.310)--(5.000,7.130);
\node[gp node center] at (5.000,1.058) {$10$};
\draw[gp path] (5.699,1.366)--(5.699,1.546);
\draw[gp path] (5.699,7.310)--(5.699,7.130);
\node[gp node center] at (5.699,1.058) {$12$};
\draw[gp path] (6.398,1.366)--(6.398,1.546);
\draw[gp path] (6.398,7.310)--(6.398,7.130);
\node[gp node center] at (6.398,1.058) {$14$};
\draw[gp path] (7.097,1.366)--(7.097,1.546);
\draw[gp path] (7.097,7.310)--(7.097,7.130);
\node[gp node center] at (7.097,1.058) {$16$};
\draw[gp path] (1.504,7.310)--(1.504,1.366)--(7.447,1.366)--(7.447,7.310)--cycle;
\node[gp node center,rotate=-270] at (0.292,4.338) {$y$ / $\times 10^5$};
\node[gp node center] at (4.475,0.596) {$x$};
\draw[gp path] (1.688,5.744)--(1.688,7.130)--(4.996,7.130)--(4.996,5.744)--cycle;
\gpcolor{rgb color={0.000,0.000,0.000}}
\gpsetpointsize{4.00}
\gppoint{gp mark 6}{(7.097,6.823)}
\gppoint{gp mark 6}{(3.602,5.173)}
\gppoint{gp mark 6}{(1.504,4.394)}
\gppoint{gp mark 6}{(7.097,1.853)}
\gppoint{gp mark 6}{(3.602,3.503)}
\gppoint{gp mark 6}{(1.504,4.282)}
\gpcolor{color=gp lt color border}
\node[gp node right] at (3.712,6.668) {$K_1$, $K_2$: 正};
\gpcolor{rgb color={0.000,0.447,0.698}}
\draw[gp path] (3.896,6.668)--(4.812,6.668);
\draw[gp path] (1.504,4.340)--(1.564,4.366)--(1.624,4.392)--(1.684,4.419)--(1.744,4.445)%
  --(1.804,4.471)--(1.864,4.498)--(1.924,4.524)--(1.984,4.550)--(2.044,4.576)--(2.104,4.603)%
  --(2.164,4.629)--(2.224,4.655)--(2.284,4.682)--(2.344,4.708)--(2.404,4.734)--(2.464,4.761)%
  --(2.525,4.787)--(2.585,4.813)--(2.645,4.839)--(2.705,4.866)--(2.765,4.892)--(2.825,4.918)%
  --(2.885,4.945)--(2.945,4.971)--(3.005,4.997)--(3.065,5.024)--(3.125,5.050)--(3.185,5.076)%
  --(3.245,5.103)--(3.305,5.129)--(3.365,5.155)--(3.425,5.181)--(3.485,5.208)--(3.545,5.234)%
  --(3.605,5.260)--(3.665,5.287)--(3.725,5.313)--(3.785,5.339)--(3.845,5.366)--(3.905,5.392)%
  --(3.965,5.418)--(4.025,5.444)--(4.085,5.471)--(4.145,5.497)--(4.205,5.523)--(4.265,5.550)%
  --(4.325,5.576)--(4.385,5.602)--(4.445,5.629)--(4.506,5.655)--(4.566,5.681)--(4.626,5.707)%
  --(4.686,5.734)--(4.746,5.760)--(4.806,5.786)--(4.866,5.813)--(4.926,5.839)--(4.986,5.865)%
  --(5.046,5.892)--(5.106,5.918)--(5.166,5.944)--(5.226,5.970)--(5.286,5.997)--(5.346,6.023)%
  --(5.406,6.049)--(5.466,6.076)--(5.526,6.102)--(5.586,6.128)--(5.646,6.155)--(5.706,6.181)%
  --(5.766,6.207)--(5.826,6.233)--(5.886,6.260)--(5.946,6.286)--(6.006,6.312)--(6.066,6.339)%
  --(6.126,6.365)--(6.186,6.391)--(6.246,6.418)--(6.306,6.444)--(6.366,6.470)--(6.426,6.496)%
  --(6.487,6.523)--(6.547,6.549)--(6.607,6.575)--(6.667,6.602)--(6.727,6.628)--(6.787,6.654)%
  --(6.847,6.681)--(6.907,6.707)--(6.967,6.733)--(7.027,6.760)--(7.087,6.786)--(7.147,6.812)%
  --(7.207,6.838)--(7.267,6.865)--(7.327,6.891)--(7.387,6.917)--(7.447,6.944);
\gpcolor{color=gp lt color border}
\node[gp node right] at (3.712,6.206) {$K_1$, $K_2$: 負};
\gpcolor{rgb color={0.898,0.118,0.063}}
\draw[gp path] (3.896,6.206)--(4.812,6.206);
\draw[gp path] (1.504,4.336)--(1.564,4.310)--(1.624,4.284)--(1.684,4.257)--(1.744,4.231)%
  --(1.804,4.205)--(1.864,4.178)--(1.924,4.152)--(1.984,4.126)--(2.044,4.100)--(2.104,4.073)%
  --(2.164,4.047)--(2.224,4.021)--(2.284,3.994)--(2.344,3.968)--(2.404,3.942)--(2.464,3.915)%
  --(2.525,3.889)--(2.585,3.863)--(2.645,3.837)--(2.705,3.810)--(2.765,3.784)--(2.825,3.758)%
  --(2.885,3.731)--(2.945,3.705)--(3.005,3.679)--(3.065,3.652)--(3.125,3.626)--(3.185,3.600)%
  --(3.245,3.573)--(3.305,3.547)--(3.365,3.521)--(3.425,3.495)--(3.485,3.468)--(3.545,3.442)%
  --(3.605,3.416)--(3.665,3.389)--(3.725,3.363)--(3.785,3.337)--(3.845,3.310)--(3.905,3.284)%
  --(3.965,3.258)--(4.025,3.232)--(4.085,3.205)--(4.145,3.179)--(4.205,3.153)--(4.265,3.126)%
  --(4.325,3.100)--(4.385,3.074)--(4.445,3.047)--(4.506,3.021)--(4.566,2.995)--(4.626,2.969)%
  --(4.686,2.942)--(4.746,2.916)--(4.806,2.890)--(4.866,2.863)--(4.926,2.837)--(4.986,2.811)%
  --(5.046,2.784)--(5.106,2.758)--(5.166,2.732)--(5.226,2.706)--(5.286,2.679)--(5.346,2.653)%
  --(5.406,2.627)--(5.466,2.600)--(5.526,2.574)--(5.586,2.548)--(5.646,2.521)--(5.706,2.495)%
  --(5.766,2.469)--(5.826,2.443)--(5.886,2.416)--(5.946,2.390)--(6.006,2.364)--(6.066,2.337)%
  --(6.126,2.311)--(6.186,2.285)--(6.246,2.258)--(6.306,2.232)--(6.366,2.206)--(6.426,2.180)%
  --(6.487,2.153)--(6.547,2.127)--(6.607,2.101)--(6.667,2.074)--(6.727,2.048)--(6.787,2.022)%
  --(6.847,1.995)--(6.907,1.969)--(6.967,1.943)--(7.027,1.916)--(7.087,1.890)--(7.147,1.864)%
  --(7.207,1.838)--(7.267,1.811)--(7.327,1.785)--(7.387,1.759)--(7.447,1.732);
\gpcolor{color=gp lt color border}
\draw[gp path] (1.504,7.310)--(1.504,1.366)--(7.447,1.366)--(7.447,7.310)--cycle;
%% coordinates of the plot area
\gpdefrectangularnode{gp plot 1}{\pgfpoint{1.504cm}{1.366cm}}{\pgfpoint{7.447cm}{7.310cm}}
\end{tikzpicture}
%% gnuplot variables

    \caption{最小二乗法によるフィット}
    \label{fig:cons_FeSi_fit}
  \end{center}
\end{figure}
ここでは$K_1$, $K_2$を共に正とする.さらに$n_1=-1$, $n_2=-1$, $n_3=-2$とすると$\theta_0\simeq-0.86$となるので
これを用いてFe-Si円盤の磁気トルク曲線に理論値を重ねると図\ref{fig:cons_FeSi_torque_fit}のようになる.
図\ref{fig:cons_FeSi_torque_fit}のように以上の方法で測定値とよく一致する理論値を得られた.
\begin{figure}[hptb]
  \begin{center}
    \begin{tikzpicture}[gnuplot]
%% generated with GNUPLOT 5.2p8 (Lua 5.3; terminal rev. Nov 2018, script rev. 108)
%% 2021年05月02日 00時14分42秒
\path (0.000,0.000) rectangle (12.500,8.750);
\gpcolor{color=gp lt color border}
\gpsetlinetype{gp lt border}
\gpsetdashtype{gp dt solid}
\gpsetlinewidth{1.00}
\draw[gp path] (2.997,0.985)--(3.087,0.985);
\draw[gp path] (10.454,0.985)--(10.364,0.985);
\draw[gp path] (2.997,1.731)--(3.177,1.731);
\draw[gp path] (10.454,1.731)--(10.274,1.731);
\node[gp node right] at (2.813,1.731) {$-2.0$};
\draw[gp path] (2.997,2.476)--(3.087,2.476);
\draw[gp path] (10.454,2.476)--(10.364,2.476);
\draw[gp path] (2.997,3.222)--(3.177,3.222);
\draw[gp path] (10.454,3.222)--(10.274,3.222);
\node[gp node right] at (2.813,3.222) {$-1.0$};
\draw[gp path] (2.997,3.967)--(3.087,3.967);
\draw[gp path] (10.454,3.967)--(10.364,3.967);
\draw[gp path] (2.997,4.713)--(3.177,4.713);
\draw[gp path] (10.454,4.713)--(10.274,4.713);
\node[gp node right] at (2.813,4.713) {$0.0$};
\draw[gp path] (2.997,5.459)--(3.087,5.459);
\draw[gp path] (10.454,5.459)--(10.364,5.459);
\draw[gp path] (2.997,6.204)--(3.177,6.204);
\draw[gp path] (10.454,6.204)--(10.274,6.204);
\node[gp node right] at (2.813,6.204) {$1.0$};
\draw[gp path] (2.997,6.950)--(3.087,6.950);
\draw[gp path] (10.454,6.950)--(10.364,6.950);
\draw[gp path] (2.997,7.695)--(3.177,7.695);
\draw[gp path] (10.454,7.695)--(10.274,7.695);
\node[gp node right] at (2.813,7.695) {$2.0$};
\draw[gp path] (2.997,8.441)--(3.087,8.441);
\draw[gp path] (10.454,8.441)--(10.364,8.441);
\draw[gp path] (3.193,0.985)--(3.193,1.165);
\draw[gp path] (3.193,8.441)--(3.193,8.261);
\node[gp node center] at (3.193,0.677) {$0$};
\draw[gp path] (3.684,0.985)--(3.684,1.075);
\draw[gp path] (3.684,8.441)--(3.684,8.351);
\draw[gp path] (4.174,0.985)--(4.174,1.165);
\draw[gp path] (4.174,8.441)--(4.174,8.261);
\node[gp node center] at (4.174,0.677) {$50$};
\draw[gp path] (4.665,0.985)--(4.665,1.075);
\draw[gp path] (4.665,8.441)--(4.665,8.351);
\draw[gp path] (5.156,0.985)--(5.156,1.165);
\draw[gp path] (5.156,8.441)--(5.156,8.261);
\node[gp node center] at (5.156,0.677) {$100$};
\draw[gp path] (5.646,0.985)--(5.646,1.075);
\draw[gp path] (5.646,8.441)--(5.646,8.351);
\draw[gp path] (6.137,0.985)--(6.137,1.165);
\draw[gp path] (6.137,8.441)--(6.137,8.261);
\node[gp node center] at (6.137,0.677) {$150$};
\draw[gp path] (6.627,0.985)--(6.627,1.075);
\draw[gp path] (6.627,8.441)--(6.627,8.351);
\draw[gp path] (7.118,0.985)--(7.118,1.165);
\draw[gp path] (7.118,8.441)--(7.118,8.261);
\node[gp node center] at (7.118,0.677) {$200$};
\draw[gp path] (7.609,0.985)--(7.609,1.075);
\draw[gp path] (7.609,8.441)--(7.609,8.351);
\draw[gp path] (8.099,0.985)--(8.099,1.165);
\draw[gp path] (8.099,8.441)--(8.099,8.261);
\node[gp node center] at (8.099,0.677) {$250$};
\draw[gp path] (8.590,0.985)--(8.590,1.075);
\draw[gp path] (8.590,8.441)--(8.590,8.351);
\draw[gp path] (9.080,0.985)--(9.080,1.165);
\draw[gp path] (9.080,8.441)--(9.080,8.261);
\node[gp node center] at (9.080,0.677) {$300$};
\draw[gp path] (9.571,0.985)--(9.571,1.075);
\draw[gp path] (9.571,8.441)--(9.571,8.351);
\draw[gp path] (10.062,0.985)--(10.062,1.165);
\draw[gp path] (10.062,8.441)--(10.062,8.261);
\node[gp node center] at (10.062,0.677) {$350$};
\gpsetlinetype{gp lt axes}
\gpsetdashtype{gp dt axes}
\gpsetlinewidth{1.50}
\draw[gp path] (2.997,4.713)--(10.454,4.713);
\gpsetlinetype{gp lt border}
\gpsetdashtype{gp dt solid}
\gpsetlinewidth{1.00}
\draw[gp path] (2.997,8.441)--(2.997,0.985)--(10.454,0.985)--(10.454,8.441)--cycle;
\node[gp node center,rotate=-270] at (1.785,4.713) {単位体積あたりのトルク$L$ / $\times 10^{4}\ \si{\newton.\meter^{-2}}$};
\node[gp node center] at (6.725,0.215) {角度$\theta$ / $\si{\degree}$};
\draw[gp path] (3.181,1.165)--(3.181,2.551)--(5.937,2.551)--(5.937,1.165)--cycle;
\node[gp node right] at (4.653,2.089) {測定値};
\gpcolor{rgb color={0.000,0.000,0.000}}
\gpsetpointsize{4.00}
\gppoint{gp mark 6}{(3.193,2.739)}
\gppoint{gp mark 6}{(3.252,3.098)}
\gppoint{gp mark 6}{(3.311,3.494)}
\gppoint{gp mark 6}{(3.370,3.898)}
\gppoint{gp mark 6}{(3.429,4.319)}
\gppoint{gp mark 6}{(3.488,4.696)}
\gppoint{gp mark 6}{(3.546,5.040)}
\gppoint{gp mark 6}{(3.605,5.318)}
\gppoint{gp mark 6}{(3.664,5.532)}
\gppoint{gp mark 6}{(3.723,5.677)}
\gppoint{gp mark 6}{(3.782,5.749)}
\gppoint{gp mark 6}{(3.841,5.744)}
\gppoint{gp mark 6}{(3.900,5.677)}
\gppoint{gp mark 6}{(3.959,5.556)}
\gppoint{gp mark 6}{(4.017,5.385)}
\gppoint{gp mark 6}{(4.076,5.185)}
\gppoint{gp mark 6}{(4.135,4.958)}
\gppoint{gp mark 6}{(4.194,4.714)}
\gppoint{gp mark 6}{(4.253,4.479)}
\gppoint{gp mark 6}{(4.312,4.262)}
\gppoint{gp mark 6}{(4.371,4.081)}
\gppoint{gp mark 6}{(4.430,3.930)}
\gppoint{gp mark 6}{(4.488,3.835)}
\gppoint{gp mark 6}{(4.547,3.799)}
\gppoint{gp mark 6}{(4.606,3.833)}
\gppoint{gp mark 6}{(4.665,3.946)}
\gppoint{gp mark 6}{(4.724,4.121)}
\gppoint{gp mark 6}{(4.783,4.376)}
\gppoint{gp mark 6}{(4.842,4.704)}
\gppoint{gp mark 6}{(4.900,5.083)}
\gppoint{gp mark 6}{(4.959,5.471)}
\gppoint{gp mark 6}{(5.018,5.915)}
\gppoint{gp mark 6}{(5.077,6.307)}
\gppoint{gp mark 6}{(5.136,6.683)}
\gppoint{gp mark 6}{(5.195,7.013)}
\gppoint{gp mark 6}{(5.254,7.269)}
\gppoint{gp mark 6}{(5.313,7.452)}
\gppoint{gp mark 6}{(5.371,7.556)}
\gppoint{gp mark 6}{(5.430,7.579)}
\gppoint{gp mark 6}{(5.489,7.507)}
\gppoint{gp mark 6}{(5.548,7.355)}
\gppoint{gp mark 6}{(5.607,7.118)}
\gppoint{gp mark 6}{(5.666,6.816)}
\gppoint{gp mark 6}{(5.725,6.443)}
\gppoint{gp mark 6}{(5.784,6.001)}
\gppoint{gp mark 6}{(5.842,5.598)}
\gppoint{gp mark 6}{(5.901,5.127)}
\gppoint{gp mark 6}{(5.960,4.630)}
\gppoint{gp mark 6}{(6.019,4.141)}
\gppoint{gp mark 6}{(6.078,3.703)}
\gppoint{gp mark 6}{(6.137,3.274)}
\gppoint{gp mark 6}{(6.196,2.870)}
\gppoint{gp mark 6}{(6.255,2.545)}
\gppoint{gp mark 6}{(6.313,2.270)}
\gppoint{gp mark 6}{(6.372,2.095)}
\gppoint{gp mark 6}{(6.431,2.001)}
\gppoint{gp mark 6}{(6.490,1.979)}
\gppoint{gp mark 6}{(6.549,2.056)}
\gppoint{gp mark 6}{(6.608,2.227)}
\gppoint{gp mark 6}{(6.667,2.456)}
\gppoint{gp mark 6}{(6.726,2.755)}
\gppoint{gp mark 6}{(6.784,3.134)}
\gppoint{gp mark 6}{(6.843,3.539)}
\gppoint{gp mark 6}{(6.902,3.935)}
\gppoint{gp mark 6}{(6.961,4.368)}
\gppoint{gp mark 6}{(7.020,4.673)}
\gppoint{gp mark 6}{(7.079,5.090)}
\gppoint{gp mark 6}{(7.138,5.362)}
\gppoint{gp mark 6}{(7.196,5.576)}
\gppoint{gp mark 6}{(7.255,5.715)}
\gppoint{gp mark 6}{(7.314,5.776)}
\gppoint{gp mark 6}{(7.373,5.788)}
\gppoint{gp mark 6}{(7.432,5.724)}
\gppoint{gp mark 6}{(7.491,5.595)}
\gppoint{gp mark 6}{(7.550,5.427)}
\gppoint{gp mark 6}{(7.609,5.224)}
\gppoint{gp mark 6}{(7.667,4.998)}
\gppoint{gp mark 6}{(7.726,4.764)}
\gppoint{gp mark 6}{(7.785,4.535)}
\gppoint{gp mark 6}{(7.844,4.317)}
\gppoint{gp mark 6}{(7.903,4.123)}
\gppoint{gp mark 6}{(7.962,3.972)}
\gppoint{gp mark 6}{(8.021,3.869)}
\gppoint{gp mark 6}{(8.080,3.838)}
\gppoint{gp mark 6}{(8.138,3.863)}
\gppoint{gp mark 6}{(8.197,3.960)}
\gppoint{gp mark 6}{(8.256,4.150)}
\gppoint{gp mark 6}{(8.315,4.398)}
\gppoint{gp mark 6}{(8.374,4.714)}
\gppoint{gp mark 6}{(8.433,5.083)}
\gppoint{gp mark 6}{(8.492,5.499)}
\gppoint{gp mark 6}{(8.551,5.906)}
\gppoint{gp mark 6}{(8.609,6.317)}
\gppoint{gp mark 6}{(8.668,6.700)}
\gppoint{gp mark 6}{(8.727,7.016)}
\gppoint{gp mark 6}{(8.786,7.275)}
\gppoint{gp mark 6}{(8.845,7.455)}
\gppoint{gp mark 6}{(8.904,7.563)}
\gppoint{gp mark 6}{(8.963,7.583)}
\gppoint{gp mark 6}{(9.021,7.508)}
\gppoint{gp mark 6}{(9.080,7.351)}
\gppoint{gp mark 6}{(9.139,7.108)}
\gppoint{gp mark 6}{(9.198,6.809)}
\gppoint{gp mark 6}{(9.257,6.436)}
\gppoint{gp mark 6}{(9.316,6.021)}
\gppoint{gp mark 6}{(9.375,5.568)}
\gppoint{gp mark 6}{(9.434,5.080)}
\gppoint{gp mark 6}{(9.492,4.625)}
\gppoint{gp mark 6}{(9.551,4.135)}
\gppoint{gp mark 6}{(9.610,3.660)}
\gppoint{gp mark 6}{(9.669,3.248)}
\gppoint{gp mark 6}{(9.728,2.862)}
\gppoint{gp mark 6}{(9.787,2.524)}
\gppoint{gp mark 6}{(9.846,2.268)}
\gppoint{gp mark 6}{(9.905,2.073)}
\gppoint{gp mark 6}{(9.963,1.966)}
\gppoint{gp mark 6}{(10.022,1.948)}
\gppoint{gp mark 6}{(10.081,2.024)}
\gppoint{gp mark 6}{(10.140,2.173)}
\gppoint{gp mark 6}{(10.199,2.428)}
\gppoint{gp mark 6}{(10.258,2.731)}
\gppoint{gp mark 6}{(5.295,2.089)}
\gpcolor{color=gp lt color border}
\node[gp node right] at (4.653,1.627) {理論値};
\gpcolor{rgb color={0.000,0.000,0.000}}
\gpsetdashtype{gp dt 3}
\draw[gp path] (4.837,1.627)--(5.753,1.627);
\draw[gp path] (2.997,1.797)--(3.072,2.046)--(3.148,2.420)--(3.223,2.887)--(3.298,3.414)%
  --(3.374,3.962)--(3.449,4.492)--(3.524,4.970)--(3.600,5.363)--(3.675,5.648)--(3.750,5.811)%
  --(3.826,5.846)--(3.901,5.757)--(3.976,5.561)--(4.052,5.279)--(4.127,4.940)--(4.202,4.580)%
  --(4.277,4.232)--(4.353,3.932)--(4.428,3.711)--(4.503,3.592)--(4.579,3.594)--(4.654,3.723)%
  --(4.729,3.977)--(4.805,4.344)--(4.880,4.801)--(4.955,5.321)--(5.031,5.868)--(5.106,6.404)%
  --(5.181,6.891)--(5.257,7.292)--(5.332,7.577)--(5.407,7.720)--(5.483,7.708)--(5.558,7.534)%
  --(5.633,7.204)--(5.709,6.735)--(5.784,6.150)--(5.859,5.482)--(5.935,4.770)--(6.010,4.055)%
  --(6.085,3.377)--(6.161,2.777)--(6.236,2.287)--(6.311,1.934)--(6.387,1.735)--(6.462,1.697)%
  --(6.537,1.817)--(6.613,2.080)--(6.688,2.465)--(6.763,2.941)--(6.838,3.472)--(6.914,4.019)%
  --(6.989,4.546)--(7.064,5.015)--(7.140,5.398)--(7.215,5.671)--(7.290,5.820)--(7.366,5.842)%
  --(7.441,5.742)--(7.516,5.535)--(7.592,5.245)--(7.667,4.903)--(7.742,4.542)--(7.818,4.198)%
  --(7.893,3.905)--(7.968,3.693)--(8.044,3.586)--(8.119,3.601)--(8.194,3.744)--(8.270,4.011)%
  --(8.345,4.388)--(8.420,4.854)--(8.496,5.378)--(8.571,5.926)--(8.646,6.458)--(8.722,6.938)%
  --(8.797,7.328)--(8.872,7.599)--(8.948,7.727)--(9.023,7.697)--(9.098,7.507)--(9.174,7.161)%
  --(9.249,6.678)--(9.324,6.083)--(9.399,5.408)--(9.475,4.694)--(9.550,3.981)--(9.625,3.310)%
  --(9.701,2.720)--(9.776,2.243)--(9.851,1.905)--(9.927,1.723)--(10.002,1.702)--(10.077,1.838)%
  --(10.153,2.116)--(10.228,2.512)--(10.303,2.995)--(10.379,3.529)--(10.454,4.076);
\gpcolor{color=gp lt color border}
\gpsetdashtype{gp dt solid}
\draw[gp path] (2.997,8.441)--(2.997,0.985)--(10.454,0.985)--(10.454,8.441)--cycle;
%% coordinates of the plot area
\gpdefrectangularnode{gp plot 1}{\pgfpoint{2.997cm}{0.985cm}}{\pgfpoint{10.454cm}{8.441cm}}
\end{tikzpicture}
%% gnuplot variables

    \caption{Fe-Si円盤の磁気トルク曲線の測定値と理論値}
    \label{fig:cons_FeSi_torque_fit}
  \end{center}
\end{figure}