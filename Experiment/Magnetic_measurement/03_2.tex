\subsection{10wt\%La-Gd合金の磁化測定}
\subsubsection{準備}
試料No.4のLa-Gdを試料棒にテフロンテープで貼り付け,液体窒素導入用のデュワー瓶をVSM装置に設置した.
試料棒をVSM装置に試料棒を取り付け,試料棒上部の温度センサー端子にケーブルを接続した.
その後電磁石に冷却水を循環させ, VSM装置,デジタルマルチメータ,電磁石用電源装置,温度計の電源を入れた.
\subsubsection{測定1: 室温での磁化測定}
室温を測定した後\ref{subsubsec:Ni_jika_Mennai}節と同様の手順で磁化を測定した.
\subsubsection{測定2: 77Kでの磁化測定}
次に漏斗を用いてデュワー瓶に液体窒素を入れた.その際液面が試料より$1〜2\ \si{\centi\metre}$上になるようにした.
温度計の温度が$77\si{\kelvin}$になったことを確認した後\ref{subsubsec:Ni_jika_Mennai}節と同様の手順を磁化を測定した.
\subsubsection{測定3: 磁化の温度依存性の測定}
次に電磁石用の電源装置を$0.5\si{\ampere}$に,ヒーター用電源装置を$1.5\si{\ampere}$に設定し,
液体窒素の液面がデュワー瓶の底から$5〜10\ \si{\milli\metre}$になるまで蒸発させた.
その後ヒーター電流を$0.8\si{\ampere}$に設定し, $290\si{\kelvin}$まで$1\si{\kelvin}$ごとに磁気モーメントを測定した.