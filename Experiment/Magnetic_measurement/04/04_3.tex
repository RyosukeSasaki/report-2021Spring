\subsection{Ni円盤の磁気トルク測定}
\subsubsection{磁気トルク計の校正}
この実験では\ref{subsec:res_Fe-Si}とは独立に校正を行っている.
図\ref{fig:Ni_adjust}に校正時に得た磁気トルク相殺電流の角度依存性曲線を示す.
電流の最大値は$20.3\si{\milli\ampere}$なので(\ref{equ:calc_alpha})から
\begin{align}
  \alpha=5.99\times10^{-5} \si{\newton.\metre.(\milli\ampere)^{-1}}
\end{align}
である.したがって\ref{subsubsec:res_Fe-Si_adjust}節と同様の手順により図\ref{fig:Ni_adjust_torque}の磁気トルク曲線を得る.
\begin{figure}[hptb]
  \begin{center}
    \begin{tikzpicture}[gnuplot]
%% generated with GNUPLOT 5.2p8 (Lua 5.3; terminal rev. Nov 2018, script rev. 108)
%% 2021年04月27日 14時39分14秒
\path (0.000,0.000) rectangle (12.500,8.750);
\gpcolor{color=gp lt color border}
\gpsetlinetype{gp lt border}
\gpsetdashtype{gp dt solid}
\gpsetlinewidth{1.00}
\draw[gp path] (2.905,1.163)--(3.085,1.163);
\draw[gp path] (10.362,1.163)--(10.182,1.163);
\node[gp node right] at (2.721,1.163) {$-20$};
\draw[gp path] (2.905,2.050)--(3.085,2.050);
\draw[gp path] (10.362,2.050)--(10.182,2.050);
\node[gp node right] at (2.721,2.050) {$-15$};
\draw[gp path] (2.905,2.938)--(3.085,2.938);
\draw[gp path] (10.362,2.938)--(10.182,2.938);
\node[gp node right] at (2.721,2.938) {$-10$};
\draw[gp path] (2.905,3.825)--(3.085,3.825);
\draw[gp path] (10.362,3.825)--(10.182,3.825);
\node[gp node right] at (2.721,3.825) {$-5$};
\draw[gp path] (2.905,4.713)--(3.085,4.713);
\draw[gp path] (10.362,4.713)--(10.182,4.713);
\node[gp node right] at (2.721,4.713) {$0$};
\draw[gp path] (2.905,5.601)--(3.085,5.601);
\draw[gp path] (10.362,5.601)--(10.182,5.601);
\node[gp node right] at (2.721,5.601) {$5$};
\draw[gp path] (2.905,6.488)--(3.085,6.488);
\draw[gp path] (10.362,6.488)--(10.182,6.488);
\node[gp node right] at (2.721,6.488) {$10$};
\draw[gp path] (2.905,7.376)--(3.085,7.376);
\draw[gp path] (10.362,7.376)--(10.182,7.376);
\node[gp node right] at (2.721,7.376) {$15$};
\draw[gp path] (2.905,8.263)--(3.085,8.263);
\draw[gp path] (10.362,8.263)--(10.182,8.263);
\node[gp node right] at (2.721,8.263) {$20$};
\draw[gp path] (2.905,0.985)--(2.905,1.165);
\draw[gp path] (2.905,8.441)--(2.905,8.261);
\node[gp node center] at (2.905,0.677) {$0$};
\draw[gp path] (3.734,0.985)--(3.734,1.165);
\draw[gp path] (3.734,8.441)--(3.734,8.261);
\node[gp node center] at (3.734,0.677) {$20$};
\draw[gp path] (4.562,0.985)--(4.562,1.165);
\draw[gp path] (4.562,8.441)--(4.562,8.261);
\node[gp node center] at (4.562,0.677) {$40$};
\draw[gp path] (5.391,0.985)--(5.391,1.165);
\draw[gp path] (5.391,8.441)--(5.391,8.261);
\node[gp node center] at (5.391,0.677) {$60$};
\draw[gp path] (6.219,0.985)--(6.219,1.165);
\draw[gp path] (6.219,8.441)--(6.219,8.261);
\node[gp node center] at (6.219,0.677) {$80$};
\draw[gp path] (7.048,0.985)--(7.048,1.165);
\draw[gp path] (7.048,8.441)--(7.048,8.261);
\node[gp node center] at (7.048,0.677) {$100$};
\draw[gp path] (7.876,0.985)--(7.876,1.165);
\draw[gp path] (7.876,8.441)--(7.876,8.261);
\node[gp node center] at (7.876,0.677) {$120$};
\draw[gp path] (8.705,0.985)--(8.705,1.165);
\draw[gp path] (8.705,8.441)--(8.705,8.261);
\node[gp node center] at (8.705,0.677) {$140$};
\draw[gp path] (9.533,0.985)--(9.533,1.165);
\draw[gp path] (9.533,8.441)--(9.533,8.261);
\node[gp node center] at (9.533,0.677) {$160$};
\draw[gp path] (10.362,0.985)--(10.362,1.165);
\draw[gp path] (10.362,8.441)--(10.362,8.261);
\node[gp node center] at (10.362,0.677) {$180$};
\gpsetlinetype{gp lt axes}
\gpsetdashtype{gp dt axes}
\gpsetlinewidth{1.50}
\draw[gp path] (2.905,4.713)--(10.362,4.713);
\gpsetlinetype{gp lt border}
\gpsetdashtype{gp dt solid}
\gpsetlinewidth{1.00}
\draw[gp path] (2.905,8.441)--(2.905,0.985)--(10.362,0.985)--(10.362,8.441)--cycle;
\node[gp node center,rotate=-270] at (1.877,4.713) {相殺電流$i$ / $\si{\milli\ampere}$};
\node[gp node center] at (6.633,0.215) {角度$\theta$ / $\si{\degree}$};
\gpcolor{rgb color={0.000,0.000,0.000}}
\gpsetpointsize{4.00}
\gppoint{gp mark 6}{(2.905,4.846)}
\gppoint{gp mark 6}{(3.029,4.580)}
\gppoint{gp mark 6}{(3.154,4.314)}
\gppoint{gp mark 6}{(3.278,4.047)}
\gppoint{gp mark 6}{(3.402,3.781)}
\gppoint{gp mark 6}{(3.526,3.541)}
\gppoint{gp mark 6}{(3.651,3.275)}
\gppoint{gp mark 6}{(3.775,3.035)}
\gppoint{gp mark 6}{(3.899,2.796)}
\gppoint{gp mark 6}{(4.024,2.556)}
\gppoint{gp mark 6}{(4.148,2.343)}
\gppoint{gp mark 6}{(4.272,2.130)}
\gppoint{gp mark 6}{(4.396,1.944)}
\gppoint{gp mark 6}{(4.521,1.784)}
\gppoint{gp mark 6}{(4.645,1.624)}
\gppoint{gp mark 6}{(4.769,1.491)}
\gppoint{gp mark 6}{(4.894,1.384)}
\gppoint{gp mark 6}{(5.018,1.305)}
\gppoint{gp mark 6}{(5.142,1.251)}
\gppoint{gp mark 6}{(5.266,1.225)}
\gppoint{gp mark 6}{(5.391,1.225)}
\gppoint{gp mark 6}{(5.515,1.305)}
\gppoint{gp mark 6}{(5.639,1.384)}
\gppoint{gp mark 6}{(5.764,1.544)}
\gppoint{gp mark 6}{(5.888,1.784)}
\gppoint{gp mark 6}{(6.012,2.077)}
\gppoint{gp mark 6}{(6.136,2.450)}
\gppoint{gp mark 6}{(6.261,2.849)}
\gppoint{gp mark 6}{(6.385,3.328)}
\gppoint{gp mark 6}{(6.509,3.888)}
\gppoint{gp mark 6}{(6.634,4.473)}
\gppoint{gp mark 6}{(6.758,5.112)}
\gppoint{gp mark 6}{(6.882,5.698)}
\gppoint{gp mark 6}{(7.006,6.257)}
\gppoint{gp mark 6}{(7.131,6.737)}
\gppoint{gp mark 6}{(7.255,7.163)}
\gppoint{gp mark 6}{(7.379,7.509)}
\gppoint{gp mark 6}{(7.503,7.775)}
\gppoint{gp mark 6}{(7.628,8.015)}
\gppoint{gp mark 6}{(7.752,8.148)}
\gppoint{gp mark 6}{(7.876,8.228)}
\gppoint{gp mark 6}{(8.001,8.308)}
\gppoint{gp mark 6}{(8.125,8.308)}
\gppoint{gp mark 6}{(8.249,8.281)}
\gppoint{gp mark 6}{(8.373,8.201)}
\gppoint{gp mark 6}{(8.498,8.121)}
\gppoint{gp mark 6}{(8.622,7.988)}
\gppoint{gp mark 6}{(8.746,7.855)}
\gppoint{gp mark 6}{(8.871,7.695)}
\gppoint{gp mark 6}{(8.995,7.509)}
\gppoint{gp mark 6}{(9.119,7.323)}
\gppoint{gp mark 6}{(9.243,7.110)}
\gppoint{gp mark 6}{(9.368,6.897)}
\gppoint{gp mark 6}{(9.492,6.657)}
\gppoint{gp mark 6}{(9.616,6.417)}
\gppoint{gp mark 6}{(9.741,6.151)}
\gppoint{gp mark 6}{(9.865,5.911)}
\gppoint{gp mark 6}{(9.989,5.672)}
\gppoint{gp mark 6}{(10.113,5.379)}
\gppoint{gp mark 6}{(10.238,5.112)}
\gppoint{gp mark 6}{(10.362,4.873)}
\gpcolor{color=gp lt color border}
\draw[gp path] (2.905,8.441)--(2.905,0.985)--(10.362,0.985)--(10.362,8.441)--cycle;
%% coordinates of the plot area
\gpdefrectangularnode{gp plot 1}{\pgfpoint{2.905cm}{0.985cm}}{\pgfpoint{10.362cm}{8.441cm}}
\end{tikzpicture}
%% gnuplot variables

    \caption{相殺電流の角度依存性}
    \label{fig:Ni_adjust}
  \end{center}
\end{figure}
\begin{figure}[hptb]
  \begin{center}
    \begin{tikzpicture}[gnuplot]
%% generated with GNUPLOT 5.2p8 (Lua 5.3; terminal rev. Nov 2018, script rev. 108)
%% 2021年05月02日 00時59分41秒
\path (0.000,0.000) rectangle (12.500,8.750);
\gpcolor{color=gp lt color border}
\gpsetlinetype{gp lt border}
\gpsetdashtype{gp dt solid}
\gpsetlinewidth{1.00}
\draw[gp path] (2.997,0.985)--(3.177,0.985);
\draw[gp path] (10.454,0.985)--(10.274,0.985);
\node[gp node right] at (2.813,0.985) {$-6.0$};
\draw[gp path] (2.997,2.228)--(3.177,2.228);
\draw[gp path] (10.454,2.228)--(10.274,2.228);
\node[gp node right] at (2.813,2.228) {$-4.0$};
\draw[gp path] (2.997,3.470)--(3.177,3.470);
\draw[gp path] (10.454,3.470)--(10.274,3.470);
\node[gp node right] at (2.813,3.470) {$-2.0$};
\draw[gp path] (2.997,4.713)--(3.177,4.713);
\draw[gp path] (10.454,4.713)--(10.274,4.713);
\node[gp node right] at (2.813,4.713) {$0.0$};
\draw[gp path] (2.997,5.956)--(3.177,5.956);
\draw[gp path] (10.454,5.956)--(10.274,5.956);
\node[gp node right] at (2.813,5.956) {$2.0$};
\draw[gp path] (2.997,7.198)--(3.177,7.198);
\draw[gp path] (10.454,7.198)--(10.274,7.198);
\node[gp node right] at (2.813,7.198) {$4.0$};
\draw[gp path] (2.997,8.441)--(3.177,8.441);
\draw[gp path] (10.454,8.441)--(10.274,8.441);
\node[gp node right] at (2.813,8.441) {$6.0$};
\draw[gp path] (2.997,0.985)--(2.997,1.165);
\draw[gp path] (2.997,8.441)--(2.997,8.261);
\node[gp node center] at (2.997,0.677) {$0$};
\draw[gp path] (3.826,0.985)--(3.826,1.165);
\draw[gp path] (3.826,8.441)--(3.826,8.261);
\node[gp node center] at (3.826,0.677) {$20$};
\draw[gp path] (4.654,0.985)--(4.654,1.165);
\draw[gp path] (4.654,8.441)--(4.654,8.261);
\node[gp node center] at (4.654,0.677) {$40$};
\draw[gp path] (5.483,0.985)--(5.483,1.165);
\draw[gp path] (5.483,8.441)--(5.483,8.261);
\node[gp node center] at (5.483,0.677) {$60$};
\draw[gp path] (6.311,0.985)--(6.311,1.165);
\draw[gp path] (6.311,8.441)--(6.311,8.261);
\node[gp node center] at (6.311,0.677) {$80$};
\draw[gp path] (7.140,0.985)--(7.140,1.165);
\draw[gp path] (7.140,8.441)--(7.140,8.261);
\node[gp node center] at (7.140,0.677) {$100$};
\draw[gp path] (7.968,0.985)--(7.968,1.165);
\draw[gp path] (7.968,8.441)--(7.968,8.261);
\node[gp node center] at (7.968,0.677) {$120$};
\draw[gp path] (8.797,0.985)--(8.797,1.165);
\draw[gp path] (8.797,8.441)--(8.797,8.261);
\node[gp node center] at (8.797,0.677) {$140$};
\draw[gp path] (9.625,0.985)--(9.625,1.165);
\draw[gp path] (9.625,8.441)--(9.625,8.261);
\node[gp node center] at (9.625,0.677) {$160$};
\draw[gp path] (10.454,0.985)--(10.454,1.165);
\draw[gp path] (10.454,8.441)--(10.454,8.261);
\node[gp node center] at (10.454,0.677) {$180$};
\gpsetlinetype{gp lt axes}
\gpsetdashtype{gp dt axes}
\gpsetlinewidth{1.50}
\draw[gp path] (2.997,4.713)--(10.454,4.713);
\gpsetlinetype{gp lt border}
\gpsetdashtype{gp dt solid}
\gpsetlinewidth{1.00}
\draw[gp path] (2.997,8.441)--(2.997,0.985)--(10.454,0.985)--(10.454,8.441)--cycle;
\node[gp node center,rotate=-270] at (1.785,4.713) {単位体積あたりのトルク$L$ / $\times 10^{4}\ \si{\newton.\meter^{-2}}$};
\node[gp node center] at (6.725,0.215) {角度$\theta$ / $\si{\degree}$};
\gpcolor{rgb color={0.000,0.000,0.000}}
\gpsetpointsize{4.00}
\gppoint{gp mark 6}{(2.997,4.840)}
\gppoint{gp mark 6}{(3.121,4.586)}
\gppoint{gp mark 6}{(3.246,4.332)}
\gppoint{gp mark 6}{(3.370,4.078)}
\gppoint{gp mark 6}{(3.494,3.824)}
\gppoint{gp mark 6}{(3.618,3.595)}
\gppoint{gp mark 6}{(3.743,3.341)}
\gppoint{gp mark 6}{(3.867,3.113)}
\gppoint{gp mark 6}{(3.991,2.884)}
\gppoint{gp mark 6}{(4.116,2.656)}
\gppoint{gp mark 6}{(4.240,2.453)}
\gppoint{gp mark 6}{(4.364,2.249)}
\gppoint{gp mark 6}{(4.488,2.072)}
\gppoint{gp mark 6}{(4.613,1.919)}
\gppoint{gp mark 6}{(4.737,1.767)}
\gppoint{gp mark 6}{(4.861,1.640)}
\gppoint{gp mark 6}{(4.986,1.538)}
\gppoint{gp mark 6}{(5.110,1.462)}
\gppoint{gp mark 6}{(5.234,1.411)}
\gppoint{gp mark 6}{(5.358,1.386)}
\gppoint{gp mark 6}{(5.483,1.386)}
\gppoint{gp mark 6}{(5.607,1.462)}
\gppoint{gp mark 6}{(5.731,1.538)}
\gppoint{gp mark 6}{(5.856,1.691)}
\gppoint{gp mark 6}{(5.980,1.919)}
\gppoint{gp mark 6}{(6.104,2.199)}
\gppoint{gp mark 6}{(6.228,2.554)}
\gppoint{gp mark 6}{(6.353,2.935)}
\gppoint{gp mark 6}{(6.477,3.392)}
\gppoint{gp mark 6}{(6.601,3.926)}
\gppoint{gp mark 6}{(6.726,4.484)}
\gppoint{gp mark 6}{(6.850,5.094)}
\gppoint{gp mark 6}{(6.974,5.653)}
\gppoint{gp mark 6}{(7.098,6.186)}
\gppoint{gp mark 6}{(7.223,6.643)}
\gppoint{gp mark 6}{(7.347,7.050)}
\gppoint{gp mark 6}{(7.471,7.380)}
\gppoint{gp mark 6}{(7.595,7.634)}
\gppoint{gp mark 6}{(7.720,7.862)}
\gppoint{gp mark 6}{(7.844,7.989)}
\gppoint{gp mark 6}{(7.968,8.066)}
\gppoint{gp mark 6}{(8.093,8.142)}
\gppoint{gp mark 6}{(8.217,8.142)}
\gppoint{gp mark 6}{(8.341,8.116)}
\gppoint{gp mark 6}{(8.465,8.040)}
\gppoint{gp mark 6}{(8.590,7.964)}
\gppoint{gp mark 6}{(8.714,7.837)}
\gppoint{gp mark 6}{(8.838,7.710)}
\gppoint{gp mark 6}{(8.963,7.558)}
\gppoint{gp mark 6}{(9.087,7.380)}
\gppoint{gp mark 6}{(9.211,7.202)}
\gppoint{gp mark 6}{(9.335,6.999)}
\gppoint{gp mark 6}{(9.460,6.796)}
\gppoint{gp mark 6}{(9.584,6.567)}
\gppoint{gp mark 6}{(9.708,6.339)}
\gppoint{gp mark 6}{(9.833,6.085)}
\gppoint{gp mark 6}{(9.957,5.856)}
\gppoint{gp mark 6}{(10.081,5.627)}
\gppoint{gp mark 6}{(10.205,5.348)}
\gppoint{gp mark 6}{(10.330,5.094)}
\gppoint{gp mark 6}{(10.454,4.865)}
\gpcolor{color=gp lt color border}
\draw[gp path] (2.997,8.441)--(2.997,0.985)--(10.454,0.985)--(10.454,8.441)--cycle;
%% coordinates of the plot area
\gpdefrectangularnode{gp plot 1}{\pgfpoint{2.997cm}{0.985cm}}{\pgfpoint{10.454cm}{8.441cm}}
\end{tikzpicture}
%% gnuplot variables

    \caption{Ni棒の磁気トルク曲線}
    \label{fig:Ni_adjust_torque}
  \end{center}
\end{figure}
\newpage
\subsubsection{Ni円盤の磁気トルク測定}
図\ref{fig:Ni_torque}にNi円盤の磁気トルク曲線を示す.
この最大値は$9.14\times10^4\ \si{\newton.\metre^{-2}}$である.
\begin{figure}[hptb]
  \begin{center}
    \begin{tikzpicture}[gnuplot]
%% generated with GNUPLOT 5.2p8 (Lua 5.3; terminal rev. Nov 2018, script rev. 108)
%% 2021年04月27日 14時39分47秒
\path (0.000,0.000) rectangle (12.500,8.750);
\gpcolor{color=gp lt color border}
\gpsetlinetype{gp lt border}
\gpsetdashtype{gp dt solid}
\gpsetlinewidth{1.00}
\draw[gp path] (2.997,1.358)--(3.177,1.358);
\draw[gp path] (10.454,1.358)--(10.274,1.358);
\node[gp node right] at (2.813,1.358) {$-9.0$};
\draw[gp path] (2.997,1.731)--(3.087,1.731);
\draw[gp path] (10.454,1.731)--(10.364,1.731);
\draw[gp path] (2.997,2.103)--(3.087,2.103);
\draw[gp path] (10.454,2.103)--(10.364,2.103);
\draw[gp path] (2.997,2.476)--(3.177,2.476);
\draw[gp path] (10.454,2.476)--(10.274,2.476);
\node[gp node right] at (2.813,2.476) {$-6.0$};
\draw[gp path] (2.997,2.849)--(3.087,2.849);
\draw[gp path] (10.454,2.849)--(10.364,2.849);
\draw[gp path] (2.997,3.222)--(3.087,3.222);
\draw[gp path] (10.454,3.222)--(10.364,3.222);
\draw[gp path] (2.997,3.595)--(3.177,3.595);
\draw[gp path] (10.454,3.595)--(10.274,3.595);
\node[gp node right] at (2.813,3.595) {$-3.0$};
\draw[gp path] (2.997,3.967)--(3.087,3.967);
\draw[gp path] (10.454,3.967)--(10.364,3.967);
\draw[gp path] (2.997,4.340)--(3.087,4.340);
\draw[gp path] (10.454,4.340)--(10.364,4.340);
\draw[gp path] (2.997,4.713)--(3.177,4.713);
\draw[gp path] (10.454,4.713)--(10.274,4.713);
\node[gp node right] at (2.813,4.713) {$0.0$};
\draw[gp path] (2.997,5.086)--(3.087,5.086);
\draw[gp path] (10.454,5.086)--(10.364,5.086);
\draw[gp path] (2.997,5.459)--(3.087,5.459);
\draw[gp path] (10.454,5.459)--(10.364,5.459);
\draw[gp path] (2.997,5.831)--(3.177,5.831);
\draw[gp path] (10.454,5.831)--(10.274,5.831);
\node[gp node right] at (2.813,5.831) {$3.0$};
\draw[gp path] (2.997,6.204)--(3.087,6.204);
\draw[gp path] (10.454,6.204)--(10.364,6.204);
\draw[gp path] (2.997,6.577)--(3.087,6.577);
\draw[gp path] (10.454,6.577)--(10.364,6.577);
\draw[gp path] (2.997,6.950)--(3.177,6.950);
\draw[gp path] (10.454,6.950)--(10.274,6.950);
\node[gp node right] at (2.813,6.950) {$6.0$};
\draw[gp path] (2.997,7.323)--(3.087,7.323);
\draw[gp path] (10.454,7.323)--(10.364,7.323);
\draw[gp path] (2.997,7.695)--(3.087,7.695);
\draw[gp path] (10.454,7.695)--(10.364,7.695);
\draw[gp path] (2.997,8.068)--(3.177,8.068);
\draw[gp path] (10.454,8.068)--(10.274,8.068);
\node[gp node right] at (2.813,8.068) {$9.0$};
\draw[gp path] (2.997,0.985)--(2.997,1.165);
\draw[gp path] (2.997,8.441)--(2.997,8.261);
\node[gp node center] at (2.997,0.677) {$0$};
\draw[gp path] (3.411,0.985)--(3.411,1.075);
\draw[gp path] (3.411,8.441)--(3.411,8.351);
\draw[gp path] (3.826,0.985)--(3.826,1.075);
\draw[gp path] (3.826,8.441)--(3.826,8.351);
\draw[gp path] (4.240,0.985)--(4.240,1.165);
\draw[gp path] (4.240,8.441)--(4.240,8.261);
\node[gp node center] at (4.240,0.677) {$30$};
\draw[gp path] (4.654,0.985)--(4.654,1.075);
\draw[gp path] (4.654,8.441)--(4.654,8.351);
\draw[gp path] (5.068,0.985)--(5.068,1.075);
\draw[gp path] (5.068,8.441)--(5.068,8.351);
\draw[gp path] (5.483,0.985)--(5.483,1.165);
\draw[gp path] (5.483,8.441)--(5.483,8.261);
\node[gp node center] at (5.483,0.677) {$60$};
\draw[gp path] (5.897,0.985)--(5.897,1.075);
\draw[gp path] (5.897,8.441)--(5.897,8.351);
\draw[gp path] (6.311,0.985)--(6.311,1.075);
\draw[gp path] (6.311,8.441)--(6.311,8.351);
\draw[gp path] (6.726,0.985)--(6.726,1.165);
\draw[gp path] (6.726,8.441)--(6.726,8.261);
\node[gp node center] at (6.726,0.677) {$90$};
\draw[gp path] (7.140,0.985)--(7.140,1.075);
\draw[gp path] (7.140,8.441)--(7.140,8.351);
\draw[gp path] (7.554,0.985)--(7.554,1.075);
\draw[gp path] (7.554,8.441)--(7.554,8.351);
\draw[gp path] (7.968,0.985)--(7.968,1.165);
\draw[gp path] (7.968,8.441)--(7.968,8.261);
\node[gp node center] at (7.968,0.677) {$120$};
\draw[gp path] (8.383,0.985)--(8.383,1.075);
\draw[gp path] (8.383,8.441)--(8.383,8.351);
\draw[gp path] (8.797,0.985)--(8.797,1.075);
\draw[gp path] (8.797,8.441)--(8.797,8.351);
\draw[gp path] (9.211,0.985)--(9.211,1.165);
\draw[gp path] (9.211,8.441)--(9.211,8.261);
\node[gp node center] at (9.211,0.677) {$150$};
\draw[gp path] (9.625,0.985)--(9.625,1.075);
\draw[gp path] (9.625,8.441)--(9.625,8.351);
\draw[gp path] (10.040,0.985)--(10.040,1.075);
\draw[gp path] (10.040,8.441)--(10.040,8.351);
\draw[gp path] (10.454,0.985)--(10.454,1.165);
\draw[gp path] (10.454,8.441)--(10.454,8.261);
\node[gp node center] at (10.454,0.677) {$180$};
\gpsetlinetype{gp lt axes}
\gpsetdashtype{gp dt axes}
\gpsetlinewidth{1.50}
\draw[gp path] (2.997,4.713)--(10.454,4.713);
\gpsetlinetype{gp lt border}
\gpsetdashtype{gp dt solid}
\gpsetlinewidth{1.00}
\draw[gp path] (2.997,8.441)--(2.997,0.985)--(10.454,0.985)--(10.454,8.441)--cycle;
\node[gp node center,rotate=-270] at (1.785,4.713) {単位体積あたりのトルク$L$ / $\times 10^{4}\ \si{\newton.\meter^{-2}}$};
\node[gp node center] at (6.725,0.215) {角度$\theta$ / $\si{\degree}$};
\gpcolor{rgb color={0.000,0.000,0.000}}
\gpsetpointsize{4.00}
\gppoint{gp mark 6}{(2.997,4.625)}
\gppoint{gp mark 6}{(3.121,4.403)}
\gppoint{gp mark 6}{(3.246,4.182)}
\gppoint{gp mark 6}{(3.370,3.983)}
\gppoint{gp mark 6}{(3.494,3.784)}
\gppoint{gp mark 6}{(3.618,3.585)}
\gppoint{gp mark 6}{(3.743,3.386)}
\gppoint{gp mark 6}{(3.867,3.187)}
\gppoint{gp mark 6}{(3.991,2.988)}
\gppoint{gp mark 6}{(4.116,2.811)}
\gppoint{gp mark 6}{(4.240,2.634)}
\gppoint{gp mark 6}{(4.364,2.435)}
\gppoint{gp mark 6}{(4.488,2.280)}
\gppoint{gp mark 6}{(4.613,2.126)}
\gppoint{gp mark 6}{(4.737,1.971)}
\gppoint{gp mark 6}{(4.861,1.838)}
\gppoint{gp mark 6}{(4.986,1.705)}
\gppoint{gp mark 6}{(5.110,1.595)}
\gppoint{gp mark 6}{(5.234,1.506)}
\gppoint{gp mark 6}{(5.358,1.440)}
\gppoint{gp mark 6}{(5.483,1.374)}
\gppoint{gp mark 6}{(5.607,1.352)}
\gppoint{gp mark 6}{(5.731,1.374)}
\gppoint{gp mark 6}{(5.856,1.440)}
\gppoint{gp mark 6}{(5.980,1.573)}
\gppoint{gp mark 6}{(6.104,1.794)}
\gppoint{gp mark 6}{(6.228,2.126)}
\gppoint{gp mark 6}{(6.353,2.656)}
\gppoint{gp mark 6}{(6.477,3.386)}
\gppoint{gp mark 6}{(6.601,4.293)}
\gppoint{gp mark 6}{(6.726,5.310)}
\gppoint{gp mark 6}{(6.850,6.217)}
\gppoint{gp mark 6}{(6.974,6.924)}
\gppoint{gp mark 6}{(7.098,7.411)}
\gppoint{gp mark 6}{(7.223,7.632)}
\gppoint{gp mark 6}{(7.347,7.942)}
\gppoint{gp mark 6}{(7.471,8.052)}
\gppoint{gp mark 6}{(7.595,8.119)}
\gppoint{gp mark 6}{(7.720,8.119)}
\gppoint{gp mark 6}{(7.844,8.096)}
\gppoint{gp mark 6}{(7.968,8.030)}
\gppoint{gp mark 6}{(8.093,7.964)}
\gppoint{gp mark 6}{(8.217,7.853)}
\gppoint{gp mark 6}{(8.341,7.743)}
\gppoint{gp mark 6}{(8.465,7.610)}
\gppoint{gp mark 6}{(8.590,7.477)}
\gppoint{gp mark 6}{(8.714,7.322)}
\gppoint{gp mark 6}{(8.838,7.146)}
\gppoint{gp mark 6}{(8.963,6.991)}
\gppoint{gp mark 6}{(9.087,6.814)}
\gppoint{gp mark 6}{(9.211,6.637)}
\gppoint{gp mark 6}{(9.335,6.438)}
\gppoint{gp mark 6}{(9.460,6.239)}
\gppoint{gp mark 6}{(9.584,6.040)}
\gppoint{gp mark 6}{(9.708,5.841)}
\gppoint{gp mark 6}{(9.833,5.642)}
\gppoint{gp mark 6}{(9.957,5.443)}
\gppoint{gp mark 6}{(10.081,5.222)}
\gppoint{gp mark 6}{(10.205,5.023)}
\gppoint{gp mark 6}{(10.330,4.824)}
\gppoint{gp mark 6}{(10.454,4.602)}
\gpcolor{color=gp lt color border}
\draw[gp path] (2.997,8.441)--(2.997,0.985)--(10.454,0.985)--(10.454,8.441)--cycle;
%% coordinates of the plot area
\gpdefrectangularnode{gp plot 1}{\pgfpoint{2.997cm}{0.985cm}}{\pgfpoint{10.454cm}{8.441cm}}
\end{tikzpicture}
%% gnuplot variables

    \caption{Ni円盤の磁気トルク曲線}
    \label{fig:Ni_torque}
  \end{center}
\end{figure}