\subsection{Fe-Si円盤の磁気トルク測定}
\label{subsec:res_Fe-Si}
\subsubsection{磁気トルク計の校正}
\label{subsubsec:res_Fe-Si_adjust}
図\ref{fig:Fe-Si_adjust}に校正時に得た磁気トルク相殺電流の角度依存性曲線を示す.
電流の最大値は$20.1\si{\milli\ampere}$なので(\ref{equ:calc_alpha})から
\begin{align}
  \alpha=6.04\times10^{-5} \si{\newton.\metre.(\milli\ampere)^{-1}}
\end{align}
である.この値と(\ref{equ:def_alpha})から磁気トルクを計算し,これをNi棒試料の体積で割ることにより
図\ref{fig:Fe-Si_adjust_torque}の磁気トルク曲線を得る.
ただし,トルクは単位体積あたり
\begin{figure}[hptb]
  \begin{center}
    \begin{tikzpicture}[gnuplot]
%% generated with GNUPLOT 5.2p8 (Lua 5.3; terminal rev. Nov 2018, script rev. 108)
%% 2021年04月19日 15時12分22秒
\path (0.000,0.000) rectangle (12.500,8.750);
\gpcolor{color=gp lt color border}
\gpsetlinetype{gp lt border}
\gpsetdashtype{gp dt solid}
\gpsetlinewidth{1.00}
\draw[gp path] (2.905,1.163)--(3.085,1.163);
\draw[gp path] (10.362,1.163)--(10.182,1.163);
\node[gp node right] at (2.721,1.163) {$-20$};
\draw[gp path] (2.905,2.050)--(3.085,2.050);
\draw[gp path] (10.362,2.050)--(10.182,2.050);
\node[gp node right] at (2.721,2.050) {$-15$};
\draw[gp path] (2.905,2.938)--(3.085,2.938);
\draw[gp path] (10.362,2.938)--(10.182,2.938);
\node[gp node right] at (2.721,2.938) {$-10$};
\draw[gp path] (2.905,3.825)--(3.085,3.825);
\draw[gp path] (10.362,3.825)--(10.182,3.825);
\node[gp node right] at (2.721,3.825) {$-5$};
\draw[gp path] (2.905,4.713)--(3.085,4.713);
\draw[gp path] (10.362,4.713)--(10.182,4.713);
\node[gp node right] at (2.721,4.713) {$0$};
\draw[gp path] (2.905,5.601)--(3.085,5.601);
\draw[gp path] (10.362,5.601)--(10.182,5.601);
\node[gp node right] at (2.721,5.601) {$5$};
\draw[gp path] (2.905,6.488)--(3.085,6.488);
\draw[gp path] (10.362,6.488)--(10.182,6.488);
\node[gp node right] at (2.721,6.488) {$10$};
\draw[gp path] (2.905,7.376)--(3.085,7.376);
\draw[gp path] (10.362,7.376)--(10.182,7.376);
\node[gp node right] at (2.721,7.376) {$15$};
\draw[gp path] (2.905,8.263)--(3.085,8.263);
\draw[gp path] (10.362,8.263)--(10.182,8.263);
\node[gp node right] at (2.721,8.263) {$20$};
\draw[gp path] (2.905,0.985)--(2.905,1.165);
\draw[gp path] (2.905,8.441)--(2.905,8.261);
\node[gp node center] at (2.905,0.677) {$0$};
\draw[gp path] (3.734,0.985)--(3.734,1.165);
\draw[gp path] (3.734,8.441)--(3.734,8.261);
\node[gp node center] at (3.734,0.677) {$20$};
\draw[gp path] (4.562,0.985)--(4.562,1.165);
\draw[gp path] (4.562,8.441)--(4.562,8.261);
\node[gp node center] at (4.562,0.677) {$40$};
\draw[gp path] (5.391,0.985)--(5.391,1.165);
\draw[gp path] (5.391,8.441)--(5.391,8.261);
\node[gp node center] at (5.391,0.677) {$60$};
\draw[gp path] (6.219,0.985)--(6.219,1.165);
\draw[gp path] (6.219,8.441)--(6.219,8.261);
\node[gp node center] at (6.219,0.677) {$80$};
\draw[gp path] (7.048,0.985)--(7.048,1.165);
\draw[gp path] (7.048,8.441)--(7.048,8.261);
\node[gp node center] at (7.048,0.677) {$100$};
\draw[gp path] (7.876,0.985)--(7.876,1.165);
\draw[gp path] (7.876,8.441)--(7.876,8.261);
\node[gp node center] at (7.876,0.677) {$120$};
\draw[gp path] (8.705,0.985)--(8.705,1.165);
\draw[gp path] (8.705,8.441)--(8.705,8.261);
\node[gp node center] at (8.705,0.677) {$140$};
\draw[gp path] (9.533,0.985)--(9.533,1.165);
\draw[gp path] (9.533,8.441)--(9.533,8.261);
\node[gp node center] at (9.533,0.677) {$160$};
\draw[gp path] (10.362,0.985)--(10.362,1.165);
\draw[gp path] (10.362,8.441)--(10.362,8.261);
\node[gp node center] at (10.362,0.677) {$180$};
\draw[gp path] (2.905,8.441)--(2.905,0.985)--(10.362,0.985)--(10.362,8.441)--cycle;
\node[gp node center,rotate=-270] at (1.877,4.713) {相殺電流$i$ / $\si{\milli\ampere}$};
\node[gp node center] at (6.633,0.215) {角度$\theta$ / $\si{\degree}$};
\gpcolor{rgb color={0.000,0.000,0.000}}
\gpsetpointsize{4.00}
\gppoint{gp mark 6}{(2.905,4.988)}
\gppoint{gp mark 6}{(3.029,4.713)}
\gppoint{gp mark 6}{(3.154,4.451)}
\gppoint{gp mark 6}{(3.278,4.182)}
\gppoint{gp mark 6}{(3.402,3.925)}
\gppoint{gp mark 6}{(3.526,3.670)}
\gppoint{gp mark 6}{(3.651,3.415)}
\gppoint{gp mark 6}{(3.775,3.166)}
\gppoint{gp mark 6}{(3.899,2.932)}
\gppoint{gp mark 6}{(4.024,2.692)}
\gppoint{gp mark 6}{(4.148,2.482)}
\gppoint{gp mark 6}{(4.272,2.273)}
\gppoint{gp mark 6}{(4.396,2.076)}
\gppoint{gp mark 6}{(4.521,1.890)}
\gppoint{gp mark 6}{(4.645,1.717)}
\gppoint{gp mark 6}{(4.769,1.569)}
\gppoint{gp mark 6}{(4.894,1.442)}
\gppoint{gp mark 6}{(5.018,1.336)}
\gppoint{gp mark 6}{(5.142,1.257)}
\gppoint{gp mark 6}{(5.266,1.212)}
\gppoint{gp mark 6}{(5.391,1.198)}
\gppoint{gp mark 6}{(5.515,1.228)}
\gppoint{gp mark 6}{(5.639,1.306)}
\gppoint{gp mark 6}{(5.764,1.431)}
\gppoint{gp mark 6}{(5.888,1.614)}
\gppoint{gp mark 6}{(6.012,1.866)}
\gppoint{gp mark 6}{(6.136,2.195)}
\gppoint{gp mark 6}{(6.261,2.594)}
\gppoint{gp mark 6}{(6.385,3.073)}
\gppoint{gp mark 6}{(6.509,3.610)}
\gppoint{gp mark 6}{(6.634,4.204)}
\gppoint{gp mark 6}{(6.758,4.849)}
\gppoint{gp mark 6}{(6.882,5.435)}
\gppoint{gp mark 6}{(7.006,6.038)}
\gppoint{gp mark 6}{(7.131,6.585)}
\gppoint{gp mark 6}{(7.255,6.995)}
\gppoint{gp mark 6}{(7.379,7.386)}
\gppoint{gp mark 6}{(7.503,7.715)}
\gppoint{gp mark 6}{(7.628,7.936)}
\gppoint{gp mark 6}{(7.752,8.095)}
\gppoint{gp mark 6}{(7.876,8.204)}
\gppoint{gp mark 6}{(8.001,8.261)}
\gppoint{gp mark 6}{(8.125,8.276)}
\gppoint{gp mark 6}{(8.249,8.258)}
\gppoint{gp mark 6}{(8.373,8.202)}
\gppoint{gp mark 6}{(8.498,8.120)}
\gppoint{gp mark 6}{(8.622,8.009)}
\gppoint{gp mark 6}{(8.746,7.876)}
\gppoint{gp mark 6}{(8.871,7.730)}
\gppoint{gp mark 6}{(8.995,7.549)}
\gppoint{gp mark 6}{(9.119,7.376)}
\gppoint{gp mark 6}{(9.243,7.161)}
\gppoint{gp mark 6}{(9.368,6.948)}
\gppoint{gp mark 6}{(9.492,6.731)}
\gppoint{gp mark 6}{(9.616,6.490)}
\gppoint{gp mark 6}{(9.741,6.253)}
\gppoint{gp mark 6}{(9.865,5.997)}
\gppoint{gp mark 6}{(9.989,5.745)}
\gppoint{gp mark 6}{(10.113,5.491)}
\gppoint{gp mark 6}{(10.238,5.229)}
\gppoint{gp mark 6}{(10.362,4.975)}
\gpcolor{color=gp lt color border}
\draw[gp path] (2.905,8.441)--(2.905,0.985)--(10.362,0.985)--(10.362,8.441)--cycle;
%% coordinates of the plot area
\gpdefrectangularnode{gp plot 1}{\pgfpoint{2.905cm}{0.985cm}}{\pgfpoint{10.362cm}{8.441cm}}
\end{tikzpicture}
%% gnuplot variables

    \caption{相殺電流の角度依存性}
    \label{fig:Fe-Si_adjust}
  \end{center}
\end{figure}
\begin{figure}[hptb]
  \begin{center}
    \begin{tikzpicture}[gnuplot]
%% generated with GNUPLOT 5.2p8 (Lua 5.3; terminal rev. Nov 2018, script rev. 108)
%% 2021年05月02日 00時15分59秒
\path (0.000,0.000) rectangle (12.500,8.750);
\gpcolor{color=gp lt color border}
\gpsetlinetype{gp lt border}
\gpsetdashtype{gp dt solid}
\gpsetlinewidth{1.00}
\draw[gp path] (2.997,0.985)--(3.177,0.985);
\draw[gp path] (10.454,0.985)--(10.274,0.985);
\node[gp node right] at (2.813,0.985) {$-6.0$};
\draw[gp path] (2.997,2.228)--(3.177,2.228);
\draw[gp path] (10.454,2.228)--(10.274,2.228);
\node[gp node right] at (2.813,2.228) {$-4.0$};
\draw[gp path] (2.997,3.470)--(3.177,3.470);
\draw[gp path] (10.454,3.470)--(10.274,3.470);
\node[gp node right] at (2.813,3.470) {$-2.0$};
\draw[gp path] (2.997,4.713)--(3.177,4.713);
\draw[gp path] (10.454,4.713)--(10.274,4.713);
\node[gp node right] at (2.813,4.713) {$0.0$};
\draw[gp path] (2.997,5.956)--(3.177,5.956);
\draw[gp path] (10.454,5.956)--(10.274,5.956);
\node[gp node right] at (2.813,5.956) {$2.0$};
\draw[gp path] (2.997,7.198)--(3.177,7.198);
\draw[gp path] (10.454,7.198)--(10.274,7.198);
\node[gp node right] at (2.813,7.198) {$4.0$};
\draw[gp path] (2.997,8.441)--(3.177,8.441);
\draw[gp path] (10.454,8.441)--(10.274,8.441);
\node[gp node right] at (2.813,8.441) {$6.0$};
\draw[gp path] (2.997,0.985)--(2.997,1.165);
\draw[gp path] (2.997,8.441)--(2.997,8.261);
\node[gp node center] at (2.997,0.677) {$0$};
\draw[gp path] (3.826,0.985)--(3.826,1.165);
\draw[gp path] (3.826,8.441)--(3.826,8.261);
\node[gp node center] at (3.826,0.677) {$20$};
\draw[gp path] (4.654,0.985)--(4.654,1.165);
\draw[gp path] (4.654,8.441)--(4.654,8.261);
\node[gp node center] at (4.654,0.677) {$40$};
\draw[gp path] (5.483,0.985)--(5.483,1.165);
\draw[gp path] (5.483,8.441)--(5.483,8.261);
\node[gp node center] at (5.483,0.677) {$60$};
\draw[gp path] (6.311,0.985)--(6.311,1.165);
\draw[gp path] (6.311,8.441)--(6.311,8.261);
\node[gp node center] at (6.311,0.677) {$80$};
\draw[gp path] (7.140,0.985)--(7.140,1.165);
\draw[gp path] (7.140,8.441)--(7.140,8.261);
\node[gp node center] at (7.140,0.677) {$100$};
\draw[gp path] (7.968,0.985)--(7.968,1.165);
\draw[gp path] (7.968,8.441)--(7.968,8.261);
\node[gp node center] at (7.968,0.677) {$120$};
\draw[gp path] (8.797,0.985)--(8.797,1.165);
\draw[gp path] (8.797,8.441)--(8.797,8.261);
\node[gp node center] at (8.797,0.677) {$140$};
\draw[gp path] (9.625,0.985)--(9.625,1.165);
\draw[gp path] (9.625,8.441)--(9.625,8.261);
\node[gp node center] at (9.625,0.677) {$160$};
\draw[gp path] (10.454,0.985)--(10.454,1.165);
\draw[gp path] (10.454,8.441)--(10.454,8.261);
\node[gp node center] at (10.454,0.677) {$180$};
\gpsetlinetype{gp lt axes}
\gpsetdashtype{gp dt axes}
\gpsetlinewidth{1.50}
\draw[gp path] (2.997,4.713)--(10.454,4.713);
\gpsetlinetype{gp lt border}
\gpsetdashtype{gp dt solid}
\gpsetlinewidth{1.00}
\draw[gp path] (2.997,8.441)--(2.997,0.985)--(10.454,0.985)--(10.454,8.441)--cycle;
\node[gp node center,rotate=-270] at (1.785,4.713) {単位体積あたりのトルク$L$ / $\times 10^{4}\ \si{\newton.\meter^{-2}}$};
\node[gp node center] at (6.725,0.215) {角度$\theta$ / $\si{\degree}$};
\gpcolor{rgb color={0.000,0.000,0.000}}
\gpsetpointsize{4.00}
\gppoint{gp mark 6}{(2.997,4.977)}
\gppoint{gp mark 6}{(3.121,4.713)}
\gppoint{gp mark 6}{(3.246,4.461)}
\gppoint{gp mark 6}{(3.370,4.202)}
\gppoint{gp mark 6}{(3.494,3.955)}
\gppoint{gp mark 6}{(3.618,3.709)}
\gppoint{gp mark 6}{(3.743,3.463)}
\gppoint{gp mark 6}{(3.867,3.224)}
\gppoint{gp mark 6}{(3.991,2.999)}
\gppoint{gp mark 6}{(4.116,2.768)}
\gppoint{gp mark 6}{(4.240,2.565)}
\gppoint{gp mark 6}{(4.364,2.365)}
\gppoint{gp mark 6}{(4.488,2.175)}
\gppoint{gp mark 6}{(4.613,1.996)}
\gppoint{gp mark 6}{(4.737,1.830)}
\gppoint{gp mark 6}{(4.861,1.687)}
\gppoint{gp mark 6}{(4.986,1.565)}
\gppoint{gp mark 6}{(5.110,1.463)}
\gppoint{gp mark 6}{(5.234,1.387)}
\gppoint{gp mark 6}{(5.358,1.343)}
\gppoint{gp mark 6}{(5.483,1.330)}
\gppoint{gp mark 6}{(5.607,1.359)}
\gppoint{gp mark 6}{(5.731,1.434)}
\gppoint{gp mark 6}{(5.856,1.555)}
\gppoint{gp mark 6}{(5.980,1.731)}
\gppoint{gp mark 6}{(6.104,1.973)}
\gppoint{gp mark 6}{(6.228,2.289)}
\gppoint{gp mark 6}{(6.353,2.674)}
\gppoint{gp mark 6}{(6.477,3.134)}
\gppoint{gp mark 6}{(6.601,3.651)}
\gppoint{gp mark 6}{(6.726,4.223)}
\gppoint{gp mark 6}{(6.850,4.844)}
\gppoint{gp mark 6}{(6.974,5.408)}
\gppoint{gp mark 6}{(7.098,5.988)}
\gppoint{gp mark 6}{(7.223,6.515)}
\gppoint{gp mark 6}{(7.347,6.909)}
\gppoint{gp mark 6}{(7.471,7.286)}
\gppoint{gp mark 6}{(7.595,7.602)}
\gppoint{gp mark 6}{(7.720,7.815)}
\gppoint{gp mark 6}{(7.844,7.968)}
\gppoint{gp mark 6}{(7.968,8.073)}
\gppoint{gp mark 6}{(8.093,8.128)}
\gppoint{gp mark 6}{(8.217,8.142)}
\gppoint{gp mark 6}{(8.341,8.125)}
\gppoint{gp mark 6}{(8.465,8.071)}
\gppoint{gp mark 6}{(8.590,7.992)}
\gppoint{gp mark 6}{(8.714,7.885)}
\gppoint{gp mark 6}{(8.838,7.758)}
\gppoint{gp mark 6}{(8.963,7.616)}
\gppoint{gp mark 6}{(9.087,7.442)}
\gppoint{gp mark 6}{(9.211,7.276)}
\gppoint{gp mark 6}{(9.335,7.069)}
\gppoint{gp mark 6}{(9.460,6.864)}
\gppoint{gp mark 6}{(9.584,6.655)}
\gppoint{gp mark 6}{(9.708,6.424)}
\gppoint{gp mark 6}{(9.833,6.196)}
\gppoint{gp mark 6}{(9.957,5.949)}
\gppoint{gp mark 6}{(10.081,5.706)}
\gppoint{gp mark 6}{(10.205,5.462)}
\gppoint{gp mark 6}{(10.330,5.209)}
\gppoint{gp mark 6}{(10.454,4.965)}
\gpcolor{color=gp lt color border}
\draw[gp path] (2.997,8.441)--(2.997,0.985)--(10.454,0.985)--(10.454,8.441)--cycle;
%% coordinates of the plot area
\gpdefrectangularnode{gp plot 1}{\pgfpoint{2.997cm}{0.985cm}}{\pgfpoint{10.454cm}{8.441cm}}
\end{tikzpicture}
%% gnuplot variables

    \caption{Ni棒の磁気トルク曲線}
    \label{fig:Fe-Si_adjust_torque}
  \end{center}
\end{figure}
\newpage
\subsubsection{Fe-Si円盤の磁気トルク測定}
図\ref{fig:Fe-Si_torque}にFe-Si円盤の磁気トルク曲線を示す.
\begin{figure}[hptb]
  \begin{center}
    \begin{tikzpicture}[gnuplot]
%% generated with GNUPLOT 5.2p8 (Lua 5.3; terminal rev. Nov 2018, script rev. 108)
%% 2021年04月19日 15時43分40秒
\path (0.000,0.000) rectangle (12.500,8.750);
\gpcolor{color=gp lt color border}
\gpsetlinetype{gp lt border}
\gpsetdashtype{gp dt solid}
\gpsetlinewidth{1.00}
\draw[gp path] (3.273,1.731)--(3.453,1.731);
\draw[gp path] (10.730,1.731)--(10.550,1.731);
\node[gp node right] at (3.089,1.731) {$-0.0004$};
\draw[gp path] (3.273,2.476)--(3.363,2.476);
\draw[gp path] (10.730,2.476)--(10.640,2.476);
\draw[gp path] (3.273,3.222)--(3.453,3.222);
\draw[gp path] (10.730,3.222)--(10.550,3.222);
\node[gp node right] at (3.089,3.222) {$-0.0002$};
\draw[gp path] (3.273,3.967)--(3.363,3.967);
\draw[gp path] (10.730,3.967)--(10.640,3.967);
\draw[gp path] (3.273,4.713)--(3.453,4.713);
\draw[gp path] (10.730,4.713)--(10.550,4.713);
\node[gp node right] at (3.089,4.713) {$0$};
\draw[gp path] (3.273,5.459)--(3.363,5.459);
\draw[gp path] (10.730,5.459)--(10.640,5.459);
\draw[gp path] (3.273,6.204)--(3.453,6.204);
\draw[gp path] (10.730,6.204)--(10.550,6.204);
\node[gp node right] at (3.089,6.204) {$0.0002$};
\draw[gp path] (3.273,6.950)--(3.363,6.950);
\draw[gp path] (10.730,6.950)--(10.640,6.950);
\draw[gp path] (3.273,7.695)--(3.453,7.695);
\draw[gp path] (10.730,7.695)--(10.550,7.695);
\node[gp node right] at (3.089,7.695) {$0.0004$};
\draw[gp path] (3.469,0.985)--(3.469,1.165);
\draw[gp path] (3.469,8.441)--(3.469,8.261);
\node[gp node center] at (3.469,0.677) {$0$};
\draw[gp path] (4.450,0.985)--(4.450,1.165);
\draw[gp path] (4.450,8.441)--(4.450,8.261);
\node[gp node center] at (4.450,0.677) {$50$};
\draw[gp path] (5.432,0.985)--(5.432,1.165);
\draw[gp path] (5.432,8.441)--(5.432,8.261);
\node[gp node center] at (5.432,0.677) {$100$};
\draw[gp path] (6.413,0.985)--(6.413,1.165);
\draw[gp path] (6.413,8.441)--(6.413,8.261);
\node[gp node center] at (6.413,0.677) {$150$};
\draw[gp path] (7.394,0.985)--(7.394,1.165);
\draw[gp path] (7.394,8.441)--(7.394,8.261);
\node[gp node center] at (7.394,0.677) {$200$};
\draw[gp path] (8.375,0.985)--(8.375,1.165);
\draw[gp path] (8.375,8.441)--(8.375,8.261);
\node[gp node center] at (8.375,0.677) {$250$};
\draw[gp path] (9.356,0.985)--(9.356,1.165);
\draw[gp path] (9.356,8.441)--(9.356,8.261);
\node[gp node center] at (9.356,0.677) {$300$};
\draw[gp path] (10.338,0.985)--(10.338,1.165);
\draw[gp path] (10.338,8.441)--(10.338,8.261);
\node[gp node center] at (10.338,0.677) {$350$};
\draw[gp path] (3.273,8.441)--(3.273,0.985)--(10.730,0.985)--(10.730,8.441)--cycle;
\node[gp node center,rotate=-270] at (1.509,4.713) {トルク$L$ / $\si{\newton.\meter}$};
\node[gp node center] at (7.001,0.215) {角度$\theta$ / $\si{\degree}$};
\gpcolor{rgb color={0.000,0.000,0.000}}
\gpsetpointsize{4.00}
\gppoint{gp mark 6}{(3.469,2.387)}
\gppoint{gp mark 6}{(3.528,2.811)}
\gppoint{gp mark 6}{(3.587,3.277)}
\gppoint{gp mark 6}{(3.646,3.753)}
\gppoint{gp mark 6}{(3.705,4.249)}
\gppoint{gp mark 6}{(3.764,4.693)}
\gppoint{gp mark 6}{(3.822,5.098)}
\gppoint{gp mark 6}{(3.881,5.426)}
\gppoint{gp mark 6}{(3.940,5.678)}
\gppoint{gp mark 6}{(3.999,5.849)}
\gppoint{gp mark 6}{(4.058,5.933)}
\gppoint{gp mark 6}{(4.117,5.928)}
\gppoint{gp mark 6}{(4.176,5.849)}
\gppoint{gp mark 6}{(4.235,5.706)}
\gppoint{gp mark 6}{(4.293,5.504)}
\gppoint{gp mark 6}{(4.352,5.269)}
\gppoint{gp mark 6}{(4.411,5.001)}
\gppoint{gp mark 6}{(4.470,4.714)}
\gppoint{gp mark 6}{(4.529,4.438)}
\gppoint{gp mark 6}{(4.588,4.182)}
\gppoint{gp mark 6}{(4.647,3.968)}
\gppoint{gp mark 6}{(4.706,3.791)}
\gppoint{gp mark 6}{(4.764,3.678)}
\gppoint{gp mark 6}{(4.823,3.636)}
\gppoint{gp mark 6}{(4.882,3.676)}
\gppoint{gp mark 6}{(4.941,3.809)}
\gppoint{gp mark 6}{(5.000,4.016)}
\gppoint{gp mark 6}{(5.059,4.316)}
\gppoint{gp mark 6}{(5.118,4.702)}
\gppoint{gp mark 6}{(5.176,5.149)}
\gppoint{gp mark 6}{(5.235,5.606)}
\gppoint{gp mark 6}{(5.294,6.128)}
\gppoint{gp mark 6}{(5.353,6.590)}
\gppoint{gp mark 6}{(5.412,7.033)}
\gppoint{gp mark 6}{(5.471,7.422)}
\gppoint{gp mark 6}{(5.530,7.725)}
\gppoint{gp mark 6}{(5.589,7.940)}
\gppoint{gp mark 6}{(5.647,8.063)}
\gppoint{gp mark 6}{(5.706,8.090)}
\gppoint{gp mark 6}{(5.765,8.005)}
\gppoint{gp mark 6}{(5.824,7.825)}
\gppoint{gp mark 6}{(5.883,7.546)}
\gppoint{gp mark 6}{(5.942,7.190)}
\gppoint{gp mark 6}{(6.001,6.751)}
\gppoint{gp mark 6}{(6.060,6.230)}
\gppoint{gp mark 6}{(6.118,5.755)}
\gppoint{gp mark 6}{(6.177,5.201)}
\gppoint{gp mark 6}{(6.236,4.615)}
\gppoint{gp mark 6}{(6.295,4.039)}
\gppoint{gp mark 6}{(6.354,3.523)}
\gppoint{gp mark 6}{(6.413,3.018)}
\gppoint{gp mark 6}{(6.472,2.541)}
\gppoint{gp mark 6}{(6.531,2.159)}
\gppoint{gp mark 6}{(6.589,1.835)}
\gppoint{gp mark 6}{(6.648,1.628)}
\gppoint{gp mark 6}{(6.707,1.518)}
\gppoint{gp mark 6}{(6.766,1.492)}
\gppoint{gp mark 6}{(6.825,1.583)}
\gppoint{gp mark 6}{(6.884,1.784)}
\gppoint{gp mark 6}{(6.943,2.054)}
\gppoint{gp mark 6}{(7.002,2.407)}
\gppoint{gp mark 6}{(7.060,2.853)}
\gppoint{gp mark 6}{(7.119,3.330)}
\gppoint{gp mark 6}{(7.178,3.797)}
\gppoint{gp mark 6}{(7.237,4.307)}
\gppoint{gp mark 6}{(7.296,4.666)}
\gppoint{gp mark 6}{(7.355,5.157)}
\gppoint{gp mark 6}{(7.414,5.477)}
\gppoint{gp mark 6}{(7.472,5.730)}
\gppoint{gp mark 6}{(7.531,5.894)}
\gppoint{gp mark 6}{(7.590,5.966)}
\gppoint{gp mark 6}{(7.649,5.980)}
\gppoint{gp mark 6}{(7.708,5.905)}
\gppoint{gp mark 6}{(7.767,5.752)}
\gppoint{gp mark 6}{(7.826,5.554)}
\gppoint{gp mark 6}{(7.885,5.316)}
\gppoint{gp mark 6}{(7.943,5.049)}
\gppoint{gp mark 6}{(8.002,4.773)}
\gppoint{gp mark 6}{(8.061,4.503)}
\gppoint{gp mark 6}{(8.120,4.247)}
\gppoint{gp mark 6}{(8.179,4.018)}
\gppoint{gp mark 6}{(8.238,3.840)}
\gppoint{gp mark 6}{(8.297,3.718)}
\gppoint{gp mark 6}{(8.356,3.682)}
\gppoint{gp mark 6}{(8.414,3.711)}
\gppoint{gp mark 6}{(8.473,3.826)}
\gppoint{gp mark 6}{(8.532,4.050)}
\gppoint{gp mark 6}{(8.591,4.342)}
\gppoint{gp mark 6}{(8.650,4.714)}
\gppoint{gp mark 6}{(8.709,5.149)}
\gppoint{gp mark 6}{(8.768,5.639)}
\gppoint{gp mark 6}{(8.827,6.118)}
\gppoint{gp mark 6}{(8.885,6.603)}
\gppoint{gp mark 6}{(8.944,7.054)}
\gppoint{gp mark 6}{(9.003,7.426)}
\gppoint{gp mark 6}{(9.062,7.731)}
\gppoint{gp mark 6}{(9.121,7.943)}
\gppoint{gp mark 6}{(9.180,8.071)}
\gppoint{gp mark 6}{(9.239,8.094)}
\gppoint{gp mark 6}{(9.297,8.005)}
\gppoint{gp mark 6}{(9.356,7.821)}
\gppoint{gp mark 6}{(9.415,7.535)}
\gppoint{gp mark 6}{(9.474,7.183)}
\gppoint{gp mark 6}{(9.533,6.743)}
\gppoint{gp mark 6}{(9.592,6.254)}
\gppoint{gp mark 6}{(9.651,5.720)}
\gppoint{gp mark 6}{(9.710,5.145)}
\gppoint{gp mark 6}{(9.768,4.610)}
\gppoint{gp mark 6}{(9.827,4.032)}
\gppoint{gp mark 6}{(9.886,3.472)}
\gppoint{gp mark 6}{(9.945,2.987)}
\gppoint{gp mark 6}{(10.004,2.532)}
\gppoint{gp mark 6}{(10.063,2.134)}
\gppoint{gp mark 6}{(10.122,1.832)}
\gppoint{gp mark 6}{(10.181,1.603)}
\gppoint{gp mark 6}{(10.239,1.477)}
\gppoint{gp mark 6}{(10.298,1.456)}
\gppoint{gp mark 6}{(10.357,1.545)}
\gppoint{gp mark 6}{(10.416,1.721)}
\gppoint{gp mark 6}{(10.475,2.021)}
\gppoint{gp mark 6}{(10.534,2.378)}
\gpcolor{color=gp lt color border}
\draw[gp path] (3.273,8.441)--(3.273,0.985)--(10.730,0.985)--(10.730,8.441)--cycle;
%% coordinates of the plot area
\gpdefrectangularnode{gp plot 1}{\pgfpoint{3.273cm}{0.985cm}}{\pgfpoint{10.730cm}{8.441cm}}
\end{tikzpicture}
%% gnuplot variables

    \caption{Fe-Si円盤の磁気トルク曲線}
    \label{fig:Fe-Si_torque}
  \end{center}
\end{figure}