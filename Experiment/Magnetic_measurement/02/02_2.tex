\subsection{試料について}
以下の表に試料の形状,材質,自発磁化の有無を示す.
ただし体積は質量と密度が既知の場合$(質量)/(密度)$,質量と密度が不明の場合形状から算出した.
\begin{table}[h]
\caption{試料の諸元\cite{rikanenpyo}}
\label{tab:siryou_specification}
\centering
\scalebox{0.75}{
\begin{tabular}{cc|cccccccc}
\hline
No. & 名称 & 材質 & 密度 / $\si{\gram.\centi\metre^{-3}}$ & 直径$R$ / $\si{\milli\metre}$ & 高さ$h$ / $\si{\milli\metre}$ & 質量 / $\si{\gram}$ & 体積 / $\times10^{-8}\si{\metre^3}$ & 自発磁化 \\
\hline \hline
1 & Ni円盤 & Ni & 8.91 & 5.05 & 1.85 & 0.135 & $1.52$ & 有 \\
2 & Ni棒 & Ni & 8.91 & 2 & 7 && $2.20$ & 有 \\
3 & Fe-Si円盤 & Fe-Si結晶 & & 10 & 0.3 && $2.36$\\
4 & La-Gd & 10wt\%La-Gd & 7.90 & 7 & 1.06 & 0.430 & $5.44$ & 有(@$77\si{\kelvin}$)\\
\hline
\end{tabular}
}
\end{table}
\mfig[width=6cm]{fig/def_siryou.jpg}{試料の形状}