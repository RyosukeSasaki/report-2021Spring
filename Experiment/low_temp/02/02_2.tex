\subsection{実験準備}
\subsubsection{試料の準備}
各試料を以下の手順にしたがい図\ref{fig:fig/siryou.png}のように試料台に取り付けた.以下の作業はゴム手袋をした状態で行った.
まず鉛試料は$25\ \si{\milli\metre}$,その他の試料は$100\ \si{\milli\metre}$程度にハサミで切った.
金鉄合金については表面の被覆を有機溶剤で除去した.
また,各試料の直径を5箇所程度マイクロメーターを用いて測定した.
この試料片を試料台にマスキングテープで仮留めした.
試料台は伝導性が高く安価な銅製であり, D型の断面形状をしており,表面はエポキシで絶縁されている.
ここで鉛試料に関しては紙やすりを用いて両端の酸化皮膜を削り取っている.
次に$0.3\phi$のスズメッキ線を試料上に配置し,端子台の電流電圧端子に接続した.
次に銀ペースト(DOTITE)をトルエンで希釈したものを各導通部に塗布した.
銀ペーストが乾燥した後,各端子の導通をデジタルマルチメータで確認する.その際各端子間の抵抗値が数$\si{\ohm}$以内であれば正常に導通が取れている.
その後余計なスズメッキ線を切除した.
最後に試料線,スズメッキ線などをワニスで補強した.
\mfig[width=10cm]{fig/siryou.png}{試料台}
\subsubsection{実験装置の準備}
図\ref{fig:fig/cryo.png}に本実験で用いるクライオスタットの概略図を示す.
このクライオスタットは2層の輻射シールドとGM冷凍機を用いて構成される.
GM冷凍機は住友重機械工業 SRDK-101D-A11を用いている.また各ステージ間は対流による熱の伝達を減らすため
ターボ分子ポンプとロータリーポンプにより高真空に保たれる.このクライオスタットは理想的には$3\ \si{\kelvin}$以下までの冷却が可能であるが
実際には構造物や配線などを介して熱が流入するため実験環境で達成される低温は$3.5 \si{\kelvin}$程度である.
可能な限り熱流入を抑えるため,これらの構造物や配線は熱伝導率が比較的低いステンレスやキュプロニッケル,マンガンなどの合金が用いられている.
一方で試料台との接続部など均一な温度分布を実現したい箇所については純銅を用いて熱伝導を高めている.

このクライオスタットに以下の手順で試料台を取り付けた.
まずクライオスタットの第2ステージにある同軸端子に試料台を取り付け,ロックインアンプとの接続を確認した.
そして試料台の上部に炭素抵抗温度計とマンガニン線ヒーターが取り付けられた円盤を載せ,M3のネジで固定した.
次に超電導磁石を上から被せネジで固定し,電磁石のコネクタを接続した.
更にステージと試料台の熱接触を高めるため,試料台上部の円盤とステージを銅線で接続した.
この状態で各配線が適切に接続されていることをロックインアンプの数値から確認した.
そして第1,第2ステージの輻射シールド及び断熱真空チャンバーを取り付けた.この際,ガスケットには真空用のグリスを塗布した.

次にチャンバーの真空引き,冷却を行う.まずクライオスタットのバルブを開放し真空ポンプを稼働した.
真空度が$10^{-2}\ \si{torr}$程度になったところでコンピューターで温度を監視しながら冷却を開始した.
このまま$3.5\ \si{\kelvin}$程度に温度が下がるまで,冷却を続けた.
\mfig[width=10cm]{fig/cryo.png}{クライオスタットの概略図}