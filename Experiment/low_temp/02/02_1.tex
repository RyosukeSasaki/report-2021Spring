\subsection{試料について}
今回の実験では以下の線状試料を用いた.

鉛試料は最も転移温度の高い第一種超伝導体であり$7.2\ \si{\kelvin}$で超伝導に転移する.毒性があるため取り扱う際にはゴム手袋を用いた.
金鉄合金(Au$+0.07\ {\rm at}.\%$Fe)は低温測定用熱電対などに用いられる合金であり,近藤効果を示す.
金試料は高い伝導性を示すことから電子機器などで用いられる.一方で$0.1\ \si{\milli\kelvin}$まで超電導性を示さないため,今回の実験装置では超伝導転移は観測できない.
\begin{table}[h]
\caption{測定した試料}
\label{tab:siryou}
\centering
\begin{tabular}{cccc}
\hline
試料No.&名称&断面積 / $\si{\milli\metre^2}$&電圧端子間距離 / $\si{\milli\metre}$\\
\hline \hline
1&鉛試料1&$(1.015\pm0.191)\times10^{-2}$&$10.47$\\
2&金鉄合金&$(2.987\pm0.000)\times10^{-2}$&$57.83$\\
3&鉛試料2&$(9.218\pm0.252)\times10^{-3}$&$12.20$\\
4&金試料&$(1.662\pm0.321)\times10^{-3}$&$58.33$\\
\hline
\end{tabular}
\end{table}