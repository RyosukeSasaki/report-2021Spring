\subsection{測定手順}
\subsubsection{測定1:電気抵抗の測定($3.6$-$20\ \si{\kelvin}$)}
Cryocon温度コントローラーを用いて試料温度を調整しながら試料の電気抵抗を測定した.
温度コントローラーはカーボン抵抗温度計の値を入力としたPIDによる閉ループ制御を行うことで試料温度を目標温度に近づける.
また必要であればAutotune機能を用いてPIDゲインを自動的に設定することができる.

以上の機能を用いて$3.6\ \si{\kelvin}$から$20\ \si{\kelvin}$まで温度を$0.5\ \si{\kelvin}$間隔で変化させながら,ロックインアンプの電圧値を記録した.
この際実験室のコンピュータを用いて抵抗の変化を可視化しながら実験を行った.
ただし鉛の超電導転移点付近では急激に出力が変化するため,温度刻みを$0.01\ \si{\kelvin}$程度に細かくして記録した.
以上で記録した電圧値は(\ref{equ:rho})を用いて抵抗率に変換する.
\subsubsection{測定2:電気抵抗の測定($20$-$300\ \si{\kelvin}$)}
上記の測定終了後,冷凍機を停止すると試料温度が室温まで上昇するので,この過程で$20$から$300\ \si{\kelvin}$での
ロックインアンプの電圧値を同様に記録した.これと上記の測定を合わせることで$3.6$から$300\ \si{\kelvin}$での抵抗率の温度依存性を得られる.
\subsubsection{測定3:臨界磁場の測定}
クライオスタット内の超伝導磁石は電流測定用の$0.1\ \si{\ohm}$抵抗と直列に接続されており,
DMTを用いてその両端の電圧を測定することで実際に流れている電流を測定した.
この電流値$I$から実際に発生する磁場は以下の計算式で与えられる.
\begin{align}
  H\ \si{Oe}=0.73721\ \si{Oe.\milli\ampere^{-1}}\times I\ \si{\milli\ampere}
\end{align}
温度コントローラを用いて試料温度を$7\ \si{\kelvin}$に設定した.
この状態で超伝導磁石に流す電流を徐々に増やし,試料の超伝導が破壊される磁場(臨界磁場)を探した.
臨界磁場が見つかったらその付近で電流を細かく刻みながらロックインアンプの出力を記録した.
$6.5$-$4.0\ \si{\kelvin}$まで$0.5\ \si{\kelvin}$刻みで温度を変えつつ同様の測定を行った.
以上で記録した電圧値は(\ref{equ:rho})を用いて抵抗率に変換する.