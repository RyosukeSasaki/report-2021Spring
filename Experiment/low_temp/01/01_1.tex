\subsection{電気抵抗の測定法}
電気抵抗の測定方法には二端子法と四端子法がある.
図\ref{fig:fig/2tansi.png},図\ref{fig:fig/4tansi.png}に二端子法, 四端子法の回路図を示す.
$R_L$は試料の抵抗, $R_S$はリード線など測定器の抵抗, $R_V$は電圧計の内部抵抗である.

二端子法では試料の両端にリード線を取り付け,既知の電流を流したときに両端で生じる電位差を測定することで試料の抵抗値を測定する.
ただし二端子法ではリード線などの抵抗値を含めた抵抗値しか測定できず,試料の抵抗値が小さいときは測定が困難になる.

一方で四端子法では既知電流を流すリード線と電圧計のリード線を別に試料に取り付ける.
すなわち試料に4つの端子を取り付けることから四端子法と呼ばれる.
ここで電圧計の入力インピーダンスが非常に大きくほとんど電流が流れ込まないとすると$R_{S2}$で生じる電圧降下は非常に少なく,
$R_L$が小さくてもその両端の電圧を正確に測定できる.したがって$R_L$が小さい試料においては四端子法での測定が適している.
また本実験では後述するロックインアンプを用いて試料の抵抗値を測定している.

測定により得られた抵抗値は試料の形状に依存する量なので,より本質的な物理量として電気抵抗率$\rho$を用いる.
電気抵抗率は試料が断面形状が一定の柱状であるとすると,その断面積$S$,長さ$L$,抵抗値$R$を用いて
\begin{align}
  \rho=R\frac{S}{L}=\frac{\Delta VS}{IL}
\end{align}
と表される.ここで$\Delta V$は試料の両端の電位差, $I$は流れる電流である.
\clearpage
\mfig[width=4cm]{fig/2tansi.png}{二端子法の回路図}
\mfig[width=8cm]{fig/4tansi.png}{四端子法の回路図}