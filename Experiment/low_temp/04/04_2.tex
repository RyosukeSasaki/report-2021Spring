\subsection{電気抵抗の温度依存性の解析}
超伝導状態にない金属の電気抵抗率の温度依存性は以下の表式でよくfitできる.
\begin{align}
  \rho(T)=\rho_0+a_nT^n
\end{align}
ここで$\rho_0$は残留抵抗であり,不純物などに起因する.
また$n$は整数である.ここでは鉛試料1及び金試料の常伝導状態における抵抗率を上の表式でfittingし,尤もらしい$n$を推定する.
図\ref{fig:graph/day4_4/Pb.tex},図\ref{fig:graph/day4_5/Au.tex}に各試料を$n=2,3,4,5$でfitした曲線を示す.
また表\ref{tab:an}にfittingにより得られた$a_n$の標準偏差を示す.
表から鉛試料1では$n=3$,金試料では$n=4$が最尤であると考えられる.
実際に図\ref{fig:graph/day4_4/Pb.tex},図\ref{fig:graph/day4_5/Au.tex}を見ると$n=3$, $n=4$で最も測定値と良く一致している.
\begin{table}[h]
\caption{$a_n$の標準偏差($\%$)}
\label{tab:an}
\centering
\begin{tabular}{c|cccc}
\hline
$n$&2&3&4&5\\
\hline \hline
鉛試料1&2.2&1.9&5.1&8.0\\
金試料&6.8&3.0&0.54&3.2\\
\hline
\end{tabular}
\end{table}
\gnu{鉛試料1の温度-抵抗率特性}{graph/day4_4/Pb.tex}
\gnu{金試料の温度-抵抗率特性}{graph/day4_5/Au.tex}
%\gnu{金鉄試料の温度-抵抗率特性}{graph/day4_6/AuFe.tex}