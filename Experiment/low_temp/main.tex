\documentclass[uplatex,a4j,11pt,dvipdfmx]{jsarticle}
\bibliographystyle{jplain}

\usepackage{url}

\usepackage{graphicx}
\usepackage{gnuplot-lua-tikz}
\usepackage{pgfplots}
\usepackage{tikz}
\usepackage{amsmath,amsfonts,amssymb}
\usepackage{bm}
\usepackage{siunitx}

\makeatletter
\def\fgcaption{\def\@captype{figure}\caption}
\makeatother
\newcommand{\setsections}[3]{
\setcounter{section}{#1}
\setcounter{subsection}{#2}
\setcounter{subsubsection}{#3}
}
\newcommand{\mfig}[3][width=15cm]{
\begin{center}
\includegraphics[#1]{#2}
\fgcaption{#3 \label{fig:#2}}
\end{center}
}
\newcommand{\gnu}[2]{
\begin{figure}[hptb]
\begin{center}
\input{#2}
\caption{#1}
\label{fig:#2}
\end{center}
\end{figure}
}

\begin{document}
\section{原理}
\subsection{レーザーの共振原理}
Light Amplification by Stimulated Emission of Radiation (LASER,レーザー)は誘導放射により増幅された光である.
放射とは励起した媒質原子が基底状態に遷移する際に電磁波を放出する過程であり,誘導放射と自発放射の2種類がある.
自発放射は励起した原子がある確率で自発的に基底状態に遷移する現象であるのに対し,
誘導放射は電場との相互作用により原子の状態が遷移する現象である.
誘導放射により放出される電磁波は外部の電場と同位相である.
したがって誘導放射が支配的な電磁波は位相が時間的,空間的に均一であり,
このような電磁波をコヒーレント(coherent)光と呼ぶ.

媒質中を進行する光は上記のように誘導放射と自発放射の成分があり,その輻射強度$I$の変化は(\ref{qeu:radiation})のように表される.
\begin{align}
  \label{qeu:radiation}
  \frac{{\rm d}I}{{\rm d}x}\propto AN_u+BIN_u-BIN_l=AN_u-BI(N_l-N_u)
\end{align}
ここで$N_u$は励起状態, $N_l$は基底状態の原子の密度である.
(\ref{qeu:radiation})の第1項は自発放射,第2項は誘導放射による自発放射,第3項は媒質による吸収を表す.
吸収は誘導放射の逆過程であるため同じ係数が掛かっている.
(\ref{qeu:radiation})は非同時線形微分方程式であり,その解は以下のようになる.
\begin{align}
  I(x)=\frac{AN_u}{B(N_l-N_u)}+C{\rm e}^{-B(N_l-N_u)x}
\end{align}
ここで$N_l>N_u$のときは$x\rightarrow\infty$で第2項が0に収束することから自発放射が支配的に,
$N_u>N_l$のときは第2項が発散することから誘導放射が支配的になるとわかる.
レーザーでは$N_u>N_l$なる状態を人工的に作り,共振器を用いて媒質中での進行距離を大きくすることで誘導放射が支配的な光を生成する.
また$N_u>N_l$なる分布を反転分布と呼ぶ.

共振器は図\ref{fig:fig/fig1.png}のような構成になっている.
レーザーが実際に発振するには前述の反転分布よりも強い条件が必要になる.
それは共振器から半透鏡や窓を用いて光を取り出したり,回折などにより損失が発生するためである.
したがって実際に発振が起きるには$N_u-N_l>N_{th}$なる条件が必要になる.
ここで$N_{th}$は損失の大きさによって定まる閾値である.
\mfig[width=12cm]{fig/fig1.png}{共振器の概略図}
また実際に放射される光は図\ref{fig:mode}のように媒質の熱運動によるドップラー効果や自発放射によりある程度の幅$f\pm\Delta f$を持っている.
この中で共振器の内部で定在波として存在できる波長のみが選択的に増幅される.
共振器の長さを$L$とすると周波数$\nu$の光が共振器内で定在波として存在する条件は
\begin{align}
  \nu_m=\frac{mc}{2L}\qquad (m\in\mathbb{N})
\end{align}
であり$\nu_m$が放射光の分布幅$2\Delta f$に収まっていれば増幅され発振する.
それぞれの$m$に対応する光軸方向の電磁場の分布を縦モードと呼ぶ.
また電磁場は光軸に垂直な方向にも分布を持ち,これを横モードと呼ぶ.
横モードは軸からの距離$r$と軸方向$z$について
\begin{align}
  E=E_0(z){\rm e}^{-(r/w(z))^2}
\end{align}
で表され,このようなビームをガウスビームと呼ぶ.
$w(z)$はビーム幅に関する値でスポットサイズという.
\begin{figure}[htbp]
  \begin{center}
    \begin{tikzpicture}
      \begin{axis}[ticks=none,
        axis lines=center,
        ymin=0,
        ymax=0.6,
        ylabel=$I_\nu$,
        xlabel=$\nu$,
        axis line style = thick,
        width=12cm,
        height=7cm
        ]
        \addplot[domain=0:8,samples=100]{e^(-(x-4)^2)/2};
        \draw [] (axis cs:0.7,0.35) node [above] {$m=1$};
        \draw [] (axis cs:2,0.35) node [above] {$2$};
        \draw [] (axis cs:3,0.35) node [above] {$3$};
        \draw [] (axis cs:4,0.35) node [above] {$4$};
        \draw [] (axis cs:5,0.35) node [above] {$5$};
        \draw [] (axis cs:6,0.35) node [above] {$6$};
        \draw [] (axis cs:7,0.35) node [above] {$7$};
        \draw [] (axis cs:4,0.5) node [above] {放射光の分布};
        \draw[-latex,black] (axis cs:1,0) -- (axis cs:1,0.35);
        \draw[-latex,black] (axis cs:2,0) -- (axis cs:2,0.35);
        \draw[-latex,black] (axis cs:3,0) -- (axis cs:3,0.35);
        \draw[-latex,black] (axis cs:4,0) -- (axis cs:4,0.35);
        \draw[-latex,black] (axis cs:5,0) -- (axis cs:5,0.35);
        \draw[-latex,black] (axis cs:6,0) -- (axis cs:6,0.35);
        \draw[-latex,black] (axis cs:7,0) -- (axis cs:7,0.35);
      \end{axis}
    \end{tikzpicture}
  \end{center}
  \caption{共振器のモード}
  \label{fig:mode}
\end{figure}
\subsection{熱放射と輝度温度}
\subsection{分光計について}
分光計の模式図を図\ref{fig:fig/fig2.png}に示す.
有限の幅を持ったスリット$S_1$から入射した光が回折格子で反射し$S_2$から出る.
このとき入射角$\alpha=\gamma+\theta$, 出射角$\beta=\gamma-\theta$であり,それぞれ$\delta\alpha$, $\delta\beta$の幅を持つ.
ここで回折条件は格子定数$d$として
\begin{align}
  \begin{split}
    \lambda\pm\delta\lambda&=d|\sin(\alpha\pm\delta\alpha)-\sin(\beta\pm\delta\beta)|\\
    &\simeq d|\sin\alpha-\sin\beta\pm(\cos\alpha\delta\alpha-\cos\beta\delta\beta)|\\
  \end{split}
\end{align}
となる.ここで$\delta\alpha,\delta\beta\ll1$として近似した.
$\lambda$に関する部分を見ると
\begin{align}
  \begin{split}
    \pm\lambda&=d(\sin\alpha-\sin\beta)\\
    &=2d\sin\theta\cos\gamma
  \end{split}
\end{align}
となり,回折格子を回転させることで任意の波長の光を取り出せることがわかる.
また$\delta\lambda$に関する部分を見ると
\begin{align}
    \delta\lambda&=d|\cos\alpha\delta\alpha-\cos\beta\delta\beta|
\end{align}
ここで誤差の伝搬則を用いると
\begin{align}
  |\delta\lambda|=d\sqrt{(\cos\alpha\delta\alpha)^2+(\cos\beta\delta\beta)^2}
\end{align}
となる.ここで幾何学的に
\begin{align}
  F\delta\alpha\simeq x_1\cos\gamma
\end{align}
なので$x_1=x_2=x$のもとで
\begin{align}
  |\delta\lambda|=d\sqrt{\left(\frac{x\cos\gamma}{F}\cos\alpha\right)^2+\left(\frac{x\cos\gamma}{F}\cos\beta\right)^2}
\end{align}
さらに$\gamma,\alpha,\beta\ll1$と仮定すると波長の幅は以下のようになる.
\begin{align}
  |\delta\lambda|\simeq \frac{\sqrt{2}d}{F}x
\end{align}
\clearpage
\mfig[width=6cm]{fig/fig2.png}{分光計の模式図}\clearpage
\section{測定について}
\subsection{分光計について\cite{jikken}}
分光計の模式図を図\ref{fig:fig/fig2.png}に示す.
有限の幅を持ったスリット$S_1$から入射した光が回折格子で反射し$S_2$から出る.
このとき入射角$\alpha=\gamma+\theta$, 出射角$\beta=\gamma-\theta$であり,それぞれ$\delta\alpha$, $\delta\beta$の幅を持つ.
ここで回折条件は格子定数$d$として
\begin{align}
  \begin{split}
    \lambda\pm\delta\lambda&=d|\sin(\alpha\pm\delta\alpha)-\sin(\beta\pm\delta\beta)|\\
    &\simeq d|\sin\alpha-\sin\beta\pm(\cos\alpha\delta\alpha-\cos\beta\delta\beta)|\\
  \end{split}
\end{align}
となる.ここで$\delta\alpha,\delta\beta\ll1$として近似した.
$\lambda$に関する部分を見ると
\begin{align}
  \begin{split}
    \pm\lambda&=d(\sin\alpha-\sin\beta)\\
    &=2d\sin\theta\cos\gamma
  \end{split}
\end{align}
となり,回折格子を回転させることで任意の波長の光を取り出せることがわかる.
また$\delta\lambda$に関する部分を見ると
\begin{align}
    \delta\lambda&=d|\cos\alpha\delta\alpha-\cos\beta\delta\beta|
\end{align}
ここで誤差の伝搬則を用いると
\begin{align}
  |\delta\lambda|=d\sqrt{(\cos\alpha\delta\alpha)^2+(\cos\beta\delta\beta)^2}
\end{align}
となる.ここで幾何学的に
\begin{align}
  F\delta\alpha\simeq x_1\cos\gamma
\end{align}
なので$x_1=x_2=x$のもとで
\begin{align}
  |\delta\lambda|=d\sqrt{\left(\frac{x\cos\gamma}{F}\cos\alpha\right)^2+\left(\frac{x\cos\gamma}{F}\cos\beta\right)^2}
\end{align}
さらに$\gamma,\alpha,\beta\ll1$と仮定するとスペクトルの幅は以下のようになる.
\begin{align}
  \label{equ:|deltalambda|}
  |\delta\lambda|\simeq \frac{\sqrt{2}d}{F}x
\end{align}

分光計に単色光を入射した場合の出力が半値幅$2|\delta\lambda|$すなわち$\sigma_a=\delta\lambda/2\sqrt{2\ln 2}$のガウス関数であるとする.
このとき$\sigma_\lambda$のガウス関数型スペクトルを持った光を入射すると,その出力のスペクトル$H$は以下の畳み込み積分で表される.
\begin{align}
  \begin{split}
    H(\lambda)&=\int_{-\infty}^{\infty}\frac{1}{2\pi\sigma_a\sigma_\lambda}{\rm e}^{-(t/\sqrt{2}\sigma_a)^2}{\rm e}^{-((\lambda-t)/\sqrt{2}\sigma_\lambda)^2}{\rm d}t\\
    &=\frac{{\rm e}^{-\lambda^2/2(\sigma_a^2+\sigma_\lambda^2)}}{2\pi\sigma_a\sigma_\lambda}\int^\infty_{-\infty}\exp\left(-\frac{\sigma_a^2+\sigma_\lambda^2}{2\sigma_a^2\sigma_\lambda^2}\left(t-\frac{\sigma_a^2\sigma_\lambda^2}{\sigma_a^2+\sigma_\lambda^2}\frac{\lambda}{\sigma_\lambda^2}\right)^2\right)\\
    &=\frac{1}{\sqrt{2\pi(\sigma_a^2+\sigma_\lambda^2)}}\exp\left(-\frac{t^2}{2(\sigma_a^2+\sigma_\lambda^2)}\right)
  \end{split}
\end{align}
すなわちガウス関数型のスペクトル分布を持った光を分光器に入力した場合も同様にガウス関数型のスペクトルが得られることがわかる.
\mfig[width=6cm]{fig/fig2.png}{分光計の模式図}
\subsection{FPIについて}
Fabry-Perrot Interderometer (FPI, ファブリ・ペロー干渉計)は任意の波長を選択的に取り出す光学機器である.
図\ref{fig:fig/fig3.png}にFPIの模式図を示す.
2つの半透鏡が向かいあわせに設置され,共振器長$D$が可変できるようになっている.
このとき周波数$\nu$の光が共振器内で定在波として存在する条件は
\begin{align}
  \nu_m=\frac{mc}{2D}\qquad(m\in\mathbb{N})
\end{align}
であり$D$を可変することで特定の波長を選択的に取り出すことができる.
$D$を一定の速度で掃引するとモードの異なるスペクトルパターンが周期的に現れ,これを測定することでスペクトルを高い分解能で測定できる.

FPIを用いてHe-Neレーザーのスペクトルを調べることを考える.
He-Neレーザーは多モード発振しておりそのここでは$\lambda_1$と$\lambda_2$にピークを持つとする.
したがってFPIスペクトルは図\ref{fig:He-Ne_FPI}のようになる.したがって以下が成り立つ.
\begin{align}
  \Delta D_1&=\frac{(m+1)\lambda_1}{2}-\frac{m\lambda_1}{2}=\frac{\lambda_1}{2}\\
  \Delta D_2&=\frac{m\lambda_2}{2}-\frac{m\lambda_1}{2}=\frac{m\Delta\lambda}{2}
\end{align}
ここで実際に測定されるのはピーク間の時間間隔であるが$D$の掃引速度が一定($=v$)であるならば
図\ref{fig:He-Ne_FPI}のように$\Delta_1=vT,\Delta_2=vt$となるので
\begin{align}
  \frac{t}{T}=\frac{\Delta D_2}{\Delta D_1}=\frac{m\Delta\lambda}{\lambda}
\end{align}
である.更に干渉条件$D\sim\frac{m\lambda}{2}$より$m$を消去すると
\begin{align}
  \frac{t}{T}=\frac{2\Delta\lambda}{\lambda^2}D
\end{align}
となるので$\lambda,D$が既知ならば$t/T$から$\Delta\lambda$が求まる.
\begin{figure}[htbp]
  \begin{center}
    \begin{tikzpicture}
      \begin{axis}[ticks=none,
        axis lines=center,
        ymin=0,
        ymax=0.5,
        xmin=1,
        xmax=7,
        ylabel=輻射強度$I$,
        xlabel=共振器長$D$,
        axis line style = thick,
        width=12cm,
        height=7cm,
        ylabel near ticks,
        xlabel near ticks,
        ]
        \draw [] (axis cs:4,0.37) node [above] {$\Delta D_1=vT$};
        \draw [] (axis cs:2.75,0.22) node [above] {$\Delta D_2=vt$};
        \draw [] (axis cs:5.4,0.4) node [above] {$\tfrac{(m+1)\lambda_1}{2}$};
        \draw [] (axis cs:6.1,0.4) node [above] {$\tfrac{(m+1)\lambda_2}{2}$};
        \draw [] (axis cs:2.5,0.4) node [above] {$\tfrac{m\lambda_1}{2}$};
        \draw [] (axis cs:3.0,0.4) node [above] {$\tfrac{m\lambda_2}{2}$};
        \draw[-latex,red] (axis cs:2.5,0) -- (axis cs:2.5,0.35);
        \draw[-latex,red] (axis cs:3,0) -- (axis cs:3,0.2);
        \draw[-latex,red] (axis cs:5.5,0) -- (axis cs:5.5,0.35);
        \draw[-latex,red] (axis cs:6,0) -- (axis cs:6,0.2);
        \draw[latex-latex,black] (axis cs:2.5,0.37) -- (axis cs:5.5,0.37);
        \draw[latex-latex,black] (axis cs:2.5,0.22) -- (axis cs:3,0.22);
      \end{axis}
    \end{tikzpicture}
  \end{center}
  \caption{He-NeレーザーのFPIスペクトル}
  \label{fig:He-Ne_FPI}
\end{figure}
\mfig[width=10cm]{fig/fig3.png}{FPI装置の模式図}
\subsection{測定手順}
\subsubsection{測定1:電気抵抗の測定($3.6$-$20\ \si{\kelvin}$)}
Cryocon温度コントローラーを用いて試料温度を調整しながら試料の電気抵抗を測定した.
温度コントローラーはカーボン抵抗温度計の値を入力としたPIDによる閉ループ制御を行うことで試料温度を目標温度に近づける.
また必要であればAutotune機能を用いてPIDゲインを自動的に設定することができる.

以上の機能を用いて$3.6\ \si{\kelvin}$から$20\ \si{\kelvin}$まで温度を$0.5\ \si{\kelvin}$間隔で変化させながら,ロックインアンプの電圧値を記録した.
この際実験室のコンピュータを用いて抵抗の変化を可視化しながら実験を行った.
ただし鉛の超電導転移点付近では急激に出力が変化するため,温度刻みを$0.01\ \si{\kelvin}$程度に細かくして記録した.
以上で記録した電圧値は(\ref{equ:rho})を用いて抵抗率に変換する.
\subsubsection{測定2:電気抵抗の測定($20$-$300\ \si{\kelvin}$)}
上記の測定終了後,冷凍機を停止すると試料温度が室温まで上昇するので,この過程で$20$から$300\ \si{\kelvin}$での
ロックインアンプの電圧値を同様に記録した.これと上記の測定を合わせることで$3.6$から$300\ \si{\kelvin}$での抵抗率の温度依存性を得られる.
\subsubsection{測定3:臨界磁場の測定}
クライオスタット内の超伝導磁石は電流測定用の$0.1\ \si{\ohm}$抵抗と直列に接続されており,
DMTを用いてその両端の電圧を測定することで実際に流れている電流を測定した.
この電流値$I$から実際に発生する磁場は以下の計算式で与えられる.
\begin{align}
  H\ \si{Oe}=0.73721\ \si{Oe.\milli\ampere^{-1}}\times I\ \si{\milli\ampere}
\end{align}
温度コントローラを用いて試料温度を$7\ \si{\kelvin}$に設定した.
この状態で超伝導磁石に流す電流を徐々に増やし,試料の超伝導が破壊される磁場(臨界磁場)を探した.
臨界磁場が見つかったらその付近で電流を細かく刻みながらロックインアンプの出力を記録した.
$6.5$-$4.0\ \si{\kelvin}$まで$0.5\ \si{\kelvin}$刻みで温度を変えつつ同様の測定を行った.
以上で記録した電圧値は(\ref{equ:rho})を用いて抵抗率に変換する.\clearpage
\section{結果}
\subsection{原理}
\subsubsection{回路方程式}
図\ref{fig:03/1-a.png}に反転増幅器の回路図を示す.
オペアンプの入力インピーダンスが$R_2$に比べて非常に高いことから入力電流$i$はすべて$R_2$に流れ込むとして良い.
したがってOhmの法則から
\begin{align}
  \label{equ:inv_amp_circuit_eq}
  i=\frac{v_i-v_d}{R_1}=\frac{v_d-v_o}{R_2}
\end{align}
ここで(\ref{equ:01-VoA})式から
\begin{align}
  v_o=A(0-v_d)
\end{align}
が成り立つためこれらを連立すると
\begin{align}
  v_o=-v_i\frac{R_2}{R_1}\frac{1}{1+\frac{R_1+R_2}{AR_1}}
\end{align}
$A$が十分大きいとき
\begin{align}
  \label{equ:inverted_amp_Vo}
  v_o=-\frac{R_2}{R_1}v_i
\end{align}
となる.以上から反転増幅器は入力信号に対して位相が反転し,電圧が$R_2/R_1$倍の信号が出力されることがわかる.
\mfig[width=10cm]{03/1-a.png}{反転増幅器}
\subsubsection{伝達関数}
(\ref{equ:inverted_amp_Vo})から伝達関数を求めると以下のようになる.
\begin{align}
  \begin{split}
    \label{equ:inv_amp_transform_eq}
    V_o&=-\frac{R_2}{R_1}V_i\\
    G(s)=\frac{V_o}{V_i}&=-\frac{R_2}{R_1}
  \end{split}
\end{align}
これに図\ref{fig:03/1-a.png}で示した値を代入してBode線図を描くと図\ref{fig:graph/1-A/theo-bode.tex}のようになる.
\gnu{Bode線図の理論値}{graph/1-A/theo-bode.tex}\clearpage
\subsection{測定3:臨界磁場の測定}
図\ref{fig:graph/day4/7k.tex}から図\ref{fig:graph/day4/4k.tex}に各温度における鉛試料1の抵抗率の磁場依存性及び,
最小二乗fittingによる回帰曲線を示す.回帰曲線はすべて1次関数であり,これらの交点が臨界磁場である.
表に最小二乗fittingで得た各温度における臨界磁場を示す.
\begin{table}[h]
\caption{各温度における臨界磁場}
\label{tab:rinkaijiba}
\centering
\begin{tabular}{cc}
\hline
試料台の温度$T_2$ / $\si{\kelvin}$&臨界磁場 / $\si{Oe}$\\
\hline \hline
$7.065$&$30.119$\\
$6.565$&$138.68$\\
$6.070$&$237.74$\\
$5.574$&$328.02$\\
$5.073$&$410.62$\\
$4.574$&$484.22$\\
$4.073$&$566.794$\\
\hline
\end{tabular}
\end{table}
\gnu{$7\ \si{\kelvin}$での臨界磁場}{graph/day4/7k.tex} %19.4645x-586.625 = 0.0236144x-1.09073
\gnu{$6.5\ \si{\kelvin}$での臨界磁場}{graph/day4/6_5k.tex} %7.18324x-996.503 = -0.000727453x-0.253076
\gnu{$6\ \si{\kelvin}$での臨界磁場}{graph/day4/6k.tex} %2.57464x-612.453 = -0.000507399x-0.230431
\gnu{$5.5\ \si{\kelvin}$での臨界磁場}{graph/day4/5_5k.tex} %0.961518x-315.644 = 0.00100394x-0.577902
\gnu{$5\ \si{\kelvin}$での臨界磁場}{graph/day4/5k.tex} %0.390929x-160.778 = 0.000911236x-0.630483
\gnu{$4.5\ \si{\kelvin}$での臨界磁場}{graph/day4/4_5k.tex} %0.138394x-67.2707 = -0.000567737x+0.0168057
\gnu{$4\ \si{\kelvin}$での臨界磁場}{graph/day4/4k.tex} %0.0666504x-37.977 = 0.0013983x-0.992479\clearpage
\section{考察}
\subsection{原理}
\subsubsection{回路方程式}
図\ref{fig:04/1-b.png}に非反転増幅器の回路図を示す.
仮想短絡を考えるとオペアンプの-端子の電圧$v_{-}]$は入力電圧$v_i$と等しいと考えられる.
さらに$v_-$は$v_o$を$R_1$と$R_2$で分圧した電圧になるので
\begin{align}
  \begin{split}
    \label{equ:amp_circuit_eq}
    v_i=v_-&=\frac{R_1}{R_1+R_2}v_o\\
    v_o&=\left(1+\frac{R_2}{R_1}\right)v_i
  \end{split}
\end{align}
となる.以上から非反転増幅器は入力信号に対して位相がそのままで,電圧が$1+R_2/R_1$倍の信号が出力されることがわかる.
\mfig[width=10cm]{04/1-b.png}{非反転増幅器}
\subsubsection{伝達関数}
(\ref{equ:amp_circuit_eq})から伝達関数を求めると以下のようになる.
\begin{align}
  \begin{split}
    V_o&=\left(1+\frac{R_2}{R_1}\right)V_i\\
    G(s)=\frac{V_o}{V_i}&=1+\frac{R_2}{R_1}
  \end{split}
\end{align}
これに図\ref{fig:04/1-b.png}で示した値を代入してBode線図を描くと図\ref{fig:graph/1-B/theo-bode.tex}のようになる.
\gnu{Bode線図の理論値}{graph/1-B/theo-bode.tex}
\clearpage
\subsection{方法}
\subsubsection{測定1-B-01}
図\ref{fig:04/1-b.png}の回路をQucs Spice上で作成しDC bias Simulationを行った.
まず$R_1=1.034\ \si{\kilo\ohm}$, $R_2=5.025\ \si{\kilo\ohm}$
として電圧利得6倍の非反転増幅器を構成した.この非反転増幅器上で入力電圧$V_i$を$-3\ \si{\volt}$から$3\ \si{\volt}$まで$0.5\ \si{\volt}$
刻みで変化させながら出力電圧$V_o$を記録した.
\subsubsection{測定1-B-02}
上記の回路で$R_2$を$608.4\ \si{\ohm}$に変更し,電圧利得1.6倍の非反転増幅器を構成した.
この回路においてTransient Simulationを行った.
この回路に$V_{p-p}=1\ \si{\volt}$,周波数が$10$, $100$, $1\si{\kilo}$, $10\si{\kilo}$, $100\ \si{\kilo\hertz}$の正弦波を入力し利得及び位相差を記録, Bode線図を作成した.
また周波数が$1\ \si{\kilo\hertz}$の時の波形を記録した.\clearpage
\bibliography{ref.bib}
%\gnu{$7\ \si{\kelvin}$での臨界磁場}{graph/day3/7k.tex}
%\gnu{$6.5\ \si{\kelvin}$での臨界磁場}{graph/day3/6_5k.tex}
%\gnu{$6\ \si{\kelvin}$での臨界磁場}{graph/day3/6k.tex}
%\gnu{$5.5\ \si{\kelvin}$での臨界磁場}{graph/day3/5_5k.tex}
%\gnu{$5\ \si{\kelvin}$での臨界磁場}{graph/day3/5k.tex}
%\gnu{$4.5\ \si{\kelvin}$での臨界磁場}{graph/day3/4_5k.tex}
%\gnu{$4\ \si{\kelvin}$での臨界磁場}{graph/day3/4k.tex}
\end{document}
