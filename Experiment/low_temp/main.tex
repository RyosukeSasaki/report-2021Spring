\documentclass[uplatex,a4j,11pt,dvipdfmx]{jsarticle}
\bibliographystyle{jplain}

\usepackage{url}

\usepackage{graphicx}
\usepackage{gnuplot-lua-tikz}
\usepackage{pgfplots}
\usepackage{tikz}
\usepackage{amsmath,amsfonts,amssymb}
\usepackage{bm}
\usepackage{siunitx}

\makeatletter
\def\fgcaption{\def\@captype{figure}\caption}
\makeatother
\newcommand{\setsections}[3]{
\setcounter{section}{#1}
\setcounter{subsection}{#2}
\setcounter{subsubsection}{#3}
}
\newcommand{\mfig}[3][width=15cm]{
\begin{center}
\includegraphics[#1]{#2}
\fgcaption{#3 \label{fig:#2}}
\end{center}
}
\newcommand{\gnu}[2]{
\begin{figure}[hptb]
\begin{center}
\input{#2}
\caption{#1}
\label{fig:#2}
\end{center}
\end{figure}
}

\begin{document}
%\gnu{鉛試料1,2の温度-抵抗率特性}{graph/day2/Pb.tex}
%\gnu{金鉄合金の温度-抵抗率特性}{graph/day2/AuFe.tex}
%\gnu{金試料の温度-抵抗率特性}{graph/day2/Au.tex}

%\gnu{鉛試料1,2の温度-抵抗率特性}{graph/day3_2/Pb.tex}
%\gnu{金鉄合金の温度-抵抗率特性}{graph/day3_2/AuFe.tex}
%\gnu{金試料の温度-抵抗率特性}{graph/day3_2/Au.tex}

%\gnu{$7\ \si{\kelvin}$での臨界磁場}{graph/day3/7k.tex}
%\gnu{$6.5\ \si{\kelvin}$での臨界磁場}{graph/day3/6_5k.tex}
%\gnu{$6\ \si{\kelvin}$での臨界磁場}{graph/day3/6k.tex}
%\gnu{$5.5\ \si{\kelvin}$での臨界磁場}{graph/day3/5_5k.tex}
%\gnu{$5\ \si{\kelvin}$での臨界磁場}{graph/day3/5k.tex}
%\gnu{$4.5\ \si{\kelvin}$での臨界磁場}{graph/day3/4_5k.tex}
%\gnu{$4\ \si{\kelvin}$での臨界磁場}{graph/day3/4k.tex}

%\gnu{$7\ \si{\kelvin}$での臨界磁場}{graph/day4/7k.tex} %19.4645x-586.625 = 0.0236144x-1.09073
%\gnu{$6.5\ \si{\kelvin}$での臨界磁場}{graph/day4/6_5k.tex} %7.18324x-996.503 = -0.000727453x-0.253076
%\gnu{$6\ \si{\kelvin}$での臨界磁場}{graph/day4/6k.tex} %2.57464x-612.453 = -0.000507399x-0.230431
%\gnu{$5.5\ \si{\kelvin}$での臨界磁場}{graph/day4/5_5k.tex} %0.961518x-315.644 = 0.00100394x-0.577902
%\gnu{$5\ \si{\kelvin}$での臨界磁場}{graph/day4/5k.tex} %0.390929x-160.778 = 0.000911236x-0.630483
%\gnu{$4.5\ \si{\kelvin}$での臨界磁場}{graph/day4/4_5k.tex} %0.138394x-67.2707 = -0.000567737x+0.0168057
%\gnu{$4\ \si{\kelvin}$での臨界磁場}{graph/day4/4k.tex} %0.0666504x-37.977 = 0.0013983x-0.992479

%\gnu{鉛試料1,2の温度-抵抗率特性($6$-$9\ \si{\kelvin}$)}{graph/day4_2/Pb.tex}

%\gnu{超伝導体の磁場-温度相図}{graph/day4_3/Pb1.tex}
\bibliography{ref.bib}
\end{document}
