\subsection{測定1:電気抵抗の測定($3.6$-$20\ \si{\kelvin}$)}
図\ref{fig:graph/day2/Pb.tex}から図\ref{fig:graph/day2/AuFe.tex}に鉛試料,金試料,金鉄合金の
$3.6$から$20\ \si{\kelvin}$における抵抗率の温度依存性を示す.
温度計は試料台と第2ステージに取り付けられているが,より試料に近い温度を示す試料台の温度計の値を用いた.
\gnu{鉛試料1,2の温度-抵抗率特性($3.6$-$20\ \si{\kelvin}$)}{graph/day2/Pb.tex}
\gnu{金試料の温度-抵抗率特性($3.6$-$20\ \si{\kelvin}$)}{graph/day2/Au.tex}
\gnu{金鉄合金の温度-抵抗率特性($3.6$-$20\ \si{\kelvin}$)}{graph/day2/AuFe.tex}
\clearpage
\subsection{測定2:電気抵抗の測定($20$-$300\ \si{\kelvin}$)}
図\ref{fig:graph/day3_2/Pb.tex}から図\ref{fig:graph/day3_2/AuFe.tex}に各試料の
$3.6$から$300\ \si{\kelvin}$における抵抗率の温度依存性を示す.
\gnu{鉛試料1,2の温度-抵抗率特性($3.6$-$300\ \si{\kelvin}$)}{graph/day3_2/Pb.tex}
\gnu{金試料の温度-抵抗率特性($3.6$-$300\ \si{\kelvin}$)}{graph/day3_2/Au.tex}
\gnu{金鉄合金の温度-抵抗率特性($3.6$-$300\ \si{\kelvin}$)}{graph/day3_2/AuFe.tex}
