\subsection{測定3:臨界磁場の測定}
図\ref{fig:graph/day4/7k.tex}から図\ref{fig:graph/day4/4k.tex}に各温度における鉛試料1の抵抗率の磁場依存性及び,
最小二乗fittingによる回帰曲線を示す.回帰曲線はすべて1次関数であり,これらの交点が臨界磁場である.
表に最小二乗fittingで得た各温度における臨界磁場を示す.
\begin{table}[h]
\caption{各温度における臨界磁場}
\label{tab:rinkaijiba}
\centering
\begin{tabular}{cc}
\hline
試料台の温度$T_2$ / $\si{\kelvin}$&臨界磁場 / $\si{Oe}$\\
\hline \hline
$7.065$&$30.119$\\
$6.565$&$138.68$\\
$6.070$&$237.74$\\
$5.574$&$328.02$\\
$5.073$&$410.62$\\
$4.574$&$484.22$\\
$4.073$&$566.794$\\
\hline
\end{tabular}
\end{table}
\gnu{$7\ \si{\kelvin}$での臨界磁場}{graph/day4/7k.tex} %19.4645x-586.625 = 0.0236144x-1.09073
\gnu{$6.5\ \si{\kelvin}$での臨界磁場}{graph/day4/6_5k.tex} %7.18324x-996.503 = -0.000727453x-0.253076
\gnu{$6\ \si{\kelvin}$での臨界磁場}{graph/day4/6k.tex} %2.57464x-612.453 = -0.000507399x-0.230431
\gnu{$5.5\ \si{\kelvin}$での臨界磁場}{graph/day4/5_5k.tex} %0.961518x-315.644 = 0.00100394x-0.577902
\gnu{$5\ \si{\kelvin}$での臨界磁場}{graph/day4/5k.tex} %0.390929x-160.778 = 0.000911236x-0.630483
\gnu{$4.5\ \si{\kelvin}$での臨界磁場}{graph/day4/4_5k.tex} %0.138394x-67.2707 = -0.000567737x+0.0168057
\gnu{$4\ \si{\kelvin}$での臨界磁場}{graph/day4/4k.tex} %0.0666504x-37.977 = 0.0013983x-0.992479