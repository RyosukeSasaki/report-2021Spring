\subsubsection*{(1-1)}
第2回授業資料から氷の密度を$\rho_i=917\ \si{\kilo\gram.\metre^{-3}}$,岩石の密度を$\rho_s=3000\ \si{\kilo\gram.\metre^{-3}}$,水の層の厚さを$r$とする.このとき以下が成り立つ.
\begin{align}
  \frac{4\pi r^3}{3}\times3000+\frac{4\pi}{3}\left((R_e\times1000)^3-r^3\right)\times917=M_e
\end{align}
これを解くと
\begin{align}
  \begin{split}
    r&=2.31\times10^5\ \si{\metre}\\
    &=231\ \si{\kilo\metre}
  \end{split}
\end{align}
であるので,氷の厚さは
\begin{align}
  252-231=2.10\times10^1\ \si{\kilo\metre}
\end{align}
である.
\subsubsection*{(1-2)}
岩石の質量は$M_s=4\pi/3r^3\rho_s$, 氷の質量は$M_i=4\pi/3(R_e^3-r^3)\rho_i$なので慣性モーメントは
\begin{align}
  \begin{split}
    C_E&=\frac{2}{5}M_s r^2+\frac{2}{5}M_i R_e^2\frac{1-(r/R_e)^5}{1-(r/R_e)^3}\\
    &=\frac{8\pi}{15}\rho_s r^5+\frac{8\pi}{15}\rho_i(R_e^5-r^5)
  \end{split}
\end{align}
したがって
\begin{align}
  \begin{split}
    \frac{C_E}{M_eR_e^2}&=\frac{1}{M_eR_e^2}\frac{8\pi}{15}\left(\rho_sr^5+\rho_i(R_e^5-r^5)\right)\\
    &=0.564
  \end{split}
\end{align}
である.
\subsubsection*{(1-3)}
木星軌道は雪線以遠であるため,表面はガス及び氷から成ると考えられる.
またガニメデにはオーロラが発生しているため,地磁気が存在すると考えられる.\cite{rikanenpyo-aurora:online}
したがって溶融した鉄からなる核が存在すると考えられるので,
ガニメデの内部構造は(D)のようになっていると考えられる.