\subsubsection*{(3-1)}
変形後の各辺の長さは以下で表される.
\begin{align}
  a'&=a(1+\epsilon_{xx})\\
  b'&=b(1+\epsilon_{yy})\\
  c'&=c(1+\epsilon_{zz})
\end{align}
したがって体積歪み$\Delta$は
\begin{align}
  \begin{split}
    \Delta=\frac{\delta V}{V}&=\frac{abc(1+\epsilon_{xx})(1+\epsilon_{yy})(1+\epsilon_{zz})-abc}{abc}\\
    &=1+\epsilon_{xx}+\epsilon_{yy}+\epsilon_{zz}+\epsilon_{xx}\epsilon_{yy}+\epsilon_{yy}\epsilon_{zz}+\epsilon_{zz}\epsilon_{xx}-1
  \end{split}
\end{align}
ここで$\epsilon_{ii}$の2次以上を無視すると
\begin{align}
  \Delta=\epsilon_{xx}+\epsilon_{yy}+\epsilon_{zz}
\end{align}
を得る.
\subsubsection*{(3-2)}
弾性体の構成方程式から
\begin{align}
  \sum_{i,j}\sigma_{ij}=\sum_{i,j}\left(\lambda\Delta\delta_{ij}+2\mu\epsilon_{ij}\right)
\end{align}
ここで歪みテンソル,応力テンソルは対角成分のみが非零なので
\begin{align}
  \begin{split}
    \sigma_{xx}+\sigma_{yy}+\sigma_{zz}&=3\lambda\Delta+2\mu(\epsilon_{xx}+\epsilon_{yy}+\epsilon_{zz})\\
    -3P&=(3\lambda+2\mu)\Delta\\
    -\frac{P}{\Delta}&=\lambda+\frac{2}{3}\mu=k
  \end{split}
\end{align}
を得る.