\subsubsection*{(4-1)}
グーテンベルグリヒターの式において両辺の指数を取ると
\begin{align}
  \label{equ:gutenberg-richter}
  n(M)=10^a10^{-bM}
\end{align}
すなわち$b$は減衰に関する項であり$b$が大きいほど$n(M)$は急速に減衰する.
つまり$a$が等しく$b$が高い地域はマグニチュードが大きい地震の発生頻度が高いと言える.
幾つかの$b$についてのプロットを図\ref{fig:graph/4-1.tex}に示す.
\gnu{$n(M)$に対する$b$の依存性}{graph/4-1.tex}
\subsubsection*{(4-2)}
(\ref{equ:gutenberg-richter})からマグニチュード$M$以上の地震の総数は
\begin{align}
  \begin{split}
    \int^\infty_Mn(M){\rm d}M&=10^a\int^\infty_M10^{-bM}{\rm d}M\\
    N(M)&=10^a\frac{0-10^{-bM}}{-b\ln 10}\\
    \log N(M)&=a-bM-\log b\ln 10\\
    &=A-bM
  \end{split}
\end{align}
となる.