\setcounter{equation}{5}
\subsubsection*{(1-1)}
鉛直方向の温度勾配が水平方向よりも大きいことから
\begin{align}
  \left|\frac{\partial T}{\partial z}\right|\gg\left|\frac{\partial T}{\partial x}\right|
\end{align}
を仮定する.また$x=vt$から${\rm d}x=v{\rm d}t$なので(2)式は
\begin{align}
  \frac{\partial T}{\partial t}&=\kappa\left(\frac{\partial^2T}{\partial x^2}+\frac{\partial^2 T}{\partial z^2}\right)
\end{align}
ここで(6)から$\partial^2 T/\partial x^2$を無視すると
\begin{align}
  \frac{\partial T}{\partial t}=\kappa\frac{\partial^2 T}{\partial z^2}
\end{align}
となり(1)と一致する.
\subsubsection*{(1-2)}
半無限冷却モデルでは熱伝導の影響が及ぶ範囲が無限に広がってしまう.
一方でプレートモデルではプレートの厚さにより新たに温度に関して境界条件が付加される.
この差が年代の古いプレートでは如実に現れることになる.
\subsubsection*{(2-1)}
図(b)をみると低温,高温の境界層付近では温度勾配が大きく,中心付近では温度がほぼ一定であることが読み取れる.

これは境界近傍では伝導による熱輸送が支配的なためである.
境界付近では伝導が支配的なため,温度勾配は半無限冷却モデルで用いたような熱拡散方程式によって記述される.
その温度勾配は図\ref{fig:fig/2-1.png}のような概形を示し,たしかに図(b)の境界付近と近い分布になっていることがわかる.

一方で球殻の内部ではレイリー数によっては対流による熱輸送が支配的になりうる.
対流が内部で活発に起こっているとき,より効率的に熱が拡散するため内部での温度勾配はほぼ一定になる.
図(b)ではたしかに内部の温度分布が均一になっており,対流を起こすのに十分大きなレイリー数を持っていることがわかる.

また模型として球殻を用いたため,初期状態で上層付近に存在し温度が低かった物質は量がおおく,
下層付近に存在し温度が高かった物質は量が少なくなる.したがって定常状態において球殻内部の温度は
規格化された温度において$0.5$よりも低い温度になっていることがわかる.
\mfig[width=6cm]{fig/2-1.png}{半無限冷却モデルによる温度勾配の概形(配布資料7から引用)}