\subsubsection*{(3-1)}
実験により$660\ \si{\kilo\metre}$の温度,圧力においてクラペイロン勾配が負の相転移があることがわかっている.
この相転移では温度が高いほど高密度相への相転移に必要な圧力が低い.
すると平均温度のマントルが高密度相に相転移できる深さに至っても低温のマントルが低密度相のままということになる.
また平均より温度が高いマントルは上昇して圧力が下がっても低密度相に相転移しにくくなる.
以上から低温のマントルは周囲に比べて軽く,高温のマントルが周囲に比べて重くなるため,対流が阻害されるのである.
\subsubsection*{(3-2)}
プレートの沈み込み境界では地震活動が起こりやすいが,多くの沈み込み境界において震源深さは$670\ \si{\kilo\metre}$程度まででそれ以上深い地震は観測されていない.
これは$670\ \si{\kilo\metre}$付近でスラブの沈み込みが止まっていることを示唆する.
\subsubsection*{(3-3)}
2層対流では上層と下層において対流が阻害され,その間での熱伝達は伝導が支配的ということになる.
これにより全層対流の時の温度勾配は図(b)のように球殻内部での温度がほぼ一様であったのに比べて,
2層対流では間に遷移層,すなわち温度勾配が急な箇所が発生すると考えられる.
\mfig[width=6cm]{fig/3-3.png}{全層対流と2層対流の場合の温度勾配\cite{kawakatu}}